
Nie potrafimy jeszcze sensownie wprowadzić pola powierzchni, ani tym bardziej objętości brył.
Coś o mierze Lebesgue'a.
Eves \cite[s. 194-239]{eves1_1972} poświęca cały piąty rozdział rozcięciom (dissections).

Dwa wielokąty mają to samo pole powierzchni wtedy i tylko wtedy, kiedy są równoważne przez rozcinanie.
Analogiczne twierdzenie dla wielościanów (wypukłych) jest fałszywe, jak pokaże Dehn w 1902.
To tłumaczy, czemu starożytni będą inaczej atakować problem wyznaczania objętości brył: przez infinitesimals, ciągłość, całkowanie.

Eves pisze, że ''We make no effort to define "area of a
polygon" or "volume of a polyhedron," but rely on intuitive notions of these ideas. A careful and sophisticated treatment of these ideas, developed
from some suitable postulate base, is an important matter in a discussion
f the foundations of geometry.''

The necessity of proving Theorem 5.1.3 seems first to have occurred to W. H. Jackson in 1912. % 5.1 .3 THEOREM. If P '"-J R (+) and R '"-J Q (+), then P  Q (+).

Other dissection matters in space differ from the corresponding situations in the plane. Thus, though it is always possible to dissect a polygon into a finite number of triangles having their vertices only at the vertices of the polygon, there exist polyhedra, which we shall call Lennes polyhedra, which cannot be dissected into finite numbers of tetrahedra having their vertices only at the vertices of the polyhedra. Indeed, much of the theory of the dissection of polyhedra will be found to be largely of the nature of a make- shift.

Eves 202: wielokąty można rozcinać na trójkąty tak, żeby nie dokładać nowych wierzchołków. ($n$-kąt na n-2 trójkątów)

205: trójkąt równoważny z prostokątem o tej samej podstawie co najdluzszy bok
prostokąt równoważny z kwadratem
206: dwa kwadraty rownowazne z trzecim suma; kazdy wielokat z kwadratem

207: rownowaznosc pol <=> rownowaznosc przez podzial dla wielokatow;

rownowazne przez odejmowanie => rownowazne przeez dodawANIE

209, 210: problemy

211: 
In this and the next two sections we consider some matters connected with the dissection of (solid) polyhedra into subpolyhedra. We first prove the existence of polyhedra which cannot be dissected into tetrahedra having their vertices only at the vertices of the polyhedron. This interesting fact was first published in 1911 by N. J. Lennes, who constructed a polyhedron having the property that the join of any two vertices not the endpoints of a common edge lies either wholly or partly outside the polyhedron; it is obvious that such a polyhedron cannot be dissected into tetrahedra in the desired fashion. 5.4.1 DEFINITION. A polyhedron which cannot be dissected into tetra- hedra having their vertices only at the vertices of the polyhedron will be called a Lennes polyhedron. The polyhedron constructed by Lennes possessed seven vertices. In 1928, E. Schönhardt gave a simpler example of a Lennes polyhedron having only six vertices, and he proved that there is no Lennes polyhedron with less than six vertices. In 1948, F. Bagemihl showed the existence of Lennes polyhedra having any number of vertices not less than six. We shall here establish the existence of Lennes polyhedra by describing Schönhardt's example of 1928. 5.4.2 THEOREM. Lennes polyhedra exist.

5.4.3 CAUCHY'S THEOREM. Every convex polyhedron* can be dissected into
tetrahedra having their vertices only at the vertices of the polyhedron and also
all sharing a common vertex.

213: tw. dehna intro?, pelne na 217
216: arccos 1/3

5.6.5 SÜss' THEOREM (1920). If P and Q are any two equivalent polyhedra,
then P = Q(T-).