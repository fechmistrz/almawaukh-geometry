\subsection{Elementarne wyniki}

\begin{proposition}
	\label{hartshorne_52}
    Niech $AB$ będzie odcinkiem.
	Istnieje wtedy trójkąt równoramienny, którego podstawą jest $AB$.
\end{proposition}

Powyższe stwierdzenie jest ciekawe, bo jest prawdziwe na płaszczyźnie Hilberta, tzn. jego prawdziwość nie zależy od aksjomatu Pascha.
(W geometrii nieeuklidesowej może nie istnieć trójkąt równoboczny o danej podstawie).

\begin{proposition}
	\label{hartshorne_52}
	Linia środkowa (odcinek łączący środki pewnych dwóch boków trójkąta) jest równoległa do trzeciego boku.
    Jej długość jest dwukrotnie mniejsza od długości tego boku.
\end{proposition}
% Hartshorne s. 52

Tego stwierdzenia nie ma w Elementach Euklidesa, ale można wyprowadzić je z księgi I (I.29, I.26, I.34), jak wspomina Hartshorne \cite[s. 52. 53]{hartshorne2000}.

\begin{corollary}
	Niech $ABC$ będzie trójkątem, zaś punkty $D$, $E$ i $F$ środkami jego boków.
	Wtedy cztery małe trójkąty utworzone na bokach $DE$, $EF$, $FD$ są przystające do siebie.
\end{corollary}

Do tego wniosku potrzeba dodatkowo cechy przystawania bok-bok-bok (I.8).
