\section{Elementarne wyniki}

\begin{definition}[symetralna]
    \label{def_symetralna}
	Prostą prostopadłą do odcinka i przechodzącą przez jego środek nazywamy symetralną.
    \index{symetralna}
\end{definition}

Symetralna odcinka $AB$ jest miejscem geometrycznym punktów, które są równodległe od obu końców odcinka, a zatem są środkiem pewnego okręgu przechodzącego przez punkty $A$ oraz $B$ (Guzicki \cite[s. 14, 15]{guzicki_2021}).
W szczególności, trzy współliniowe punkty nie leżą na jednym okręgu, ale już trzy niewspółliniowe punkty wyznaczają taki okrąg jednoznacznie.
\index{współliniowy}

Okrąg przechodzący przez wszystkie wierzchołki wielo-(niekoniecznie trój!)-kąta nazywamy opisanym na tym wielokącie, zatem poprzednie stwierdzenie można wysłowić krótko ,,na każdym trójkącie można opisać dokładnie jeden okrąg''.
Mówimy też, że wielokąt jest wpisany w okrąg.
\index{okrąg!opisany}%

% TODO: https://en.wikipedia.org/wiki/Bisection#Line_segment_bisector
% https://en.wikipedia.org/wiki/Circumcircle

\begin{proposition}
	\label{hartshorne_52x}
    Niech $AB$ będzie odcinkiem.
	Istnieje wtedy trójkąt równoramienny, którego podstawą jest $AB$.
    \index{trójkąt!równoramienny}%
\end{proposition}

Powyższe stwierdzenie jest ciekawe, bo jest prawdziwe na płaszczyźnie Hilberta, tzn. jego prawdziwość nie zależy od aksjomatu Pascha.
\index{płaszczyzna!Hilberta}
\index{aksjomat!Pascha}
(W geometrii nieeuklidesowej może nie istnieć trójkąt równoboczny o danej podstawie).

\begin{proposition}
	\label{hartshorne_52}
	Linia środkowa (odcinek łączący środki pewnych dwóch boków trójkąta) jest równoległa do trzeciego boku.
    \index{linia środkowa}
    Jej długość jest dwukrotnie mniejsza od długości tego boku.
\end{proposition}
% Hartshorne s. 52

Tego stwierdzenia nie ma w Elementach Euklidesa, ale można wyprowadzić je z księgi I (I.29, I.26, I.34), jak wspomina Hartshorne \cite[s. 45, 52. 53]{hartshorne2000}.
Patrz też Bogdańska, Neugebauer \cite[s. 15]{neugebauer_2018}.

\begin{corollary}
	Niech $ABC$ będzie trójkątem, zaś punkty $D$, $E$ i $F$ środkami jego boków.
	Wtedy cztery małe trójkąty utworzone na bokach $DE$, $EF$, $FD$ są przystające do siebie.
\end{corollary}

Do tego wniosku potrzeba dodatkowo cechy przystawania bok-bok-bok (I.8).
\index{cecha przystawania!bok-bok-bok}%



%