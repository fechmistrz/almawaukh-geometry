\section{Elementarne wyniki}

\begin{definition}[symetralna]
	Prostą prostopadłą do odcinka i przechodzącą przez jego środek nazywamy symetralną.
    \index{symetralna}
\end{definition}

Przez dwa punkty możemy przeprowadzić nieskończenie wiele okręgów, ich środki leżą na symetralnej odcinka łączącego wspomniane dwa punkty.
Przez trzy punkty możemy przeprowadzić jeden okrąg, jeśli nie są współliniowe (i zero w przeciwnym razie).
\index{współliniowy}
\todofoot{Czy to jest tu? Z czego to wynika? Może później, jeśli korzystamy w dowodzie ze zbyt wielu aksjomatów? Guzicki s. 14, 15.}

\begin{proposition} % Guzicki s. 126
	Punkt $P$ leży na symetralnej odcinka $AB$ wtedy i tylko wtedy, gdy jest jednakowo oddalony od końców tego odcinka: $|AP| = |BP|$.
\end{proposition}

% TODO: https://en.wikipedia.org/wiki/Bisection#Line_segment_bisector

Okrąg przechodzący przez wszystkie wierzchołki wielokąta nazywamy okręgiem opisanym na tym wielokącie. % https://en.wikipedia.org/wiki/Circumcircle
Mówimy także, że wielokąt jest wpisany w okrąg.
\index{okrąg!opisany}%

\begin{proposition}
	\label{hartshorne_52x}
    Niech $AB$ będzie odcinkiem.
	Istnieje wtedy trójkąt równoramienny, którego podstawą jest $AB$.
    \index{trójkąt!równoramienny}%
\end{proposition}

Powyższe stwierdzenie jest ciekawe, bo jest prawdziwe na płaszczyźnie Hilberta, tzn. jego prawdziwość nie zależy od aksjomatu Pascha.
\index{płaszczyzna!Hilberta}
\index{aksjomat!Pascha}
(W geometrii nieeuklidesowej może nie istnieć trójkąt równoboczny o danej podstawie).

\begin{proposition}
	\label{hartshorne_52}
	Linia środkowa (odcinek łączący środki pewnych dwóch boków trójkąta) jest równoległa do trzeciego boku.
    \index{linia środkowa}
    Jej długość jest dwukrotnie mniejsza od długości tego boku.
\end{proposition}
% Hartshorne s. 52

Tego stwierdzenia nie ma w Elementach Euklidesa, ale można wyprowadzić je z księgi I (I.29, I.26, I.34), jak wspomina Hartshorne \cite[s. 45, 52. 53]{hartshorne2000}.

\begin{corollary}
	Niech $ABC$ będzie trójkątem, zaś punkty $D$, $E$ i $F$ środkami jego boków.
	Wtedy cztery małe trójkąty utworzone na bokach $DE$, $EF$, $FD$ są przystające do siebie.
\end{corollary}

Do tego wniosku potrzeba dodatkowo cechy przystawania bok-bok-bok (I.8).
\index{cecha przystawania!bok-bok-bok}%

\subsection{Okręgi?}
\begin{proposition}
    Niech $\Gamma$ będzie okręgiem o środku $O$ oraz promieniu $OA$.
    Wtedy prosta prostopadła do $OA$, która przechodzi przez $A$, jest styczną do okręgu, leżącą (poza punktem $A$) na zewnątrz okręgu $\Gamma$.
    Odwrotnie, każda prosta, która jest styczna w punkcie $A$ do okręgu $\Gamma$, musi być prostopadła do prostej $OA$.
    \index{styczność}
\end{proposition} % Hartshorne 105

\begin{corollary}
    Przez każdy punkt okręgu przechodzi dokładnie jedna styczna do tego okręgu.
\end{corollary} % Hartshorne 105

\begin{corollary}
    Prosta, która nie jest styczna do okręgu i nie jest z nim rozłączna, musi przecinać go w dokładnie dwóch punktach.
\end{corollary} % Hartshorne 106

\begin{proposition}
    Niech $O_1, O_2, A$ będą trzema punktami.
    Następujące warunki są równoważne: punkty $A, O_1, O_2$ są współliniowe; okręgi o promieniach $O_1A$, $O_2A$ są styczne.
\end{proposition} % Hartshorne 105

\begin{corollary}
    Dwa okręgi, które nie są rozłączne i nie są styczne, mają dokładnie dwa punkty wspólne.
\end{corollary} % Hartshorne 106

\index{twierdzenie!Sylvestera-Gallaia|(}%
\subsection{Twierdzenie Sylvestera-Gallaia}
\begin{theorem}[Sylvestera-Gallaia]
	Dla każdego skończonego zbioru punktów na płaszczyźnie istnieje prosta, która przechodzi przez dokładnie dwa albo wszystkie punkty.
\end{theorem}

Mamy wrażenie, że zaczęło się w 1893 roku, kiedy James Sylvester postawił problem.
Być może zainspirowała go konfiguracją Hessego\footnote{Konfiguracja Hessego to 12 prostych przez 9 punktów na zespolonej płaszczyźnie rzutowej, gdzie każdy punkt leży na 4 prostych, a każda prosta przechodzi przez 3 punkty}.
Herbert Woodall szybko zaproponował rozwiązanie, gdzie równie szybko wychwycono usterkę.
Dopiero w 1941 roku Eberhard Melchior udowodnił trochę mocniejsze stwierdzenie niż rzutowy dual ówczesnej hipotezy (że prostych przez dokładnie dwa punkty jest co najmniej trzy).
Nieświadomy tego, Paul ErdErdős postawił hipotezę na nowo w~1943 roku, a Tibor Gallai w 1944 roku dodał swój dowód (ponownie wykorzystując elementy geometrii rzutowej).
Wraz z upływem czasu pojawiały się inne, ciekawe rozumowania.
Na przykład Leroy Kelly wykorzystał własności metryki, co oburzyło Harolda Coxetera i skłoniło go do opublikowania kolejnego dowodu, korzystającego jedynie z aksjomatów geometrii uporządkowania.
(Aigner, Ziegler uważają dowód Kelly'ego za najlepszy).

Niech $t_2(n)$ oznacza minimalną liczbę prostych przez dwa punkty w dowolnym ułożeniu $n$ punktów.
Melchior pokazał, że $t_2(n) \ge 3$.
Wynik sukcesywnie poprawiano:
de Bruijn \cite{debruijn_1948} zapytał, czy $t_2(n)$ dąży do nieskończoności,
Theodore Motzkin \cite{motzkin_1951} udzielił twierdzącej odpowiedz, bo $t_2(n) \ge \sqrt{n}$.
Potem Gabriel Dirac \cite{dirac_1951} przypuścił, że $t_2(n) \ge \lfloor n/2\rfloor$, co nie zostawia wiele miejsca na poprawki, bo dla parzystych $n \ge 6$ zachodzi $t_2(n) \le n/2$, jak pokazał pomysłową konstrukcją Károly Böröczky.
Dla nieparzystych $n$ wiemy tylko, że ten kres jest realizowany dla $n = 7$ (Kelly, Moser \cite{kelly_1958} w 1958) i $n = 13$ (Crowe, McKee \cite{mckee_1968} w 1968).
Najnowszy wynik, o jakim nam wiadomo, to Csimy, Sawyera \cite{csima_1993}: że $t_2(n) \ge \lceil 6n/13 \rceil$.
\index{twierdzenie!Sylvestera-Gallaia|)}%

%