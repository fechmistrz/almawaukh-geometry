\subsection{Księga I}	
\subsubsection{Definicje}	
\begin{enumerate}	
    \item [1.1] Definicja ... % Definicja 1. % Punkt to jest to, co nie składa się z części.
    \item [1.2] Definicja ... % Definicja 2. % Linia jest długością bez szerokości.
    \item [1.3] Definicja ... % Definicja 3. % Końcami linii są punkty.
    \item [1.4] Definicja ... % Definicja 4. % Linia jest prosta, jeżeli położona jest między swoimi punktami w równym i jednostajnym kierunku.
    \item [1.5] Definicja ... % Definicja 5. % Powierzchnia jest to, co ma tylko długość i szerokość.
    \item [1.6] Definicja ... % Definicja 6. % Krawędzie powierzchni są liniami.
    \item [1.7] Definicja ... % Definicja 7. % Płaska powierzchnia albo płaszczyzna jest ta, na której biorąc gdziekolwiek dwa punkty linia prosta między tymi punktami cała leży na tej powierzchni.
    \item [1.8] Definicja ... % Definicja 8. % Kąt płaski to nachylenie dwóch linii na płaszczyźnie w miejscu, w którym jedna spotyka drugą i nie leżą w linii prostej.
    \item [1.9] Definicja ... % Definicja 9. % Kiedy linie są proste i tworzą kąt, wtedy kąt zwany jest prostoliniowym.
    \item [1.10] Definicja ... % Definicja 10. % Kiedy linia prosta padająca na drugą linie prostą, tworzy z nią kąty przyległe równe między sobą, to każdy z kątów równych nazywamy prostym, a padająca linia prostą nazywa się prostopadłą do tej linii, na którą pada.
    \item [1.11] Definicja ... % Definicja 11. % Kąt rozwarty jest większy od kąta prostego.
    \item [1.12] Definicja ... % Definicja 12. % Kąt ostry jest mniejszy od kąta prostego.
    \item [1.13] Definicja ... % Definicja 13. % Kresem albo granicą jest to, na czym się dana rzecz kończy.
    \item [1.14] Definicja ... % Definicja 14. % Figurą nazywamy to co jest ograniczone granicą lub granicami.
    \item [1.15] Definicja ... % Definicja 15. % Koło jest figurą płaską zawarta linią zwaną okręgiem, do której wszystkie linie proste poprowadzone z jednego punktu wewnątrz figury położonego, są między sobą równe.
    \item [1.16] Definicja ... % Definicja 16. % I ten punkt nazywa się centrum lub środkiem koła.
    \item [1.17] Definicja ... % Definicja 17. % Średnicą koła jest każda linia narysowana przez środek koła, przedłużona w dwóch kierunkach do jego obwodu, przepoławiająca go.
    \item [1.18] Definicja ... % Definicja 18. % Półokręgiem jest figura zawarta między średnicą i częscia okręgu odciętą tą średnicą. Środek półokregu jest też środkiem okręgu.
    \item [1.19] Definicja ... % Definicja 19. % Figury prostokreślne to figury ograniczone prostymi. Trójkąt to figura prostokreślna ograniczona trzema prostymi. Czworobok lub czworokąt to figura prostokreślna, która jest ograniczona czterema prostymi. Wielobok lub wielokąt to figura prostokreślna ograniczona więcej niż czterema prostymi.
    \item [1.20] Definicja ... % Definicja 20. % Trójkąt równoboczny to trójkąt, który ma trzy boki równe. Trójkąt równoramienny to trójkąt, który ma tylko dwa boki równe. Trójkąt różnoboczny to trójkąt, który ma trzy boki różne.
    \item [1.21] Definicja ... % Definicja 21. % Ponadto: trójkąt prostokątny to trójkąt, który na kąt prosty. Trójkąt rozwartokątny to trójkąt, który ma kąt rozwarty. Trójkąt ostrokątny to trójkąt, który ma trzy kąty ostre.
    \item [1.22] Definicja ... % Definicja 22. % Kwadrat jest to czworobok mający równe boki i równe kąty. Prostokąt jest to czworobok mający kąty proste, ale boki nierówne. Romb (kwadrat ukośny) jest to czworobok mający równe boki, ale nie mający kątów prostych. Równoległobok jest to czworobok mający boki przeciwległe równe, ale nie mający katów prostych. Wszystkie czworoboki inne niż wyżej wymienione nazywamy czworokątami.
    \item [1.23] Definicja ... % Definicja 23. % Linie równoległe, czyli mówiąc krócej równoległe są to proste, które leżą na tej samej płaszczyźnie i przedłużone z obu stron w nieskończoność, z żadnej strony nie przetną się.
\end{enumerate}	
	
\subsubsection{Postulaty}	
\begin{enumerate}	
    \item [1.1] Przez każde dwa punkty przechodzi prosta.
    \item [1.2] Postulat ... % Postulat 2. % Ograniczoną prostą można przedłużyć nieskończenie.
    \item [1.3] Postulat ... % Postulat 3. % Można zakreślić okrąg z któregokolwiek punktu jako środka dowolną odległością.
    \item [1.4] Wszystkie kąty proste są sobie równe.
    \item [1.5] Postulat ... % Postulat 5. % Jeżeli prosta przecinająca dwie proste tworzy z nimi kąty jednostronnie wewnętrzne o sumie mniejszej niż dwa kąty proste, to te dwie proste przedłużone nieskończenie przecinają się po tej stronie, po której znajdują się kąty o sumie mniejszej od dwóch kątów prostych.
\end{enumerate}	
	
Jak łatwo zauważyć, sformułowanie ostatniego postulatu używa więcej słów niż pozostałe razem wzięte; wbrew przekonaniu, że postulaty miały wyrażać treści oczywiste i proste.	
Piąty postulat wydawał się bardziej skomplikowany, więc nasuwał podejrzenie, że wynika z poprzednich czterech.	
Zauważył to już Proklos zwany Diadochem (410-485):
\index[persons]{Proklos zwany Diadochem}%
\emph{,,Nie jest możliwe, aby uczony tej miary co Euklides godził się na obecność tak długiego postulatu w aksjomatyce -- obecność postulatu wzięła się z pospiesznego kończenia przez niego Elementów, tak aby zdążyć przed nadejściem słusznie oczekiwanej rychłej śmierci; my zatem -- czcząc jego pamięć -- powinniśmy ten postulat usunąć lub co najmniej znacznie uprościć.''}	
	
Wiele osób próbowało stawić czoło wyzwaniu postawionemu przez Proklosa.	
Bezskutecznie, ponieważ piąty postulat jest niezależny od pozostałych, zaś zastąpienie go jego zaprzeczeniem prowadzi do geometrii nieeuklidesowych.	
Piszą o tym Audin \cite[s. 13]{audin_2003}.
	
\subsubsection{Pojęcia pierwotne}	
\begin{enumerate}	
    \item [1.1] Wyrażenia, które są równe się temu samemu wyrażeniowi, są sobie równe.
    \item [1.2] Równania można dodawać stronami.
    \item [1.3] Równania można odejmować stronami.
    \item [1.4] Wyrażenia, które się pokrywają, są sobie równe.
    \item [1.5] Całość jest większa od części.
\end{enumerate}	
	
\subsubsection{Twierdzenia}	
\begin{enumerate}	
    \item [1.1] Skonstruować trójkąt równoboczny o zadanym boku.
    \item [1.2] Twierdzenie ... % Twierdzenie 2. % Skonstruuj odcinek równy danemu odcinkowi którego koniec jest zadanym punktem.
    \item [1.3] Skonstruować różnicę dwóch odcinków.
    \item [1.4] Twierdzenie ... \hfill \emph{(przystawanie bok-kąt-bok)} % Twierdzenie 4. % Jeśli dwa trójkąty mają dwa boki odpowiednio równe dwóm innym, i jeżeli kąty zawarte między bokami równoległymi są równe, wtedy ich podstawy również są sobie równe i pozostałe kąty równe są odpowiednim kątom.
    \index{cecha przystawania!bok-kąt-bok}%
    \item [1.5] Twierdzenie ... % Twierdzenie 5. % W trójkątach równoramiennych kąty przy podstawie są sobie równe oraz kąty powstałe przez przedłużenie boków równych są sobie równe.
    \item [1.6] Boki trójkąta leżące naprzeciw przystających kątów są przystające.
    \item [1.7] Twierdzenie ... % Twierdzenie 7. % Na tej samej podstawie i z tej samej strony nie mogą być wykreślone dwa trójkąty takie, żeby boki w tych trójkątach przy obydwu końcach wspólnej podstawy były między sobą równe.
    \item [1.8] Twierdzenie ... \hfill \emph{(przystawanie bok-bok-bok)} % Twierdzenie 8. % Jeżeli dwa boki jednego trójkąta są równe dwóm bokom drugiego trójkąta, to kąty zawarte między równymi bokami są sobie równe.
    \index{cecha przystawania!bok-bok-bok}%
    \item [1.9] Podzielić dany kąt na dwie równe części.
    \item [1.10] Podzielić dany odcinek na dwie równe części.
    \item [1.11] Twierdzenie ... % Twierdzenie 11. % Z punktu danego na danej linii prostej wyprowadzić linie prostopadłą do danej linii prostej.
    \item [1.12] Twierdzenie ... % Twierdzenie 12. % Z punktu danego leżącego poza linią prostą nieograniczoną, wyprowadzić prostą linię prostopadłą do niej.
    \item [1.13] Twierdzenie ... % Twierdzenie 13. % Jeżeli linia prosta przecinająca drugą prostą tworzy z nią dwa kąty, to są one proste, albo równe dwóm kątom prostym.
    \item [1.14] Twierdzenie ... % Twierdzenie 14. % Jeżeli przy linii prostej i przy punkcie na niej leżącym dwie linie proste nie po jednej stronie położone czynią kąty przyległe równe dwóm kątom prostym, to te linie proste będą w tym samym kierunku.
    \item [1.15] Twierdzenie ... % Twierdzenie 15. % Jeżeli dwie linie proste przecinają się, to utworzone przez nie kąty przeciwległe są sobie równe.
    \item [1.16] Twierdzenie ... % Twierdzenie 16. % W dowolnym trójkącie kąt zewnętrzny powstały przez przedłużenie jednego boku jest większy od każdego z dwóch kątów wewnętrznych przeciwległych jemu.
    \item [1.17] W każdym trójkącie suma dwóch kątów jest mniejsza od $\pi$.
    \item [1.18] Twierdzenie ... % Twierdzenie 18. % W każdym trójkącie bok większy przeciwległy jest kątowi większemu.
    \item [1.19] Twierdzenie ... % Twierdzenie 19. % W każdym trójkącie kąt większy przeciwległy jest bokowi większemu.
    \item [1.20] Twierdzenie ... % Twierdzenie 20. % W każdym trójkącie suma dwóch dowolnych boków jest większa od boku trzeciego.
    \item [1.21] Twierdzenie ... % Twierdzenie 21. % Jeżeli z końców jednego boku trójkąta poprowadzone będą dwie linie proste wewnątrz trójkąta, aż do zejścia się z sobą, to te dwie linie proste będą mniejsze od dwóch pozostałych boków trójkąta, lecz zawierać jednak będą kąt większy od kąta zawartego między pozostałymi bokami trójkąta.
    \item [1.22] Twierdzenie ... % Twierdzenie 22. % Aby z trzech danych linii prostych wykreślić trójkąt, potrzeba aby z tych trzech danych linii prostych suma dwóch którychkolwiek była większa od trzeciej.
    \item [1.23] Twierdzenie ... % Twierdzenie 23. % Na danej linii prostej i punkcie na niej danym wykreślić kąt prostokreślny równy kątowi prostokreślnemu danemu.
    \item [1.24] Twierdzenie ... % Twierdzenie 24. % Jeżeli dwa boki jednego trójkąta, są równe dwóm bokom trójkąta drugiego, z kątów zaś między bokami równymi jeden większy jest od drugiego; to będzie też podstawa jednego trójkąta większa od podstawy drugiego trójkąta.
    % TODO: https://en.wikipedia.org/wiki/Hinge_theorem
    \item [1.25] Twierdzenie ... % Twierdzenie 25. % Jeżeli dwa boki jednego trójkąta, są równe dwóm bokom trójkąta drugiego, lecz podstawa jednego trójkąta większa jest od podstawy drugiego trójkąta, to i kąty między bokami równymi zawarte będą jeden większy od drugiego.
    \item [1.26] Twierdzenie ... % Twierdzenie 26. % Jeżeli dwa kąty jednego trójkąta są równe dwóm kątom drugiego trójkąta, i bok jeden przyległy obydwu kątom, albo jednemu w pierwszym trójkącie równa się bokowi jednemu przyległemu obydwu katom, albo jednemu w drugim trójkącie; będą i dwa boki pozostałe równe dwóm bokom pozostałym i kąt trzeci w jednym trójkącie będzie równy katowi trzeciemu w drugim trójkącie.
    \item [1.27] Twierdzenie ... % Twierdzenie 27. % Jeżeli na dwie linie proste, pada linia prosta czyniąca kąty naprzemian równe między sobą, to te dwie linie proste będą równoległe.
    \item [1.28] Twierdzenie ... % Twierdzenie 28. % Jeśli linia prosta opada na dwie linie proste, tworząc kąt zewnętrzny równy wewnętrznemu i przeciwny do kąta na tym samym boku lub suma kątów wewnętrznych na tym samym boku jest równa dwóm kątom prostym, wtedy linie proste są równoległe do siebie.
    \item [1.29] Twierdzenie ... % Twierdzenie 29. % Linia prosta opada na równoległą linie prostą tworząc alternatywne kąty równe sobie, kąt zewnętrzny równy wewnętrznemu i przeciwległy i suma kątów wewnętrznych na tym samym boku jest równa dwóm kątom prostym.
    \item [1.30] Twierdzenie ... % Twierdzenie 30. % Linie proste, które są równoległe do linii prostej są również równoległe do siebie.
    \item [1.31] Twierdzenie ... % Twierdzenie 31. % Poprowadzić przez dany punkt linię prostą równoległą względem danej lini prostej.
    \item [1.32] Twierdzenie ... % Twierdzenie 32. % W jakimkolwiek trójkącie, jeśli jeden z boków jest znany wtedy kąt zewnętrzny jest równy sumie dwóch kątów wewnętrznych i przeciwnych i suma trzech wewnętrznych kątów trójkąta jest równa dwóm kątom prostym.
    \item [1.33] Twierdzenie ... % Twierdzenie 33. % Linie proste, które łączą końce równych i równoległych linii prostych w tym samym kierunku są sobie równe i równoległe.
    \item [1.34] Twierdzenie ... % Twierdzenie 34. % W równoległobokach boki i kąty przeciwne są między sobą równe, a przekątna dzieli je na dwie równe części.
    \item [1.35] Twierdzenie ... % Twierdzenie 35. % Równoległoboki, które są na takiej samej podstawie i są porównywalne są sobie równe.
    \item [1.36] Twierdzenie ... % Twierdzenie 36. % Równoległoboki, które mają równe podstawy i są porównywalne są sobie równe.
    \item [1.37] Twierdzenie ... % Twierdzenie 37. % Trójkąty, które mają takie same podstawy i są porównywalne są sobie równe.
    \item [1.38] Twierdzenie ... % Twierdzenie 38. % Trójkąty, których podstawy są równe i są one porównywalne są sobie równe.
    \item [1.39] Twierdzenie ... % Twierdzenie 39. % Równe trójkąty, które są na takich samych podstawach i mające te same boki również są porównywalne.
    \item [1.40] Twierdzenie ... % Twierdzenie 40. % Równe trójkąty, które mają takie same podstawy i mają te same boki również są porównywalne.
    \item [1.41] Twierdzenie ... % Twierdzenie 41. % Jeśli równoległobok i trójkąt mają tą samą podstawę i są tymi samymi liniami zakończone, to trójkąt jest połową równoległoboku.
    \item [1.42] Twierdzenie ... % Twierdzenie 42. % Skonstruować równoległobok równy danemu trójkątowi o podanym prostoliniowym kącie.
    \item [1.43] Twierdzenie ... % Twierdzenie 43. % W każdym równoległoboku, dopełnienia równoległoboków koło przekątnych położonych są między sobą równe.
    \item [1.44] Twierdzenie ... % Twierdzenie 44. % Na danej linii prostej wykreślić równy danemu równoległobok, którego jeden kąt będzie równy danemu.
    \item [1.45] Twierdzenie ... % Twierdzenie 45. % Wykreślić równy danej figurze prostokreślny równoległobok, którego jeden kąt będzie równy danemu.
    \item [1.46] Skonstruować kwadrat.
    \item [1.47] Twierdzenie ... \hfill \emph{(twierdzenie Pitagorasa)} % Twierdzenie 47. % W trójkącie prostokątnym, kwadrat zbudowany na boku przeciwnym kątowi prostemu, równy jest kwadratom zbudowanym na bokach, które kąt prosty zawierają.
    \index{twierdzenie!Pitagorasa}
    \item [1.48] Twierdzenie ... \hfill \emph{(twierdzenie odwrotne do twierdzenia Pitagorasa)} % Twierdzenie 48. % Jeżeli kwadrat zbudowany na jednym z boków trójkąta, jest równy kwadratom wykreślonym na dwóch pozostałych bokach trójkąta, to kąt zawarty między dwoma pozostałymi bokami będzie prosty.
\end{enumerate}

%