%

% http://aleph0.clarku.edu/~djoyce/java/elements/bookIII/bookIII.html

\subsection{Księga III}
\subsubsection{Definicje}
\begin{enumerate}
    \item [3.1] Dwa okręgi są \emph{równe}, kiedy mają równe średnice (lub równoważnie, promienie).
    \item [3.2] Prosta jest \emph{styczna} do okręgu, jeśli ma z nim punkt wspólny, ale po przedłużeniu nie ma drugiego takiego punktu.
    \item [3.3] Dwa okręgi są \emph{styczne}, kiedy mają punkt wspólny, ale nie przecinają się.
    \item [3.4] Dwie cięciwy (oryginalnie: proste) w okręgu są \emph{równoodległe} od jego środka, kiedy odcinki prostopadłe do nich opuszczone ze środka okręgu są równe.
    \item [3.5] Spośród dwóch cięciw ta jest \emph{bardziej oddalona}, do której odcinek prostopadły ze środka okręgu jest dłuższy.
    \item [3.6] \emph{Odcinek kołowy} to figura ograniczona przez cięciwę okręgu (a właściwie koła) oraz jego brzeg.
    \item [3.7] \emph{Kątem odcinka kołowego} jest kąt między cięciwą i brzegiem koła\footnote{To sformułowanie jest dziwne, ale jest potrzebne tylko raz, w twierdzeniu 3.16}.
    \item [3.8] Kątem w odcinku kołowym, co nie jest (!) tym samym, co wyżej, nazywamy kąt, którego wierzchołek leży na brzegu koła, zaś ramiona przechodzą przez cięciwę, która wyznacza odcinek kołowy.
    \item [3.9] Kąt w odcinku kołowym wygodniej jest nazywać kątem wpisanym opartym na łuku, który łączy końce odcinka kołowego (są dwa takie łuki).
    \item [3.10] Wycinek kołowy to figura ograniczona przez dwa promienie oraz łuk koła łączący te promienie.
    \item [3.11] Dwa odcinki kołowe są \emph{podobne}, kiedy mają te same kąty.
\end{enumerate}

\subsubsection{Twierdzenia}
\begin{enumerate}
    \item [3.1] Skonstruować środek danego okręgu.      
    \item [3.2] Cięciwa znajduje się wewnątrz koła.
    \item [3.3] Średnica (cięciwa przechodząca przez środek koła), która połowi inną cięciwę niebędącą średnicą, jest do niej prostopadła. Średnica, która jest prostopadła do innej cięciwy, połowi ją.
    \item [3.4] Dwie cięciwy, które nie są średnicami, nie połowią się nawzajem.
    \item [3.5] Dwa okręgi, które się przecinają, nie mogą być współśrodkowe. 
    \item [3.6] Dwa okręgi, które są styczne, nie są koncentryczne.
    \item [3.7] Dane jest koło oraz punkt $F$ leżący na jego średnicy $AD$, który nie jest środkiem. Rozpatrujemy odcinki łączące punkt $F$ z innymi punktami na brzegu koła. Wtedy:
        \begin{itemize}
        \item odcinek, który przechodzi przez środek jest najdłuższy;
        \item odcinek, który nie przechodzi przez środek, ale leży na średnicy, jest najkrótszy;
        \item odcinek, którego koniec leży dalej od wspomnianej średnicy jest dłuższy od odcinka, którego koniec leży bliżej.
        \end{itemize}
    (To twierdzenie nie jest później do niczego używane.)
    \item [3.8] Twierdzenie ...
    % Twierdzenie 8. % Jeżeli z punktu zewnątrz koła obranego, poprowadzone będą do okręgu linie proste, z których jedna przechodziła by przez środek koła a inne padały gdziekolwiek, z linii prostych padających na część okręgu wklęsłą, największa jest linia poprowadzona przez środek koła, z innych zaś linii każda bliższa przechodzącej przez środek jest większa od odleglejszej. Lecz z linii padających na cześć okręgu wypukłą, najmniejsza jest linia prosta zawarta między punktem zewnętrz koła i średnicą, z innych zaś linii prostych każda bliższa najmniejszej, mniejsza jest odleglejsza; na koniec dwie tylko równe linie proste z tego punktu po obydwu stronach najmniejszej linii prostej mogą być do okręgu poprowadzone.
    \item [3.9] Jeżeli przez punkt wewnątrz koła przechodzą trzy cięciwy równej długości, to jest środkiem koła.
    \item [3.10] Dwa okręgi, które się przecinają, przecinają się w dwóch punktach. 
    \item [3.11] Prosta łącząca środki kół stycznych wewnętrznie przechodzi przez punkt styczności kół.
    \item [3.12] Odcinek łączący środki kół stycznych zewnętrznie przechodzi przez punkt styczności kół.
    \item [! 3.13] Dwa różne okręgi nie mogą być styczne (wewnętrznie lub zewnętrznie) w więcej niż jednym punkcie.
    \item [3.14] Następujące warunki są równoważne:
        \begin{itemize}
        \item dwie cięciwy są równej długości; 
        \item dwie cięciwy są równoodległe od środka koła.
        \end{itemize}
    \item [! 3.15] Spośród dwóch cięciw ta jest dłuższa, która leży bliżej środka koła; średnica to najdłuższa cięciwa.
    \item [3.16] Prosta prostopadła do średnicy przechodząca przez jej środek nie przechodzi przez wnętrze koła. (To twierdzenie mówi też, że ,,kąt'' między prostopadłą a brzegiem koła jest mniejszy niż dowolny kąt płaski, co trudno sformalizować i nie jest później wykorzystywane).
    \item [3.17] Skonstruować styczną do danego okręgu, która przechodzi przez dany punkt.
    \item [3.18] Odcinek łączący środek koła z punktem styczności tego koła i pewnej prostej jest prostopadły do tej prostej.
    \item [3.19] Prosta prostopadła do stycznej do okręgu, przechodząca przez punkt styczności, przechodzi także przez środek koła.
    \item [3.20] Twierdzenie ...
    % Twierdzenie 20. % W kole, kąt mający wierzchołek we środku jest podwojeniem kata mającego swój wierzchołek na okręgu koła, gdyż tę samą podstawę okręgu mają za podstawę, czyli to samo gdy ramionami swymi tej samej części okręgu obejmują.
    \item [3.21] Twierdzenie ...
    % Twierdzenie 21. % Kąty w tym samym odcinku koła są między sobą równe.
    \item [3.22] Twierdzenie ...
    % Twierdzenie 22. % Kąty przeciwne czworokąta w koło wpisane są równe dwóm kątom prostym.
    \item [3.23] Twierdzenie ...
    % Twierdzenie 23. % Na tej samej linii prostej nie można wykreślić dwóch odcinków kół po tej samej stronie podobnych, które by nie przystawały do siebie.
    \item [3.24] Twierdzenie ...
    % Twierdzenie 24. % Wykreślone na równych liniach prostych podobne odcinki kół, są między sobą równe.
    \item [3.25] Twierdzenie ...
    % Twierdzenie 25. % Mając dany odcinek koła, opisać koła którego jest odcinkiem.
    \item [3.26] Twierdzenie ...
    % Twierdzenie 26. % W kołach równych, kąty równe w środkach lub przy okręgach wspierają się na równych łukach.
    \item [3.27] Twierdzenie ...
    % Twierdzenie 27. % W kołach równych, kąty we środkach lub przy okręgach, na równych łukach wspierające się, są między sobą równe.
    \item [3.28] Twierdzenie ...
    % Twierdzenie 28. % W kołach równych, cięciwy równe obejmują łuki równe, tak, że łuk większy większemu, mniejszy mniejszemu jest równy.
    \item [3.29] Twierdzenie ...
    % Twierdzenie 29. % W kołach równych, równe łuki obejmują cięciwy równe.
    \item [3.30] Podzielić dany Twierdzenie ...
    % Twierdzenie 30. % Dany łuk podzielić na dwie części.
    \item [3.31] Twierdzenie ...
    % Twierdzenie 31. % W kole, kąt w półkolu jest prosty; z katów zaś w odcinkach nierównych, kąt w większym odcinku mniejszy jest od prostego; a w mniejszym odcinku większy od prostego.
    \item [3.32] Twierdzenie ...
    % Twierdzenie 32. % Jeżeli okręgu koła dotyka linia prosta, z punktu zaś dotknięcia poprowadzona będzie cięciwa, kąty zawarte miedzy cięciwową i styczną, będą równe kątom w odcinkach koła na przemian.
    \item [3.33] Twierdzenie ...
    % Twierdzenie 33. % Na danej linii prostej wykreślić odcinek koła który by zawierał kąt równy kątowi danemu.
    \item [3.34] Twierdzenie ...
    % Twierdzenie 34. % Z koła danego oddzielić odcinek któryby zawierał kąt równy danemu kątowi.
    \item [3.35] Twierdzenie ...
    % Twierdzenie 35. % Jeżeli w kole dwie cięciwy przecinają się nawzajem, prostokąt zawarty odcinkami jednej cięciwy będzie równy prostokątowi zawartemu odcinkami drugiej cięciwy.
    \item [3.36] Twierdzenie ...
    % Twierdzenie 36. % Jeżeli z punktu za kołem obranego, poprowadzimy dwie linie proste, których jedna przecinałaby koło, a druga byłaby styczną; to prostokąt zawarty całą linia przecinającą i odcinkiem jej za kołem będzie równy kwadratowi ze stycznej.
    \item [3.37] Twierdzenie ...
    % Twierdzenie 37. % Jeżeli z dwóch linii prostych, od jednego punktu zewnątrz koła obranego poprowadzonych, jedna przecina koło, a druga pada na okrąg tego koła: i jeżeli prostokąt z całej linii przecinającej i odcinka jej za kołem będącego jest równy kwadratowi z linii padającej na okrąg koła, to linia będzie padająca na okrąg koła styczną.
\end{enumerate}

%