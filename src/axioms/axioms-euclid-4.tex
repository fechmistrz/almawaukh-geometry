%

\subsection{Księga IV}
\subsubsection{Definicje}
\begin{enumerate}
	\item [4.1] Definicja ...
	% Definicja 1. % Mówi się że figura prostokreślna wpisuje się w figurę prostokreślną, wtedy kiedy każdy kąt figury wpisanej dotyka się każdego boku figury, w który się wpisuje.
	\item [4.2] Definicja ...
	% Definicja 2. % Podobnie się mówi, że figura opisuje się na figurze, kiedy każdy bok figury opisanej dotyka każdego kąta figury na której się opisuje.
	\item [4.3] Definicja ...
	% Definicja 3. % Figura prostokreślna wpisuje się w koło, kiedy każdy kąt figury wpisanej dotyka okręgu koła.
	\item [4.4] Definicja ...
	% Definicja 4. % Figura prostokreślna opisuje się na kole kiedy każdy bok figury opisanej dotyka okręgu koła.
	\item [4.5] Definicja ...
	% Definicja 5. % Podobnież koło wpisuje się w figurę prostokreślną, kiedy każdy bok figury w którą koło się wpisuje, dotyka okręgu koła.
	\item [4.6] Definicja ...
	% Definicja 6. % Koło opisuje się na figurze prostokreślne wtedy gdy okrąg dotyka do każdego kąta figury na której opisujemy koło.
	\item [4.7] Definicja ...
	% Definicja 7. % Mówi się, że linie proste kreśli się w kole, gdy jej końce są na okręgu danego koła.
\end{enumerate}

\subsubsection{Twierdzenia}
\begin{enumerate}
	\item [4.1] Wpisać odcinek krótszy od średnicy w dany okrąg.
	\item [4.2] Twierdzenie ...
	% Twierdzenie 2. % W dane koło wpisać trójkąt równoramienny względem danego trójkąt.
	\item [4.3] Twierdzenie ...
	% Twierdzenie 3. % Na danym kole opisać trójkąt równokątny względem danego trójkąta.
	\item [4.4] Twierdzenie ...
	% Twierdzenie 4. % W dany trójkąt wpisać koło.
	\item [4.5] Opisać okrąg na danym kwadracie. \index{okrąg!opisany} % TODO OKRG NA OKREGU???
	\item [4.6] Wpisać kwadrat w dany okrąg.
	\item [4.7] Opisać kwadrat na danym okręgu.
	\item [4.8] Wpisać okrąg w dany kwadrat.
	\item [4.9] Opisać okrąg na danym kwadracie.
	\item [4.10] Wykreślić trójkąt równoramienny, którego kąt przy podstawie jest podwojeniem kąta przy wierzchołku (o kątach $\pi/5$, $2\pi/5$, $2\pi/5$; taki trójkąt nazywa się czasami złotym). \index{trójkąt!złoty}
	\item [4.11] Twierdzenie ...
	% Twierdzenie 11. % W dane koło wpisać pięciokąt równoboczny i równokątny.
	\item [4.12] Opisać pięciokąt równoboczny i równokątny na danym okręgu. \index{pięciokąt|see{wielokąt}}
	\index{wielokąt!pięciokąt}
	\item [4.13] Wpisać okrąg w dany pięciokąt równoboczny i równokątny.
	\item [4.14] Opisać okrąg na danym pięciokącie równobocznym i równokątnym
	\item [4.15] Wpisać sześciokąt równoboczny i równokątny w dany okrąg.
	\index{sześciokąt|see{wielokąt}}
	\index{wielokąt!sześciokąt}
	\item [4.16] Wpisać piętnastokąt równoboczny i równokątny w dany okrąg.
	\index{piętnastokąt|see{wielokąt}}
	\index{wielokąt!piętnastokąt}
\end{enumerate}

%