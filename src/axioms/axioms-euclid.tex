%

\section{Aksjomaty Euklidesa}
\todofoot{Considered the "father of geometry",[3] he is chiefly known for the Elements treatise, which established the foundations of geometry that largely dominated the field until the early 19th century.}
\todofoot{Very little is known of Euclid's life, and most information comes from the scholars Proclus and Pappus of Alexandria many centuries later. }
\todofoot{Simply discarding it gives absolute geometry, while negating it yields hyperbolic geometry. Other consistent axiom sets can yield other geometries, such as projective, elliptic, spherical or affine geometry.}
% https://en.wikipedia.org/wiki/Absolute_geometry
% https://en.wikipedia.org/wiki/Hyperbolic_geometry
% https://en.wikipedia.org/wiki/Projective_geometry
% https://en.wikipedia.org/wiki/Elliptic_geometry
% https://en.wikipedia.org/wiki/Spherical_geometry
% https://en.wikipedia.org/wiki/Affine_geometry
\todofoot{The theorem of the gnomon was described as early as in Euclid's Elements (around 300 BC), and there it plays an important role in the derivation of other theorems. It is given as proposition 43 in Book I of the Elements, where it is phrased as a statement about parallelograms without using the term "gnomon". The latter is introduced by Euclid as the second definition of the second book of Elements. Further theorems for which the gnomon and its properties play an important role are proposition 6 in Book II, proposition 29 in Book VI and propositions 1 to 4 in Book XIII.[5][4][6]} % https://en.wikipedia.org/wiki/Theorem_of_the_gnomon
\todofoot{Strona ,,Euclidean geometry'' na en-wiki} % https://en.wikipedia.org/wiki/Euclidean_geometry
% https://en.wikipedia.org/wiki/Euclid's_Elements

Wiele twierdzeń zawartych w Elementach Euklidesa to jawne przepisy, jak korzystając z cyrkla i~linijki otrzymać żądane figury.
Dlatego nigdzie w tekście nie pojawia się na przykład siedmiokąt foremny, chociaż w trzecim tysiącleciu każdy akceptuje, że ten istnieje.
Ale nie można go wykreślić wspomnianymi przed chwilą przyrządami, więc Euklides nie zaczyna nawet badania jego własności.

\subsection{Księga I}	
\subsubsection{Definicje}	
\begin{enumerate}	
    \item [1.1] Definicja ... % Definicja 1. % Punkt to jest to, co nie składa się z części.
    \item [1.2] Definicja ... % Definicja 2. % Linia jest długością bez szerokości.
    \item [1.3] Definicja ... % Definicja 3. % Końcami linii są punkty.
    \item [1.4] Definicja ... % Definicja 4. % Linia jest prosta, jeżeli położona jest między swoimi punktami w równym i jednostajnym kierunku.
    \item [1.5] Definicja ... % Definicja 5. % Powierzchnia jest to, co ma tylko długość i szerokość.
    \item [1.6] Definicja ... % Definicja 6. % Krawędzie powierzchni są liniami.
    \item [1.7] Definicja ... % Definicja 7. % Płaska powierzchnia albo płaszczyzna jest ta, na której biorąc gdziekolwiek dwa punkty linia prosta między tymi punktami cała leży na tej powierzchni.
    \item [1.8] Definicja ... % Definicja 8. % Kąt płaski to nachylenie dwóch linii na płaszczyźnie w miejscu, w którym jedna spotyka drugą i nie leżą w linii prostej.
    \item [1.9] Definicja ... % Definicja 9. % Kiedy linie są proste i tworzą kąt, wtedy kąt zwany jest prostoliniowym.
    \item [1.10] Definicja ... % Definicja 10. % Kiedy linia prosta padająca na drugą linie prostą, tworzy z nią kąty przyległe równe między sobą, to każdy z kątów równych nazywamy prostym, a padająca linia prostą nazywa się prostopadłą do tej linii, na którą pada.
    \item [1.11] Definicja ... % Definicja 11. % Kąt rozwarty jest większy od kąta prostego.
    \item [1.12] Definicja ... % Definicja 12. % Kąt ostry jest mniejszy od kąta prostego.
    \item [1.13] Definicja ... % Definicja 13. % Kresem albo granicą jest to, na czym się dana rzecz kończy.
    \item [1.14] Definicja ... % Definicja 14. % Figurą nazywamy to co jest ograniczone granicą lub granicami.
    \item [1.15] Definicja ... % Definicja 15. % Koło jest figurą płaską zawarta linią zwaną okręgiem, do której wszystkie linie proste poprowadzone z jednego punktu wewnątrz figury położonego, są między sobą równe.
    \item [1.16] Definicja ... % Definicja 16. % I ten punkt nazywa się centrum lub środkiem koła.
    \item [1.17] Definicja ... % Definicja 17. % Średnicą koła jest każda linia narysowana przez środek koła, przedłużona w dwóch kierunkach do jego obwodu, przepoławiająca go.
    \item [1.18] Definicja ... % Definicja 18. % Półokręgiem jest figura zawarta między średnicą i częscia okręgu odciętą tą średnicą. Środek półokregu jest też środkiem okręgu.
    \item [1.19] Definicja ... % Definicja 19. % Figury prostokreślne to figury ograniczone prostymi. Trójkąt to figura prostokreślna ograniczona trzema prostymi. Czworobok lub czworokąt to figura prostokreślna, która jest ograniczona czterema prostymi. Wielobok lub wielokąt to figura prostokreślna ograniczona więcej niż czterema prostymi.
    \item [1.20] Definicja ... % Definicja 20. % Trójkąt równoboczny to trójkąt, który ma trzy boki równe. Trójkąt równoramienny to trójkąt, który ma tylko dwa boki równe. Trójkąt różnoboczny to trójkąt, który ma trzy boki różne.
    \item [1.21] Definicja ... % Definicja 21. % Ponadto: trójkąt prostokątny to trójkąt, który na kąt prosty. Trójkąt rozwartokątny to trójkąt, który ma kąt rozwarty. Trójkąt ostrokątny to trójkąt, który ma trzy kąty ostre.
    \item [1.22] Definicja ... % Definicja 22. % Kwadrat jest to czworobok mający równe boki i równe kąty. Prostokąt jest to czworobok mający kąty proste, ale boki nierówne. Romb (kwadrat ukośny) jest to czworobok mający równe boki, ale nie mający kątów prostych. Równoległobok jest to czworobok mający boki przeciwległe równe, ale nie mający katów prostych. Wszystkie czworoboki inne niż wyżej wymienione nazywamy czworokątami.
    \item [1.23] Definicja ... % Definicja 23. % Linie równoległe, czyli mówiąc krócej równoległe są to proste, które leżą na tej samej płaszczyźnie i przedłużone z obu stron w nieskończoność, z żadnej strony nie przetną się.
\end{enumerate}	
	
\subsubsection{Postulaty}	
\begin{enumerate}	
    \item [1.1] Przez każde dwa punkty przechodzi prosta.
    \item [1.2] Postulat ... % Postulat 2. % Ograniczoną prostą można przedłużyć nieskończenie.
    \item [1.3] Postulat ... % Postulat 3. % Można zakreślić okrąg z któregokolwiek punktu jako środka dowolną odległością.
    \item [1.4] Wszystkie kąty proste są sobie równe.
    \item [1.5] Postulat ... % Postulat 5. % Jeżeli prosta przecinająca dwie proste tworzy z nimi kąty jednostronnie wewnętrzne o sumie mniejszej niż dwa kąty proste, to te dwie proste przedłużone nieskończenie przecinają się po tej stronie, po której znajdują się kąty o sumie mniejszej od dwóch kątów prostych.
\end{enumerate}	
	
Jak łatwo zauważyć, sformułowanie ostatniego postulatu używa więcej słów niż pozostałe razem wzięte; wbrew przekonaniu, że postulaty miały wyrażać treści oczywiste i proste.	
Piąty postulat wydawał się bardziej skomplikowany, więc nasuwał podejrzenie, że wynika z poprzednich czterech.	
Zauważył to już Proklos zwany Diadochem (410-485):
\index[persons]{Proklos zwany Diadochem}%
\emph{,,Nie jest możliwe, aby uczony tej miary co Euklides godził się na obecność tak długiego postulatu w aksjomatyce -- obecność postulatu wzięła się z pospiesznego kończenia przez niego Elementów, tak aby zdążyć przed nadejściem słusznie oczekiwanej rychłej śmierci; my zatem -- czcząc jego pamięć -- powinniśmy ten postulat usunąć lub co najmniej znacznie uprościć.''}	
	
Wiele osób próbowało stawić czoło wyzwaniu postawionemu przez Proklosa.	
Bezskutecznie, ponieważ piąty postulat jest niezależny od pozostałych, zaś zastąpienie go jego zaprzeczeniem prowadzi do geometrii nieeuklidesowych.	
Piszą o tym Audin \cite[s. 13]{audin_2003}.
	
\subsubsection{Pojęcia pierwotne}	
\begin{enumerate}	
    \item [1.1] Wyrażenia, które są równe się temu samemu wyrażeniowi, są sobie równe.
    \item [1.2] Równania można dodawać stronami.
    \item [1.3] Równania można odejmować stronami.
    \item [1.4] Wyrażenia, które się pokrywają, są sobie równe.
    \item [1.5] Całość jest większa od części.
\end{enumerate}	
	
\subsubsection{Twierdzenia}	
\begin{enumerate}	
    \item [1.1] Skonstruować trójkąt równoboczny o zadanym boku.
    \item [1.2] Twierdzenie ... % Twierdzenie 2. % Skonstruuj odcinek równy danemu odcinkowi którego koniec jest zadanym punktem.
    \item [1.3] Skonstruować różnicę dwóch odcinków.
    \item [1.4] Twierdzenie ... \hfill \emph{(przystawanie bok-kąt-bok)} % Twierdzenie 4. % Jeśli dwa trójkąty mają dwa boki odpowiednio równe dwóm innym, i jeżeli kąty zawarte między bokami równoległymi są równe, wtedy ich podstawy również są sobie równe i pozostałe kąty równe są odpowiednim kątom.
    \index{cecha przystawania!bok-kąt-bok}%
    \item [1.5] Twierdzenie ... % Twierdzenie 5. % W trójkątach równoramiennych kąty przy podstawie są sobie równe oraz kąty powstałe przez przedłużenie boków równych są sobie równe.
    \item [1.6] Boki trójkąta leżące naprzeciw przystających kątów są przystające.
    \item [1.7] Twierdzenie ... % Twierdzenie 7. % Na tej samej podstawie i z tej samej strony nie mogą być wykreślone dwa trójkąty takie, żeby boki w tych trójkątach przy obydwu końcach wspólnej podstawy były między sobą równe.
    \item [1.8] Twierdzenie ... \hfill \emph{(przystawanie bok-bok-bok)} % Twierdzenie 8. % Jeżeli dwa boki jednego trójkąta są równe dwóm bokom drugiego trójkąta, to kąty zawarte między równymi bokami są sobie równe.
    \index{cecha przystawania!bok-bok-bok}%
    \item [1.9] Podzielić dany kąt na dwie równe części.
    \item [1.10] Podzielić dany odcinek na dwie równe części.
    \item [1.11] Twierdzenie ... % Twierdzenie 11. % Z punktu danego na danej linii prostej wyprowadzić linie prostopadłą do danej linii prostej.
    \item [1.12] Twierdzenie ... % Twierdzenie 12. % Z punktu danego leżącego poza linią prostą nieograniczoną, wyprowadzić prostą linię prostopadłą do niej.
    \item [1.13] Twierdzenie ... % Twierdzenie 13. % Jeżeli linia prosta przecinająca drugą prostą tworzy z nią dwa kąty, to są one proste, albo równe dwóm kątom prostym.
    \item [1.14] Twierdzenie ... % Twierdzenie 14. % Jeżeli przy linii prostej i przy punkcie na niej leżącym dwie linie proste nie po jednej stronie położone czynią kąty przyległe równe dwóm kątom prostym, to te linie proste będą w tym samym kierunku.
    \item [1.15] Twierdzenie ... % Twierdzenie 15. % Jeżeli dwie linie proste przecinają się, to utworzone przez nie kąty przeciwległe są sobie równe.
    \item [1.16] Twierdzenie ... % Twierdzenie 16. % W dowolnym trójkącie kąt zewnętrzny powstały przez przedłużenie jednego boku jest większy od każdego z dwóch kątów wewnętrznych przeciwległych jemu.
    \item [1.17] W każdym trójkącie suma dwóch kątów jest mniejsza od $\pi$.
    \item [1.18] Twierdzenie ... % Twierdzenie 18. % W każdym trójkącie bok większy przeciwległy jest kątowi większemu.
    \item [1.19] Twierdzenie ... % Twierdzenie 19. % W każdym trójkącie kąt większy przeciwległy jest bokowi większemu.
    \item [1.20] Twierdzenie ... % Twierdzenie 20. % W każdym trójkącie suma dwóch dowolnych boków jest większa od boku trzeciego.
    \item [1.21] Twierdzenie ... % Twierdzenie 21. % Jeżeli z końców jednego boku trójkąta poprowadzone będą dwie linie proste wewnątrz trójkąta, aż do zejścia się z sobą, to te dwie linie proste będą mniejsze od dwóch pozostałych boków trójkąta, lecz zawierać jednak będą kąt większy od kąta zawartego między pozostałymi bokami trójkąta.
    \item [1.22] Twierdzenie ... % Twierdzenie 22. % Aby z trzech danych linii prostych wykreślić trójkąt, potrzeba aby z tych trzech danych linii prostych suma dwóch którychkolwiek była większa od trzeciej.
    \item [1.23] Twierdzenie ... % Twierdzenie 23. % Na danej linii prostej i punkcie na niej danym wykreślić kąt prostokreślny równy kątowi prostokreślnemu danemu.
    \item [1.24] Twierdzenie ... % Twierdzenie 24. % Jeżeli dwa boki jednego trójkąta, są równe dwóm bokom trójkąta drugiego, z kątów zaś między bokami równymi jeden większy jest od drugiego; to będzie też podstawa jednego trójkąta większa od podstawy drugiego trójkąta.
    % TODO: https://en.wikipedia.org/wiki/Hinge_theorem
    \item [1.25] Twierdzenie ... % Twierdzenie 25. % Jeżeli dwa boki jednego trójkąta, są równe dwóm bokom trójkąta drugiego, lecz podstawa jednego trójkąta większa jest od podstawy drugiego trójkąta, to i kąty między bokami równymi zawarte będą jeden większy od drugiego.
    \item [1.26] Twierdzenie ... % Twierdzenie 26. % Jeżeli dwa kąty jednego trójkąta są równe dwóm kątom drugiego trójkąta, i bok jeden przyległy obydwu kątom, albo jednemu w pierwszym trójkącie równa się bokowi jednemu przyległemu obydwu katom, albo jednemu w drugim trójkącie; będą i dwa boki pozostałe równe dwóm bokom pozostałym i kąt trzeci w jednym trójkącie będzie równy katowi trzeciemu w drugim trójkącie.
    \item [1.27] Twierdzenie ... % Twierdzenie 27. % Jeżeli na dwie linie proste, pada linia prosta czyniąca kąty naprzemian równe między sobą, to te dwie linie proste będą równoległe.
    \item [1.28] Twierdzenie ... % Twierdzenie 28. % Jeśli linia prosta opada na dwie linie proste, tworząc kąt zewnętrzny równy wewnętrznemu i przeciwny do kąta na tym samym boku lub suma kątów wewnętrznych na tym samym boku jest równa dwóm kątom prostym, wtedy linie proste są równoległe do siebie.
    \item [1.29] Twierdzenie ... % Twierdzenie 29. % Linia prosta opada na równoległą linie prostą tworząc alternatywne kąty równe sobie, kąt zewnętrzny równy wewnętrznemu i przeciwległy i suma kątów wewnętrznych na tym samym boku jest równa dwóm kątom prostym.
    \item [1.30] Twierdzenie ... % Twierdzenie 30. % Linie proste, które są równoległe do linii prostej są również równoległe do siebie.
    \item [1.31] Twierdzenie ... % Twierdzenie 31. % Poprowadzić przez dany punkt linię prostą równoległą względem danej lini prostej.
    \item [1.32] Twierdzenie ... % Twierdzenie 32. % W jakimkolwiek trójkącie, jeśli jeden z boków jest znany wtedy kąt zewnętrzny jest równy sumie dwóch kątów wewnętrznych i przeciwnych i suma trzech wewnętrznych kątów trójkąta jest równa dwóm kątom prostym.
    \item [1.33] Twierdzenie ... % Twierdzenie 33. % Linie proste, które łączą końce równych i równoległych linii prostych w tym samym kierunku są sobie równe i równoległe.
    \item [1.34] Twierdzenie ... % Twierdzenie 34. % W równoległobokach boki i kąty przeciwne są między sobą równe, a przekątna dzieli je na dwie równe części.
    \item [1.35] Twierdzenie ... % Twierdzenie 35. % Równoległoboki, które są na takiej samej podstawie i są porównywalne są sobie równe.
    \item [1.36] Twierdzenie ... % Twierdzenie 36. % Równoległoboki, które mają równe podstawy i są porównywalne są sobie równe.
    \item [1.37] Twierdzenie ... % Twierdzenie 37. % Trójkąty, które mają takie same podstawy i są porównywalne są sobie równe.
    \item [1.38] Twierdzenie ... % Twierdzenie 38. % Trójkąty, których podstawy są równe i są one porównywalne są sobie równe.
    \item [1.39] Twierdzenie ... % Twierdzenie 39. % Równe trójkąty, które są na takich samych podstawach i mające te same boki również są porównywalne.
    \item [1.40] Twierdzenie ... % Twierdzenie 40. % Równe trójkąty, które mają takie same podstawy i mają te same boki również są porównywalne.
    \item [1.41] Twierdzenie ... % Twierdzenie 41. % Jeśli równoległobok i trójkąt mają tą samą podstawę i są tymi samymi liniami zakończone, to trójkąt jest połową równoległoboku.
    \item [1.42] Twierdzenie ... % Twierdzenie 42. % Skonstruować równoległobok równy danemu trójkątowi o podanym prostoliniowym kącie.
    \item [1.43] Twierdzenie ... % Twierdzenie 43. % W każdym równoległoboku, dopełnienia równoległoboków koło przekątnych położonych są między sobą równe.
    \item [1.44] Twierdzenie ... % Twierdzenie 44. % Na danej linii prostej wykreślić równy danemu równoległobok, którego jeden kąt będzie równy danemu.
    \item [1.45] Twierdzenie ... % Twierdzenie 45. % Wykreślić równy danej figurze prostokreślny równoległobok, którego jeden kąt będzie równy danemu.
    \item [1.46] Skonstruować kwadrat.
    \item [1.47] Twierdzenie ... \hfill \emph{(twierdzenie Pitagorasa)} % Twierdzenie 47. % W trójkącie prostokątnym, kwadrat zbudowany na boku przeciwnym kątowi prostemu, równy jest kwadratom zbudowanym na bokach, które kąt prosty zawierają.
    \index{twierdzenie!Pitagorasa}
    \item [1.48] Twierdzenie ... \hfill \emph{(twierdzenie odwrotne do twierdzenia Pitagorasa)} % Twierdzenie 48. % Jeżeli kwadrat zbudowany na jednym z boków trójkąta, jest równy kwadratom wykreślonym na dwóch pozostałych bokach trójkąta, to kąt zawarty między dwoma pozostałymi bokami będzie prosty.
\end{enumerate}

%
\input{axioms/axioms-euclid-2}
%

\subsection{Księga III}
\subsubsection{Definicje}
\begin{enumerate}
    \item [3.1] Dwa okręgi są przystające, kiedy mają równe średnice (lub równoważnie, promienie).
    \item [3.2] Definicja ...
    % Definicja 2. % Mówi się, że linia prosta dotyka koła, gdy będąc styczną z kołem przedłużona z obydwu stron nie przecina się z żadnej strony okręgu koła.
    \item [3.3] Dwa okręgi nazywamy stycznymi, kiedy mają dokładnie jeden punkt wspólny.
    \item [3.4] Definicja ...
    % Definicja 4. % Mówi się, że linie proste równoodległe są od środka koła, gdy prostopadłe ze środka koła na nie spuszczone są równe.
    \item [3.5] Definicja ...
    % Definicja 5. % Mówi się, że ta linia prosta bardziej jest odległa od środka koła, na którą prostopadła ze środka koła spuszczona jest większa.
    \item [3.6] Definicja ...
    % Definicja 6. % Odcinkiem koła jest figura czyli część koła ograniczona linią prostą i okręgiem koła.
    \item [3.7] Definicja ...
    % Definicja 7. % Kąt zaś odcinka jest ten, który się linią prostą i okręgiem koła zawiera.
    \item [3.8] Definicja ...
    % Definicja 8. % Jeżeli na okręgu koła wzięty będzie punkt i od niego będą poprowadzone linie proste do końców linii prostej za podstawę odcinkami służącej, kąt między tymi liniami prostymi zawarty jest kątem w odcinku.
    \item [3.9] Definicja ...
    % Definicja 9. % Kiedy zaś linie proste kąt zawierające zajmują część okręgu, mówi się, że kąt ten opiera się na okręgu koła.
    \item [3.10] Definicja ...
    % Definicja 10. % Jeżeli kąt ma swój wierzchołek we środku koła; figura czyli część koła zawarta między ramionami tegoż koła, to jest między promieniami i łukiem koła nazywa się wycinkiem koła.
    \item [3.11] Definicja ...
    % Definicja 11. % Odcinkami podobnymi kół nazywają się te, które zajmują kąty równe, lub w których kąty są równe między sobą.
\end{enumerate}

\subsubsection{Twierdzenia}
\begin{enumerate}
    \item [3.1] Skonstruować środek danego okręgu. 
    \item [3.2] Twierdzenie ...
    % Twierdzenie 2. % Jeżeli na okręgu obierzemy dwa gdziekolwiek punkty, linia prosta łącząca te punkty padnie wewnątrz koła.
    \item [3.3] Twierdzenie ...
    % Twierdzenie 3. % Jeżeli w kole linia prosta przez środek poprowadzona przecina linie nie przez środek poprowadzoną na dwie równe części, będzie pierwsza prostopadła do drugiej; i jeżeli pierwsza jest prostopadła do drugiej, przecina ja na dwie równe części.
    \item [3.4] Twierdzenie ...
    % Twierdzenie 4. % Jeżeli w kole dwie linie proste, nie przez środek koła poprowadzone przecinają się nawzajem, nie przetną się na dwie równe części.
    \item [3.5] Dwa okręgi, które się przecinają, nie mogą być współśrodkowe. 
    \item [3.6] Twierdzenie ...
    % Twierdzenie 6. % Jeżeli dwa koła dotykają się wzajemnie, to wspólnego środka mieć nie mogą.
    \item [3.7] Twierdzenie ...
    % Twierdzenie 7. % Jeżeli na średnicy koła wzięty będzie punkt którykolwiek oprócz średnicy koła i od tego punktu poprowadzone linie proste do okręgu, ze wszystkich linii największa będzie część średnicy, na której znajduje się środek koła, a najmniejsza pozostała część średnicy, z innych zaś linii prostych każda bliższa przechodząca przez środek koła, większa będzie od odleglejszej, z tego na koniec punktu dwie tylko równe linie proste z obydwu stron najmniejszej linii prostej mogą być do okręgu poprowadzone.
    \item [3.8] Twierdzenie ...
    % Twierdzenie 8. % Jeżeli z punktu zewnątrz koła obranego, poprowadzone będą do okręgu linie proste, z których jedna przechodziła by przez środek koła a inne padały gdziekolwiek, z linii prostych padających na część okręgu wklęsłą, największa jest linia poprowadzona przez środek koła, z innych zaś linii każda bliższa przechodzącej przez środek jest większa od odleglejszej. Lecz z linii padających na cześć okręgu wypukłą, najmniejsza jest linia prosta zawarta między punktem zewnętrz koła i średnicą, z innych zaś linii prostych każda bliższa najmniejszej, mniejsza jest odleglejsza; na koniec dwie tylko równe linie proste z tego punktu po obydwu stronach najmniejszej linii prostej mogą być do okręgu poprowadzone.
    \item [3.9] Twierdzenie ...
    % Twierdzenie 9. % Jeżeli z punktu danego wewnątrz koła poprowadzimy do okręgu więcej niż dwie linie proste i te proste są miedzy sobą równe, punkt ten będzie środkiem koła.
    \item [3.10] Dwa okręgi, które się przecinają, przecinają się w dwóch punktach. 
    \item [3.11] Twierdzenie ...
    % Twierdzenie 11. % Jeżeli dwa koła stykają się ze sobą wewnątrz, linia łącząca środki tychże kół przedłużona pada na punkt dotykania się kół.
    \item [3.12] Twierdzenie ...
    % Twierdzenie 12. % Jeżeli dwa koła dotykają się ze sobą zewnętrznie, to linia prosta łącząca ich środki przechodzi przez punkt dotykania się.
    \item [3.13] Twierdzenie ...
    % Twierdzenie 13. % Okrąg koła nie może dotykać okręgu drugiego koła w więcej niż jednym punkcie, nieważne jest czy dotkniecie jest zewnętrzne bądź wewnętrzne.
    \item [3.14] Twierdzenie ...
    % Twierdzenie 14. % W kole linie proste równe, na okręgu jego zakończone, są równoodległe od środka; i linie proste które na okręgu jego zakończone są równoodległe od środka, są też miedzy sobą równe.
    \item [3.15] Twierdzenie ...
    % Twierdzenie 15. % Ze wszystkich linii prostych w kole poprowadzonych i na okręgu jego zakończonych, największa jest średnica, z innych zaś każda bliższa środka koła, większa jest od odleglejszej; i z dwóch linii prostych nierównych, większa bliższa jest środka koła od mniejszej.
    \item [3.16] Twierdzenie ... 
    % Twierdzenie 16. % Prostopadła do średnicy koła z końca jej wyprowadzona, pada cała zewnątrz koła, a między tą prostopadłą i okręgiem żadna inna linia prosta nie pada; albo tak samo: okrąg koła przechodzi miedzy prostopadłą do średnicy i linią prostą, która ze średnicą kąt ostry jakokolwiek wielki zawiera, czyli która zawiera kąt jakokolwiek mały z prostopadłą do średnicy.
    \item [3.17] Skonstruować styczną do danego okręgu, która przechodzi przez dany punkt.
    \item [3.18] Twierdzenie ...
    % Twierdzenie 18. % Jeżeli linia prosta dotyka się okręgu koła, a ze środka koła wyprowadzona będzie linia prosta do punktu dotykania się, to ta będzie prostopadła do stycznej.
    \item [3.19] Twierdzenie ...
    % Twierdzenie 19. % Jeżeli linia prosta dotyka okręgu koła, z punktu zaś dotknięcia wyprowadzona będzie do tej stycznej prostopadła, to na prostopadłej będzie środek koła.
    \item [3.20] Twierdzenie ...
    % Twierdzenie 20. % W kole, kąt mający wierzchołek we środku jest podwojeniem kata mającego swój wierzchołek na okręgu koła, gdyż tę samą podstawę okręgu mają za podstawę, czyli to samo gdy ramionami swymi tej samej części okręgu obejmują.
    \item [3.21] Twierdzenie ...
    % Twierdzenie 21. % Kąty w tym samym odcinku koła są między sobą równe.
    \item [3.22] Twierdzenie ...
    % Twierdzenie 22. % Kąty przeciwne czworokąta w koło wpisane są równe dwóm kątom prostym.
    \item [3.23] Twierdzenie ...
    % Twierdzenie 23. % Na tej samej linii prostej nie można wykreślić dwóch odcinków kół po tej samej stronie podobnych, które by nie przystawały do siebie.
    \item [3.24] Twierdzenie ...
    % Twierdzenie 24. % Wykreślone na równych liniach prostych podobne odcinki kół, są między sobą równe.
    \item [3.25] Twierdzenie ...
    % Twierdzenie 25. % Mając dany odcinek koła, opisać koła którego jest odcinkiem.
    \item [3.26] Twierdzenie ...
    % Twierdzenie 26. % W kołach równych, kąty równe w środkach lub przy okręgach wspierają się na równych łukach.
    \item [3.27] Twierdzenie ...
    % Twierdzenie 27. % W kołach równych, kąty we środkach lub przy okręgach, na równych łukach wspierające się, są między sobą równe.
    \item [3.28] Twierdzenie ...
    % Twierdzenie 28. % W kołach równych, cięciwy równe obejmują łuki równe, tak, że łuk większy większemu, mniejszy mniejszemu jest równy.
    \item [3.29] Twierdzenie ...
    % Twierdzenie 29. % W kołach równych, równe łuki obejmują cięciwy równe.
    \item [3.30] Podzielić dany Twierdzenie ...
    % Twierdzenie 30. % Dany łuk podzielić na dwie części.
    \item [3.31] Twierdzenie ...
    % Twierdzenie 31. % W kole, kąt w półkolu jest prosty; z katów zaś w odcinkach nierównych, kąt w większym odcinku mniejszy jest od prostego; a w mniejszym odcinku większy od prostego.
    \item [3.32] Twierdzenie ...
    % Twierdzenie 32. % Jeżeli okręgu koła dotyka linia prosta, z punktu zaś dotknięcia poprowadzona będzie cięciwa, kąty zawarte miedzy cięciwową i styczną, będą równe kątom w odcinkach koła na przemian.
    \item [3.33] Twierdzenie ...
    % Twierdzenie 33. % Na danej linii prostej wykreślić odcinek koła który by zawierał kąt równy kątowi danemu.
    \item [3.34] Twierdzenie ...
    % Twierdzenie 34. % Z koła danego oddzielić odcinek któryby zawierał kąt równy danemu kątowi.
    \item [3.35] Twierdzenie ...
    % Twierdzenie 35. % Jeżeli w kole dwie cięciwy przecinają się nawzajem, prostokąt zawarty odcinkami jednej cięciwy będzie równy prostokątowi zawartemu odcinkami drugiej cięciwy.
    \item [3.36] Twierdzenie ...
    % Twierdzenie 36. % Jeżeli z punktu za kołem obranego, poprowadzimy dwie linie proste, których jedna przecinałaby koło, a druga byłaby styczną; to prostokąt zawarty całą linia przecinającą i odcinkiem jej za kołem będzie równy kwadratowi ze stycznej.
    \item [3.37] Twierdzenie ...
    % Twierdzenie 37. % Jeżeli z dwóch linii prostych, od jednego punktu zewnątrz koła obranego poprowadzonych, jedna przecina koło, a druga pada na okrąg tego koła: i jeżeli prostokąt z całej linii przecinającej i odcinka jej za kołem będącego jest równy kwadratowi z linii padającej na okrąg koła, to linia będzie padająca na okrąg koła styczną.
\end{enumerate}

%
%

\subsubsection{Księga IV}
\paragraph{Definicje}
\begin{enumerate}
	\item [4.1] Definicja ...
	% Definicja 1. % Mówi się że figura prostokreślna wpisuje się w figurę prostokreślną, wtedy kiedy każdy kąt figury wpisanej dotyka się każdego boku figury, w który się wpisuje.
	\item [4.2] Definicja ...
	% Definicja 2. % Podobnie się mówi, że figura opisuje się na figurze, kiedy każdy bok figury opisanej dotyka każdego kąta figury na której się opisuje.
	\item [4.3] Definicja ...
	% Definicja 3. % Figura prostokreślna wpisuje się w koło, kiedy każdy kąt figury wpisanej dotyka okręgu koła.
	\item [4.4] Definicja ...
	% Definicja 4. % Figura prostokreślna opisuje się na kole kiedy każdy bok figury opisanej dotyka okręgu koła.
	\item [4.5] Definicja ...
	% Definicja 5. % Podobnież koło wpisuje się w figurę prostokreślną, kiedy każdy bok figury w którą koło się wpisuje, dotyka okręgu koła.
	\item [4.6] Definicja ...
	% Definicja 6. % Koło opisuje się na figurze prostokreślne wtedy gdy okrąg dotyka do każdego kąta figury na której opisujemy koło.
	\item [4.7] Definicja ...
	% Definicja 7. % Mówi się, że linie proste kreśli się w kole, gdy jej końce są na okręgu danego koła.
\end{enumerate}

\paragraph{Twierdzenia}
\begin{enumerate}
	\item [4.1] Wpisać odcinek krótszy od średnicy w dany okrąg.
	\item [4.2] Twierdzenie ...
	% Twierdzenie 2. % W dane koło wpisać trójkąt równoramienny względem danego trójkąt.
	\item [4.3] Twierdzenie ...
	% Twierdzenie 3. % Na danym kole opisać trójkąt równokątny względem danego trójkąta.
	\item [4.4] Twierdzenie ...
	% Twierdzenie 4. % W dany trójkąt wpisać koło.
	\item [4.5] Twierdzenie ...
	% Twierdzenie 5. % Na danym trójkącie opisać koło.
	\item [4.6] Wpisać kwadrat w dany okrąg.
	\item [4.7] Opisać kwadrat na danym okręgu.
	\item [4.8] Wpisać okrąg w dany kwadrat.
	\item [4.9] Opisać okrąg na danym kwadracie.
	\item [4.10] Wykreślić trójkąt równoramienny, którego kąt przy podstawie jest podwojeniem kąta przy wierzchołku (o kątach $\pi/5$, $2\pi/5$, $2\pi/5$).
	\item [4.11] Twierdzenie ...
	% Twierdzenie 11. % W dane koło wpisać pięciokąt równoboczny i równokątny.
	\item [4.12] Opisać pięciokąt równoboczny i równokątny na danym okręgu.
	\item [4.13] Wpisać okrąg w dany pięciokąt równoboczny i równokątny.
	\item [4.14] Opisać okrąg na danym pięciokącie równobocznym i równokątnym
	\item [4.15] Wpisać sześciokąt równoboczny i równokątny w dany okrąg.
	\item [4.16] Wpisać piętnastokąt równoboczny i równokątny w dany okrąg.
\end{enumerate}

%

% TODO: https://kpbc.umk.pl/dlibra/publication/37/edition/66/content
% TODO: Pojęcia pierwotne i aksjomaty Euklidesa nie są jednak idealne.
% TODO: Dlatego zamiast nich będziemy używać aksjomatów Hilberta podanych około 1899 roku.

% https://www.claymath.org/library/historical/euclid/
% BOOK I	Triangles, parallels, and area
% BOOK II	Geometric algebra
% BOOK III	Circles
% BOOK IV	Constructions for inscribed and circumscribed figures
% BOOK V	Theory of proportions
% BOOK VI	Similar figures and proportions
% BOOK VII	Fundamentals of number theory
% BOOK VIII	Continued proportions in number theory
% BOOK IX	Number theory
% BOOK X	Classification of incommensurables
% BOOK XI	Solid geometry
% BOOK XII	Measurement of figures
% BOOK XIII	Regular solids

% TODO: https://en.wikipedia.org/wiki/Arbelos (tego może nie być u Euklidesa)

%