%

\subsubsection{Aksjomaty leżenia pomiędzy}
\begin{axiom}[leżenia pomiędzy, B1]
    Jeśli punkt $B$ leży między punktami $A$ i $C$, to punkty $A$, $B$, $C$ są różnymi punktami tej samej prostej oraz punkt $B$ leży także między punktami $C$ i $A$.
\end{axiom}

\begin{axiom}[leżenia pomiędzy, B2]
    Dla każdej pary punktów $A$ i $B$ istnieje punkt $C$ taki, że punkt $B$ leży między punktami $A$ i $C$.
\end{axiom}

\begin{axiom}[leżenia pomiędzy, B3]
    Spośród trzech punktów leżących na prostej, dokładnie jeden leży pomiędzy pozostałymi dwoma.
\end{axiom}

\begin{axiom}[leżenia pomiędzy, B4]
    Niech $A$, $B$ i $C$ będą trzema niewspółliniowymi punktami, zaś $l$ prostą, która nie przechodzi przez żaden z nich.
    Jeśli prosta $l$ przechodzi przez punkt między punktami $A$ i $B$, to przechodzi też przez punkt między punktami $A$ i $C$ albo $B$ i $C$, ale nie przez obydwa.
\end{axiom}

Powyższy aksjomat nazywany jest też aksjomatem Pascha, ponieważ Moritz Pasch \cite{pasch_1882} przyłapał dopiero w 1882 roku geometrów całego świata na tym, że korzystali z takiej przesłanki.
\index{aksjomat!Pascha}%
\index[persons]{Pasch, Moritz}%

\begin{proposition}
    Z aksojmatów I1, I2, I3, B1, B2, B3, B4 wynika, że każda prosta ma nieskończenie wiele punktów.
\end{proposition}

\begin{definition}[odcinek]
    Niech $A$, $B$ będą punktami.
    Zbiór złożony z punktów $A$, $B$ oraz punktów, które leżą między nimi, nazywamy odcinkiem i oznaczamy $\overline {AB}$.
\end{definition} % Hartshorne 74

\begin{definition}[trójkąt]
    Niech $A$, $B$, $C$ będą punktami.
    Sumę odcinków $AB$, $BC$, $AC$ nazywamy trójkątem, wspomniane odcinki -- jego bokami, zaś punkty $A$, $B$ i $C$ -- wierzchołkami.
\end{definition} % Hartshorne 74

\begin{proposition}
    Niech $l$ będzie prostą.
    Wtedy zbiór punktów, które nie leżą na prostej $l$ można rozbić na dwa niepuste zbiory $S_1$, $S_2$ takie, że: dwa punkty, które nie leżą na prostej $l$, należą do tego samego zbioru ($S_1$ lub $S_2$) wtedy i~tylko wtedy, gdy odcinek $AB$ nie przecina prostej $l$.
\end{proposition} % Hartshorne 74

Zbiory $S_1$, $S_2$ nazywamy stronami prostej $l$.
Podobnie punkt wyznacza na prostej dwa zbiory, które leżą po różnych stronach tego punktu.

\begin{definition}[półprosta]
    Niech $A$, $B$ będą punktami.
    Zbiór złożony z punktów $A$, $B$ oraz punktów, które leżą po tej samej stronie punktu $A$ na prostej $AB$ co punkt $B$, nazywamy półprostą i oznaczamy $NIE WIEM JAK AB$.
\end{definition} % Hartshorne 77

\begin{definition}[kąt]
    Sumę dwóch półprostych $AB$, $AC$, które nie leżą na jednej prostej, nazywamy kątem, zaś punkt $A$ wierzchołkiem tego kąta.
    Wnętrze kąta $\angle BACS$ składa się z tych punktów $D$ takich, że $D$ i $C$ leżą po tej samej stronie prostej $AB$ oraz $D$ i $B$ leżą po tej samej stronie prostej $AC$.
\end{definition} % Hartshorne 77

W myśl tej definicji, nie ma kąta zerowego ani półpełnego.
Wnętrze trójkąta $ABC$ to część wspólna wnętrz kątów $\angle ABC$, $\angle BCA$, $\angle CAB$; jest wypukłe i niepuste.

%