%

\subsubsection{Aksjomaty incydencji}
\begin{axiom}[incydencji, I1]
    Dla każdej pary punktów $A$ oraz $B$ istnieje dokładnie jedna prosta $l$, która przechodzi przez te punkty.
\end{axiom}

\begin{axiom}[incydencji, I2]
    Na każdej prostej istnieją co najmniej dwa punkty.
\end{axiom}

\begin{axiom}[incydencji, I3]
    Istnieją co najmniej trzy punkty, które nie są współliniowe.
\end{axiom}

,,Być współliniowym'' to synonim leżenia na jednej prostej.

\begin{proposition}
    Dwie różne proste mogą mieć co najwyżej jeden punkt wspólny.
\end{proposition}

\begin{definition}
    Dwie różne proste, które nie mają punktów wspólnych, nazywamy równoległymi.
    Każda prosta jest też równoległa do siebie.
\end{definition}

\begin{axiom}[Playfaira, P]
    Dla każdej prostej $l$ oraz punktu $A$, istnieje co najwyżej jedna prosta przechodząca przez $A$, równoległa do $l$.
\end{axiom}

John Playfair opublikował ten aksjomat w 1795 roku, chociaż już wtedy twierdził, że inni używali go przed nim (np. William Ludlam).
\index[persons]{Playfair, John}%
Aksjomat Playfaira jest równoważny z (piątym) postulatem Euklidesa i dlatego wiele osób próbowało wyprowadzić go z czterech wcześniejszych postulatów.
Za każdym razem okazywało się, że w ,,dowodzie'' użyte jest zdanie będące równoważnikiem piątego postulatu, na przykład:
\begin{itemize}
    \item suma kątów wewnętrznych każdego trójkąta jest kątem półpełnym,
    \item suma kątów wewnętrznych każdego trójkąta jest taka sama,
    \item istnieje trójkąt, którego suma kątów wewnętrznych jest kątem półpełnym,
    \item istnieją dwa trójkąty, które są do siebie podobne, ale nie przystające,
    \item na każdym trójkącie można opisać okrąg,
    \item jeśli trzy kąty wewnętrzne czworokąta są proste, to czwarty kąt także jest prosty,
    \item istnieje para prostych, które są w stałej odległości od siebie,
    \item dwie proste, które są równoległe do danej prostej, są też równoległe do siebie,
    \item twierdzenie Pitagorasa lub jego uogólnienie, twierdzenie cosinusów
    \item aksjomat Wallisa: nie ma górnego ograniczenia na pole trójkąta,
    \item kąty przy górnej podstawie czworokąta Saccheriego (czworokąta $ABCD$ o dwóch kątach prostych przy dolnej podstawie $AB$, w którym boki $AD$ i $BC$ mają równe długości) są proste,
    \item aksjomat Proklosa: jeśli prosta przecina jedną z dwóch równoległych prostych, to przecina także tę drugą.
\end{itemize}

% TODO: https://en.wikipedia.org/wiki/Parallel_postulate

\begin{example}
    Rozważmy geometrię, gdzie płaszczyzna składa się z pięciu punktów $A$, $B$, $C$, $D$, $E$ i leży na niej dziesięć różnych prostych, każda z nich przechodząca przez dokładnie dwa różne punkty.
    Wtedy proste $AB$ i $AC$ mają punkt wspólny $A$, chociaż obydwie są równoległe do prostej $DE$.
    Aksjomat Playfaira nie jest spełniony.
\end{example}

\begin{proposition}
    Aksjomaty incydencji I1, I2, I3 oraz aksjomat P (Playfaira) są od siebie niezależne.
\end{proposition}

Hartshorne \cite[s. 69-70]{hartshorne2000} konstruuje modele geometrii, w których spełnione są dowolne trzy, ale nie czwarty z nich.

\begin{proposition}
    Płaszczyzna rzutowa to taki zbiór punktów oraz prostych (podzbiorów zbioru punktów), że:
    \begin{itemize}
        \item przez dwa różne punkty przechodzi dokładnie jedna prosta,
        \item każde dwie proste mają punkt wspólny,
        \item każda prosta ma co najmniej trzy punkty i
        \item nie wszystkie punkty są współliniowe.
    \end{itemize}
    (Każdy z wymienionych aksjomatów jest niezależny od pozostałych, ze wszystkich razem wynikają aksjomaty incydencji).
    Każda płaszczyzna rzutowa ma co najmniej 7 punktów, dokładnie jedna płaszczyzna rzutowa ma dokładnie 7 punktów.
    Jeśli istnieje prosta, która ma $n+1$ punktów, to płaszczyzna ma $n^2 + n + 1$ punktów.
\end{proposition} % Hartshorne 71

\begin{example}[płaszczyzna Fana]
    Zbiór złożony z siedmiu elementów (,,punktów''), w którym wyróżniono rodzinę siedmiu podzbiorów (,,prostych'') spełniający następujące warunki:
    \begin{itemize}
        \item każde dwie różne proste mają dokładnie jeden punkt wspólny,
        \item każde dwa różne punkty należą do dokładnie jednej prostej
    \end{itemize}
    nazywamy płaszczyzną Fana.
    Jest przykładem płaszczyzny rzutowej, która nie spełnia aksjomatu Fana (że trzy punkty przekątne czworoboku zupełnego nie są współliniowe).
    \begin{figure}[H]
        \centering
        \begin{tikzpicture}[
        mydot/.style={
        draw,
        circle,
        fill=black,
        inner sep=1.5pt}
        ]
        \draw
        (0,0) coordinate (A) --
        (4cm,0) coordinate (B) --
        ($ (A)!.5!(B) ! {sin(60)*2} ! 90:(B) $) coordinate (C) -- cycle;
        \coordinate (O) at
        (barycentric cs:A=1,B=1,C=1);
        \draw (O) circle [radius=4cm*1.717/6];
        \draw (C) -- ($ (A)!.5!(B) $) coordinate (LC); 
        \draw (A) -- ($ (B)!.5!(C) $) coordinate (LA); 
        \draw (B) -- ($ (C)!.5!(A) $) coordinate (LB); 
        \foreach \Nodo in {A,B,C,O,LC,LA,LB}
        \node[mydot] at (\Nodo) {};    
    \end{tikzpicture}%
    \caption{Płaszczyzna Fana}
\end{figure}
\end{example}

W momencie pisania tego tekstu nie jesteśmy jeszcze zainteresowani geometrią rzutową, więc przytoczymy tylko wymagania wobec studenta, który ukończył kurs ,,Geometria III'' na uniwersytecie w Warszawie.

Zna pojęcie płaszczyzny rzutowej rzeczywistej (równoważne sformułowania), dwustosunku, definicję przekształceń rzutowych łańcuchów, stożkowych, pęków stycznych do stożkowych.
\index{dwustosunek}%
\index{pęk}%
\index{stożkowa}%
Zna i~potrafi stosować twierdzenia Steinera i Braikenridge'a-Maclaurina.
\index{twierdzenie!Steinera}%
\index{twierdzenie!Braikendridge'a-Maclaurina}%
Wie w jaki sposób określa się rzutowo ogniska i kierownice stożkowych.
\index{ognisko}%
\index{kierownica}%

\begin{proposition}
    Płaszczyzna afiniczna to taki zbiór punktów i prostych, które spełniają:
    \begin{itemize}
        \item aksjomaty incydencji oraz
        \item mocniejszą wersję aksjomatu Playfaira: dla każdej prostej $l$ i punktu $A$, dokładnie jedna prosta przechodzi przez punkt $A$ i jest równoległa do $l$.
    \end{itemize}
    Każda prosta na płaszczyźnie afinicznej ma tyle samo punktów.
    Jeśli pewna prosta ma $n$ punktów, to płaszczyzna ma dokładnie $n^2$ punktów.
    Istnieją płaszczyzny afiniczne o $4$, $9$, $16$ i $25$ punktach, ale nie istnieje taka, która miałaby $36$ punktów.
\end{proposition} % Hartshorne 71, 72

Geometrii afinicznej też nie rozumiemy, dlatego zaznaczymy tylko, co mogłoby się pojawić w kolejnym wydaniu.
Grupa przekształceń afinicznych od strony geometrycznej: powinowactwa osiowe, rozkład przekształcenia afinicznego na podobieństwo i powinowactwo osiowe, kierunki główne przekształcenia afinicznego.
Niezmienniczość stosunku pól przy przekształceniu afinicznym
Obraz okręgu przy przekształceniu afinicznym.

%