%

\subsection{Aksjomaty Hilberta}
W aksjomatyce Hilberta (płaskiej geometrii euklidesowej) pojęciami pierwotnymi są punkt, prosta, płaszczyzna, relacja incydencji (leżeć na, zawierać się w), relacja uporządkowania (leżenia między) oraz relacja przystawania.

%

\subsubsection{Aksjomaty incydencji}
\begin{axiom}[incydencji, I1]
    Dla każdej pary punktów $A$ oraz $B$ istnieje dokładnie jedna prosta $l$, która przechodzi przez te punkty.
\end{axiom}

\begin{axiom}[incydencji, I2]
    Na każdej prostej istnieją co najmniej dwa punkty.
\end{axiom}

\begin{axiom}[incydencji, I3]
    Istnieją co najmniej trzy punkty, które nie są współliniowe.
\end{axiom}

,,Być współliniowym'' to synonim leżenia na jednej prostej.

\begin{proposition}
    Dwie różne proste mogą mieć co najwyżej jeden punkt wspólny.
\end{proposition}

\begin{definition}
    Dwie różne proste, które nie mają punktów wspólnych, nazywamy równoległymi.
    Każda prosta jest też równoległa do siebie.
\end{definition}

\begin{axiom}[Playfaira, P]
    Dla każdej prostej $l$ oraz punktu $A$, istnieje co najwyżej jedna prosta przechodząca przez $A$, równoległa do $l$.
\end{axiom}
% TODO: https://en.wikipedia.org/wiki/Playfair%27s_axiom

John Playfair opublikował ten aksjomat w 1795 roku, chociaż już wtedy twierdził, że inni używali go przed nim (np. William Ludlam).
\index[persons]{Playfair, John}%
Aksjomat Playfaira jest równoważny z (piątym) postulatem Euklidesa i dlatego wiele osób próbowało wyprowadzić go z czterech wcześniejszych postulatów.
Za każdym razem okazywało się, że w ,,dowodzie'' użyte jest zdanie będące równoważnikiem piątego postulatu, na przykład:
\begin{itemize}
    \item suma kątów wewnętrznych każdego trójkąta jest kątem półpełnym,
    \item suma kątów wewnętrznych każdego trójkąta jest taka sama,
    \item istnieje trójkąt, którego suma kątów wewnętrznych jest kątem półpełnym,
    \item istnieją dwa trójkąty, które są do siebie podobne, ale nie przystające,
    \item na każdym trójkącie można opisać okrąg,
    \item jeśli trzy kąty wewnętrzne czworokąta są proste, to czwarty kąt także jest prosty,
    \item istnieje para prostych, które są w stałej odległości od siebie,
    \item dwie proste, które są równoległe do danej prostej, są też równoległe do siebie,
    \item twierdzenie Pitagorasa lub jego uogólnienie, twierdzenie cosinusów
    \item aksjomat Wallisa: nie ma górnego ograniczenia na pole trójkąta,
    \item kąty przy górnej podstawie czworokąta Saccheriego (czworokąta $ABCD$ o dwóch kątach prostych przy dolnej podstawie $AB$, w którym boki $AD$ i $BC$ mają równe długości) są proste,
    \item aksjomat Proklosa: jeśli prosta przecina jedną z dwóch równoległych prostych, to przecina także tę drugą.
\end{itemize}

% TODO: https://en.wikipedia.org/wiki/Parallel_postulate

\begin{example}
    Rozważmy geometrię, gdzie płaszczyzna składa się z pięciu punktów $A$, $B$, $C$, $D$, $E$ i leży na niej dziesięć różnych prostych, każda z nich przechodząca przez dokładnie dwa różne punkty.
    Wtedy proste $AB$ i $AC$ mają punkt wspólny $A$, chociaż obydwie są równoległe do prostej $DE$.
    Aksjomat Playfaira nie jest spełniony.
\end{example}

\begin{proposition}
    Aksjomaty incydencji I1, I2, I3 oraz aksjomat P (Playfaira) są od siebie niezależne.
\end{proposition}

Hartshorne \cite[s. 69-70]{hartshorne2000} konstruuje modele geometrii, w których spełnione są dowolne trzy, ale nie czwarty z nich.

\begin{proposition}
    Płaszczyzna rzutowa to taki zbiór punktów oraz prostych (podzbiorów zbioru punktów), że:
    \begin{itemize}
        \item przez dwa różne punkty przechodzi dokładnie jedna prosta,
        \item każde dwie proste mają punkt wspólny,
        \item każda prosta ma co najmniej trzy punkty i
        \item nie wszystkie punkty są współliniowe.
    \end{itemize}
    (Każdy z wymienionych aksjomatów jest niezależny od pozostałych, ze wszystkich razem wynikają aksjomaty incydencji).
    Każda płaszczyzna rzutowa ma co najmniej 7 punktów, dokładnie jedna płaszczyzna rzutowa ma dokładnie 7 punktów.
    Jeśli istnieje prosta, która ma $n+1$ punktów, to płaszczyzna ma $n^2 + n + 1$ punktów.
\end{proposition} % Hartshorne 71

\begin{example}[płaszczyzna Fana]
    Zbiór złożony z siedmiu elementów (,,punktów''), w którym wyróżniono rodzinę siedmiu podzbiorów (,,prostych'') spełniający następujące warunki:
    \begin{itemize}
        \item każde dwie różne proste mają dokładnie jeden punkt wspólny,
        \item każde dwa różne punkty należą do dokładnie jednej prostej
    \end{itemize}
    nazywamy płaszczyzną Fana.
    Jest przykładem płaszczyzny rzutowej, która nie spełnia aksjomatu Fana (że trzy punkty przekątne czworoboku zupełnego nie są współliniowe).
    \begin{figure}[H]
        \centering
        \begin{tikzpicture}[
        mydot/.style={
        draw,
        circle,
        fill=black,
        inner sep=1.5pt}
        ]
        \draw
        (0,0) coordinate (A) --
        (4cm,0) coordinate (B) --
        ($ (A)!.5!(B) ! {sin(60)*2} ! 90:(B) $) coordinate (C) -- cycle;
        \coordinate (O) at
        (barycentric cs:A=1,B=1,C=1);
        \draw (O) circle [radius=4cm*1.717/6];
        \draw (C) -- ($ (A)!.5!(B) $) coordinate (LC); 
        \draw (A) -- ($ (B)!.5!(C) $) coordinate (LA); 
        \draw (B) -- ($ (C)!.5!(A) $) coordinate (LB); 
        \foreach \Nodo in {A,B,C,O,LC,LA,LB}
        \node[mydot] at (\Nodo) {};    
    \end{tikzpicture}%
    \caption{Płaszczyzna Fana}
\end{figure}
\end{example}

W momencie pisania tego tekstu nie jesteśmy jeszcze zainteresowani geometrią rzutową, więc przytoczymy tylko wymagania wobec studenta, który ukończył kurs ,,Geometria III'' na uniwersytecie w Warszawie.
Audin \cite[s. 143-182]{audin_2003}.


Zna pojęcie płaszczyzny rzutowej rzeczywistej (równoważne sformułowania), dwustosunku, definicję przekształceń rzutowych łańcuchów, stożkowych, pęków stycznych do stożkowych.
\index{dwustosunek}%
\index{pęk}%
\index{stożkowa}%
Zna i~potrafi stosować twierdzenia Steinera i Braikenridge'a-Maclaurina.
\index{twierdzenie!Steinera}%
\index{twierdzenie!Braikendridge'a-Maclaurina}%
Wie w jaki sposób określa się rzutowo ogniska i kierownice stożkowych.
\index{ognisko}%
\index{kierownica}%

\begin{proposition}
    Płaszczyzna afiniczna to taki zbiór punktów i prostych, które spełniają:
    \begin{itemize}
        \item aksjomaty incydencji oraz
        \item mocniejszą wersję aksjomatu Playfaira: dla każdej prostej $l$ i punktu $A$, dokładnie jedna prosta przechodzi przez punkt $A$ i jest równoległa do $l$.
    \end{itemize}
    Każda prosta na płaszczyźnie afinicznej ma tyle samo punktów.
    Jeśli pewna prosta ma $n$ punktów, to płaszczyzna ma dokładnie $n^2$ punktów.
    Istnieją płaszczyzny afiniczne o $4$, $9$, $16$ i $25$ punktach, ale nie istnieje taka, która miałaby $36$ punktów.
\end{proposition} % Hartshorne 71, 72

Podręcznik Audina \cite[s. 7]{audin_2003} zaczyna się od definicji przestrzeni afinicznej.
My nie będziemy poświęcać im za dużo miejsca.
\todofoot{Grupa przekształceń afinicznych od strony geometrycznej: powinowactwa osiowe, rozkład przekształcenia afinicznego na podobieństwo i powinowactwo osiowe, kierunki główne przekształcenia afinicznego.
Niezmienniczość stosunku pól przy przekształceniu afinicznym
Obraz okręgu przy przekształceniu afinicznym.}

%
%

\subsection{Aksjomaty leżenia pomiędzy}
\begin{axiom}[leżenia pomiędzy, B1]
    Jeśli punkt $B$ leży między punktami $A$ i $C$, to punkty $A$, $B$, $C$ są różnymi punktami tej samej prostej oraz punkt $B$ leży także między punktami $C$ i $A$.
\end{axiom}

\begin{axiom}[leżenia pomiędzy, B2]
    Dla każdej pary punktów $A$ i $B$ istnieje punkt $C$ taki, że punkt $B$ leży między punktami $A$ i $C$.
\end{axiom}

\begin{axiom}[leżenia pomiędzy, B3]
    Spośród trzech punktów leżących na prostej, dokładnie jeden leży pomiędzy pozostałymi dwoma.
\end{axiom}

\begin{axiom}[leżenia pomiędzy, B4]
    Niech $A$, $B$ i $C$ będą trzema niewspółliniowymi punktami, zaś $l$ prostą, która nie przechodzi przez żaden z nich.
    Jeśli prosta $l$ przechodzi przez punkt między punktami $A$ i $B$, to przechodzi też przez punkt między punktami $A$ i $C$ albo $B$ i $C$, ale nie przez obydwa.
\end{axiom}

Powyższy aksjomat nazywany jest też aksjomatem Pascha, ponieważ Moritz Pasch \cite{pasch_1882} przyłapał dopiero w 1882 roku geometrów całego świata na tym, że korzystali z takiej przesłanki.
\index{aksjomat!Pascha}%
\index[persons]{Pasch, Moritz}%
Aksjomatu Pascha (B4) nie wolno mylić z aksjomatem Veblena-Younga geometrii rzutowej\footnote{Jeśli prosta przecina dwa boki trójkąta, to przecina także jego trzeci bok. \index{aksjomat!Veblena-Younga}}!
% PRZEJRZANO: https://en.wikipedia.org/wiki/Pasch%27s_axiom

\begin{proposition}
    Z aksojmatów I1 do I3, B1 do B4 wynika, że każda prosta ma nieskończenie wiele punktów.
\end{proposition}

\begin{definition}[odcinek]
    Niech $A$, $B$ będą punktami.
    Zbiór złożony z punktów $A$, $B$ oraz punktów, które leżą między nimi, nazywamy odcinkiem i oznaczamy:
\index{odcinek}%
    \begin{equation}
        \overline{AB}.
    \end{equation}
\end{definition}

Jeżeli nie prowadzi to do nieporozumień, kreskę pomijamy.

\begin{definition}[trójkąt]
    Niech $A$, $B$, $C$ będą punktami.
    Sumę odcinków $AB$, $BC$, $AC$ nazywamy trójkątem, wspomniane odcinki -- jego bokami, zaś punkty $A$, $B$ i $C$ -- wierzchołkami.
\index{trójkąt}%
\end{definition} % Hartshorne 74

Trójkąty zazwyczaj oznaczamy tak: $\triangle ABC$ i nie odróżniamy go od trójkąta z wnętrzem.

\begin{proposition}
    Niech $l$ będzie prostą.
    Wtedy zbiór punktów, które nie leżą na prostej $l$ można rozbić na dwa niepuste zbiory $S_1$, $S_2$ takie, że: dwa punkty, które nie leżą na prostej $l$, należą do tego samego zbioru ($S_1$ lub $S_2$) wtedy i~tylko wtedy, gdy odcinek $AB$ nie przecina prostej $l$.
\end{proposition} % Hartshorne 74

Zbiory $S_1$, $S_2$ nazywamy stronami prostej $l$.
\index{prosta!dwie strony prostej}%
Podobnie punkt wyznacza na prostej dwa zbiory, które leżą po różnych stronach tego punktu.

\begin{definition}[półprosta]
    Niech $A$, $B$ będą punktami.
    Zbiór złożony z punktów $A$, $B$ oraz punktów, które leżą po tej samej stronie punktu $A$ na prostej $AB$ co punkt $B$, nazywamy półprostą i oznaczamy:
\index{półprosta}%
    \begin{equation}
        \overrightarrow{AB}.
    \end{equation}
\end{definition} % Hartshorne 77

Ponownie, strzałkę psującą interlinię często pomijamy.

\begin{definition}[kąt]
    Sumę dwóch półprostych $AB$, $AC$, które nie leżą na jednej prostej, nazywamy kątem, zaś punkt $A$ wierzchołkiem tego kąta.
    Wnętrze kąta $\angle BACS$ składa się z tych punktów $D$ takich, że $D$ i $C$ leżą po tej samej stronie prostej $AB$ oraz $D$ i $B$ leżą po tej samej stronie prostej $AC$. 
    \index{kąt}%
\end{definition} % Hartshorne 77

W myśl tej definicji, nie ma kąta zerowego, wklęsłego ani półpełnego.
\index{kąt!zerowy}%
\index{kąt!półpełny}%
\index{kąt!wklęsły}%
Wnętrze trójkąta $ABC$ to część wspólna wnętrz kątów $\angle ABC$, $\angle BCA$, $\angle CAB$; jest wypukłe i niepuste.

%
\subsection{Aksjomaty przystawania odcinków}
Listę aksjomatów rozszerzymy o trzy, które opisują niezdefiniowane pojęcie przystawania odcinków: relacji między odcinkami, którą oznaczamy przez $\cong$.

\begin{axiom}[przystawania, C1]
    Niech $A$ i $B$ będą punktami, zaś $l$ prostą przechodzącą przez punkt $C$.
    Wtedy po każdej stronie punktu $C$ istnieje punkt $D$ taki, że odcinki $AB \cong CD$ są przystające.
\end{axiom}

Ten aksjomat odpowiada konstrukcji (I.3) Euklidesa i pozwala przenosić odcinki.

\begin{axiom}[przystawania, C2]
    Jeśli $AB \cong CD$ oraz $AB \cong EF$, to $CD \cong EF$.
    Każdy odcinek jest przystający do siebie.
\end{axiom}

To było pierwsze pojęcie pierwotne dla Euklidesa.

\begin{axiom}[przystawania, C3]
    Niech $A$, $B$, $C$, $D$, $E$, $F$ będą takimi punktami, że $B$ leży między $A$ i $C$, zaś $E$ leży między $D$ i $F$.
    Jeśli $AB \cong DE$ i $BC \cong EF$, to $AC \cong DF$.
\end{axiom}

Ten aksjomat pozwala nam dodawać odcinki i znowu zastępuje pojęcie pierwotne Euklidesa.
Dodawanie jest łączne i przemienne.

\begin{proposition}[odejmowanie odcinków]
    Niech $A$, $B$, $C$, $D$, $E$, $F$ będą takimi punktami, że $B$ leży między $A$ i $C$, zaś $E$ i $F$ leżą na półprostej zaczynającej się w $D$.
    Jeśli $AB \cong DE$ i $AC \cong DF$, to punkt $E$ leży między $D$ i $F$, co więcej $BC \cong EF$.
\end{proposition}

Odcinek $BC$ traktujemy jako różnicę między $AC$ i $AB$.
Dla Euklidesa to było pojęcie pierwotne (że całość jest większa od części).

\begin{definition}
    Niech $AB$ i $CD$ będą odcinkami.
    Jeśli istnieje punkt $E$ między $C$ i $D$ taki, że $AB \cong CE$, to powiemy, że odcinek $AB$ jest krótszy niż odcinek $CD$ (zaś odcinek $CD$ jest dłuższy niż odcinek $AB$).
\end{definition} % Hartshorne 85

Relacja bycia krótszym jest zgodna z przystawaniem i zadaje zupełny porządek na klasach równoważności.
Dodawanie odcinków zachowuje nierówności między nimi.

\begin{definition}[okrąg]
    Niech $O$, $A$ będą dwoma różnymi punktami.
    Zbiór punktów $B$ takich, że $OA \cong OB$ nazywamy okręgiem o środku $O$ oraz promieniu $OA$; okręgi często oznacza się literą $\Gamma$.
\end{definition} % Hartshorne 89

\begin{proposition}
    Każda prosta, która przechodzi przez środek, przecina okrąg w dwóch punktach.
    Okrąg składa się z nieskończenie wielu punktów.
\end{proposition}

(Nie jest jasne, ile środków może mieć okrąg, ale Hartshorne \cite[s. 89]{hartshorne2000} obiecuje pokazać póżniej, że tylko jeden.
Później ma miejsce na stronie 104).

\begin{definition}[styczna]
    Niech $\Gamma$ będzie okręgiem, zaś $l$ prostą, która przecina $\Gamma$ w dokładnie jednym punkcie $A$.
    Mówimy, że $l$ jest styczną do okręgu $\Gamma$ w punkcie $A$.
\end{definition}

Podobnie mówimy, że dwa okręgi są styczne, jeśli mają jeden punkt wspólny.
\subsection{Aksjomaty przystawania kątów}
Podobnie jak dla odcinków, wprowadzamy niezdefiniowaną relację przystawania kątów, oznaczaną znowu przez $\cong$, ponieważ nie prowadzi to do nieporozumień.

\begin{axiom}[przystawania, C4]
    Niech $\angle BAC$ będzie kątem, zaś $DF$ półprostą.
    Istnieje wtedy dokładnie jedna półprosta $DE$ po ustalonej stronie prostej $DF$ taka, że $\angle BAC \cong \angle EDF$.
\end{axiom}

Możemy traktować ten aksjomat jako odpowiednik cyrkla: pozwala przenosić kąty tak, jak Euklides (I.23).

\begin{axiom}[przystawania, C5]
    Niech $\alpha, \beta, \gamma$ będą kątami.
    Jeśli $\alpha \cong \beta$ oraz $\alpha \cong \gamma$, to $\beta \cong \gamma$.
    Każdy kąt przystaje do siebie.
\end{axiom}

Mamy wreszcie cechę przystawania bok-kąt-bok, z której można wyprowadzić dodawanie kątów:

\begin{axiom}[przystawania, C6]
    Niech $ABC$ i $DEF$ będą dwoma trójkątami takimi, że $AB \cong DE$, $AC \cong DF$ i $\angle BAC \cong \angle EDF$.
    Wtedy pozostałe boki i kąty są przystające, a razem z nimi całe trójkąty: $BC \cong EF$, $\angle ABC \cong \angle DEF$ i $\angle ACB \cong \angle DFE$.
\end{axiom}

Euklides udawał, że dowodzi aksjomatu C6 metodą ,,nakładania'' figur, ale Hilbert znalazł model będący świadkiem, że nie wynika on z poprzednich aksjomatów; patrz \cite[paragraf 11]{hilbert_1988}, jak wiemy od Greenberga \cite[s. 200]{greenberg_2010}.

Jeżeli chodzi o dodawanie kątów, musimy być ostrożni.
Suma dwóch kątów może okazać się prostą (lub ,,dwoma kątami prostymi'' jak pisze Euklides) albo mieć miarę większą niż $\pi$, wtedy składniki sumy nie znajdują się we wnętrzu otrzymanego kąta.
Kąty można także odejmować.

\begin{definition}[kąt przyległy]
    Niech $\angle BAC$ będzie kątem, zaś $D$ punktem na prostej $AC$ po przeciwnej stronie $A$ niż $C$.
    Wtedy kąty $\angle BAC$ i $\angle BAD$ nazywamy przyległymi. % supplementary.
\end{definition}

Warto teraz spojrzeć na (I.13) Euklidesa.

\begin{proposition}
    Kąty wierzchołkowe (para kątów przy dwóch przecinających się prostych, które nie są przyległe) są przystające.
\end{proposition}

Kąty wierzchołkowe, przyległe, odpowiadające i naprzemianległe.
\todofoot{Pompe s. 5: dwie proste przecięte trzecią wyznaczają kąty tej samej miary <=> proste są równoległe (naprzemianległe). Pompe s. 5: suma kątów trójkąta to 180}

Istnieje odpowiednik relacji między odcinkami ,,bycia mniejszym'' dla kątów:

\begin{definition}
    Niech $\angle BAC$ i $\angle EDF$ będą kątami.
    Jeśli istnieje półprosta $DG$ wewnątrz kąta $EDF$ taka, że $\angle BAC \cong \angle GDF$, to powiemy, że kąt $\angle BAC$ jest mniejszy niż kąt $\angle EDF$ (a ten drugi jest większy niż pierwszy).
\end{definition}

\begin{definition}
    Kąt, który przystaje do kąta przyległego, nazywamy prostym.
    Dwie proste, które przecinają się i wyznaczają tam cztery kąty proste, będziemy nazywać prostopadłymi.
\end{definition}

Nie jest ważne, który kąt przyległy wybierzemy, ponieważ kąty wierzchołkowe są przystające.


\subsubsection{Płaszczyzna Hilberta}

Przedstawione do tej pory aksjomaty są niezbędnym minimum do uprawiania geometrii.

\begin{definition}
    Zbiór punktów i prostych, razem z relacjami leżenia pomiędzy, przystawania odcinków i kątów, które spełniają aksjomaty I1 do I3, B1 do B4, C1 do C6, nazywamy płaszczyzną Hilberta.
\end{definition}

W szczególności, nie zakładamy aksjomatu równoległości (P).
Wszystkie twierdzenia Euklidesa od I.1 do I.28 (bez I.1, I.22) można udowodnić na dowolnej płaszczyźnie Hilberta.
Do dwóch wyjątków potrzebujemy dodatkowego aksjomatu \ref{axiom_e}.

%