
\subsection{Aksjomaty Hilberta}
{\color{red}
Aksjomatyka Hilberta używa trzech pojęć pierwotnych punktu, prostej, płaszczyzny oraz trzech relacji pierwotnych:
\begin{itemize}
	\item leżenia pomiędzy (jedna relacja między trójkami punktów),
	\item zawierania się  w (trzy relacje: między punktami i prostymi; punktami i płaszczyznami; prostymi i płaszczyznami) oraz
	\item przystawania (dwie relacje: między odcinkami; między kątami).
\end{itemize}
Będziemy czasem używać synonimów, takich jak: ,,punkt $A$ leży na prostej $a$'', ,,prosta $a$ przechodzi przez punkt $A$'', ,,prosta $a$ łączy punkty $A$ i $B$''.
Wymienimy najpierw wszystkie aksjomaty, a potem przeanalizujemy ich treść.

\begin{itemize}
	\item \textbf{aksjomaty incydencji}:
\begin{enumerate}
	\item Przez każde dwa punkty przechodzi dokładnie jedna prosta.
	\item Na każdej prostej leżą co najmniej dwa różne punkty.
	\item Pewne trzy punkty nie są współliniowe.
	\item (Pozostałe aksjomaty incydencji dotyczą przestrzeni trójwymiarowej).
\end{enumerate}
\item \textbf{aksjomat Playfaire'a}:
\begin{enumerate}
	\item Dla każdego punktu $A$ i każdej prostej $l$, istnieje co najwyżej jedna prosta równoległa do $l$, zawierająca $A$.
\end{enumerate}
\item \textbf{aksjomaty uporządkowania}: \begin{enumerate}
	\item Jeżeli punkt $B$ leży pomiędzy punktami $A$ i $C$, to leży też pomiędzy punktami $C$ i $A$, a wszystkie trzy leżą na jednej prostej.
	\item Między każdą parą punktów leży trzeci punkt.
	\item Dla każdych trzech punktów na prostej, tylko jeden z nich leży pomiędzy pozostałymi dwoma.
	\item (Pascha) Niech $A, B, C$ będą trzema niewspółliniiowymi punktami, zaś $l$ prostą, która nie przechodzi przez żaden z nich. Jeśli prosta $l$ zawiera punkt $D$ leżący między $A$ i $B$, to musi też zawierać punkt leżący między $A$ i $C$ albo punkt leżący między $B$ i $C$, ale nie obydwa te punkty.
\end{enumerate}
\item \textbf{aksjomaty przystawania} (zapis $\overline{AB} \cong \overline{CD}$ oznacza, że odcinki są przystające): \begin{enumerate}
	\item Niech $\overline{AB}$ będzie odcinkiem, a $r$ półprostą o początku w punkcie $C$. Istnieje dokładnie jeden punkt $D$ leżący na $r$ taki, że $\overline{AB} \cong \overline{CD}$.
	\item Jeśli $\overline{AB} \cong \overline{CD}$ i $\overline{AB} \cong \overline{EF}$, to $\overline{CD} \cong \overline{EF}$. Każdy odcinek przystaje do siebie.
	\item (dodawanie) Dane są trzy punkty $A, B, C$ na prostej takie, że $B$ leży pomiędzy $A$ i $C$; oraz trzy punkty $D, E, F$ na (być może innej) prostej takie, że $E$ leży pomiędzy $D$ i $F$.
	Jeśli $\overline{AB} \cong \overline{DE}$ i $\overline{BC} \cong \overline{EF}$, to $\overline{AC} \cong \overline{DF}$.
	\item (aksjomaty dla kątów)
	\item (aksjomaty dla kątów)
	\item (cecha przystawania bok-kąt-bok)
\end{enumerate}
\end{itemize}

Punkty nazywamy współliniowymi, kiedy istnieje prosta, która przechodzi przez każdy z~nich.
Aksjomat Playfaira został nazwany na cześć szkockiego matematyka, który podał jego treść w podręczniku \emph{Elements of Geometry} z 1795 roku.
% % https://en.wikipedia.org/wiki/Playfair%27s_axiom
\index[persons]{Playfair, John}%
\index{aksjomat!Playfaira}%
Potrzebna jest jeszcze definicja prostych równoległych:

\begin{definition}[równoległość]
	Dwie proste, które pokrywają się albo nie mają żadnych punktów wspólnych, nazywamy równoległymi.
\end{definition}

Czwarty aksjomat uporządkowania znalazł Moritz Pasch \cite{pasch_1882} w 1882 roku.
\index[persons]{Pasch, Moritz}
\index{aksjomat!Pascha}
Aksjomaty uporządkowania pozwalają mówić o odcinkach:

\begin{definition}[odcinek]
\index{odcinek}%
	Niech $A, B$ będą dwoma różnymi punktami.
	Zbiór punktów $A$, $B$ oraz wszystkich punktów leżących pomiędzy $A$ i $B$ nazywamy odcinkiem i oznaczamy $\overline{AB}$.
\end{definition}

\begin{definition}[trójkąt]
\index{trójkąt}%
\index{trójkąt!bok trójkąta}%
\index{trójkąt!wierzchołek trójkąta}%
	Niech $A, B, C$ będą trzema niewspółliniowymi punktami.
	Sumę odcinków $\overline{AB}$, $\overline{BC}$ i $\overline{AC}$ nazywamy trójkątem i oznaczamy $\triangle ABC$.
	Punkty $A, B, C$ są jego wierzchołkami, odcinki $\overline{AB}$, $\overline{BC}$ i $\overline{AC}$ bokami.
\end{definition}

Znając trójkąty, możemy wysłowić aksjomat Pascha inaczej: jeśli prosta $l$ przechodzi przez bok $\overline{AB}$ trójkąta $\triangle ABC$ i nie przechodzi przez wierzchołki $A, B$, to musi przecinać dokładnie jeden z boków $\overline{AC}$, $\overline{BC}$.

\begin{proposition}[rozdzielanie płaszczyzny]
	Niech $l$ będzie prostą.
	Zbiór punktów, które nie leżą na $l$ można podzielić na dwa niepuste podzbiory $S_1, S_2$ takie, że dwa punkty $A, B$ należą do tego samego zbioru ($S_1$ lub $S_2$) wtedy i tylko wtedy, gdy odcinek $\overline{AB}$ nie przecina prostej $l$.
\end{proposition}

Będziemy mówić, że dwa punkty leżą po tej samej stronie (albo po różnych stronach) prostej.

\begin{proof}
	Hartshorne \cite[s. 74--76]{hartshorne_2010}.
\end{proof}

\begin{proposition}[rozdzielanie prostej]
	Niech $l$ będzie prostą przechodzącą przez punkt $A$.
	Zbiór pozostałych punktów prostej $l$ można podzielić na dwa niepuste podzbiory $S_1, S_2$ takie, że dwa punkty $B, C$ należą do tego samego zbioru ($S_1$ lub $S_2$) wtedy i tylko wtedy, gdy punkt $A$ nie leży na odcinku $\overline{BC}$.
\end{proposition}

Znowu, pozwala to mówić o dwóch stronach prostej.

\begin{proof}
	Hartshorne \cite[s. 76--77]{hartshorne_2010}.
\end{proof}

\begin{definition}[półprosta]
	Niech $A, B$ będą dwoma punktami.
	Zbiór, do którego należą punkt $A$ oraz wszystkie punkty prostej $AB$, które leżą po tej samej stronie, co punkt $B$, nazywamy półprostą $\overrightarrow{AB}$ o początku w $A$.
\end{definition}

\begin{definition}[kąt]
	Sumę dwóch półprostych $\overrightarrow{AB}$ i $\overrightarrow{AC}$, które nie leżą na jednej prostej, nazywamy kątem i oznaczamy $\angle BAC$.
	Wnętrzem takiego kąta nazywamy zbiór punktów $D$ takich, że $D$ i $C$ leżą po tej samej stronie prostej $AB$, zaś $D$ i $B$ po tej samej stronie prostej $AC$.
\end{definition}

(W myśl tej definicji, nie istnieje kąt zerowy ani półpełny!).
Część wspólną wnętrz kątów $\angle ABC$, $\angle BCA$ i $\angle CAB$ nazywamy wnętrzem trójkąta $\triangle ABC$.

\begin{proposition}[o kuszy]
	Niech $\angle BAC$ będzie kątem, we wnętrzu którego leży punkt $D$.
	Wtedy półprosta $\overrightarrow{AD}$ przecina odcinek $\overline{BC}$.
\end{proposition}

\begin{proof}
	Hartshorne \cite[s. 77--78]{hartshorne_2010}.
\end{proof}

Trzeci aksjomat przystawania pozwala nam dodawać odcinki: jeśli dane są odcinki $\overline{AB}$ i $\overline{CD}$, zaś $r$ jest półprostą $\overrightarrow{AB}$ z punktem $E$ na sobie takim, że $\overline{CD} \cong \overline{BE}$, to możemy skonstruować sumę $AE = AB + CD$.

(Odejmowanie, porządek...)

\begin{definition}[płaszczyzna Hilberta]
	Zbiór punktów $\Pi$ z wyróżnionymi pewnymi podzbiorami (zwanymi liniami) oraz pojęciami leżenia pomiędzy, przystawania odcinków i przystawania kątów (tak jak opisaliśmy to wyżej), który spełnia wszystkie aksjomaty poza, być może, aksjomatem Pascha, nazywamy płaszczyzną Hilberta.
\end{definition}


Twierdzenie o dwusiecznej % https://en.wikipedia.org/wiki/Angle_bisector_theorem
The angle bisector theorem appears as Proposition 3 of Book VI in Euclid's Elements. 

The exterior angle theorem is Proposition 1.16 in Euclid's Elements, which states that the measure of an exterior angle of a triangle is greater than either of the measures of the remote interior angles. This is a fundamental result in absolute geometry because its proof does not depend upon the parallel postulate. % https://en.wikipedia.org/wiki/Exterior_angle_theorem

Konstrukcja pierwiastka z iloczynu:
The theorem is usually attributed to Euclid (ca. 360–280 BC), who stated it as a corollary to proposition 8 in book VI of his Elements. In proposition 14 of book II Euclid gives a method for squaring a rectangle, which essentially matches the method given here. Euclid however provides a different slightly more complicated proof for the correctness of the construction rather than relying on the geometric mean theorem.
% https://en.wikipedia.org/wiki/Geometric_mean_theorem


Hinge theorem % https://en.wikipedia.org/wiki/Hinge_theorem

twierdzenia geometrii koła:
- % https://en.wikipedia.org/wiki/Thales%27s_theorem
- The inscribed angle theorem states that an angle $\theta$ inscribed in a circle is half of the central angle $2\theta$ that subtends the same arc on the circle. 

% https://en.wikipedia.org/wiki/Intercept_theorem

% https://en.wikipedia.org/wiki/Inscribed_angle#Theorem

% https://en.wikipedia.org/wiki/Intersecting_chords_theorem
% https://en.wikipedia.org/wiki/Intersecting_secants_theorem
% https://en.wikipedia.org/wiki/Tangent%E2%80%93secant_theorem
% https://en.wikipedia.org/wiki/Power_of_a_point#Theorems
twierdzenie o siecznych


\subsection{Elementy, księga I}
Hartshorne analizuje teraz, które stwierdzenia z Elementów Euklidesa są nadal prawdziwe na płaszczyźnie Hilberta.
Nie opiszę tego lepiej, dlatego przedstawiam jedynie podsumowanie.

(I.1), czyli konstrukcja trójkąta równobocznego, nie wynika z aksjomatów płaszczyzny Hilberta (ćwiczenie 39.31).
(I.2) i (I.3) zastąpiliśmy aksjomatem C1, zaś (I.4) aksjomatem C6.
(I.5), że kąty przy podstawie trójkąta równoramiennego są przystające, nie wymaga zmian.
Tezę V w I księdze nazywa się często \emph{pons asinorum}, czyli mostem osłów; jeśli ktoś nie jest w stanie samodzielnie przeprowadzić tego dowodu, to nie może przekroczyć mostu i dalej studiować geometrii.
\index{pons asinorum}%
\index{most osłów|see {pons asinorum}}%
(I.6) to twierdzenie odwrotne do (I.5) i wymaga kosmetycznych zmian, podobnie jak (I.7).

Ale (I.8), czyli cecha przystawania bok-bok-bok musi zostać udowodniona zupełnie inaczej; nowe uzasadnienie zaproponował Hilbert.
\index{cecha przystawania bok-bok-bok}
Zaczynając od (I.9) mamy do czynienia z konstrukcjami cyrklem i linijką, co stanowi pewien problem, bo nie wiemy jeszcze, czy proste zawsze przecinają okręgi.
Hartshorne posiłkuje się słabszym twierdzeniem, że każdy odcinek może być podstawą trójkąta równoramiennego.
To wystarcza do naprawy (I.10) i (I.11), ale nie (I.12), że dowolny punkt można zrzutować prostopadle na prostą, która przez niego nie przechodzi.
Potrzeba znowu całkiem nowego rozumowania.

Tezy (I.13) do (I.21) są w porządku.
Teza (I.22) jest nie do uratowania; nie wiemy, czy dwa okręgi muszą się zawsze przecinać tak, jak oczekuje tego Euklides i istotnie ćwiczenie 16.11 u Hartshorne'a mówi, że w pewnych płaszczyznach Hilberta trójkąty użyte w dowodzie nie istnieją.
Dalej, (I.23) było dowodzone przy użyciu (I.22), ale u nas to jest po prostu aksjomat C4.
Tezy (I.24) do (I.27) i (I.31) znowu są w porządku.

Zatem wszystko, co pisze Euklides, od I.1 do I.28 bez I.1, I.22 można uratować.

Hartshorne 104
Definicja okręgu, środka, promienia

Hartshorne 105
Definicja stycznej


}