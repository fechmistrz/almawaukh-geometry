\subsection{Pons asinorum}
In geometry, the theorem that the angles opposite the equal sides of an isosceles triangle are themselves equal is known as the pons asinorum Latin for "bridge of asses", or more descriptively as the isosceles triangle theorem. The theorem appears as Proposition 5 of Book 1 in Euclid's Elements[1]. Its converse is also true: if two angles of a triangle are equal, then the sides opposite them are also equal. 

Dowody:
- Euclid and Proclus
% https://en.wikipedia.org/wiki/Pons_asinorum#Euclid_and_Proclus

- Pappus
% https://en.wikipedia.org/wiki/Pons_asinorum#Pappus

- współcześni, Legendre, Garfield
% https://en.wikipedia.org/wiki/Pons_asinorum#Others