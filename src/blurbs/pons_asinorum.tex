%

\index{pons asinorum|(}

\subsection{Pons asinorum}
\index{most osłów patrz pons asinorum}
Most osłów (łacińskie ,,pons asinorum'') to tradycyjna nazwa dowodu twierdzenia, że kąty przy podstawie trójkąta równoramiennego są równe.
Podał go Euklides jako teza V w księdze I Elementów.
Mawiało się, że ci, którzy nie są w stanie samodzielnie przeprowadzić tego dedukcyjnego dowodu opartego na własnościach trójkątów przystających, nie może przekroczyć mostu i studiować dalej geometrii.

Pierwsze dowody tego faktu podali jeszcze Euklides, komentujący jego prace Proklos zwany Diadochem oraz Pappus z Aleksandrii.
Współcześnie podaje się krótkie uzasadnienie w oparciu o dwusieczną kąta, ale Euklides nie mógł tak uczynić, ponieważ definiuje ją dopiero cztery tezy później w swoich Elementach.

\index{pons asinorum|)}

%