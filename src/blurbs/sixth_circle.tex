%
\subsection{Twierdzenie o sześciu okręgach}

\begin{proposition}[twierdzenie o~sześciu okręgach]
\index{twierdzenie!o sześciu okręgach}%
    Dany są trójkąt $\triangle ABC$ oraz okręgi $K_1$, $K_2$, \ldots, $K_7$ zawarte w~tym trójkącie, wpisane kolejno w~kąty $\angle A$, $\angle B$, $\angle C$, $\angle A$, $\angle B$, $\angle C$, $\angle A$ takie, że okręgi $K_i$ oraz $K_{i+1}$ dla $i = 1, 2, \ldots, 6$ są styczne.
    Wtedy $K_1 = K_7$.
\end{proposition}

(Neugebauer \cite[s. 101]{neugebauer_2018} nazywa to twierdzeniem o~siódmym okręgu).
Tabacznikow, Iwanow \cite{ivanov_tabachnikov_2016} pokazali, że jeśli osłabimy założenia: okręgi nie muszą zawierać się w~trójkącie i~wystarczy, że będą styczne do prostych zawierających boki trójkąta, to nadal ciąg okręgów jest od pewnego miejsca okresowy z okresem równym sześć, ale osiągnięcie tego stanu może wymagać dowolnie wielu kroków.
\index[persons]{Tabacznikow, Siergiej (Табачников, Сергей Львович)}%
\index[persons]{Iwanow, Denis (Иванов, Денис)}%

\begin{proof}
    Evelyn, Money-Coutts, Tyrrell \cite[s. 49–58]{evelyn_money_coutts_tyrrell_1974}.
\index[persons]{Evelyn, Cecil John Alvin}%
\index[persons]{Money-Coutts, Godfrey Burdett}%
\index[persons]{Tyrrell, John Alfred}%
\end{proof}

%