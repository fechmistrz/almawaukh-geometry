
% intercept theorem, also known as Thales's theorem, basic proportionality theorem or side splitter theorem
\begin{theorem}[Talesa]
    Jeśli ramiona kąta płaskiego przetnie się 2 równoległymi prostymi:
    \begin{center}
        \begin{tikzpicture}
            \tkzDefPoint(0, 0.5){O}
            \tkzDefPoint(1.5, 0){A}
            \tkzDefPoint(2, 1){Ap}
            \tkzDefPointBy[homothety=center O ratio 1.618](A) \tkzGetPoint{B}
            \tkzDefLine[parallel=through B](A,Ap) \tkzGetPoint{Bp}
            \tkzInterLL(O,Ap)(B,Bp) \tkzGetPoint{Bpp}
            \tkzDrawPoints[fill=gray,opacity=.9](O,A,B,Ap,Bpp)
            \tkzLabelPoint[above](O){$O$}
            \tkzLabelPoint[below](A){$A$}
            \tkzLabelPoint[below](B){$B$}
            \tkzLabelPoint[above left](Bpp){$B'$}
            \tkzLabelPoint[above left](Ap){$A'$}
            \tkzDrawLine[thick](O,B)
            \tkzDrawLine[thick](O,Bpp)
            \tkzDrawLine[color=red, thick](A,Ap)
            \tkzDrawLine[color=red, thick](B,Bpp)
        \end{tikzpicture}
        \end{center}
    to długości odcinków wyznaczonych przez te proste na jednym z ramion kąta są proporcjonalne do długości odpowiednich odcinków na drugim ramieniu kąta, a zatem
    \begin{equation}
        \frac{|OA|}{|OB|} = \frac{|OA'|}{|OB'|} = \frac{|AA'|}{|BB'|}.
    \end{equation}
\end{theorem}

Tradycja przypisuje jego sformułowanie Talesowi z Miletu, chociaż znane było starożytnym Babilończykom i Egipcjanom.
\index[persons]{Tales z Miletu}%
Pierwszy znany dowód pojawia się w Elementach Euklidesa.

\begin{proof}
    Do napisania.
\end{proof}
