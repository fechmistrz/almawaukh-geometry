
\section{Do zrobienia}
Potęga punktu względem okręgu?
Twierdzenie o odcinku środkowym: odcinek łączący środki dwóch boków trójkąta jest równoległy do podstawy i ma połowę jej długości.
Symetria osiowa.
Symetralna: przecinają się w jednym punkcie.
Okrąg.
Styczne, sieczne.
Twierdzenia geometrii koła o miarach kątów.
Twierdzenie Apoloniusza.
Czworokąt cykliczny.
Twierdzenie o prostej Wallace'a-Simsona.
Ortocentrum i trójkąt ortyczny.
Twierdzenie Miquela.
Twierdzenie Pitagorasa.
Twierdzenie Varignona.
Podobieństwo, skala.
Twierdzenie Ptolemeusza.
Twierdzenie Carnota.
Sieczne i styczne.
Potęga punktu względem okręgu.
Twierdzenie o prostej Auberta.
Twierdzenie o dwusiecznej.
Twierdzenie o okręgu Apoloniusza.
Dwustosunek.
Pęki okręgów.
twierdzenia:
- Ptolemeusza
- trójkąty
twierdzenie Pitagorasa, wzór Herona (uogólnienie do czworokątów itd.), twierdzenie Carnota
okrąg opisany, wpisany, ortrocentrym, środek ciężkości
prosta Eulera?
okrąg Feuerbacha?
punkt Torricellego = punkt Fermata
- czworokąty:
opisany/wpisany na okręgu, 
twierdzenia Newtona/Brianchona
nierówności:
- Mikołaja z Kuzy: sinx / x < 2 + cos x / 3 (Guzicki, s. 390)
- Eulera (R >= 2r), izoperymetryczna (S <= pp/3sqrt3), Mitrinovica (r <= ... <= R/2), Leibniza (aa + bb + cc <= 9RR), Weitzenbocka (aa + bb + cc >= 4sqrt 3 S)
Konstrukcje z cyrklem i linijką:
- wielokątów (3, 4, 6, 5, 17, 257, ...)
- okręgi Apoloniusza
Inwersja, Feuerbach.
Okręgi Apoloniusza: \cite[s. 444-461]{guzicki_2021}.
Dwustosunek.
Izometrie, punkty stałe.
Translacje, symetrie osiowe, symetrie środkowe, obroty.
Twierdzenie Chasles'a: każda izometria płaszczyzny jest złożeniem co najwyżej trzech symetrii osiowych.
Symetria osiowa z poślizgiem.
Słowo Banacha.
Klasyfikacja podobieństw.
Okrąg siedmiu punktów. % https://mathworld.wolfram.com/BrocardCircle.html ?
Przekształcenia afiniczne i rzutowe.
% https://www.cut-the-knot.org/Curriculum/Geometry/HeronsProblem.shtml
% This one is a basic optimization problem. It's quite famous, being discussed in Heron's Catoptrica (On Mirrors from the Greek word Katoptron Catoptron = Mirror) that, in all likelihood, saw the light of day some 2000 years ago.
Pitagorasa % https://en.wikipedia.org/wiki/Pythagorean_theorem
% https://en.wikipedia.org/wiki/Spiral_of_Theodorus
Twierdzenie Ponceleta.
Prosta/twierdzenie Eulera.
Twierdzenie Morleya
Okrąg dziewięciu punktóws
Trygonometria - sinusów, cosinusów, Stewarta.
Wzór Herona.
Wzór Brahmagupty
Twierdzenie Urquharta
Punkt i kąt Crelle'a-Brocarda
Aksjomaty. Kąty naprzemianległe i odpowiadające.
Przystawanie trójkątów.
Łamane i wielokąty.
Równoległobok.
Równoważność wektorów.
Symetria osiowa.
Symetralna.
Styczna do okręgu.
gnomon % https://en.wikipedia.org/wiki/Theorem_of_the_gnomon
Kąty środkowe i wpisane.
Cykliczność. Prosta Wallace'a.
Ortocentrum i trójkąt ortyczny.
Twierdzenie Miquela.
Dwusieczna. Okrąg wpisany i dopisane.
Twierdzenie Pitagorasa.

Podobieństwo.
Twierdzenie Ptolemeusza.
Twierdzenie Carnota.
Potęga punktu względem okręgu.
Pęki okręgów.
Twierdzenie Eulera.
Twierdzenie Morleya.
Trygonometria. Wzór Herona.
Twierdzenie Urquharta.
Kąt Crelle'a-Brocarda.
Twierdzenie o siódmym okręgu.
Współliniowość.
Współpękowość.
Ceva i Menelaos.
Twierdzenie Ponceleta.
Jednokładność.
Inwersja.
Dwustosunek.