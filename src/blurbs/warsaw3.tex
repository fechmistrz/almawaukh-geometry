\subsection{Geometria III}
Geometria rzutowa: ujęcie od strony geometrycznej. Płaszczyzna rzutowa (rzeczywista), przekształcenia rzutowe prostych, pęków, stożkowych, pęków stycznych do stożkowych.
Twierdzenia Desarguesa, Pappusa, Pascala, Brianchona.
Dualność: biegun i biegunowa względem okręgu i stożkowych. Sprzężenie biegunowe. Inwolucje rzutowe, twierdzenia inwolucyjne. Pęki okręgów i stożkowych jako generatory inwolucji. Twierdzenie Ponceleta. Stożkowe w ujęciu rzutowym, twierdzenia Steinera i Braikenridge'a-Maclaurina. Rzutowe określenie ogniska i kierownicy stożkowych. Punkty urojone przecięcia prostej ze stożkową w ujęciu czysto geometrycznym.
