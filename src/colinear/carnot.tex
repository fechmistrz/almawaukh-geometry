%

Uogólnieniem twierdzenia o współpękowości symetralnych boków trójkąta jest:

\begin{proposition}[twierdzenie Carnota]
\label{guzicki_6_13}%
	Dany jest trójkąt $ABC$ i punkty $D, E, F$ leżące odpowiednio na prostych $BC, CA, AB$.
	Niech prosta $k$ (odpowiednio: $l$, $m$) przechodzi przez punkt $D$ ($E$, $F$) i będzie prostopadła do prostej $BC$ ($CA$, $AB$).
	Wtedy proste $k$, $l$, $m$ mają punkt wspólny wtedy i tylko wtedy, gdy
	\begin{equation}
		|AF|^2 + |BD|^2 + |CE|^2 = |AE|^2 + |BF|^2 + |CD|^2.
	\end{equation}
\end{proposition}
% TODO: https://en.wikipedia.org/wiki/Carnot%27s_theorem_(perpendiculars)

Guzicki \cite[s. 176]{guzicki_2021} wyprowadza je z twierdzenia Pitagorasa, co pozwala mu dojść do trzech wniosków dotyczących istnienia punktów szczególnych trójkąta: \ref{guzicki_6_17}, \ref{guzicki_6_18}, \ref{guzicki_6_20}.

\begin{corollary}
\label{guzicki_6_17}%
    Symetralne trzech boków trójkąta mają punkt wspólny (środek okręgu opisanego na tym trójkącie).
\end{corollary}

Audin \cite[s. 61]{audin_2003} podaje ten fakt w formie ćwiczenia.


\begin{corollary}
\label{guzicki_6_18}%
    Proste zawierające wysokości trójkąta mają punkt wspólny (ortocentrum).
\index{ortocentrum}%
\end{corollary}

Tego samego dowodzi Pompe \cite[s. 38]{pompe_2022}, zmyślnie używając równoległoboków.
Audin \cite[s. 61]{audin_2003} podaje ten fakt w formie ćwiczenia.

\begin{corollary} % Guzicki, s. 132
\label{guzicki_6_20}%
    Dwusieczne kątów trójkąta mają punkt wspólny (środek okręgu wpisanego w ten trójkąt).
\end{corollary}

% Nie wiem, czy tu:
Środkowe przecinają się w jednym punkcie. % Coxeter, Introduction to Geometry, s. 10 <- przeczytaj to, nie tylko cytuj! + ćwiczenia: 3/4 <= 1

Punkt ten nazywa się po angielsku centroid, dla Archimedesa był środkiem ciężkości trójkąta o równomiernie rozłożonej masie.

%