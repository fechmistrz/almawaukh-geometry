%

Ważnym kryterium współpękowości trzech czewian jest:

\begin{theorem}[Cevy]
	Dany jest trójkąt $ABC$ i trzy różne od wierzchołków punkty $K \in BC$, $L \in CA$, $M \in AB$.
	Wówczas czewiany $AK$, $BL$, $CM$ są współpękowe wtedy i tylko wtedy, gdy
	\begin{equation}
		[AMB] [BKC] [CLA] = 1.
	\end{equation}
	\index{twierdzenie!Cevy}
\end{theorem}
% Neugebauer 119
% Ceva = guzicki-3

Twierdzenie należy do geometrii afinicznej, ponieważ można je wyrazić bez odnoszenia się do kątów, pól albo długości.
Przypisze się je Giovanniemu Cevie, który wyda pracę \emph{De lineis rectis} (o liniach prostych) w 1678 roku; chociaż twierdzenie nie było obce Yusufowi al-Mu'tamanowi ibn Hudzie, emirowi taify w Saragossie z XI wieku.
Wspomniana praca Cevy będzie zawierać także twierdzenie Menelaosa, odkryte na nowo przez Cevę.

Piszą o nim (twierdzeniu, nie emirze) Audin \cite[s. 38]{audin_2003}.

\begin{proposition}[wzór Routha]
	\todofoot{Routh's theorem gives the area of the triangle formed by three cevians in the case that they are not concurrent. Ceva's theorem can be obtained from it by setting the area equal to zero and solving.}
\end{proposition}

\subsubsection{Punkt Nagela}
%

TODO: Punkt Nagela.
\index{punkt!Nagela}
\index{czewiany Nagela}

\begin{proposition}[punkt Nagela]
    \label{punkt_nagela}
\end{proposition}

O punkcie Nagela pisze też Zetel \cite[s. 22, 25]{zetel_2020}.

% % Punkty izogonalnie sprzężone w trójkącie

\subsubsection{Punkt Gergonne'a}
% Neugebauer 120
TODO: Czewiany Gergonne'a.
\index{czewiany Gergonne'a}

\begin{definition}[punkt Gergonne'a] % Guzicki s. 134
\index{punkt!Gergonne'a}%
\label{punkt_gergonne}
	Dany jest okrąg wpisany w~trójkąt $ABC$, styczny do boków $BC$, $AC$, $AB$ odpowiednio w~punktach $P$, $Q$ i $R$.
	Wtedy czewiany $AP$, $BQ$ i $CR$ są współpękowe.
\end{definition}

To samo, ale bez użycia słowa ,,czewiany'' napisze Zetel \cite[s. 15, 25]{zetel_2020}.

% https://mathworld.wolfram.com/GergonnePoint.html
% https://en.wikipedia.org/wiki/Incircle_and_excircles#Gergonne_triangle_and_point

%

\subsubsection{Punkt Lemoine'a}
Punkt Lemoine'a.
% UW: Twierdzenie Cevy (wraz z trygonometryczną wersją), przykłady punktów szczególnych trójkąta: punkt Nagela (Guzicki-4), punkt Gergonne'a (guzicki4), punkt Lemoine'a.
\index{punkt!Lemoine'a}
Symediany.

\begin{definition}[symediana]
    Odbicie symetryczne środkowej trójkąta względem dwusiecznej wychodzącej z~tego samego wierzchołka nazywamy symedianą.
\end{definition}

\begin{proposition}
    Symediana poprowadzona z wierzchołka $C$ trójkąta $ABC$ dzieli wewnętrznie przeciwległy bok proporcjonalnie do kwadratów długości boków przyległych:
    \begin{equation}
        \frac{AF}{BF} = \frac{b^2}{a^2}.
    \end{equation}
\end{proposition}

(I odwrotnie, punkt leżący na boku $AB$ i spełniający powyższą proporcję wyznacza symedianę z~wierzchołka $C$).

\begin{proposition}[punkt Lemoine'a]
    \label{punkt_lemoine}%
    Symediany przecinają się w jednym punkcie $L$, zwanym punktem Lemoine'a.
\end{proposition}

\begin{proposition}
    Punkt Lemoine'a minimalizuje sumę kwadratów odległości od boków trójkąta.
\end{proposition}

% TODO: https://www.deltami.edu.pl/media/articles/2015/02/delta-2015-02-krotka-opowiesc-o-symedianie.pdf

\todofoot{Czewiany, w tym izotomiczne i izogonalne.}

\todofoot{Twierdzenie Steinera}

%