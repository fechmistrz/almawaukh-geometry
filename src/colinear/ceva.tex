Ważnym kryterium współpękowości trzech czewian jest:

\begin{proposition}[twierdzenie Cevy (1678)]
	Dany jest trójkąt $ABC$ i trzy różne od wierzchołków punkty $K \in BC$, $L \in CA$, $M \in AB$.
	Wówczas czewiany $AK$, $BL$, $CM$ są współpękowe wtedy i tylko wtedy, gdy
	\begin{equation}
		[AMB] [BKC] [CLA] = 1.
	\end{equation}
	\index{twierdzenie!Cevy}
\end{proposition}
% Neugebauer 119
% Ceva = guzicki-3

Piszą o nim Audin \cite[s. 38]{audin_2003}.