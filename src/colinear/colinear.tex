
%

\section{Twierdzenia Menelaosa, Cevy, van Aubela}
\subsection{Twierdzenie Menelaosa}
Znamy trzy twierdzenia o współliniowości: o prostej Eulera, o prostej Wallace'a-Simsona, o prostej Auberta.

\begin{proposition}[twierdzenie Menelaosa]
	...
	Wówczas punkty $K, L, M$ są współliniowe wtedy i tylko wtedy, gdy zachodzi
	\begin{equation}
		[AMB] [BKC] [CLA] = -1.
	\end{equation}
	\index{twierdzenie!Menelaosa}
\end{proposition}

\index[persons]{Menelaos z Aleksandrii}%

Piszą o nim Audin \cite[s. 38]{audin_2003}, Zetel \cite[s. 43]{zetel_2020}.
Uogólnieniem twierdzenia Menelaosa będzie twierdzenie Carnota (udowodnione w 1803 roku w \emph{,,Géométrie de position}''), że iloczyn długości odcinków wyznaczonych przez prostą na bokach wielokąta i niemających wspólnych końców jest równy iloczynowi długości pozostałych odcinków; 
% https://encyclopediaofmath.org/wiki/Carnot_theorem
% Zetel 45.

% TODO: https://en.wikipedia.org/wiki/Menelaus%27s_theorem
% (MENELAOS = guzicki-3)
% It is uncertain who actually discovered the theorem; however, the oldest extant exposition appears in Spherics by Menelaus. In this book, the plane version of the theorem is used as a lemma to prove a spherical version of the theorem.
% twierdzenie o środkach jednokładności trzech okręgów, patrz TODO w kodzie źródłowym % (chyba https://atcm.mathandtech.org/EP2016/contributed/4052016_21160.pdf), na UW po: 	- Twierdzenia o składaniu jednokładności i przesunięć, 
\subsection{Twierdzenie Cevy}
%

Ważnym kryterium współpękowości trzech czewian jest:

\begin{theorem}[Cevy]
	Dany jest trójkąt $ABC$ i trzy różne od wierzchołków punkty $K \in BC$, $L \in CA$, $M \in AB$.
	Wówczas czewiany $AK$, $BL$, $CM$ są współpękowe wtedy i tylko wtedy, gdy
	\begin{equation}
		[AMB] [BKC] [CLA] = 1.
	\end{equation}
	\index{twierdzenie!Cevy}
\end{theorem}
% Neugebauer 119
% Ceva = guzicki-3

Twierdzenie należy do geometrii afinicznej, ponieważ można je wyrazić bez odnoszenia się do kątów, pól albo długości.
Przypisze się je Giovanniemu Cevie, który wyda pracę \emph{De lineis rectis} (o liniach prostych) w 1678 roku; chociaż twierdzenie nie było obce Yusufowi al-Mu'tamanowi ibn Hudzie, emirowi taify w Saragossie z XI wieku.
Wspomniana praca Cevy będzie zawierać także twierdzenie Menelaosa, odkryte na nowo przez Cevę.

Piszą o nim (twierdzeniu, nie emirze) Audin \cite[s. 38]{audin_2003}.

\begin{proposition}[wzór Routha]
	\todofoot{Routh's theorem gives the area of the triangle formed by three cevians in the case that they are not concurrent. Ceva's theorem can be obtained from it by setting the area equal to zero and solving.}
\end{proposition}

\subsubsection{Punkt Nagela}
%

TODO: Punkt Nagela.
\index{punkt!Nagela}
\index{czewiany Nagela}

\begin{proposition}[punkt Nagela]
    \label{punkt_nagela}
\end{proposition}

O punkcie Nagela pisze też Zetel \cite[s. 22, 25]{zetel_2020}.

% % Punkty izogonalnie sprzężone w trójkącie

\subsubsection{Punkt Gergonne'a}
% Neugebauer 120
TODO: Czewiany Gergonne'a.
\index{czewiany Gergonne'a}

\begin{definition}[punkt Gergonne'a] % Guzicki s. 134
\index{punkt!Gergonne'a}%
\label{punkt_gergonne}
	Dany jest okrąg wpisany w~trójkąt $ABC$, styczny do boków $BC$, $AC$, $AB$ odpowiednio w~punktach $P$, $Q$ i $R$.
	Wtedy czewiany $AP$, $BQ$ i $CR$ są współpękowe.
\end{definition}

To samo, ale bez użycia słowa ,,czewiany'' napisze Zetel \cite[s. 15, 25]{zetel_2020}.

% https://mathworld.wolfram.com/GergonnePoint.html
% https://en.wikipedia.org/wiki/Incircle_and_excircles#Gergonne_triangle_and_point

%

\subsubsection{Punkt Lemoine'a}
Punkt Lemoine'a.
% UW: Twierdzenie Cevy (wraz z trygonometryczną wersją), przykłady punktów szczególnych trójkąta: punkt Nagela (Guzicki-4), punkt Gergonne'a (guzicki4), punkt Lemoine'a.
\index{punkt!Lemoine'a}
Symediany.

\begin{definition}[symediana]
    Odbicie symetryczne środkowej trójkąta względem dwusiecznej wychodzącej z~tego samego wierzchołka nazywamy symedianą.
\end{definition}

\begin{proposition}
    Symediana poprowadzona z wierzchołka $C$ trójkąta $ABC$ dzieli wewnętrznie przeciwległy bok proporcjonalnie do kwadratów długości boków przyległych:
    \begin{equation}
        \frac{AF}{BF} = \frac{b^2}{a^2}.
    \end{equation}
\end{proposition}

(I odwrotnie, punkt leżący na boku $AB$ i spełniający powyższą proporcję wyznacza symedianę z~wierzchołka $C$).

\begin{proposition}[punkt Lemoine'a]
    \label{punkt_lemoine}%
    Symediany przecinają się w jednym punkcie $L$, zwanym punktem Lemoine'a.
\end{proposition}

\begin{proposition}
    Punkt Lemoine'a minimalizuje sumę kwadratów odległości od boków trójkąta.
\end{proposition}

% TODO: https://www.deltami.edu.pl/media/articles/2015/02/delta-2015-02-krotka-opowiesc-o-symedianie.pdf

\todofoot{Czewiany, w tym izotomiczne i izogonalne.}

\todofoot{Twierdzenie Steinera}

% 

\subsection{Twierdzenie van Aubela}
% van aubel
O twierdzeniu van Aubela pisze Zetel \cite[s. 24]{zetel_2020}.

% \subsection{Twierdzenie Carnota???}
% \begin{proposition}[twierdzenie Carnota???]
	% Neugebauer, strona 108.
% \end{proposition}

\section{Twierdzenia rzutowe: Desargues'a, Pappusa, Pascala, Brianchona}
% kolejnosć jak na UW: https://informatorects.uw.edu.pl/pl/courses/view?prz_kod=1000-1M14GM3
\begin{proposition}[twierdzenie Desargues'a]
	Neugebauer, strona 109.
	% TODO: https://en.wikipedia.org/wiki/Desargues%27s_theorem
	\index{twierdzenie!Desargues'a}
\end{proposition}
% 1. Zna pojęcie inwolucji rzutowych.   Zna i potrafi stosować twierdzenia inwolucyjne Desarguesa.  

Piszą o nim Audin \cite[s. 26, 151]{audin_2003}.
Eves \cite[s. 362]{eves1_1972} poda przykład Foresta Raya Moultona z 1902 roku płaszczyzny rzutowej, na której nie jest spełnione (!). % Desargues two triangle theorem for the plane: copolar triangles are coaxial
\index[persons]{Moulton, Forest Ray}% https://en.wikipedia.org/wiki/Forest_Ray_Moulton
% Eves 365: Desaurgian/Pappian/classic

\begin{proposition}[twierdzenie Pappusa]
\index{twierdzenie!Pappusa}%
	Neugebauer, strona 114.
	% https://en.wikipedia.org/wiki/Pappus%27s_hexagon_theorem
\end{proposition}
Piszą o nim Audin \cite[s. 25, 151, 171]{audin_2003} (w wersji afinicznej, potem rzutowej).
Hartshorne: 62, inne twierdzenie Pappusa?

% \item Zna przykłady przekształceń rzutowych i umie je stosować w zadaniach i dowodach twierdzeń rzutowych (Desarguesa, Pappusa, Pascala, Brianchona). zna pojęcia: biegun i biegunowa i potrafi formułować twierdzenia dualne.  
% wg Neugebauera, Brianchon jest dualny do Pascala

\begin{proposition}[twierdzenie Pascala]
	Neugebauer, strona 113.
	% TODO: https://en.wikipedia.org/wiki/Pascal%27s_theorem
	\index{twierdzenie!Pascala}
\end{proposition}

Audin \cite[s. 103, 107, 209]{audin_2003} podaje warianty tego twierdzenia.

\begin{theorem}[Pascala]
	Jeżeli punkty $p_1$, $p_2$, $p_3$, $p_4$, $p_5$, $p_6$ leżą na pewnej stożkowej, to punkty $p_1p_2 \cdot p_4p_5$, $p_2p_3 \cdot p_5p_6$ oraz $p_3p_4 \cdot p_6p_1$ są współliniowe.
	\index{twierdzenie!Pascala}
\end{theorem}

\begin{theorem}[Brianchona]
	Jeżeli proste $l_1$, $l_2$, $l_3$, $l_4$, $l_5$, $l_6$ są styczne do pewnej stożkowej, to proste $p = (l_1 \cdot l_2)(l_4 \cdot l_5)$, $q = (l_2 \cdot l_3)(l_5 \cdot l_6)$ oraz $r = (l_3 \cdot l_4)(l_6 \cdot l_1)$ są współpękowe.
	\index{twierdzenie!Brianchona}
\end{theorem}

%  Coxeter, H. S. M. (1987). Projective Geometry (2nd ed.). Springer-Verlag. Theorem 9.15, p. 83. ISBN 0-387-96532-7.
% The polar reciprocal and projective dual of this theorem give Pascal's theorem.

To sformułowanie pojawi się u Bogdańskiej, Neugebauera \cite[s. 265, 266]{neugebauer_2018}.
Napisze o nim także nieznany autor w $\Delta_{84}^{11}$, Eves \cite[s. 143]{eves1_1972}, a w prostszej wersji także Guzicki \cite[s. 238]{guzicki_2021}.
\todofoot{Twierdzenie Kirkmana: jeśli część wspólna dwóch trójkątów wpisanych w okrąg jest sześciokątem wypukłym, to główne przekątne tego sześciokąta przecinają się w jednym punkcie.}
\index{twierdzenie!Kirkmana}

% TODO: rewrite
The introduction into geometry of the notion of points at infinity is usually credited to Johann Kepler (1571-1630), but it was Gerard
Desargues (1593-1662) who, in a treatment of the conic sections
(his Brouillon projet) published in 1639, first used the idea systematically. This work of Desargues marks the first essential advance in synthetic geometry since the time of the ancient Greeks.

\section{Różności}
Twierdzenie Schloemilcha: trzy proste łączące środki boków trójkąta ze środkami odpowiednich wysokości są współpękowe % Neugebauer 195
\index{twierdzenie!Schloemilcha}
Twierdzenie Hirotaki: dany jest cykliczny, wówczas proste są współpękowe.
\index{twierdzenie!Hirotaki}

%