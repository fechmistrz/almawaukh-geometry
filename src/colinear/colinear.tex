
\subsection{Współliniowość}
\subsubsection{Neugebauer: Menelaosa}
Znamy trzy twierdzenia o współliniowości: ..., ... i twierdzenie o prostej Auberta ...

\begin{proposition}[twierdzenie Salmona]
	Dany jest okrąg oraz trzy jego różne cięciwy $PA$, $PB$, $PC$ takie, że przekrojem okręgów na średnicach $PA$, $PB$ (odpowiednio: $PB$, $PC$ i $PA$, $PC$) są punkty $P$, $M$ (odpowiednio: $P$, $K$ oraz $P$, $L$).
	Wtedy punkty $K$, $L$, $M$ są współliniowe.
\end{proposition}

\begin{proposition}[twierdzenie Menelaosa]
	...
	Wówczas punkty $K, L, M$ są współliniowe wtedy i tylko wtedy, gdy zachodzi
	\begin{equation}
		[AMB] [BKC] [CLA] = -1.
	\end{equation}
\end{proposition}
% https://en.wikipedia.org/wiki/Menelaus%27s_theorem
% (MENELAOS = guzicki-3)
% It is uncertain who actually discovered the theorem; however, the oldest extant exposition appears in Spherics by Menelaus. In this book, the plane version of the theorem is used as a lemma to prove a spherical version of the theorem.

% \begin{proposition}[twierdzenie Carnota???]
	% Neugebauer, strona 108.
% \end{proposition}

% https://en.wikipedia.org/wiki/Newton%E2%80%93Gauss_line#Existence_of_the_Newton%E2%88%92Gauss_line

\begin{proposition}
	Środki trzech przekątnych czworoboku zupełnego leżą na jednej prostej, zwaną prostą Newtona-Gaussa.
\end{proposition}

\subsubsection{Neugebauer: Desargues, płaszczyzna rzutowa}

\begin{proposition}[twierdzenie Desargues'a]
	Neugebauer, strona 109.
	% https://en.wikipedia.org/wiki/Desargues%27s_theorem
\end{proposition}
% 1. Zna pojęcie inwolucji rzutowych.   Zna i potrafi stosować twierdzenia inwolucyjne Desarguesa.  

\subsubsection{Neugebauer: Pascal}

\begin{proposition}[twierdzenie Pascala]
	Neugebauer, strona 113.
	% https://en.wikipedia.org/wiki/Pascal%27s_theorem
\end{proposition}


\begin{theorem}[Pascala]
	Jeżeli punkty $p_1$, $p_2$, $p_3$, $p_4$, $p_5$, $p_6$ leżą na pewnej stożkowej, to punkty $p_1p_2 \cdot p_4p_5$, $p_2p_3 \cdot p_5p_6$ oraz $p_3p_4 \cdot p_6p_1$ są współliniowe.
\end{theorem}

\begin{theorem}[Brianchona]
	Jeżeli proste $l_1$, $l_2$, $l_3$, $l_4$, $l_5$, $l_6$ są styczne do pewnej stożkowej, to proste $p = (l_1 \cdot l_2)(l_4 \cdot l_5)$, $q = (l_2 \cdot l_3)(l_5 \cdot l_6)$ oraz $r = (l_3 \cdot l_4)(l_6 \cdot l_1)$ są współpękowe.
\end{theorem}

%  Coxeter, H. S. M. (1987). Projective Geometry (2nd ed.). Springer-Verlag. Theorem 9.15, p. 83. ISBN 0-387-96532-7.
% The polar reciprocal and projective dual of this theorem give Pascal's theorem.

To sformułowanie pojawia się u Bogdańskiej, Neugebauera \cite[s. 265, 266]{neugebauer_2018}.


\subsubsection{Neugebauer: Pappus}

\begin{proposition}[twierdzenie Pappusa]
	Neugebauer, strona 114.
	% https://en.wikipedia.org/wiki/Pappus%27s_hexagon_theorem
\end{proposition}

% Hartshorne: 62, inne twierdzenie Pappusa?

% \item Zna przykłady przekształceń rzutowych i umie je stosować w zadaniach i dowodach twierdzeń rzutowych (Desarguesa, Pappusa, Pascala, Brianchona). zna pojęcia: biegun i biegunowa i potrafi formułować twierdzenia dualne.  
% wg Neugebauera, Brianchon jest dualny do Pascala

\subsection{Współpękowość}
\subsubsection{Zadanie Napoleona}

Zadanie Napoleona -- Neugebauer s. 116, okręgi Torricelliego.
\index{zadanie Napoleona}
\index{okręgi Torricelliego}
\index{punkt Torricelliego}

Zadanie Fermata -- Neugebauer, s. 117.
\index{zadanie Fermata}


\subsubsection{Twierdzenie Cevy}
Ważnym kryterium współpękowości trzech czewian jest:

\begin{proposition}[twierdzenie Cevy (1678)]
	Dany jest trójkąt $ABC$ i trzy różne od wierzchołków punkty $K \in BC$, $L \in CA$, $M \in AB$.
	Wówczas czewiany $AK$, $BL$, $CM$ są współpękowe wtedy i tylko wtedy, gdy
	\begin{equation}
		[AMB] [BKC] [CLA] = 1.
	\end{equation}
\end{proposition}
% Neugebauer 119
% Ceva = guzicki-3

% Neugebauer 120
TODO: Czewiany Gergonne'a.
\index{czewiany Gergonne'a}

\begin{definition}[punkt Gergonne'a] % Guzicki s. 134
\index{punkt!Gergonne'a}%
	Dany jest okrąg wpisany w~trójkąt $ABC$, styczny do boków $BC$, $AC$, $AB$ odpowiednio w~punktach $P$, $Q$ i $R$.
	Wtedy czewiany $AP$, $BQ$ i $CR$ są współpękowe.
\end{definition}

Są jeszcze trzy inne okręgi styczne do wszystkich trzech prostych, na których leżą boki trójkąta.
Nazywamy je okręgami dopisanymi.
\index{okrąg dopisany}

TODO: Punkt Nagela.
\index{punkt!Nagela}
\index{czewiany Nagela}



Punkt Lemoine'a.
% UW: Twierdzenie Cevy (wraz z trygonometryczną wersją), przykłady punktów szczególnych trójkąta: punkt Nagela (Guzicki-4), punkt Gergonne'a (guzicki4), punkt Lemoine'a.

\subsubsection{Twierdzenie Carnota (Neugebauer: przed Cevą)}
%

Uogólnieniem twierdzenia o współpękowości symetralnych boków trójkąta jest:

\begin{proposition}[twierdzenie Carnota]
\label{guzicki_6_13}%
	Dany jest trójkąt $ABC$ i punkty $D, E, F$ leżące odpowiednio na prostych $BC, CA, AB$.
	Niech prosta $k$ (odpowiednio: $l$, $m$) przechodzi przez punkt $D$ ($E$, $F$) i będzie prostopadła do prostej $BC$ ($CA$, $AB$).
	Wtedy proste $k$, $l$, $m$ mają punkt wspólny wtedy i tylko wtedy, gdy
	\begin{equation}
		|AF|^2 + |BD|^2 + |CE|^2 = |AE|^2 + |BF|^2 + |CD|^2.
	\end{equation}
\end{proposition}
% TODO: https://en.wikipedia.org/wiki/Carnot%27s_theorem_(perpendiculars)

Guzicki \cite[s. 176]{guzicki_2021} wyprowadza je z twierdzenia Pitagorasa, co pozwala mu dojść do trzech wniosków dotyczących istnienia punktów szczególnych trójkąta: \ref{guzicki_6_17}, \ref{guzicki_6_18}, \ref{guzicki_6_20}.

\begin{corollary}
\label{guzicki_6_17}%
    Symetralne trzech boków trójkąta mają punkt wspólny (środek okręgu opisanego na tym trójkącie).
\end{corollary}

Audin \cite[s. 61]{audin_2003} podaje ten fakt w formie ćwiczenia.


\begin{corollary}
\label{guzicki_6_18}%
    Proste zawierające wysokości trójkąta mają punkt wspólny (ortocentrum).
\index{ortocentrum}%
\end{corollary}

Tego samego dowodzi Pompe \cite[s. 38]{pompe_2022}, zmyślnie używając równoległoboków.
Audin \cite[s. 61]{audin_2003} podaje ten fakt w formie ćwiczenia.

\begin{corollary} % Guzicki, s. 132
\label{guzicki_6_20}%
    Dwusieczne kątów trójkąta mają punkt wspólny (środek okręgu wpisanego w ten trójkąt).
\end{corollary}

% Nie wiem, czy tu:
Środkowe przecinają się w jednym punkcie. % Coxeter, Introduction to Geometry, s. 10 <- przeczytaj to, nie tylko cytuj! + ćwiczenia: 3/4 <= 1

Punkt ten nazywa się po angielsku centroid, dla Archimedesa był środkiem ciężkości trójkąta o równomiernie rozłożonej masie.

%
% twierdzenie Carnota: trzy proste są współpunktowe wtw AF2 + BD2 + CE2 = AE2 + BF2 + CD2. Wniosek: symetralne są współpunktowe. GUZICKI-6

\subsection{Czewiany i symediany}
\subsubsection{Twierdzenie van Aubela, wzór Routha, równość Gergonne'a}
\subsubsection{Czewiany izotomiczne i izogonalne, twierdzenie Steinera}
\subsubsection{Symediany, punkt Lemoine'a}


\subsubsection{Do włączenia w powyższe podpodsekcje}
\begin{enumerate}
    \item twierdzenia Newtona i Brianchona (s. 237) - GUZICKI 9
    \item Twierdzenie Kirkmana: jeśli część wspólna dwóch trójkątów wpisanych w okrąg jest sześciokątem wypukłym, to główne przekątne tego sześciokąta przecinają się w jednym punkcie. - TO JEST BARDZIEJ POD JEDNOKŁADNOŚĆ (UW)
    \item Wg Wiki, to jest wniosek z Desarguesa/Menelaos: twierdzenie o środkach jednokładności trzech okręgów, patrz TODO w kodzie źródłowym % (chyba https://atcm.mathandtech.org/EP2016/contributed/4052016_21160.pdf), na UW po: 	- Twierdzenia o składaniu jednokładności i przesunięć, 
\end{enumerate}

\subsection{Inwersja względem okręgu}
Patrz Guzicki-20: twierdzenie Ptolemeusza, zadanie Apolloniusza, zadanie Sangaku.
% Coxeter s. 77: Magnus 1831 wymyślił ten termin
% tamże: Peaucellier's cell; Hart's linkage

\subsection{Izogonalne}
Punkty izogonalnie sprzężone w trójkącie. + Twierdzenie Menelausa. (UW1)

\subsection{Ćwiczenia Neugebauer}
Punkt Apoloniusza

Twierdzenie Schloemilcha: trzy proste łączące środki boków trójkąta ze środkami odpowiednich wysokości są współpękowe % Neugebauer 195

Twierdzenie Hirotaki: dany jest cykliczny, wówczas proste są współpękowe.