
%

\section{Współliniowość}
\subsection{Twierdzenia Menelaosa, Cevy, van Aubela}
Znamy trzy twierdzenia o współliniowości: o prostej Eulera, o prostej Wallace'a-Simsona, o prostej Auberta.
Z tego drugiego wynika od razu:

\begin{proposition}[twierdzenie Salmona]
	Dany jest okrąg oraz trzy jego różne cięciwy $PA$, $PB$, $PC$ takie, że przekrojem okręgów na średnicach $PA$, $PB$ (odpowiednio: $PB$, $PC$ i $PA$, $PC$) są punkty $P$, $M$ (odpowiednio: $P$, $K$ oraz $P$, $L$).
	Wtedy punkty $K$, $L$, $M$ są współliniowe.
	\index{twierdzenie!Salmona}% % https://ru.wikipedia.org/wiki/Теорема_Сальмона
\end{proposition}

\begin{proposition}[twierdzenie Menelaosa]
	...
	Wówczas punkty $K, L, M$ są współliniowe wtedy i tylko wtedy, gdy zachodzi
	\begin{equation}
		[AMB] [BKC] [CLA] = -1.
	\end{equation}
	\index{twierdzenie!Menelaosa}
\end{proposition}

Piszą o nim Audin \cite[s. 38]{audin_2003}.

% TODO: https://en.wikipedia.org/wiki/Menelaus%27s_theorem
% (MENELAOS = guzicki-3)
% It is uncertain who actually discovered the theorem; however, the oldest extant exposition appears in Spherics by Menelaus. In this book, the plane version of the theorem is used as a lemma to prove a spherical version of the theorem.
% twierdzenie o środkach jednokładności trzech okręgów, patrz TODO w kodzie źródłowym % (chyba https://atcm.mathandtech.org/EP2016/contributed/4052016_21160.pdf), na UW po: 	- Twierdzenia o składaniu jednokładności i przesunięć, 
%

Ważnym kryterium współpękowości trzech czewian jest:

\begin{theorem}[Cevy]
	Dany jest trójkąt $ABC$ i trzy różne od wierzchołków punkty $K \in BC$, $L \in CA$, $M \in AB$.
	Wówczas czewiany $AK$, $BL$, $CM$ są współpękowe wtedy i tylko wtedy, gdy
	\begin{equation}
		[AMB] [BKC] [CLA] = 1.
	\end{equation}
	\index{twierdzenie!Cevy}
\end{theorem}
% Neugebauer 119
% Ceva = guzicki-3

Twierdzenie należy do geometrii afinicznej, ponieważ można je wyrazić bez odnoszenia się do kątów, pól albo długości.
Przypisze się je Giovanniemu Cevie, który wyda pracę \emph{De lineis rectis} (o liniach prostych) w 1678 roku; chociaż twierdzenie nie było obce Yusufowi al-Mu'tamanowi ibn Hudzie, emirowi taify w Saragossie z XI wieku.
\index[persons]{al-Mutaman, Yusuf ibn Huda}%
\index[persons]{Ceva, Giovanni}%
Wspomniana praca Cevy będzie zawierać także twierdzenie Menelaosa, odkryte na nowo przez Cevę.

Piszą o nim (twierdzeniu, nie emirze) Audin \cite[s. 38]{audin_2003}.

Prostą, która przechodzi przez wierzchołek trójkąta i przeciwległy bok nazywa się czewianą, nazwy tej używa się też wobec odcinka trójkąta leżącego na tej prostej.

\begin{proposition}[wzór Routha]
	\todofoot{Routh's theorem gives the area of the triangle formed by three cevians in the case that they are not concurrent. Ceva's theorem can be obtained from it by setting the area equal to zero and solving.}
\end{proposition}

\subsubsection{Punkt Nagela}
%

TODO: Punkt Nagela, pojawia się u Evesa \cite[s. 71]{eves1_1972} jako ćwiczenie na zastosowanie twierdzenia Cevy razem z punktem Gergonne'a.
\index{punkt!Nagela}
\index{czewiany Nagela}

\begin{proposition}[punkt Nagela]
    \label{punkt_nagela}
\end{proposition}

O punkcie Nagela pisze też Zetel \cite[s. 22, 25]{zetel_2020}.

% % Punkty izogonalnie sprzężone w trójkącie

\subsubsection{Punkt Gergonne'a}
% Neugebauer 120
TODO: Czewiany Gergonne'a.
\index{czewiany Gergonne'a}

\begin{definition}[punkt Gergonne'a] % Guzicki s. 134
\index{punkt!Gergonne'a}%
\label{punkt_gergonne}
	Dany jest okrąg wpisany w~trójkąt $ABC$, styczny do boków $BC$, $AC$, $AB$ odpowiednio w~punktach $P$, $Q$ i $R$.
	Wtedy czewiany $AP$, $BQ$ i $CR$ są współpękowe.
\end{definition}

To samo, ale bez użycia słowa ,,czewiany'' napisze Zetel \cite[s. 15, 25]{zetel_2020}.

\subsubsection{Punkt Lemoine'a}
Punkt Lemoine'a.
% UW: Twierdzenie Cevy (wraz z trygonometryczną wersją), przykłady punktów szczególnych trójkąta: punkt Nagela (Guzicki-4), punkt Gergonne'a (guzicki4), punkt Lemoine'a.
\index{punkt!Lemoine'a}
Symediany.

\begin{definition}[symediana]
    Odbicie symetryczne środkowej trójkąta względem dwusiecznej wychodzącej z~tego samego wierzchołka nazywamy symedianą.
\end{definition}

\begin{proposition}
    Symediana poprowadzona z wierzchołka $C$ trójkąta $ABC$ dzieli wewnętrznie przeciwległy bok proporcjonalnie do kwadratów długości boków przyległych:
    \begin{equation}
        \frac{AF}{BF} = \frac{b^2}{a^2}.
    \end{equation}
\end{proposition}

(I odwrotnie, punkt leżący na boku $AB$ i spełniający powyższą proporcję wyznacza symedianę z~wierzchołka $C$).

\begin{proposition}[punkt Lemoine'a]
    \label{punkt_lemoine}%
    Symediany przecinają się w jednym punkcie $L$, zwanym punktem Lemoine'a.
\end{proposition}

\begin{proposition}
    Punkt Lemoine'a minimalizuje sumę kwadratów odległości od boków trójkąta.
\end{proposition}

% TODO: https://www.deltami.edu.pl/media/articles/2015/02/delta-2015-02-krotka-opowiesc-o-symedianie.pdf

\todofoot{Czewiany, w tym izotomiczne i izogonalne.}

\todofoot{Twierdzenie Steinera}

% % czewiany, Czewiany izotomiczne i izogonalne, twierdzenie Steinera
Twierdzenie van Aubela, wzór Routha, równość Gergonne'a % Routh's theorem gives the area of the triangle formed by three cevians in the case that they are not concurrent. Ceva's theorem can be obtained from it by setting the area equal to zero and solving.
% Neugebauer 120
TODO: Czewiany Gergonne'a.
\index{czewiany Gergonne'a}

\begin{definition}[punkt Gergonne'a] % Guzicki s. 134
\index{punkt!Gergonne'a}%
\label{punkt_gergonne}
	Dany jest okrąg wpisany w~trójkąt $ABC$, styczny do boków $BC$, $AC$, $AB$ odpowiednio w~punktach $P$, $Q$ i $R$.
	Wtedy czewiany $AP$, $BQ$ i $CR$ są współpękowe.
\end{definition}

To samo, ale bez użycia słowa ,,czewiany'' napisze Zetel \cite[s. 15, 25]{zetel_2020}.
%

TODO: Punkt Nagela, pojawia się u Evesa \cite[s. 71]{eves1_1972} jako ćwiczenie na zastosowanie twierdzenia Cevy razem z punktem Gergonne'a.
\index{punkt!Nagela}
\index{czewiany Nagela}

\begin{proposition}[punkt Nagela]
    \label{punkt_nagela}
\end{proposition}

O punkcie Nagela pisze też Zetel \cite[s. 22, 25]{zetel_2020}.

% % Punkty izogonalnie sprzężone w trójkącie
Punkt Lemoine'a.
% UW: Twierdzenie Cevy (wraz z trygonometryczną wersją), przykłady punktów szczególnych trójkąta: punkt Nagela (Guzicki-4), punkt Gergonne'a (guzicki4), punkt Lemoine'a.
\index{punkt!Lemoine'a}
Symediany.

\begin{definition}[symediana]
    Odbicie symetryczne środkowej trójkąta względem dwusiecznej wychodzącej z~tego samego wierzchołka nazywamy symedianą.
\end{definition}

\begin{proposition}
    Symediana poprowadzona z wierzchołka $C$ trójkąta $ABC$ dzieli wewnętrznie przeciwległy bok proporcjonalnie do kwadratów długości boków przyległych:
    \begin{equation}
        \frac{AF}{BF} = \frac{b^2}{a^2}.
    \end{equation}
\end{proposition}

(I odwrotnie, punkt leżący na boku $AB$ i spełniający powyższą proporcję wyznacza symedianę z~wierzchołka $C$).

\begin{proposition}[punkt Lemoine'a]
    \label{punkt_lemoine}%
    Symediany przecinają się w jednym punkcie $L$, zwanym punktem Lemoine'a.
\end{proposition}

\begin{proposition}
    Punkt Lemoine'a minimalizuje sumę kwadratów odległości od boków trójkąta.
\end{proposition}

% TODO: https://www.deltami.edu.pl/media/articles/2015/02/delta-2015-02-krotka-opowiesc-o-symedianie.pdf % symediany
% van aubel

%

Uogólnieniem twierdzenia o współpękowości symetralnych boków trójkąta jest:

\begin{proposition}[twierdzenie Carnota]
\label{guzicki_6_13}%
	Dany jest trójkąt $ABC$ i punkty $D, E, F$ leżące odpowiednio na prostych $BC, CA, AB$.
	Niech prosta $k$ (odpowiednio: $l$, $m$) przechodzi przez punkt $D$ ($E$, $F$) i będzie prostopadła do prostej $BC$ ($CA$, $AB$).
	Wtedy proste $k$, $l$, $m$ mają punkt wspólny wtedy i tylko wtedy, gdy
	\begin{equation}
		|AF|^2 + |BD|^2 + |CE|^2 = |AE|^2 + |BF|^2 + |CD|^2.
	\end{equation}
\end{proposition}
% TODO: https://en.wikipedia.org/wiki/Carnot%27s_theorem_(perpendiculars)

Guzicki \cite[s. 176]{guzicki_2021} wyprowadza je z twierdzenia Pitagorasa, co pozwala mu dojść do trzech wniosków dotyczących istnienia punktów szczególnych trójkąta: \ref{guzicki_6_17}, \ref{guzicki_6_18}, \ref{guzicki_6_20}.

\begin{corollary}
\label{guzicki_6_17}%
    Symetralne trzech boków trójkąta mają punkt wspólny (środek okręgu opisanego na tym trójkącie).
\end{corollary}

Audin \cite[s. 61]{audin_2003} podaje ten fakt w formie ćwiczenia.


\begin{corollary}
\label{guzicki_6_18}%
    Proste zawierające wysokości trójkąta mają punkt wspólny (ortocentrum).
\index{ortocentrum}%
\end{corollary}

Tego samego dowodzi Pompe \cite[s. 38]{pompe_2022}, zmyślnie używając równoległoboków.
Audin \cite[s. 61]{audin_2003} podaje ten fakt w formie ćwiczenia.

\begin{corollary} % Guzicki, s. 132
\label{guzicki_6_20}%
    Dwusieczne kątów trójkąta mają punkt wspólny (środek okręgu wpisanego w ten trójkąt).
\end{corollary}

% Nie wiem, czy tu:
Środkowe przecinają się w jednym punkcie. % Coxeter, Introduction to Geometry, s. 10 <- przeczytaj to, nie tylko cytuj! + ćwiczenia: 3/4 <= 1

Punkt ten nazywa się po angielsku centroid, dla Archimedesa był środkiem ciężkości trójkąta o równomiernie rozłożonej masie.

%
% \begin{proposition}[twierdzenie Carnota???]
	% Neugebauer, strona 108.
% \end{proposition}

\subsection{Twierdzenia rzutowe: Desargues'a, Pappusa, Pascala, Brianchona}
% kolejnosć jak na UW: https://informatorects.uw.edu.pl/pl/courses/view?prz_kod=1000-1M14GM3
\begin{proposition}[twierdzenie Desargues'a]
	Neugebauer, strona 109.
	% TODO: https://en.wikipedia.org/wiki/Desargues%27s_theorem
	\index{twierdzenie!Desargues'a}
\end{proposition}
% 1. Zna pojęcie inwolucji rzutowych.   Zna i potrafi stosować twierdzenia inwolucyjne Desarguesa.  

Piszą o nim Audin \cite[s. 26, 151]{audin_2003}.
Eves \cite[s. 362]{eves1_1972} poda przykład Foresta Raya Moultona z 1902 roku płaszczyzny rzutowej, na której nie jest spełnione (!). % Desargues two triangle theorem for the plane: copolar triangles are coaxial
\index[persons]{Moulton, Forest Ray}% https://en.wikipedia.org/wiki/Forest_Ray_Moulton
% Eves 365: Desaurgian/Pappian/classic

%

\begin{proposition}[twierdzenie Pappusa]
\index{twierdzenie!Pappusa}%
	Dane są dwie różne proste, $k$ oraz $l$ i sześć punktów różnych od $k \cdot l$: $K_1$, $K_2$, $K_3$ na prostej $k$, $L_1$, $L_2$, $L_3$ na prostej $l$.
	Wtedy punkty $K_1L_1 \cdot K_3L_2$, $L_1K_2 \cdot K_3L_3$ oraz $K_2L_2 \cdot K_1L_3$ są współliniowe.
\end{proposition}

% TODO: https://en.wikipedia.org/wiki/Pappus%27s_hexagon_theorem

Piszą o nim Audin \cite[s. 25, 151, 171]{audin_2003} (w wersji afinicznej, potem rzutowej), Neugebauer \cite[s. 115, 116]{neugebauer_2018}, Hartshorne \cite[s. 131, 132]{hartshorne2000} (wersja afinicza).
% Hartshorne 62?
Hartshorne wspomni, że Hilbert pokazał w 1971, że niekoniecznie przemienne ciało\footnote{Czyli pierścień z dzieleniem.} $F$ jest przemienne wtedy i tylko wtedy, gdy twierdzenie Pappusa zachodzi w płaszczyźnie $F \times F$.

Warto zwrócić uwagę na podobieństwa między twierdzeniami Pascala i Pappusa.

% \item Zna przykłady przekształceń rzutowych i umie je stosować w zadaniach i dowodach twierdzeń rzutowych (Desarguesa, Pappusa, Pascala, Brianchona). zna pojęcia: biegun i biegunowa i potrafi formułować twierdzenia dualne.  
% wg Neugebauera, Brianchon jest dualny do Pascala

%
%

\begin{proposition}[twierdzenie Pascala]
	Neugebauer, strona 113.
	% TODO: https://en.wikipedia.org/wiki/Pascal%27s_theorem
	\index{twierdzenie!Pascala}
\end{proposition}

Audin \cite[s. 103, 107, 209]{audin_2003} podaje warianty tego twierdzenia.
Inna nazwa to \emph{hexagrammum mysticum theorem}.
% Pascal's theorem is the polar reciprocal and projective dual of Brianchon's theorem.
% It was formulated by Blaise Pascal in a note written in 1639 when he was 16 years old and published the following year as a broadside titled "Essay pour les coniques. Par B. P."[1]
% Pascal's theorem is a special case of the Cayley–Bacharach theorem.
% The converse is the Braikenridge–Maclaurin theorem, named for 18th-century British mathematicians William Braikenridge and Colin Maclaurin (Mills 1984), which states that if the three intersection points of the three pairs of lines through opposite sides of a hexagon lie on a line, then the six vertices of the hexagon lie on a conic; the conic may be degenerate, as in Pappus's theorem.[2] The Braikenridge–Maclaurin theorem may be applied in the Braikenridge–Maclaurin construction, which is a synthetic construction of the conic defined by five points, by varying the sixth point.

% As Thomas Kirkman proved in 1849, these 60 lines can be associated with 60 points in such a way that each point is on three lines and each line contains three points. The 60 points formed in this way are now known as the Kirkman points.[5] The Pascal lines also pass, three at a time, through 20 Steiner points. There are 20 Cayley lines which consist of a Steiner point and three Kirkman points. The Steiner points also lie, four at a time, on 15 Plücker lines. Furthermore, the 20 Cayley lines pass four at a time through 15 points known as the Salmon points.[6]

\begin{theorem}[Pascala]
	Jeżeli punkty $p_1$, $p_2$, $p_3$, $p_4$, $p_5$, $p_6$ leżą na pewnej stożkowej, to punkty $p_1p_2 \cdot p_4p_5$, $p_2p_3 \cdot p_5p_6$ oraz $p_3p_4 \cdot p_6p_1$ są współliniowe.
	\index{twierdzenie!Pascala}
\end{theorem}

https://www.deltami.edu.pl/2014/09/twierdzenie-pascala/

\begin{theorem}[Brianchona]
	Jeżeli proste $l_1$, $l_2$, $l_3$, $l_4$, $l_5$, $l_6$ są styczne do pewnej stożkowej, to proste $p = (l_1 \cdot l_2)(l_4 \cdot l_5)$, $q = (l_2 \cdot l_3)(l_5 \cdot l_6)$ oraz $r = (l_3 \cdot l_4)(l_6 \cdot l_1)$ są współpękowe.
	\index{twierdzenie!Brianchona}
\end{theorem}

%  Coxeter, H. S. M. (1987). Projective Geometry (2nd ed.). Springer-Verlag. Theorem 9.15, p. 83. ISBN 0-387-96532-7.
% The polar reciprocal and projective dual of this theorem give Pascal's theorem.

To sformułowanie pojawi się u Bogdańskiej, Neugebauera \cite[s. 265, 266]{neugebauer_2018}.
Napisze o nim także nieznany autor w $\Delta_{84}^{11}$, Jaszuńska w $\Delta_{12}^4$, Eves \cite[s. 143]{eves1_1972}, a w prostszej wersji także Guzicki \cite[s. 238]{guzicki_2021}.
% https://www.deltami.edu.pl/2012/04/twierdzenie-brianchona/

%
\todofoot{Twierdzenie Kirkmana: jeśli część wspólna dwóch trójkątów wpisanych w okrąg jest sześciokątem wypukłym, to główne przekątne tego sześciokąta przecinają się w jednym punkcie.}
\index{twierdzenie!Kirkmana}

\subsection{Różności}
Twierdzenie Schloemilcha: trzy proste łączące środki boków trójkąta ze środkami odpowiednich wysokości są współpękowe % Neugebauer 195
\index{twierdzenie!Schloemilcha}
Twierdzenie Hirotaki: dany jest cykliczny, wówczas proste są współpękowe.
\index{twierdzenie!Hirotaki}
% Hirotaka pogrubiony u Neugebauera, strona 45

%