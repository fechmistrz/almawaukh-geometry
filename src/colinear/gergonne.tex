% Neugebauer 120
TODO: Czewiany Gergonne'a.
\index{czewiany Gergonne'a}

\begin{definition}[punkt Gergonne'a] % Guzicki s. 134
\index{punkt!Gergonne'a}%
\label{punkt_gergonne}
	Dany jest okrąg wpisany w~trójkąt $ABC$, styczny do boków $BC$, $AC$, $AB$ odpowiednio w~punktach $P$, $Q$ i $R$.
	Wtedy czewiany $AP$, $BQ$ i $CR$ są współpękowe.
\end{definition}
% Mathworld: https://mathworld.wolfram.com/GergonnePoint.html odsyła do perspector trójkąta ABC i A'B'C', gdzie... (perspector: przecięcie trzech prostych przez dwa trójkąty perspetktywiczne; inaczej: środek perspektywiczny, środek homologiczny (?) albo biegun (??))
% https://mathworld.wolfram.com/Perspector.html dużo przykładów, np. conic = Mandart inellipse odpowiada point = Nagel point

TODO: rysunek

To samo, ale bez użycia słowa ,,czewiany'' napisze Zetel \cite[s. 15, 25]{zetel_2020}.

% https://en.wikipedia.org/wiki/Incircle_and_excircles#Gergonne_triangle_and_point

%