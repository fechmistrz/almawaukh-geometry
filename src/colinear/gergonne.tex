% Neugebauer 120
TODO: Czewiany Gergonne'a.
\index{czewiany Gergonne'a}

\begin{definition}[punkt Gergonne'a] % Guzicki s. 134
\index{punkt!Gergonne'a}%
	Dany jest okrąg wpisany w~trójkąt $ABC$, styczny do boków $BC$, $AC$, $AB$ odpowiednio w~punktach $P$, $Q$ i $R$.
	Wtedy czewiany $AP$, $BQ$ i $CR$ są współpękowe.
\end{definition}
