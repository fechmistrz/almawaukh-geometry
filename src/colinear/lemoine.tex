Punkt Lemoine'a.
% UW: Twierdzenie Cevy (wraz z trygonometryczną wersją), przykłady punktów szczególnych trójkąta: punkt Nagela (Guzicki-4), punkt Gergonne'a (guzicki4), punkt Lemoine'a.
\index{punkt!Lemoine'a}
Symediany.

\begin{definition}[symediana]
    Odbicie symetryczne środkowej trójkąta względem dwusiecznej wychodzącej z~tego samego wierzchołka nazywamy symedianą.
\end{definition}

\begin{proposition}
    Symediana poprowadzona z wierzchołka $C$ trójkąta $ABC$ dzieli wewnętrznie przeciwległy bok proporcjonalnie do kwadratów długości boków przyległych:
    \begin{equation}
        \frac{AF}{BF} = \frac{b^2}{a^2}.
    \end{equation}
\end{proposition}

(I odwrotnie, punkt leżący na boku $AB$ i spełniający powyższą proporcję wyznacza symedianę z~wierzchołka $C$).

\begin{proposition}[punkt Lemoine'a]
    \label{punkt_lemoine}%
    Symediany przecinają się w jednym punkcie $L$, zwanym punktem Lemoine'a.
\end{proposition}

\begin{proposition}
    Punkt Lemoine'a minimalizuje sumę kwadratów odległości od boków trójkąta.
\end{proposition}

% TODO: https://www.deltami.edu.pl/media/articles/2015/02/delta-2015-02-krotka-opowiesc-o-symedianie.pdf