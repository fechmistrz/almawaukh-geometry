Punkt Lemoine'a.
% UW: Twierdzenie Cevy (wraz z trygonometryczną wersją), przykłady punktów szczególnych trójkąta: punkt Nagela (Guzicki-4), punkt Gergonne'a (guzicki4), punkt Lemoine'a.
\index{punkt!Lemoine'a}
Symediany.

https://www.deltami.edu.pl/2016/05/izogonalne-sprzezenie-i-symediany/

\begin{definition}[symediana]
    Odbicie symetryczne środkowej trójkąta względem dwusiecznej wychodzącej z~tego samego wierzchołka nazywamy symedianą.
\end{definition}

https://www.deltami.edu.pl/2015/02/krotka-opowiesc-o-symedianie/

Innymi słowy symediana jest to prosta izogonalna ze środkową (Zetel \cite[s. 111]{zetel_2020}).
Symediana jest miejscem geometrycznym punktów, których odległości od dwóch boków trójkąta są proporcjonalne do tych boków.

\begin{proposition}
    Symediana poprowadzona z wierzchołka $C$ trójkąta $ABC$ dzieli wewnętrznie przeciwległy bok proporcjonalnie do kwadratów długości boków przyległych:
    \begin{equation}
        \frac{AF}{BF} = \frac{b^2}{a^2}.
    \end{equation}
\end{proposition}

(I odwrotnie, punkt leżący na boku $AB$ i spełniający powyższą proporcję wyznacza symedianę z~wierzchołka $C$).
Zetel \cite[s. 111, 112]{zetel_2020} uważa to za szczególny przypadek twierdzenie Steinera o~prostych izogonalnych.
(Chyba nie ma tego twierdzenia w tej książce).

\begin{proposition}
    Symediana wychodząca z wierzchołka $A$ trójkąta $ABC$ ma długość
    \begin{equation}
        \frac{bc}{b^2 + c^2} \cdot \sqrt {2b^2 + 2c^2 - a^2}.
    \end{equation}
\end{proposition}

Zetel \cite[s. 119]{zetel_2020} pisze, że można dostać tę długość z twierdzenia Stewarta, ale nie trzeba.

\begin{proposition}[punkt Lemoine'a]
    \label{punkt_lemoine}%
    Symediany przecinają się w jednym punkcie $L$, zwanym punktem Lemoine'a.
\end{proposition}

Zetel \cite[s. 114]{zetel_2020} sugeruje udowodnić na podstawie twierdzenia odwrotnego do twierdzenia Cevy, że punkt Lemoine'a istnieje.

\begin{proposition}
    Punkt Lemoine'a minimalizuje sumę kwadratów odległości od boków trójkąta.
\end{proposition}

% TODO: https://www.deltami.edu.pl/media/articles/2015/02/delta-2015-02-krotka-opowiesc-o-symedianie.pdf