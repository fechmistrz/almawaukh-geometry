Znamy trzy twierdzenia o współliniowości: o prostej Eulera, o prostej Wallace'a-Simsona, o prostej Auberta.

\begin{proposition}[twierdzenie Menelaosa]
	...
	Wówczas punkty $K, L, M$ są współliniowe wtedy i tylko wtedy, gdy zachodzi
	\begin{equation}
		[AMB] [BKC] [CLA] = -1.
	\end{equation}
	\index{twierdzenie!Menelaosa}
\end{proposition}

\index[persons]{Menelaos z Aleksandrii}%

Piszą o nim Audin \cite[s. 38]{audin_2003}, Zetel \cite[s. 43]{zetel_2020}.
Uogólnieniem twierdzenia Menelaosa będzie twierdzenie Carnota (udowodnione w 1803 roku w \emph{,,Géométrie de position}''), że iloczyn długości odcinków wyznaczonych przez prostą na bokach wielokąta i niemających wspólnych końców jest równy iloczynowi długości pozostałych odcinków; 
% https://encyclopediaofmath.org/wiki/Carnot_theorem
% Zetel 45.

% TODO: https://en.wikipedia.org/wiki/Menelaus%27s_theorem
% (MENELAOS = guzicki-3)
% It is uncertain who actually discovered the theorem; however, the oldest extant exposition appears in Spherics by Menelaus. In this book, the plane version of the theorem is used as a lemma to prove a spherical version of the theorem.