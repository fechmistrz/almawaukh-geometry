%

TODO: Punkt Nagela, pojawia się u Evesa \cite[s. 71]{eves1_1972} jako ćwiczenie na zastosowanie twierdzenia Cevy razem z punktem Gergonne'a.
\index{punkt!Nagela}
\index{czewiany Nagela}

\begin{proposition}[punkt Nagela]
    \label{punkt_nagela}
\end{proposition}

O punkcie Nagela pisze też Zetel \cite[s. 22, 25, 64]{zetel_2020} (punkt Nagela, punkt przecięcia środkowych -- środek ciężkości trójkąta, środek okręgu wpisanego oraz środek ciężkości obwodu trójkąta leżą na jednej prostej zwanej drugą prostą Eulera albo prostą Eulera-Nagela).

%