%

\begin{proposition}[twierdzenie Pappusa]
\index{twierdzenie!Pappusa}%
	Dane są dwie różne proste, $k$ oraz $l$ i sześć punktów różnych od $k \cdot l$: $K_1$, $K_2$, $K_3$ na prostej $k$, $L_1$, $L_2$, $L_3$ na prostej $l$.
	Wtedy punkty $K_1L_1 \cdot K_3L_2$, $L_1K_2 \cdot K_3L_3$ oraz $K_2L_2 \cdot K_1L_3$ są współliniowe.
\end{proposition}

% TODO: https://en.wikipedia.org/wiki/Pappus%27s_hexagon_theorem

Piszą o nim Audin \cite[s. 25, 151, 171]{audin_2003} (w wersji afinicznej, potem rzutowej), Neugebauer \cite[s. 115, 116]{neugebauer_2018}, Hartshorne \cite[s. 131, 132]{hartshorne2000} (wersja afinicza).
% Hartshorne 62?
Hartshorne wspomni, że Hilbert pokazał w 1971, że niekoniecznie przemienne ciało\footnote{Czyli pierścień z dzieleniem.} $F$ jest przemienne wtedy i tylko wtedy, gdy twierdzenie Pappusa zachodzi w płaszczyźnie $F \times F$.

Warto zwrócić uwagę na podobieństwa między twierdzeniami Pascala i Pappusa.

% \item Zna przykłady przekształceń rzutowych i umie je stosować w zadaniach i dowodach twierdzeń rzutowych (Desarguesa, Pappusa, Pascala, Brianchona). zna pojęcia: biegun i biegunowa i potrafi formułować twierdzenia dualne.  
% wg Neugebauera, Brianchon jest dualny do Pascala

%