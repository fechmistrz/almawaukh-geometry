\begin{proposition}[twierdzenie Pappusa]
	Neugebauer, strona 114.
	% https://en.wikipedia.org/wiki/Pappus%27s_hexagon_theorem
	\index{twierdzenie!Pappusa}
\end{proposition}
Piszą o nim Audin \cite[s. 25, 151, 171]{audin_2003} (w wersji afinicznej, potem rzutowej).
Hartshorne: 62, inne twierdzenie Pappusa?

% \item Zna przykłady przekształceń rzutowych i umie je stosować w zadaniach i dowodach twierdzeń rzutowych (Desarguesa, Pappusa, Pascala, Brianchona). zna pojęcia: biegun i biegunowa i potrafi formułować twierdzenia dualne.  
% wg Neugebauera, Brianchon jest dualny do Pascala
