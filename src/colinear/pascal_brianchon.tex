%

\begin{proposition}[twierdzenie Pascala]
	Neugebauer, strona 113.
	% TODO: https://en.wikipedia.org/wiki/Pascal%27s_theorem
	\index{twierdzenie!Pascala}
\end{proposition}

Audin \cite[s. 103, 107, 209]{audin_2003} podaje warianty tego twierdzenia.
Inna nazwa to \emph{hexagrammum mysticum theorem}.
% Pascal's theorem is the polar reciprocal and projective dual of Brianchon's theorem.
% It was formulated by Blaise Pascal in a note written in 1639 when he was 16 years old and published the following year as a broadside titled "Essay pour les coniques. Par B. P."[1]
% Pascal's theorem is a special case of the Cayley–Bacharach theorem.
% The converse is the Braikenridge–Maclaurin theorem, named for 18th-century British mathematicians William Braikenridge and Colin Maclaurin (Mills 1984), which states that if the three intersection points of the three pairs of lines through opposite sides of a hexagon lie on a line, then the six vertices of the hexagon lie on a conic; the conic may be degenerate, as in Pappus's theorem.[2] The Braikenridge–Maclaurin theorem may be applied in the Braikenridge–Maclaurin construction, which is a synthetic construction of the conic defined by five points, by varying the sixth point.

% As Thomas Kirkman proved in 1849, these 60 lines can be associated with 60 points in such a way that each point is on three lines and each line contains three points. The 60 points formed in this way are now known as the Kirkman points.[5] The Pascal lines also pass, three at a time, through 20 Steiner points. There are 20 Cayley lines which consist of a Steiner point and three Kirkman points. The Steiner points also lie, four at a time, on 15 Plücker lines. Furthermore, the 20 Cayley lines pass four at a time through 15 points known as the Salmon points.[6]

\begin{theorem}[Pascala]
	Jeżeli punkty $p_1$, $p_2$, $p_3$, $p_4$, $p_5$, $p_6$ leżą na pewnej stożkowej, to punkty $p_1p_2 \cdot p_4p_5$, $p_2p_3 \cdot p_5p_6$ oraz $p_3p_4 \cdot p_6p_1$ są współliniowe.
	\index{twierdzenie!Pascala}
\end{theorem}

https://www.deltami.edu.pl/2014/09/twierdzenie-pascala/

\begin{theorem}[Brianchona]
	Jeżeli proste $l_1$, $l_2$, $l_3$, $l_4$, $l_5$, $l_6$ są styczne do pewnej stożkowej, to proste $p = (l_1 \cdot l_2)(l_4 \cdot l_5)$, $q = (l_2 \cdot l_3)(l_5 \cdot l_6)$ oraz $r = (l_3 \cdot l_4)(l_6 \cdot l_1)$ są współpękowe.
	\index{twierdzenie!Brianchona}
\end{theorem}

%  Coxeter, H. S. M. (1987). Projective Geometry (2nd ed.). Springer-Verlag. Theorem 9.15, p. 83. ISBN 0-387-96532-7.
% The polar reciprocal and projective dual of this theorem give Pascal's theorem.

To sformułowanie pojawi się u Bogdańskiej, Neugebauera \cite[s. 265, 266]{neugebauer_2018}.
Napisze o nim także nieznany autor w $\Delta_{84}^{11}$, Jaszuńska w $\Delta_{12}^4$, Eves \cite[s. 143]{eves1_1972}, a w prostszej wersji także Guzicki \cite[s. 238]{guzicki_2021}.
% https://www.deltami.edu.pl/2012/04/twierdzenie-brianchona/

%