\begin{proposition}[twierdzenie Pascala]
	Neugebauer, strona 113.
	% TODO: https://en.wikipedia.org/wiki/Pascal%27s_theorem
	\index{twierdzenie!Pascala}
\end{proposition}

Audin \cite[s. 103, 107, 209]{audin_2003} podaje warianty tego twierdzenia.

\begin{theorem}[Pascala]
	Jeżeli punkty $p_1$, $p_2$, $p_3$, $p_4$, $p_5$, $p_6$ leżą na pewnej stożkowej, to punkty $p_1p_2 \cdot p_4p_5$, $p_2p_3 \cdot p_5p_6$ oraz $p_3p_4 \cdot p_6p_1$ są współliniowe.
	\index{twierdzenie!Pascala}
\end{theorem}

\begin{theorem}[Brianchona]
	Jeżeli proste $l_1$, $l_2$, $l_3$, $l_4$, $l_5$, $l_6$ są styczne do pewnej stożkowej, to proste $p = (l_1 \cdot l_2)(l_4 \cdot l_5)$, $q = (l_2 \cdot l_3)(l_5 \cdot l_6)$ oraz $r = (l_3 \cdot l_4)(l_6 \cdot l_1)$ są współpękowe.
	\index{twierdzenie!Brianchona}
\end{theorem}

%  Coxeter, H. S. M. (1987). Projective Geometry (2nd ed.). Springer-Verlag. Theorem 9.15, p. 83. ISBN 0-387-96532-7.
% The polar reciprocal and projective dual of this theorem give Pascal's theorem.

To sformułowanie pojawi się u Bogdańskiej, Neugebauera \cite[s. 265, 266]{neugebauer_2018}.
Napisze o nim także nieznany autor w $\Delta_{84}^{11}$, Eves \cite[s. 143]{eves1_1972}