Nazwy elipsa, parabola i hiperbola zostały wprowadzone przez Apoloniusza i zaczerpnięte z wczesnej pitagorejskiej terminologii dotyczącej przykładania pól. Gdy Pitagorejczycy przykładali prostokąt do odcinka (czyli ustawiali jego podstawę wzdłuż odcinka tak, że jeden koniec podstawy pokrywał się z końcem odcinka), mówili o „ellipsie”, „paraboli” lub „hyperboli” w zależności od tego, czy podstawa prostokąta była krótsza od odcinka, dokładnie mu odpowiadała, czy też go przekraczała.
% to jest Eves, s. 30, s. 31

\todofoot{Ogniska elipsy i hiperboli, ognisko, kierownica i mimośród stożkowych, asymptoty hiperboli, konstrukcja stycznej do stożkowej, rzuty ustalonego ogniska na styczne, własności izogonalne stożkowych, równania kanoniczne stożkowych, elipsa jako przekrój walca.}
\todofoot{Ognisko, kierownica i mimośród stożkowej na przekroju stożka.}
\todofoot{Przekroje stożków ze sferami wpisanymi.}
\todofoot{Równanie ogólne stożkowej w układzie współrzędnych, duży i mały wyznacznik.}
\todofoot{Równania stożkowych we współrzędnych biegunowych.}

Audin \cite[s. 183]{audin_2003} % conics, quadrics
% https://en.wikipedia.org/wiki/Circumconic_and_inconic

% https://en.wikipedia.org/wiki/Five_points_determine_a_conic#Synthetic_proof
\todofoot{pięć punktów wyznacza stożkową}

% https://en.wikipedia.org/wiki/Parabola
Parabola: krzywa będąca zbiorem punktów równoodległych od prostej zwanej kierownicą paraboli i punktu zwanego ogniskiem paraboli.

% https://en.wikipedia.org/wiki/Hyperbola
% https://en.wikipedia.org/wiki/Conic_section + https://en.wikipedia.org/wiki/Degenerate_conic + https://en.wikipedia.org/wiki/Generalized_conic

% https://en.wikipedia.org/wiki/Dandelin_spheres
In geometry, the Dandelin spheres are one or two spheres that are tangent both to a plane and to a cone that intersects the plane. The intersection of the cone and the plane is a conic section, and the point at which either sphere touches the plane is a focus of the conic section, so the Dandelin spheres are also sometimes called focal spheres.[1]
The Dandelin spheres were discovered in 1822.[1][2] They are named in honor of the French mathematician Germinal Pierre Dandelin, though Adolphe Quetelet is sometimes given partial credit as well.[3][4][5]
The Dandelin spheres can be used to give elegant modern proofs of two classical theorems known to Apollonius. The first theorem is that a closed conic section (i.e. an ellipse) is the locus of points such that the sum of the distances to two fixed points (the foci) is constant. The second theorem is that for any conic section, the distance from a fixed point (the focus) is proportional to the distance from a fixed line (the directrix), the constant of proportionality being called the eccentricity.[6]