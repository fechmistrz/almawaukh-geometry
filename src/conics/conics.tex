The names ellipse, parabola, and hyperbola were supplied by Apollo
nius, and were borrowed from the early Pythagorean terminology of
application of areas. When the Pythagoreans applied a rectangle to a
line segment (that is, placed the base of the rectangle along the line
segment, with one end of the base coinciding with one end of the seg
ment) they said they had a case of “ellipsis,” “parabole,” or “hyper¬bole” according as the base of the applied rectangle fell short of the
line segment, exactly coincided with it, or exceeded it.
% to jest Eves, s. 30, s. 31



\todofoot{Ogniska elipsy i hiperboli, ognisko, kierownica i mimośród stożkowych, asymptoty hiperboli, konstrukcja stycznej do stożkowej, rzuty ustalonego ogniska na styczne, własności izogonalne stożkowych, równania kanoniczne stożkowych, elipsa jako przekrój walca.}
\todofoot{Ognisko, kierownica i mimośród stożkowej na przekroju stożka.}
\todofoot{Przekroje stożków ze sferami wpisanymi.}
\todofoot{Równanie ogólne stożkowej w układzie współrzędnych, duży i mały wyznacznik.}
\todofoot{Równania stożkowych we współrzędnych biegunowych.}

Audin \cite[s. 183]{audin_2003} % conics, quadrics
% https://en.wikipedia.org/wiki/Circumconic_and_inconic

% https://en.wikipedia.org/wiki/Five_points_determine_a_conic#Synthetic_proof
\todofoot{pięć punktów wyznacza stożkową}

% https://en.wikipedia.org/wiki/Parabola
% https://en.wikipedia.org/wiki/Hyperbola
% https://en.wikipedia.org/wiki/Conic_section + https://en.wikipedia.org/wiki/Degenerate_conic + https://en.wikipedia.org/wiki/Generalized_conic
% https://en.wikipedia.org/wiki/Circular_section
% https://en.wikipedia.org/wiki/Dandelin_spheres