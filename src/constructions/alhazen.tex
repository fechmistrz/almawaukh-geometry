% TODO: Alhazen => https://en.wikipedia.org/wiki/Straightedge_and_compass_construction => niemożliwy
% https://en.wikipedia.org/wiki/Alhazen%27s_problem

s. 278 Hartshorne: problem Alhazen, równokąty widziane z dwóch punktów na okręgu.
\todofoot{Problem bilardowy Alhazena
Opisać w danym okręgu trójkąt równoramienny, którego ramiona przechodzą przez dwa dane punkty wewnątrz okręgu.
Problem ten pochodzi od arabskiego matematyka Abu Alego al Hassana ibn al Hassana ibn Alhajtama (ok. 965-ok. 1039), którego imię zostało przekształcone na Alhazen przez tłumaczy jego dzieła Optyka.
W Optyce problem ten ma następującą postać:
„Znajdź punkt na wklęsłym zwierciadle sferycznym, w który musi trafić promień światła wychodzący z danego punktu, aby po odbiciu dotarł do innego danego punktu.”
Problem ten można sformułować także w inny sposób, np.:
„Na okrągłym stole bilardowym znajdują się dwie bile; w jaki sposób należy uderzyć jedną z nich, aby po odbiciu od bandy trafiła w drugą?”
lub
„Na okręgu znajdź punkt, którego suma odległości od dwóch danych punktów wewnątrz okręgu jest minimalna (lub maksymalna).”
Z problemem tym mierzyło się wielu wybitnych matematyków po Alhazenie, m.in. Huygens, Barrow, de L’Hôpital, Riccati i Quetelet.
% In 1997, the Oxford mathematician Peter M. Neumann proved the theorem that there is no ruler-and-compass construction for the general solution of the ancient Alhazen's problem (billiard problem or reflection from a spherical mirror).[10][11]
}