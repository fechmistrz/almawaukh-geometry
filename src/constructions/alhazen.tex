%

% TODO: Alhazen => https://en.wikipedia.org/wiki/Straightedge_and_compass_construction => niemożliwy
% https://en.wikipedia.org/wiki/Alhazen%27s_problem

\begin{problem}[bilardowy Alhazena]
    Dany jest okrąg $\Gamma$ oraz dwa punkty $A$, $B$, obydwa w jego wnętrzu lub obydwa na zewnątrz.
    Skonstruować trójkąt równoramienny wpisany w okrąg $\Gamma$ tak, żeby punkt $A$ leżał na jednym ramieniu, zaś $B$ na drugim.
\end{problem}

Tematyką zajmie się już Ptolemeusz około II wieku naszej ery, ale dopiero XI-wieczny matematyk arabski, Alhazen (Hasan Ibn al-Haytham), sformułuje problem ogólniej, a oprócz tego przedstawi też jego rozwiązanie w książce \emph{Kitab al-Manazur} (łacińskie \emph{De aspectibus}, o optyce).
Około 1965 roku, amerykański matematyk Jack Elkin rozwiąże problem algebraicznie; pojawi się tam pierwiastek równania czwartego stopnia, więc trójkąta nie da się wykreślić samym cyrklem z linijką.
Al-Haytham i inni (na przykład Christiaan Huygens) wykorzystają wcześniej przecięcie stożkowych.

Problem ma prostą interpretację fizyczną, dlatego też znalazł się w dziele o optyce.
Promień światłą biegnie z punktu $A$ aż uderzy w powierzchnię kuli, po czym zmienia kierunek tak, że kąty między promieniem przed i po uderzeniu oraz normalną do powierzchni sfery są równe.
Szukamy takiego punktu, by po odbiciu trafić w punkt $B$.

Więcej przeczytać można u Hartshorne'a \cite[s. 278]{hartshorne2000}.

https://www.deltami.edu.pl/2017/06/zadanie-alhazena/

%