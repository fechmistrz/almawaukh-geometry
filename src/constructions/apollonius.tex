%

\todofoot{GUZICKI-19 -- wykorzystuje twierdzenie menelaosa}
% https://en.wikipedia.org/wiki/Problem_of_Apollonius
\begin{problem}
    \label{problem_apolloniusza}%
    ...
    Skonstruować ...
    \index{okrąg!Apoloniusza}
\end{problem}

Eves \cite[s. 164]{eves1_1972} w szczególnym przypadku, kiedy trzy okręgi są współpękowe

Z zadaniem wiąże się dużo późniejszy wynik.
W dwóch listach z 1643 roku do czeskiej księżniczki Elżbiety z Palatynatu Kartezjusz przedstawi uproszczoną wersję, w której trzy okręgi są parami styczne i znajdzie (bez dowodu) relację wiążącą promienie okręgów.

\begin{theorem}[Kartezjusza]
    Dane są trzy okręgi, parami styczne o promieniach $r_1, r_2, r_3$ oraz czwarty styczny do każdego z nich (istnieją dwa takie, jeden z~nich zazwyczaj styczny wewnętrznie, a w szczególnym przypadku może okazać się prostą).
    Definiujemy krzywiznę ze znakiem jako $\pm 1 / r$, gdzie znak $-$ odpowiada okręgowi stycznemu zewnętrznie.
    Wtedy
    \begin{equation}
        \left(\pm \frac{1}{r_1} \pm  \frac{1}{r_2} \pm  \frac{1}{r_3} \pm  \frac{1}{r_4}\right)^2 = 2 \left(\frac{1}{r_1^2} + \frac{1}{r_2^2} + \frac{1}{r_3^2} + \frac{1}{r_4^2}\right),
    \end{equation}
    gdzie $r_4$ oznacza promień czwartego okręgu.
\end{theorem}

Jeżeli jeden z okręgów (dajmy na to ten o promieniu $r_1$) staje się prostą, to jego zerowa krzywizna znika z równania.
Jeżeli dwa okręgi stają się dwiema prostymi, to pozostałem dwa muszą być do siebie przystające, co należy uznać za zdegenerowany przypadek.

Twierdzenie Kartezjusza odkryją na nowo Yamaji Nuchizumi (1751), Jakob Steiner (\emph{Fortsetzung der geometrischen Betrachtungen} z 1826), Philip Beecroft (\emph{Properties of circles in mutual contac} z 1842) i~wreszcie Frederick Soddy (\emph{The Kiss Precise} z 1936 jak opisane u Coxetera \cite[s. 29-31]{coxeter_1967}).
\index[persons]{Nuchizumi, Yamaji}%
\index[persons]{Steiner, Jakob}%
\index[persons]{Beecroft, Philip}%
\index[persons]{Soddy, Frederick}%
Ten ostatni poda uogólnienie do sfer, a rok później Thorold Gosset przeskoczy do przestrzeni dowolnego wymiaru ($\mathbb R^n$), z $n$ w miejsce czynnika $2$.
\index[persons]{Gosset, Thorold}%

Dowody korzystają z inwersji, wzoru Herona, wyznacznika Cayleya-Mengera, albo łańcuchów Pappusa i twierdzenia Vivianiego.
Patrz też $\Delta_{10}^6$. %. % https://www.deltami.edu.pl/media/articles/2010/06/delta-2010-06-czwarty-okrag.pdf

\begin{example}
    Dane są trzy styczne zewnętrznie okręgi o promieniach $r_1 = r_2 = r_3 = \sqrt{3}$.
    Czwarty okrąg styczny do nich wszystkich ma wtedy krzywiznę $\sqrt 3 \pm 2$.
\end{example}

Jeszcze nie skończyliśmy z Soddym.
W wierzchołkach trójkąta o obwodzie $2p = a + b + c$ wstawiamy okręgi o promieniach $p - a$, $p - b$, $p - c$.
Wtedy czwarty okrąg (zwany okręgiem Soddy'ego) z twierdzenia Kartezjusza ma promień
\begin{equation}
    \frac{4R + r \pm 2p}{S},
\end{equation}
gdzie $R, r$ to promienie okręgu opisanego i wpisanego, zaś $S$ to pole powierzchni.
Napisze o tym Coxeter \cite[s. 29-31]{coxeter_1967}

\begin{definition}[prosta Soddy'ego]
    Prostą przechodzącą przez środki okręgów Soddy'ego nazywamy prostą Soddy'ego.
\end{definition}

\begin{proposition}
    Prosta Soddy'ego przechodzi przez środek okręgu wpisanego, punkt Gergonne'a i punkt Eppsteina (punkt, gdzie przecinają się trzy proste przechodzące przez trzy pary spośród sześciu punktów styczności czterech okręgów...)
\index{punkt!Eppsteina}%
\end{proposition}

Prosta Soddy'ego ma pewien związek z punktem de Lonchampsa, którego nie opiszemy.
\index{punkt!de Longchampsa}%
% bo nie chce mi się wczytywać w szczegóły artykułu na wiki o Soddy'ego

%