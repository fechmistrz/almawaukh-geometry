%

Problemy od \ref{broken_ruler_compass_hartshorne_start} do \ref{broken_ruler_compass_hartshorne_end} pochodzą z książki Hartshorne'a \cite[s. 25, 26]{hartshorne2000}.

\begin{geoconstruction}[połamana linijka]
\label{broken_ruler_compass_hartshorne_start}%
\index{linijka!połamana}%
    Dane są dwa punkty $A$ i $B$ na płaszczyźnie, odległe od siebie o około trzy nible.
    Mając do dyspozycji fragment linijki o długości jednej nibli oraz sprawny cyrkiel, narysować odcinek $AB$.
\end{geoconstruction}
% Hartshorne s. 25
O problemie pisze Eves \cite[s. 181]{eves1_1972}.

\begin{geoconstruction}[zardzewiały cyrkiel]
    \index{cyrkiel!zardzewiały}
    Dane są dwa punkty $A$ i $B$ na płaszczyźnie, odległe od siebie o około pięć nibli.
    Mając do dyspozycji zardzewiały cyrkiel, którym można kreślić jedynie okręgi o promieniu dwóch nibli, skonstruować trójkąt równoboczny oparty o bok $AB$.
\end{geoconstruction}

Konstrukcje zardzewiałym cyrklem były rozpatrywane przez perskiego matematyka Abu al-Wafę Buzjaniego (940-998).
\index[persons]{Buzjani, Abu al-Wafa}%
Miały praktyczne znaczenie, ponieważ stosowali je Leonardo da Vinci czy też Albrecht Dürer pod koniec piętnastego wieku.
\index[persons]{Dürer, Albrech}%
\index[persons]{da Vinci, Leonardo}%
% ŹRÓDŁO: https://en.wikipedia.org/wiki/Poncelet-Steiner_theorem#History

\begin{geoconstruction}
    Dany jest punkt $A$ leżący na prostej $l$.
    Skonstruować prostą prostopadłą do $l$ przechodzącą przez $A$ przy użyciu linijki i zardzewiałego cyrkla.
\end{geoconstruction}
% Hartshorne s. 25

\begin{geoconstruction}
    Dany jest punkt $A$ leżący ponad cztery nible od prostej $l$.
    Skonstruować prostą prostopadłą do $l$ przechodzącą przez $A$ przy użyciu linijki i zardzewiałego cyrkla.
\end{geoconstruction}
% Hartshorne s. 25

\begin{geoconstruction}
    Dane są trzy niewspółliniowe punkty $A$, $B$ oraz $C$.
    Skonstrować punkt $D$ na prostej $AC$ tak, żeby odcinki $AD$ oraz $AB$ były równej długości, przy użyciu linijki i zardzewiałego cyrkla.
\end{geoconstruction}
% Hartshorne s. 26

\begin{geoconstruction}
    Dany jest odcinek $AB$ o długości ponad dwóch nibli oraz prosta $l$, która nie przechodzi przez końce odcinka.
    Skonstrować punkt $C$ na prostej $l$ tak, żeby odcinki $AB$ oraz $AC$ były równej długości, przy użyciu linijki i zardzewiałego cyrkla.
\end{geoconstruction}
% Hartshorne s. 26

\begin{proposition}
\label{broken_ruler_compass_hartshorne_end}%
    Każdą konstrukcję geometryczną, którą można wykonać cyrklem i linijką, można powtórzyć przy użyciu zardzewiałego cyrkla i linijki.
\end{proposition}
% Hartshorne s. 26

Wynik ten zostanie odkryty relatywnie wcześnie.
Pierwszy pokaże go Lodovico Ferrari, uczeń Gerolamo Cardano i odkrywca metody rozwiązywania równań czwartego stopnia w pojedynku przeciwko Niccolò Fontanie Tartaglii.
W przeciągu dziesięciu lat wyczyn ten powtórzą Cardano, Tartaglia oraz uczeń Tartaglii, Benedetti.

Ciąg dalszy to oczywiście twierdzenie \ref{thm:poncelet_steiner}, że zardzewiałego cyrkla wystarczy użyć raz.

%