%

Problemy od \ref{broken_ruler_compass_hartshorne_start} do \ref{broken_ruler_compass_hartshorne_end} pochodzą z książki Hartshorne'a \cite[s. 25, 26]{hartshorne2000}.

\begin{problem}[połamana linijka]
    \label{broken_ruler_compass_hartshorne_start}
    Dane są dwa punkty $A$ i $B$ na płaszczyźnie, odległe od siebie o około trzy nible.
    Mając do dyspozycji fragment linijki o długości jednej nibli oraz sprawny cyrkiel, narysować odcinek $AB$.
\end{problem}
% Hartshorne s. 25

\begin{problem}[zardzewiały cyrkiel]
    Dane są dwa punkty $A$ i $B$ na płaszczyźnie, odległe od siebie o około pięć nibli.
    Mając do dyspozycji zardzewiały cyrkiel, którym można kreślić jedynie okręgi o promieniu dwóch nibli, skonstruować trójkąt równoboczny oparty o bok $AB$.
\end{problem}

Konstrukcje zardzewiałym cyrklem były rozpatrywane przez perskiego matematyka Abu al-Wafę Buzjaniego (940-998).
Miały praktyczne znaczenie, ponieważ stosowali je Leonardo da Vinci czy też Albrecht Dürer pod koniec piętnastego wieku.
\index[persons]{Buzjani, Abu al-Wafa}%
\index[persons]{Dürer, Albrech}%
\index[persons]{da Vinci, Leonardo}%
% ŹRÓDŁO: https://en.wikipedia.org/wiki/Poncelet–Steiner_theorem#History

\begin{problem}
    Dany jest punkt $A$ leżący na prostej $l$.
    Skonstruować prostą prostopadłą do $l$ przechodzącą przez $A$ przy użyciu linijki i zardzewiałego cyrkla.
\end{problem}
% Hartshorne s. 25

\begin{problem}
    Dany jest punkt $A$ leżący ponad cztery nible od prostej $l$.
    Skonstruować prostą prostopadłą do $l$ przechodzącą przez $A$ przy użyciu linijki i zardzewiałego cyrkla.
\end{problem}
% Hartshorne s. 25

\begin{problem}
    Dane są trzy niewspółliniowe punkty $A$, $B$ oraz $C$.
    Skonstrować punkt $D$ na prostej $AC$ tak, żeby odcinki $AD$ oraz $AB$ były równej długości przy użyciu linijki i zardzewiałego cyrkla.
\end{problem}
% Hartshorne s. 26

\begin{problem}
    Dany jest odcinek $AB$ o długości ponad dwóch nibli oraz prosta $l$, która nie przechodzi przez końce odcinka.
    Skonstrować punkt $C$ na prostej $l$ tak, żeby odcinki $AB$ oraz $AC$ były równej długości przy użyciu linijki i zardzewiałego cyrkla.
\end{problem}
% Hartshorne s. 26

\begin{problem}
    \label{broken_ruler_compass_hartshorne_end}
    Czy wszystkie konstrukcje, które można wykonać cyrklem i linijką, można wykonać też zardzewiałym cyrklem i linijką?
\end{problem}
% Hartshorne s. 26

(Odpowiedź na to pytali znali Ferrari, Cardano, Tartaglia).
% ŹRÓDŁO: https://en.wikipedia.org/wiki/Mohr–Mascheroni_theorem#Restrictions_involving_the_compass
% Retz, Merlyn; Keihn, Meta Darlene (1989), "Compass and Straightedge Constructions", Historical Topics for the Mathematics Classroom, National Council of Teachers of Mathematics (NCTM), p. 195, ISBN 9780873532815
% TODO: wpisy do indeksu

%