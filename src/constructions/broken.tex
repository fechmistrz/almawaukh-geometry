\begin{problem}[połamana linijka]
    Dane są dwa punkty $A$ i $B$ na płaszczyźnie, odległe od siebie o około trzy nible.
    Mając do dyspozycji fragment linijki o długości jednej nibli oraz sprawny cyrkiel, narysować odcinek $AB$.
\end{problem}
% Hartshorne s. 25

\begin{problem}[zardzewiały cyrkiel]
    Dane są dwa punkty $A$ i $B$ na płaszczyźnie, odległe od siebie o około pięć nibli.
    Mając do dyspozycji zardzewiały cyrkiel, którym można kreślić jedynie okręgi o promieniu dwóch nibli, skonstruować trójkąt równoboczny oparty o bok $AB$.
\end{problem}
% In the tenth century, the Persian mathematician Abu al-Wafa' Buzjani (940−998) considered geometric constructions using a straightedge and a compass with a fixed opening, a so-called rusty compass. Constructions of this type appeared to have some practical significance as they were used by artists Leonardo da Vinci and Albrecht Dürer in Europe in the late fifteenth century. 

\begin{problem}
    Dany jest punkt $A$ leżący na prostej $l$.
    Skonstruować prostą prostopadłą do $l$ przechodzącą przez $A$ przy użyciu linijki i zardzewiałego cyrkla.
\end{problem}
% Hartshorne s. 25

\begin{problem}
    Dany jest punkt $A$ leżący ponad cztery nible od prostej $l$.
    Skonstruować prostą prostopadłą do $l$ przechodzącą przez $A$ przy użyciu linijki i zardzewiałego cyrkla.
\end{problem}
% Hartshorne s. 25

\begin{problem}
    Dane są trzy niewspółliniowe punkty $A$, $B$ oraz $C$.
    Skonstrować punkt $D$ na prostej $AC$ tak, żeby odcinki $AD$ oraz $AB$ były równej długości przy użyciu linijki i zardzewiałego cyrkla.
\end{problem}
% Hartshorne s. 26

\begin{problem}
    Dany jest odcinek $AB$ o długości ponad dwóch nibli oraz prosta $l$, która nie przechodzi przez końce odcinka.
    Skonstrować punkt $C$ na prostej $l$ tak, żeby odcinki $AB$ oraz $AC$ były równej długości przy użyciu linijki i zardzewiałego cyrkla.
\end{problem}
% Hartshorne s. 26

\begin{problem}
    Czy wszystkie konstrukcje, które można wykonać cyrklem i linijką, można wykonać też zardzewiałym cyrklem i linijką?
\end{problem}
% Hartshorne s. 26