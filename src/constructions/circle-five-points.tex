Konstrukcje od \ref{delta_2024_12_start} do \ref{delta_2024_12_end} opisane są w czasopiśmie Delta, w numerze grudniowym z 2024 roku.
% TODO: opisać wszystkie siedem konstrukcji

\begin{geoconstruction}
    \label{delta_2024_12_start}
    Znając pięć punktów okręgu $\omega$, skonstruować styczną do $\omega$ w jednym z tych punktów.
\end{geoconstruction}
% Niech tymi punktami będą A, B, C, D, E. Przecinamy AB i CD w P, AC i BE w Q oraz PQ i DE w R. Wówczas prosta AR jest szukaną styczną (rys. 1). Podkreślmy, że do przeprowadzenia powyższej konstrukcji nie potrzebowaliśmy mieć narysowanego całego okręgu ω – wystarczyło tylko pięć znajdujących się na nim punktów. Uzasadnienie poprawności wymaga znajomości twierdzenia Pascala (patrz Deltoid z ∆9 14).

\begin{geoconstruction}
    Znając pięć punktów okręgu $\omega$, dla danej prostej $l$ przechodzącej przez jeden z nich wyznaczyć drugi punkt przecięcia $l$ i $\omega$.
\end{geoconstruction}

% W przypadku problemów ze znalezieniem rozwiązania polecam poszukać go
% w ∆6 17. Jesteśmy już gotowi do znalezienia środka okręgu samą linijką, jeśli
% mamy do dyspozycji jeszcze jeden, przecinający go okrąg.
% Konstrukcja 3. 
% Oznaczmy te okręgi przez ω1 i ω2, a ich punkty przecięcia przez A i B.
% Korzystając z konstrukcji 1, konstruujemy styczną do ω1 w punkcie B
% i przecinamy z ω2 w C. Przez A rysujemy prostą, która przecina ω1 w D,
% a ω2 w E. Oznaczmy przez F drugi punkt przecięcia prostej BD z ω2. Na koniec
% niech P będzie przecięciem BC i EF, a Q przecięciem BE i CF. Wówczas
% ?EFB= ?BAD= 180◦
% −?DBA−?ADB= 180◦
% −?DBA−?ABC= ?CBF.
% Oznacza to, że EB= CF, a prosta PQ zawiera średnicę ω2 (rys. 2).
% Po wybraniu innej prostej przechodzącej przez A skonstruujemy inną średnicę,
% i w konsekwencji środek ω2.
% Czytelnik z pewnością sam bez problemu wymyśli konstrukcje środka okręgu
% przy zadanych dwóch okręgach stycznych, a także przy zadanych dwóch
% okręgach współśrodkowych.
% W kolejnych konstrukcjach przyda się kilka pojęć.

\begin{geoconstruction}
    Skonstruować środek jednego z dwóch okręgów mających dwa punkty wspólne.
\end{geoconstruction}

% Rozważmy okrąg ω i dowolny punkt P nieleżący na tym okręgu. Przez punkt P
% poprowadźmy dwie sieczne, które przecinają ω w A i B oraz C i D. Niech proste
% AD i BC przecinają się w Q, a AC i BD przecinają się w R. Prostą QRbędziemy
% nazywać biegunową punktu P względem okręgu ω (rys. 3). Zauważmy, że może
% ona być wyznaczona wyłącznie przy użyciu linijki, nawet jeśli okrąg ω dany jest
% tylko w pięciu punktach (w takim przypadku korzystamy z konstrukcji 2).
% Biegunowe mają liczne i użyteczne własności. Na przykład jeśli P leży na
% zewnątrz ω,to biegunowa P przechodzi przez punkty styczności prostych stycznych
% do ω przechodzących przez P. Stąd dla punktów leżących na okręgu przyjmujemy,
% że biegunową jest styczna w tym punkcie. Zatem, wyznaczając biegunową, możemy
% skonstruować styczną do okręgu przechodzącą przez punkt na nim nieleżący.
% Inną użyteczną własnością jest fakt, że każda sieczna okręgu ω
% przechodząca przez P przecina ω w takich punktach A, B oraz biegunową
% P w takim Q, że AB dzieli harmonicznie PQ, tzn. AP
% BP = AQ
% BQ. Czytelnik
% może spróbować wymyślić, jak podzielić harmonicznie odcinek przy
% użyciu wyłącznie linijki (podpowiedź: warto przypomnieć sobie
% twierdzenia Cevy i Menelaosa).
% Kolejnym przydatnym obiektem będzie pęk okręgów. Jest to rodzina
% okręgów, którą jednoznacznie wyznaczają dwa niewspółśrodkowe okręgi.
% Pęki okręgów mają taką własność, że jeśli dwa okręgi należące do pęku
% przecinają się w dwóch punktach, to każdy okrąg z tego pęku przechodzi
% przez te dwa punkty (rys. 4), jeśli są styczne, to wszystkie są do siebie
% styczne w tym samym punkcie, oraz jeśli się nie przecinają, to żadne
% dwa się nie przecinają (rys. 5). Na potrzeby tego artykułu potraktujmy
% pęki okręgów jako „czarną skrzynkę”, zainteresowanych szczegółami
% odsyłam do krótkiego tekstu w tym wydaniu Delty (s. 20), który jest
% im poświęcony.
% Zachodzi następujące twierdzenie:
% Twierdzenie. Biegunowe dowolnego punktu P względem okręgów
% należących do jednego pęku są współpękowe.
% Punkt ten będziemy nazywali biegunowo sprzężonym do punktu P względem
% odpowiedniego pęku. Ponieważ pęk jest wyznaczony przez dwa okręgi,
% możemy też mówić o dwóch punktach sprzężonych względem pary okręgów.
% Powyższe twierdzenie wykorzystamy w kolejnych konstrukcjach. Ponieważ
% linijka nie pozwala na narysowanie okręgu, przez wyrażenie „skonstruować
% okrąg” będziemy określać wyznaczenie dowolnie wielu jego punktów.
% Konstrukcja 4. Mając dane okręgi λ i µ oraz punkt A na zewnątrz jednego
% z nich, skonstruować okrąg przechodzący przez A oraz należący do pęku
% wyznaczanego przez te okręgi.
% Niech A leży na zewnątrz okręgu λ. Z punktu A skonstruujmy styczną do λ
% w punkcie B. Następnie niech C będzie punktem biegunowo sprzężonym do
% punktu B względem λ i µ (zauważmy, że leży na AB). Konstruujemy teraz taki
% punkt D, że AD dzieli harmonicznie BC. Punkt D jest drugim obok A punktem
% szukanego okręgu (rys. 6). Gdyby okazało się, że D= C= A (tzn. gdyby AB było
% styczne do konstruowanego okręgu), to na początku konstrukcji powinniśmy wziąć
% „drugą styczną” z A do λ. Całą procedurę możemy teraz powtórzyć, biorąc D jako
% punkt startowy (i oczywiście punkt styczności do λ różny od B).

% Konstrukcja 5. Mając dane okręgi λ i µ oraz punkt A leżący wewnątrz nich,
% skonstruować okrąg przechodzący przez A oraz należący do pęku wyznaczanego
% przez te okręgi.
% W tym przypadku wyznaczamy punkt B, biegunowo sprzężony do A. Punkt ten
% leży na zewnątrz okręgów λ i µ, zatem możemy skonstruować dowolną liczbę
% punktów okręgu β przechodzącego przez B i należącego do pęku wyznaczanego
% przez te dwa okręgi (konstrukcja 4). Punkt A leży na zewnątrz β. Pokażemy, jak
% wykorzystać ten „dziurkowany” okrąg do odtworzenia konstrukcji 4.
% Problematyczny jest tylko pierwszy krok, czyli konstrukcja stycznej do β.
% Aby ją wyznaczyć, postępujemy następująco. Niech C będzie różnym od B
% punktem okręgu β. Wyznaczmy punkt D przecięcia prostej AC z okręgiem β
% (korzystamy z konstrukcji 2). Dalej konstruujemy taki E na AC, że AE dzieli
% harmonicznie CD. Prosta BE jest biegunową punktu A względem β, więc jej
% drugi punkt przecięcia z β to taki punkt F (rys. 7), że AF jest styczna do β
% (ponownie skorzystaliśmy z konstrukcji 2). Teraz na AF możemy wyznaczyć
% drugi obok A punkt szukanego okręgu i powtórzyć procedurę, rozpoczynając od
% tego punktu.
% Konstrukcja 6. Skonstruować środek przynajmniej jednego z czterech okręgów,
% z których żadne trzy nie należą do jednego pęku.
% Oznaczmy dane okręgi przez κ, λ, µ, ν. Zakładamy, że żadne dwa z nich nie
% mają punktów wspólnych ani nie są współśrodkowe.
% Wybierzmy punkt A na κ. Konstruujemy okręgi α i β przechodzące przez A oraz
% należące do pęków wyznaczonych odpowiednio przez λ i µ oraz µ i ν. Następnie
% wybieramy taki punkt B na α, że skonstruowana styczna w B do α przecina
% okrąg κ. Niech C będzie tym punktem przecięcia. Niech ponadto D i F będą
% punktami biegunowo sprzężonymi do punktów odpowiednio B i C względem
% pęku wyznaczonego przez okręgi α i β (rys. 8).
% Zauważmy, że E taki, że CE dzieli harmonicznie BD, leży na
% okręgu γ należącym do pęku wyznaczonego przez α i β oraz
% przechodzącym przez C (rozważamy ten okrąg, ale go nie
% konstruujemy). Ponadto prosta CF jest styczna do γ. Czytelnik,
% γ
% analizując ponownie rysunek 2, przekona się, że mamy wystarczająco
% danych, aby zastosować konstrukcję 3 dla okręgów γ i κ i uzyskać
% E
% średnicę κ. Drugą średnicę konstruujemy, zaczynając od innego
% punktu A.
% Odnotujmy, że konstrukcję da się powtórzyć, jeśli jeden z okręgów
% (u nas okrąg µ) jest dany tylko w pięciu punktach. Wynika to
% z możliwości wykonania konstrukcji, gdy okrąg µ jest dany tylko
% w 5 punktach, co z kolei jest konsekwencją poczynionej wcześniej
% uwagi o konstruowalności biegunowych.

\begin{geoconstruction}
    \label{delta_2024_12_end}
    Skonstruować środek przynajmniej jednego z trzech okręgów nienależących do jednego pęku.
\end{geoconstruction}
% Oznaczmy te okręgi przez κ, λ, µ. Wybieramy punkt A na κ i konstruujemy
% okrąg α należący do pęku wyznaczanego przez okręgi λ i µ oraz przechodzący
% przez A. Na okręgach κ i α wybieramy odpowiednio punkty B i C. Następnie
% prowadzimy dowolną prostą przez A i oznaczamy jej punkty przecięcia z κ
% i α przez P i Q, odpowiednio. Zauważmy, że jeśli prosta PQ będzie obracać
% się wokół punktu A, to punkt przecięcia R prostych PB i QC będzie zakreślał
% okrąg (jest to okrąg opisany na trójkącie, którego wierzchołkami są punkty B,
% C i różny od A punkt przecięcia α i κ). Oznaczmy go przez ν. Rysując zatem
% kolejne położenia prostej PQ, będziemy mogli konstruować kolejne punkty
% okręgu ν (rys. 9). Żadne trzy spośród κ, λ, µ, ν nie należą do jednego pęku.
% Stąd po skonstruowaniu pięciu punktów ν możemy powtórzyć konstrukcję 6.
% Czytelnikowi zastanawiającemu się, co z przypadkiem rozłącznych okręgów
% należących do jednego pęku, odpowiem, że wówczas środka okręgu nie da się
% skonstruować. Omówienie tego zagadnienia byłoby jednak zbyt długie, aby
% można je było zawrzeć w tym artykule.