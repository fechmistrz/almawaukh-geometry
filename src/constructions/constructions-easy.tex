% TODO: Hartshorne s. 103
\begin{problem}
    Dwusieczna kąta.
    \index{dwusieczna}
\end{problem}

Hartshorne \cite[s. 23]{hartshorne2000}. % par = 4

\begin{problem}
    Środek odcinka.
\end{problem}

Hartshorne \cite[s. 23]{hartshorne2000}. % par = 3

\begin{problem}
    Prosta prostopadła do prostej, przechodząca przez punkt.
    \index{prosta!prostopadła}
\end{problem} % par 3

Hartshorne \cite[s. 23, 24]{hartshorne2000}.

\begin{problem}
    Prosta równoległa do prostej, przechodząca przez punkt.
    \index{prosta!równoległa}
\end{problem} % par 3 

Hartshorne \cite[s. 24]{hartshorne2000}.

\begin{problem}
    Środek danego okręgu.
\end{problem} % par = 5

Hartshorne \cite[s. 24]{hartshorne2000}.
Porównaj z zadaniem Napoleona \ref{napoleon_problem}.
\index{zadanie!Napoleona}

\begin{problem}
    Styczna do okręgu przez punkt poza nim.
    \index{styczna}
\end{problem} % par = 6

Hartshorne \cite[s. 24]{hartshorne2000}.

Trójkąt wpisany 13, opisany 7. % 24/25

\begin{problem}
    Dana jest prosta, odcinek i punkt poza prostą.
    Wykreślić trójkąt równoramienny, z wierzchołkiem w punkie i z podstawą równą odcinkowi na prostej.
\end{problem} % par = 9

Hartshorne \cite[s. 25]{hartshorne2000}.

\begin{problem}
    Dana jest prosta przez punkt $B$, punkt $A$ poza prostą.
    Znaleźć okrąg, który przechodzi przez obydwa punkty, styczny do prostej.
\end{problem} % par = 8

Hartshorne \cite[s. 25]{hartshorne2000}.

\begin{problem}
    Trzy okręgi przecinające się parami pod kątem prostym
\end{problem} % par = 10

Hartshorne \cite[s. 25]{hartshorne2000}.

\begin{problem}
    Podzielić odcinek na trzy równe części.
    \index{trysekcja!odcinka}
\end{problem} % par = 6

Hartshorne \cite[s. 25]{hartshorne2000}.


% TODO: Neugebauer s. 67
\begin{problem}
    Okrąg styczny do prostej, przechodzący przez dwa punkty poza nią.
\end{problem}

Hartshorne \cite[s. 43]{hartshorne2000}.

\begin{problem}
    Okrąg styczny do dwóch prostych, przechodzący przez punkt poza nimi.
\end{problem}

Hartshorne \cite[s. 44]{hartshorne2000}.


\begin{problem}
    Prostokąt o polu takim, jak dany trójkąt.
    Dany jest jeden bok prostokąta.
\end{problem}
Hartshorne \cite[s. 43]{hartshorne2000}.

\begin{problem}
    Kwadrat o polu takim, jak dany prostokąt.
\end{problem}
Hartshorne \cite[s. 43]{hartshorne2000}.


\begin{problem}
    Podzielić trójkąt prostą przez dany punkt na dwie części o równych polach.
\end{problem}
Hartshorne \cite[s. 44]{hartshorne2000}.




\begin{problem}
    Dany jest odcinek $AB$ oraz punkt $P$ wewnątrz okręgu.
    Skonstruować cięciwę tego okręgu, która przechodzi przez punkt $P$ o długości takiej samej, jak odcinek $AB$.
    \index{cięciwa}
\end{problem} % par 5
% Hartshorne s. 26

\begin{problem}
    Dany jest odcinek $AB$, inny odcinek o długości $d$ oraz kąt $\alpha$.
    Skonstruować trójkąt $ABC$ tak, by kąt przy wierzchołku $C$ miał miarę $\alpha$, zaś suma długości ramion tego kąta była równa $d$.
\end{problem} % par ? 
% Hartshorne s. 26

\begin{problem}
    Dane są dwa okręgi takie, że żaden nie jest zawarty w drugim.
    Skonstruować styczną do obydwu okręgów.
    \index{dwustyczna}
\end{problem}
% Hartshorne s. 26

\begin{problem}
    Dany jest okrąg $\Gamma$ oraz jego środek $O$.
    Skonstruować trzy przystające okręgi, które są styczne do pozostałych dwóch oraz do $\Gamma$. \hfill \emph{(13 kroków)}. % Hartshorne s. 51
\end{problem}

\begin{problem}
    Dany jest okrąg $\Gamma$ oraz dwa punkty $A$ i $B$.
    Skonstrować punkt $C$ na okręgu $\Gamma$ tak, by odcinek łączący punkty przecięcia prostych $CA$, $CB$ z okręgiem $\Gamma$ był równoległy do odcinka $AB$.
\end{problem}
% Hartshorne s. 58-59

\begin{problem}
    Skonstruować trzy parami styczne okręgi, każdy o innym promieniu, których środki nie są współliniowe. \hfill \emph{(7 kroków)}. % Hartshorne s. 62
\end{problem}

Skonstruować oś potęgową.
\index{oś!potęgowa}
Audin \cite[s. 107]{audin_2003} podaje ten fakt w formie ćwiczenia.
