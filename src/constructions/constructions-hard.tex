%

\subsection{Okręgi Apoloniusza i Soddy'ego}
%

\todofoot{GUZICKI-19 -- wykorzystuje twierdzenie menelaosa}
% https://en.wikipedia.org/wiki/Problem_of_Apollonius
\begin{problem}
    \label{problem_apolloniusza}%
    ...
    Skonstruować ...
    \index{okrąg!Apoloniusza}
\end{problem}

Eves \cite[s. 164]{eves1_1972} w szczególnym przypadku, kiedy trzy okręgi są współpękowe

Z zadaniem wiąże się dużo późniejszy wynik.
W dwóch listach z 1643 roku do czeskiej księżniczki Elżbiety z Palatynatu Kartezjusz przedstawi uproszczoną wersję, w której trzy okręgi są parami styczne i znajdzie (bez dowodu) relację wiążącą promienie okręgów.

\begin{theorem}[Kartezjusza]
    Dane są trzy okręgi, parami styczne o promieniach $r_1, r_2, r_3$ oraz czwarty styczny do każdego z nich (istnieją dwa takie, jeden z~nich zazwyczaj styczny wewnętrznie, a w szczególnym przypadku może okazać się prostą).
    Definiujemy krzywiznę ze znakiem jako $\pm 1 / r$, gdzie znak $-$ odpowiada okręgowi stycznemu zewnętrznie.
    Wtedy
    \begin{equation}
        \left(\pm \frac{1}{r_1} \pm  \frac{1}{r_2} \pm  \frac{1}{r_3} \pm  \frac{1}{r_4}\right)^2 = 2 \left(\frac{1}{r_1^2} + \frac{1}{r_2^2} + \frac{1}{r_3^2} + \frac{1}{r_4^2}\right),
    \end{equation}
    gdzie $r_4$ oznacza promień czwartego okręgu.
\end{theorem}

Jeżeli jeden z okręgów (dajmy na to ten o promieniu $r_1$) staje się prostą, to jego zerowa krzywizna znika z równania.
Jeżeli dwa okręgi stają się dwiema prostymi, to pozostałem dwa muszą być do siebie przystające, co należy uznać za zdegenerowany przypadek.

Twierdzenie Kartezjusza odkryją na nowo Yamaji Nuchizumi (1751), Jakob Steiner (\emph{Fortsetzung der geometrischen Betrachtungen} z 1826), Philip Beecroft (\emph{Properties of circles in mutual contac} z 1842) i~wreszcie Frederick Soddy (\emph{The Kiss Precise} z 1936 jak opisane u Coxetera \cite[s. 29-31]{coxeter_1967}).
\index[persons]{Nuchizumi, Yamaji}%
\index[persons]{Steiner, Jakob}%
\index[persons]{Beecroft, Philip}%
\index[persons]{Soddy, Frederick}%
Ten ostatni poda uogólnienie do sfer, a rok później Thorold Gosset przeskoczy do przestrzeni dowolnego wymiaru ($\mathbb R^n$), z $n$ w miejsce czynnika $2$.
\index[persons]{Gosset, Thorold}%

Dowody korzystają z inwersji, wzoru Herona, wyznacznika Cayleya-Mengera, albo łańcuchów Pappusa i twierdzenia Vivianiego.
Patrz też $\Delta_{10}^6$. %. % https://www.deltami.edu.pl/media/articles/2010/06/delta-2010-06-czwarty-okrag.pdf

\begin{example}
    Dane są trzy styczne zewnętrznie okręgi o promieniach $r_1 = r_2 = r_3 = \sqrt{3}$.
    Czwarty okrąg styczny do nich wszystkich ma wtedy krzywiznę $\sqrt 3 \pm 2$.
\end{example}

Jeszcze nie skończyliśmy z Soddym.
W wierzchołkach trójkąta o obwodzie $2p = a + b + c$ wstawiamy okręgi o promieniach $p - a$, $p - b$, $p - c$.
Wtedy czwarty okrąg (zwany okręgiem Soddy'ego) z twierdzenia Kartezjusza ma promień
\begin{equation}
    \frac{4R + r \pm 2p}{S},
\end{equation}
gdzie $R, r$ to promienie okręgu opisanego i wpisanego, zaś $S$ to pole powierzchni.
Napisze o tym Coxeter \cite[s. 29-31]{coxeter_1967}

\begin{definition}[prosta Soddy'ego]
    Prostą przechodzącą przez środki okręgów Soddy'ego nazywamy prostą Soddy'ego.
\end{definition}

\begin{proposition}
    Prosta Soddy'ego przechodzi przez środek okręgu wpisanego, punkt Gergonne'a i punkt Eppsteina (punkt, gdzie przecinają się trzy proste przechodzące przez trzy pary spośród sześciu punktów styczności czterech okręgów...)
\index{punkt!Eppsteina}%
\end{proposition}

Prosta Soddy'ego ma pewien związek z punktem de Lonchampsa, którego nie opiszemy.
\index{punkt!de Longchampsa}%
% bo nie chce mi się wczytywać w szczegóły artykułu na wiki o Soddy'ego

%

\subsection{Problem Cramera-Castillona}
%

Poniższy problem będzie mieć ciekawą historię.

\index{zadanie!Cramera-Castillona}
\begin{problem}
    \label{problem_cramera_castillona}%
    Dane jest $n$ punktów oraz okrąg.
    Znaleźć taki wielokąt wpisany w okrąg, że na każdym jego boku (lub jego przedłużeniu) znajduje się dokładnie jeden z danych punktów.
\end{problem}
% TODO https://ems.press/content/serial-article-files/45288 ładny obrazek, związek z Urquhartem

Szczególny przypadek $n = 3$ współliniowych punktów zainteresuje Pappusa.
\index[persons]{Pappus}%
Pewien nieznany, ale stary geometra przedstawi problem dla dowolnych trzech punktów Gabrielowi Cramerowi, który w 1742 przekaże go dalej do Giovanniego Salveminiego\footnote{Giovanni Francesco Mauro Melchiorre Salvemini di Castiglione przyjmie w pewnym roku nazwisko od swojego miejsca urodzenia, Castiglione del Valdarno w Toskanii.}.
\index[persons]{Castiglione, Giovanni}%
\index[persons]{Cramer, Gabriel}%
Ten znajdzie geometryczne rozwiązanie już po śmierci Cramera, w~1776 roku; wyczyn powtórzą Euler (1783 rok),
Lagrange, Malfatti, Lhuilier, Servois, Poncelet
\index[persons]{Lagrange, ?}%
\index[persons]{Malfatti, ?}%
\index[persons]{Lhuilier, ?}%
\index[persons]{Servois, ?}%
\index[persons]{Poncelet, ?}%
% https://cms.math.ca/wp-content/uploads/crux-pdfs/Crux_v9n04_Apr.pdf Crux Mathematicorum, Vol 9, No 4, page 125
i szesnastoletni Annibale\footnote{Carnot uzna, że Ottajano (miejsce urodzenia Giordana) jest tytułem szlacheckim, a nie nazwą miejscowości; w swoich publikacjach nazwie młodego matematyka właśnie Ottajano. Niestety, ale później będzie tak określany także w kolejnych pracach naukowych.} Giordano.
\index[persons]{Giordano, Annibale}%
Carnot uprości analityczne rozwiązanie Lagrange'a i~uogólni je do $n \ge 3$ punktów.
\index[persons]{Carnot, ?}%

% TODO: https://en.wikipedia.org/wiki/Cramer-Castillon_problem

%

\index{zadanie!Monge'a}
\subsection{Problem Monge'a}
\begin{problem}[Monge'a?]
    Okrąg, który przecina trzy okręgi $\Gamma_1$, $\Gamma_2$, $\Gamma_3$ pod kątem prostym.
\end{problem}

% https://mathworld.wolfram.com/MongesProblem.html
Dla każdej pary okręgów $\Gamma_i$, $\Gamma_j$ znajdujemy oś potęgową; jeśli trzy osie przecinają się w punkcie $O$ leżącym na zewnątrz okręgów $\Gamma_i$, to jest to środek szukanego okręgu.
Promieniem szukanego okręgu jest odcinek styczny do $\Gamma_i$ oraz przechodzący przez $O$.
Jeśli jednak środek okręgu leży wewnątrz któregoś okręgu albo osie nie przecinają się, to problem nie ma rozwiąania.

\subsection{Zadanie Malfattiego}
\index{zadanie!Malfattiego|(}
%

W 1803 roku Malfatti \cite{malfatti_1803} zainspirowany pewnym praktycznym zagadnieniem (wycinanie walców z graniastosłupa) postawi następujący problem:
\index[persons]{Malfatti, Gian Francesco}%

\begin{problem}[zadanie Malfattiego]
	\label{malfatti_problem}
	\index{zadanie!Malfattiego}%
	Dany jest trójkąt $\triangle ABC$.
	Skonstruować takie trzy parami styczne okręgi $\Gamma_A, \Gamma_B, \Gamma_C$, że okrąg $\Gamma_A$ (odpowiednio: $\Gamma_B$, $\Gamma_C$) jest wpisany w~kąt $\angle A$ (odpowiednio: $\angle B$, $\angle C$).
\end{problem}

\begin{figure}[H]
\begin{center}
\begin{tikzpicture}[scale=.5]
\tkzDefPoints{0/0/A,10/2/B,6/7/C}
\tkzDefPoints{4.43012726/2.59439459/Oa}
\tkzDefCircle[R](Oa,1.67519375895) \tkzGetPoint{Oaa}
\tkzDrawCircle[line width=0.2mm](Oa,Oaa)

\tkzDefPoints{7.48168986/2.91734309/Ob}
\tkzDefCircle[R](Ob,1.39341015784) \tkzGetPoint{Obb}
\tkzDrawCircle[line width=0.2mm](Ob,Obb)

\tkzDefPoints{5.96721113/5.06490116/Oc}
\tkzDefCircle[R](Oc,1.23445046858) \tkzGetPoint{Occ}
\tkzDrawCircle[line width=0.2mm](Oc,Occ)

\tkzLabelPoint(A){$A$}
\tkzLabelPoint[anchor=center](Oa){$\Gamma_A$}
\tkzLabelPoint(B){$B$}
\tkzLabelPoint[anchor=center](Ob){$\Gamma_B$}
\tkzLabelPoint[above](C){$C$}
\tkzLabelPoint[anchor=center](Oc){$\Gamma_C$}
\tkzDrawPolygon[line width=0.3mm](A,B,C)
\end{tikzpicture}
\end{center}
\caption{Trzy okręgi Malfattiego}
\end{figure}

Problem będzie rozważany na długo przed Malfattim, zajmie się nim Ajima Naonobu\footnote{Matematyk japoński, przypisze się mu wprowadzenie rachunku różniczkowo-całkowego do matematyki japońskiej.} w~XVIII wieku, a~jeszcze wcześniej Gilio de Cecco da Montepulciano w~rękopisie z~1384 roku.
\index[persons]{Ajima, Naonobu}%
\index[persons]{de Cecco da Montepulciano, Gilio}%

Malfatti wyprowadzi co następuje.
Niech $p$ będzie połową obwodu trójkąta, $r$ będzie promieniem okręgu wpisanego w~ten trójkąt zaś $d_A$, $d_B$, $d_C$ odległościami wierzchołków $A, B, C$ od środka tego okręgu.
Wtedy promienie okręgów Malfattiego wyrażają się wzorami
\begin{align}
	r_A & = \frac r 2 \cdot {\frac {s-r+d_A-d_B-d_C}{p-a}}, \\
	r_B & = \frac r 2 \cdot {\frac {s-r+d_B-d_A-d_C}{p-b}}, \\
	r_C & = \frac r 2 \cdot {\frac {s-r+d_C-d_A-d_B}{p-c}}.
\end{align}

Prostą konstrukcję okręgów opartą na dwustycznych zawdzięczymy Steinerowi \cite{steiner_1826} w~1826 roku;
\index[persons]{Steiner, Jakob}%
inne rozwiązania podadzą Lehmus \cite{lehmus_1819}, Catalan \cite{catalan_1846}, Adams \cite{adams_1846}, Derousseau \cite{derousseau_1895}, Pampuch \cite{pampuch_1904}.
% TODO: po poprawie bibliografii, podać tu index persons

(O~problemie napiszą też Bogdańska, Neugebauer \cite[s. 102]{neugebauer_2018}).

Malfatti postawi tak naprawdę inny problem: znalezienia trzech rozłącznych kół zawartych w~trójkącie, których suma pól jest maksymalna i~błędnie założy, że opisane wyżej okręgi stanowią rozwiązanie.
Pomyłkę zauważą najpierw bez dowodu Lob, Richmond \cite{lob_richmond_1930} w~1930 roku: z trójkąta równobocznego można wyciąć zachłannie kolejno trzy koła, ich łączna powierzchnia jest większa od powierzchni kół znalezionych przez Malfattiego o 1\%.
\index[persons]{Richmond, ?}%
\index[persons]{Lob, ?}%
Howard Eves powtórzy to dla stromych trójkątów równoramiennych o bardzo wąskiej podstawie i dużej wysokości około 1946 roku.
\index[persons]{Eves, Howard}%
% https://en.wikipedia.org/w/index.php?title=Howard_Eves&diff=831382284&oldid=750910758
Goldberg \cite{goldberg_1967} wykaże, że domniemanie Malfattiego nie daje nigdy kół o maksymalnej łącznej powierzchni.
Ostatnie słowo należy zaś do Zalgallera, Losa \cite{zalgaller_los_1992}, którzy znajdą trzy koła rozwiązujące problem Malfattiego w dowolnym trójkącie.
% TODO: Goldberg M., On the original Malfatti problem, Math. Mag. 40 (1967), 241–247.
\index[persons]{Zalgaller, VA?}%
\index[persons]{Los, GA?}%
% TODO: Zalgaller V.A., Los’ G.A., Solution of the Malfatti problem, Ukrain. Geom. Sb. 35 (1992), 14–33 (ang. J. Math. Sci. 72 (1994), 3163–3177).
% TODO: po poprawie bibliografii, podać tu index persons
% TODO: Lob, H.; Richmond, H. W. (1930), "On the Solutions of Malfatti's Problem for a Triangle", Proceedings of the London Mathematical Society, 2nd ser., 30 (1): 287–304, doi:10.1112/plms/s2-30.1.287.

Kryształowa kula nie potrafi przewidzieć, kto oceni, czy algorytm zachłanny zawsze znajduje $n \ge 4$ rozłącznych kół w trójkącie o maksymalnej łącznej powierzchni.

(O więcej niż jednym okręgu wpisanym w trójkąt pisaliśmy w podpodsekcji \ref{sssection_6_7_9_circles}).

\index{zadanie!Malfattiego|)}

\subsection{Punkty Crelle'a-Brocarda}
\index{punkt!Crelle'a-Brocarda}
\todofoot{pl-wiki ,,Punkty Brocarda''}
% TODO: ttps://pl.wikipedia.org/wiki/Punkty_Brocarda

\subsection{Lista Wernicksa}
\emph{Sixteen key points of a triangle are its vertices, the midpoints of its sides, the feet of its altitudes, the feet of its internal angle bisectors, and its circumcenter, centroid, orthocenter, and incenter. These can be taken three at a time to yield 139 distinct nontrivial problems of constructing a triangle from three points.[12] Of these problems, three involve a point that can be uniquely constructed from the other two points; 23 can be non-uniquely constructed (in fact for infinitely many solutions) but only if the locations of the points obey certain constraints; in 74 the problem is constructible in the general case; and in 39 the required triangle exists but is not constructible.}
%  => https://en.wikipedia.org/wiki/Straightedge_and_compass_construction

% TODO: https://en.wikipedia.org/wiki/Malfatti_circles

%