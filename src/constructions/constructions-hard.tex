\subsection{Okręgi Apoloniusza}
\todofoot{GUZICKI-19 -- wykorzystuje twierdzenie menelaosa}
% https://en.wikipedia.org/wiki/Problem_of_Apollonius
\begin{problem}
    \label{problem_apolloniusza}%
    ...
    Skonstruować ...
    \index{okrąg!Apoloniusza}
\end{problem}

\subsection{Problem Cramera-Castillona}
\index{zadanie!Cramera-Castillona}
Wpisać w dany okrąg trójkąt, którego boki przechodzą przez trzy dane punkty.
Ten problem, postawiony przez szwajcarskiego matematyka Cramera, nosi nazwę włoskiego matematyka Castillona, który rozwiązał go w 1776 roku.
(Gabriel Cramer, 1704–1752, opublikował w 1750 roku swoje główne dzieło Introduction à l’analyse des lignes courbes algébriques, w którym po raz pierwszy rozwiązano układ równań liniowych za pomocą wyznaczników.
I. F. Salvemini, 1709–1791, przyjął nazwisko Castillon od swojego miejsca urodzenia – Castiglione w Toskanii.)
Poniższe proste, choć niełatwe do zauważenia rozwiązanie problemu Castillona pochodzi od Włocha Giordano.
% TODO: https://en.wikipedia.org/wiki/Cramer–Castillon_problem
\todofoot{https://en.wikipedia.org/wiki/Cramer–Castillon\_problem 1776}

\index{zadanie!Monge'a}
\subsection{Problem Monge'a}
\begin{problem}[Monge'a?]
    Okrąg, który przecina trzy okręgi $\Gamma_1$, $\Gamma_2$, $\Gamma_3$ pod kątem prostym.
\end{problem}

% https://mathworld.wolfram.com/MongesProblem.html
Dla każdej pary okręgów $\Gamma_i$, $\Gamma_j$ znajdujemy oś potęgową; jeśli trzy osie przecinają się w punkcie $O$ leżącym na zewnątrz okręgów $\Gamma_i$, to jest to środek szukanego okręgu.
Promieniem szukanego okręgu jest odcinek styczny do $\Gamma_i$ oraz przechodzący przez $O$.
Jeśli jednak środek okręgu leży wewnątrz któregoś okręgu albo osie nie przecinają się, to problem nie ma rozwiąania.

\subsection{Zadanie Malfattiego}
\index{zadanie!Malfattiego|(}
%

W 1803 roku Malfatti \cite{malfatti_1803} zainspirowany pewnym praktycznym zagadnieniem (wycinanie walców z graniastosłupa) postawi następujący problem:
\index[persons]{Malfatti, Gian Francesco}%

\begin{problem}[zadanie Malfattiego]
	\label{malfatti_problem}
	\index{zadanie!Malfattiego}%
	Dany jest trójkąt $\triangle ABC$.
	Skonstruować takie trzy parami styczne okręgi $\Gamma_A, \Gamma_B, \Gamma_C$, że okrąg $\Gamma_A$ (odpowiednio: $\Gamma_B$, $\Gamma_C$) jest wpisany w~kąt $\angle A$ (odpowiednio: $\angle B$, $\angle C$).
\end{problem}

% https://www.desmos.com/calculator/mqzextwkad?lang=pl
\begin{figure}[H] \centering
\begin{comment}
\begin{tikzpicture}[scale=.5]
\tkzDefPoints{0/0/A,10/2/B,6/7/C}
\tkzDefPoints{4.43012726/2.59439459/Oa}
\tkzDefCircle[R](Oa,1.67519375895) \tkzGetPoint{Oaa}
\tkzDrawCircle[line width=0.2mm](Oa,Oaa)

\tkzDefPoints{7.48168986/2.91734309/Ob}
\tkzDefCircle[R](Ob,1.39341015784) \tkzGetPoint{Obb}
\tkzDrawCircle[line width=0.2mm](Ob,Obb)

\tkzDefPoints{5.96721113/5.06490116/Oc}
\tkzDefCircle[R](Oc,1.23445046858) \tkzGetPoint{Occ}
\tkzDrawCircle[line width=0.2mm](Oc,Occ)

\tkzLabelPoint(A){$A$}
\tkzLabelPoint[anchor=center](Oa){$\Gamma_A$}
\tkzLabelPoint(B){$B$}
\tkzLabelPoint[anchor=center](Ob){$\Gamma_B$}
\tkzLabelPoint[above](C){$C$}
\tkzLabelPoint[anchor=center](Oc){$\Gamma_C$}
\tkzDrawPolygon[line width=0.3mm](A,B,C)
\end{tikzpicture}
\end{comment}
\caption{Trzy okręgi Malfattiego}
\end{figure}

Problem będzie rozważany na długo przed Malfattim, zajmie się nim Ajima Naonobu\footnote{Matematyk japoński, przypisze się mu wprowadzenie rachunku różniczkowo-całkowego do matematyki japońskiej.} w~XVIII wieku, a~jeszcze wcześniej Gilio de Cecco da Montepulciano w~rękopisie z~1384 roku.
\index[persons]{Ajima, Naonobu}%
\index[persons]{de Cecco da Montepulciano, Gilio}%

Malfatti wyprowadzi co następuje.
Niech $p$ będzie połową obwodu trójkąta, $r$ będzie promieniem okręgu wpisanego w~ten trójkąt zaś $d_A$, $d_B$, $d_C$ odległościami wierzchołków $A, B, C$ od środka tego okręgu.
Wtedy promienie okręgów Malfattiego wyrażają się wzorami
\begin{align}
	r_A & = \frac r 2 \cdot {\frac {s-r+d_A-d_B-d_C}{p-a}}, \\
	r_B & = \frac r 2 \cdot {\frac {s-r+d_B-d_A-d_C}{p-b}}, \\
	r_C & = \frac r 2 \cdot {\frac {s-r+d_C-d_A-d_B}{p-c}}.
\end{align}

Prostą konstrukcję okręgów opartą na dwustycznych zawdzięczymy Steinerowi \cite{steiner_1826} w~1826 roku;
\index[persons]{Steiner, Jakob}%
inne rozwiązania podadzą Lehmus \cite{lehmus_1819}, Catalan \cite{catalan_1846}, Adams \cite{adams_1846}, Derousseau \cite{derousseau_1895}, Pampuch \cite{pampuch_1904}.
% TODO: po poprawie bibliografii, podać tu index persons

(O~problemie napiszą też Bogdańska, Neugebauer \cite[s. 102]{neugebauer_2018}).

Malfatti postawi tak naprawdę inny problem: znalezienia trzech rozłącznych kół zawartych w~trójkącie, których suma pól jest maksymalna i~błędnie założy, że opisane wyżej okręgi stanowią rozwiązanie.
Pomyłkę zauważą najpierw bez dowodu Lob, Richmond \cite{lob_richmond_1930} w~1930 roku: z trójkąta równobocznego można wyciąć zachłannie kolejno trzy koła, ich łączna powierzchnia jest większa od powierzchni kół znalezionych przez Malfattiego o 1\%.
\index[persons]{Richmond, ?}%
\index[persons]{Lob, ?}%
Howard Eves powtórzy to dla stromych trójkątów równoramiennych o bardzo wąskiej podstawie i dużej wysokości około 1946 roku.
\index[persons]{Eves, Howard}%
% https://en.wikipedia.org/w/index.php?title=Howard_Eves&diff=831382284&oldid=750910758
Goldberg \cite{goldberg_1967} wykaże, że domniemanie Malfattiego nie daje nigdy kół o maksymalnej łącznej powierzchni.
Ostatnie słowo należy zaś do Zalgallera, Losa \cite{zalgaller_los_1992}, którzy znajdą trzy koła rozwiązujące problem Malfattiego w dowolnym trójkącie.
% TODO: Goldberg M., On the original Malfatti problem, Math. Mag. 40 (1967), 241-247.
\index[persons]{Zalgaller, VA?}%
\index[persons]{Los, GA?}%
% TODO: Zalgaller V.A., Los’ G.A., Solution of the Malfatti problem, Ukrain. Geom. Sb. 35 (1992), 14-33 (ang. J. Math. Sci. 72 (1994), 3163-3177).
% TODO: po poprawie bibliografii, podać tu index persons
% TODO: Lob, H.; Richmond, H. W. (1930), "On the Solutions of Malfatti's Problem for a Triangle", Proceedings of the London Mathematical Society, 2nd ser., 30 (1): 287-304, doi:10.1112/plms/s2-30.1.287.

Kryształowa kula nie potrafi przewidzieć, kto oceni, czy algorytm zachłanny zawsze znajduje $n \ge 4$ rozłącznych kół w trójkącie o maksymalnej łącznej powierzchni.

(O więcej niż jednym okręgu wpisanym w trójkąt pisaliśmy w podpodsekcji \ref{sssection_6_7_9_circles}).

\index{zadanie!Malfattiego|)}

\subsection{Punkty Crelle'a-Brocarda}
\index{punkt!Crelle'a-Brocarda}
\todofoot{pl-wiki ,,Punkty Brocarda''}
% TODO: ttps://pl.wikipedia.org/wiki/Punkty_Brocarda

% TODO: https://en.wikipedia.org/wiki/Malfatti_circles