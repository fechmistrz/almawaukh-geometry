%

% TODO: https://en.wikipedia.org/wiki/Straightedge_and_compass_construction
% https://en.wikipedia.org/wiki/Compass_equivalence_theorem
% https://en.wikipedia.org/wiki/4,294,967,295
\section{Konstrukcje klasyczne}

% The compass equivalence theorem proves that the rigid compass (also called the modern compass) - one that holds its spacing when lifted from the plane - is equivalent to the traditional collapsing compass (also called divider) - one that does not retain its spacing, thus "resetting to zero", every time it is lifted from the plane. The ability to transfer distances (i.e. construct congruent circles, translate a circle in the plane) - an operation made trivial by the fixable aperture of a rigid compass - was proven by Euclid to be possible with the collapsing compass. Consequently, the rigid compass and the collapsing compass are equivalent; what can be constructed by one can be constructed by the other. => https://en.wikipedia.org/wiki/Poncelet%E2%80%93Steiner_theorem#Other_types_of_restricted_construction

\todofoot{Klasycznie zakłada się, że cyrkiel nie może przenosić odcinków, bo się rozpada po podniesieniu: However, by the compass equivalence theorem in Proposition 2 of Book 1 of Euclid's Elements, no power is lost by using a collapsing compass.} % The idealized ruler, known as a straightedge, is assumed to be infinite in length, have only one edge, and no markings on it. The compass is assumed to have no maximum or minimum radius, and is assumed to "collapse" when lifted from the page, so it may not be directly used to transfer distances. (This is an unimportant restriction since, using a multi-step procedure, a distance can be transferred even with a collapsing compass; see compass equivalence theorem. Note however that whilst a non-collapsing compass held against a straightedge might seem to be equivalent to marking it, the neusis construction is still impermissible and this is what unmarked really means: see Markable rulers below.) More formally, the only permissible constructions are those granted by the first three postulates of Euclid's Elements.
%  

\todofoot{Hipokrates podwoił sześcian: Hippocrates and Menaechmus showed that the volume of the cube could be doubled by finding the intersections of hyperbolas and parabolas, but these cannot be constructed by straightedge and compass.[2]: p. 30  In the fifth century BCE, Hippias used a curve that he called a quadratrix to both trisect the general angle and square the circle, and Nicomedes in the second century BCE showed how to use a conchoid to trisect an arbitrary angle;[2]: p. 37  but these methods also cannot be followed with just straightedge and compass.}



\subsection{Proste}
% TODO: Hartshorne s. 103
\begin{problem}
    Dwusieczna kąta.
\end{problem}

\begin{problem}
    Środek odcinka.
\end{problem}

\begin{problem}
    Prosta prostopadła do prostej, przechodząca przez punkt.
\end{problem}

\begin{problem}
    Prosta równoległa do prostej, przechodząca przez punkt.
\end{problem}

% TODO: Neugebauer s. 67
\begin{problem}
    Okrąg styczny do prostej, przechodzący przez dwa punkty.
\end{problem}



\subsection{Trudniejsze}

\begin{problem}[Monge'a?]
    Okrąg, który przecina trzy okręgi $\Gamma_1$, $\Gamma_2$, $\Gamma_3$ pod kątem prostym.
\end{problem}

% https://mathworld.wolfram.com/MongesProblem.html
Dla każdej pary okręgów $\Gamma_i$, $\Gamma_j$ znajdujemy oś potęgową; jeśli trzy osie przecinają się w punkcie $O$ leżącym na zewnątrz okręgów $\Gamma_i$, to jest to środek szukanego okręgu.
Promieniem szukanego okręgu jest odcinek styczny do $\Gamma_i$ oraz przechodzący przez $O$.
Jeśli jednak środek okręgu leży wewnątrz któregoś okręgu albo osie nie przecinają się, to problem nie ma rozwiąania.

\begin{problem}
    Dany jest odcinek $AB$ oraz punkt $P$ wewnątrz okręgu.
    Skonstruować cięciwę tego okręgu, która przechodzi przez punkt $P$ o długości takiej samej, jak odcinek $AB$.
\end{problem}
% Hartshorne s. 26

\begin{problem}
    Dany jest odcinek $AB$, inny odcinek o długości $d$ oraz kąt $\alpha$.
    Skonstruować trójkąt $ABC$ tak, by kąt przy wierzchołku $C$ miał miarę $\alpha$, zaś suma długości ramion tego kąta była równa $d$.
\end{problem}
% Hartshorne s. 26

\begin{problem}
    Dane są dwa okręgi takie, że żaden nie jest zawarty w drugim.
    Skonstruować styczną do obydwu okręgów.
\end{problem}
% Hartshorne s. 26

\begin{problem}
    Dany jest okrąg $\Gamma$ oraz jego środek $O$.
    Skonstruować trzy przystające okręgi, które są styczne do pozostałych dwóch oraz do $\Gamma$. \hfill \emph{(13 kroków)}. % Hartshorne s. 51
\end{problem}
% Hartshorne s. 26

\begin{problem}
    Dany jest okrąg $\Gamma$ oraz dwa punkty $A$ i $B$.
    Skonstrować punkt $C$ na okręgu $\Gamma$ tak, by odcinek łączący punkty przecięcia prostych $CA$, $CB$ z okręgiem $\Gamma$ był równoległy do odcinka $AB$.
\end{problem}
% Hartshorne s. 58-59

\begin{problem}
    Skonstruować trzy parami styczne okręgi, każdy o innym promieniu, których środki nie są współliniowe. \hfill \emph{(7 kroków)}. % Hartshorne s. 62
\end{problem}

\subsection{Wyrocznia w Delfach}
\begin{problem}[problem delijski]
    Zbudować sześcian o objętości dwa razy większej niż sześcian dany.
    \index{problem!delijski}
\end{problem}
\todofoot{Theon Smyrny} % https://en.wikipedia.org/wiki/Theon_of_Smyrna  It is also one of the sources of our knowledge of the origins of the classical problem of Doubling the cube.[5]

Według legendy wyrocznia na wyspie Delos poradziła, by dla powstrzymania panującej tam zarazy dwukrotnie powiększyć ołtarz ofiarny, nie zmieniając jego kształtu.
\index{podwojenie sześcianu}
Problem ten zajmował matematyków przez wiele stuleci; w efekcie istnieją konstrukcje przybliżone, a także konstrukcje wykorzystujące dodatkowo pewne krzywe, np. konstrukcja Dioklesa wykorzystująca cysoidę, konstrukcja Nikomedesa wykorzystująca konchoidę.
\index[persons]{Dikoles}
\index{cysoida}
\index[persons]{Nikomedes}
\index{konchoida}
% Hippocrates and Menaechmus showed that the volume of the cube could be doubled by finding the intersections of hyperbolas and parabolas, but these cannot be constructed by straightedge and compass. => https://en.wikipedia.org/wiki/Straightedge_and_compass_construction
\todofoot{Hipokrates podwoił sześcian: Hippocrates and Menaechmus showed that the volume of the cube could be doubled by finding the intersections of hyperbolas and parabolas, but these cannot be constructed by straightedge and compass.[2]: p. 30  In the fifth century BCE, Hippias used a curve that he called a quadratrix to both trisect the general angle and square the circle, and Nicomedes in the second century BCE showed how to use a conchoid to trisect an arbitrary angle;[2]: p. 37  but these methods also cannot be followed with just straightedge and compass.}

\begin{problem}[trysekcja kąta]
    Podzielić dowolny kąt na trzy równe części.
    \index{trysekcja!kąta}
\end{problem}
% TODO: % TODO: Coxeter, s. 28
% TODO: https://en.wikipedia.org/wiki/Square_trisection

Około V wieku próbowano rozwiązać to zagadnienie za pomocą innych środków konstrukcyjnych, np. wykorzystując kwadratrysę, konchoidę; Archimedes np. zaproponował następujący sposób trysekcji kąta za pomocą cyrkla i linijki z podziałką.
\index[persons]{Archimedes}
% promieniem BA (równym np. jedności) zakreśla się okrąg (otrzymując punkty A i C), przedłuża średnicę AD poza okrąg i tak się umieszcza linijkę z zaznaczonym na niej odcinkiem EF = AB, by przechodziła ona przez punkt C oraz by zaznaczony na niej punkt F leżał na okręgu, punkt E zaś na przedłużeniu średnicy AD; wówczas DEF = 1/3ABC.
W średniowieczu stwierdzono, że zagadnienie trysekcji kąta prowadzi do rozwiązania równania algebraicznego stopnia trzeciego postaci
\begin{equation}
    x^3 - 3 x - 2 \cos 3\alpha = 0,
\end{equation}
którego pierwiastek konstruuje się następnie za pomocą rozmaitych środków konstrukcyjnych.

\begin{proposition}
    Rozwiązanie problemu delijskiego oraz trysekcja kąta są niewykonalne za pomocą cyrkla i linijki.
\end{proposition}
% https://en.wikipedia.org/wiki/Menaechmus

Pierwszy dowód tego faktu podał P. L. Wantzel (1837).
Być może było to powodem, dla którego dopiero w 1899 roku Frank Morley odkrył (a w 1924 opublikował):
\index[persons]{Morley, Frank}

\begin{theorem}[Morleya]
    Punkty przeciecięcia tych trójsiecznych kątów trójkąta, które sąsiadują z którymś z boków trójkąta, są wierzchołkami trójkąta równobocznego.
    \index{trójkąt!Morleya}
    \index{twierdzenie!Morleya}
\end{theorem}

(W tym sformułowaniu nie ma mowy, że chodzi o kąty wewnętrzne -- ponieważ twierdzenie jest prawdziwe także dla kątów zewnętrznych, z prawie takim samym dowodem).
Trójkąt równoboczny, jaki powstaje, ma bok długości
\begin{equation}
    8 R \sin \frac \alpha 3 \sin \frac \beta 3 \sin \frac \gamma 3.
\end{equation}
% Coxeter s. 23-25
% https://en.wikipedia.org/wiki/Hofstadter_points

\begin{problem}[kwadratura koła]
    Wykreślić kwadrat o polu równym polu danego koła.
    \index{kwadratura koła}
\end{problem}
% https://en.wikipedia.org/wiki/Adam_Adamandy_Kocha%C5%84ski
% TODO: https://en.wikipedia.org/wiki/Lune_of_Hippocrates
% Without the constraint of requiring solution by ruler and compass alone, the problem is easily solvable by a wide variety of geometric and algebraic means, and was solved many times in antiquity. => https://en.wikipedia.org/wiki/Straightedge_and_compass_construction
% https://en.wikipedia.org/wiki/Wallace–Bolyai–Gerwien_theorem 
% ciekawe, bo dla 3d nie działa, bo mają dziwną teorię pola, bo nie mają R

1833 F. Lindemann udowodnił, że kwadratura koła jest niewykonalna za pomocą linijki i cyrkla.

% Sixteen key points of a triangle are its vertices, the midpoints of its sides, the feet of its altitudes, the feet of its internal angle bisectors, and its circumcenter, centroid, orthocenter, and incenter. These can be taken three at a time to yield 139 distinct nontrivial problems of constructing a triangle from three points.[12] Of these problems, three involve a point that can be uniquely constructed from the other two points; 23 can be non-uniquely constructed (in fact for infinitely many solutions) but only if the locations of the points obey certain constraints; in 74 the problem is constructible in the general case; and in 39 the required triangle exists but is not constructible. => https://en.wikipedia.org/wiki/Straightedge_and_compass_construction

\todofoot{Trysekcja kąta angle trisection} % https://en.wikipedia.org/wiki/Angle_trisection

% Konstruowalna => stopień Q(x) nad Q to potęga 2, ale nie w drugą stronę.
% Podwojenie sześcianu. % https://en.wikipedia.org/wiki/Pandrosion
% Trysekcja kąta.

\subsection{Wielokąty foremne}

\subsection{Wielokąty foremne}
Euklides potrafił skonstruować wielokąt foremny o $n$ bokach dla $n = 3$ (I.1), $n = 4$ (I.46), $n = 5$ (IV.11), $n = 6$ (IV.15) oraz $n = 15$ (IV.16).
Ptolemeusz podał konstrukcję pięciokąta foremnego w Almageście.

Mając wielokąt o $n$ bokach, bisekcja kąta środkowego pozwala podwoić liczbę boków do $2n$.
Możemy więc uznać, że Euklides potrafił wykreślić wielokąty, które miały
\begin{equation}
    3, 4, 5, 6, 8, 10, 12, 15, 16, 20, \ldots
\end{equation}
boków.
Lista została rozszerzona dopiero w 1796 roku przez Gaussa: pokazał, że siedemnastokąt foremny jest konstruowalny.
Był tak zadowolony z tego wyniku, że zażyczył sobie wyryć właśnie tę figurę na swoim grobie.
Faktycznej konstrukcji dokonał Johannes Erchinger w 1800 roku.
\index[persons]{Erchinger, Johannes}%
Hartstone \cite[s. 250-259]{hartshorne2000} poświęca tej konstrukcji całą sekcję 29.

W 1832 roku Friederich Richelot skonstruował $257$-kąt.
\index[persons]{Richelot, Friederich}
Jego cierpliwość przyćmiewa konstrukcja $65\,537$-kąta Johanna Hermesa z 1896 roku, na którą poświęcił dziesięć lat swojego życia (i dwieście stron rękopisu).
% The construction is very complex; Hermes spent 10 years completing the 200-page manuscript

W ogólności, mamy:
\begin{theorem}[Gaussa-Wantzla]
    Wielokąt foremny o $n$ bokach jest konstruowalny przy pomocy cyrkla i linijki wtedy i tylko wtedy, kiedy $n$ jest postaci
    \begin{equation}
        n = 2^r p_1 \cdot \ldots \cdot p_s,
    \end{equation}
    $r, s \ge 0$, gdzie $p_i$ są różnymi liczbami pierwszymi postaci $2^{2^k} + 1$, na przykład: $3$, $5$, $17$, $257$, $65537$
\end{theorem}

Uzasadnienie można znaleźć u Hartstone'a \cite[s. 258]{hartshorne2000}.
Gauss znalazł warunek wystarczający i napisał bez dowodu w \emph{Disquisitiones Arithmeticae} (1801), że jest też konieczny.
Brakujący dowód został dodany przez Pierre'a Wantzla w 1837 roku.

Carl Friedrich Gauss proved the constructibility of the regular 17-gon in 1796. Five years later, he developed the theory of Gaussian periods in his . This theory allowed him to formulate a sufficient condition for the constructibility of regular polygons. Gauss stated without proof that this condition was also necessary,[2] but never published his proof.

% TODO: Coxeter, s. 26




GUZICKI-12 **wielokąty foremne** które są konstruowalne? (tw. wantzla itd.) konstrukcje przybliżone pięciokąta - durer i da vinci.
$n = 3$, $n = 4$, $n = 6$ (proste)

\begin{problem}
    Skonstrować trójkąt równoboczny wpisany w okrąg, którego środek nie jest znany. \hfill \emph{(7 kroków)}
\end{problem}

\begin{problem}
    Skonstrować kwadrat. \hfill \emph{(9 kroków)}
\end{problem}

\begin{problem}
    Skonstrować pięciokąt foremny.
\end{problem}

Piszą o tym Hartshorne \cite[s. 45-49]{hartshorne2000}.
Jeśli mamy zadany jeden z jego boków, konstrukcja wymaga 11 kroków. % Hartshorne s. 51




$n = 17$

$n = 7$ (niemożliwe), możliwe ze znaczoną linijką: Hartshorne rozdział 30/31


\subsection{Zadanie Malfattiego}
%

W 1803 roku Malfatti \cite{malfatti_1803} zainspirowany pewnym praktycznym zagadnieniem (wycinanie walców z graniastosłupa) postawi następujący problem:
\index[persons]{Malfatti, Gian Francesco}%

\begin{problem}[zadanie Malfattiego]
	\label{malfatti_problem}
	\index{zadanie!Malfattiego}%
	Dany jest trójkąt $\triangle ABC$.
	Skonstruować takie trzy parami styczne okręgi $\Gamma_A, \Gamma_B, \Gamma_C$, że okrąg $\Gamma_A$ (odpowiednio: $\Gamma_B$, $\Gamma_C$) jest wpisany w~kąt $\angle A$ (odpowiednio: $\angle B$, $\angle C$).
\end{problem}

% https://www.desmos.com/calculator/mqzextwkad?lang=pl
\begin{figure}[H] \centering
\begin{comment}
\begin{tikzpicture}[scale=.5]
\tkzDefPoints{0/0/A,10/2/B,6/7/C}
\tkzDefPoints{4.43012726/2.59439459/Oa}
\tkzDefCircle[R](Oa,1.67519375895) \tkzGetPoint{Oaa}
\tkzDrawCircle[line width=0.2mm](Oa,Oaa)

\tkzDefPoints{7.48168986/2.91734309/Ob}
\tkzDefCircle[R](Ob,1.39341015784) \tkzGetPoint{Obb}
\tkzDrawCircle[line width=0.2mm](Ob,Obb)

\tkzDefPoints{5.96721113/5.06490116/Oc}
\tkzDefCircle[R](Oc,1.23445046858) \tkzGetPoint{Occ}
\tkzDrawCircle[line width=0.2mm](Oc,Occ)

\tkzLabelPoint(A){$A$}
\tkzLabelPoint[anchor=center](Oa){$\Gamma_A$}
\tkzLabelPoint(B){$B$}
\tkzLabelPoint[anchor=center](Ob){$\Gamma_B$}
\tkzLabelPoint[above](C){$C$}
\tkzLabelPoint[anchor=center](Oc){$\Gamma_C$}
\tkzDrawPolygon[line width=0.3mm](A,B,C)
\end{tikzpicture}
\end{comment}
\caption{Trzy okręgi Malfattiego}
\end{figure}

Problem będzie rozważany na długo przed Malfattim, zajmie się nim Ajima Naonobu\footnote{Matematyk japoński, przypisze się mu wprowadzenie rachunku różniczkowo-całkowego do matematyki japońskiej.} w~XVIII wieku, a~jeszcze wcześniej Gilio de Cecco da Montepulciano w~rękopisie z~1384 roku.
\index[persons]{Ajima, Naonobu}%
\index[persons]{de Cecco da Montepulciano, Gilio}%

Malfatti wyprowadzi co następuje.
Niech $p$ będzie połową obwodu trójkąta, $r$ będzie promieniem okręgu wpisanego w~ten trójkąt zaś $d_A$, $d_B$, $d_C$ odległościami wierzchołków $A, B, C$ od środka tego okręgu.
Wtedy promienie okręgów Malfattiego wyrażają się wzorami
\begin{align}
	r_A & = \frac r 2 \cdot {\frac {s-r+d_A-d_B-d_C}{p-a}}, \\
	r_B & = \frac r 2 \cdot {\frac {s-r+d_B-d_A-d_C}{p-b}}, \\
	r_C & = \frac r 2 \cdot {\frac {s-r+d_C-d_A-d_B}{p-c}}.
\end{align}

Prostą konstrukcję okręgów opartą na dwustycznych zawdzięczymy Steinerowi \cite{steiner_1826} w~1826 roku;
\index[persons]{Steiner, Jakob}%
inne rozwiązania podadzą Lehmus \cite{lehmus_1819}, Catalan \cite{catalan_1846}, Adams \cite{adams_1846}, Derousseau \cite{derousseau_1895}, Pampuch \cite{pampuch_1904}.
% TODO: po poprawie bibliografii, podać tu index persons

(O~problemie napiszą też Bogdańska, Neugebauer \cite[s. 102]{neugebauer_2018}).

Malfatti postawi tak naprawdę inny problem: znalezienia trzech rozłącznych kół zawartych w~trójkącie, których suma pól jest maksymalna i~błędnie założy, że opisane wyżej okręgi stanowią rozwiązanie.
Pomyłkę zauważą najpierw bez dowodu Lob, Richmond \cite{lob_richmond_1930} w~1930 roku: z trójkąta równobocznego można wyciąć zachłannie kolejno trzy koła, ich łączna powierzchnia jest większa od powierzchni kół znalezionych przez Malfattiego o 1\%.
\index[persons]{Richmond, ?}%
\index[persons]{Lob, ?}%
Howard Eves powtórzy to dla stromych trójkątów równoramiennych o bardzo wąskiej podstawie i dużej wysokości około 1946 roku.
\index[persons]{Eves, Howard}%
% https://en.wikipedia.org/w/index.php?title=Howard_Eves&diff=831382284&oldid=750910758
Goldberg \cite{goldberg_1967} wykaże, że domniemanie Malfattiego nie daje nigdy kół o maksymalnej łącznej powierzchni.
Ostatnie słowo należy zaś do Zalgallera, Losa \cite{zalgaller_los_1992}, którzy znajdą trzy koła rozwiązujące problem Malfattiego w dowolnym trójkącie.
% TODO: Goldberg M., On the original Malfatti problem, Math. Mag. 40 (1967), 241-247.
\index[persons]{Zalgaller, VA?}%
\index[persons]{Los, GA?}%
% TODO: Zalgaller V.A., Los’ G.A., Solution of the Malfatti problem, Ukrain. Geom. Sb. 35 (1992), 14-33 (ang. J. Math. Sci. 72 (1994), 3163-3177).
% TODO: po poprawie bibliografii, podać tu index persons
% TODO: Lob, H.; Richmond, H. W. (1930), "On the Solutions of Malfatti's Problem for a Triangle", Proceedings of the London Mathematical Society, 2nd ser., 30 (1): 287-304, doi:10.1112/plms/s2-30.1.287.

Kryształowa kula nie potrafi przewidzieć, kto oceni, czy algorytm zachłanny zawsze znajduje $n \ge 4$ rozłącznych kół w trójkącie o maksymalnej łącznej powierzchni.

(O więcej niż jednym okręgu wpisanym w trójkąt pisaliśmy w podpodsekcji \ref{sssection_6_7_9_circles}).


\subsection{Apolloniusz}
GUZICKI-19 **zadanie konstrukcyjne apolloniusza** wykorzystuje twierdzenie menelaosa

Konstrukcje od \ref{delta_2024_12_start} do \ref{delta_2024_12_end} opisane są w czasopiśmie Delta, w numerze grudniowym z 2024 roku.
\todofoot{Dopisać cytowanie w formacie BibTeX}

\begin{geoconstruction}
    \label{delta_2024_12_start}
    Znając pięć punktów okręgu $\omega$, skonstruować styczną do $\omega$ w jednym z tych punktów.
\end{geoconstruction}

\begin{geoconstruction}
    Znając pięć punktów okręgu $\omega$, dla danej prostej $l$ przechodzącej przez jeden z nich wyznaczyć drugi punkt przecięcia $l$ i $\omega$.
\end{geoconstruction}

\begin{geoconstruction}
    Skonstruować środek jednego z dwóch okręgów mających dwa punkty wspólne.
\end{geoconstruction}

\begin{geoconstruction}
    \label{delta_2024_12_end}
    Skonstruować środek przynajmniej jednego z trzech okręgów nienależących do jednego pęku.
\end{geoconstruction}

\subsection{Stożkowe}
przecięcie prostej z parabolą (hartshorne s. 247)

s. 278 Hartshorne: problem Alhazen, równokąty widziane z dwóch punktów na okręgu.

\subsection{Konstrukcje bez linijki}
%

\begin{theorem}[Mohra-Mascheroniego]
\index{twierdzenie!Mohra-Mascheroniego}%
    Jeśli dana konstrukcja jest wykonalna za pomocą cyrkla i~linijki, to jest ona wykonalna za pomocą samego cyrkla, o ile pominiemy rysowanie linii i ograniczymy się do wyznaczania punktów konstrukcji.
\end{theorem}
% PRZECZYTANO: https://en.wikipedia.org/wiki/Mohr-Mascheroni_theorem

Wynik ma ciekawą historię.
Pierwszy znajdzie go Georg Mohr, włoski geometra i poeta w \emph{Euclides Danicus} (1672), ale jego odkrycie popadnie w zapomnienie i będzie tak marnieć aż do roku 1928!
Mohr wykorzysta odbicia względem prostych. % established his results by using the idea of reflection in a line. In 1890, the
Niezależnie twierdzenie odkryje Lorenzo Mascheroniego i~opisze je w~\emph{La Geometria del Compasso} (1797).
W 1890 jeszcze raz to samo, ale nie tak samo (inwersjami) zrobi wiedeński August Adler w pracy, której tytuł nadgryzie ząb czasu.
Wreszcie w 1928 student Hjelmsleva przeglądając księgarnię w Kopenhadze i jej zawartość znajdzie książkę Mohra.
Będzie to duża niespodzianka dla Hjelmsleva!

Dla dowodu wystarczy pokazać, że cyrkel wystarcza do przeprowadzenia pięciu konstrukcji:
\begin{enumerate}
    \item poprowadzenia prostej przez dwa punkty,
    \item wykreślenia okręgu o danym środku i promieniu,
    \item znalezienia punktu przecięcia dwóch nierównoległych prostych,
    \item znalezienia punktu (lub punktów) przecięcia prostej z okręgiem,
    \item znalezienia punktu (lub punktów) przecięcia dwóch okręgów.
\end{enumerate}

Coxeter \cite[s. 95]{coxeter_1967} sugeruje skonstruować cyrklem wierzchołki sześciokąta foremnego, odcinek dwa razy dłuższy (i krótszy) od danego, obraz inwersyjny danego punktu leżącego dowolnie blisko środka i wreszcie podzielić dany odcinek na $n$ równych części.

%
\color{red}

\begin{problem}[zadanie Napoleona]
	Podzielić dany okrąg (bez znanego środka) na cztery łuki równej miary korzystając z cyrkla, ale nie linijki.
\end{problem}

Nie wiadomo, czy Napoleon wymyślił albo rozwiązał przedstawione wyżej zadanie konstrukcyjne.
Rozwiązanie: \cite[s. 116]{neugebauer} z wykorzystaniem okręgów Torricelliego.
\index{okrąg Torricelliego}%

\begin{problem}[zadanie Fermata]
	Dany jest trójkąt $ABC$.
	Znaleźć punkt $F$ taki, by suma $|FA| + |FB| + |FC|$ była możliwie najmniejsza.
\end{problem}

Powyższe zadanie rozwiązał Evangelista Torricelli, który dostał je w formie wyzwania od Fermata.
Rozwiązanie opublikował student Torricelliego, Viviani, w 1659 roku.
% TODO: Johnson, R. A. Modern Geometry: An Elementary Treatise on the Geometry of the Triangle and the Circle. Boston, MA: Houghton Mifflin, pp. 221-222, 1929.

% TODO: rozwiązanie https://en.wikipedia.org/wiki/Napoleon%27s_problem

\color{black}


\subsection{Konstrukcje bez cyrkla}
\begin{theorem}[Ponceleta-Steinera]
    Jeśli dana konstrukcja jest wykonalna za pomocą cyrkla i linijki, to jest ona wykonalna za pomocą samej linijki, o ile dany jest na płaszczyźnie pewien okrąg wraz ze środkiem.
\end{theorem}
% Renaissance mathematician Lodovico Ferrari, a student of Gerolamo Cardano in a "mathematical challenge" against Niccolò Fontana Tartaglia was able to show that "all of Euclid" (that is, the straightedge and compass constructions in the first six books of Euclid's Elements) could be accomplished with a straightedge and rusty compass. Within ten years additional sets of solutions were obtained by Cardano, Tartaglia and Tartaglia's student Benedetti.[2] During the next century these solutions were generally forgotten until, in 1673, Georg Mohr published (anonymously and in Dutch) Euclidis Curiosi containing his own solutions. Mohr had only heard about the existence of the earlier results and this led him to work on the problem.[3]
% https://en.wikipedia.org/wiki/Poncelet%E2%80%93Steiner_theorem

Jean Poncelet postawił powyższe jako hipotezę w 1822 roku.
\index{Poncelet, Jean}%
Jakob Steiner przedstawił dowód w 1833 roku.
\index{Steiner, Jakob}
Sama linijka nie jest wystarczająca, bo nie pozwala na wyciąganie pierwiastków kwadratowych.
Francesco Severi wzmocnił to twierdzenie w 1904 roku: zamiast całego okręgu, wystarczy, że mamy do dyspozycji mały jego łuk.
\index{Severi, Francesco}

Dodatkowo, zamiast środka okręgu możemy wymagać drugiego koncentrycznego okręgu, drugiego okręgu przecinającego pierwszy, drugiego okręgu rozłącznego z pierwszym i punktu na prostej łączącej ich środki albo ich osi potęgowej.
Zapewne istnieje jeszcze więcej takich warunków.

\begin{problem}
    Dany jest okrąg $\Gamma$ ze środkiem $O$ oraz odcinek.
    Skonstruować środek odcinka. \hfill \emph{(15 kroków)}. % Hartshorne s. 192
\end{problem}

\begin{problem}
    Dany jest okrąg $\Gamma$ ze środkiem $O$, prosta $l$ oraz punkt $P$.
    Skonstruować prostą równoległą do $l$, która przechodzi przez $P$. \hfill \emph{(16 kroków)}. % Hartshorne s. 192
\end{problem}

\begin{problem}
    Dany jest okrąg $\Gamma$ ze środkiem $O$, odcinek $OA$ oraz półprosta zaczynająca się w $O$.
    Skonstruować punkt $B$ na półprostej taki, że $OA$ i $OB$ są przystające. \hfill \emph{(17 kroków)}. % Hartshorne s. 193
\end{problem}

\begin{problem}
    Dany jest okrąg $\Gamma$ ze środkiem $O$, prosta $l$ oraz punkt $P$.
    Skonstruować prostą prostopadłą do $l$, która przechodzi przez $P$. \hfill \emph{(33 kroki)}. % Hartshorne s. 193
\end{problem}

\begin{problem}
    Dany jest okrąg $\Gamma$ ze środkiem $O$, prosta $l$ oraz dwa punkty $A$, $B$.
    Skonstruować punkt, gdzie prosta $l$ przecina okrąg o środku $A$ i promieniu $AB$. \hfill \emph{(54 kroki)}. % Hartshorne s. 193
\end{problem}

\begin{problem}
    Dany jest okrąg $\Gamma$ ze środkiem $O$.
    Skonstruować pięciokąt foremny wpisany w $\Gamma$. \hfill \emph{(około 50 kroków)}. % Hartshorne s. 193
\end{problem}

\todofoot{konstrukcja stycznej do okręgu samą linijką}

\subsection{Uszkodzone przyrządy}
%

Problemy od \ref{broken_ruler_compass_hartshorne_start} do \ref{broken_ruler_compass_hartshorne_end} pochodzą z książki Hartshorne'a \cite[s. 25, 26]{hartshorne2000}.

\begin{geoconstruction}[połamana linijka]
\label{broken_ruler_compass_hartshorne_start}%
\index{linijka!połamana}%
    Dane są dwa punkty $A$ i $B$ na płaszczyźnie, odległe od siebie o około trzy nible.
    Mając do dyspozycji fragment linijki o długości jednej nibli oraz sprawny cyrkiel, narysować odcinek $AB$.
\end{geoconstruction}
% Hartshorne s. 25

\begin{geoconstruction}[zardzewiały cyrkiel]
    \index{cyrkiel!zardzewiały}
    Dane są dwa punkty $A$ i $B$ na płaszczyźnie, odległe od siebie o około pięć nibli.
    Mając do dyspozycji zardzewiały cyrkiel, którym można kreślić jedynie okręgi o promieniu dwóch nibli, skonstruować trójkąt równoboczny oparty o bok $AB$.
\end{geoconstruction}

Konstrukcje zardzewiałym cyrklem były rozpatrywane przez perskiego matematyka Abu al-Wafę Buzjaniego (940-998).
\index[persons]{Buzjani, Abu al-Wafa}%
Miały praktyczne znaczenie, ponieważ stosowali je Leonardo da Vinci czy też Albrecht Dürer pod koniec piętnastego wieku.
\index[persons]{Dürer, Albrech}%
\index[persons]{da Vinci, Leonardo}%
% ŹRÓDŁO: https://en.wikipedia.org/wiki/Poncelet-Steiner_theorem#History

\begin{geoconstruction}
    Dany jest punkt $A$ leżący na prostej $l$.
    Skonstruować prostą prostopadłą do $l$ przechodzącą przez $A$ przy użyciu linijki i zardzewiałego cyrkla.
\end{geoconstruction}
% Hartshorne s. 25

\begin{geoconstruction}
    Dany jest punkt $A$ leżący ponad cztery nible od prostej $l$.
    Skonstruować prostą prostopadłą do $l$ przechodzącą przez $A$ przy użyciu linijki i zardzewiałego cyrkla.
\end{geoconstruction}
% Hartshorne s. 25

\begin{geoconstruction}
    Dane są trzy niewspółliniowe punkty $A$, $B$ oraz $C$.
    Skonstrować punkt $D$ na prostej $AC$ tak, żeby odcinki $AD$ oraz $AB$ były równej długości, przy użyciu linijki i zardzewiałego cyrkla.
\end{geoconstruction}
% Hartshorne s. 26

\begin{geoconstruction}
    Dany jest odcinek $AB$ o długości ponad dwóch nibli oraz prosta $l$, która nie przechodzi przez końce odcinka.
    Skonstrować punkt $C$ na prostej $l$ tak, żeby odcinki $AB$ oraz $AC$ były równej długości, przy użyciu linijki i zardzewiałego cyrkla.
\end{geoconstruction}
% Hartshorne s. 26

\begin{geoconstruction}
\label{broken_ruler_compass_hartshorne_end}%
    Czy wszystkie konstrukcje, które można wykonać cyrklem i linijką, można wykonać też zardzewiałym cyrklem i linijką?
\end{geoconstruction}
% Hartshorne s. 26

(Odpowiedź na to pytali znali Ferrari, Cardano, Tartaglia).
% ŹRÓDŁO: https://en.wikipedia.org/wiki/Mohr-Mascheroni_theorem#Restrictions_involving_the_compass
% Retz, Merlyn; Keihn, Meta Darlene (1989), "Compass and Straightedge Constructions", Historical Topics for the Mathematics Classroom, National Council of Teachers of Mathematics (NCTM), p. 195, ISBN 9780873532815
% TODO: wpisy do indeksu

%

%

% Konstruowalna => stopień Q(x) nad Q to potęga 2, ale nie w drugą stronę.
% Podwojenie sześcianu.
% Trysekcja kąta.

% https://pl.wikipedia.org/wiki/Punkty_Brocarda

\todofoot{konstrukcja neusis z linijką z podziałką} % https://pl.wikipedia.org/wiki/Konstrukcja_neusis

% https://en.wikipedia.org/wiki/Straightedge_and_compass_construction#Solid_constructions
% https://en.wikipedia.org/wiki/Straightedge_and_compass_construction#Angle_trisection_2
% https://en.wikipedia.org/wiki/Straightedge_and_compass_construction#Origami
% https://en.wikipedia.org/wiki/Straightedge_and_compass_construction#Markable_rulers