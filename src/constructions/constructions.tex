%

% TODO: https://en.wikipedia.org/wiki/Straightedge_and_compass_construction
% https://en.wikipedia.org/wiki/Compass_equivalence_theorem
% https://en.wikipedia.org/wiki/4,294,967,295
\section{Konstrukcje klasyczne}

% The compass equivalence theorem proves that the rigid compass (also called the modern compass) - one that holds its spacing when lifted from the plane - is equivalent to the traditional collapsing compass (also called divider) - one that does not retain its spacing, thus "resetting to zero", every time it is lifted from the plane. The ability to transfer distances (i.e. construct congruent circles, translate a circle in the plane) - an operation made trivial by the fixable aperture of a rigid compass - was proven by Euclid to be possible with the collapsing compass. Consequently, the rigid compass and the collapsing compass are equivalent; what can be constructed by one can be constructed by the other. => https://en.wikipedia.org/wiki/Poncelet%E2%80%93Steiner_theorem#Other_types_of_restricted_construction

\todofoot{Klasycznie zakłada się, że cyrkiel nie może przenosić odcinków, bo się rozpada po podniesieniu: However, by the compass equivalence theorem in Proposition 2 of Book 1 of Euclid's Elements, no power is lost by using a collapsing compass.} % The idealized ruler, known as a straightedge, is assumed to be infinite in length, have only one edge, and no markings on it. The compass is assumed to have no maximum or minimum radius, and is assumed to "collapse" when lifted from the page, so it may not be directly used to transfer distances. (This is an unimportant restriction since, using a multi-step procedure, a distance can be transferred even with a collapsing compass; see compass equivalence theorem. Note however that whilst a non-collapsing compass held against a straightedge might seem to be equivalent to marking it, the neusis construction is still impermissible and this is what unmarked really means: see Markable rulers below.) More formally, the only permissible constructions are those granted by the first three postulates of Euclid's Elements.
%  

\todofoot{Hipokrates podwoił sześcian: Hippocrates and Menaechmus showed that the volume of the cube could be doubled by finding the intersections of hyperbolas and parabolas, but these cannot be constructed by straightedge and compass.[2]: p. 30  In the fifth century BCE, Hippias used a curve that he called a quadratrix to both trisect the general angle and square the circle, and Nicomedes in the second century BCE showed how to use a conchoid to trisect an arbitrary angle;[2]: p. 37  but these methods also cannot be followed with just straightedge and compass.}



\subsection{Proste}
% TODO: Hartshorne s. 103
\begin{problem}
    Dwusieczna kąta.
\end{problem}

\begin{problem}
    Środek odcinka.
\end{problem}

\begin{problem}
    Prosta prostopadła do prostej, przechodząca przez punkt.
\end{problem}

\begin{problem}
    Prosta równoległa do prostej, przechodząca przez punkt.
\end{problem}

% TODO: Neugebauer s. 67
\begin{problem}
    Okrąg styczny do prostej, przechodzący przez dwa punkty.
\end{problem}



\subsection{Trudniejsze}
% https://en.wikipedia.org/wiki/Cramer–Castillon_problem

\begin{problem}[Monge'a?]
    Okrąg, który przecina trzy okręgi $\Gamma_1$, $\Gamma_2$, $\Gamma_3$ pod kątem prostym.
\end{problem}

% https://mathworld.wolfram.com/MongesProblem.html
Dla każdej pary okręgów $\Gamma_i$, $\Gamma_j$ znajdujemy oś potęgową; jeśli trzy osie przecinają się w punkcie $O$ leżącym na zewnątrz okręgów $\Gamma_i$, to jest to środek szukanego okręgu.
Promieniem szukanego okręgu jest odcinek styczny do $\Gamma_i$ oraz przechodzący przez $O$.
Jeśli jednak środek okręgu leży wewnątrz któregoś okręgu albo osie nie przecinają się, to problem nie ma rozwiąania.

\begin{problem}
    Dany jest odcinek $AB$ oraz punkt $P$ wewnątrz okręgu.
    Skonstruować cięciwę tego okręgu, która przechodzi przez punkt $P$ o długości takiej samej, jak odcinek $AB$.
\end{problem}
% Hartshorne s. 26

\begin{problem}
    Dany jest odcinek $AB$, inny odcinek o długości $d$ oraz kąt $\alpha$.
    Skonstruować trójkąt $ABC$ tak, by kąt przy wierzchołku $C$ miał miarę $\alpha$, zaś suma długości ramion tego kąta była równa $d$.
\end{problem}
% Hartshorne s. 26

\begin{problem}
    Dane są dwa okręgi takie, że żaden nie jest zawarty w drugim.
    Skonstruować styczną do obydwu okręgów.
\end{problem}
% Hartshorne s. 26

\begin{problem}
    Dany jest okrąg $\Gamma$ oraz jego środek $O$.
    Skonstruować trzy przystające okręgi, które są styczne do pozostałych dwóch oraz do $\Gamma$. \hfill \emph{(13 kroków)}. % Hartshorne s. 51
\end{problem}
% Hartshorne s. 26

\begin{problem}
    Dany jest okrąg $\Gamma$ oraz dwa punkty $A$ i $B$.
    Skonstrować punkt $C$ na okręgu $\Gamma$ tak, by odcinek łączący punkty przecięcia prostych $CA$, $CB$ z okręgiem $\Gamma$ był równoległy do odcinka $AB$.
\end{problem}
% Hartshorne s. 58-59

\begin{problem}
    Skonstruować trzy parami styczne okręgi, każdy o innym promieniu, których środki nie są współliniowe. \hfill \emph{(7 kroków)}. % Hartshorne s. 62
\end{problem}

\subsection{Wyrocznia w Delfach}
%

Przytoczymy teraz trzy problemy geometryczne, zaprzątające głowy przez blisko dwa tysiąclecia.
Wszystkie trzy znajdziemy też u Evesa \cite[s. 156]{eves1_1972}.

Według legendy wyrocznia na wyspie Delos poradzi, by dla powstrzymania panującej tam zarazy trzeba dwukrotnie powiększyć ołtarz ofiarny, nie zmieniając jego sześciennego kształtu.
\index{podwojenie sześcianu}

\begin{problem}[problem delijski]
    \label{problem_delijski}%
    Zbudować sześcian o objętości dwa razy większej niż sześcian dany.
    \index{problem!delijski}
\end{problem}

O problemie następne pokolenia dowiedzą się między innymi z prac Teona ze Smyrny.
\index[persons]{Teon ze Smyrny}%
% https://en.wikipedia.org/wiki/Theon_of_Smyrna  It is also one of the sources of our knowledge of the origins of the classical problem of Doubling the cube.[5]
Choć odkryte zostanie wiele konstrukcji przybliżonych albo takich, które wykorzystują poza cyrklem i~linijką dodatkowe narzędzia (Hipokrates razem z Menaichmosem i hiperbola z parabolą, których nie da się wykreślić; Diokles i cysoida; Nikomedes i konchoida, Hippiasz z Elidy i kwadratrysa), to mimo upływu setek lat nie zostanie zrobiony żaden postęp.
\index[persons]{Hipokrates z Chios}%
\index[persons]{Menaichmos}%
\index{hiperbola}%
\index{parabola}%
\index[persons]{Diokles}%
\index{cysoida}%
\index[persons]{Nikomedes}%
\index{konchoida}%
\index[persons]{Hippiasz z Elidy}%
\index[persons]{kwadratrysa}%

\begin{problem}[trysekcja kąta]
    \label{trysekcja_kata}%
    Podzielić dowolny kąt na trzy równe części.
    \index{trysekcja!kąta}
\end{problem}
% TODO: % TODO: Coxeter, s. 28
% TODO: https://en.wikipedia.org/wiki/Square_trisection
% https://en.wikipedia.org/wiki/Angle_trisection

Tu też pojawią się próby użycia różnorakich przyrządów.
Archmiedes zaproponuje dodanie podziałki do linijki, co okaże się wystarczające do wykreślenia kątów.
\index[persons]{Archimedes}
W średniowieczu zostanie zauważone, że zagadnienie trysekcji kąta prowadzi do rozwiązania równania algebraicznego stopnia trzeciego postaci
\begin{equation}
    x^3 - 3 x - 2 \cos 3\alpha = 0.
\end{equation}
W szczególności, jeśli $\alpha = \pi / 9$, to pierwiastki nie są konstruowalne.
Patrz też do Coxetera \cite[s. 44]{coxeter_1967}.

\begin{proposition}
    Rozwiązanie problemu delijskiego oraz trysekcja kąta są niewykonalne za pomocą cyrkla i linijki.
\end{proposition}

Pierwszy dowód tego faktu poda Pierre Wantzel \cite{wantzel_1837} w 1837 roku.
Być może to będzie powodem, dla którego dopiero w 1899 roku Frank Morley \cite{morley_1907} odkryje:
\index[persons]{Morley, Frank}

\begin{theorem}[Morleya]
    Punkty przeciecięcia tych spośród trójsiecznych kątów trójkąta, które sąsiadują z~którymś z~boków trójkąta, są wierzchołkami trójkąta równobocznego.
    \index{trójkąt!Morleya}%
    \index{twierdzenie!Morleya}%
\end{theorem}

(W tym sformułowaniu nie ma mowy, że chodzi o kąty wewnętrzne -- ponieważ twierdzenie jest prawdziwe także dla kątów zewnętrznych, z prawie takim samym dowodem).
Trójkąt równoboczny, jaki powstaje, ma bok długości
\begin{equation}
    8 R \sin \frac \alpha 3 \sin \frac \beta 3 \sin \frac \gamma 3,
\end{equation}
% Coxeter s. 23-25
% https://en.wikipedia.org/wiki/Hofstadter_points
a samo twierdzenie opiszą Coxeter \cite[s. 39-41]{coxeter_1967}.

Jak pamiętamy, pola powierzchnie wszystkich figur płaskich (trójkąta, równoległoboku, itd.) zostaną wyznaczone w starożytności przez podział figury na mniejsze części i ułożenie z nich prostokąta.
Próbowano tego samego wobec koła, bez powodzenia.

\begin{problem}[kwadratura koła]
\index{kwadratura koła}%
    Wykreślić kwadrat o polu równym polu danego koła.
\end{problem}
% https://en.wikipedia.org/wiki/Wallace-Bolyai-Gerwien_theorem 

Opowiadaliśmy już o nie do końca trafionych pomysłach Mikołaja z Kuzy na stronie \pageref{nierownosc_mikolaja_z_kuzy}.
Adam Adamandy Kochański herbu Lubicz, dworzanin króla Jana Sobieskiego i członek zakonu jezuitów znajdzie przybliżenie
\begin{equation}
    \pi \approx \sqrt{\frac{40}{3} - 2 \sqrt{3}} = 3.14153\,33387...
\end{equation}
w pracy \emph{Observationes Cyclometricae ad facilitandam Praxin accommodatae} (Obserwacje cyklometryczne przystosowane dla ułatwienia praktycznego użycia) z 1685 roku.
Ludzie nie przestaną szukać innych niezbyt czasochłonnych przepisów.
Jacob de Gelder w 1849 znajdzie dokładniejsze przybliżenie
\begin{equation}
    \pi \approx 3 + \frac{4^2}{7^2 + 8^2} = \frac{355}{113} = 3.14159\,29203...
\end{equation}

Wreszcie Ramanujan w 1914 poda inny przepis oparty o przybliżenie zgadzające się na ośmiu miejscach po przecinku:
\begin{equation}
    \pi \approx \sqrt[4]{9^2 + \frac{19^2}{22}} = 3.14159\,26525...
\end{equation}

Ferdinand Lindemann odkryje w 1882 roku, że liczba $\pi$ pojawiająca się we wzorze na pole koła jest przestępna, zatem rozszerzenie $\mathbb Q(\sqrt{\pi}) / \mathbb Q$ jest za dużego stopnia i kwadratura koła, podobnie jak poprzednie dwa problemy, również nie jest wykonalna standardowymi narzędziami. (Jeśli liczba $x$ jest konstruowalna, to stopień rozszerzenia $\mathbb Q(x) / \mathbb Q$ jest potęgą $2$).

Kwadratura koła jest równoważna rektyfikacji okręgu, czyli zadaniu, by wykreślić odcinek tej samej długości, co obwód danego okręgu.
To również nie jest wykonalne.

%

\subsection{Wielokąty foremne}

\subsection{Wielokąty foremne}
Euklides potrafił skonstruować wielokąt foremny o $n$ bokach dla $n = 3$ (I.1), $n = 4$ (I.46), $n = 5$ (IV.11), $n = 6$ (IV.15) oraz $n = 15$ (IV.16).
Ptolemeusz podał konstrukcję pięciokąta foremnego w Almageście.

Mając wielokąt o $n$ bokach, bisekcja kąta środkowego pozwala podwoić liczbę boków do $2n$.
Możemy więc uznać, że Euklides potrafił wykreślić wielokąty, które miały
\begin{equation}
    3, 4, 5, 6, 8, 10, 12, 15, 16, 20, \ldots
\end{equation}
boków.
Lista została rozszerzona dopiero w 1796 roku przez Gaussa: pokazał, że siedemnastokąt foremny jest konstruowalny.
Był tak zadowolony z tego wyniku, że zażyczył sobie wyryć właśnie tę figurę na swoim grobie.
Faktycznej konstrukcji dokonał Johannes Erchinger w 1800 roku.
\index[persons]{Erchinger, Johannes}%
Hartstone \cite[s. 250-259]{hartshorne2000} poświęca tej konstrukcji całą sekcję 29.

W 1832 roku Friederich Richelot skonstruował $257$-kąt.
\index[persons]{Richelot, Friederich}
Jego cierpliwość przyćmiewa konstrukcja $65\,537$-kąta Johanna Hermesa z 1896 roku, na którą poświęcił dziesięć lat swojego życia (i dwieście stron rękopisu).
% The construction is very complex; Hermes spent 10 years completing the 200-page manuscript

W ogólności, mamy:
\begin{theorem}[Gaussa-Wantzla]
    Wielokąt foremny o $n$ bokach jest konstruowalny przy pomocy cyrkla i linijki wtedy i tylko wtedy, kiedy $n$ jest postaci
    \begin{equation}
        n = 2^r p_1 \cdot \ldots \cdot p_s,
    \end{equation}
    $r, s \ge 0$, gdzie $p_i$ są różnymi liczbami pierwszymi postaci $2^{2^k} + 1$, na przykład: $3$, $5$, $17$, $257$, $65537$
\end{theorem}

Uzasadnienie można znaleźć u Hartstone'a \cite[s. 258]{hartshorne2000}.
Gauss znalazł warunek wystarczający i napisał bez dowodu w \emph{Disquisitiones Arithmeticae} (1801), że jest też konieczny.
Brakujący dowód został dodany przez Pierre'a Wantzla w 1837 roku.

Carl Friedrich Gauss proved the constructibility of the regular 17-gon in 1796. Five years later, he developed the theory of Gaussian periods in his . This theory allowed him to formulate a sufficient condition for the constructibility of regular polygons. Gauss stated without proof that this condition was also necessary,[2] but never published his proof.

% TODO: Coxeter, s. 26




GUZICKI-12 **wielokąty foremne** które są konstruowalne? (tw. wantzla itd.) konstrukcje przybliżone pięciokąta - durer i da vinci.
$n = 3$, $n = 4$, $n = 6$ (proste)

\begin{problem}
    Skonstrować trójkąt równoboczny wpisany w okrąg, którego środek nie jest znany. \hfill \emph{(7 kroków)}
\end{problem}

\begin{problem}
    Skonstrować kwadrat. \hfill \emph{(9 kroków)}
\end{problem}

\begin{problem}
    Skonstrować pięciokąt foremny.
\end{problem}

Piszą o tym Hartshorne \cite[s. 45-49]{hartshorne2000}.
Jeśli mamy zadany jeden z jego boków, konstrukcja wymaga 11 kroków. % Hartshorne s. 51




$n = 17$

$n = 7$ (niemożliwe), możliwe ze znaczoną linijką: Hartshorne rozdział 30/31


\subsection{Zadanie Malfattiego}
%

W 1803 roku Malfatti \cite{malfatti_1803} zainspirowany pewnym praktycznym zagadnieniem (wycinanie walców z graniastosłupa) postawi następujący problem:
\index[persons]{Malfatti, Gian Francesco}%

\begin{problem}[zadanie Malfattiego]
	\label{malfatti_problem}
	\index{zadanie!Malfattiego}%
	Dany jest trójkąt $\triangle ABC$.
	Skonstruować takie trzy parami styczne okręgi $\Gamma_A, \Gamma_B, \Gamma_C$, że okrąg $\Gamma_A$ (odpowiednio: $\Gamma_B$, $\Gamma_C$) jest wpisany w~kąt $\angle A$ (odpowiednio: $\angle B$, $\angle C$).
\end{problem}

\begin{figure}[H]
\begin{center}
\begin{tikzpicture}[scale=.5]
\tkzDefPoints{0/0/A,10/2/B,6/7/C}
\tkzDefPoints{4.43012726/2.59439459/Oa}
\tkzDefCircle[R](Oa,1.67519375895) \tkzGetPoint{Oaa}
\tkzDrawCircle[line width=0.2mm](Oa,Oaa)

\tkzDefPoints{7.48168986/2.91734309/Ob}
\tkzDefCircle[R](Ob,1.39341015784) \tkzGetPoint{Obb}
\tkzDrawCircle[line width=0.2mm](Ob,Obb)

\tkzDefPoints{5.96721113/5.06490116/Oc}
\tkzDefCircle[R](Oc,1.23445046858) \tkzGetPoint{Occ}
\tkzDrawCircle[line width=0.2mm](Oc,Occ)

\tkzLabelPoint(A){$A$}
\tkzLabelPoint[anchor=center](Oa){$\Gamma_A$}
\tkzLabelPoint(B){$B$}
\tkzLabelPoint[anchor=center](Ob){$\Gamma_B$}
\tkzLabelPoint[above](C){$C$}
\tkzLabelPoint[anchor=center](Oc){$\Gamma_C$}
\tkzDrawPolygon[line width=0.3mm](A,B,C)
\end{tikzpicture}
\end{center}
\caption{Trzy okręgi Malfattiego}
\end{figure}

Problem będzie rozważany na długo przed Malfattim, zajmie się nim Ajima Naonobu\footnote{Matematyk japoński, przypisze się mu wprowadzenie rachunku różniczkowo-całkowego do matematyki japońskiej.} w~XVIII wieku, a~jeszcze wcześniej Gilio de Cecco da Montepulciano w~rękopisie z~1384 roku.
\index[persons]{Ajima, Naonobu}%
\index[persons]{de Cecco da Montepulciano, Gilio}%

Malfatti wyprowadzi co następuje.
Niech $p$ będzie połową obwodu trójkąta, $r$ będzie promieniem okręgu wpisanego w~ten trójkąt zaś $d_A$, $d_B$, $d_C$ odległościami wierzchołków $A, B, C$ od środka tego okręgu.
Wtedy promienie okręgów Malfattiego wyrażają się wzorami
\begin{align}
	r_A & = \frac r 2 \cdot {\frac {s-r+d_A-d_B-d_C}{p-a}}, \\
	r_B & = \frac r 2 \cdot {\frac {s-r+d_B-d_A-d_C}{p-b}}, \\
	r_C & = \frac r 2 \cdot {\frac {s-r+d_C-d_A-d_B}{p-c}}.
\end{align}

Prostą konstrukcję okręgów opartą na dwustycznych zawdzięczymy Steinerowi \cite{steiner_1826} w~1826 roku;
\index[persons]{Steiner, Jakob}%
inne rozwiązania podadzą Lehmus \cite{lehmus_1819}, Catalan \cite{catalan_1846}, Adams \cite{adams_1846}, Derousseau \cite{derousseau_1895}, Pampuch \cite{pampuch_1904}.
% TODO: po poprawie bibliografii, podać tu index persons

(O~problemie napiszą też Bogdańska, Neugebauer \cite[s. 102]{neugebauer_2018}).

Malfatti postawi tak naprawdę inny problem: znalezienia trzech rozłącznych kół zawartych w~trójkącie, których suma pól jest maksymalna i~błędnie założy, że opisane wyżej okręgi stanowią rozwiązanie.
Pomyłkę zauważą najpierw bez dowodu Lob, Richmond \cite{lob_richmond_1930} w~1930 roku: z trójkąta równobocznego można wyciąć zachłannie kolejno trzy koła, ich łączna powierzchnia jest większa od powierzchni kół znalezionych przez Malfattiego o 1\%.
\index[persons]{Richmond, ?}%
\index[persons]{Lob, ?}%
Howard Eves powtórzy to dla stromych trójkątów równoramiennych o bardzo wąskiej podstawie i dużej wysokości około 1946 roku.
\index[persons]{Eves, Howard}%
% https://en.wikipedia.org/w/index.php?title=Howard_Eves&diff=831382284&oldid=750910758
Goldberg \cite{goldberg_1967} wykaże, że domniemanie Malfattiego nie daje nigdy kół o maksymalnej łącznej powierzchni.
Ostatnie słowo należy zaś do Zalgallera, Losa \cite{zalgaller_los_1992}, którzy znajdą trzy koła rozwiązujące problem Malfattiego w dowolnym trójkącie.
% TODO: Goldberg M., On the original Malfatti problem, Math. Mag. 40 (1967), 241–247.
\index[persons]{Zalgaller, VA?}%
\index[persons]{Los, GA?}%
% TODO: Zalgaller V.A., Los’ G.A., Solution of the Malfatti problem, Ukrain. Geom. Sb. 35 (1992), 14–33 (ang. J. Math. Sci. 72 (1994), 3163–3177).
% TODO: po poprawie bibliografii, podać tu index persons
% TODO: Lob, H.; Richmond, H. W. (1930), "On the Solutions of Malfatti's Problem for a Triangle", Proceedings of the London Mathematical Society, 2nd ser., 30 (1): 287–304, doi:10.1112/plms/s2-30.1.287.

Kryształowa kula nie potrafi przewidzieć, kto oceni, czy algorytm zachłanny zawsze znajduje $n \ge 4$ rozłącznych kół w trójkącie o maksymalnej łącznej powierzchni.

(O więcej niż jednym okręgu wpisanym w trójkąt pisaliśmy w podpodsekcji \ref{sssection_6_7_9_circles}).


\subsection{Apolloniusz}
GUZICKI-19 **zadanie konstrukcyjne apolloniusza** wykorzystuje twierdzenie menelaosa

Konstrukcje od \ref{delta_2024_12_start} do \ref{delta_2024_12_end} opisane są w czasopiśmie Delta, w numerze grudniowym z 2024 roku.
% TODO: opisać wszystkie siedem konstrukcji

\begin{geoconstruction}
    \label{delta_2024_12_start}
    Znając pięć punktów okręgu $\omega$, skonstruować styczną do $\omega$ w jednym z tych punktów.
\end{geoconstruction}
% Niech tymi punktami będą A, B, C, D, E. Przecinamy AB i CD w P, AC i BE w Q oraz PQ i DE w R. Wówczas prosta AR jest szukaną styczną (rys. 1). Podkreślmy, że do przeprowadzenia powyższej konstrukcji nie potrzebowaliśmy mieć narysowanego całego okręgu ω – wystarczyło tylko pięć znajdujących się na nim punktów. Uzasadnienie poprawności wymaga znajomości twierdzenia Pascala (patrz Deltoid z ∆9 14).

\begin{geoconstruction}
    Znając pięć punktów okręgu $\omega$, dla danej prostej $l$ przechodzącej przez jeden z nich wyznaczyć drugi punkt przecięcia $l$ i $\omega$.
\end{geoconstruction}

% W przypadku problemów ze znalezieniem rozwiązania polecam poszukać go
% w ∆6 17. Jesteśmy już gotowi do znalezienia środka okręgu samą linijką, jeśli
% mamy do dyspozycji jeszcze jeden, przecinający go okrąg.
% Konstrukcja 3. 
% Oznaczmy te okręgi przez ω1 i ω2, a ich punkty przecięcia przez A i B.
% Korzystając z konstrukcji 1, konstruujemy styczną do ω1 w punkcie B
% i przecinamy z ω2 w C. Przez A rysujemy prostą, która przecina ω1 w D,
% a ω2 w E. Oznaczmy przez F drugi punkt przecięcia prostej BD z ω2. Na koniec
% niech P będzie przecięciem BC i EF, a Q przecięciem BE i CF. Wówczas
% ?EFB= ?BAD= 180◦
% −?DBA−?ADB= 180◦
% −?DBA−?ABC= ?CBF.
% Oznacza to, że EB= CF, a prosta PQ zawiera średnicę ω2 (rys. 2).
% Po wybraniu innej prostej przechodzącej przez A skonstruujemy inną średnicę,
% i w konsekwencji środek ω2.
% Czytelnik z pewnością sam bez problemu wymyśli konstrukcje środka okręgu
% przy zadanych dwóch okręgach stycznych, a także przy zadanych dwóch
% okręgach współśrodkowych.
% W kolejnych konstrukcjach przyda się kilka pojęć.

\begin{geoconstruction}
    Skonstruować środek jednego z dwóch okręgów mających dwa punkty wspólne.
\end{geoconstruction}

% Rozważmy okrąg ω i dowolny punkt P nieleżący na tym okręgu. Przez punkt P
% poprowadźmy dwie sieczne, które przecinają ω w A i B oraz C i D. Niech proste
% AD i BC przecinają się w Q, a AC i BD przecinają się w R. Prostą QRbędziemy
% nazywać biegunową punktu P względem okręgu ω (rys. 3). Zauważmy, że może
% ona być wyznaczona wyłącznie przy użyciu linijki, nawet jeśli okrąg ω dany jest
% tylko w pięciu punktach (w takim przypadku korzystamy z konstrukcji 2).
% Biegunowe mają liczne i użyteczne własności. Na przykład jeśli P leży na
% zewnątrz ω,to biegunowa P przechodzi przez punkty styczności prostych stycznych
% do ω przechodzących przez P. Stąd dla punktów leżących na okręgu przyjmujemy,
% że biegunową jest styczna w tym punkcie. Zatem, wyznaczając biegunową, możemy
% skonstruować styczną do okręgu przechodzącą przez punkt na nim nieleżący.
% Inną użyteczną własnością jest fakt, że każda sieczna okręgu ω
% przechodząca przez P przecina ω w takich punktach A, B oraz biegunową
% P w takim Q, że AB dzieli harmonicznie PQ, tzn. AP
% BP = AQ
% BQ. Czytelnik
% może spróbować wymyślić, jak podzielić harmonicznie odcinek przy
% użyciu wyłącznie linijki (podpowiedź: warto przypomnieć sobie
% twierdzenia Cevy i Menelaosa).
% Kolejnym przydatnym obiektem będzie pęk okręgów. Jest to rodzina
% okręgów, którą jednoznacznie wyznaczają dwa niewspółśrodkowe okręgi.
% Pęki okręgów mają taką własność, że jeśli dwa okręgi należące do pęku
% przecinają się w dwóch punktach, to każdy okrąg z tego pęku przechodzi
% przez te dwa punkty (rys. 4), jeśli są styczne, to wszystkie są do siebie
% styczne w tym samym punkcie, oraz jeśli się nie przecinają, to żadne
% dwa się nie przecinają (rys. 5). Na potrzeby tego artykułu potraktujmy
% pęki okręgów jako „czarną skrzynkę”, zainteresowanych szczegółami
% odsyłam do krótkiego tekstu w tym wydaniu Delty (s. 20), który jest
% im poświęcony.
% Zachodzi następujące twierdzenie:
% Twierdzenie. Biegunowe dowolnego punktu P względem okręgów
% należących do jednego pęku są współpękowe.
% Punkt ten będziemy nazywali biegunowo sprzężonym do punktu P względem
% odpowiedniego pęku. Ponieważ pęk jest wyznaczony przez dwa okręgi,
% możemy też mówić o dwóch punktach sprzężonych względem pary okręgów.
% Powyższe twierdzenie wykorzystamy w kolejnych konstrukcjach. Ponieważ
% linijka nie pozwala na narysowanie okręgu, przez wyrażenie „skonstruować
% okrąg” będziemy określać wyznaczenie dowolnie wielu jego punktów.
% Konstrukcja 4. Mając dane okręgi λ i µ oraz punkt A na zewnątrz jednego
% z nich, skonstruować okrąg przechodzący przez A oraz należący do pęku
% wyznaczanego przez te okręgi.
% Niech A leży na zewnątrz okręgu λ. Z punktu A skonstruujmy styczną do λ
% w punkcie B. Następnie niech C będzie punktem biegunowo sprzężonym do
% punktu B względem λ i µ (zauważmy, że leży na AB). Konstruujemy teraz taki
% punkt D, że AD dzieli harmonicznie BC. Punkt D jest drugim obok A punktem
% szukanego okręgu (rys. 6). Gdyby okazało się, że D= C= A (tzn. gdyby AB było
% styczne do konstruowanego okręgu), to na początku konstrukcji powinniśmy wziąć
% „drugą styczną” z A do λ. Całą procedurę możemy teraz powtórzyć, biorąc D jako
% punkt startowy (i oczywiście punkt styczności do λ różny od B).

% Konstrukcja 5. Mając dane okręgi λ i µ oraz punkt A leżący wewnątrz nich,
% skonstruować okrąg przechodzący przez A oraz należący do pęku wyznaczanego
% przez te okręgi.
% W tym przypadku wyznaczamy punkt B, biegunowo sprzężony do A. Punkt ten
% leży na zewnątrz okręgów λ i µ, zatem możemy skonstruować dowolną liczbę
% punktów okręgu β przechodzącego przez B i należącego do pęku wyznaczanego
% przez te dwa okręgi (konstrukcja 4). Punkt A leży na zewnątrz β. Pokażemy, jak
% wykorzystać ten „dziurkowany” okrąg do odtworzenia konstrukcji 4.
% Problematyczny jest tylko pierwszy krok, czyli konstrukcja stycznej do β.
% Aby ją wyznaczyć, postępujemy następująco. Niech C będzie różnym od B
% punktem okręgu β. Wyznaczmy punkt D przecięcia prostej AC z okręgiem β
% (korzystamy z konstrukcji 2). Dalej konstruujemy taki E na AC, że AE dzieli
% harmonicznie CD. Prosta BE jest biegunową punktu A względem β, więc jej
% drugi punkt przecięcia z β to taki punkt F (rys. 7), że AF jest styczna do β
% (ponownie skorzystaliśmy z konstrukcji 2). Teraz na AF możemy wyznaczyć
% drugi obok A punkt szukanego okręgu i powtórzyć procedurę, rozpoczynając od
% tego punktu.
% Konstrukcja 6. Skonstruować środek przynajmniej jednego z czterech okręgów,
% z których żadne trzy nie należą do jednego pęku.
% Oznaczmy dane okręgi przez κ, λ, µ, ν. Zakładamy, że żadne dwa z nich nie
% mają punktów wspólnych ani nie są współśrodkowe.
% Wybierzmy punkt A na κ. Konstruujemy okręgi α i β przechodzące przez A oraz
% należące do pęków wyznaczonych odpowiednio przez λ i µ oraz µ i ν. Następnie
% wybieramy taki punkt B na α, że skonstruowana styczna w B do α przecina
% okrąg κ. Niech C będzie tym punktem przecięcia. Niech ponadto D i F będą
% punktami biegunowo sprzężonymi do punktów odpowiednio B i C względem
% pęku wyznaczonego przez okręgi α i β (rys. 8).
% Zauważmy, że E taki, że CE dzieli harmonicznie BD, leży na
% okręgu γ należącym do pęku wyznaczonego przez α i β oraz
% przechodzącym przez C (rozważamy ten okrąg, ale go nie
% konstruujemy). Ponadto prosta CF jest styczna do γ. Czytelnik,
% γ
% analizując ponownie rysunek 2, przekona się, że mamy wystarczająco
% danych, aby zastosować konstrukcję 3 dla okręgów γ i κ i uzyskać
% E
% średnicę κ. Drugą średnicę konstruujemy, zaczynając od innego
% punktu A.
% Odnotujmy, że konstrukcję da się powtórzyć, jeśli jeden z okręgów
% (u nas okrąg µ) jest dany tylko w pięciu punktach. Wynika to
% z możliwości wykonania konstrukcji, gdy okrąg µ jest dany tylko
% w 5 punktach, co z kolei jest konsekwencją poczynionej wcześniej
% uwagi o konstruowalności biegunowych.

\begin{geoconstruction}
    \label{delta_2024_12_end}
    Skonstruować środek przynajmniej jednego z trzech okręgów nienależących do jednego pęku.
\end{geoconstruction}
% Oznaczmy te okręgi przez κ, λ, µ. Wybieramy punkt A na κ i konstruujemy
% okrąg α należący do pęku wyznaczanego przez okręgi λ i µ oraz przechodzący
% przez A. Na okręgach κ i α wybieramy odpowiednio punkty B i C. Następnie
% prowadzimy dowolną prostą przez A i oznaczamy jej punkty przecięcia z κ
% i α przez P i Q, odpowiednio. Zauważmy, że jeśli prosta PQ będzie obracać
% się wokół punktu A, to punkt przecięcia R prostych PB i QC będzie zakreślał
% okrąg (jest to okrąg opisany na trójkącie, którego wierzchołkami są punkty B,
% C i różny od A punkt przecięcia α i κ). Oznaczmy go przez ν. Rysując zatem
% kolejne położenia prostej PQ, będziemy mogli konstruować kolejne punkty
% okręgu ν (rys. 9). Żadne trzy spośród κ, λ, µ, ν nie należą do jednego pęku.
% Stąd po skonstruowaniu pięciu punktów ν możemy powtórzyć konstrukcję 6.
% Czytelnikowi zastanawiającemu się, co z przypadkiem rozłącznych okręgów
% należących do jednego pęku, odpowiem, że wówczas środka okręgu nie da się
% skonstruować. Omówienie tego zagadnienia byłoby jednak zbyt długie, aby
% można je było zawrzeć w tym artykule.

\subsection{Stożkowe}
przecięcie prostej z parabolą (hartshorne s. 247)

% TODO: Alhazen => https://en.wikipedia.org/wiki/Straightedge_and_compass_construction
% https://en.wikipedia.org/wiki/Alhazen%27s_problem

s. 278 Hartshorne: problem Alhazen, równokąty widziane z dwóch punktów na okręgu.
\todofoot{Problem bilardowy Alhazena
Opisać w danym okręgu trójkąt równoramienny, którego ramiona przechodzą przez dwa dane punkty wewnątrz okręgu.
Problem ten pochodzi od arabskiego matematyka Abu Alego al Hassana ibn al Hassana ibn Alhajtama (ok. 965–ok. 1039), którego imię zostało przekształcone na Alhazen przez tłumaczy jego dzieła Optyka.
W Optyce problem ten ma następującą postać:
„Znajdź punkt na wklęsłym zwierciadle sferycznym, w który musi trafić promień światła wychodzący z danego punktu, aby po odbiciu dotarł do innego danego punktu.”
Problem ten można sformułować także w inny sposób, np.:
„Na okrągłym stole bilardowym znajdują się dwie bile; w jaki sposób należy uderzyć jedną z nich, aby po odbiciu od bandy trafiła w drugą?”
lub
„Na okręgu znajdź punkt, którego suma odległości od dwóch danych punktów wewnątrz okręgu jest minimalna (lub maksymalna).”
Z problemem tym mierzyło się wielu wybitnych matematyków po Alhazenie, m.in. Huygens, Barrow, de L’Hôpital, Riccati i Quetelet.
% In 1997, the Oxford mathematician Peter M. Neumann proved the theorem that there is no ruler-and-compass construction for the general solution of the ancient Alhazen's problem (billiard problem or reflection from a spherical mirror).[10][11]
}

\subsection{Konstrukcje bez linijki}
\begin{theorem}[Mohra-Mascheroniego]
    Jeśli dana konstrukcja jest wykonalna za pomocą cyrkla i linijki, to jest ona wykonalna za pomocą samego cyrkla, o ile pominiemy rysowanie linii i ograniczymy się do wyznaczania punktów konstrukcji.
\end{theorem}
% TODO: https://en.wikipedia.org/wiki/Mohr–Mascheroni_theorem

Wynik ten został opublikowany w roku 1672 przez Georga Mohra, był jednak nieznany aż do roku 1928. Niezależnie od Mohra twierdzenie zostało odkryte przez Lorenzo Mascheroniego w roku 1797.
\color{red}

\begin{problem}[zadanie Napoleona]
	Podzielić dany okrąg (bez znanego środka) na cztery łuki równej miary korzystając z cyrkla, ale nie linijki.
\end{problem}

Nie wiadomo, czy Napoleon wymyślił albo rozwiązał przedstawione wyżej zadanie konstrukcyjne.
Rozwiązanie: \cite[s. 116]{neugebauer} z wykorzystaniem okręgów Torricelliego.
\index{okrąg Torricelliego}%



% TODO: rozwiązanie https://en.wikipedia.org/wiki/Napoleon%27s_problem

\color{black}


\subsection{Konstrukcje bez cyrkla}
\begin{theorem}[Ponceleta-Steinera]
    Jeśli dana konstrukcja jest wykonalna za pomocą cyrkla i linijki, to jest ona wykonalna za pomocą samej linijki, o ile dany jest na płaszczyźnie pewien okrąg wraz ze środkiem.
\end{theorem}

Jean Poncelet postawił powyższe jako hipotezę w 1822 roku.
\index{Poncelet, Jean}%
Jakob Steiner przedstawił dowód w 1833 roku.
\index{Steiner, Jakob}
Sama linijka nie jest wystarczająca, bo nie pozwala na wyciąganie pierwiastków kwadratowych.
Francesco Severi wzmocnił to twierdzenie w 1904 roku: zamiast całego okręgu, wystarczy, że mamy do dyspozycji mały jego łuk.
\index{Severi, Francesco}

Dodatkowo, zamiast środka okręgu możemy wymagać drugiego koncentrycznego okręgu, drugiego okręgu przecinającego pierwszy, drugiego okręgu rozłącznego z pierwszym i punktu na prostej łączącej ich środki albo ich osi potęgowej.
Zapewne istnieje jeszcze więcej takich warunków.

\begin{problem}
    Dany jest okrąg $\Gamma$ ze środkiem $O$ oraz odcinek.
    Skonstruować środek odcinka. \hfill \emph{(15 kroków)}. % Hartshorne s. 192
\end{problem}

\begin{problem}
    Dany jest okrąg $\Gamma$ ze środkiem $O$, prosta $l$ oraz punkt $P$.
    Skonstruować prostą równoległą do $l$, która przechodzi przez $P$. \hfill \emph{(16 kroków)}. % Hartshorne s. 192
\end{problem}

\begin{problem}
    Dany jest okrąg $\Gamma$ ze środkiem $O$, odcinek $OA$ oraz półprosta zaczynająca się w $O$.
    Skonstruować punkt $B$ na półprostej taki, że $OA$ i $OB$ są przystające. \hfill \emph{(17 kroków)}. % Hartshorne s. 193
\end{problem}

\begin{problem}
    Dany jest okrąg $\Gamma$ ze środkiem $O$, prosta $l$ oraz punkt $P$.
    Skonstruować prostą prostopadłą do $l$, która przechodzi przez $P$. \hfill \emph{(33 kroki)}. % Hartshorne s. 193
\end{problem}

\begin{problem}
    Dany jest okrąg $\Gamma$ ze środkiem $O$, prosta $l$ oraz dwa punkty $A$, $B$.
    Skonstruować punkt, gdzie prosta $l$ przecina okrąg o środku $A$ i promieniu $AB$. \hfill \emph{(54 kroki)}. % Hartshorne s. 193
\end{problem}

\begin{problem}
    Dany jest okrąg $\Gamma$ ze środkiem $O$.
    Skonstruować pięciokąt foremny wpisany w $\Gamma$. \hfill \emph{(około 50 kroków)}. % Hartshorne s. 193
\end{problem}

\subsection{Uszkodzone przyrządy}
%

Problemy od \ref{broken_ruler_compass_hartshorne_start} do \ref{broken_ruler_compass_hartshorne_end} pochodzą z książki Hartshorne'a \cite[s. 25, 26]{hartshorne2000}.

\begin{geoconstruction}[połamana linijka]
\label{broken_ruler_compass_hartshorne_start}%
\index{linijka!połamana}%
    Dane są dwa punkty $A$ i $B$ na płaszczyźnie, odległe od siebie o około trzy nible.
    Mając do dyspozycji fragment linijki o długości jednej nibli oraz sprawny cyrkiel, narysować odcinek $AB$.
\end{geoconstruction}
% Hartshorne s. 25
O problemie pisze Eves \cite[s. 181]{eves1_1972}.

\begin{geoconstruction}[zardzewiały cyrkiel]
    \index{cyrkiel!zardzewiały}
    Dane są dwa punkty $A$ i $B$ na płaszczyźnie, odległe od siebie o około pięć nibli.
    Mając do dyspozycji zardzewiały cyrkiel, którym można kreślić jedynie okręgi o promieniu dwóch nibli, skonstruować trójkąt równoboczny oparty o bok $AB$.
\end{geoconstruction}

Konstrukcje zardzewiałym cyrklem były rozpatrywane przez perskiego matematyka Abu al-Wafę Buzjaniego (940-998).
\index[persons]{Buzjani, Abu al-Wafa}%
Miały praktyczne znaczenie, ponieważ stosowali je Leonardo da Vinci czy też Albrecht Dürer pod koniec piętnastego wieku.
\index[persons]{Dürer, Albrech}%
\index[persons]{da Vinci, Leonardo}%
% ŹRÓDŁO: https://en.wikipedia.org/wiki/Poncelet-Steiner_theorem#History

\begin{geoconstruction}
    Dany jest punkt $A$ leżący na prostej $l$.
    Skonstruować prostą prostopadłą do $l$ przechodzącą przez $A$ przy użyciu linijki i zardzewiałego cyrkla.
\end{geoconstruction}
% Hartshorne s. 25

\begin{geoconstruction}
    Dany jest punkt $A$ leżący ponad cztery nible od prostej $l$.
    Skonstruować prostą prostopadłą do $l$ przechodzącą przez $A$ przy użyciu linijki i zardzewiałego cyrkla.
\end{geoconstruction}
% Hartshorne s. 25

\begin{geoconstruction}
    Dane są trzy niewspółliniowe punkty $A$, $B$ oraz $C$.
    Skonstrować punkt $D$ na prostej $AC$ tak, żeby odcinki $AD$ oraz $AB$ były równej długości, przy użyciu linijki i zardzewiałego cyrkla.
\end{geoconstruction}
% Hartshorne s. 26

\begin{geoconstruction}
    Dany jest odcinek $AB$ o długości ponad dwóch nibli oraz prosta $l$, która nie przechodzi przez końce odcinka.
    Skonstrować punkt $C$ na prostej $l$ tak, żeby odcinki $AB$ oraz $AC$ były równej długości, przy użyciu linijki i zardzewiałego cyrkla.
\end{geoconstruction}
% Hartshorne s. 26

\begin{proposition}
\label{broken_ruler_compass_hartshorne_end}%
    Każdą konstrukcję geometryczną, którą można wykonać cyrklem i linijką, można powtórzyć przy użyciu zardzewiałego cyrkla i linijki.
\end{proposition}
% Hartshorne s. 26

Wynik ten zostanie odkryty relatywnie wcześnie.
Pierwszy pokaże go Lodovico Ferrari, uczeń Gerolamo Cardano i odkrywca metody rozwiązywania równań czwartego stopnia w pojedynku przeciwko Niccolò Fontanie Tartaglii.
W przeciągu dziesięciu lat wyczyn ten powtórzą Cardano, Tartaglia oraz uczeń Tartaglii, Benedetti.

Ciąg dalszy to oczywiście twierdzenie \ref{thm:poncelet_steiner}, że zardzewiałego cyrkla wystarczy użyć raz.

%

%

% Konstruowalna => stopień Q(x) nad Q to potęga 2, ale nie w drugą stronę.
% Podwojenie sześcianu. % https://en.wikipedia.org/wiki/Pandrosion
% Trysekcja kąta.

% https://pl.wikipedia.org/wiki/Punkty_Brocarda

\todofoot{konstrukcja neusis z linijką z podziałką} % https://pl.wikipedia.org/wiki/Konstrukcja_neusis

% https://en.wikipedia.org/wiki/Straightedge_and_compass_construction#Solid_constructions
% https://en.wikipedia.org/wiki/Straightedge_and_compass_construction#Angle_trisection_2
% https://en.wikipedia.org/wiki/Straightedge_and_compass_construction#Origami
% https://en.wikipedia.org/wiki/Straightedge_and_compass_construction#Markable_rulers

Wpisać w dany okrąg trójkąt, którego boki przechodzą przez trzy dane punkty.
Ten problem, postawiony przez szwajcarskiego matematyka Cramera, nosi nazwę włoskiego matematyka Castillona, który rozwiązał go w 1776 roku.
(Gabriel Cramer, 1704–1752, opublikował w 1750 roku swoje główne dzieło Introduction à l’analyse des lignes courbes algébriques, w którym po raz pierwszy rozwiązano układ równań liniowych za pomocą wyznaczników.
I. F. Salvemini, 1709–1791, przyjął nazwisko Castillon od swojego miejsca urodzenia – Castiglione w Toskanii.)
Poniższe proste, choć niełatwe do zauważenia rozwiązanie problemu Castillona pochodzi od Włocha Giordano.
% https://en.wikipedia.org/wiki/Cramer–Castillon_problem