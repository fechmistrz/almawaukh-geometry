%

Poniższy problem będzie mieć ciekawą historię.

\index{zadanie!Cramera-Castillona}
\begin{problem}
    \label{problem_cramera_castillona}%
    Dane jest $n$ punktów oraz okrąg.
    Znaleźć taki wielokąt wpisany w okrąg, że na każdym jego boku (lub jego przedłużeniu) znajduje się dokładnie jeden z danych punktów.
\end{problem}
% TODO https://ems.press/content/serial-article-files/45288 ładny obrazek, związek z Urquhartem

Szczególny przypadek $n = 3$ współliniowych punktów zainteresuje Pappusa.
\index[persons]{Pappus}%
Pewien nieznany, ale stary geometra przedstawi problem dla dowolnych trzech punktów Gabrielowi Cramerowi, który w 1742 przekaże go dalej do Giovanniego Salveminiego\footnote{Giovanni Francesco Mauro Melchiorre Salvemini di Castiglione przyjmie w pewnym roku nazwisko od swojego miejsca urodzenia, Castiglione del Valdarno w Toskanii.}.
\index[persons]{Castiglione, Giovanni}%
\index[persons]{Cramer, Gabriel}%
Ten znajdzie geometryczne rozwiązanie już po śmierci Cramera, w~1776 roku; wyczyn powtórzą Euler (1783 rok),
Lagrange, Malfatti, Lhuilier, Servois, Poncelet
\index[persons]{Lagrange, ?}%
\index[persons]{Malfatti, ?}%
\index[persons]{Lhuilier, ?}%
\index[persons]{Servois, ?}%
\index[persons]{Poncelet, ?}%
% https://cms.math.ca/wp-content/uploads/crux-pdfs/Crux_v9n04_Apr.pdf Crux Mathematicorum, Vol 9, No 4, page 125
i szesnastoletni Annibale\footnote{Carnot uzna, że Ottajano (miejsce urodzenia Giordana) jest tytułem szlacheckim, a nie nazwą miejscowości; w swoich publikacjach nazwie młodego matematyka właśnie Ottajano. Niestety, ale później będzie tak określany także w kolejnych pracach naukowych.} Giordano.
\index[persons]{Giordano, Annibale}%
Carnot uprości analityczne rozwiązanie Lagrange'a i~uogólni je do $n \ge 3$ punktów.
\index[persons]{Carnot, ?}%

% TODO: https://en.wikipedia.org/wiki/Cramer-Castillon_problem

%