\begin{problem}[problem delijski]
    Zbudować sześcian o objętości dwa razy większej niż sześcian dany.
\end{problem}
\todofoot{Theon Smyrny} % https://en.wikipedia.org/wiki/Theon_of_Smyrna  It is also one of the sources of our knowledge of the origins of the classical problem of Doubling the cube.[5]

Według legendy wyrocznia na wyspie Delos poradziła, by dla powstrzymania panującej tam zarazy dwukrotnie powiększyć ołtarz ofiarny, nie zmieniając jego kształtu.
Problem ten zajmował matematyków przez wiele stuleci; w efekcie istnieją konstrukcje przybliżone, a także konstrukcje wykorzystujące dodatkowo pewne krzywe, np. konstrukcja Dioklesa wykorzystująca cysoidę, konstrukcja Nikomedesa wykorzystująca konchoidę.
% Hippocrates and Menaechmus showed that the volume of the cube could be doubled by finding the intersections of hyperbolas and parabolas, but these cannot be constructed by straightedge and compass. => https://en.wikipedia.org/wiki/Straightedge_and_compass_construction

\begin{problem}[trysekcja kąta]
    Podzielić dowolny kąt na trzy równe części.
\end{problem}
% TODO: % TODO: Coxeter, s. 28

Około V wieku próbowano rozwiązać to zagadnienie za pomocą innych środków konstrukcyjnych, np. wykorzystując kwadratrysę, konchoidę; Archimedes np. zaproponował następujący sposób trysekcji kąta za pomocą cyrkla i linijki z podziałką.
% promieniem BA (równym np. jedności) zakreśla się okrąg (otrzymując punkty A i C), przedłuża średnicę AD poza okrąg i tak się umieszcza linijkę z zaznaczonym na niej odcinkiem EF = AB, by przechodziła ona przez punkt C oraz by zaznaczony na niej punkt F leżał na okręgu, punkt E zaś na przedłużeniu średnicy AD; wówczas DEF = 1/3ABC.
W średniowieczu stwierdzono, że zagadnienie trysekcji kąta prowadzi do rozwiązania równania algebraicznego stopnia trzeciego postaci
\begin{equation}
    x^3 - 3 x - 2 \cos 3\alpha = 0,
\end{equation}
którego pierwiastek konstruuje się następnie za pomocą rozmaitych środków konstrukcyjnych.

\begin{proposition}
    Rozwiązanie problemu delijskiego oraz trysekcja kąta są niewykonalne za pomocą cyrkla i linijki.
\end{proposition}

Pierwszy dowód tego faktu podał P. L. Wantzel (1837).
Być może było to powodem, dla którego dopiero w 1899 roku Frank Morley odkrył (a w 1924 opublikował):
\index[persons]{Morley, Frank}

\begin{theorem}[Morleya]
    Punkty przeciecięcia tych trójsiecznych kątów trójkąta, które sąsiadują z którymś z boków trójkąta,są wierzchołkami trójkąta równobocznego.
\end{theorem}

(W tym sformułowaniu nie ma mowy, że chodzi o kąty wewnętrzne -- ponieważ twierdzenie jest prawdziwe także dla kątów zewnętrznych, z prawie takim samym dowodem).
Trójkąt równoboczny, jaki powstaje, ma bok długości
\begin{equation}
    8 R \sin \frac \alpha 3 \sin \frac \beta 3 \sin \frac \gamma 3.
\end{equation}
% Coxeter s. 23-25
% https://en.wikipedia.org/wiki/Hofstadter_points

\begin{problem}[kwadratura koła]
    Wykreślić kwadrat o polu równym polu danego koła.
\end{problem}
% https://en.wikipedia.org/wiki/Adam_Adamandy_Kocha%C5%84ski
% TODO: https://en.wikipedia.org/wiki/Lune_of_Hippocrates
% Without the constraint of requiring solution by ruler and compass alone, the problem is easily solvable by a wide variety of geometric and algebraic means, and was solved many times in antiquity. => https://en.wikipedia.org/wiki/Straightedge_and_compass_construction

1833 F. Lindemann udowodnił, że kwadratura koła jest niewykonalna za pomocą linijki i cyrkla.

% TODO: Alhazen => https://en.wikipedia.org/wiki/Straightedge_and_compass_construction

% Sixteen key points of a triangle are its vertices, the midpoints of its sides, the feet of its altitudes, the feet of its internal angle bisectors, and its circumcenter, centroid, orthocenter, and incenter. These can be taken three at a time to yield 139 distinct nontrivial problems of constructing a triangle from three points.[12] Of these problems, three involve a point that can be uniquely constructed from the other two points; 23 can be non-uniquely constructed (in fact for infinitely many solutions) but only if the locations of the points obey certain constraints; in 74 the problem is constructible in the general case; and in 39 the required triangle exists but is not constructible. => https://en.wikipedia.org/wiki/Straightedge_and_compass_construction

\todofoot{Trysekcja kąta angle trisection} % https://en.wikipedia.org/wiki/Angle_trisection