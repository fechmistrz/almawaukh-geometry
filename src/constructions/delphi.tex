Według legendy wyrocznia na wyspie Delos poradziła, by dla powstrzymania panującej tam zarazy dwukrotnie powiększyć ołtarz ofiarny, nie zmieniając jego kształtu.
\index{podwojenie sześcianu}

\begin{problem}[problem delijski]
    Zbudować sześcian o objętości dwa razy większej niż sześcian dany.
    \index{problem!delijski}
\end{problem}

O problemie następne pokolenia dowiedzą się między innymi z prac Teona ze Smyrny.
\index[persons]{Teon ze Smyrny}%
% https://en.wikipedia.org/wiki/Theon_of_Smyrna  It is also one of the sources of our knowledge of the origins of the classical problem of Doubling the cube.[5]
Choć odkryte zostanie wiele konstrukcji przybliżonych albo takich, które wykorzystują poza cyrklem i~linijką dodatkowe narzędzia (Hipokrates razem z Menaichmosem i hiperbola z parabolą, których nie da się wykreślić; Diokles i cysoida; Nikomedes i konchoida, Hippiasz z Elidy i kwadratrysa), to mimo upływu setek lat nie zostanie zrobiony żaden postęp.
\index[persons]{Hipokrates}%
\index[persons]{Menaichmos}%
\index{hiperbola}%
\index{parabola}%
\index[persons]{Diokles}%
\index{cysoida}%
\index[persons]{Nikomedes}%
\index{konchoida}%
\index[persons]{Hippiasz z Elidy}%
\index[persons]{kwadratrysa}%

\begin{problem}[trysekcja kąta]
    Podzielić dowolny kąt na trzy równe części.
    \index{trysekcja!kąta}
\end{problem}
% TODO: % TODO: Coxeter, s. 28
% TODO: https://en.wikipedia.org/wiki/Square_trisection

Tu też pojawią się próby użycia różnorakich przyrządów.
Archmiedes zaproponuje dodanie podziałki do linijki, co okaże się wystarczające do wykreślenia kątów.
\index[persons]{Archimedes}
W średniowieczu zostanie zauważone, że zagadnienie trysekcji kąta prowadzi do rozwiązania równania algebraicznego stopnia trzeciego postaci
\begin{equation}
    x^3 - 3 x - 2 \cos 3\alpha = 0.
\end{equation}
W szczególności, jeśli $\alpha = \pi / 9$, to pierwiastki okażą się nie być konstruowalne.

\begin{proposition}
    Rozwiązanie problemu delijskiego oraz trysekcja kąta są niewykonalne za pomocą cyrkla i linijki.
\end{proposition}

Pierwszy dowód tego faktu poda Pierre Wantzel (1837).
Być może to będzie powodem, dla którego dopiero w 1899 roku Frank Morley odkryje (a w 1924 opublikuje):
\index[persons]{Morley, Frank}

\begin{theorem}[Morleya]
    Punkty przeciecięcia tych trójsiecznych kątów trójkąta, które sąsiadują z którymś z boków trójkąta, są wierzchołkami trójkąta równobocznego.
    \index{trójkąt!Morleya}
    \index{twierdzenie!Morleya}
\end{theorem}

(W tym sformułowaniu nie ma mowy, że chodzi o kąty wewnętrzne -- ponieważ twierdzenie jest prawdziwe także dla kątów zewnętrznych, z prawie takim samym dowodem).
Trójkąt równoboczny, jaki powstaje, ma bok długości
\begin{equation}
    8 R \sin \frac \alpha 3 \sin \frac \beta 3 \sin \frac \gamma 3.
\end{equation}
% Coxeter s. 23-25
% https://en.wikipedia.org/wiki/Hofstadter_points

Jak pamiętamy, pola powierzchnie wszystkich figur płaskich (trójkąta, równoległoboku, itd.) zostaną wyznaczone w starożytności przez podział figury na mniejsze części i ułożenie z nich prostokąta.
Próbowano tego samego wobec koła, bez powodzenia.

\begin{problem}[kwadratura koła]
\index{kwadratura koła}%
    Wykreślić kwadrat o polu równym polu danego koła.
\end{problem}
% https://en.wikipedia.org/wiki/Adam_Adamandy_Kocha%C5%84ski
% TODO: https://en.wikipedia.org/wiki/Lune_of_Hippocrates
% Without the constraint of requiring solution by ruler and compass alone, the problem is easily solvable by a wide variety of geometric and algebraic means, and was solved many times in antiquity. => https://en.wikipedia.org/wiki/Straightedge_and_compass_construction
% https://en.wikipedia.org/wiki/Wallace-Bolyai-Gerwien_theorem 
% ciekawe, bo dla 3d nie działa, bo mają dziwną teorię pola, bo nie mają R

Ferdinand Lindemann odkryje w 1882 roku, że liczba $\pi$ pojawiajaca się we wzorze na pole koła jest przestępna, zatem rozszerzenie $\mathbb Q(\sqrt{\pi}) / \mathbb Q$ jest za dużego stopnia i kwadratura koła, podobnie jak poprzednie dwa problemy, również nie jest wykonalna standardowymi narzędziami.

\todofoot{Trysekcja kąta angle trisection} % https://en.wikipedia.org/wiki/Angle_trisection

% Konstruowalna => stopień Q(x) nad Q to potęga 2, ale nie w drugą stronę.
% Podwojenie sześcianu. % https://en.wikipedia.org/wiki/Pandrosion
% Trysekcja kąta.