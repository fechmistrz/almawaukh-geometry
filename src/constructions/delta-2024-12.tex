%

Konstrukcje od \ref{delta_2024_12_start} do \ref{delta_2024_12_end} opisze Stanisław Majchrzak w $\Delta_{24}^{12}$.

\begin{geoconstruction}
    \label{delta_2024_12_start}
    Dane jest pięć punktów leżących na okręgu $\Gamma$.
    Wykreślić styczną do $\Gamma$ w jednym z tych punktów.
\end{geoconstruction}

\begin{geoconstruction}
    Dane jest pięć punktów leżących na okręgu $\Gamma$ oraz prosta $l$ przechodząca przez jeden z nich.
    Wyznaczyć drugi punkt przecięcia okręgu $\Gamma$ oraz prostej $l$.
\end{geoconstruction}

\begin{geoconstruction}
    Dane są dwa okręgi przecinające się w dwóch punktach.
    Wyznaczyć środek jednego z nich.
\end{geoconstruction}

\begin{geoconstruction}
    Dane są dwa okręgi oraz punkt $A$ na zewnątrz jednego z nich (wewnątrz obydwu).
    Wykreślić okrąg przechodzący przez $A$, który jest współpękowy z dwoma danymi.
\end{geoconstruction}

\begin{geoconstruction}
    \label{delta_2024_12_end}
    Dane są cztery (w trudniejszej wersji: trzy) okręgi takie, że żadne trzy nie są współpękowe.
    Wyznaczyć środek przynajmniej jednego z nich.
\end{geoconstruction}

Dla rozłącznych okręgów należących do jednego pęku, środka nie da się (!) skonstruować.

%