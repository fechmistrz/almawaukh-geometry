%

W 1803 roku Malfatti \cite{malfatti_1803} zainspirowany pewnym praktycznym zagadnieniem (wycinanie walców z graniastosłupa) postawił następujący problem:
\index[persons]{Malfatti, Gian Francesco}%

\begin{problem}[zadanie Malfattiego]
	\label{malfatti_problem}
\index{zadanie Malfattiego}%
	Dany jest trójkąt $\triangle ABC$.
	Skonstruować takie trzy parami styczne okręgi $K_A, K_B, K_C$, że okrąg $K_A$ (odpowiednio: $K_B$, $K_C$) jest wpisany w~kąt $\angle A$ (odpowiednio: $\angle B$, $\angle C$).
\end{problem}

Problem był rozważany na długo przed Malfattim, zajmował się nim Ajima Naonobu\footnote{Matematyk japoński, przypisuje się mu wprowadzenie rachunku różniczkowo-całkowego do matematyki japońskiej.} w~XVIII wieku, a~jeszcze wcześniej zadanie Gilio de Cecco da Montepulciano w~rękopisie z~1384 roku.
\index[persons]{Ajima, Naonobu}%
\index[persons]{de Cecco da Montepulciano, Gilio}%
Malfatti wyprowadził co następuje.
Niech $r$ będzie promieniem koła wpisanego w~trójkąt, $s$ połową jego obwodu, a~$d_A, d_B, d_C$ odległościami wierzchołków $A, B, C$ od środka koła wpisanego.
Wtedy promienie kół Malfattiego wyrażają się wzorami
\begin{align}
	r_1 & = {\frac {r}{2(s-a)}}(s-r+d_A-d_B-d_C), \\
	r_2 & = {\frac {r}{2(s-b)}}(s-r-d_A+d_B-d_C), \\
	r_3 & = {\frac {r}{2(s-c)}}(s-r-d_A-d_B+d_C).
\end{align}

Prostą konstrukcję okręgów opartą na dwustycznych zawdzięczamy Steinerowi \cite{steiner_1826} w~1826 roku;
\index[persons]{Steiner, Jakob}%
inne rozwiązania podali Lehmus \cite{lehmus_1819}, Catalan \cite{catalan_1846}, Adams \cite{adams_1846}, Derousseau \cite{derousseau_1895}, Pampuch \cite{pampuch_1904}.
% TODO: po poprawie bibliografii, podać tu index persons
(O problemie pisze też Neugebauer \cite[s. 102]{neugebauer_2018}).

Malfatti postawił tak naprawdę inny problem: znalezienia trzech rozłącznych kół zawartych w~trójkącie, których suma pól jest maksymalna i~błędnie założył, że opisane wyżej okręgi stanowią rozwiązanie.
Pomyłkę zauważyli najpierw bez dowodu Lob, Richmond \cite{lob_richmond_1930} w~1930 roku: z trójkąta równobocznego można wyciąć zachłannie kolejno trzy koła, ich łączna powierzchnia jest większa od powierzchni kół znalezionych przez Malfattiego o 1\%.
\index[persons]{Richmond, ?}%
\index[persons]{Lob, ?}%
Goldberg \cite{goldberg_1967} wykazał, że domniemanie Malfattiego nie daje nigdy kół o maksymalnej łącznej powierzchni.
Ostatnie słowo należało zaś do Zalgallera, Losa \cite{zalgaller_los_1992}, którzy znaleźli trzy koła rozwiązujące problem Malfattiego w dowolnym trójkącie.
% TODO: Goldberg M., On the original Malfatti problem, Math. Mag. 40 (1967), 241–247.
\index[persons]{Zalgaller, VA?}%
\index[persons]{Los, GA?}%
% TODO: Zalgaller V.A., Los’ G.A., Solution of the Malfatti problem, Ukrain. Geom. Sb. 35 (1992), 14–33 (ang. J. Math. Sci. 72 (1994), 3163–3177).
% TODO: po poprawie bibliografii, podać tu index persons
% TODO: Lob, H.; Richmond, H. W. (1930), "On the Solutions of Malfatti's Problem for a Triangle", Proceedings of the London Mathematical Society, 2nd ser., 30 (1): 287–304, doi:10.1112/plms/s2-30.1.287.

Nie wiadomo, czy algorytm zachłanny zawsze znajduje $n \ge 4$ rozłącznych kół w trójkącie o maksymalnej łącznej powierzchni.

\subsubsection{Twierdzenia o sześciu, siedmiu i dziewięciu okręgach}
% TODO: pięciu, Guzicki ps. 36
Cecil John Alvin Evelyn, G. B. (pełne imiona nieznane) Money-Coutts i John Alfred Tyrrell znaleźli około 1974 roku piękne twierdzenia o okręgach wpisanych, które pokazują nam, jak wiele wyników geometrii elementarnej oczekuje na swoje odkrycie.

\begin{proposition}[o sześciu okręgach]
	Dany jest trójkąt $\triangle ABC$.
	Niech $\Gamma_1, \Gamma_2, \ldots, \Gamma_6$ będą okręgami wpisanymi kolejno w kąty przy wierzchołkach $A$, $B$, $C$, $A$, $B$, $C$, $A$ takimi, że każdy jest styczny do poprzedniego.
	Wtedy okręgi $\Gamma_6$ i $\Gamma_1$ też są do siebie styczne.
\end{proposition}

Pisze o tym Bogdańska, Neugebauer \cite[s. 101]{neugebauer_2018}: wykorzystują punkt Crelle'a-Brocarda, wzór Herona i trochę trygonometrii.
Tabacznikow, Iwanow \cite{ivanov_tabachnikov_2016} pokazali, że jeśli osłabimy założenia: okręgi nie muszą zawierać się w~trójkącie i~wystarczy, że będą styczne do prostych zawierających boki trójkąta, to nadal ciąg okręgów jest od pewnego miejsca okresowy z okresem równym sześć, ale osiągnięcie tego stanu może wymagać dowolnie wielu kroków.
\index[persons]{Tabacznikow, Siergiej (Табачников, Сергей Львович)}%
\index[persons]{Iwanow, Denis (Иванов, Денис)}%

\begin{proposition}[o siedmiu okręgach]
	Dany jest okrąg $\Gamma$ oraz sześć okręgów stycznych do niego tak, że każdy jest też styczny do swoich dwóch sąsiadów.
	Wtedy trzy proste łączące przeciwległe punkty styczności przecinają się w jednym punkcie.
\end{proposition}

\begin{proposition}[o dziewięciu okręgach]
	Niech $\Gamma_1$, $\Gamma_2$, $\Gamma_3$ będą trzema okręgami na płaszczyźnie, zaś okrąg $\Gamma_4$ będzie styczny do $\Gamma_2$ i $\Gamma_3$.
	Kreślimy ciąg okręgów stycznych do poprzednich:
	$\Gamma_5$ do $\Gamma_1$, $\Gamma_3$ i $\Gamma_4$,
	$\Gamma_6$ do $\Gamma_1$, $\Gamma_2$ i $\Gamma_5$,
	$\Gamma_7$ do $\Gamma_2$, $\Gamma_3$ i $\Gamma_6$,
	$\Gamma_8$ do $\Gamma_1$, $\Gamma_3$ i $\Gamma_7$,
	$\Gamma_9$ do $\Gamma_1$, $\Gamma_2$ i $\Gamma_8$.
	Wtedy przy dobrym wyborze, który okrąg styczny wziąć, okrąg $\Gamma_4$ jest też styczny do $\Gamma_2$, $\Gamma_3$ i $\Gamma_9$.
\end{proposition}
