%

\begin{problem}[zadanie Napoleona]
\index{zadanie!Napoleona}
	Dany jest okrąg oraz jego środek.
	Podzielić okrąg na cztery łuki równej miary korzystając z cyrkla, ale nie linijki.
\end{problem}
% PRZECZYTANO: https://en.wikipedia.org/wiki/Napoleon%27s_problem

Jak zawsze w przypadku Napoleona i geometrii, nie wiadomo, czy  wymyślił albo choć rozwiązał przedstawione wyżej zadanie konstrukcyjne.
\index[persons]{Bonaparte, Napoleon}%
W trudniejszej wersji okrąg nie ma środka i trzeba go najpierw znaleźć.
Rozwiązanie podają Bogdańska, Neugebauer \cite[s. 116]{neugebauer_2018} (z wykorzystaniem okręgów Torricelliego); 
u Audina \cite[s. 105]{audin_2003} jest to ćwiczenie. 
\index{okrąg Torricelliego}%

%