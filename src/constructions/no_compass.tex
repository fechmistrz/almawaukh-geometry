%

\begin{theorem}[Ponceleta-Steinera]
\label{thm:poncelet_steiner}%
\index{twierdzenie!Ponceleta-Steinera}%
    Jeśli dana konstrukcja jest wykonalna za pomocą cyrkla i linijki, to jest ona wykonalna za pomocą samej linijki, o ile dany jest na płaszczyźnie pewien okrąg wraz ze środkiem.
    \index{twierdzenie!Ponceleta-Steinera}%
\end{theorem}

Nazwa twierdzenia pochodzi od dwóch panów: Jeana Ponceleta, który postawi w 1822 roku hipotezę oraz Jakoba Steinera, który dostarczy brakujący dowód w 1833 roku.
\index[persons]{Poncelet, Jean}%
\index[persons]{Steiner, Jakob}%
Zamiast całego okręgu, wystarczy, że mamy do dyspozycji mały jego łuk -- to wzmocnienie, o które pokusi się Francesco Severi w 1904 roku.
\index[persons]{Severi, Francesco}

Sama linijka to za mało, bo nie pozwala na wyciąganie pierwiastków kwadratowych.
Ale zamiast środka okręgu możemy wymagać:
\begin{itemize}
    \item równoległoboku,
    \item drugiego koncentrycznego okręgu,
    \item drugiego okręgu przecinającego pierwszy,
    \item drugiego okręgu rozłącznego z pierwszym i punktu na prostej łączącej ich środki albo ich osi potęgowej...
\end{itemize}

Zapewne istnieje jeszcze więcej takich warunków; o części napisze Eves \cite[s. 174-182]{eves1_1972}.
August Adler wspomniany we wcześniejszej sekcji zauważy, że linijka z dwoma brzegami albo ekierka pozwalają wykonać wszystkie klasyczne konstrukcje.

Dwie trójkątne linijki (w kształcie litery L) pozwalają podwoić sześcian (problem \ref{problem_delijski}) i dzielić kąt na trzy równe (problem \ref{trysekcja_kata}).

\begin{problem}
    Dany jest okrąg $\Gamma$ ze środkiem $O$ oraz odcinek.
    Skonstruować środek odcinka. \hfill \emph{(15 kroków)}. % Hartshorne s. 192
\end{problem}

\begin{problem}
    Dany jest okrąg $\Gamma$ ze środkiem $O$, prosta $l$ oraz punkt $P$.
    Skonstruować prostą równoległą do $l$, która przechodzi przez $P$. \hfill \emph{(16 kroków)}. % Hartshorne s. 192
\end{problem}

\begin{problem}
    Dany jest okrąg $\Gamma$ ze środkiem $O$, odcinek $OA$ oraz półprosta $OB$.
    Skonstruować punkt $C$ na półprostej taki, że $OA$ i $OC$ są przystające. \hfill \emph{(17 kroków)}. % Hartshorne s. 193
\end{problem}

\begin{problem}
    Dany jest okrąg $\Gamma$ ze środkiem $O$, prosta $l$ oraz punkt $P$.
    Skonstruować prostą prostopadłą do $l$, która przechodzi przez $P$. \hfill \emph{(33 kroki)}. % Hartshorne s. 193
\end{problem}

Zetel \cite[s. 17]{zetel_2020} poda wariację tego problemu: jego prosta $l$ jest średnicą okręgu.
Premier Rosji Michaił Miszustin odwiedzi we wrześniu 2021 roku liceum imienia P. L. Kapicy w Dołgoprudnym i~poprosi uczniów o~rozwiązanie jeszcze prostszego zadania, gdzie punkt $P$ leży na okręgu $\Gamma$. % https://www.theguardian.com/science/2021/sep/20/did-you-solve-it-russias-prime-minister-sets-a-geometry-puzzle

\begin{problem}
    Dany jest okrąg $\Gamma$ ze środkiem $O$, prosta $l$ oraz dwa punkty $A$, $B$.
    Skonstruować punkt, gdzie prosta $l$ przecina okrąg o środku $A$ i promieniu $AB$. \hfill \emph{(54 kroki)}. % Hartshorne s. 193
\end{problem}

\begin{problem}
    Dany jest okrąg $\Gamma$ ze środkiem $O$ oraz punkt $A$ poza okręgiem.
    Skonstruować styczne do $\Gamma$ przechodzące przez $A$.
\end{problem}

\begin{problem}
    Dany jest okrąg $\Gamma$ ze środkiem $O$.
    Skonstruować pięciokąt foremny wpisany w $\Gamma$. \hfill \emph{(około 50 kroków)}. % Hartshorne s. 193
\end{problem}

%