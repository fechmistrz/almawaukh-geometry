\begin{theorem}[Ponceleta-Steinera]
    Jeśli dana konstrukcja jest wykonalna za pomocą cyrkla i linijki, to jest ona wykonalna za pomocą samej linijki, o ile dany jest na płaszczyźnie pewien okrąg wraz ze środkiem.
\end{theorem}
% Renaissance mathematician Lodovico Ferrari, a student of Gerolamo Cardano in a "mathematical challenge" against Niccolò Fontana Tartaglia was able to show that "all of Euclid" (that is, the straightedge and compass constructions in the first six books of Euclid's Elements) could be accomplished with a straightedge and rusty compass. Within ten years additional sets of solutions were obtained by Cardano, Tartaglia and Tartaglia's student Benedetti.[2] During the next century these solutions were generally forgotten until, in 1673, Georg Mohr published (anonymously and in Dutch) Euclidis Curiosi containing his own solutions. Mohr had only heard about the existence of the earlier results and this led him to work on the problem.[3]
% https://en.wikipedia.org/wiki/Poncelet%E2%80%93Steiner_theorem

Jean Poncelet postawił powyższe jako hipotezę w 1822 roku.
\index{Poncelet, Jean}%
Jakob Steiner przedstawił dowód w 1833 roku.
\index{Steiner, Jakob}
Sama linijka nie jest wystarczająca, bo nie pozwala na wyciąganie pierwiastków kwadratowych.
Francesco Severi wzmocnił to twierdzenie w 1904 roku: zamiast całego okręgu, wystarczy, że mamy do dyspozycji mały jego łuk.
\index{Severi, Francesco}

Dodatkowo, zamiast środka okręgu możemy wymagać drugiego koncentrycznego okręgu, drugiego okręgu przecinającego pierwszy, drugiego okręgu rozłącznego z pierwszym i punktu na prostej łączącej ich środki albo ich osi potęgowej.
Zapewne istnieje jeszcze więcej takich warunków.

\begin{problem}
    Dany jest okrąg $\Gamma$ ze środkiem $O$ oraz odcinek.
    Skonstruować środek odcinka. \hfill \emph{(15 kroków)}. % Hartshorne s. 192
\end{problem}

\begin{problem}
    Dany jest okrąg $\Gamma$ ze środkiem $O$, prosta $l$ oraz punkt $P$.
    Skonstruować prostą równoległą do $l$, która przechodzi przez $P$. \hfill \emph{(16 kroków)}. % Hartshorne s. 192
\end{problem}

\begin{problem}
    Dany jest okrąg $\Gamma$ ze środkiem $O$, odcinek $OA$ oraz półprosta zaczynająca się w $O$.
    Skonstruować punkt $B$ na półprostej taki, że $OA$ i $OB$ są przystające. \hfill \emph{(17 kroków)}. % Hartshorne s. 193
\end{problem}

\begin{problem}
    Dany jest okrąg $\Gamma$ ze środkiem $O$, prosta $l$ oraz punkt $P$.
    Skonstruować prostą prostopadłą do $l$, która przechodzi przez $P$. \hfill \emph{(33 kroki)}. % Hartshorne s. 193
\end{problem}

\begin{problem}
    Dany jest okrąg $\Gamma$ ze środkiem $O$, prosta $l$ oraz dwa punkty $A$, $B$.
    Skonstruować punkt, gdzie prosta $l$ przecina okrąg o środku $A$ i promieniu $AB$. \hfill \emph{(54 kroki)}. % Hartshorne s. 193
\end{problem}

\begin{problem}
    Dany jest okrąg $\Gamma$ ze środkiem $O$.
    Skonstruować pięciokąt foremny wpisany w $\Gamma$. \hfill \emph{(około 50 kroków)}. % Hartshorne s. 193
\end{problem}

\todofoot{konstrukcja stycznej do okręgu samą linijką}