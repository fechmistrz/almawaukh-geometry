%

Coxeter \cite[s. 95]{coxeter_1967} sugeruje skonstruować (samym) cyrklem wierzchołki sześciokąta foremnego, odcinek dwa razy dłuższy (i krótszy) od danego, obraz inwersyjny danego punktu leżącego dowolnie blisko środka i wreszcie podzielić dany odcinek na $n$ równych części.

Okazuje się, że do żadnej klasycznej konstrukcji nie jest potrzebna linijka.

\begin{theorem}[Mohra-Mascheroniego]
\index{twierdzenie!Mohra-Mascheroniego}%
    Jeśli dana konstrukcja jest wykonalna za pomocą cyrkla i~linijki, to jest ona wykonalna za pomocą samego cyrkla, o ile pominiemy rysowanie linii i ograniczymy się do wyznaczania punktów konstrukcji.
\end{theorem}
% PRZECZYTANO: https://en.wikipedia.org/wiki/Mohr-Mascheroni_theorem

Wynik będzie mieć ciekawą historię.
Pierwszy znajdzie go Georg Mohr, włoski geometra i~poeta w \emph{Euclides Danicus} (1672), ale jego odkrycie popadnie w zapomnienie, po czym będzie tak marnieć aż do roku 1928! % established his results by using the idea of reflection in a line.
% Euclides Danicus = duński Euklides
\index[persons]{Mohr, Georg}%
Niezależnie twierdzenie odkryje Lorenzo Mascheroni; wykorzysta odbicia względem prostych i~opisze je w~\emph{La Geometria del Compasso} (1797).
\index[persons]{Mascheroni, Lorenzo}%
W 1890 jeszcze raz to samo, ale nie tak samo (inwersjami) zrobi śląski\footnote{Urodzony w Księstwie Górnego i Dolnego Śląska, znanym także jako Śląsk Austriacki} matematyk August Adler w pracy, której tytuł nadgryzie ząb czasu.
% August Adler (24 January 1863, Opava, Austrian Silesia – 17 October 1923, Vienna) was a Czech and Austrian mathematician noted for using the theory of inversion to provide an alternate proof of Mascheroni's compass and straightedge construction theorem. Czyli nie był z niczego innego znany?
Wreszcie w 1928 student Hjelmsleva przeglądając księgarnię w Kopenhadze i jej zawartość znajdzie książkę Mohra.
% źródło tej rewelacji: George Martin - Geometric Constructions (1998), s. 54
Będzie to duża niespodzianka dla Hjelmsleva.

Dla dowodu wystarczy pokazać, że cyrkel wystarcza do przeprowadzenia pięciu konstrukcji:
\begin{enumerate}
    \item poprowadzenia prostej przez dwa punkty,
    \item wykreślenia okręgu o danym środku i promieniu,
    \item znalezienia punktu przecięcia dwóch nierównoległych prostych,
    \item znalezienia punktu (lub punktów) przecięcia prostej z okręgiem,
    \item znalezienia punktu (lub punktów) przecięcia dwóch okręgów.
\end{enumerate}

O wyniku Mohra-Mascheroniego napisze Eves \cite[s. 169-172]{eves1_1972}.

%