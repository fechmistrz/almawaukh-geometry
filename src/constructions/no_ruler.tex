%

\begin{theorem}[Mohra-Mascheroniego]
\index{twierdzenie!Mohra-Mascheroniego}%
    Jeśli dana konstrukcja jest wykonalna za pomocą cyrkla i linijki, to jest ona wykonalna za pomocą samego cyrkla, o ile pominiemy rysowanie linii i ograniczymy się do wyznaczania punktów konstrukcji.
\end{theorem}
% PRZECZYTANO: https://en.wikipedia.org/wiki/Mohr-Mascheroni_theorem

Wynik ma ciekawą historię.
Pierwszy znajdzie go Georg Mohr, włoski geometra i poeta w \emph{Euclides Danicus} (1672), ale jego odkrycie popadnie w zapomnienie i będzie tak marnieć aż do roku 1928!
Mohr wykorzysta odbicia względem prostych. % established his results by using the idea of reflection in a line. In 1890, the
Niezależnie twierdzenie odkryje Lorenzo Mascheroniego i~opisze je w~\emph{La Geometria del Compasso} (1797).
W 1890 jeszcze raz to samo, ale nie tak samo (inwersjami) zrobi wiedeński August Adler w pracy, której tytuł nadgryzie ząb czasu.
Wreszcie w 1928 student Hjelmsleva przeglądając księgarnię w Kopenhadze i jej zawartość znajdzie książkę Mohra.
Będzie to duża niespodzianka dla Hjelmsleva!

Dla dowodu wystarczy pokazać, że cyrkel wystarcza do przeprowadzenia pięciu konstrukcji:
\begin{enumerate}
    \item poprowadzenia prostej przez dwa punkty,
    \item wykreślenia okręgu o danym środku i promieniu,
    \item znalezienia punktu przecięcia dwóch nierównoległych prostych,
    \item znalezienia punktu (lub punktów) przecięcia prostej z okręgiem,
    \item znalezienia punktu (lub punktów) przecięcia dwóch okręgów.
\end{enumerate}

Coxeter \cite[s. 95]{coxeter_1967} sugeruje skonstruować cyrklem wierzchołki sześciokąta foremnego, odcinek dwa razy dłuższy (i krótszy) od danego, obraz inwersyjny danego punktu leżącego dowolnie blisko środka i wreszcie podzielić dany odcinek na $n$ równych części.

%