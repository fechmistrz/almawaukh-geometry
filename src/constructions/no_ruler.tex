%

\begin{theorem}[Mohra-Mascheroniego]
    Jeśli dana konstrukcja jest wykonalna za pomocą cyrkla i linijki, to jest ona wykonalna za pomocą samego cyrkla, o ile pominiemy rysowanie linii i ograniczymy się do wyznaczania punktów konstrukcji.
    \index{twierdzenie!Mohra-Mascheroniego}
\end{theorem}
% PRZECZYTANO: https://en.wikipedia.org/wiki/Mohr-Mascheroni_theorem

Wynik ten został opublikowany przez Georga Mohra w \emph{,,Euclides Danicus''} (1672), był jednak nieznany aż do roku 1928. %  languished in obscurity
Niezależnie twierdzenie zostało odkryte przez Lorenzo Mascheroniego w~\emph{,,La Geometria del Compasso''} (1797).
\index[persons]{Mohr, Georg}%
\index[persons]{Mascheroni, Lorenzo}%

Dla dowodu wystarczy pokazać, że cyrkel wystarcza do przeprowadzenia pięciu konstrukcji:
\begin{enumerate}
    \item poprowadzenia prostej przez dwa punkty,
    \item wykreślenia okręgu o danym środku i promieniu,
    \item znalezienia punktu przecięcia dwóch nierównoległych prostych,
    \item znalezienia punktu (lub punktów) przecięcia prostej z okręgiem,
    \item znalezienia punktu (lub punktów) przecięcia dwóch okręgów.
\end{enumerate}

%