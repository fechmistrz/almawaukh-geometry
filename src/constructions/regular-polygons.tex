
\subsection{Wielokąty foremne}
Euklides potrafił skonstruować wielokąt foremny o $n$ bokach dla $n = 3$ (I.1), $n = 4$ (I.46), $n = 5$ (IV.11), $n = 6$ (IV.15) oraz $n = 15$ (IV.16).
Ptolemeusz podał konstrukcję pięciokąta foremnego w Almageście.

Mając wielokąt o $n$ bokach, bisekcja kąta środkowego pozwala podwoić liczbę boków do $2n$.
Możemy więc uznać, że Euklides potrafił wykreślić wielokąty, które miały
\begin{equation}
    3, 4, 5, 6, 8, 10, 12, 15, 16, 20, \ldots
\end{equation}
boków.
Lista została rozszerzona dopiero w 1796 roku przez Gaussa: pokazał, że siedemnastokąt foremny jest konstruowalny.
Był tak zadowolony z tego wyniku, że zażyczył sobie wyryć właśnie tę figurę na swoim grobie.
Faktycznej konstrukcji dokonał Johannes Erchinger w 1800 roku.
\index[persons]{Erchinger, Johannes}%
Hartstone \cite[s. 250-259]{hartshorne2000} poświęca tej konstrukcji całą sekcję 29.

W 1832 roku Friederich Richelot skonstruował $257$-kąt.
\index[persons]{Richelot, Friederich}
Jego cierpliwość przyćmiewa konstrukcja $65\,537$-kąta Johanna Hermesa z 1896 roku, na którą poświęcił dziesięć lat swojego życia (i dwieście stron rękopisu).
% The construction is very complex; Hermes spent 10 years completing the 200-page manuscript

W ogólności, mamy:
\begin{theorem}[Gaussa-Wantzla]
    Wielokąt foremny o $n$ bokach jest konstruowalny przy pomocy cyrkla i linijki wtedy i tylko wtedy, kiedy $n$ jest postaci
    \begin{equation}
        n = 2^r p_1 \cdot \ldots \cdot p_s,
    \end{equation}
    $r, s \ge 0$, gdzie $p_i$ są różnymi liczbami pierwszymi postaci $2^{2^k} + 1$, na przykład: $3$, $5$, $17$, $257$, $65537$
\end{theorem}

Uzasadnienie można znaleźć u Hartstone'a \cite[s. 258]{hartshorne2000}.
Gauss znalazł warunek wystarczający i napisał bez dowodu w \emph{Disquisitiones Arithmeticae} (1801), że jest też konieczny.
Brakujący dowód został dodany przez Pierre'a Wantzla w 1837 roku.

Carl Friedrich Gauss proved the constructibility of the regular 17-gon in 1796. Five years later, he developed the theory of Gaussian periods in his . This theory allowed him to formulate a sufficient condition for the constructibility of regular polygons. Gauss stated without proof that this condition was also necessary,[2] but never published his proof.

% TODO: Coxeter, s. 26




GUZICKI-12 **wielokąty foremne** które są konstruowalne? (tw. wantzla itd.) konstrukcje przybliżone pięciokąta - durer i da vinci.
$n = 3$, $n = 4$, $n = 6$ (proste)

\begin{problem}
    Skonstrować trójkąt równoboczny wpisany w okrąg, którego środek nie jest znany. \hfill \emph{(7 kroków)}
\end{problem}

\begin{problem}
    Skonstrować kwadrat. \hfill \emph{(9 kroków)}
\end{problem}

\begin{problem}
    Skonstrować pięciokąt foremny.
\end{problem}

Piszą o tym Hartshorne \cite[s. 45-49]{hartshorne2000}.
Jeśli mamy zadany jeden z jego boków, konstrukcja wymaga 11 kroków. % Hartshorne s. 51




$n = 17$

$n = 7$ (niemożliwe), możliwe ze znaczoną linijką: Hartshorne rozdział 30/31
