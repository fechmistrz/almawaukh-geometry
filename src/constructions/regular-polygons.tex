\begin{problem}
    Skonstruować wielokąt foremny o $n$ kątach.
\end{problem}

Konstrukcję trójkąta równobocznego oraz kwadratu była znana dobrze w starożytności i nie wiadomo, kto pierwszy na nią wpadł.
Euklides potrafił skonstruować wielokąt foremny o $n$ bokach dla $n = 3$ (I.1), $n = 4$ (I.46), $n = 5$ (IV.11), $n = 6$ (IV.15) oraz $n = 15$ (IV.16).
\index[persons]{Euklides}
Ptolemeusz podał inną konstrukcję pięciokąta foremnego w Almageście.
\index[persons]{Ptolemeusz}

Mając wielokąt o $n$ bokach, bisekcja kąta środkowego pozwala podwoić liczbę boków do $2n$.
Możemy więc uznać, że Euklides potrafił wykreślić wielokąty, które miały
\begin{equation}
    3, 4, 5, 6, 8, 10, 12, 15, 16, 20, \ldots
\end{equation}
boków.
Lista została rozszerzona dopiero w 1796 roku przez Gaussa: pokazał, że siedemnastokąt foremny jest konstruowalny.
\index{Gauss, Carl Friedrich}%
\index{wielokąt!siedemnastokąt}%
Był tak zadowolony z tego wyniku, że zażyczył sobie wyryć właśnie tę figurę na swoim grobie.
Faktycznej konstrukcji dokonał Johannes Erchinger w 1800 roku.
\index[persons]{Erchinger, Johannes}%
Hartstone \cite[s. 250-259]{hartshorne2000} poświęca tej konstrukcji całą sekcję.

W ogólności, mamy:
\begin{theorem}[Gaussa-Wantzla]
    \label{gauss_wantzel}
    Wielokąt foremny o $n$ bokach jest konstruowalny przy pomocy cyrkla i~linijki bez podziałki wtedy i tylko wtedy, kiedy $n$ jest postaci
    \begin{equation}
        n = 2^r p_1 \cdot \ldots \cdot p_s,
    \end{equation}
    $r, s \ge 0$, gdzie $p_i$ są różnymi liczbami pierwszymi postaci $2^{2^k} + 1$, na przykład: $3$, $5$, $17$, $257$, $65537$.
    \index{liczba pierwsza!Fermata}%
\end{theorem}

Gauss znalazł warunek wystarczający i napisał bez dowodu w \emph{Disquisitiones Arithmeticae} (1801), że jest też konieczny.
Brakujący dowód został dodany przez Pierre'a Wantzla w 1837 roku.
\index[persons]{Wantzel, Pierre}
Spisany współczesnym językiem znajdzie się u Hartstone'a \cite[s. 258]{hartshorne2000}.

Twierdzenie Gaussa zachęcało do znalezienia przepisu na wielokąty foremne o większej liczbie boków.
W 1822 roku Magnus Paucker \cite{paucker_1822}, a dziesięć lat później i prawie pewne, że niezależnie także Friederich Richelot \cite{richelot_1832} znaleźli konstrukcję $257$-kąta foremnego. 
\index[persons]{Paucker, Magnus}%
\index[persons]{Richelot, Friederich}%
Ich cierpliwość przyćmiewa konstrukcja $65\,537$-kąta Johanna Hermesa z 1896 roku, na którą poświęcił dziesięć lat swojego życia (i dwieście stron rękopisu).

Pełna lista konstruowalnych wielokątów foremnych o nieparzyście wielu bokach to: 3, 5, 15 (znane starożytnym), 17, 51, 85, 255 (znane Gaussowi), 257, 771, 1 285, 3 855, 4 369, 13 107, 21 845, 65 535 (znane Pauckerowi/Richelotowi), 65 537, 196 611, 327 685, 983 055, 1 114 129, 3 342 387, 5 570 645, 16 711 935, 16 843 009, 50 529 027, 84 215 045, 252 645 135, 286 331 153, 858 993 459, 1 431 655 765, 4 294 967 295; zakładamy przy tym, że istnieje tylko pięć liczb pierwszych Fermata.

\todofoot{Coxeter, s. 26}

\begin{problem}
    Skonstrować trójkąt równoboczny wpisany w okrąg, którego środek nie jest znany. \hfill \emph{(7 kroków)}
\end{problem}

\begin{problem}
    Skonstrować kwadrat. \hfill \emph{(9 kroków)}
\end{problem}

\begin{problem}
    Skonstrować pięciokąt foremny.
\end{problem}

Piszą o tym Hartshorne \cite[s. 45-49]{hartshorne2000}.
Jeśli mamy zadany jeden z jego boków, konstrukcja wymaga 11 kroków. % Hartshorne s. 51

\subsection{Konstrukcje przybliżone}
GUZICKI-12 pięciokąta - durer i da vinci.

\subsection{Konstrukcje z niestandardowymi przyborami}
Jeżeli mamy do dyspozycji trysektor kąta, potrafimy zbudować więcej wielokątów foremnych niż sugeruje to twierdzenie \ref{gauss_wantzel}:
\index{trysekcja!kąta}
\index{trysektor kąta}

\begin{proposition}
    Wielokąt foremny o $n$ bokach jest konstruowalny przy pomocy cyrkla, linijki bez podziałki oraz trysektora kąta wtedy i tylko wtedy, kiedy $n$ jest postaci
    \begin{equation}
        n = 2^r 3^q p_1 \cdot \ldots \cdot p_s,
    \end{equation}
    $q, r, s \ge 0$, gdzie $p_i$ są różnymi liczbami pierwszymi Pierponta, postaci $2^t3^u + 1$.
    \index{liczba pierwsza!Pierponta}
\end{proposition}

To są dokładnie wielokąty, które można skonstruować używając stożkowych, albo równoważnie origami.
Lista nowych nieparzystych liczb $n$, który dostarcza ten fakt, to \textbf{7}, $9$, \textbf{13}, \textbf{19}, $21$, $27$, $35$, \textbf{37}, $39$, $45$, $57$, $63$, $65$, \textbf{73}, $81$, $91$, $95$, \textbf{97}, $105$, \textbf{109}, $111$, $117$, $119$, $133$, $135$, $153$, \textbf{163}, $171$, $185$, $189$, \textbf{193}, $195$, $219$, $221$, $243$, $247$, \textbf{257}, $259$, $273$, $285$, $291\ldots$ (pogrubiono liczby pierwsze).
% gdzie 25???

Szczegóły zapewnia artykuł Gleasona \cite{gleason_1988}.
\index[persons]{Gleason, Andrew}%
Konstrukcję siedmiokąta przy użyciu znaczonej linijki opisuje Hartshorne \todofoot{Hartshorne rozdział 30/31.}.
\index{linijka!znaczona}