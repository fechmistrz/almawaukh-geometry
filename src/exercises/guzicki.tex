\textbf{Zadanie} (Guzicki, s. 304).
Na bokach $AB$, $BC$, $CD$ i $DA$ czworokąta wypukłego $ABCD$ zbudowano, na zewnątrz czworokąta, kwadraty $ABFE$, $BCHG$, $CDJI$ i $DALK$.
Punkty $P$, $Q$, $R$ i $S$ są odpowiednio środkami kwadratów $ABFE$, $BCHG$, $CDJI$ i $DALK$.
Udowodnij, że odcinki $PR$ i $QS$ są równej długości oraz wzajemnie prostopadłe.

\textbf{Zadanie} (Guzicki, s. 306).
(XLIV OM, zadanie 5/I).
Dana jest półpłaszczyzna oraz punkty $A$ i $C$ na jej krawędzi.
Dla każdego punktu $B$ tej półpłaszczyzny rozważamy kwadraty $ABKL$ i $BCMN$ leżące na zewnątrz trójkąta $ABC$.
Wyznaczają one odpowiadającą punktowi $B$ prostą $LM$.
Udowodnij, że wszystkie proste odpowiadające różnym położeniom punktu $B$ przechodzą przez jeden punkt.

\textbf{Zadanie} (Guzicki, s. 306).
Na bokach $AB$ i $AC$ trójkąta $ABC$ zbudowano, po jego zewnętrznej stronie, kwadraty $ABDE$ i $ACFG$.
Punkty $M$ i $N$ są odpowiednio środkami odcinków $DG$ i $EF$.
Wyznacz możliwe wartości wyrażenia $MN / BC$.

\textbf{Zadanie} (Guzicki, s. 307)
(TWIERDZENIE NAPOLEONA)
Na bokach $AB$, $BC$ i $CA$ trójkąta $ABC$ zbudowano, na zewnątrz trójkąta, trójkąty równoboczne $ABF$, $BCD$ i $CAE$.
Udowodnij, że środki tych trójkątów równobocznych są wierzchołkami trójkąta równobocznego.

\textbf{Zadanie} (Guzicki, s. 308)
Na bokach $AB$, $BC$ i $CA$ trójkąta $ABC$ wybrano odpowiednio punkty $D$, $E$ i $F$ tak, że $AD : DB = BE : EC = CF : FA$.
Udowodnij, że jeśli trójkąt $DEF$ jest równoboczny, to trójkąt $ABC$ też jest równoboczny.

\textbf{Zadanie} (Guzicki, s. 310)
(XLV OM, zadanie 7/I)
Na zewnątrz czworokąta wypukłego $ABCD$ budujemy trójkąty podobone $APB$, $BQC$, $CRD$, $DSA$ w ten sposób, że kąty $PAB, QBC, RCD, SDA$ są sobie równe i że kąty $PBA, QCB, RDS, SAD$ też są sobie równe.
Udowodnij, że jeśli czworokąt PQRS jest równoległobokiem, to czworokąt $ABCD$ też jest równoległobokiem.