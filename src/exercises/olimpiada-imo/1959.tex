%

\begin{exercise}[4/1959]
    Construct a right triangle with given hypotenuse $c$ such that the median drawn to the hypotenuse is the geometric mean of the two legs of the triangle.
\end{exercise}

\begin{exercise}[5/1959]
    An arbitrary point $M$ is selected in the interior of the segment $AB$.
    The squares $AMCD$ and $MBEF$ are constructed on the same side of $AB$, with the segments $AM$ and $MB$ as their respective bases.
    The circles circumscribed about these squares, with centers $P$ and $Q$, intersect at $M$ and also at another point $N$.
    Let $N'$ denote the point of intersection of the straight lines $AF$ and $BC$.
    Prove that the points $N$ and $N'$ coincide.
    Prove that the straight lines $MN$ pass through a fixed point $S$ independent of the choice of $M$.
    Find the locus of the midpoints of the segments $PQ$ as $M$ varies between $A$ and $B$.
\end{exercise}

\begin{exercise}[6/1959]
    Two planes, $P$ and $Q$, intersect along the line $p$.
    The point $A$ is given in the plane $P$, and the point $C$ in the plane $Q$;
    neither of these points lies on the straight line $p$.
    Construct an isosceles trapezoid $ABCD$ (with $AB$ parallel to $CD$) in which a circle can be inscribed, and with vertices $B$ and $D$ lying in the planes $P$ and $Q$ respectively.
\end{exercise}

%