
\begin{exercise}[3/2014]
    W czworokącie wypukłym $ABCD$ zachodzą równości $\angle ABC = \angle CDA = \pi/2$.
    Punkt $H$ jest rzutem prostopadłym punktu $A$ na prostą $BD$.
    Punkty $S$ i $T$ leżą odpowiednio na bokach $AB$ i $AD$, przy czym $H$ leży wewnątrz trójkąta $SCT$ oraz $\angle CHS - \angle CSV = \pi/2$, $\angle THC - \angle DTC = \pi/2$.
    Udowodnić, że prosta $BD$ jest styczna do okręgu opisanego na trójkącie $TSH$.
\end{exercise}

\begin{exercise}[4/2014]
    Punkty $P$ i $Q$ leżą na boku $BC$ trójkąta ostrokątnego $ABC$, przy czym $\angle PAB = \angle BCA$ oraz $\angle CAQ = \angle ABC$.
    Punkty $M$ i $N$ leżą odpowiednio na prostych $AP$ i $AQ$, przy czym $P$ jest środkiem odcinka $AM$ oraz $Q$ jest środkiem odcinka $AN$.
    Udowodnić, że proste $BM$ i $CN$ przecinają się w punkcie leżącym na okręgu opisanym na trójkącie $ABC$.
\end{exercise}
