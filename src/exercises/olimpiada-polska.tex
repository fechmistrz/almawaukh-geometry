Polska olimpiada matematyczna.

\subsection{LX Olimpiada w roku 2008/2009}
\begin{exercise}[4/2024]
    Dany jest trójkąt $ABC$, w którym $AB < AC < BC$.
    Niech $\omega$ będzie okręgiem wpisanym w trójkąt $ABC$ o środku $I$.
    Niech $X$ będzie takim punktem na prostej $BC$ różnym od $C$, że prosta przechodząca przez $X$ i równoległa do $AC$ jest styczna do $\omega$.
    Podobnie, niech $Y$ będzie takim punktem na prostej $BC$ różnym od $B$, że prosta przechodząca przez $Y$ i równoległa do $AB$ jest styczna do $\omega$.
    Niech prosta $AI$ przecina okrąg opisany na trójkącie $ABC$ ponownie w punkcie $P \neq A$.
    Niech $K$ i $L$ będą odpowiednio środkami odcinków $AC$ i $AB$.
    Dowieść, że $\angle KIL + \angle YPX = \pi$.
\end{exercise}

\textbf{Zadanie} (Guzicki, s. 310)
(XLV OM, zadanie 7/I)
Na zewnątrz czworokąta wypukłego $ABCD$ budujemy trójkąty podobone $APB$, $BQC$, $CRD$, $DSA$ w ten sposób, że kąty $PAB, QBC, RCD, SDA$ są sobie równe i że kąty $PBA, QCB, RDS, SAD$ też są sobie równe.
Udowodnij, że jeśli czworokąt PQRS jest równoległobokiem, to czworokąt $ABCD$ też jest równoległobokiem.

\textbf{Zadanie} (Guzicki, s. 306).
(XLIV OM, zadanie 5/I).
Dana jest półpłaszczyzna oraz punkty $A$ i $C$ na jej krawędzi.
Dla każdego punktu $B$ tej półpłaszczyzny rozważamy kwadraty $ABKL$ i $BCMN$ leżące na zewnątrz trójkąta $ABC$.
Wyznaczają one odpowiadającą punktowi $B$ prostą $LM$.
Udowodnij, że wszystkie proste odpowiadające różnym położeniom punktu $B$ przechodzą przez jeden punkt.