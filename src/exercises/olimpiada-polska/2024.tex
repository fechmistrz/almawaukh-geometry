
\subsubsection{Etap 1, zadanie 3}
Okrąg wpisany w trójkąt $ABC$ jest styczny do boków $BC$, $CA$, $AB$ w~punktach $D$, $E$, $F$.
Punkty $M$, $N$, $J$ są środkami okręgów wpisanych odpowiednio w trójkąty $AEF$, $BDF$, $DEF$.
Dowieść, że punkty $F$ i $J$ są symetryczne względem prostej $MN$.

\subsubsection{Etap 1, zadanie 6}
Dany jest trójkąt $ABC$, w którym $AB = AC$. Na półprostych $AB$ i $AC$ obrano odpowiednio takie punkty $K$ i $L$ leżące poza bokami trójkąta, że $4 \cdot BK \cdot CL = BC^2$.
Punkt $M$ jest środkiem boku $BC$.
Proste $KM$ i $LM$ przecinają po raz drugi okrąg opisany na trójkącie $AKL$ odpowiednio w punktach $P$ i $Q$.
Wykazać, że proste $PQ$ i $BC$ są równoległe.

\subsubsection{Etap 1, zadanie 8}
Przekątne podstawy $ABCD$ ostrosłupa $ABCDS$ przecinają się pod kątem prostym w punkcie $H$, będącym spodkiem wysokości ostrosłupa.
Niech $K$, $L$, $M$, $N$ będą rzutami prostokątnymi punktu $H$ odpowiednio na ściany $ABS$, $BCS$, $CDS$, $DAS$.
Dowieść, że proste $KL$, $MN$ i $AC$ są równoległe lub przecinają się w jednym punkcie.

\subsubsection{Etap 1, zadanie 10}
Punkt $P$ jest środkiem krótszego łuku $BC$ okręgu opisanego na trójkącie $ABC$, w którym $\angle BAC = \pi/3$.
Punkt $M$ jest środkiem odcinka łączącego środki dwóch okręgów dopisanych do danego trójkąta, stycznych odpowiednio do boków $AB$ i $AC$. Wykazać, że $PM = 2 \cdot BP$.

\subsubsection{Etap 2, zadanie 3}
Rozłączne okręgi $o_1$ i $o_2$ o środkach odpowiednio $I_1$ i $I_2$ są styczne do prostej $k$ odpowiednio w punktach $A_1$ i $A_2$ oraz leżą po tej samej jej stronie.
Punkt $C$ leży na odcinku $I_1I_2$, przy czym $\angle A_1 C A_2 = \pi/2$.
Dla $i = 1, 2$ niech $B_i$ będzie punktem różnym od $A_i$, w~którym prosta $A_iC$ przecina okrąg $o_i$.
Dowieść, że prosta $B_1B_2$ jest styczna do okręgów $o_1$ i $o_2$.

\subsubsection{Etap 2, zadanie 4}
Odcinek $AB$ jest średnicą okręgu $o$ opisanego na czworokącie wypukłym $ABCD$, którego przekątne przecinają się w punkcie $E$. Proste styczne do okręgu $o$ w~punktach $C$ i~$D$ przecinają się w punkcie $P$.
Udowodnić, że $PC = PE$.

\subsubsection{Etap 3, zadanie 1}
Każdy z wierzchołków sześciokąta wypukłego jest środkiem koła o promieniu równym długości nie dłuższego z boków sześciokąta zawierających ten wierzchołek.
Udowodnić, że jeśli część wspólna wszystkich sześciu kół (rozważanych wraz z brzegiem) jest niepusta, to sześciokąt jest foremny.

\subsubsection{Etap 3, zadanie 5}
Sfera wpisana w czworościan $ABCD$ jest styczna do jego ścian $BCD$, $ACD$, $ABD$, $ABC$ odpowiednio w punktach $P$, $Q$, $R$, $S$.
Odcinek $PT$ jest średnicą tej sfery, zaś punkty $A'$, $Q'$, $R'$, $S'$ są punktami przecięcia prostych $TA$, $TQ$, $TR$, $TS$ z płaszczyzną $BCD$.
Wykazać, że $A'$ jest środkiem okręgu opisanego na trójkącie $Q'R'S'$.
