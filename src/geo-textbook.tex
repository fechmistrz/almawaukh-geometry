\documentclass{greaseproof}
\usepackage{lipsum}
\newcommand{\loremipsum}{ {\color{gray}  Lorem ipsum dolor sit amet, consectetur adipiscing elit. Suspendisse nisl purus, ultricies et ante quis, iaculis imperdiet justo. Proin tristique turpis a tortor eleifend lobortis. Vestibulum odio nisi, tempor sed scelerisque ac, aliquet a lorem. Proin ullamcorper nibh eget augue placerat lobortis. Quisque ac commodo libero, fringilla dictum purus. Proin nec massa vitae lorem eleifend lacinia. Praesent rhoncus ultricies ullamcorper. Fusce vel viverra purus, nec rhoncus massa. Etiam nisl lorem, cursus vel est sit amet, mattis venenatis magna. Vestibulum id risus sit amet nisi congue ultrices id nec ex. } }
\begin{document}

\input{blurbs/page-1-5}

\section{Aksjomatyka}
Tekst podsekcji Aksjomatyka. \loremipsum
%

\subsection{Aksjomaty Euklidesa}
\subsection{Księga I}	
\subsubsection{Definicje}	
\begin{enumerate}	
    \item [1.1] Definicja ... % Definicja 1. % Punkt to jest to, co nie składa się z części.
    \item [1.2] Definicja ... % Definicja 2. % Linia jest długością bez szerokości.
    \item [1.3] Definicja ... % Definicja 3. % Końcami linii są punkty.
    \item [1.4] Definicja ... % Definicja 4. % Linia jest prosta, jeżeli położona jest między swoimi punktami w równym i jednostajnym kierunku.
    \item [1.5] Definicja ... % Definicja 5. % Powierzchnia jest to, co ma tylko długość i szerokość.
    \item [1.6] Definicja ... % Definicja 6. % Krawędzie powierzchni są liniami.
    \item [1.7] Definicja ... % Definicja 7. % Płaska powierzchnia albo płaszczyzna jest ta, na której biorąc gdziekolwiek dwa punkty linia prosta między tymi punktami cała leży na tej powierzchni.
    \item [1.8] Definicja ... % Definicja 8. % Kąt płaski to nachylenie dwóch linii na płaszczyźnie w miejscu, w którym jedna spotyka drugą i nie leżą w linii prostej.
    \item [1.9] Definicja ... % Definicja 9. % Kiedy linie są proste i tworzą kąt, wtedy kąt zwany jest prostoliniowym.
    \item [1.10] Definicja ... % Definicja 10. % Kiedy linia prosta padająca na drugą linie prostą, tworzy z nią kąty przyległe równe między sobą, to każdy z kątów równych nazywamy prostym, a padająca linia prostą nazywa się prostopadłą do tej linii, na którą pada.
    \item [1.11] Definicja ... % Definicja 11. % Kąt rozwarty jest większy od kąta prostego.
    \item [1.12] Definicja ... % Definicja 12. % Kąt ostry jest mniejszy od kąta prostego.
    \item [1.13] Definicja ... % Definicja 13. % Kresem albo granicą jest to, na czym się dana rzecz kończy.
    \item [1.14] Definicja ... % Definicja 14. % Figurą nazywamy to co jest ograniczone granicą lub granicami.
    \item [1.15] Definicja ... % Definicja 15. % Koło jest figurą płaską zawarta linią zwaną okręgiem, do której wszystkie linie proste poprowadzone z jednego punktu wewnątrz figury położonego, są między sobą równe.
    \item [1.16] Definicja ... % Definicja 16. % I ten punkt nazywa się centrum lub środkiem koła.
    \item [1.17] Definicja ... % Definicja 17. % Średnicą koła jest każda linia narysowana przez środek koła, przedłużona w dwóch kierunkach do jego obwodu, przepoławiająca go.
    \item [1.18] Definicja ... % Definicja 18. % Półokręgiem jest figura zawarta między średnicą i częscia okręgu odciętą tą średnicą. Środek półokregu jest też środkiem okręgu.
    \item [1.19] Definicja ... % Definicja 19. % Figury prostokreślne to figury ograniczone prostymi. Trójkąt to figura prostokreślna ograniczona trzema prostymi. Czworobok lub czworokąt to figura prostokreślna, która jest ograniczona czterema prostymi. Wielobok lub wielokąt to figura prostokreślna ograniczona więcej niż czterema prostymi.
    \item [1.20] Definicja ... % Definicja 20. % Trójkąt równoboczny to trójkąt, który ma trzy boki równe. Trójkąt równoramienny to trójkąt, który ma tylko dwa boki równe. Trójkąt różnoboczny to trójkąt, który ma trzy boki różne.
    \item [1.21] Definicja ... % Definicja 21. % Ponadto: trójkąt prostokątny to trójkąt, który na kąt prosty. Trójkąt rozwartokątny to trójkąt, który ma kąt rozwarty. Trójkąt ostrokątny to trójkąt, który ma trzy kąty ostre.
    \item [1.22] Definicja ... % Definicja 22. % Kwadrat jest to czworobok mający równe boki i równe kąty. Prostokąt jest to czworobok mający kąty proste, ale boki nierówne. Romb (kwadrat ukośny) jest to czworobok mający równe boki, ale nie mający kątów prostych. Równoległobok jest to czworobok mający boki przeciwległe równe, ale nie mający katów prostych. Wszystkie czworoboki inne niż wyżej wymienione nazywamy czworokątami.
    \item [1.23] Definicja ... % Definicja 23. % Linie równoległe, czyli mówiąc krócej równoległe są to proste, które leżą na tej samej płaszczyźnie i przedłużone z obu stron w nieskończoność, z żadnej strony nie przetną się.
\end{enumerate}	
	
\subsubsection{Postulaty}	
\begin{enumerate}	
    \item [1.1] Przez każde dwa punkty przechodzi prosta.
    \item [1.2] Postulat ... % Postulat 2. % Ograniczoną prostą można przedłużyć nieskończenie.
    \item [1.3] Postulat ... % Postulat 3. % Można zakreślić okrąg z któregokolwiek punktu jako środka dowolną odległością.
    \item [1.4] Wszystkie kąty proste są sobie równe.
    \item [1.5] Postulat ... % Postulat 5. % Jeżeli prosta przecinająca dwie proste tworzy z nimi kąty jednostronnie wewnętrzne o sumie mniejszej niż dwa kąty proste, to te dwie proste przedłużone nieskończenie przecinają się po tej stronie, po której znajdują się kąty o sumie mniejszej od dwóch kątów prostych.
\end{enumerate}	
	
Jak łatwo zauważyć, sformułowanie ostatniego postulatu używa więcej słów niż pozostałe razem wzięte; wbrew przekonaniu, że postulaty miały wyrażać treści oczywiste i proste.	
Piąty postulat wydawał się bardziej skomplikowany, więc nasuwał podejrzenie, że wynika z poprzednich czterech.	
Zauważył to już Proklos zwany Diadochem (410-485):
\index[persons]{Proklos zwany Diadochem}%
\emph{,,Nie jest możliwe, aby uczony tej miary co Euklides godził się na obecność tak długiego postulatu w aksjomatyce -- obecność postulatu wzięła się z pospiesznego kończenia przez niego Elementów, tak aby zdążyć przed nadejściem słusznie oczekiwanej rychłej śmierci; my zatem -- czcząc jego pamięć -- powinniśmy ten postulat usunąć lub co najmniej znacznie uprościć.''}	
	
Wiele osób próbowało stawić czoło wyzwaniu postawionemu przez Proklosa.	
Bezskutecznie, ponieważ piąty postulat jest niezależny od pozostałych, zaś zastąpienie go jego zaprzeczeniem prowadzi do geometrii nieeuklidesowych.	
Piszą o tym Audin \cite[s. 13]{audin_2003}.
	
\subsubsection{Pojęcia pierwotne}	
\begin{enumerate}	
    \item [1.1] Wyrażenia, które są równe się temu samemu wyrażeniowi, są sobie równe.
    \item [1.2] Równania można dodawać stronami.
    \item [1.3] Równania można odejmować stronami.
    \item [1.4] Wyrażenia, które się pokrywają, są sobie równe.
    \item [1.5] Całość jest większa od części.
\end{enumerate}	
	
\subsubsection{Twierdzenia}	
\begin{enumerate}	
    \item [1.1] Skonstruować trójkąt równoboczny o zadanym boku.
    \item [1.2] Twierdzenie ... % Twierdzenie 2. % Skonstruuj odcinek równy danemu odcinkowi którego koniec jest zadanym punktem.
    \item [1.3] Skonstruować różnicę dwóch odcinków.
    \item [1.4] Twierdzenie ... \hfill \emph{(przystawanie bok-kąt-bok)} % Twierdzenie 4. % Jeśli dwa trójkąty mają dwa boki odpowiednio równe dwóm innym, i jeżeli kąty zawarte między bokami równoległymi są równe, wtedy ich podstawy również są sobie równe i pozostałe kąty równe są odpowiednim kątom.
    \index{cecha przystawania!bok-kąt-bok}%
    \item [1.5] Twierdzenie ... % Twierdzenie 5. % W trójkątach równoramiennych kąty przy podstawie są sobie równe oraz kąty powstałe przez przedłużenie boków równych są sobie równe.
    \item [1.6] Boki trójkąta leżące naprzeciw przystających kątów są przystające.
    \item [1.7] Twierdzenie ... % Twierdzenie 7. % Na tej samej podstawie i z tej samej strony nie mogą być wykreślone dwa trójkąty takie, żeby boki w tych trójkątach przy obydwu końcach wspólnej podstawy były między sobą równe.
    \item [1.8] Twierdzenie ... \hfill \emph{(przystawanie bok-bok-bok)} % Twierdzenie 8. % Jeżeli dwa boki jednego trójkąta są równe dwóm bokom drugiego trójkąta, to kąty zawarte między równymi bokami są sobie równe.
    \index{cecha przystawania!bok-bok-bok}%
    \item [1.9] Podzielić dany kąt na dwie równe części.
    \item [1.10] Podzielić dany odcinek na dwie równe części.
    \item [1.11] Twierdzenie ... % Twierdzenie 11. % Z punktu danego na danej linii prostej wyprowadzić linie prostopadłą do danej linii prostej.
    \item [1.12] Twierdzenie ... % Twierdzenie 12. % Z punktu danego leżącego poza linią prostą nieograniczoną, wyprowadzić prostą linię prostopadłą do niej.
    \item [1.13] Twierdzenie ... % Twierdzenie 13. % Jeżeli linia prosta przecinająca drugą prostą tworzy z nią dwa kąty, to są one proste, albo równe dwóm kątom prostym.
    \item [1.14] Twierdzenie ... % Twierdzenie 14. % Jeżeli przy linii prostej i przy punkcie na niej leżącym dwie linie proste nie po jednej stronie położone czynią kąty przyległe równe dwóm kątom prostym, to te linie proste będą w tym samym kierunku.
    \item [1.15] Twierdzenie ... % Twierdzenie 15. % Jeżeli dwie linie proste przecinają się, to utworzone przez nie kąty przeciwległe są sobie równe.
    \item [1.16] Twierdzenie ... % Twierdzenie 16. % W dowolnym trójkącie kąt zewnętrzny powstały przez przedłużenie jednego boku jest większy od każdego z dwóch kątów wewnętrznych przeciwległych jemu.
    \item [1.17] W każdym trójkącie suma dwóch kątów jest mniejsza od $\pi$.
    \item [1.18] Twierdzenie ... % Twierdzenie 18. % W każdym trójkącie bok większy przeciwległy jest kątowi większemu.
    \item [1.19] Twierdzenie ... % Twierdzenie 19. % W każdym trójkącie kąt większy przeciwległy jest bokowi większemu.
    \item [1.20] Twierdzenie ... % Twierdzenie 20. % W każdym trójkącie suma dwóch dowolnych boków jest większa od boku trzeciego.
    \item [1.21] Twierdzenie ... % Twierdzenie 21. % Jeżeli z końców jednego boku trójkąta poprowadzone będą dwie linie proste wewnątrz trójkąta, aż do zejścia się z sobą, to te dwie linie proste będą mniejsze od dwóch pozostałych boków trójkąta, lecz zawierać jednak będą kąt większy od kąta zawartego między pozostałymi bokami trójkąta.
    \item [1.22] Twierdzenie ... % Twierdzenie 22. % Aby z trzech danych linii prostych wykreślić trójkąt, potrzeba aby z tych trzech danych linii prostych suma dwóch którychkolwiek była większa od trzeciej.
    \item [1.23] Twierdzenie ... % Twierdzenie 23. % Na danej linii prostej i punkcie na niej danym wykreślić kąt prostokreślny równy kątowi prostokreślnemu danemu.
    \item [1.24] Twierdzenie ... % Twierdzenie 24. % Jeżeli dwa boki jednego trójkąta, są równe dwóm bokom trójkąta drugiego, z kątów zaś między bokami równymi jeden większy jest od drugiego; to będzie też podstawa jednego trójkąta większa od podstawy drugiego trójkąta.
    % TODO: https://en.wikipedia.org/wiki/Hinge_theorem
    \item [1.25] Twierdzenie ... % Twierdzenie 25. % Jeżeli dwa boki jednego trójkąta, są równe dwóm bokom trójkąta drugiego, lecz podstawa jednego trójkąta większa jest od podstawy drugiego trójkąta, to i kąty między bokami równymi zawarte będą jeden większy od drugiego.
    \item [1.26] Twierdzenie ... % Twierdzenie 26. % Jeżeli dwa kąty jednego trójkąta są równe dwóm kątom drugiego trójkąta, i bok jeden przyległy obydwu kątom, albo jednemu w pierwszym trójkącie równa się bokowi jednemu przyległemu obydwu katom, albo jednemu w drugim trójkącie; będą i dwa boki pozostałe równe dwóm bokom pozostałym i kąt trzeci w jednym trójkącie będzie równy katowi trzeciemu w drugim trójkącie.
    \item [1.27] Twierdzenie ... % Twierdzenie 27. % Jeżeli na dwie linie proste, pada linia prosta czyniąca kąty naprzemian równe między sobą, to te dwie linie proste będą równoległe.
    \item [1.28] Twierdzenie ... % Twierdzenie 28. % Jeśli linia prosta opada na dwie linie proste, tworząc kąt zewnętrzny równy wewnętrznemu i przeciwny do kąta na tym samym boku lub suma kątów wewnętrznych na tym samym boku jest równa dwóm kątom prostym, wtedy linie proste są równoległe do siebie.
    \item [1.29] Twierdzenie ... % Twierdzenie 29. % Linia prosta opada na równoległą linie prostą tworząc alternatywne kąty równe sobie, kąt zewnętrzny równy wewnętrznemu i przeciwległy i suma kątów wewnętrznych na tym samym boku jest równa dwóm kątom prostym.
    \item [1.30] Twierdzenie ... % Twierdzenie 30. % Linie proste, które są równoległe do linii prostej są również równoległe do siebie.
    \item [1.31] Twierdzenie ... % Twierdzenie 31. % Poprowadzić przez dany punkt linię prostą równoległą względem danej lini prostej.
    \item [1.32] Twierdzenie ... % Twierdzenie 32. % W jakimkolwiek trójkącie, jeśli jeden z boków jest znany wtedy kąt zewnętrzny jest równy sumie dwóch kątów wewnętrznych i przeciwnych i suma trzech wewnętrznych kątów trójkąta jest równa dwóm kątom prostym.
    \item [1.33] Twierdzenie ... % Twierdzenie 33. % Linie proste, które łączą końce równych i równoległych linii prostych w tym samym kierunku są sobie równe i równoległe.
    \item [1.34] Twierdzenie ... % Twierdzenie 34. % W równoległobokach boki i kąty przeciwne są między sobą równe, a przekątna dzieli je na dwie równe części.
    \item [1.35] Twierdzenie ... % Twierdzenie 35. % Równoległoboki, które są na takiej samej podstawie i są porównywalne są sobie równe.
    \item [1.36] Twierdzenie ... % Twierdzenie 36. % Równoległoboki, które mają równe podstawy i są porównywalne są sobie równe.
    \item [1.37] Twierdzenie ... % Twierdzenie 37. % Trójkąty, które mają takie same podstawy i są porównywalne są sobie równe.
    \item [1.38] Twierdzenie ... % Twierdzenie 38. % Trójkąty, których podstawy są równe i są one porównywalne są sobie równe.
    \item [1.39] Twierdzenie ... % Twierdzenie 39. % Równe trójkąty, które są na takich samych podstawach i mające te same boki również są porównywalne.
    \item [1.40] Twierdzenie ... % Twierdzenie 40. % Równe trójkąty, które mają takie same podstawy i mają te same boki również są porównywalne.
    \item [1.41] Twierdzenie ... % Twierdzenie 41. % Jeśli równoległobok i trójkąt mają tą samą podstawę i są tymi samymi liniami zakończone, to trójkąt jest połową równoległoboku.
    \item [1.42] Twierdzenie ... % Twierdzenie 42. % Skonstruować równoległobok równy danemu trójkątowi o podanym prostoliniowym kącie.
    \item [1.43] Twierdzenie ... % Twierdzenie 43. % W każdym równoległoboku, dopełnienia równoległoboków koło przekątnych położonych są między sobą równe.
    \item [1.44] Twierdzenie ... % Twierdzenie 44. % Na danej linii prostej wykreślić równy danemu równoległobok, którego jeden kąt będzie równy danemu.
    \item [1.45] Twierdzenie ... % Twierdzenie 45. % Wykreślić równy danej figurze prostokreślny równoległobok, którego jeden kąt będzie równy danemu.
    \item [1.46] Skonstruować kwadrat.
    \item [1.47] Twierdzenie ... \hfill \emph{(twierdzenie Pitagorasa)} % Twierdzenie 47. % W trójkącie prostokątnym, kwadrat zbudowany na boku przeciwnym kątowi prostemu, równy jest kwadratom zbudowanym na bokach, które kąt prosty zawierają.
    \index{twierdzenie!Pitagorasa}
    \item [1.48] Twierdzenie ... \hfill \emph{(twierdzenie odwrotne do twierdzenia Pitagorasa)} % Twierdzenie 48. % Jeżeli kwadrat zbudowany na jednym z boków trójkąta, jest równy kwadratom wykreślonym na dwóch pozostałych bokach trójkąta, to kąt zawarty między dwoma pozostałymi bokami będzie prosty.
\end{enumerate}

%
\input{axioms/axioms-euclid-2}
%

\subsection{Księga III}
\subsubsection{Definicje}
\begin{enumerate}
    \item [3.1] Dwa okręgi są przystające, kiedy mają równe średnice (lub równoważnie, promienie).
    \item [3.2] Definicja ...
    % Definicja 2. % Mówi się, że linia prosta dotyka koła, gdy będąc styczną z kołem przedłużona z obydwu stron nie przecina się z żadnej strony okręgu koła.
    \item [3.3] Dwa okręgi nazywamy stycznymi, kiedy mają dokładnie jeden punkt wspólny.
    \item [3.4] Definicja ...
    % Definicja 4. % Mówi się, że linie proste równoodległe są od środka koła, gdy prostopadłe ze środka koła na nie spuszczone są równe.
    \item [3.5] Definicja ...
    % Definicja 5. % Mówi się, że ta linia prosta bardziej jest odległa od środka koła, na którą prostopadła ze środka koła spuszczona jest większa.
    \item [3.6] Definicja ...
    % Definicja 6. % Odcinkiem koła jest figura czyli część koła ograniczona linią prostą i okręgiem koła.
    \item [3.7] Definicja ...
    % Definicja 7. % Kąt zaś odcinka jest ten, który się linią prostą i okręgiem koła zawiera.
    \item [3.8] Definicja ...
    % Definicja 8. % Jeżeli na okręgu koła wzięty będzie punkt i od niego będą poprowadzone linie proste do końców linii prostej za podstawę odcinkami służącej, kąt między tymi liniami prostymi zawarty jest kątem w odcinku.
    \item [3.9] Definicja ...
    % Definicja 9. % Kiedy zaś linie proste kąt zawierające zajmują część okręgu, mówi się, że kąt ten opiera się na okręgu koła.
    \item [3.10] Definicja ...
    % Definicja 10. % Jeżeli kąt ma swój wierzchołek we środku koła; figura czyli część koła zawarta między ramionami tegoż koła, to jest między promieniami i łukiem koła nazywa się wycinkiem koła.
    \item [3.11] Definicja ...
    % Definicja 11. % Odcinkami podobnymi kół nazywają się te, które zajmują kąty równe, lub w których kąty są równe między sobą.
\end{enumerate}

\subsubsection{Twierdzenia}
\begin{enumerate}
    \item [3.1] Skonstruować środek danego okręgu. 
    \item [3.2] Twierdzenie ...
    % Twierdzenie 2. % Jeżeli na okręgu obierzemy dwa gdziekolwiek punkty, linia prosta łącząca te punkty padnie wewnątrz koła.
    \item [3.3] Twierdzenie ...
    % Twierdzenie 3. % Jeżeli w kole linia prosta przez środek poprowadzona przecina linie nie przez środek poprowadzoną na dwie równe części, będzie pierwsza prostopadła do drugiej; i jeżeli pierwsza jest prostopadła do drugiej, przecina ja na dwie równe części.
    \item [3.4] Twierdzenie ...
    % Twierdzenie 4. % Jeżeli w kole dwie linie proste, nie przez środek koła poprowadzone przecinają się nawzajem, nie przetną się na dwie równe części.
    \item [3.5] Dwa okręgi, które się przecinają, nie mogą być współśrodkowe. 
    \item [3.6] Twierdzenie ...
    % Twierdzenie 6. % Jeżeli dwa koła dotykają się wzajemnie, to wspólnego środka mieć nie mogą.
    \item [3.7] Twierdzenie ...
    % Twierdzenie 7. % Jeżeli na średnicy koła wzięty będzie punkt którykolwiek oprócz średnicy koła i od tego punktu poprowadzone linie proste do okręgu, ze wszystkich linii największa będzie część średnicy, na której znajduje się środek koła, a najmniejsza pozostała część średnicy, z innych zaś linii prostych każda bliższa przechodząca przez środek koła, większa będzie od odleglejszej, z tego na koniec punktu dwie tylko równe linie proste z obydwu stron najmniejszej linii prostej mogą być do okręgu poprowadzone.
    \item [3.8] Twierdzenie ...
    % Twierdzenie 8. % Jeżeli z punktu zewnątrz koła obranego, poprowadzone będą do okręgu linie proste, z których jedna przechodziła by przez środek koła a inne padały gdziekolwiek, z linii prostych padających na część okręgu wklęsłą, największa jest linia poprowadzona przez środek koła, z innych zaś linii każda bliższa przechodzącej przez środek jest większa od odleglejszej. Lecz z linii padających na cześć okręgu wypukłą, najmniejsza jest linia prosta zawarta między punktem zewnętrz koła i średnicą, z innych zaś linii prostych każda bliższa najmniejszej, mniejsza jest odleglejsza; na koniec dwie tylko równe linie proste z tego punktu po obydwu stronach najmniejszej linii prostej mogą być do okręgu poprowadzone.
    \item [3.9] Twierdzenie ...
    % Twierdzenie 9. % Jeżeli z punktu danego wewnątrz koła poprowadzimy do okręgu więcej niż dwie linie proste i te proste są miedzy sobą równe, punkt ten będzie środkiem koła.
    \item [3.10] Dwa okręgi, które się przecinają, przecinają się w dwóch punktach. 
    \item [3.11] Twierdzenie ...
    % Twierdzenie 11. % Jeżeli dwa koła stykają się ze sobą wewnątrz, linia łącząca środki tychże kół przedłużona pada na punkt dotykania się kół.
    \item [3.12] Twierdzenie ...
    % Twierdzenie 12. % Jeżeli dwa koła dotykają się ze sobą zewnętrznie, to linia prosta łącząca ich środki przechodzi przez punkt dotykania się.
    \item [3.13] Twierdzenie ...
    % Twierdzenie 13. % Okrąg koła nie może dotykać okręgu drugiego koła w więcej niż jednym punkcie, nieważne jest czy dotkniecie jest zewnętrzne bądź wewnętrzne.
    \item [3.14] Twierdzenie ...
    % Twierdzenie 14. % W kole linie proste równe, na okręgu jego zakończone, są równoodległe od środka; i linie proste które na okręgu jego zakończone są równoodległe od środka, są też miedzy sobą równe.
    \item [3.15] Twierdzenie ...
    % Twierdzenie 15. % Ze wszystkich linii prostych w kole poprowadzonych i na okręgu jego zakończonych, największa jest średnica, z innych zaś każda bliższa środka koła, większa jest od odleglejszej; i z dwóch linii prostych nierównych, większa bliższa jest środka koła od mniejszej.
    \item [3.16] Twierdzenie ... 
    % Twierdzenie 16. % Prostopadła do średnicy koła z końca jej wyprowadzona, pada cała zewnątrz koła, a między tą prostopadłą i okręgiem żadna inna linia prosta nie pada; albo tak samo: okrąg koła przechodzi miedzy prostopadłą do średnicy i linią prostą, która ze średnicą kąt ostry jakokolwiek wielki zawiera, czyli która zawiera kąt jakokolwiek mały z prostopadłą do średnicy.
    \item [3.17] Skonstruować styczną do danego okręgu, która przechodzi przez dany punkt.
    \item [3.18] Twierdzenie ...
    % Twierdzenie 18. % Jeżeli linia prosta dotyka się okręgu koła, a ze środka koła wyprowadzona będzie linia prosta do punktu dotykania się, to ta będzie prostopadła do stycznej.
    \item [3.19] Twierdzenie ...
    % Twierdzenie 19. % Jeżeli linia prosta dotyka okręgu koła, z punktu zaś dotknięcia wyprowadzona będzie do tej stycznej prostopadła, to na prostopadłej będzie środek koła.
    \item [3.20] Twierdzenie ...
    % Twierdzenie 20. % W kole, kąt mający wierzchołek we środku jest podwojeniem kata mającego swój wierzchołek na okręgu koła, gdyż tę samą podstawę okręgu mają za podstawę, czyli to samo gdy ramionami swymi tej samej części okręgu obejmują.
    \item [3.21] Twierdzenie ...
    % Twierdzenie 21. % Kąty w tym samym odcinku koła są między sobą równe.
    \item [3.22] Twierdzenie ...
    % Twierdzenie 22. % Kąty przeciwne czworokąta w koło wpisane są równe dwóm kątom prostym.
    \item [3.23] Twierdzenie ...
    % Twierdzenie 23. % Na tej samej linii prostej nie można wykreślić dwóch odcinków kół po tej samej stronie podobnych, które by nie przystawały do siebie.
    \item [3.24] Twierdzenie ...
    % Twierdzenie 24. % Wykreślone na równych liniach prostych podobne odcinki kół, są między sobą równe.
    \item [3.25] Twierdzenie ...
    % Twierdzenie 25. % Mając dany odcinek koła, opisać koła którego jest odcinkiem.
    \item [3.26] Twierdzenie ...
    % Twierdzenie 26. % W kołach równych, kąty równe w środkach lub przy okręgach wspierają się na równych łukach.
    \item [3.27] Twierdzenie ...
    % Twierdzenie 27. % W kołach równych, kąty we środkach lub przy okręgach, na równych łukach wspierające się, są między sobą równe.
    \item [3.28] Twierdzenie ...
    % Twierdzenie 28. % W kołach równych, cięciwy równe obejmują łuki równe, tak, że łuk większy większemu, mniejszy mniejszemu jest równy.
    \item [3.29] Twierdzenie ...
    % Twierdzenie 29. % W kołach równych, równe łuki obejmują cięciwy równe.
    \item [3.30] Podzielić dany Twierdzenie ...
    % Twierdzenie 30. % Dany łuk podzielić na dwie części.
    \item [3.31] Twierdzenie ...
    % Twierdzenie 31. % W kole, kąt w półkolu jest prosty; z katów zaś w odcinkach nierównych, kąt w większym odcinku mniejszy jest od prostego; a w mniejszym odcinku większy od prostego.
    \item [3.32] Twierdzenie ...
    % Twierdzenie 32. % Jeżeli okręgu koła dotyka linia prosta, z punktu zaś dotknięcia poprowadzona będzie cięciwa, kąty zawarte miedzy cięciwową i styczną, będą równe kątom w odcinkach koła na przemian.
    \item [3.33] Twierdzenie ...
    % Twierdzenie 33. % Na danej linii prostej wykreślić odcinek koła który by zawierał kąt równy kątowi danemu.
    \item [3.34] Twierdzenie ...
    % Twierdzenie 34. % Z koła danego oddzielić odcinek któryby zawierał kąt równy danemu kątowi.
    \item [3.35] Twierdzenie ...
    % Twierdzenie 35. % Jeżeli w kole dwie cięciwy przecinają się nawzajem, prostokąt zawarty odcinkami jednej cięciwy będzie równy prostokątowi zawartemu odcinkami drugiej cięciwy.
    \item [3.36] Twierdzenie ...
    % Twierdzenie 36. % Jeżeli z punktu za kołem obranego, poprowadzimy dwie linie proste, których jedna przecinałaby koło, a druga byłaby styczną; to prostokąt zawarty całą linia przecinającą i odcinkiem jej za kołem będzie równy kwadratowi ze stycznej.
    \item [3.37] Twierdzenie ...
    % Twierdzenie 37. % Jeżeli z dwóch linii prostych, od jednego punktu zewnątrz koła obranego poprowadzonych, jedna przecina koło, a druga pada na okrąg tego koła: i jeżeli prostokąt z całej linii przecinającej i odcinka jej za kołem będącego jest równy kwadratowi z linii padającej na okrąg koła, to linia będzie padająca na okrąg koła styczną.
\end{enumerate}

%
%

\subsubsection{Księga IV}
\paragraph{Definicje}
\begin{enumerate}
	\item [4.1] Definicja ...
	% Definicja 1. % Mówi się że figura prostokreślna wpisuje się w figurę prostokreślną, wtedy kiedy każdy kąt figury wpisanej dotyka się każdego boku figury, w który się wpisuje.
	\item [4.2] Definicja ...
	% Definicja 2. % Podobnie się mówi, że figura opisuje się na figurze, kiedy każdy bok figury opisanej dotyka każdego kąta figury na której się opisuje.
	\item [4.3] Definicja ...
	% Definicja 3. % Figura prostokreślna wpisuje się w koło, kiedy każdy kąt figury wpisanej dotyka okręgu koła.
	\item [4.4] Definicja ...
	% Definicja 4. % Figura prostokreślna opisuje się na kole kiedy każdy bok figury opisanej dotyka okręgu koła.
	\item [4.5] Definicja ...
	% Definicja 5. % Podobnież koło wpisuje się w figurę prostokreślną, kiedy każdy bok figury w którą koło się wpisuje, dotyka okręgu koła.
	\item [4.6] Definicja ...
	% Definicja 6. % Koło opisuje się na figurze prostokreślne wtedy gdy okrąg dotyka do każdego kąta figury na której opisujemy koło.
	\item [4.7] Definicja ...
	% Definicja 7. % Mówi się, że linie proste kreśli się w kole, gdy jej końce są na okręgu danego koła.
\end{enumerate}

\paragraph{Twierdzenia}
\begin{enumerate}
	\item [4.1] Wpisać odcinek krótszy od średnicy w dany okrąg.
	\item [4.2] Twierdzenie ...
	% Twierdzenie 2. % W dane koło wpisać trójkąt równoramienny względem danego trójkąt.
	\item [4.3] Twierdzenie ...
	% Twierdzenie 3. % Na danym kole opisać trójkąt równokątny względem danego trójkąta.
	\item [4.4] Twierdzenie ...
	% Twierdzenie 4. % W dany trójkąt wpisać koło.
	\item [4.5] Twierdzenie ...
	% Twierdzenie 5. % Na danym trójkącie opisać koło.
	\item [4.6] Wpisać kwadrat w dany okrąg.
	\item [4.7] Opisać kwadrat na danym okręgu.
	\item [4.8] Wpisać okrąg w dany kwadrat.
	\item [4.9] Opisać okrąg na danym kwadracie.
	\item [4.10] Wykreślić trójkąt równoramienny, którego kąt przy podstawie jest podwojeniem kąta przy wierzchołku (o kątach $\pi/5$, $2\pi/5$, $2\pi/5$).
	\item [4.11] Twierdzenie ...
	% Twierdzenie 11. % W dane koło wpisać pięciokąt równoboczny i równokątny.
	\item [4.12] Opisać pięciokąt równoboczny i równokątny na danym okręgu.
	\item [4.13] Wpisać okrąg w dany pięciokąt równoboczny i równokątny.
	\item [4.14] Opisać okrąg na danym pięciokącie równobocznym i równokątnym
	\item [4.15] Wpisać sześciokąt równoboczny i równokątny w dany okrąg.
	\item [4.16] Wpisać piętnastokąt równoboczny i równokątny w dany okrąg.
\end{enumerate}

%

% TODO: https://kpbc.umk.pl/dlibra/publication/37/edition/66/content
% TODO: Pojęcia pierwotne i aksjomaty Euklidesa nie są jednak idealne.
% TODO: Dlatego zamiast nich będziemy używać aksjomatów Hilberta podanych około 1899 roku.

% https://www.claymath.org/library/historical/euclid/
% BOOK I	Triangles, parallels, and area
% BOOK II	Geometric algebra
% BOOK III	Circles
% BOOK IV	Constructions for inscribed and circumscribed figures
% BOOK V	Theory of proportions
% BOOK VI	Similar figures and proportions
% BOOK VII	Fundamentals of number theory
% BOOK VIII	Continued proportions in number theory
% BOOK IX	Number theory
% BOOK X	Classification of incommensurables
% BOOK XI	Solid geometry
% BOOK XII	Measurement of figures
% BOOK XIII	Regular solids

%
%

\subsection{Aksjomaty Hilberta}
W aksjomatyce Hilberta (płaskiej geometrii euklidesowej) pojęciami pierwotnymi są punkt, prosta, płaszczyzna, relacja incydencji (leżeć na, zawierać się w), relacja uporządkowania (leżenia między) oraz relacja przystawania.

%

\subsubsection{Aksjomaty incydencji}
\begin{axiom}[incydencji, I1]
    Dla każdej pary punktów $A$ oraz $B$ istnieje dokładnie jedna prosta $l$, która przechodzi przez te punkty.
\end{axiom}

\begin{axiom}[incydencji, I2]
    Na każdej prostej istnieją co najmniej dwa punkty.
\end{axiom}

\begin{axiom}[incydencji, I3]
    Istnieją co najmniej trzy punkty, które nie są współliniowe.
\end{axiom}

,,Być współliniowym'' to synonim leżenia na jednej prostej.

\begin{proposition}
    Dwie różne proste mogą mieć co najwyżej jeden punkt wspólny.
\end{proposition}

\begin{definition}
    Dwie różne proste, które nie mają punktów wspólnych, nazywamy równoległymi.
    Każda prosta jest też równoległa do siebie.
\end{definition}

\begin{axiom}[Playfaira, P]
    Dla każdej prostej $l$ oraz punktu $A$, istnieje co najwyżej jedna prosta przechodząca przez $A$, równoległa do $l$.
\end{axiom}

John Playfair opublikował ten aksjomat w 1795 roku, chociaż już wtedy twierdził, że inni używali go przed nim (np. William Ludlam).
\index[persons]{Playfair, John}%
Aksjomat Playfaira jest równoważny z (piątym) postulatem Euklidesa i dlatego wiele osób próbowało wyprowadzić go z czterech wcześniejszych postulatów.
Za każdym razem okazywało się, że w ,,dowodzie'' użyte jest zdanie będące równoważnikiem piątego postulatu, na przykład:
\begin{itemize}
    \item suma kątów wewnętrznych każdego trójkąta jest kątem półpełnym,
    \item suma kątów wewnętrznych każdego trójkąta jest taka sama,
    \item istnieje trójkąt, którego suma kątów wewnętrznych jest kątem półpełnym,
    \item istnieją dwa trójkąty, które są do siebie podobne, ale nie przystające,
    \item na każdym trójkącie można opisać okrąg,
    \item jeśli trzy kąty wewnętrzne czworokąta są proste, to czwarty kąt także jest prosty,
    \item istnieje para prostych, które są w stałej odległości od siebie,
    \item dwie proste, które są równoległe do danej prostej, są też równoległe do siebie,
    \item twierdzenie Pitagorasa lub jego uogólnienie, twierdzenie cosinusów
    \item aksjomat Wallisa: nie ma górnego ograniczenia na pole trójkąta,
    \item kąty przy górnej podstawie czworokąta Saccheriego (czworokąta $ABCD$ o dwóch kątach prostych przy dolnej podstawie $AB$, w którym boki $AD$ i $BC$ mają równe długości) są proste,
    \item aksjomat Proklosa: jeśli prosta przecina jedną z dwóch równoległych prostych, to przecina także tę drugą.
\end{itemize}

\begin{example}
    Rozważmy geometrię, gdzie płaszczyzna składa się z pięciu punktów $A$, $B$, $C$, $D$, $E$ i leży na niej dziesięć różnych prostych, każda z nich przechodząca przez dokładnie dwa różne punkty.
    Wtedy proste $AB$ i $AC$ mają punkt wspólny $A$, chociaż obydwie są równoległe do prostej $DE$.
    Aksjomat Playfaira nie jest spełniony.
\end{example}

\begin{proposition}
    Aksjomaty incydencji I1, I2, I3 oraz aksjomat P (Playfaira) są od siebie niezależne.
\end{proposition}

Hartshorne \cite[s. 69-70]{hartshorne2000} konstruuje modele geometrii, w których spełnione są dowolne trzy, ale nie czwarty z nich.

\begin{proposition}
    Płaszczyzna rzutowa to taki zbiór punktów oraz prostych (podzbiorów zbioru punktów), że:
    \begin{itemize}
        \item przez dwa różne punkty przechodzi dokładnie jedna prosta,
        \item każde dwie proste mają punkt wspólny,
        \item każda prosta ma co najmniej trzy punkty i
        \item nie wszystkie punkty są współliniowe.
    \end{itemize}
    (Każdy z wymienionych aksjomatów jest niezależny od pozostałych, ze wszystkich razem wynikają aksjomaty incydencji).
    Każda płaszczyzna rzutowa ma co najmniej 7 punktów, dokładnie jedna płaszczyzna rzutowa ma dokładnie 7 punktów.
    Jeśli istnieje prosta, która ma $n+1$ punktów, to płaszczyzna ma $n^2 + n + 1$ punktów.
\end{proposition} % Hartshorne 71

\begin{example}[płaszczyzna Fana]
    Zbiór złożony z siedmiu elementów (,,punktów''), w którym wyróżniono rodzinę siedmiu podzbiorów (,,prostych'') spełniający następujące warunki:
    \begin{itemize}
        \item każde dwie różne proste mają dokładnie jeden punkt wspólny,
        \item każde dwa różne punkty należą do dokładnie jednej prostej
    \end{itemize}
    nazywamy płaszczyzną Fana.
    Jest przykładem płaszczyzny rzutowej, która nie spełnia aksjomatu Fana (że trzy punkty przekątne czworoboku zupełnego nie są współliniowe).
    \begin{figure}[H]
        \centering
        \begin{tikzpicture}[
        mydot/.style={
        draw,
        circle,
        fill=black,
        inner sep=1.5pt}
        ]
        \draw
        (0,0) coordinate (A) --
        (4cm,0) coordinate (B) --
        ($ (A)!.5!(B) ! {sin(60)*2} ! 90:(B) $) coordinate (C) -- cycle;
        \coordinate (O) at
        (barycentric cs:A=1,B=1,C=1);
        \draw (O) circle [radius=4cm*1.717/6];
        \draw (C) -- ($ (A)!.5!(B) $) coordinate (LC); 
        \draw (A) -- ($ (B)!.5!(C) $) coordinate (LA); 
        \draw (B) -- ($ (C)!.5!(A) $) coordinate (LB); 
        \foreach \Nodo in {A,B,C,O,LC,LA,LB}
        \node[mydot] at (\Nodo) {};    
    \end{tikzpicture}%
    \caption{Płaszczyzna Fana}
\end{figure}
\end{example}

W momencie pisania tego tekstu nie jesteśmy jeszcze zainteresowani geometrią rzutową, więc przytoczymy tylko wymagania wobec studenta, który ukończył kurs ,,Geometria III'' na uniwersytecie w Warszawie.

Zna pojęcie płaszczyzny rzutowej rzeczywistej (równoważne sformułowania), dwustosunku, definicję przekształceń rzutowych łańcuchów, stożkowych, pęków stycznych do stożkowych.
\index{dwustosunek}%
\index{pęk}%
\index{stożkowa}%
Zna i~potrafi stosować twierdzenia Steinera i Braikenridge'a-Maclaurina.
\index{twierdzenie!Steinera}%
\index{twierdzenie!Braikendridge'a-Maclaurina}%
Wie w jaki sposób określa się rzutowo ogniska i kierownice stożkowych.
\index{ognisko}%
\index{kierownica}%

\begin{proposition}
    Płaszczyzna afiniczna to taki zbiór punktów i prostych, które spełniają:
    \begin{itemize}
        \item aksjomaty incydencji oraz
        \item mocniejszą wersję aksjomatu Playfaira: dla każdej prostej $l$ i punktu $A$, dokładnie jedna prosta przechodzi przez punkt $A$ i jest równoległa do $l$.
    \end{itemize}
    Każda prosta na płaszczyźnie afinicznej ma tyle samo punktów.
    Jeśli pewna prosta ma $n$ punktów, to płaszczyzna ma dokładnie $n^2$ punktów.
    Istnieją płaszczyzny afiniczne o $4$, $9$, $16$ i $25$ punktach, ale nie istnieje taka, która miałaby $36$ punktów.
\end{proposition} % Hartshorne 71, 72

Geometrii afinicznej też nie rozumiemy, dlatego zaznaczymy tylko, co mogłoby się pojawić w kolejnym wydaniu.
Grupa przekształceń afinicznych od strony geometrycznej: powinowactwa osiowe, rozkład przekształcenia afinicznego na podobieństwo i powinowactwo osiowe, kierunki główne przekształcenia afinicznego.
Niezmienniczość stosunku pól przy przekształceniu afinicznym
Obraz okręgu przy przekształceniu afinicznym.

%
%

\subsubsection{Aksjomaty leżenia pomiędzy}
\begin{axiom}[leżenia pomiędzy, B1]
    Jeśli punkt $B$ leży między punktami $A$ i $C$, to punkty $A$, $B$, $C$ są różnymi punktami tej samej prostej oraz punkt $B$ leży także między punktami $C$ i $A$.
\end{axiom}

\begin{axiom}[leżenia pomiędzy, B2]
    Dla każdej pary punktów $A$ i $B$ istnieje punkt $C$ taki, że punkt $B$ leży między punktami $A$ i $C$.
\end{axiom}

\begin{axiom}[leżenia pomiędzy, B3]
    Spośród trzech punktów leżących na prostej, dokładnie jeden leży pomiędzy pozostałymi dwoma.
\end{axiom}

\begin{axiom}[leżenia pomiędzy, B4]
    Niech $A$, $B$ i $C$ będą trzema niewspółliniowymi punktami, zaś $l$ prostą, która nie przechodzi przez żaden z nich.
    Jeśli prosta $l$ przechodzi przez punkt między punktami $A$ i $B$, to przechodzi też przez punkt między punktami $A$ i $C$ albo $B$ i $C$, ale nie przez obydwa.
\end{axiom}

Powyższy aksjomat nazywany jest też aksjomatem Pascha, ponieważ Moritz Pasch \cite{pasch_1882} przyłapał dopiero w 1882 roku geometrów całego świata na tym, że korzystali z takiej przesłanki.
\index{aksjomat!Pascha}%
\index[persons]{Pasch, Moritz}%

\begin{proposition}
    Z aksojmatów I1, I2, I3, B1, B2, B3, B4 wynika, że każda prosta ma nieskończenie wiele punktów.
\end{proposition}

\begin{definition}[odcinek]
    Niech $A$, $B$ będą punktami.
    Zbiór złożony z punktów $A$, $B$ oraz punktów, które leżą między nimi, nazywamy odcinkiem i oznaczamy $\overline {AB}$.
\end{definition} % Hartshorne 74

\begin{definition}[trójkąt]
    Niech $A$, $B$, $C$ będą punktami.
    Sumę odcinków $AB$, $BC$, $AC$ nazywamy trójkątem, wspomniane odcinki -- jego bokami, zaś punkty $A$, $B$ i $C$ -- wierzchołkami.
\end{definition} % Hartshorne 74

\begin{proposition}
    Niech $l$ będzie prostą.
    Wtedy zbiór punktów, które nie leżą na prostej $l$ można rozbić na dwa niepuste zbiory $S_1$, $S_2$ takie, że: dwa punkty, które nie leżą na prostej $l$, należą do tego samego zbioru ($S_1$ lub $S_2$) wtedy i~tylko wtedy, gdy odcinek $AB$ nie przecina prostej $l$.
\end{proposition} % Hartshorne 74

Zbiory $S_1$, $S_2$ nazywamy stronami prostej $l$.
Podobnie punkt wyznacza na prostej dwa zbiory, które leżą po różnych stronach tego punktu.

\begin{definition}[półprosta]
    Niech $A$, $B$ będą punktami.
    Zbiór złożony z punktów $A$, $B$ oraz punktów, które leżą po tej samej stronie punktu $A$ na prostej $AB$ co punkt $B$, nazywamy półprostą i oznaczamy $NIE WIEM JAK AB$.
\end{definition} % Hartshorne 77

\begin{definition}[kąt]
    Sumę dwóch półprostych $AB$, $AC$, które nie leżą na jednej prostej, nazywamy kątem, zaś punkt $A$ wierzchołkiem tego kąta.
    Wnętrze kąta $\angle BACS$ składa się z tych punktów $D$ takich, że $D$ i $C$ leżą po tej samej stronie prostej $AB$ oraz $D$ i $B$ leżą po tej samej stronie prostej $AC$.
\end{definition} % Hartshorne 77

W myśl tej definicji, nie ma kąta zerowego ani półpełnego.
Wnętrze trójkąta $ABC$ to część wspólna wnętrz kątów $\angle ABC$, $\angle BCA$, $\angle CAB$; jest wypukłe i niepuste.

%
\input{section-01/axioms-hilbert-congruence-1}
\subsubsection{Aksjomaty przystawania kątów}
\begin{axiom}[przystawania, C4]
    \loremipsum
\end{axiom}

\begin{axiom}[przystawania, C5]
    \loremipsum
\end{axiom}

\begin{axiom}[przystawania, C6]
    \loremipsum
\end{axiom}

Kąty przyległe, prosty.

\subsubsection{Płaszczyzna Hilberta}

. . .

%

\section{Przekształcenia geometryczne}
\subsection{Symetria osiowa}
\subsection{Symetria środkowa}
\subsection{Jednokładność}

\subsection{Geometria pretalesowska}
\subsubsection{Cechy przystawania}
Cechy przystawania

\subsubsection{Kąty wierzchołkowe, naprzemianległe}
Kąty wierzchołkowe, naprzemianległe

%

\index{pons asinorum|(}

\subsubsection{Pons asinorum}
\index{most osłów patrz pons asinorum}
Most osłów (łacińskie \emph{,,pons asinorum''}) to tradycyjna nazwa dowodu twierdzenia, że kąty przy podstawie trójkąta równoramiennego są równe.
Podał go Euklides jako teza V w księdze I Elementów.
Mawiało się, że ci, którzy nie są w stanie samodzielnie przeprowadzić tego dedukcyjnego dowodu opartego na własnościach trójkątów przystających, nie może przekroczyć mostu i studiować dalej geometrii.

Bardziej przyziemnie Coxter \cite[s. 6-9]{coxeter_1991} zauważa, że rysunek wykonany przez Euklidesa przypomina most.
Wśród konsekwencji wymienia kilka wyników z Elementów: III.3, III.20, III.21, III.22, III.32, VI.2, VI.4, a potem III.35, III.36, VI.19, co prowadzi do dowodu twierdzenia Pitagorasa, czyli I.47. % TODO: sprawdzić, czy numeracja moja i Coxetera jest taka sama.
\index{twierdzenie!Pitagorasa}%
Coxeter podaje w formie ćwiczeń nierówność Erdős-Mordella (u nas podsekcja \ref{subsection_erdos_mordell}) oraz twierdzenie Steinera-Lehmusa (twierdzenie \ref{theorem_steiner_lehmus}).
% TODO: https://www.deltami.edu.pl/1990/08/elementarny-dowod-nierownosci-erdosa-mordella/
\todofoot{Przeczytać artykuł z Delty 1990, elementarny-dowod-nierownosci-erdosa-mordella}

Pierwsze dowody tego faktu podali jeszcze Euklides, komentujący jego prace Proklos zwany Diadochem oraz Pappus z Aleksandrii.
Współcześnie podaje się krótkie uzasadnienie w oparciu o dwusieczną kąta, ale Euklides nie mógł tak uczynić, ponieważ definiuje ją dopiero cztery tezy później w swoich Elementach.

O moście osłów piszą Coxeter 

\index{pons asinorum|)}

%

\subsubsection{Równoległobok (problem Fagnano i Fermata?)}
Równoległobok (problem Fagnano i Fermata?)

\subsubsection{Symetralna, okrąg opisany na trójkącie.}
Symetralna, okrąg opisany na trójkącie.

\subsubsection{Geometria koła i kątów, twierdzenie Apolloniusza (s. 22)}
Geometria koła i kątów, twierdzenie Apolloniusza (s. 22)

\subsubsection{Okrąg opisany na czworokącie}
Okrąg opisany na czworokącie.
Iloczynowe warunki istnienia okręgu przechodzącego przez cztery punkty.

Twierdzneie o prostej Wallace'a-Simsona.
Na UW: Zastosowanie: okrąg dziewięciu punktów, twierdzenie o prostej Simsona.

\subsubsection{Kąty w okręgu}
Kąty w okręgu: wpisane, kąty środkowe i kąty dopisane.
Twierdzenia o kątach wpisanych, kątach środkowych i kątach dopisanych do okręgu.
Kątowe warunki na istnienie okręgu przechodzącego przez cztery punkty.
Styczna do okręgu, okrąg wpisany w kąt.
Okrąg wpisany w trójkąt, okręgi dopisane do trójkąta.
Warunki istnienia okręgu stycznego do czterech prostych.

\subsubsection{Twierdzenie Miquela}
Twierdzenie Miquela

\subsubsection{Okrąg wpisany w czworokąt}
Okrąg wpisany w czworokąt

\subsubsection{Twierdzenie Pitagorasa}
Twierdzenie Pitagorasa

%

\section{Podobieństwo}
\subsection{Jednokładność}
Podobieństwo figur, trójkątów (cechy), stosunek pól figur podobnych.

%

Guzicki-3

\begin{theorem}[Talesa]
    Jeśli ramiona kąta płaskiego przetnie się 2 równoległymi prostymi:
    \begin{center}
\begin{comment}
        \begin{tikzpicture}
            \tkzDefPoint(0, 0.5){O}
            \tkzDefPoint(1.5, 0){A}
            \tkzDefPoint(2, 1){Ap}
            \tkzDefPointBy[homothety=center O ratio 1.618](A) \tkzGetPoint{B}
            \tkzDefLine[parallel=through B](A,Ap) \tkzGetPoint{Bp}
            \tkzInterLL(O,Ap)(B,Bp) \tkzGetPoint{Bpp}
            \tkzDrawPoints[fill=gray,opacity=.9](O,A,B,Ap,Bpp)
            \tkzLabelPoint[above](O){$O$}
            \tkzLabelPoint[below](A){$A$}
            \tkzLabelPoint[below](B){$A'$}
            \tkzLabelPoint[above left](Bpp){$B'$}
            \tkzLabelPoint[above left](Ap){$B$}
            \tkzDrawLine[thick](O,B)
            \tkzDrawLine[thick](O,Bpp)
            \tkzDrawLine[color=blue, thick](A,Ap)
            \tkzDrawLine[color=blue, thick](B,Bpp)
        \end{tikzpicture}
\end{comment}
        \end{center}
    to długości odcinków wyznaczonych przez te proste na jednym z ramion kąta są proporcjonalne do długości odpowiednich odcinków na drugim ramieniu kąta, a zatem
    \begin{equation}
        \label{thales_ratio}
        \frac{|OA|}{|OA'|} = \frac{|OB|}{|OB'|} = \frac{|AB|}{|A'B'|}.
    \end{equation}
\end{theorem}
% TODO: https://en.wikipedia.org/wiki/Thales's_theorem

Tradycja przypisuje jego sformułowanie Talesowi z Miletu, chociaż znane było starożytnym Babilończykom i Egipcjanom.
\index[persons]{Tales z Miletu}%
% Pierwszy znany dowód pojawia się w Elementach Euklidesa.
Najstarszy zachowany dowód twierdzenia Talesa zamieszczony jest w VI. księdze Elementów Euklidesa. 
% https://en.wikipedia.org/wiki/Intercept_theorem#Claim_3

Piszą o nim Neugebauer, Bogdańska \cite[s. 48-56]{neugebauer_2018}; Audin \cite[s. 24, 173]{audin_2003}.

Po angielsku znane jest jako \emph{Thales's theorem}, \emph{intercept theorem}, \emph{basic proportionality theorem} albo \emph{side splitter theorem}.

Prawdziwe jest również twierdzenie odwrotne:

\begin{proposition}[twierdzenie odwrotne do tw. Talesa]
    Jeżeli pewna prosta przecina boki $OA'$, $OB'$ trójkąta $OA'B'$ w różnych punktach $A$ i $B$ odpowiednio, a przy tym zachodzi równość \ref{thales_ratio}, to prosta ta jest równoległa do prostej $A'B'$.
\end{proposition}

Prostym wnioskiem z twierdzenia Talesa jest fakt \ref{hartshorne_52}, znajduje on zastosowanie w dowodzie:
% Neugebauer s. 52

\begin{theorem}[Varignona]
    Czworokąt $PQRS$, którego wierzchołki leżą na środkach boków $AB$, $BC$, $CD$, $DA$ czworokąta $ABCD$, jest równoległobokiem.
    Jego znakowane  (!) pole jest równe połowie pola czworokąta $ABCD$. % Neugebauer s. 61
\end{theorem}

% * The area of the Varignon parallelogram equals half the area of the original quadrilateral. This is true in convex, concave and crossed quadrilaterals provided the area of the latter is defined to be the difference of the areas of the two triangles it is composed of. => [[Varignon's theorem]]


W szczególności, czworokąt $ABCD$ nie musi być wypukły\footnote{Może być nawet ,,motylkiem'', to znaczy łamaną zamkniętą o czterech bokach, która ma samoprzecięcia.}.
Twierdzenie zostało nazwane na cześć Pierre'a Varignona pośmiertnie w 1731 roku.
\index[persons]{Varignon, Pierre}%
Co więcej,

\begin{proposition}
    Równoległobok Varignona jest rombem (prostokątem) wtedy i tylko wtedy, gdy przekątne czworokąta $ABCD$ są równej długości (są prostopadłe do siebie).
\index{równoległobok!Varignona}%
\index{romb}%
\index{prostokąt}%
% de Villiers, Michael (2009), Some Adventures in Euclidean Geometry, Dynamic Mathematics Learning, p. 58, 169. ISBN 9780557102952.
\end{proposition}

%

% \subsection{Pole?}

\input{section-01/ptolemy}


\subsection{Potęga punktu względem okręgu}

\begin{proposition}
\label{guzicki_6_11}%
    Dane są dwa niewspółśrodkowe okręgi $\omega_1$ i $\omega_2$.
    Miejscem geometrycznym punktów $P$ mających równe potęgi względem obu okręgów jest prosta prostopadła do prostje przechodzącej przez środki obu okręgów.
\index{potęga punktu}%
\end{proposition}

Patrz Guzicki \cite[s. 173, 174]{guzicki_2021}.
Prostą, której istnienie właśnie zasugerowaliśmy, nazywamy \textbf{osią potęgową} okręgów $\omega_1, \omega_2$.

\begin{corollary}
	Dane są trzy parami niewspółśrodkowe okręgi na płaszczyźnie: $\omega_1, \omega_2, \omega_3$.
	Jeśli środki tych okręgów są współliniowe, to osie potęgowe każdej pary są równoległe.
	W przeciwnym razie wszystkie trzy osie przecinają się w~jednym punkcie zwanym \textbf{środkiem potęgowym} tych trzech okręgów.
\end{corollary}

Patrz Guzicki \cite[s. 174]{guzicki_2021}.


UW zrobione:
Potęga punktu względem okręgu, oś potęgowa dwóch okręgów, środek potęgowy trzech okręgów.

UW niezrobione:
twierdzenie Brianchona, konstrukcja stycznej do okręgu samą linijką, okręgi współpękowe, twierdzenie Gaussa-Bodenmillera, twierdzenie o motylku, formuła Eulera na odległość między środkami okręgu opisanego i wpisanego (dla trójkąta), twierdzenie Ponceleta dla trójkąta.

\begin{definition}[potęga punktu względem okręgu]
	Jeżeli...
\end{definition}
\begin{proposition}[potęgowe kryterium współokręgowości]
	Jeżeli...
\end{proposition}
\begin{definition}[oś potęgowa]
	Jeżeli...
\end{definition}
\begin{theorem}[Monge'a]
	Jeżeli...
\end{theorem}
\begin{theorem}[Auberta]
	Jeżeli...
\end{theorem}


\subsection{Twierdzenie o dwusiecznej}
Okrąg Apolloniusza, Guzicki-4

\begin{proposition}[twierdzenie o dwusiecznej]
	Jeżeli...
\end{proposition}
\begin{theorem}[Lehmusa-Steinera]
	Jeżeli...
\end{theorem}
\begin{definition}[okrąg Apoloniusza]
	Jeżeli...
\end{definition}


\subsection{Twierdzenie Newtona i Gaussa?}
Twierdzenie Newtona: środek okręgu wpisanego w czworokąt i środki przekątnych tego czworokąta są współliniowe.
Twierdzenie Gaussa: środki przekątnych czworokąta zupełnego są współliniowe.

\subsection{Dwustosunek}

\subsection{Okręgi ortogonalne, pęki okręgów.}
Wie czym są pęki okręgów, zna ich podstawowe własności i potrafi stosować w konfiguracjach spokrewnionych z twierdzeniem Ponceleta.   

% T2.19 tutaj

Bogdańska, Neugebauer \cite[s. 267]{neugebauer_2018} na ostatniej stronie podają niespodziewanie informacją, że twierdzenie Ponceleta {\color{red}\textbf{(TODO: T2.19)}\color{black}} było motywem przewodnim całego skryptu.
% todo: podlinkować te cztery dowody po ich spisaniu
Zachęcają do uogólnienia czwartego dowodu dla poniższej wersji:

\begin{theorem}[Ponceleta, małe]
	Niech trójkąt $A_0 A_1 A_2$ będzie wpisany w~stożkową $C$ oraz opisany na stożkowej $D$.
	Wtedy każdy punkt $B_0$ stożkowej $C$ jest wierzchołkiem dokładnie jednego trójkąta $B_0 B_1 B_2$ wpisanego w~stożkową $C$ oraz opisanego na stożkowej $D$.
\end{theorem}

Oczywiście jest też wielkie twierdzenie Ponceleta, udowodnione przez, jak niezbyt trudno się domyślić, Victora Ponceleta \cite[s. 311-317]{poncelet_1865} (wg Bogdańskiej, Neugebauera w 1813 roku, wg angielskiej Wikipedii w 1822 roku):x

\begin{theorem}[Ponceleta, wielkie]
	Niech $C$ i $D$ będą dwiema stożkowymi, zaś $A_0, A_1, \ldots, A_{n-1}$ takimi punktami na stożkowej $C$, że proste $A_0A_1$, $A_1A_2$, \ldots, $A_{n-1}A_0$ są styczne do stożkowej $D$.
	Wtedy dla każdego punktu $B_0$ na stożkowej $C$ istnieją różne punkty $B_1, \ldots, B_{n-1}$, też na stożkowej $C$, że proste $B_0B_1$, $B_1B_2$, \ldots, $B_{n-1}B_0$ są styczne do stożkowej $D$.
\end{theorem}

Dowód można znaleźć na przykład u Akopiana, Zasławskiego \cite[s. 93, 61, 67, 115, 124]{akopyan_2007}.


\subsection{Prosta Eulera i okrąg Feuerbacha}
Prosta Eulera w trójkącie (środek okręgu opisanego, środek ciężkości, ortocentrum).
Wszystkie wysokości itd. przecinają się w jednym punkcie; prosta Eulera, okrąg Feuerbacha, punkt Torricellego/Fermata (Guzicki-8)
\input{section-01/feuerbach}

\subsection{Trygonometria}

\subsubsection{Twierdzenie sinusów}

$$\frac{a}{\sin \alpha} = \frac{b}{\sin \beta} = \frac{c}{\sin \gamma} = 2R$$
% https://en.wikipedia.org/wiki/Law_of_sines

\subsubsection{Twierdzenie cosinusów}
\begin{proposition}[twierdzenie cosinusów]
	\label{twierdzenie_cosinusow}%
	\begin{equation}
		c^2 = a^2 + b^2 - 2ab \cos \gamma.
	\end{equation}
	% https://en.wikipedia.org/wiki/Law_of_cosines
\end{proposition}

\index[persons]{Archimedes}%

Wzory na promienie okręgów wpisanych, dopisanych.


\subsubsection{Zastosowania trygonometrii -- twierdzenie Urquharta}
Twierdzenie Urquharta

\subsubsection{Zastosowania trygonometrii -- punkt i kąt Crelle'a-Brocarda}
Punkt i kąt Crelle'a-Brocarda.

\subsubsection{Zastosowania trygonometrii -- twierdzenie o siódmym okręgu}
Twierdzenie o siódmym okręgu.

\subsubsection{Rozwiązywanie trójkątów}
Wzór Mollweide'a.
\index{wzór!Mollweide'a}%

Problem Hansena
\index{problem!Hansena}%

Problem Snelliusa-Pothenota.
\index{problem!Snelliusa-Pothenota}%

% https://en.wikipedia.org/wiki/Mollweide%27s_formula
% https://en.wikipedia.org/wiki/Snellius%E2%80%93Pothenot_problem
% https://en.wikipedia.org/wiki/Hansen%27s_problem


Twierdzenie Malfattiego.
Guzicki-11

\subsection{Bałagan}

\textbf{Twierdzenie Caseya} (nie duplikat Ptolemeusza?)

\textbf{Twierdzenie Taylora, okrąg, sześciokąt}
% https://en.wikipedia.org/wiki/Taylor_circle
{
    \emph{WIP: Taylor w 1882 roku zauważył, że rzuty spodków wysokości na pozostałe boki leżą na jednym okręgu.}
}

\textbf{Twierdzenie Eulera $1/4R^2$}

% https://en.wikipedia.org/wiki/Law_of_tangents


%


\subsection{Współliniowość, współpękowość}



Znamy trzy twierdzenia o współliniowości: ..., ... i twierdzenie o prostej Auberta ...

\begin{proposition}[twierdzenie Salmona]
	Dany jest okrąg oraz trzy jego różne cięciwy $PA$, $PB$, $PC$ takie, że przekrojem okręgów na średnicach $PA$, $PB$ (odpowiednio: $PB$, $PC$ i $PA$, $PC$) są punkty $P$, $M$ (odpowiednio: $P$, $K$ oraz $P$, $L$).
	Wtedy punkty $K$, $L$, $M$ są współliniowe.
\end{proposition}

\begin{proposition}[twierdzenie Menelaosa]
	...
	Wówczas punkty $K, L, M$ są współliniowe wtedy i tylko wtedy, gdy zachodzi
	\begin{equation}
		[AMB] [BKC] [CLA] = -1.
	\end{equation}
\end{proposition}
% https://en.wikipedia.org/wiki/Menelaus%27s_theorem
It is uncertain who actually discovered the theorem; however, the oldest extant exposition appears in Spherics by Menelaus. In this book, the plane version of the theorem is used as a lemma to prove a spherical version of the theorem.

% \begin{proposition}[twierdzenie Carnota???]
	% Neugebauer, strona 108.
% \end{proposition}

% https://en.wikipedia.org/wiki/Newton%E2%80%93Gauss_line#Existence_of_the_Newton%E2%88%92Gauss_line

\begin{proposition}
	Środki trzech przekątnych czworoboku zupełnego leżą na jednej prostej, zwaną prostą Newtona-Gaussa.
\end{proposition}

\begin{proposition}[twierdzenie Desargues'a]
	Neugebauer, strona 109.
	% https://en.wikipedia.org/wiki/Desargues%27s_theorem
\end{proposition}

\begin{proposition}[twierdzenie Pascala]
	Neugebauer, strona 113.
	% https://en.wikipedia.org/wiki/Pascal%27s_theorem
\end{proposition}

\begin{proposition}[twierdzenie Pappusa]
	Neugebauer, strona 114.
	% https://en.wikipedia.org/wiki/Pappus%27s_hexagon_theorem
\end{proposition}






1. menelaos, desargues, pascal, pappus (MENELAOS = guzicki-3)
1. Zna pojęcie inwolucji rzutowych.   Zna i potrafi stosować twierdzenia inwolucyjne Desarguesa.  
 
\subsubsection{Współpękowość}
zadanie newtona, okręgi torricelliego, ceva, carnot (suma kwadratów = suma kwadratów). Ceva = guzicki-3
twierdzenie Carnota: trzy proste są współpunktowe wtw AF2 + BD2 + CE2 = AE2 + BF2 + CD2. Wniosek: symetralne są współpunktowe. GUZICKI-6
UW: Twierdzenie Cevy (wraz z trygonometryczną wersją), przykłady punktów szczególnych trójkąta: punkt Nagela (Guzicki-4), punkt Gergonne'a (guzicki4), punkt Lemoine'a.


Ważnym kryterium współpękowości trzech czewian jest:

\begin{proposition}[twierdzenie Cevy (1678)]
	Dany jest trójkąt $ABC$ i trzy różne od wierzchołków punkty $K \in BC$, $L \in CA$, $M \in AB$.
	Wówczas czewiany $AK$, $BL$, $CM$ są współpękowe wtedy i tylko wtedy, gdy
	\begin{equation}
		[AMB] [BKC] [CLA] = 1.
	\end{equation}
\end{proposition}

% CARNOT
Uogólnieniem twierdzenia o współpękowości symetralnych boków trójkąta jest:

\begin{proposition}[twierdzenie Carnota]
	... wtedy i tylko wtedy, gdy
	\begin{equation}
		|AM|^2 + |BK|^2 + |CL|^2 = |AL|^2 + |CK|^2 + |BM|^2.
	\end{equation}
\end{proposition}
% CARNOT


\subsubsection{Do włączenia w powyższe podpodsekcje}
Twierdzenie Brianchona.
\begin{enumerate}
    \item Zna przykłady przekształceń rzutowych i umie je stosować w zadaniach i dowodach twierdzeń rzutowych (Desarguesa, Pappusa, Pascala, Brianchona). zna pojęcia: biegun i biegunowa i potrafi formułować twierdzenia dualne.  
    \item twierdzenia Newtona i Brianchona (s. 237) - GUZICKI 9
    \item Twierdzenie Kirkmana: jeśli część wspólna dwóch trójkątów wpisanych w okrąg jest sześciokątem wypukłym, to główne przekątne tego sześciokąta przecinają się w jednym punkcie. - TO JEST BARDZIEJ POD JEDNOKŁADNOŚĆ (UW)
    \item Wg Wiki, to jest wniosek z Desarguesa/Menelaos: twierdzenie o środkach jednokładności trzech okręgów, patrz TODO w kodzie źródłowym % (chyba https://atcm.mathandtech.org/EP2016/contributed/4052016_21160.pdf), na UW po: 	- Twierdzenia o składaniu jednokładności i przesunięć, 
\end{enumerate}

\subsubsection{Czewiany, symediany, twierdzenie van Aubela}
+ twierdzenie Steinera?

\subsection{Współrzędne barycentryczne}
Współrzędne barycentryczne

\subsection{Inwersja względem okręgu}
Patrz Guzicki-20: twierdzenie Ptolemeusza, zadanie Apolloniusza, zadanie Sangaku.

\subsection{Izogonalne}
Punkty izogonalnie sprzężone w trójkącie. + Twierdzenie Menelausa. (UW1)

%

\subsection{Przekształcenia płaszczyzny}
Tekst.

\subsubsection{Izometrie}
\begin{enumerate}
\item Konstrukcja obrazu punktu, okręgu, prostej przy translacji, obrocie i symetrii osiowej.
\item Złożenie dwóch i złożenie trzech symetrii osiowych.
\item Twierdzenia o składaniu izometrii.
\item Klasyfikacja izometrii na płaszczyźnie.
\item Izometrie parzyste i izometrie nieparzyste.
\item Twierdzenie o redukcji.
\item Twierdzenie Napoleona: środki ciężkości trójkątów równobocznych zbudowanych na bokach dowolnego trójkąta są wierzchołkami trójkąta równobocznego.
\end{enumerate}

\subsubsection{Podobieństwa}
\begin{enumerate}
\item Podobieństwa spiralne i odbicia dylatacyjne.
\item Klasyfikacja podobieństw płaszczyzny.
\end{enumerate}

%


%

\section{Konstrukcje klasyczne}
\subsection{Proste}
% TODO: Hartshorne s. 103
\begin{problem}
    Dwusieczna kąta.
\end{problem}

\begin{problem}
    Środek odcinka.
\end{problem}

\begin{problem}
    Prosta prostopadła do prostej, przechodząca przez punkt.
\end{problem}

\begin{problem}
    Prosta równoległa do prostej, przechodząca przez punkt.
\end{problem}





\subsection{Trudniejsze}


\begin{problem}
    Dany jest odcinek $AB$ oraz punkt $P$ wewnątrz okręgu.
    Skonstruować cięciwę tego okręgu, która przechodzi przez punkt $P$ o długości takiej samej, jak odcinek $AB$.
\end{problem}
% Hartshorne s. 26

\begin{problem}
    Dany jest odcinek $AB$, inny odcinek o długości $d$ oraz kąt $\alpha$.
    Skonstruować trójkąt $ABC$ tak, by kąt przy wierzchołku $C$ miał miarę $\alpha$, zaś suma długości ramion tego kąta była równa $d$.
\end{problem}
% Hartshorne s. 26

\begin{problem}
    Dane są dwa okręgi takie, że żaden nie jest zawarty w drugim.
    Skonstruować styczną do obydwu okręgów.
\end{problem}
% Hartshorne s. 26

\begin{problem}
    Dany jest okrąg $\Gamma$ oraz jego środek $O$.
    Skonstruować trzy przystające okręgi, które są styczne do pozostałych dwóch oraz do $\Gamma$. \hfill \emph{(13 kroków)}. % Hartshorne s. 51
\end{problem}
% Hartshorne s. 26

\begin{problem}
    Dany jest okrąg $\Gamma$ oraz dwa punkty $A$ i $B$.
    Skonstrować punkt $C$ na okręgu $\Gamma$ tak, by odcinek łączący punkty przecięcia prostych $CA$, $CB$ z okręgiem $\Gamma$ był równoległy do odcinka $AB$.
\end{problem}
% Hartshorne s. 58-59

\begin{problem}
    Skonstruować trzy parami styczne okręgi, każdy o innym promieniu, których środki nie są współliniowe. \hfill \emph{(7 kroków)}. % Hartshorne s. 62
\end{problem}

\subsection{Uszkodzone przyrządy}
\begin{problem}[połamana linijka]
    Dane są dwa punkty $A$ i $B$ na płaszczyźnie, odległe od siebie o około trzy nible.
    Mając do dyspozycji fragment linijki o długości jednej nibli oraz sprawny cyrkiel, narysować odcinek $AB$.
\end{problem}
% Hartshorne s. 25

\begin{problem}[zardzewiały cyrkiel]
    Dane są dwa punkty $A$ i $B$ na płaszczyźnie, odległe od siebie o około pięć nibli.
    Mając do dyspozycji zardzewiały cyrkiel, którym można kreślić jedynie okręgi o promieniu dwóch nibli, skonstruować trójkąt równoboczny oparty o bok $AB$.
\end{problem}

\begin{problem}
    Dany jest punkt $A$ leżący na prostej $l$.
    Skonstruować prostą prostopadłą do $l$ przechodzącą przez $A$ przy użyciu linijki i zardzewiałego cyrkla.
\end{problem}
% Hartshorne s. 25

\begin{problem}
    Dany jest punkt $A$ leżący ponad cztery nible od prostej $l$.
    Skonstruować prostą prostopadłą do $l$ przechodzącą przez $A$ przy użyciu linijki i zardzewiałego cyrkla.
\end{problem}
% Hartshorne s. 25

\begin{problem}
    Dane są trzy niewspółliniowe punkty $A$, $B$ oraz $C$.
    Skonstrować punkt $D$ na prostej $AC$ tak, żeby odcinki $AD$ oraz $AB$ były równej długości przy użyciu linijki i zardzewiałego cyrkla.
\end{problem}
% Hartshorne s. 26

\begin{problem}
    Dany jest odcinek $AB$ o długości ponad dwóch nibli oraz prosta $l$, która nie przechodzi przez końce odcinka.
    Skonstrować punkt $C$ na prostej $l$ tak, żeby odcinki $AB$ oraz $AC$ były równej długości przy użyciu linijki i zardzewiałego cyrkla.
\end{problem}
% Hartshorne s. 26

\begin{problem}
    Czy wszystkie konstrukcje, które można wykonać cyrklem i linijką, można wykonać też zardzewiałym cyrklem i linijką?
\end{problem}
% Hartshorne s. 26

\subsection{Wyrocznia w Delfach}

\subsection{Wielokąty foremne}
GUZICKI-12 **wielokąty foremne** które są konstruowalne? (tw. wantzla itd.) konstrukcje przybliżone pięciokąta - durer i da vinci.
$n = 3$, $n = 4$, $n = 6$ (proste)

\begin{problem}
    Skonstrować trójkąt równoboczny wpisany w okrąg, którego środek nie jest znany. \hfill \emph{(7 kroków)}
\end{problem}

\begin{problem}
    Skonstrować kwadrat. \hfill \emph{(9 kroków)}
\end{problem}

\begin{problem}
    Skonstrować pięciokąt foremny.
\end{problem}

Piszą o tym Hartshorne \cite[s. 45-49]{hartshorne2000}.
Jeśli mamy zadany jeden z jego boków, konstrukcja wymaga 11 kroków. % Hartshorne s. 51



$n = 17$

$n = 7$ (niemożliwe), możliwe ze znaczoną linijką: Hartshorne rozdział 30/31

\subsection{Stożkowe}
przecięcie prostej z parabolą (hartshorne s. 247)

s. 278 Hartshorne: problem Alhazen, równokąty widziane z dwóch punktów na okręgu

\subsection{Apolloniusz}
GUZICKI-19 **zadanie konstrukcyjne apolloniusza** wykorzystuje twierdzenie menelaosa



%

W 1803 roku Malfatti \cite{malfatti_1803} zainspirowany pewnym praktycznym zagadnieniem (wycinanie walców z graniastosłupa) postawi następujący problem:
\index[persons]{Malfatti, Gian Francesco}%

\begin{problem}[zadanie Malfattiego]
	\label{malfatti_problem}
	\index{zadanie!Malfattiego}%
	Dany jest trójkąt $\triangle ABC$.
	Skonstruować takie trzy parami styczne okręgi $\Gamma_A, \Gamma_B, \Gamma_C$, że okrąg $\Gamma_A$ (odpowiednio: $\Gamma_B$, $\Gamma_C$) jest wpisany w~kąt $\angle A$ (odpowiednio: $\angle B$, $\angle C$).
\end{problem}

\begin{figure}[H]
\begin{center}
\begin{tikzpicture}[scale=.5]
\tkzDefPoints{0/0/A,10/2/B,6/7/C}
\tkzDefPoints{4.43012726/2.59439459/Oa}
\tkzDefCircle[R](Oa,1.67519375895) \tkzGetPoint{Oaa}
\tkzDrawCircle[line width=0.2mm](Oa,Oaa)

\tkzDefPoints{7.48168986/2.91734309/Ob}
\tkzDefCircle[R](Ob,1.39341015784) \tkzGetPoint{Obb}
\tkzDrawCircle[line width=0.2mm](Ob,Obb)

\tkzDefPoints{5.96721113/5.06490116/Oc}
\tkzDefCircle[R](Oc,1.23445046858) \tkzGetPoint{Occ}
\tkzDrawCircle[line width=0.2mm](Oc,Occ)

\tkzLabelPoint(A){$A$}
\tkzLabelPoint[anchor=center](Oa){$\Gamma_A$}
\tkzLabelPoint(B){$B$}
\tkzLabelPoint[anchor=center](Ob){$\Gamma_B$}
\tkzLabelPoint[above](C){$C$}
\tkzLabelPoint[anchor=center](Oc){$\Gamma_C$}
\tkzDrawPolygon[line width=0.3mm](A,B,C)
\end{tikzpicture}
\end{center}
\caption{Trzy okręgi Malfattiego}
\end{figure}

Problem będzie rozważany na długo przed Malfattim, zajmie się nim Ajima Naonobu\footnote{Matematyk japoński, przypisze się mu wprowadzenie rachunku różniczkowo-całkowego do matematyki japońskiej.} w~XVIII wieku, a~jeszcze wcześniej Gilio de Cecco da Montepulciano w~rękopisie z~1384 roku.
\index[persons]{Ajima, Naonobu}%
\index[persons]{de Cecco da Montepulciano, Gilio}%

Malfatti wyprowadzi co następuje.
Niech $p$ będzie połową obwodu trójkąta, $r$ będzie promieniem okręgu wpisanego w~ten trójkąt zaś $d_A$, $d_B$, $d_C$ odległościami wierzchołków $A, B, C$ od środka tego okręgu.
Wtedy promienie okręgów Malfattiego wyrażają się wzorami
\begin{align}
	r_A & = \frac r 2 \cdot {\frac {s-r+d_A-d_B-d_C}{p-a}}, \\
	r_B & = \frac r 2 \cdot {\frac {s-r+d_B-d_A-d_C}{p-b}}, \\
	r_C & = \frac r 2 \cdot {\frac {s-r+d_C-d_A-d_B}{p-c}}.
\end{align}

Prostą konstrukcję okręgów opartą na dwustycznych zawdzięczymy Steinerowi \cite{steiner_1826} w~1826 roku;
\index[persons]{Steiner, Jakob}%
inne rozwiązania podadzą Lehmus \cite{lehmus_1819}, Catalan \cite{catalan_1846}, Adams \cite{adams_1846}, Derousseau \cite{derousseau_1895}, Pampuch \cite{pampuch_1904}.
% TODO: po poprawie bibliografii, podać tu index persons

(O~problemie napiszą też Bogdańska, Neugebauer \cite[s. 102]{neugebauer_2018}).

Malfatti postawi tak naprawdę inny problem: znalezienia trzech rozłącznych kół zawartych w~trójkącie, których suma pól jest maksymalna i~błędnie założy, że opisane wyżej okręgi stanowią rozwiązanie.
Pomyłkę zauważą najpierw bez dowodu Lob, Richmond \cite{lob_richmond_1930} w~1930 roku: z trójkąta równobocznego można wyciąć zachłannie kolejno trzy koła, ich łączna powierzchnia jest większa od powierzchni kół znalezionych przez Malfattiego o 1\%.
\index[persons]{Richmond, ?}%
\index[persons]{Lob, ?}%
Howard Eves powtórzy to dla stromych trójkątów równoramiennych o bardzo wąskiej podstawie i dużej wysokości około 1946 roku.
\index[persons]{Eves, Howard}%
% https://en.wikipedia.org/w/index.php?title=Howard_Eves&diff=831382284&oldid=750910758
Goldberg \cite{goldberg_1967} wykaże, że domniemanie Malfattiego nie daje nigdy kół o maksymalnej łącznej powierzchni.
Ostatnie słowo należy zaś do Zalgallera, Losa \cite{zalgaller_los_1992}, którzy znajdą trzy koła rozwiązujące problem Malfattiego w dowolnym trójkącie.
% TODO: Goldberg M., On the original Malfatti problem, Math. Mag. 40 (1967), 241–247.
\index[persons]{Zalgaller, VA?}%
\index[persons]{Los, GA?}%
% TODO: Zalgaller V.A., Los’ G.A., Solution of the Malfatti problem, Ukrain. Geom. Sb. 35 (1992), 14–33 (ang. J. Math. Sci. 72 (1994), 3163–3177).
% TODO: po poprawie bibliografii, podać tu index persons
% TODO: Lob, H.; Richmond, H. W. (1930), "On the Solutions of Malfatti's Problem for a Triangle", Proceedings of the London Mathematical Society, 2nd ser., 30 (1): 287–304, doi:10.1112/plms/s2-30.1.287.

Kryształowa kula nie potrafi przewidzieć, kto oceni, czy algorytm zachłanny zawsze znajduje $n \ge 4$ rozłącznych kół w trójkącie o maksymalnej łącznej powierzchni.

(O więcej niż jednym okręgu wpisanym w trójkąt pisaliśmy w podpodsekcji \ref{sssection_6_7_9_circles}).


\color{red}

\begin{problem}[zadanie Napoleona]
	Podzielić dany okrąg (bez znanego środka) na cztery łuki równej miary korzystając z cyrkla, ale nie linijki.
\end{problem}

Nie wiadomo, czy Napoleon wymyślił albo rozwiązał przedstawione wyżej zadanie konstrukcyjne.
Rozwiązanie: \cite[s. 116]{neugebauer} z wykorzystaniem okręgów Torricelliego.
\index{okrąg Torricelliego}%



% TODO: rozwiązanie https://en.wikipedia.org/wiki/Napoleon%27s_problem

\color{black}

Konstrukcje od \ref{delta_2024_12_start} do \ref{delta_2024_12_end} opisane są w czasopiśmie Delta, w numerze grudniowym z 2024 roku.
% TODO: opisać wszystkie siedem konstrukcji

\begin{geoconstruction}
    \label{delta_2024_12_start}
    Znając pięć punktów okręgu $\omega$, skonstruować styczną do $\omega$ w jednym z tych punktów.
\end{geoconstruction}
% Niech tymi punktami będą A, B, C, D, E. Przecinamy AB i CD w P, AC i BE w Q oraz PQ i DE w R. Wówczas prosta AR jest szukaną styczną (rys. 1). Podkreślmy, że do przeprowadzenia powyższej konstrukcji nie potrzebowaliśmy mieć narysowanego całego okręgu ω – wystarczyło tylko pięć znajdujących się na nim punktów. Uzasadnienie poprawności wymaga znajomości twierdzenia Pascala (patrz Deltoid z ∆9 14).

\begin{geoconstruction}
    Znając pięć punktów okręgu $\omega$, dla danej prostej $l$ przechodzącej przez jeden z nich wyznaczyć drugi punkt przecięcia $l$ i $\omega$.
\end{geoconstruction}

% W przypadku problemów ze znalezieniem rozwiązania polecam poszukać go
% w ∆6 17. Jesteśmy już gotowi do znalezienia środka okręgu samą linijką, jeśli
% mamy do dyspozycji jeszcze jeden, przecinający go okrąg.
% Konstrukcja 3. 
% Oznaczmy te okręgi przez ω1 i ω2, a ich punkty przecięcia przez A i B.
% Korzystając z konstrukcji 1, konstruujemy styczną do ω1 w punkcie B
% i przecinamy z ω2 w C. Przez A rysujemy prostą, która przecina ω1 w D,
% a ω2 w E. Oznaczmy przez F drugi punkt przecięcia prostej BD z ω2. Na koniec
% niech P będzie przecięciem BC i EF, a Q przecięciem BE i CF. Wówczas
% ?EFB= ?BAD= 180◦
% −?DBA−?ADB= 180◦
% −?DBA−?ABC= ?CBF.
% Oznacza to, że EB= CF, a prosta PQ zawiera średnicę ω2 (rys. 2).
% Po wybraniu innej prostej przechodzącej przez A skonstruujemy inną średnicę,
% i w konsekwencji środek ω2.
% Czytelnik z pewnością sam bez problemu wymyśli konstrukcje środka okręgu
% przy zadanych dwóch okręgach stycznych, a także przy zadanych dwóch
% okręgach współśrodkowych.
% W kolejnych konstrukcjach przyda się kilka pojęć.

\begin{geoconstruction}
    Skonstruować środek jednego z dwóch okręgów mających dwa punkty wspólne.
\end{geoconstruction}

% Rozważmy okrąg ω i dowolny punkt P nieleżący na tym okręgu. Przez punkt P
% poprowadźmy dwie sieczne, które przecinają ω w A i B oraz C i D. Niech proste
% AD i BC przecinają się w Q, a AC i BD przecinają się w R. Prostą QRbędziemy
% nazywać biegunową punktu P względem okręgu ω (rys. 3). Zauważmy, że może
% ona być wyznaczona wyłącznie przy użyciu linijki, nawet jeśli okrąg ω dany jest
% tylko w pięciu punktach (w takim przypadku korzystamy z konstrukcji 2).
% Biegunowe mają liczne i użyteczne własności. Na przykład jeśli P leży na
% zewnątrz ω,to biegunowa P przechodzi przez punkty styczności prostych stycznych
% do ω przechodzących przez P. Stąd dla punktów leżących na okręgu przyjmujemy,
% że biegunową jest styczna w tym punkcie. Zatem, wyznaczając biegunową, możemy
% skonstruować styczną do okręgu przechodzącą przez punkt na nim nieleżący.
% Inną użyteczną własnością jest fakt, że każda sieczna okręgu ω
% przechodząca przez P przecina ω w takich punktach A, B oraz biegunową
% P w takim Q, że AB dzieli harmonicznie PQ, tzn. AP
% BP = AQ
% BQ. Czytelnik
% może spróbować wymyślić, jak podzielić harmonicznie odcinek przy
% użyciu wyłącznie linijki (podpowiedź: warto przypomnieć sobie
% twierdzenia Cevy i Menelaosa).
% Kolejnym przydatnym obiektem będzie pęk okręgów. Jest to rodzina
% okręgów, którą jednoznacznie wyznaczają dwa niewspółśrodkowe okręgi.
% Pęki okręgów mają taką własność, że jeśli dwa okręgi należące do pęku
% przecinają się w dwóch punktach, to każdy okrąg z tego pęku przechodzi
% przez te dwa punkty (rys. 4), jeśli są styczne, to wszystkie są do siebie
% styczne w tym samym punkcie, oraz jeśli się nie przecinają, to żadne
% dwa się nie przecinają (rys. 5). Na potrzeby tego artykułu potraktujmy
% pęki okręgów jako „czarną skrzynkę”, zainteresowanych szczegółami
% odsyłam do krótkiego tekstu w tym wydaniu Delty (s. 20), który jest
% im poświęcony.
% Zachodzi następujące twierdzenie:
% Twierdzenie. Biegunowe dowolnego punktu P względem okręgów
% należących do jednego pęku są współpękowe.
% Punkt ten będziemy nazywali biegunowo sprzężonym do punktu P względem
% odpowiedniego pęku. Ponieważ pęk jest wyznaczony przez dwa okręgi,
% możemy też mówić o dwóch punktach sprzężonych względem pary okręgów.
% Powyższe twierdzenie wykorzystamy w kolejnych konstrukcjach. Ponieważ
% linijka nie pozwala na narysowanie okręgu, przez wyrażenie „skonstruować
% okrąg” będziemy określać wyznaczenie dowolnie wielu jego punktów.
% Konstrukcja 4. Mając dane okręgi λ i µ oraz punkt A na zewnątrz jednego
% z nich, skonstruować okrąg przechodzący przez A oraz należący do pęku
% wyznaczanego przez te okręgi.
% Niech A leży na zewnątrz okręgu λ. Z punktu A skonstruujmy styczną do λ
% w punkcie B. Następnie niech C będzie punktem biegunowo sprzężonym do
% punktu B względem λ i µ (zauważmy, że leży na AB). Konstruujemy teraz taki
% punkt D, że AD dzieli harmonicznie BC. Punkt D jest drugim obok A punktem
% szukanego okręgu (rys. 6). Gdyby okazało się, że D= C= A (tzn. gdyby AB było
% styczne do konstruowanego okręgu), to na początku konstrukcji powinniśmy wziąć
% „drugą styczną” z A do λ. Całą procedurę możemy teraz powtórzyć, biorąc D jako
% punkt startowy (i oczywiście punkt styczności do λ różny od B).

% Konstrukcja 5. Mając dane okręgi λ i µ oraz punkt A leżący wewnątrz nich,
% skonstruować okrąg przechodzący przez A oraz należący do pęku wyznaczanego
% przez te okręgi.
% W tym przypadku wyznaczamy punkt B, biegunowo sprzężony do A. Punkt ten
% leży na zewnątrz okręgów λ i µ, zatem możemy skonstruować dowolną liczbę
% punktów okręgu β przechodzącego przez B i należącego do pęku wyznaczanego
% przez te dwa okręgi (konstrukcja 4). Punkt A leży na zewnątrz β. Pokażemy, jak
% wykorzystać ten „dziurkowany” okrąg do odtworzenia konstrukcji 4.
% Problematyczny jest tylko pierwszy krok, czyli konstrukcja stycznej do β.
% Aby ją wyznaczyć, postępujemy następująco. Niech C będzie różnym od B
% punktem okręgu β. Wyznaczmy punkt D przecięcia prostej AC z okręgiem β
% (korzystamy z konstrukcji 2). Dalej konstruujemy taki E na AC, że AE dzieli
% harmonicznie CD. Prosta BE jest biegunową punktu A względem β, więc jej
% drugi punkt przecięcia z β to taki punkt F (rys. 7), że AF jest styczna do β
% (ponownie skorzystaliśmy z konstrukcji 2). Teraz na AF możemy wyznaczyć
% drugi obok A punkt szukanego okręgu i powtórzyć procedurę, rozpoczynając od
% tego punktu.
% Konstrukcja 6. Skonstruować środek przynajmniej jednego z czterech okręgów,
% z których żadne trzy nie należą do jednego pęku.
% Oznaczmy dane okręgi przez κ, λ, µ, ν. Zakładamy, że żadne dwa z nich nie
% mają punktów wspólnych ani nie są współśrodkowe.
% Wybierzmy punkt A na κ. Konstruujemy okręgi α i β przechodzące przez A oraz
% należące do pęków wyznaczonych odpowiednio przez λ i µ oraz µ i ν. Następnie
% wybieramy taki punkt B na α, że skonstruowana styczna w B do α przecina
% okrąg κ. Niech C będzie tym punktem przecięcia. Niech ponadto D i F będą
% punktami biegunowo sprzężonymi do punktów odpowiednio B i C względem
% pęku wyznaczonego przez okręgi α i β (rys. 8).
% Zauważmy, że E taki, że CE dzieli harmonicznie BD, leży na
% okręgu γ należącym do pęku wyznaczonego przez α i β oraz
% przechodzącym przez C (rozważamy ten okrąg, ale go nie
% konstruujemy). Ponadto prosta CF jest styczna do γ. Czytelnik,
% γ
% analizując ponownie rysunek 2, przekona się, że mamy wystarczająco
% danych, aby zastosować konstrukcję 3 dla okręgów γ i κ i uzyskać
% E
% średnicę κ. Drugą średnicę konstruujemy, zaczynając od innego
% punktu A.
% Odnotujmy, że konstrukcję da się powtórzyć, jeśli jeden z okręgów
% (u nas okrąg µ) jest dany tylko w pięciu punktach. Wynika to
% z możliwości wykonania konstrukcji, gdy okrąg µ jest dany tylko
% w 5 punktach, co z kolei jest konsekwencją poczynionej wcześniej
% uwagi o konstruowalności biegunowych.

\begin{geoconstruction}
    \label{delta_2024_12_end}
    Skonstruować środek przynajmniej jednego z trzech okręgów nienależących do jednego pęku.
\end{geoconstruction}
% Oznaczmy te okręgi przez κ, λ, µ. Wybieramy punkt A na κ i konstruujemy
% okrąg α należący do pęku wyznaczanego przez okręgi λ i µ oraz przechodzący
% przez A. Na okręgach κ i α wybieramy odpowiednio punkty B i C. Następnie
% prowadzimy dowolną prostą przez A i oznaczamy jej punkty przecięcia z κ
% i α przez P i Q, odpowiednio. Zauważmy, że jeśli prosta PQ będzie obracać
% się wokół punktu A, to punkt przecięcia R prostych PB i QC będzie zakreślał
% okrąg (jest to okrąg opisany na trójkącie, którego wierzchołkami są punkty B,
% C i różny od A punkt przecięcia α i κ). Oznaczmy go przez ν. Rysując zatem
% kolejne położenia prostej PQ, będziemy mogli konstruować kolejne punkty
% okręgu ν (rys. 9). Żadne trzy spośród κ, λ, µ, ν nie należą do jednego pęku.
% Stąd po skonstruowaniu pięciu punktów ν możemy powtórzyć konstrukcję 6.
% Czytelnikowi zastanawiającemu się, co z przypadkiem rozłącznych okręgów
% należących do jednego pęku, odpowiem, że wówczas środka okręgu nie da się
% skonstruować. Omówienie tego zagadnienia byłoby jednak zbyt długie, aby
% można je było zawrzeć w tym artykule.


%

\input{section-03/section-03}

\section{Stereometria (wielościany)}
Tekst sekcji stereometria, na podstawie 8 rozdziału Hartshorne'a.



\subsection{Twierdzenie Sylvestera-Gallaia}
\begin{theorem}[Sylvestera-Gallaia]
	Dla każdego skończonego zbioru punktów na płaszczyźnie istnieje prosta, która przechodzi przez dokładnie dwa albo wszystkie punkty.
\end{theorem}

Mamy wrażenie, że zaczęło się w 1893 roku, kiedy James Sylvester postawił problem.
Być może zainspirowała go konfiguracją Hessego\footnote{Konfiguracja Hessego to 12 prostych przez 9 punktów na zespolonej płaszczyźnie rzutowej, gdzie każdy punkt leży na 4 prostych, a każda prosta przechodzi przez 3 punkty}.
Herbert Woodall szybko zaproponował rozwiązanie, gdzie równie szybko wychwycono usterkę.
Dopiero w 1941 roku Eberhard Melchior udowodnił trochę mocniejsze stwierdzenie niż rzutowy dual ówczesnej hipotezy (że prostych przez dokładnie dwa punkty jest co najmniej trzy).
Nieświadomy tego, Paul ErdErdős postawił hipotezę na nowo w~1943 roku, a Tibor Gallai w 1944 roku dodał swój dowód (ponownie wykorzystując elementy geometrii rzutowej).
Wraz z upływem czasu pojawiały się inne, ciekawe rozumowania.
Na przykład Leroy Kelly wykorzystał własności metryki, co oburzyło Harolda Coxetera i skłoniło go do opublikowania kolejnego dowodu, korzystającego jedynie z aksjomatów geometrii uporządkowania.
(Aigner, Ziegler uważają dowód Kelly'ego za najlepszy).

Niech $t_2(n)$ oznacza minimalną liczbę prostych przez dwa punkty w dowolnym ułożeniu $n$ punktów.
Melchior pokazał, że $t_2(n) \ge 3$.
Wynik sukcesywnie poprawiano:
de Bruijn \cite{debruijn_1948} zapytał, czy $t_2(n)$ dąży do nieskończoności,
Theodore Motzkin \cite{motzkin_1951} udzielił twierdzącej odpowiedz, bo $t_2(n) \ge \sqrt{n}$.
Potem Gabriel Dirac \cite{dirac_1951} przypuścił, że $t_2(n) \ge \lfloor n/2\rfloor$, co nie zostawia wiele miejsca na poprawki, bo dla parzystych $n \ge 6$ zachodzi $t_2(n) \le n/2$, jak pokazał pomysłową konstrukcją Károly Böröczky.
Dla nieparzystych $n$ wiemy tylko, że ten kres jest realizowany dla $n = 7$ (Kelly, Moser \cite{kelly_1958} w 1958) i $n = 13$ (Crowe, McKee \cite{mckee_1968} w 1968).
Najnowszy wynik, o jakim nam wiadomo, to Csimy, Sawyera \cite{csima_1993}: że $t_2(n) \ge \lceil 6n/13 \rceil$.

\subsection{Inwersje (UW-2)}
\begin{enumerate}
	\item Obrazy inwersyjne okręgów i prostych, konforemność inwersji, okręgi stałe inwersji, okręgi prostopadłe
	\item zmiana odległości przy inwersji, zmiana promienia okręgu przy inwersji,
	\item twierdzenie Ptolemeusza,
	\item łańcuchy Steinera
	\item formuła Kartezjusza
	\item formuła Fussa dla czworokątów,
	\item twierdzenie Feuerbacha.
\end{enumerate}

\subsection{Stożkowe (UW-2)}
\begin{enumerate}
	\item Ogniska elipsy i hiperboli, ognisko, kierownica i mimośród stożkowych, asymptoty hiperboli, konstrukcja stycznej do stożkowej, rzuty ustalonego ogniska na styczne, własności izogonalne stożkowych, równania kanoniczne stożkowych, elipsa jako przekrój walca.
	\item Ognisko, kierownica i mimośród stożkowej na przekroju stożka.
	\item Przekroje stożków ze sferami wpisanymi.
	\item Równanie ogólne stożkowej w układzie współrzędnych, duży i mały wyznacznik.
	\item Równania stożkowych we współrzędnych biegunowych.
\end{enumerate}

Neugebauer 262: w każdy właściwy czworobok zupełny da się wpisać dokładnie jedną parabolę, jej ogniskiem jest punkt Miquela czworoboku.
Jemieljanow: punkt Miquela właściwego czworoboku zupełnego leży na okręgu dziewięciu punktów trójkąta przekątnego tego czworoboku.
Droz-Farny: proste przechodzą przez ortocentrum trójkąta i są prostopadłe, wtedy środki odcinków leżą na jednej prostej.

\subsection{Przekształcenia afiniczne (UW-2)}
\begin{enumerate}
	\item Grupa przekształceń afinicznych od strony geometrycznej: powinowactwa osiowe, rozkład przekształcenia afinicznego na podobieństwo i powinowactwo osiowe, kierunki główne przekształcenia afinicznego.
	\item niezmienniczość stosunku pól przy przekształceniu afinicznym
	\item obraz okręgu przy przekształceniu afinicznym
\end{enumerate}

\subsection{UW-3}
\begin{enumerate}
	\item zna pojęcie płaszczyzny rzutowej rzeczywistej (równoważne sformułowania), dwustosunku, definicję przekształceń rzutowych łańcuchów, pęków, stożkowych, pęków stycznych do stożkowych. 
	\item Rozumie, czym są stożkowe w ujęciu rzutowym, zna typy stożkowych.
	\item Zna i potrafi stosować twierdzenia Steinera i Braikenridge'a-Maclaurina.
	\item Wie w jaki sposób określa się rzutowo ogniska i kierownice stożkowych.
\end{enumerate}

\subsection{Guzicki}
\begin{enumerate}
	\item Złoty podział i pięciokąt.
	\item Zagadnienie izoperymetryczne (6)
	\item nierówności geometryczne: stosunek sumy środkowych do obwodu leży między 3/4 i 1 (s. 355), $s <= p^2 / 3 \sqrt 3$ - przypomnienie nierówności izoperymetrycznej. nierówność eulera (R >= 2r), Mitrinovica, Leibniza, Weitzenbocka (s. 362). Twierdzenie Eulera: $d^2 = R^2 - 2Rr$. nierówność Erdosa-Mordella: P leży wewnątrz trójkąta, K L M to rzuty na boki. Wtedy PA + PB + PC >= 2 (PK + PL + PM). Mikołaj z Kuzy: $\sin x / x < (2 + \cos x) / 3$. Snellius-Huygens: $2 \sin x + \tan x > 3x$.
	\item przekątne w wielokącie, tw. Heinekena % n nieparzyste -> w n-kącie foremnym żadne trzy przekątne nie przecinają się -> https://arxiv.org/pdf/math/9508209v3 ... In the 1960s, Heineken [6] gave a delightful argument which generalized this to all odd n,
\end{enumerate}

\subsection{Starocie}
Twierdzenie Chasles'a: każda izometria płaszczyzny jest złożeniem co najwyżej trzech symetrii osiowych.
Symetria osiowa z poślizgiem.
Słowo Banacha.
Klasyfikacja podobieństw.
Okrąg siedmiu punktów. % https://mathworld.wolfram.com/BrocardCircle.html ?
Przekształcenia afiniczne i rzutowe.
% https://www.cut-the-knot.org/Curriculum/Geometry/HeronsProblem.shtml
% This one is a basic optimization problem. It's quite famous, being discussed in Heron's Catoptrica (On Mirrors from the Greek word Katoptron Catoptron = Mirror) that, in all likelihood, saw the light of day some 2000 years ago.
Pitagorasa % https://en.wikipedia.org/wiki/Pythagorean_theorem
% https://en.wikipedia.org/wiki/Spiral_of_Theodorus

gnomon % https://en.wikipedia.org/wiki/Theorem_of_the_gnomon

Czwarty aksjomat uporządkowania znalazł Moritz Pasch \cite{pasch_1882} w 1882 roku.
\index[persons]{Pasch, Moritz}
\index{aksjomat!Pascha}

\subsection{Zadania}
\textbf{Zadanie} (Guzicki, s. 304).
Na bokach $AB$, $BC$, $CD$ i $DA$ czworokąta wypukłego $ABCD$ zbudowano, na zewnątrz czworokąta, kwadraty $ABFE$, $BCHG$, $CDJI$ i $DALK$.
Punkty $P$, $Q$, $R$ i $S$ są odpowiednio środkami kwadratów $ABFE$, $BCHG$, $CDJI$ i $DALK$.
Udowodnij, że odcinki $PR$ i $QS$ są równej długości oraz wzajemnie prostopadłe.

\textbf{Zadanie} (Guzicki, s. 306).
(XLIV OM, zadanie 5/I).
Dana jest półpłaszczyzna oraz punkty $A$ i $C$ na jej krawędzi.
Dla każdego punktu $B$ tej półpłaszczyzny rozważamy kwadraty $ABKL$ i $BCMN$ leżące na zewnątrz trójkąta $ABC$.
Wyznaczają one odpowiadającą punktowi $B$ prostą $LM$.
Udowodnij, że wszystkie proste odpowiadające różnym położeniom punktu $B$ przechodzą przez jeden punkt.

\textbf{Zadanie} (Guzicki, s. 306).
Na bokach $AB$ i $AC$ trójkąta $ABC$ zbudowano, po jego zewnętrznej stronie, kwadraty $ABDE$ i $ACFG$.
Punkty $M$ i $N$ są odpowiednio środkami odcinków $DG$ i $EF$.
Wyznacz możliwe wartości wyrażenia $MN / BC$.

\textbf{Zadanie} (Guzicki, s. 307)
(TWIERDZENIE NAPOLEONA)
Na bokach $AB$, $BC$ i $CA$ trójkąta $ABC$ zbudowano, na zewnątrz trójkąta, trójkąty równoboczne $ABF$, $BCD$ i $CAE$.
Udowodnij, że środki tych trójkątów równobocznych są wierzchołkami trójkąta równobocznego.

\textbf{Zadanie} (Guzicki, s. 308)
Na bokach $AB$, $BC$ i $CA$ trójkąta $ABC$ wybrano odpowiednio punkty $D$, $E$ i $F$ tak, że $AD : DB = BE : EC = CF : FA$.
Udowodnij, że jeśli trójkąt $DEF$ jest równoboczny, to trójkąt $ABC$ też jest równoboczny.

\textbf{Zadanie} (Guzicki, s. 310)
(XLV OM, zadanie 7/I)
Na zewnątrz czworokąta wypukłego $ABCD$ budujemy trójkąty podobone $APB$, $BQC$, $CRD$, $DSA$ w ten sposób, że kąty $PAB, QBC, RCD, SDA$ są sobie równe i że kąty $PBA, QCB, RDS, SAD$ też są sobie równe.
Udowodnij, że jeśli czworokąt PQRS jest równoległobokiem, to czworokąt $ABCD$ też jest równoległobokiem.

%










% https://bookstore.ams.org/browse?Author=%22A.%20V.%20Akopyan%22
% https://geometry.ru/books/conic_e.pdf












\bibliography{geo-textbook}{}
\bibliographystyle{plain}

\raggedright
\indexprologue{\small Tekst prologu...}
\printindex

\indexprologue{\small Tekst prologu...}
\printindex[persons]

\end{document}







\section{Aksjomatyka}

\input{blurbs/axioms/euclides} % aksjomaty Euklidesa
\input{blurbs/axioms/hilbert} % aksjomaty Hilberta

\section{W przygotowaniu -- Geometria -- Uniwersytet Warszawski}
\input{blurbs/warsaw1}
\input{blurbs/warsaw2}
\input{blurbs/warsaw3}


\section{Trygonometria}
Nie będziemy zanudzać Czytelnika sto sześćdziesięcioma trzema wzorami redukcyjnymi, na funkcje i kofunkcje, na sumy i różnice kątów, blablabla.
Jest pełno innych książek, które to robią.
Opowiemy trochę o etymologii różnych słów i tym, jak trygonometria rozwijać się będzie na przestrzeni wieków, co powinno być dużo ciekawsze.

O funkcjach trygonometrycznych kąta ostrego pisze Guzicki \cite[s. 240-254]{guzicki_2021}.

Współczesne ,,sinus'' i ,,kosinus'' wywodzą się z błędu w tłumaczeniu!
Na pczątku będzie sanskryt i \emph{jyā} (cięciwa łuku, ze starogreckiego χορδή -- sznurek, struna), a potem transkrypcja do arabskiego \emph{jība}, gdzie pomija się samogłoski (!) i prowadzi do jb, słowa bez żadnego znaczenia. % (جب)
Jb zinterpretują jako \emph{jayb}, które już coś znaczy (po angielsku: ,,bosom, pocket, fold''); Gerard z~Cremony przetłumaczy to do łacińskiego \emph{sinus} (czyli ,,zatoki'', ale też ,,zwisającego fałdu togi na piersi'')...
Kosinus to skrót od \emph{complementi sinus} jak w \emph{Canon triangulorum} Edmunda Guntera z 1620 roku; po drodze był jeszcze Fibonacci oraz jego \emph{sinus rectus arcus}, co pomogło w przyjęciu się ostatecznej wersji.

Nazwy dwóch kolejnych funkcji trygonometrycznych także mają korzenie w łacinie, bowiem \emph{tangens} znaczy ,,dotykający'', zaś \emph{secans} to ,,przecinający''.
Ma to sens, jeśli popatrzymy na okrąg jednostkowy.
Wreszcie termin ,,trygonometria'' bierze się z greckiego τρίγωνον (trigonon -- trójkąt) oraz μέτρον (metron -- miara).
% TODO: https://en.wikipedia.org/wiki/History_of_trigonometry#Ancient_Near_East
% TODO: https://en.wikipedia.org/wiki/History_of_trigonometry#Classical_antiquity
% TODO: https://en.wikipedia.org/wiki/History_of_trigonometry#Indian_mathematics
% TODO: https://en.wikipedia.org/wiki/History_of_trigonometry#Chinese_mathematics
% TODO: https://en.wikipedia.org/wiki/History_of_trigonometry#Islamic_world
% TODO: https://en.wikipedia.org/wiki/History_of_trigonometry#European_renaissance
% TODO: https://en.wikipedia.org/wiki/History_of_trigonometry#Modern

Podstawy trygonometrii rozwinie Hipparchos z Nikei: około 140 r.p.n.e opracuje tablicę cięciw, czyli odpowiednik współczesnych sinusów.
Pierwsza księga poświęcona temu niezależnie od astronomii wyjdzie spod pióra Nasira al Dina al-Tusiego, \emph{,,Kitāb al-Shakl al-qattā''} (czyli Rozprawa o czworobokach) około 1250 roku. % https://en.wikipedia.org/wiki/Nasir_al-Din_al-Tusi#Mathematics

\begin{proposition}
	Niech $\alpha, \beta, \gamma$ będą miarami kątów trójkąta o bokach $a, b, c$, polu $S$ i promieniu okręgu opisanego $R$.
	Wtedy
	\begin{equation}
		\tan \alpha + \tan \beta + \tan \gamma = \frac{4S}{a^2 + b^2 + c^2 - 8R^2}
	\end{equation}
\end{proposition}

\begin{theorem}[nierówność Arystarchusa]
\index{nierówność!Arystarchusa}
	% https://en.wikipedia.org/wiki/Aristarchus's_inequality
\end{theorem}

% https://en.wikipedia.org/wiki/List_of_trigonometric_identities#Identities_without_variables

% https://en.wikipedia.org/wiki/Kunstweg

\subsection{Twierdzenie sinusów}
\todofoot{Law of sines}
\index{twierdzenie!sinusów}
% TODO: https://en.wikipedia.org/wiki/Law_of_sines
\todofoot{ca. 1000 - Law of sines is discovered by Muslim mathematicians, but it is uncertain who discovers it first between Abu-Mahmud al-Khujandi, Abu Nasr Mansur, and Abu al-Wafa.}

$$\frac{a}{\sin \alpha} = \frac{b}{\sin \beta} = \frac{c}{\sin \gamma} = 2R$$
Coxeter \cite[s. 28, 29]{coxeter_1967} wyprowadza twierdzenie sinusów z symetrii okręgu opisanego na trójkącie.

\subsection{Twierdzenie cosinusów}
\index{twierdzenie!cosinusów|(}%
Twierdzenie cosinusów jest bardzo stare.
Euklides rozpatrzy je w Elementach osobno dla trójkątów rozwartokątnych (II.12) i ostrokątnych (II.13).
% TODO: https://en.wikipedia.org/wiki/Law_of_cosines The cases of obtuse triangles and acute triangles (corresponding to the two cases of negative or positive cosine) are treated separately, in Propositions II.12 and II.13:[1]
Perski matematyk Jamshid al-Kashi znajdzie wartość $2\pi$ z~dokładnością do szesnastu cyfr, będzie autorem najdokładniejszych tablic trygonometrycznych swoich czasów i poda równoważną postać wzoru,
\index[persons]{al-Kashi, Jamshid}
\begin{equation}
	c = \sqrt{(b - a \cos \gamma)^2 + (a \sin \gamma)^2}
\end{equation}
dla ostrego kąta $\gamma$.
Taka sama metoda rozwiązywania trójkątów pojawi się w Europie w 1464 roku, kiedy Regiomontanus opublikuje \emph{De triangulis omnimodis} (czyli ,,O trójkątach wszelkiego rodzaju'').
\index[persons]{Regiomontanus}%
Współczesna forma twierdzenia to zasługa François Viète'a.
\index[persons]{Viète, François}%

\begin{proposition}[twierdzenie cosinusów]
	W trójkącie o bokach długości $a, b, c$ z kątem $\gamma$ naprzeciwko krawędzi $c$ zachodzi
	\label{twierdzenie_cosinusow}%
	\begin{equation}
		c^2 = a^2 + b^2 - 2ab \cos \gamma.
	\end{equation}
	% https://en.wikipedia.org/wiki/Law_of_cosines
\end{proposition}

https://www.deltami.edu.pl/2017/04/niemozliwe-wycinanki/

Znamy różne dowody tego twierdzenia: korzystające z twierdzenia Pitagorasa, trzech wysokości trójkąta, twierdzenia Ptolemeusza, geometrii koła albo twierdzenia sinusów.
We Francji twierdzenie cosinusów do późna będzie nazywane \emph{théorème d'Al-Kashi}, na cześć wspomnianego wyżej Persa.

Piszą o nim Guzicki \cite[s. 258]{guzicki_2021}.

\begin{theorem}[Stewarta, 1746]
\index{twierdzenie!Stewarta}
	W trójkącie $\triangle ABC$ o bokach długości $a, b, c$ poprowadzono czewianę z~wierzchołka $C$ do boku $AB$ o długości $d$, dzieląc ten bok na odcinki długości $n$ oraz $m$, jak na rysunku:
	\begin{center}
\begin{comment}
    \begin{tikzpicture}[scale=.4]
        %\tkzInit[xmin=-0.5,xmax=6.5, ymin=-0.5,ymax=4.5]
        % \tkzClip
        \tkzDefPoint(0, 0){A}
		\tkzDefPoint(3.25, 3.25){d}

		\tkzDefPoint(6, 0){AB}
        \tkzDefPoint(10, 0){B}
        \tkzDefPoint(1, 7){C}
        \tkzDefPoint(35:4.75){CC}
		\tkzDrawSegments(C,AB)
        \tkzDrawPolygon[line width=0.3mm](A,B,C)

        \tkzLabelPoint[below left](A){$A$}
        \tkzLabelPoint[below right](B){$B$}
        \tkzLabelPoint[above](C){$C$}

        \tkzLabelPoint(d){$d$}

		\tkzDrawPoints[size=3,color=black,fill=black!80](A,B,C,AB)
		\tkzDrawSegment[dim={$\,\,c\,\,$,-16pt,transform shape}](A,B)
		\tkzDrawSegment[dim={$\,\,n\,\,$,-8pt,transform shape}](A,AB)
		\tkzDrawSegment[dim={$\,\,m\,\,$,-8pt,transform shape}](AB,B)
		\tkzDrawSegment[dim={$\,\,b\,\,$,8pt,transform shape,sloped}](A,C)
		\tkzDrawSegment[dim={$\,\,a\,\,$,-8pt,transform shape,sloped}](B,C)
    \end{tikzpicture}
\end{comment}
    \end{center}
	Wtedy
	\begin{equation}
		b^2 m + c^2 n = a (d^2 + mn).
	\end{equation}
\end{theorem}

Twierdzenie bez dowodu opublikuje Matthew Stewart w 1746 roku, chociaż Coxeter przypuści, że mogło być znane nawet Archimedesowi.
\index[persons]{Stewart, Matthew}%
\index[persons]{Archimedes}%
% Coxeter, H.S.M.; Greitzer, S.L. (1967), Geometry Revisited, New Mathematical Library #19, The Mathematical Association of America, ISBN 0-88385-619-0 strona 6
Dowody podadzą też Simpson, Euler, Carnot i pewnie jeszcze ktoś.
\index[persons]{Simpson, Thomas}%
\index[persons]{Euler, Leonhard}%
\index[persons]{Carnot, L.M.N.}%
W przyszłości często pokazywać się będzie je jako zastosowanie twierdzenia cosinusów, tak jak Guzicki \cite[s. 265]{guzicki_2021}, Bogdańska, Neugebauer \cite[s. 90-91]{neugebauer_2018}; inne rozwiązanie skorzysta z odcinków skierowanych.
Znalezienie go to ćwiczenie podane przez Evesa \cite[s. 58]{eves1_1972}.

\begin{corollary}[twierdzenie Apoloniusza]
	% https://en.wikipedia.org/wiki/Apollonius%27s_theorem
	The theorem is found as proposition VII.122 of Pappus of Alexandria's Collection (c. 340 AD). It may have been in Apollonius of Perga's lost treatise Plane Loci (c. 200 BC), and was included in Robert Simson's 1749 reconstruction of that work.[1]
\end{corollary}

\begin{corollary}
	W trójkącie $\triangle ABC$ o bokach długości $a, b, c$ poprowadzono środkową oraz dwusieczną z~wierzchołka $C$.
	Długość środkowej wynosi
	\begin{equation}
		\sqrt{\frac{a^2 + b^2}{2} - \frac{c^2}{4}},
	\end{equation}
	zaś dwusieczna ma długość
	\begin{equation}
		\frac{\sqrt{ab (a+b+c)(a+b-c)}}{a+b}.
	\end{equation}
\end{corollary}

\begin{corollary}[z twierdzenia Stewarta]
	Odległość między środkiem ciężkości i środkiem okręgu opisanego na trójkącie wynosi
	\begin{equation}
		\sqrt{R^2 - \frac 19 \left(a^2 + b^2 + c^2\right)},
	\end{equation}
	w szczególności więc $(3R)^2 \ge a^2 + b^2 + c^2$, z równością dla trójkąta równobocznego.
\end{corollary}

Wniosek ten znajdziemy u Zetela \cite[s. 72]{zetel_2020}
\index{twierdzenie!cosinusów|)}%

\subsection{Więcej wzorów w powijakach z okręgami}
Wzory na promienie okręgów wpisanych, dopisanych.
$4R = r_a + r_b + r_c - r$ % Coxeter, s. 13; dopisz też bend okręgi Kartezjusza

Cztery okręgi ze znakiem Kartezjusza; Soddy.
Cztery nowe okręgi Beecrofta.
\index{okrąg!Beecrofta}%
% ćwiczenie 2. ze strony 16
% https://en.wikipedia.org/wiki/Descartes%27_theorem
% https://dept.math.lsa.umich.edu/~lagarias/doc/descartes.pdf Soddy-Gossett,

\subsection{Rozwiązywanie trójkątów}
% https://en.wikipedia.org/wiki/Solution_of_triangles

Podamy teraz kilka nowych zależności trygonometrycznych, które pomagają w rozwiązywaniu trójkątów.
Znamy już twierdzenia sinusów i cosinusów; w ogólności to drugie jest bezpieczniejsze od pierwszego (ponieważ z faktu, że $\sin \alpha = \frac 1 2$ nie wynika, czy kąt $\alpha$ jest ostry, czy rozwarty, cosinus zaś jest jednoznaczny).

W każdym z poniższych stwierdzeń mamy trójkąt o kątach $\alpha, \beta, \gamma$, bokach $a, b, c$, połowie obsowdu $p = \frac 1 2 (a + b + c)$, promieniu okręgu opisanego $R$ i wpisanego $r$.

\begin{proposition}
    Zachodzi
    \begin{equation}
    S = 2 R^2 \sin \alpha \sin \beta \sin \gamma.
    \end{equation}
\end{proposition}

\begin{proposition}
    Zachodzi
    \begin{equation}
        p = R (\sin \alpha + \sin \beta + \sin \theta).
    \end{equation}
\end{proposition}

\begin{proposition}
    Zachodzi
    \begin{equation}
        r = 4R \sin \frac \alpha 2 \sin \frac \beta 2 \sin \frac \gamma 2.
    \end{equation}
\end{proposition}

\begin{proposition}[wzór Newtona]
\index{wzór!Newtona}%
    Zachodzi
    \begin{equation}
        \frac{a + b}{c} = \frac{\cos \frac 1 2 (\alpha - \beta)}{\cos \frac 1 2 (\alpha + \beta)}.
    \end{equation}
\end{proposition}

\begin{proposition}[wzór Mollweidego]
\index{wzór!Mollweidego}%
    Zachodzi
    \begin{equation}
        \frac{a - b}{c} = \frac{\sin \frac 1 2 (\alpha - \beta)}{\sin \frac 1 2 (\alpha + \beta)}.
    \end{equation}
\end{proposition}

(Nie ma zgodności co do powyższych nazw).
Nieco bardziej geometryczną wersję wzorów podał Izaak Newton w 1707 roku, potem Friedrich von Oppel w 1746.
\index[persons]{Newton, Izaak}%
\index[persons]{von Oppel, Friedrich}%
Wyrażenia, których używamy po dziś dzień zawdzięczamy Thomasowi Simpsonowi z 1748 roku, oraz Karlowi Mollweidemu, który opublikował to samo w 1808 roku bez cytowania poprzedników.
\index[persons]{Simpson, Thomas}%
\index[persons]{Mollweide, Karl}%

\begin{proposition}[wzór Regiomontanusa]
\index{wzór!Regiomontanusa}%
    Zachodzi
    \begin{equation}
        \frac{a + b}{a- b} = \frac{\tan \frac 1 2 (\alpha + \beta)}{\tan \frac 1 2 (\alpha - \beta)}.
    \end{equation}
\end{proposition}

Regiomontanus był znany gorzej jako Johannes Müller von Königsberg.
\index[persons]{von Königsberg, Johannes|see{Regiomontanus}}%
\index[persons]{Regiomontanus}%


\subsection{Zastosowania trygonometrii -- twierdzenie Urquharta}
%

\begin{theorem}[Urquharta]
\label{theorem_urquhart}%
\index{twierdzenie!Urquharta}%
    Przy oznaczeniach jak na rysunku:
    \begin{center}
\begin{comment}
    \begin{tikzpicture}[scale=.4]
        \tkzDefPoint(0, 0){E}
        \tkzDefPoint(267:5){E1}
        \tkzDefPoint(250:5){E2}
        \tkzDefPoint(165:5){E3}
        \tkzDefPoint(135:5){E4}
        
        \tkzDefLine[tangent at=E1](E) \tkzGetPoint{tan1}
        \tkzDefLine[tangent at=E2](E) \tkzGetPoint{tan2}
        \tkzDefLine[tangent at=E3](E) \tkzGetPoint{tan3}
        \tkzDefLine[tangent at=E4](E) \tkzGetPoint{tan4}
        
        \tkzInterLL(E4,tan4)(E1,tan1) \tkzGetPoint{S1}
        \tkzInterLL(E4,tan4)(E2,tan2) \tkzGetPoint{S2}
        \tkzInterLL(E3,tan3)(E1,tan1) \tkzGetPoint{S3}
        \tkzInterLL(E3,tan3)(E2,tan2) \tkzGetPoint{S4}

        \tkzInterLL(S3,S4)(S1,S2) \tkzGetPoint{Gorny}
        \tkzInterLL(S1,S3)(S2,S4) \tkzGetPoint{Dolny}

        \tkzLabelPoint[below](S1){$A$}
        \tkzLabelPoint[above left](S2){$D$}
        \tkzLabelPoint(S3){$B$}
        \tkzLabelPoint[above right](S4){$C$}
        \tkzLabelPoint(Dolny){$E$}
        \tkzLabelPoint[right](Gorny){$F$}

        \tkzDrawSegments[line width=0.2mm](S1,Dolny S1,Gorny S3,Gorny S2,Dolny)
        \tkzDrawPoints[size=3,color=black,fill=black!50](S1,S2,S3,S4,Dolny,Gorny)
        % konstrukcja z https://en.wikipedia.org/wiki/Ex-tangential_quadrilateral
\end{tikzpicture}
\end{comment}
    \end{center}
    niech $|AB| + |BC| = |AD| + |CD|$.
    Wtedy $|AE| + |CE| = |AF| + |CF|$.
\end{theorem}

Twierdzenie \ref{theorem_urquhart} nie jest bardzo popularne, napiszą o nim Bogdańska, Neugebauer \cite[s. 97]{neugebauer_2018}, nieznany autor w $\Delta_{84}^{11}$ i pewnie ktoś jeszcze.
Nie do końca wiadomo, kto udowodni je pierwszy: być może będzie to de Morgan w 1841 roku; teza jest też przypadkiem granicznym innego wyniku Chaslesa, znanego około roku 1860.
Wreszcie Dan Pedoe \cite{pedoe_1976} przypisze to twierdzenie Malcolmowi Livingstonowi Urquhartowi\footnote{Matematyk australijski, będzie żyć w latach 1902-1966 i nie opublikuje żadnej pracy.}.
\index[persons]{Pedoe, Daniel}%
\index[persons]{Urquhart, Malcolm Livingston}%

%

\subsection{Zastosowania trygonometrii -- punkt i kąt Crelle'a-Brocarda}

Poznamy teraz dwa wyróżnione punkty trójkąta, opisane w 1875 roku przez oficera francuskiej armii, Henriego Brocarda oraz wiele lat wcześniej przez Augusta Crelle'a \cite{crelle_1816}, założyciela słynnego czasopisma matematycznego.
\index[persons]{Brocard, Henri}%
\index[persons]{Crelle, August}%

\begin{definition}
    Niech $\triangle ABC$ będzie trójkątem.
    Punkt $X$ leżący w jego wnętrzu taki, że kąty $\angle XAB$, $\angle XBC$, $\angle XCA$ są równej miary, nazywamy (pierwszym) punktem Crelle'a-Brocarda.
\end{definition}

\index{punkt Crelle'a-Brocarda}%
\index{kąt Crelle'a-Brocarda}%

Miarę wspomnianych kątów nazywamy kątem Crelle'a-Brocarda.
Jest jeszcze drugi punkt Crelle'a-Brocarda, gdzie odwracamy kolejność punktów: $\angle XBA = \angle XCB = \angle XAC$.

\begin{proposition}
    W każdym trójkącie istnieje (jedyny) punkt Crelle'a-Brocarda.
    Miara kąta Crelle'a-Brocarda $\omega$ spełnia związek
    \begin{equation}
        \cot \omega = \cot \alpha + \cot \beta + \cot \gamma,
    \end{equation}
    gdzie $\alpha, \beta, \gamma$ to miary kątów trójkąta.
    Ponadto, $0 \le \omega \le \pi/6$.
\end{proposition}

Bogdańska, Neugebauer \cite[s. 100]{neugebauer_2018} podają jako ćwiczenie:

\begin{proposition}
    Niech $X$ będzie punktem Crelle'a-Brocarda trójkąta $\triangle ABC$.
    Niech $R_a$, $R_b$ i $R_c$ oznaczają promienie okręgów opisanych na trójkątach $\triangle XBC$, $\triangle AXC$, $\triangle ABX$, zaś $R$ będzie jak zwykle promieniem okręgu opisanego na trójkącie $\triangle ABC$.
    Wtedy
    \begin{equation}
        R = \sqrt[3]{R_a R_b R_c}.
    \end{equation}
\end{proposition}

Peter Yiff \cite{yff_1963} postawił w 1963 roku (!) hipotezę, że $8 \omega^3 \le \alpha \beta \gamma$.
\index[persons]{Yiff, Peter}%
Dowód znalazł w 1974 roku Faruk Abi-Khuzam \cite{abikhuzam_1974}.
\index[persons]{Abi-Khuzam, Faruk}%
% P. Yff, "An analogue of the Brocard points" Amer. Math. Monthly , 70 (1963) pp. 495–501
% F. Abi–Khuzam, "Proof of Yff's conjecture on the Brocard angle of a triangle" Elem. Math. , 29 (1974) pp. 141–142



\subsection{Problem Hansena}
Problem Hansena
\index{problem!Hansena}%

\subsection{Zastosowania trygonometrii -- Problem Snelliusa-Pothenota}
% TODO: https://en.wikipedia.org/wiki/Snellius-Pothenot_problem
Problem Snelliusa-Pothenota.
\index{problem!Snelliusa-Pothenota}%

% https://en.wikipedia.org/wiki/Skinny_triangle

% https://en.wikipedia.org/wiki/Rule_of_marteloio - napisać, że tego nie będzie

% https://en.wikipedia.org/wiki/Mollweide%27s_formula
% https://en.wikipedia.org/wiki/Mollweide's_formula
% https://en.wikipedia.org/wiki/Hansen%27s_problem


\section{Konstrukcje geometryczne}
\subsection{Twierdzenie Mohra-Mascheroniego}
Konstrukcje od \ref{delta_2024_12_start} do \ref{delta_2024_12_end} opisane są w czasopiśmie Delta, w numerze grudniowym z 2024 roku.
% TODO: opisać wszystkie siedem konstrukcji

\begin{geoconstruction}
    \label{delta_2024_12_start}
    Znając pięć punktów okręgu $\omega$, skonstruować styczną do $\omega$ w jednym z tych punktów.
\end{geoconstruction}
% Niech tymi punktami będą A, B, C, D, E. Przecinamy AB i CD w P, AC i BE w Q oraz PQ i DE w R. Wówczas prosta AR jest szukaną styczną (rys. 1). Podkreślmy, że do przeprowadzenia powyższej konstrukcji nie potrzebowaliśmy mieć narysowanego całego okręgu ω – wystarczyło tylko pięć znajdujących się na nim punktów. Uzasadnienie poprawności wymaga znajomości twierdzenia Pascala (patrz Deltoid z ∆9 14).

\begin{geoconstruction}
    Znając pięć punktów okręgu $\omega$, dla danej prostej $l$ przechodzącej przez jeden z nich wyznaczyć drugi punkt przecięcia $l$ i $\omega$.
\end{geoconstruction}

% W przypadku problemów ze znalezieniem rozwiązania polecam poszukać go
% w ∆6 17. Jesteśmy już gotowi do znalezienia środka okręgu samą linijką, jeśli
% mamy do dyspozycji jeszcze jeden, przecinający go okrąg.
% Konstrukcja 3. 
% Oznaczmy te okręgi przez ω1 i ω2, a ich punkty przecięcia przez A i B.
% Korzystając z konstrukcji 1, konstruujemy styczną do ω1 w punkcie B
% i przecinamy z ω2 w C. Przez A rysujemy prostą, która przecina ω1 w D,
% a ω2 w E. Oznaczmy przez F drugi punkt przecięcia prostej BD z ω2. Na koniec
% niech P będzie przecięciem BC i EF, a Q przecięciem BE i CF. Wówczas
% ?EFB= ?BAD= 180◦
% −?DBA−?ADB= 180◦
% −?DBA−?ABC= ?CBF.
% Oznacza to, że EB= CF, a prosta PQ zawiera średnicę ω2 (rys. 2).
% Po wybraniu innej prostej przechodzącej przez A skonstruujemy inną średnicę,
% i w konsekwencji środek ω2.
% Czytelnik z pewnością sam bez problemu wymyśli konstrukcje środka okręgu
% przy zadanych dwóch okręgach stycznych, a także przy zadanych dwóch
% okręgach współśrodkowych.
% W kolejnych konstrukcjach przyda się kilka pojęć.

\begin{geoconstruction}
    Skonstruować środek jednego z dwóch okręgów mających dwa punkty wspólne.
\end{geoconstruction}

% Rozważmy okrąg ω i dowolny punkt P nieleżący na tym okręgu. Przez punkt P
% poprowadźmy dwie sieczne, które przecinają ω w A i B oraz C i D. Niech proste
% AD i BC przecinają się w Q, a AC i BD przecinają się w R. Prostą QRbędziemy
% nazywać biegunową punktu P względem okręgu ω (rys. 3). Zauważmy, że może
% ona być wyznaczona wyłącznie przy użyciu linijki, nawet jeśli okrąg ω dany jest
% tylko w pięciu punktach (w takim przypadku korzystamy z konstrukcji 2).
% Biegunowe mają liczne i użyteczne własności. Na przykład jeśli P leży na
% zewnątrz ω,to biegunowa P przechodzi przez punkty styczności prostych stycznych
% do ω przechodzących przez P. Stąd dla punktów leżących na okręgu przyjmujemy,
% że biegunową jest styczna w tym punkcie. Zatem, wyznaczając biegunową, możemy
% skonstruować styczną do okręgu przechodzącą przez punkt na nim nieleżący.
% Inną użyteczną własnością jest fakt, że każda sieczna okręgu ω
% przechodząca przez P przecina ω w takich punktach A, B oraz biegunową
% P w takim Q, że AB dzieli harmonicznie PQ, tzn. AP
% BP = AQ
% BQ. Czytelnik
% może spróbować wymyślić, jak podzielić harmonicznie odcinek przy
% użyciu wyłącznie linijki (podpowiedź: warto przypomnieć sobie
% twierdzenia Cevy i Menelaosa).
% Kolejnym przydatnym obiektem będzie pęk okręgów. Jest to rodzina
% okręgów, którą jednoznacznie wyznaczają dwa niewspółśrodkowe okręgi.
% Pęki okręgów mają taką własność, że jeśli dwa okręgi należące do pęku
% przecinają się w dwóch punktach, to każdy okrąg z tego pęku przechodzi
% przez te dwa punkty (rys. 4), jeśli są styczne, to wszystkie są do siebie
% styczne w tym samym punkcie, oraz jeśli się nie przecinają, to żadne
% dwa się nie przecinają (rys. 5). Na potrzeby tego artykułu potraktujmy
% pęki okręgów jako „czarną skrzynkę”, zainteresowanych szczegółami
% odsyłam do krótkiego tekstu w tym wydaniu Delty (s. 20), który jest
% im poświęcony.
% Zachodzi następujące twierdzenie:
% Twierdzenie. Biegunowe dowolnego punktu P względem okręgów
% należących do jednego pęku są współpękowe.
% Punkt ten będziemy nazywali biegunowo sprzężonym do punktu P względem
% odpowiedniego pęku. Ponieważ pęk jest wyznaczony przez dwa okręgi,
% możemy też mówić o dwóch punktach sprzężonych względem pary okręgów.
% Powyższe twierdzenie wykorzystamy w kolejnych konstrukcjach. Ponieważ
% linijka nie pozwala na narysowanie okręgu, przez wyrażenie „skonstruować
% okrąg” będziemy określać wyznaczenie dowolnie wielu jego punktów.
% Konstrukcja 4. Mając dane okręgi λ i µ oraz punkt A na zewnątrz jednego
% z nich, skonstruować okrąg przechodzący przez A oraz należący do pęku
% wyznaczanego przez te okręgi.
% Niech A leży na zewnątrz okręgu λ. Z punktu A skonstruujmy styczną do λ
% w punkcie B. Następnie niech C będzie punktem biegunowo sprzężonym do
% punktu B względem λ i µ (zauważmy, że leży na AB). Konstruujemy teraz taki
% punkt D, że AD dzieli harmonicznie BC. Punkt D jest drugim obok A punktem
% szukanego okręgu (rys. 6). Gdyby okazało się, że D= C= A (tzn. gdyby AB było
% styczne do konstruowanego okręgu), to na początku konstrukcji powinniśmy wziąć
% „drugą styczną” z A do λ. Całą procedurę możemy teraz powtórzyć, biorąc D jako
% punkt startowy (i oczywiście punkt styczności do λ różny od B).

% Konstrukcja 5. Mając dane okręgi λ i µ oraz punkt A leżący wewnątrz nich,
% skonstruować okrąg przechodzący przez A oraz należący do pęku wyznaczanego
% przez te okręgi.
% W tym przypadku wyznaczamy punkt B, biegunowo sprzężony do A. Punkt ten
% leży na zewnątrz okręgów λ i µ, zatem możemy skonstruować dowolną liczbę
% punktów okręgu β przechodzącego przez B i należącego do pęku wyznaczanego
% przez te dwa okręgi (konstrukcja 4). Punkt A leży na zewnątrz β. Pokażemy, jak
% wykorzystać ten „dziurkowany” okrąg do odtworzenia konstrukcji 4.
% Problematyczny jest tylko pierwszy krok, czyli konstrukcja stycznej do β.
% Aby ją wyznaczyć, postępujemy następująco. Niech C będzie różnym od B
% punktem okręgu β. Wyznaczmy punkt D przecięcia prostej AC z okręgiem β
% (korzystamy z konstrukcji 2). Dalej konstruujemy taki E na AC, że AE dzieli
% harmonicznie CD. Prosta BE jest biegunową punktu A względem β, więc jej
% drugi punkt przecięcia z β to taki punkt F (rys. 7), że AF jest styczna do β
% (ponownie skorzystaliśmy z konstrukcji 2). Teraz na AF możemy wyznaczyć
% drugi obok A punkt szukanego okręgu i powtórzyć procedurę, rozpoczynając od
% tego punktu.
% Konstrukcja 6. Skonstruować środek przynajmniej jednego z czterech okręgów,
% z których żadne trzy nie należą do jednego pęku.
% Oznaczmy dane okręgi przez κ, λ, µ, ν. Zakładamy, że żadne dwa z nich nie
% mają punktów wspólnych ani nie są współśrodkowe.
% Wybierzmy punkt A na κ. Konstruujemy okręgi α i β przechodzące przez A oraz
% należące do pęków wyznaczonych odpowiednio przez λ i µ oraz µ i ν. Następnie
% wybieramy taki punkt B na α, że skonstruowana styczna w B do α przecina
% okrąg κ. Niech C będzie tym punktem przecięcia. Niech ponadto D i F będą
% punktami biegunowo sprzężonymi do punktów odpowiednio B i C względem
% pęku wyznaczonego przez okręgi α i β (rys. 8).
% Zauważmy, że E taki, że CE dzieli harmonicznie BD, leży na
% okręgu γ należącym do pęku wyznaczonego przez α i β oraz
% przechodzącym przez C (rozważamy ten okrąg, ale go nie
% konstruujemy). Ponadto prosta CF jest styczna do γ. Czytelnik,
% γ
% analizując ponownie rysunek 2, przekona się, że mamy wystarczająco
% danych, aby zastosować konstrukcję 3 dla okręgów γ i κ i uzyskać
% E
% średnicę κ. Drugą średnicę konstruujemy, zaczynając od innego
% punktu A.
% Odnotujmy, że konstrukcję da się powtórzyć, jeśli jeden z okręgów
% (u nas okrąg µ) jest dany tylko w pięciu punktach. Wynika to
% z możliwości wykonania konstrukcji, gdy okrąg µ jest dany tylko
% w 5 punktach, co z kolei jest konsekwencją poczynionej wcześniej
% uwagi o konstruowalności biegunowych.

\begin{geoconstruction}
    \label{delta_2024_12_end}
    Skonstruować środek przynajmniej jednego z trzech okręgów nienależących do jednego pęku.
\end{geoconstruction}
% Oznaczmy te okręgi przez κ, λ, µ. Wybieramy punkt A na κ i konstruujemy
% okrąg α należący do pęku wyznaczanego przez okręgi λ i µ oraz przechodzący
% przez A. Na okręgach κ i α wybieramy odpowiednio punkty B i C. Następnie
% prowadzimy dowolną prostą przez A i oznaczamy jej punkty przecięcia z κ
% i α przez P i Q, odpowiednio. Zauważmy, że jeśli prosta PQ będzie obracać
% się wokół punktu A, to punkt przecięcia R prostych PB i QC będzie zakreślał
% okrąg (jest to okrąg opisany na trójkącie, którego wierzchołkami są punkty B,
% C i różny od A punkt przecięcia α i κ). Oznaczmy go przez ν. Rysując zatem
% kolejne położenia prostej PQ, będziemy mogli konstruować kolejne punkty
% okręgu ν (rys. 9). Żadne trzy spośród κ, λ, µ, ν nie należą do jednego pęku.
% Stąd po skonstruowaniu pięciu punktów ν możemy powtórzyć konstrukcję 6.
% Czytelnikowi zastanawiającemu się, co z przypadkiem rozłącznych okręgów
% należących do jednego pęku, odpowiem, że wówczas środka okręgu nie da się
% skonstruować. Omówienie tego zagadnienia byłoby jednak zbyt długie, aby
% można je było zawrzeć w tym artykule.


Konstruowalna => stopień Q(x) nad Q to potęga 2, ale nie w drugą stronę.
Podwojenie sześcianu.
Trysekcja kąta.
<=: Hartshorne, papierowa strona 245.
17-kąt

\section{Stereometria}
\subsection{Pięć wielościanów}
Hartshorne: rozdział 8

\subsection{Cauchy's rigidity theorem}
Hartshorne: section 45

\subsection{Siamese dodecahedron}
John solid?
% https://en.wikipedia.org/wiki/Johnson_solid




% IMO problems
1959/4.
Construct a right triangle with given hypotenuse c such that the median
drawn to the hypotenuse is the geometric mean of the two legs of the triangle.

1959/5.
An arbitrary point M is selected in the interior of the segment AB. The
squares AM CD and M BEF are constructed on the same side of AB, with
the segments AM and M B as their respective bases. The circles circum-
scribed about these squares, with centers P and Q, intersect at M and also
at another point N. Let N ′ denote the point of intersection of the straight
lines AF and BC.
(a) Prove that the points N and N ′ coincide.
(b) Prove that the straight lines M N pass through a fixed point S indepen-
dent of the choice of M.
(c) Find the locus of the midpoints of the segments P Q as M varies between
A and B.
1959/6.
Two planes, P and Q, intersect along the line p. The point A is given in the
plane P, and the point C in the plane Q; neither of these points lies on the
straight line p. Construct an isosceles trapezoid ABCD (with AB parallel to
CD) in which a circle can be inscribed, and with vertices B and D lying in
the planes P and Q respectively.

% https://www.imo-official.org/problems.aspx


