\documentclass{greaseproof}
\usepackage{lipsum}
\newcommand{\loremipsum}{ {\color{gray}  Lorem ipsum dolor sit amet, consectetur adipiscing elit. Suspendisse nisl purus, ultricies et ante quis, iaculis imperdiet justo. Proin tristique turpis a tortor eleifend lobortis. Vestibulum odio nisi, tempor sed scelerisque ac, aliquet a lorem. Proin ullamcorper nibh eget augue placerat lobortis. Quisque ac commodo libero, fringilla dictum purus. Proin nec massa vitae lorem eleifend lacinia. Praesent rhoncus ultricies ullamcorper. Fusce vel viverra purus, nec rhoncus massa. Etiam nisl lorem, cursus vel est sit amet, mattis venenatis magna. Vestibulum id risus sit amet nisi congue ultrices id nec ex. } }
\begin{document}

% strona pierwsza
\thispagestyle{empty}
{\noindent\fontsize{18pt}{18pt}\selectfont Biblioteka Aleksandryjska, tom I}

\noindent\makebox[\linewidth]{\rule{\textwidth}{1pt}}

\newpage

% strona druga
\thispagestyle{empty}
\phantom{nothing}
\newpage

% strona trzecia
\thispagestyle{empty}
{\noindent\fontsize{18pt}{18pt}\selectfont Epafrodyt z Ptolemais}

\noindent\makebox[\linewidth]{\rule{\textwidth}{1pt}}

\vspace{10mm}

{\noindent\fontsize{24pt}{24pt}\selectfont \textbf{Geometria}}
\vspace{10mm}

{\noindent\fontsize{14pt}{14pt}\selectfont Wydanie zerowe (eksperymentalne)}

\newpage

% strona czwarta
\thispagestyle{empty}
\begin{figure}[H]
\begin{minipage}[b]{.48\linewidth}
{\noindent Epafrodyt z Eudoksos\\
do napisania\\
do napisania\\
do napisania}
\end{minipage}
\begin{minipage}[b]{.48\linewidth}
{\noindent do napisania\\
do napisania\\
do napisania\\
do napisania}
\end{minipage}
\end{figure}

{\noindent \textbf{Kategorie MSC 2020}\\do napisania} \vspace{5mm}

{\noindent \textbf{Tytuł oryginału}\\do napisania} \vspace{5mm}

{\noindent \textbf{Z greki tłumaczyła}\\do napisania} \vspace{5mm}

{\noindent \textbf{Okładkę zaprojektował}\\do napisania} \vspace{5mm}

{\noindent \textbf{Zredagował}\\do napisania} \vspace{5mm}

{\noindent \textbf{Zredagowała technicznie}\\do napisania} \vspace{5mm}

{\noindent \textbf{Złożyli i połamali}\\do napisania} \vspace{5mm}

{\noindent \textbf{Korekty dokonali}\\do napisania} \vfill

{\noindent Copyleft © 2024 by Antykwariat Czarnoksięski.
Książka, a żeby było śmieszniej także każda jej część, mogą być przedrukowywane oraz w jakikolwiek inny sposób reprodukowane czy powielane mechanicznie, fotooptycznie, zapisywane elektronicznie lub magnetycznie, oraz odczytywane w środkach publicznego przekazu bez pisemnej zgody wydawcy.
}

\vspace{5mm}
{
    \noindent
    Tekst udostępniany na licencji Creative Commons: uznanie autorstwa, użycie niekomercyjne. Przeczytaj więcej na \texttt{https://creativecommons.org/licenses/by-nc/4.0/deed.pl}.
}

\vspace{5mm}

{\noindent Przygotowano w systemie \TeX, wydrukowano na siarczystym papierze.}

% strona piąta
\newpage
\section*{Przedmowa}
Do napisania.

\begin{flushright}
Epafrodyt,\\gdzie, kiedy
\end{flushright}

\tableofcontents
% \pagestyle{fancy} % Enable default headers and footers again
\cleardoublepage % Start the following content on a new page

Tekst podsekcji Aksjomatyka. \loremipsum
%

\section{Aksjomaty Euklidesa}
\todofoot{Considered the "father of geometry",[3] he is chiefly known for the Elements treatise, which established the foundations of geometry that largely dominated the field until the early 19th century.}
\todofoot{Very little is known of Euclid's life, and most information comes from the scholars Proclus and Pappus of Alexandria many centuries later. }

\todofoot{The theorem of the gnomon was described as early as in Euclid's Elements (around 300 BC), and there it plays an important role in the derivation of other theorems. It is given as proposition 43 in Book I of the Elements, where it is phrased as a statement about parallelograms without using the term "gnomon". The latter is introduced by Euclid as the second definition of the second book of Elements. Further theorems for which the gnomon and its properties play an important role are proposition 6 in Book II, proposition 29 in Book VI and propositions 1 to 4 in Book XIII.[5][4][6]} % https://en.wikipedia.org/wiki/Theorem_of_the_gnomon
\todofoot{Strona ,,Euclidean geometry'' na en-wiki} % https://en.wikipedia.org/wiki/Euclidean_geometry

\subsection{Księga I}	
\subsubsection{Definicje}	
\begin{enumerate}	
    \item [1.1] Definicja ... % Definicja 1. % Punkt to jest to, co nie składa się z części.
    \item [1.2] Definicja ... % Definicja 2. % Linia jest długością bez szerokości.
    \item [1.3] Definicja ... % Definicja 3. % Końcami linii są punkty.
    \item [1.4] Definicja ... % Definicja 4. % Linia jest prosta, jeżeli położona jest między swoimi punktami w równym i jednostajnym kierunku.
    \item [1.5] Definicja ... % Definicja 5. % Powierzchnia jest to, co ma tylko długość i szerokość.
    \item [1.6] Definicja ... % Definicja 6. % Krawędzie powierzchni są liniami.
    \item [1.7] Definicja ... % Definicja 7. % Płaska powierzchnia albo płaszczyzna jest ta, na której biorąc gdziekolwiek dwa punkty linia prosta między tymi punktami cała leży na tej powierzchni.
    \item [1.8] Definicja ... % Definicja 8. % Kąt płaski to nachylenie dwóch linii na płaszczyźnie w miejscu, w którym jedna spotyka drugą i nie leżą w linii prostej.
    \item [1.9] Definicja ... % Definicja 9. % Kiedy linie są proste i tworzą kąt, wtedy kąt zwany jest prostoliniowym.
    \item [1.10] Definicja ... % Definicja 10. % Kiedy linia prosta padająca na drugą linie prostą, tworzy z nią kąty przyległe równe między sobą, to każdy z kątów równych nazywamy prostym, a padająca linia prostą nazywa się prostopadłą do tej linii, na którą pada.
    \item [1.11] Definicja ... % Definicja 11. % Kąt rozwarty jest większy od kąta prostego.
    \item [1.12] Definicja ... % Definicja 12. % Kąt ostry jest mniejszy od kąta prostego.
    \item [1.13] Definicja ... % Definicja 13. % Kresem albo granicą jest to, na czym się dana rzecz kończy.
    \item [1.14] Definicja ... % Definicja 14. % Figurą nazywamy to co jest ograniczone granicą lub granicami.
    \item [1.15] Definicja ... % Definicja 15. % Koło jest figurą płaską zawarta linią zwaną okręgiem, do której wszystkie linie proste poprowadzone z jednego punktu wewnątrz figury położonego, są między sobą równe.
    \item [1.16] Definicja ... % Definicja 16. % I ten punkt nazywa się centrum lub środkiem koła.
    \item [1.17] Definicja ... % Definicja 17. % Średnicą koła jest każda linia narysowana przez środek koła, przedłużona w dwóch kierunkach do jego obwodu, przepoławiająca go.
    \item [1.18] Definicja ... % Definicja 18. % Półokręgiem jest figura zawarta między średnicą i częscia okręgu odciętą tą średnicą. Środek półokregu jest też środkiem okręgu.
    \item [1.19] Definicja ... % Definicja 19. % Figury prostokreślne to figury ograniczone prostymi. Trójkąt to figura prostokreślna ograniczona trzema prostymi. Czworobok lub czworokąt to figura prostokreślna, która jest ograniczona czterema prostymi. Wielobok lub wielokąt to figura prostokreślna ograniczona więcej niż czterema prostymi.
    \item [1.20] Definicja ... % Definicja 20. % Trójkąt równoboczny to trójkąt, który ma trzy boki równe. Trójkąt równoramienny to trójkąt, który ma tylko dwa boki równe. Trójkąt różnoboczny to trójkąt, który ma trzy boki różne.
    \item [1.21] Definicja ... % Definicja 21. % Ponadto: trójkąt prostokątny to trójkąt, który na kąt prosty. Trójkąt rozwartokątny to trójkąt, który ma kąt rozwarty. Trójkąt ostrokątny to trójkąt, który ma trzy kąty ostre.
    \item [1.22] Definicja ... % Definicja 22. % Kwadrat jest to czworobok mający równe boki i równe kąty. Prostokąt jest to czworobok mający kąty proste, ale boki nierówne. Romb (kwadrat ukośny) jest to czworobok mający równe boki, ale nie mający kątów prostych. Równoległobok jest to czworobok mający boki przeciwległe równe, ale nie mający katów prostych. Wszystkie czworoboki inne niż wyżej wymienione nazywamy czworokątami.
    \item [1.23] Definicja ... % Definicja 23. % Linie równoległe, czyli mówiąc krócej równoległe są to proste, które leżą na tej samej płaszczyźnie i przedłużone z obu stron w nieskończoność, z żadnej strony nie przetną się.
\end{enumerate}	
	
\subsubsection{Postulaty}	
\begin{enumerate}	
    \item [1.1] Postulat ... % Postulat 1 % Można poprowadzić prostą od któregokolwiek punktu do któregokolwiek punktu.
    \item [1.2] Postulat ... % Postulat 2. % Ograniczoną prostą można przedłużyć nieskończenie.
    \item [1.3] Postulat ... % Postulat 3. % Można zakreślić okrąg z któregokolwiek punktu jako środka dowolną odległością.
    \item [1.4] Postulat ... % Postulat 4. % Wszystkie kąty proste są między sobą równe.
    \item [1.5] Postulat ... % Postulat 5. % Jeżeli prosta przecinająca dwie proste tworzy z nimi kąty jednostronnie wewnętrzne o sumie mniejszej niż dwa kąty proste, to te dwie proste przedłużone nieskończenie przecinają się po tej stronie, po której znajdują się kąty o sumie mniejszej od dwóch kątów prostych.
\end{enumerate}	
	
Jak łatwo zauważyć, sformułowanie ostatniego postulatu używa więcej słów niż pozostałe razem wzięte; wbrew przekonaniu, że postulaty miały wyrażać treści oczywiste i proste.	
Piąty postulat wydawał się bardziej skomplikowany, więc nasuwał podejrzenie, że wynika z poprzednich czterech.	
Zauważył to już Proklos (410-485): \emph{,,Nie jest możliwe, aby uczony tej miary co Euklides godził się na obecność tak długiego postulatu w aksjomatyce -- obecność postulatu wzięła się z pospiesznego kończenia przez niego Elementów, tak aby zdążyć przed nadejściem słusznie oczekiwanej rychłej śmierci; my zatem -- czcząc jego pamięć -- powinniśmy ten postulat usunąć lub co najmniej znacznie uprościć.''}	
	
Wiele osób próbowało stawić czoło wyzwaniu postawionemu przez Proklosa.	
Było to bezskuteczne, ponieważ piąty postulat jest niezależny od pozostałych, zaś zastąpienie go jego zaprzeczeniem prowadzi do geometrii nieeuklidesowych.	
Piszą o nim Audin \cite[s. 13]{audin_2003}.
	
\subsubsection{Pojęcia pierwotne}	
\begin{enumerate}	
    \item [1.1] Pojęcie pierwotne ... % Pojęcie podstawowe 1 % Wyrażenia, które są równe się temu samemu wyrażeniowi, są sobie równe.
    \item [1.2] Pojęcie pierwotne ... % Pojęcie podstawowe 2 % Jeżeli równania dodawane są do równań, wtedy całości są sobie równe.
    \item [1.3] Pojęcie pierwotne ... % Pojęcie podstawowe 3 % Jeżeli równania odejmowane są do równań, wtedy całości są sobie równe.
    \item [1.4] Pojęcie pierwotne ... % Pojęcie podstawowe 4 % Wyrażenia, które się pokrywają, są sobie równe.
    \item [1.5] Pojęcie pierwotne ... % Pojęcie podstawowe 5 % Całość jest większa od części.
\end{enumerate}	
	
\subsubsection{Twierdzenia}	
\begin{enumerate}	
    \item [1.1] Twierdzenie ... % Twierdzenie 1. % Na danej linii prostej skonstruuj trójkąt równoboczny o żadanych bokach.
    \item [1.2] Twierdzenie ... % Twierdzenie 2. % Skonstruuj odcinek równy danemu odcinkowi którego koniec jest zadanym punktem.
    \item [1.3] Twierdzenie ... % Twierdzenie 3. % Mając dane dwie linie proste nierówne, od większej odciąć linię równą mniejszej.
    \item [1.4] Twierdzenie ... \hfill \emph{(przystawanie bok-kąt-bok)} % Twierdzenie 4. % Jeśli dwa trójkąty mają dwa boki odpowiednio równe dwóm innym, i jeżeli kąty zawarte między bokami równoległymi są równe, wtedy ich podstawy również są sobie równe i pozostałe kąty równe są odpowiednim kątom.
    \item [1.5] Twierdzenie ... % Twierdzenie 5. % W trójkątach równoramiennych kąty przy podstawie są sobie równe oraz kąty powstałe przez przedłużenie boków równych są sobie równe.
    \item [1.6] Boki trójkąta leżące naprzeciw przystających kątów są przystające.
    \item [1.7] Twierdzenie ... % Twierdzenie 7. % Na tej samej podstawie i z tej samej strony nie mogą być wykreślone dwa trójkąty takie, żeby boki w tych trójkątach przy obydwu końcach wspólnej podstawy były między sobą równe.
    \item [1.8] Twierdzenie ... \hfill \emph{(przystawanie bok-bok-bok)} % Twierdzenie 8. % Jeżeli dwa boki jednego trójkąta są równe dwóm bokom drugiego trójkąta, to kąty zawarte między równymi bokami są sobie równe.
    \item [1.9] Podzielić dany kąt na dwie równe części.
    \item [1.10] Podzielić dany odcinek na dwie równe części.
    \item [1.11] Twierdzenie ... % Twierdzenie 11. % Z punktu danego na danej linii prostej wyprowadzić linie prostopadłą do danej linii prostej.
    \item [1.12] Twierdzenie ... % Twierdzenie 12. % Z punktu danego leżącego poza linią prostą nieograniczoną, wyprowadzić prostą linię prostopadłą do niej.
    \item [1.13] Twierdzenie ... % Twierdzenie 13. % Jeżeli linia prosta przecinająca drugą prostą tworzy z nią dwa kąty, to są one proste, albo równe dwóm kątom prostym.
    \item [1.14] Twierdzenie ... % Twierdzenie 14. % Jeżeli przy linii prostej i przy punkcie na niej leżącym dwie linie proste nie po jednej stronie położone czynią kąty przyległe równe dwóm kątom prostym, to te linie proste będą w tym samym kierunku.
    \item [1.15] Twierdzenie ... % Twierdzenie 15. % Jeżeli dwie linie proste przecinają się, to utworzone przez nie kąty przeciwległe są sobie równe.
    \item [1.16] Twierdzenie ... % Twierdzenie 16. % W dowolnym trójkącie kąt zewnętrzny powstały przez przedłużenie jednego boku jest większy od każdego z dwóch kątów wewnętrznych przeciwległych jemu.
    \item [1.17] Twierdzenie ... % Twierdzenie 17. % W każdym trójkącie suma dwóch dowolnych kątów jest mniejsza od dwóch kątów prostych.
    \item [1.18] Twierdzenie ... % Twierdzenie 18. % W każdym trójkącie bok większy przeciwległy jest kątowi większemu.
    \item [1.19] Twierdzenie ... % Twierdzenie 19. % W każdym trójkącie kąt większy przeciwległy jest bokowi większemu.
    \item [1.20] Twierdzenie ... % Twierdzenie 20. % W każdym trójkącie suma dwóch dowolnych boków jest większa od boku trzeciego.
    \item [1.21] Twierdzenie ... % Twierdzenie 21. % Jeżeli z końców jednego boku trójkąta poprowadzone będą dwie linie proste wewnątrz trójkąta, aż do zejścia się z sobą, to te dwie linie proste będą mniejsze od dwóch pozostałych boków trójkąta, lecz zawierać jednak będą kąt większy od kąta zawartego między pozostałymi bokami trójkąta.
    \item [1.22] Twierdzenie ... % Twierdzenie 22. % Aby z trzech danych linii prostych wykreślić trójkąt, potrzeba aby z tych trzech danych linii prostych suma dwóch którychkolwiek była większa od trzeciej.
    \item [1.23] Twierdzenie ... % Twierdzenie 23. % Na danej linii prostej i punkcie na niej danym wykreślić kąt prostokreślny równy kątowi prostokreślnemu danemu.
    \item [1.24] Twierdzenie ... % Twierdzenie 24. % Jeżeli dwa boki jednego trójkąta, są równe dwóm bokom trójkąta drugiego, z kątów zaś między bokami równymi jeden większy jest od drugiego; to będzie też podstawa jednego trójkąta większa od podstawy drugiego trójkąta.
    \item [1.25] Twierdzenie ... % Twierdzenie 25. % Jeżeli dwa boki jednego trójkąta, są równe dwóm bokom trójkąta drugiego, lecz podstawa jednego trójkąta większa jest od podstawy drugiego trójkąta, to i kąty między bokami równymi zawarte będą jeden większy od drugiego.
    \item [1.26] Twierdzenie ... % Twierdzenie 26. % Jeżeli dwa kąty jednego trójkąta są równe dwóm kątom drugiego trójkąta, i bok jeden przyległy obydwu kątom, albo jednemu w pierwszym trójkącie równa się bokowi jednemu przyległemu obydwu katom, albo jednemu w drugim trójkącie; będą i dwa boki pozostałe równe dwóm bokom pozostałym i kąt trzeci w jednym trójkącie będzie równy katowi trzeciemu w drugim trójkącie.
    \item [1.27] Twierdzenie ... % Twierdzenie 27. % Jeżeli na dwie linie proste, pada linia prosta czyniąca kąty naprzemian równe między sobą, to te dwie linie proste będą równoległe.
    \item [1.28] Twierdzenie ... % Twierdzenie 28. % Jeśli linia prosta opada na dwie linie proste, tworząc kąt zewnętrzny równy wewnętrznemu i przeciwny do kąta na tym samym boku lub suma kątów wewnętrznych na tym samym boku jest równa dwóm kątom prostym, wtedy linie proste są równoległe do siebie.
    \item [1.29] Twierdzenie ... % Twierdzenie 29. % Linia prosta opada na równoległą linie prostą tworząc alternatywne kąty równe sobie, kąt zewnętrzny równy wewnętrznemu i przeciwległy i suma kątów wewnętrznych na tym samym boku jest równa dwóm kątom prostym.
    \item [1.30] Twierdzenie ... % Twierdzenie 30. % Linie proste, które są równoległe do linii prostej są również równoległe do siebie.
    \item [1.31] Twierdzenie ... % Twierdzenie 31. % Poprowadzić przez dany punkt linię prostą równoległą względem danej lini prostej.
    \item [1.32] Twierdzenie ... % Twierdzenie 32. % W jakimkolwiek trójkącie, jeśli jeden z boków jest znany wtedy kąt zewnętrzny jest równy sumie dwóch kątów wewnętrznych i przeciwnych i suma trzech wewnętrznych kątów trójkąta jest równa dwóm kątom prostym.
    \item [1.33] Twierdzenie ... % Twierdzenie 33. % Linie proste, które łączą końce równych i równoległych linii prostych w tym samym kierunku są sobie równe i równoległe.
    \item [1.34] Twierdzenie ... % Twierdzenie 34. % W równoległobokach boki i kąty przeciwne są między sobą równe, a przekątna dzieli je na dwie równe części.
    \item [1.35] Twierdzenie ... % Twierdzenie 35. % Równoległoboki, które są na takiej samej podstawie i są porównywalne są sobie równe.
    \item [1.36] Twierdzenie ... % Twierdzenie 36. % Równoległoboki, które mają równe podstawy i są porównywalne są sobie równe.
    \item [1.37] Twierdzenie ... % Twierdzenie 37. % Trójkąty, które mają takie same podstawy i są porównywalne są sobie równe.
    \item [1.38] Twierdzenie ... % Twierdzenie 38. % Trójkąty, których podstawy są równe i są one porównywalne są sobie równe.
    \item [1.39] Twierdzenie ... % Twierdzenie 39. % Równe trójkąty, które są na takich samych podstawach i mające te same boki również są porównywalne.
    \item [1.40] Twierdzenie ... % Twierdzenie 40. % Równe trójkąty, które mają takie same podstawy i mają te same boki również są porównywalne.
    \item [1.41] Twierdzenie ... % Twierdzenie 41. % Jeśli równoległobok i trójkąt mają tą samą podstawę i są tymi samymi liniami zakończone, to trójkąt jest połową równoległoboku.
    \item [1.42] Twierdzenie ... % Twierdzenie 42. % Skonstruować równoległobok równy danemu trójkątowi o podanym prostoliniowym kącie.
    \item [1.43] Twierdzenie ... % Twierdzenie 43. % W każdym równoległoboku, dopełnienia równoległoboków koło przekątnych położonych są między sobą równe.
    \item [1.44] Twierdzenie ... % Twierdzenie 44. % Na danej linii prostej wykreślić równy danemu równoległobok, którego jeden kąt będzie równy danemu.
    \item [1.45] Twierdzenie ... % Twierdzenie 45. % Wykreślić równy danej figurze prostokreślny równoległobok, którego jeden kąt będzie równy danemu.
    \item [1.46] Twierdzenie ... % Twierdzenie 46. % Na danej linii prostej wykreślić kwadrat.
    \item [1.47] Twierdzenie ... \hfill \emph{(twierdzenie Pitagorasa)} % Twierdzenie 47. % W trójkącie prostokątnym, kwadrat zbudowany na boku przeciwnym kątowi prostemu, równy jest kwadratom zbudowanym na bokach, które kąt prosty zawierają.
    \item [1.48] Twierdzenie ... \hfill \emph{(twierdzenie odwrotne do twierdzenia Pitagorasa)} % Twierdzenie 48. % Jeżeli kwadrat zbudowany na jednym z boków trójkąta, jest równy kwadratom wykreślonym na dwóch pozostałych bokach trójkąta, to kąt zawarty między dwoma pozostałymi bokami będzie prosty.
\end{enumerate}	
\subsection{Księga II}	
\subsubsection{Definicje}	
\begin{enumerate}
    \item [2.1] Definicja ...
    % Definicja 1 % Każdy równoległobok prostokątny wyraża i wykreśla się dwiema liniami prostymi które zawierają właściwy kąt.
    \item [2.2] Definicja ...
    % Definicja 2 % W równoległoboku jeżeli poprowadzimy przekątną i przez punkt gdziekolwiek obrany na tej przekątnej poprowadzimy dwie linie równoległe do boków równoległoboku, równoległobok podzieli się na cztery części, każda z dwóch części której przekątna jest częścią przekątnej całego równoległoboku, wzięta z dwiema jej przyległymi zwać będziemy węgielnicą.
\end{enumerate}	
	
\subsubsection{Twierdzenia}	
\begin{enumerate}	
    \item [2.1] Twierdzenie ...
    % Twierdzenie 1 % Jeżeli z dwóch linii prostych podzielimy jedną którąkolwiek na ilekolwiek części (które będziemy nazywać odcinkami), prostokąt zawarty dwiema liniami prostymi, równy będzie prostokątom wykreślonym z linii prostej nieprzecietej i z odcinków drugiej linii prostej.
    \item [2.2] Twierdzenie ...
    % Twierdzenie 2 % Jeżeli linie prostą podzielimy jakkolwiek, prostokąty zawarte całą linią i jej oddzielnymi odcinkami będą równe kwadratowi z całej linii.
    \item [2.3] Twierdzenie ...
    % Twierdzenie 3 % Jeżeli linie prostą podzielimy na dwa jakiekolwiek odcinki; prostokąt całą linią i jednym odcinkiem, zawarty, będzie równy prostokątowi odcinkami linii prostej zawartymi wraz z kwadratem wyrażonym na odcinku wziętym z boku drugiego prostokąta pierwszego.
    \item [2.4] Twierdzenie ...
    % Twierdzenie 4 % Jeżeli linię prostą podzielimy na dwa jakiekolwiek odcinki, kwadrat z całej linii będzie równy kwadratom z obydwu odcinków linii dwa razy wziętemu prostokątowi zawartemu odcinkami linii.
    \item [2.5] Twierdzenie ...
    % Twierdzenie 5 % Jeżeli linię prostą podzielimy na dwa równe odcinki i na dwa odcinki nierówne; to prostokąt odcinkami nierównymi zawartymi wraz z kwadratem wystawionym na odcinkach między podziałami zawartymi będzie równy kwadratowi wystawionemu na połowie linii.
    \item [2.6] Twierdzenie ...
    % Twierdzenie 6 % Jeżeli linię prostą na dwa różne odcinki podzieloną przedłużymy podług upodobani; prostokąt zawarty linią prostą wraz z przedłużeniem wziętą i samym przedłużeniem, wraz z kwadratem wystawionym na połowie linii, będzie równy kwadratowi wystawionemu na połowie linii wraz z przedłużeniem wziętym.
    \item [2.7] Twierdzenie ...
    % Twierdzenie 7 % Jeżeli linię prostą podzielimy na dwa różne odcinki nierówne; kwadraty: pierwszy z całej linii, drugi z jej odcinka, będą równe dwa razy wziętemu prostokątowi całą linią i tym samym odcinkiem zawartym wraz z kwadratem z odcinka drugiego.
    \item [2.8] Twierdzenie ...
    % Twierdzenie 8 % Jeżeli linię podzielimy na dwa odcinki nierówne; cztery razy wzięty prostokąt całą linią i jej jednym odcinkiem zawarty wraz z kwadratem z odcinka drugiego, będzie równy kwadratowi wystawionemu na linii złożonej z całej linii i z odcinka pierwszego.
    \item [2.9] Twierdzenie ...
    % Twierdzenie 9 % Jeżeli linię prostą podzielimy na dwa odcinki równe, i na dwa odcinki nierówne; kwadraty z odcinków nierównych będą dwa razy większe od kwadratów, z których jeden byłby wystawiony na połowie linii, drugi na linii miedzy podziałami zawartej.
    \item [2.10] Twierdzenie ...
    % Twierdzenie 10 % Jeżeli linię prostą na dwa odcinki równe podzieloną przedłużymy według upodobania; kwadraty: pierwszy z całej linii wraz z przedłużeniem, drugi z samego przedłużenia, będą dwa razy większe od kwadratów, z których pierwszy byłby wystawiony na połowie linii, a drugi na połowie linii wraz z przedłużeniem wziętym.
    \item [2.11] Twierdzenie ...
    % Twierdzenie 11 % Daną linię prostą podzielić na dwa odcinki tak, aby prostokąt całą linią i jednym jej odcinkiem zawarty, był równy kwadratowi z odcinka drugiego.
    \item [2.12] Twierdzenie ...
    % Twierdzenie 12 % W trójkątach rozwartokątnych, kwadrat z boku kątowi przeciwnemu rozwartemu, większy jest od kwadratów z ramion kąta rozwartego o dwa razy wzięty prostokąt, zawarty ramionami kąta rozwartego i przedłużeniem tego ramienia zamkniętym między wierzchołkiem kąta rozwartego i punktem w którym prostopadła z końca drugiego ramienia kąta rozwartego spuszczona na pierwsze ramie, spotyka przedłużenie odcinka.
    \item [2.13] Twierdzenie ...
    % Twierdzenie 13 % W każdym trójkącie, kwadrat z boku przeciwnego kątowi ostremu, mniejszy jest od kwadratów z ramion ten kąt obejmujących, o dwa razy wzięty prostokąt zawarty ramieniem tego kąta ostrego i odcinka, lub przedłużeniem tego ramienia zamkniętym między wierzchołkami kata ostrego i punktem, w którym linia prostopadła z końca drugiego ramienia kąta ostrego spuszczona na pierwsze ramię spotyka to ramię lub przedłużenie danego ramienia.
    \item [2.14] Twierdzenie ...
    % Twierdzenie 14 % Na danej figurze prostokreślnej równy kwadrat wykreślić.
\end{enumerate}	
%

\subsection{Księga III}
\subsubsection{Definicje}
\begin{enumerate}
    \item [3.1] Dwa okręgi są przystające, kiedy mają równe średnice (lub równoważnie, promienie).
    \index{przystawanie!okręgów}%
    \item [3.2] Definicja ...
    % Definicja 2. % Mówi się, że linia prosta dotyka koła, gdy będąc styczną z kołem przedłużona z obydwu stron nie przecina się z żadnej strony okręgu koła.
    \item [3.3] Dwa okręgi nazywamy stycznymi, kiedy mają dokładnie jeden punkt wspólny.
    \index{styczność}%
    \item [3.4] Definicja ...
    % Definicja 4. % Mówi się, że linie proste równoodległe są od środka koła, gdy prostopadłe ze środka koła na nie spuszczone są równe.
    \item [3.5] Definicja ...
    % Definicja 5. % Mówi się, że ta linia prosta bardziej jest odległa od środka koła, na którą prostopadła ze środka koła spuszczona jest większa.
    \item [3.6] Definicja ...
    % Definicja 6. % Odcinkiem koła jest figura czyli część koła ograniczona linią prostą i okręgiem koła.
    \item [3.7] Definicja ...
    % Definicja 7. % Kąt zaś odcinka jest ten, który się linią prostą i okręgiem koła zawiera.
    \item [3.8] Definicja ...
    % Definicja 8. % Jeżeli na okręgu koła wzięty będzie punkt i od niego będą poprowadzone linie proste do końców linii prostej za podstawę odcinkami służącej, kąt między tymi liniami prostymi zawarty jest kątem w odcinku.
    \item [3.9] Definicja ...
    % Definicja 9. % Kiedy zaś linie proste kąt zawierające zajmują część okręgu, mówi się, że kąt ten opiera się na okręgu koła.
    \item [3.10] Definicja ...
    % Definicja 10. % Jeżeli kąt ma swój wierzchołek we środku koła; figura czyli część koła zawarta między ramionami tegoż koła, to jest między promieniami i łukiem koła nazywa się wycinkiem koła.
    \item [3.11] Definicja ...
    % Definicja 11. % Odcinkami podobnymi kół nazywają się te, które zajmują kąty równe, lub w których kąty są równe między sobą.
\end{enumerate}

\subsubsection{Twierdzenia}
\begin{enumerate}
    \item [3.1] Skonstruować środek danego okręgu. 
    \item [3.2] Twierdzenie ...
    % Twierdzenie 2. % Jeżeli na okręgu obierzemy dwa gdziekolwiek punkty, linia prosta łącząca te punkty padnie wewnątrz koła.
    \item [3.3] Twierdzenie ...
    % Twierdzenie 3. % Jeżeli w kole linia prosta przez środek poprowadzona przecina linie nie przez środek poprowadzoną na dwie równe części, będzie pierwsza prostopadła do drugiej; i jeżeli pierwsza jest prostopadła do drugiej, przecina ja na dwie równe części.
    \item [3.4] Twierdzenie ...
    % Twierdzenie 4. % Jeżeli w kole dwie linie proste, nie przez środek koła poprowadzone przecinają się nawzajem, nie przetną się na dwie równe części.
    \item [3.5] Dwa okręgi, które się przecinają, nie mogą być współśrodkowe. 
    \item [3.6] Twierdzenie ...
    % Twierdzenie 6. % Jeżeli dwa koła dotykają się wzajemnie, to wspólnego środka mieć nie mogą.
    \item [3.7] Twierdzenie ...
    % Twierdzenie 7. % Jeżeli na średnicy koła wzięty będzie punkt którykolwiek oprócz średnicy koła i od tego punktu poprowadzone linie proste do okręgu, ze wszystkich linii największa będzie część średnicy, na której znajduje się środek koła, a najmniejsza pozostała część średnicy, z innych zaś linii prostych każda bliższa przechodząca przez środek koła, większa będzie od odleglejszej, z tego na koniec punktu dwie tylko równe linie proste z obydwu stron najmniejszej linii prostej mogą być do okręgu poprowadzone.
    \item [3.8] Twierdzenie ...
    % Twierdzenie 8. % Jeżeli z punktu zewnątrz koła obranego, poprowadzone będą do okręgu linie proste, z których jedna przechodziła by przez środek koła a inne padały gdziekolwiek, z linii prostych padających na część okręgu wklęsłą, największa jest linia poprowadzona przez środek koła, z innych zaś linii każda bliższa przechodzącej przez środek jest większa od odleglejszej. Lecz z linii padających na cześć okręgu wypukłą, najmniejsza jest linia prosta zawarta między punktem zewnętrz koła i średnicą, z innych zaś linii prostych każda bliższa najmniejszej, mniejsza jest odleglejsza; na koniec dwie tylko równe linie proste z tego punktu po obydwu stronach najmniejszej linii prostej mogą być do okręgu poprowadzone.
    \item [3.9] Twierdzenie ...
    % Twierdzenie 9. % Jeżeli z punktu danego wewnątrz koła poprowadzimy do okręgu więcej niż dwie linie proste i te proste są miedzy sobą równe, punkt ten będzie środkiem koła.
    \item [3.10] Dwa okręgi, które się przecinają, przecinają się w dwóch punktach. 
    \item [3.11] Twierdzenie ...
    % Twierdzenie 11. % Jeżeli dwa koła stykają się ze sobą wewnątrz, linia łącząca środki tychże kół przedłużona pada na punkt dotykania się kół.
    \item [3.12] Twierdzenie ...
    % Twierdzenie 12. % Jeżeli dwa koła dotykają się ze sobą zewnętrznie, to linia prosta łącząca ich środki przechodzi przez punkt dotykania się.
    \item [3.13] Twierdzenie ...
    % Twierdzenie 13. % Okrąg koła nie może dotykać okręgu drugiego koła w więcej niż jednym punkcie, nieważne jest czy dotkniecie jest zewnętrzne bądź wewnętrzne.
    \item [3.14] Twierdzenie ...
    % Twierdzenie 14. % W kole linie proste równe, na okręgu jego zakończone, są równoodległe od środka; i linie proste które na okręgu jego zakończone są równoodległe od środka, są też miedzy sobą równe.
    \item [3.15] Twierdzenie ...
    % Twierdzenie 15. % Ze wszystkich linii prostych w kole poprowadzonych i na okręgu jego zakończonych, największa jest średnica, z innych zaś każda bliższa środka koła, większa jest od odleglejszej; i z dwóch linii prostych nierównych, większa bliższa jest środka koła od mniejszej.
    \item [3.16] Twierdzenie ... 
    % Twierdzenie 16. % Prostopadła do średnicy koła z końca jej wyprowadzona, pada cała zewnątrz koła, a między tą prostopadłą i okręgiem żadna inna linia prosta nie pada; albo tak samo: okrąg koła przechodzi miedzy prostopadłą do średnicy i linią prostą, która ze średnicą kąt ostry jakokolwiek wielki zawiera, czyli która zawiera kąt jakokolwiek mały z prostopadłą do średnicy.
    \item [3.17] Skonstruować styczną do danego okręgu, która przechodzi przez dany punkt.
    \index{styczność}%
    \item [3.18] Twierdzenie ...
    % Twierdzenie 18. % Jeżeli linia prosta dotyka się okręgu koła, a ze środka koła wyprowadzona będzie linia prosta do punktu dotykania się, to ta będzie prostopadła do stycznej.
    \item [3.19] Twierdzenie ...
    % Twierdzenie 19. % Jeżeli linia prosta dotyka okręgu koła, z punktu zaś dotknięcia wyprowadzona będzie do tej stycznej prostopadła, to na prostopadłej będzie środek koła.
    \item [3.20] Twierdzenie ...
    % Twierdzenie 20. % W kole, kąt mający wierzchołek we środku jest podwojeniem kata mającego swój wierzchołek na okręgu koła, gdyż tę samą podstawę okręgu mają za podstawę, czyli to samo gdy ramionami swymi tej samej części okręgu obejmują.
    \item [3.21] Twierdzenie ...
    % Twierdzenie 21. % Kąty w tym samym odcinku koła są między sobą równe.
    \item [3.22] Twierdzenie ...
    % Twierdzenie 22. % Kąty przeciwne czworokąta w koło wpisane są równe dwóm kątom prostym.
    \item [3.23] Twierdzenie ...
    % Twierdzenie 23. % Na tej samej linii prostej nie można wykreślić dwóch odcinków kół po tej samej stronie podobnych, które by nie przystawały do siebie.
    \item [3.24] Twierdzenie ...
    % Twierdzenie 24. % Wykreślone na równych liniach prostych podobne odcinki kół, są między sobą równe.
    \item [3.25] Twierdzenie ...
    % Twierdzenie 25. % Mając dany odcinek koła, opisać koła którego jest odcinkiem.
    \item [3.26] Twierdzenie ...
    % Twierdzenie 26. % W kołach równych, kąty równe w środkach lub przy okręgach wspierają się na równych łukach.
    \item [3.27] Twierdzenie ...
    % Twierdzenie 27. % W kołach równych, kąty we środkach lub przy okręgach, na równych łukach wspierające się, są między sobą równe.
    \item [3.28] Twierdzenie ...
    % Twierdzenie 28. % W kołach równych, cięciwy równe obejmują łuki równe, tak, że łuk większy większemu, mniejszy mniejszemu jest równy.
    \item [3.29] Twierdzenie ...
    % Twierdzenie 29. % W kołach równych, równe łuki obejmują cięciwy równe.
    \item [3.30] Podzielić dany Twierdzenie ...
    % Twierdzenie 30. % Dany łuk podzielić na dwie części.
    \item [3.31] Twierdzenie ...
    % Twierdzenie 31. % W kole, kąt w półkolu jest prosty; z katów zaś w odcinkach nierównych, kąt w większym odcinku mniejszy jest od prostego; a w mniejszym odcinku większy od prostego.
    \item [3.32] Twierdzenie ...
    % Twierdzenie 32. % Jeżeli okręgu koła dotyka linia prosta, z punktu zaś dotknięcia poprowadzona będzie cięciwa, kąty zawarte miedzy cięciwową i styczną, będą równe kątom w odcinkach koła na przemian.
    \item [3.33] Twierdzenie ...
    % Twierdzenie 33. % Na danej linii prostej wykreślić odcinek koła który by zawierał kąt równy kątowi danemu.
    \item [3.34] Twierdzenie ...
    % Twierdzenie 34. % Z koła danego oddzielić odcinek któryby zawierał kąt równy danemu kątowi.
    \item [3.35] Twierdzenie ...
    % Twierdzenie 35. % Jeżeli w kole dwie cięciwy przecinają się nawzajem, prostokąt zawarty odcinkami jednej cięciwy będzie równy prostokątowi zawartemu odcinkami drugiej cięciwy.
    \item [3.36] Twierdzenie ...
    % Twierdzenie 36. % Jeżeli z punktu za kołem obranego, poprowadzimy dwie linie proste, których jedna przecinałaby koło, a druga byłaby styczną; to prostokąt zawarty całą linia przecinającą i odcinkiem jej za kołem będzie równy kwadratowi ze stycznej.
    \item [3.37] Twierdzenie ...
    % Twierdzenie 37. % Jeżeli z dwóch linii prostych, od jednego punktu zewnątrz koła obranego poprowadzonych, jedna przecina koło, a druga pada na okrąg tego koła: i jeżeli prostokąt z całej linii przecinającej i odcinka jej za kołem będącego jest równy kwadratowi z linii padającej na okrąg koła, to linia będzie padająca na okrąg koła styczną.
\end{enumerate}

%
%

\subsection{Księga IV}
\subsubsection{Definicje}
\begin{enumerate}
	\item [4.1] Definicja ...
	% Definicja 1. % Mówi się że figura prostokreślna wpisuje się w figurę prostokreślną, wtedy kiedy każdy kąt figury wpisanej dotyka się każdego boku figury, w który się wpisuje.
	\item [4.2] Definicja ...
	% Definicja 2. % Podobnie się mówi, że figura opisuje się na figurze, kiedy każdy bok figury opisanej dotyka każdego kąta figury na której się opisuje.
	\item [4.3] Definicja ...
	% Definicja 3. % Figura prostokreślna wpisuje się w koło, kiedy każdy kąt figury wpisanej dotyka okręgu koła.
	\item [4.4] Definicja ...
	% Definicja 4. % Figura prostokreślna opisuje się na kole kiedy każdy bok figury opisanej dotyka okręgu koła.
	\item [4.5] Definicja ...
	% Definicja 5. % Podobnież koło wpisuje się w figurę prostokreślną, kiedy każdy bok figury w którą koło się wpisuje, dotyka okręgu koła.
	\item [4.6] Definicja ...
	% Definicja 6. % Koło opisuje się na figurze prostokreślne wtedy gdy okrąg dotyka do każdego kąta figury na której opisujemy koło.
	\item [4.7] Definicja ...
	% Definicja 7. % Mówi się, że linie proste kreśli się w kole, gdy jej końce są na okręgu danego koła.
\end{enumerate}

\subsubsection{Twierdzenia}
\begin{enumerate}
	\item [4.1] Wpisać odcinek krótszy od średnicy w dany okrąg.
	\item [4.2] Twierdzenie ...
	% Twierdzenie 2. % W dane koło wpisać trójkąt równoramienny względem danego trójkąt.
	\item [4.3] Twierdzenie ...
	% Twierdzenie 3. % Na danym kole opisać trójkąt równokątny względem danego trójkąta.
	\item [4.4] Twierdzenie ...
	% Twierdzenie 4. % W dany trójkąt wpisać koło.
	\item [4.5] Opisać okrąg na danym okręgu.
	\item [4.6] Wpisać kwadrat w dany okrąg.
	\item [4.7] Opisać kwadrat na danym okręgu.
	\item [4.8] Wpisać okrąg w dany kwadrat.
	\item [4.9] Opisać okrąg na danym kwadracie.
	\item [4.10] Wykreślić trójkąt równoramienny, którego kąt przy podstawie jest podwojeniem kąta przy wierzchołku (o kątach $\pi/5$, $2\pi/5$, $2\pi/5$).
	\item [4.11] Twierdzenie ...
	% Twierdzenie 11. % W dane koło wpisać pięciokąt równoboczny i równokątny.
	\item [4.12] Opisać pięciokąt równoboczny i równokątny na danym okręgu.
	\item [4.13] Wpisać okrąg w dany pięciokąt równoboczny i równokątny.
	\item [4.14] Opisać okrąg na danym pięciokącie równobocznym i równokątnym
	\item [4.15] Wpisać sześciokąt równoboczny i równokątny w dany okrąg.
	\item [4.16] Wpisać piętnastokąt równoboczny i równokątny w dany okrąg.
\end{enumerate}

%

% TODO: https://kpbc.umk.pl/dlibra/publication/37/edition/66/content
% TODO: Pojęcia pierwotne i aksjomaty Euklidesa nie są jednak idealne.
% TODO: Dlatego zamiast nich będziemy używać aksjomatów Hilberta podanych około 1899 roku.

% https://www.claymath.org/library/historical/euclid/
% BOOK I	Triangles, parallels, and area
% BOOK II	Geometric algebra
% BOOK III	Circles
% BOOK IV	Constructions for inscribed and circumscribed figures
% BOOK V	Theory of proportions
% BOOK VI	Similar figures and proportions
% BOOK VII	Fundamentals of number theory
% BOOK VIII	Continued proportions in number theory
% BOOK IX	Number theory
% BOOK X	Classification of incommensurables
% BOOK XI	Solid geometry
% BOOK XII	Measurement of figures
% BOOK XIII	Regular solids

%
%

\subsection{Aksjomaty Hilberta}
\subsubsection{Aksjomaty incydencji}
Aksjomaty incydencji I1, I2, I3.

\begin{proposition}
    Dwie różne proste mogą mieć co najwyżej jeden punkt wspólny.
\end{proposition}

\begin{definition}
    Dwie różne proste, które nie mają punktów wspólnych, nazywamy równoległymi.
    Każda prosta jest też równoległa do siebie.
\end{definition}

\begin{definition}[aksjomat Playfaira]
    Dla każdej prostej $l$ oraz punktu $A$, istnieje dokładnie jedna prosta przechodząca przez $A$, równoległa do $l$.
\end{definition}

\begin{example}
    Rozważmy zbiór pięciu punktów $A$, $B$, $C$, $D$, $E$, w którym proste są dowolnymi dwuelementowymi podzbiorami.
    Wtedy proste $AB$ i $AC$ mają punkt wspólny $A$, chociaż obydwie są równoległe do prostej $DE$.
    Aksjomat Playfaira nie jest spełniony.
\end{example}

\begin{proposition}
    Aksjomaty I1, I2, I3, P (Playfaira) są od siebie niezależne.
\end{proposition}

Hartshorne \cite[s. 69-70]{hartshorne2000} konstruuje modele geometrii, w których spełnione są dowolne trzy, ale nie czwarty z nich.

\begin{proposition}
    Płaszczyzna rzutowa to taki zbiór punktów oraz prostych (podzbiorów zbioru punktów), że: przez dwa różne punkty przechodzi dokładnie jedna prosta, każde dwie proste mają punkt wspólny, każda prosta ma co najmniej trzy punkty i nie wszystkie punkty są współliniowe.
    Każda płaszczyzna rzutowa ma co najmniej siedem punktów, dokładnie jedna płaszczyzna rzutowa ma dokładnie siedem punktów, każdy z wymienionych zdanie wcześniej aksjomatów jest niezależny od pozostałych.
    Co więcej, wynikają z nich wszystkie trzy aksjomaty incydencji.
    Jeśli istnieje prosta, która ma $n+1$ punktów, to płaszczyzna ma $n^2 + n + 1$ punktów.
\end{proposition} % Hartshorne 71

\begin{proposition}
    Płaszczyzna afiniczna to taki zbiór punktów i prostych, które spełniają aksjomaty incydencji oraz mocniejszą wersję aksjomatu Playfaira: dla każdej prostej $l$ i punktu $A$, dokładnie jedna prosta przechodzi przez punkt $A$ i jest równoległa do $l$>
    Każda prosta na płaszczyźnie afinicznej ma tyle samo punktów.
    Jeśli pewna prosta ma $n$ punktów, to płaszczyzna ma dokładnie $n^2$ punktów.
    Istnieją płaszczyzny rzutowe o $4$, $9$, $16$ i $25$ punktach, ale nie istnieje taka, która miałaby $36$ punktów.
\end{proposition} % Hartshorne 71, 72

\subsubsection{Aksjomaty leżenia pomiędzy}
B1, B2, B3, B4 (Pascha).

I1-I3 + B1-B4 wynika stąd, że każda prosta ma nieskończenie wiele punktów.

\begin{definition}[odcinek]
    Niech $A$, $B$ będą punktami.
    Zbiór złożony z punktów $A$, $B$ oraz punktów, które leżą między nimi, nazywamy odcinkiem i oznaczamy $\overline {AB}$.
\end{definition} % Hartshorne 74

\begin{definition}[trójkąt]
    Niech $A$, $B$, $C$ będą punktami.
    Sumę odcinków $AB$, $BC$, $AC$ nazywamy trójkątem, wspomniane odcinki -- jego bokami, zaś punkty $A$, $B$ i $C$ -- wierzchołkami.
\end{definition} % Hartshorne 74

\begin{proposition}
    Niech $l$ będzie prostą.
    Wtedy zbiór punktów, które nie leżą na prostej $l$ można rozbić na dwa niepuste zbiory $S_1$, $S_2$ takie, że: dwa punkty, które nie leżą na prostej $l$, należą do tego samego zbioru ($S_1$ lub $S_2$) wtedy i tylko wtedy, gdy odcinek $AB$ nie przecina prostej $l$.
\end{proposition} % Hartshorne 74

Zbiory $S_1$, $S_2$ nazywamy stronami prostej $l$.
Podobnie punkt wyznacza na prostej dwa zbiory, które leżą po różnych stronach tego punktu.

\begin{definition}[półprosta]
    Niech $A$, $B$ będą punktami.
    Zbiór złożony z punktów $A$, $B$ oraz punktów, które leżą po tej samej stronie punktu $A$ na prostej $AB$ co punkt $B$, nazywamy półprostą i oznaczamy $NIE WIEM JAK AB$.
\end{definition} % Hartshorne 77

\begin{definition}[kąt]
    Sumę dwóch półprostych $AB$, $AC$, które nie leżą na jednej prostej, nazywamy kątem, zaś punkt $A$ wierzchołkiem tego kąta.
    Wnętrze kąta $\angle BACS$ składa się z tych punktów $D$ takich, że $D$ i $C$ leżą po tej samej stronie prostej $AB$ oraz $D$ i $B$ leżą po tej samej stronie prostej $AC$.
\end{definition} % Hartshorne 77

W myśl tej definicji, nie ma kąta zerowego ani półpełnego.
Wnętrze trójkąta $ABC$ to część wspólna wnętrz kątów $\angle ABC$, $\angle BCA$, $\angle CAB$; jest wypukłe i niepuste.

\subsubsection{Aksjomaty przystawania odcinków}
Aksjomaty C1-C3

\begin{definition}[okrąg]
    Niech $O$, $A$ będą dwoma różnymi punktami.
    Zbiór punktów $B$ takich, że odcinki $OA$ i $OB$ są przystające nazywamy okręgiem o środku $O$ oraz promieniu $OA$; okręgi często oznacza się literą $\Gamma$.
\end{definition} % Hartshorne 89

\begin{proposition}
    Każda prosta, która przechodzi przez środek, przecina okrąg w dwóch punktach.
    Okrąg składa się z nieskończenie wielu punktów.
\end{proposition}

(Nie jest jasne, ile środków może mieć okrąg, ale Hartshorne \cite[s. 89]{hartshorne2000} obiecuje pokazać póżniej, że tylko jeden.
Później ma miejsce na stronie 104).

\begin{definition}[styczna]
    Niech $\Gamma$ będzie okręgiem, zaś $l$ prostą, która przecina $\Gamma$ w dokładnie jednym punkcie $A$.
    Mówimy, że $l$ jest styczną do okręgu $\Gamma$ w punkcie $A$.
\end{definition}

Podobnie mówimy, że dwa okręgi są styczne, jeśli mają jeden punkt wspólny.

\subsubsection{Aksjomaty przystawania kątów}
C4-C6

Kąty przyległe, prosty.

\subsubsection{Płaszczyzna Hilberta}

. . .

%
\subsection{Elementarne wyniki}

\begin{definition}[symetralna]
	Prostą prostopadłą do odcinka i przechodzącą przez jego środek nazywamy symetralną.
\end{definition}
Przez dwa punkty możemy przeprowadzić nieskończenie wiele okręgów, ich środki leżą na symetralnej odcinka łączącego wspomniane dwa punkty.
Przez trzy punkty możemy przeprowadzić jeden okrąg, jeśli nie są współliniowe (i zero w przeciwnym razie).
\todofoot{Czy to jest tu? Z czego to wynika? Może później, jeśli korzystamy w dowodzie ze zbyt wielu aksjomatów? Guzicki s. 14, 15.}
\begin{proposition} % Guzicki s. 126
	Punkt $P$ leży na symetralnej odcinka $AB$ wtedy i tylko wtedy, gdy jest jednakowo oddalony od końców tego odcinka: $|AP| = |BP|$.
\end{proposition}

% TODO: https://en.wikipedia.org/wiki/Bisection#Line_segment_bisector

Okrąg przechodzący przez wszystkie wierzchołki wielokąta nazywamy okręgiem opisanym na tym wielokącie.
Mówimy także, że wielokąt jest wpisany w okrąg.

\begin{proposition}
	\label{hartshorne_52}
    Niech $AB$ będzie odcinkiem.
	Istnieje wtedy trójkąt równoramienny, którego podstawą jest $AB$.
\end{proposition}

Powyższe stwierdzenie jest ciekawe, bo jest prawdziwe na płaszczyźnie Hilberta, tzn. jego prawdziwość nie zależy od aksjomatu Pascha.
(W geometrii nieeuklidesowej może nie istnieć trójkąt równoboczny o danej podstawie).

\begin{proposition}
	\label{hartshorne_52}
	Linia środkowa (odcinek łączący środki pewnych dwóch boków trójkąta) jest równoległa do trzeciego boku.
    Jej długość jest dwukrotnie mniejsza od długości tego boku.
\end{proposition}
% Hartshorne s. 52

Tego stwierdzenia nie ma w Elementach Euklidesa, ale można wyprowadzić je z księgi I (I.29, I.26, I.34), jak wspomina Hartshorne \cite[s. 52. 53]{hartshorne2000}.

\begin{corollary}
	Niech $ABC$ będzie trójkątem, zaś punkty $D$, $E$ i $F$ środkami jego boków.
	Wtedy cztery małe trójkąty utworzone na bokach $DE$, $EF$, $FD$ są przystające do siebie.
\end{corollary}

Do tego wniosku potrzeba dodatkowo cechy przystawania bok-bok-bok (I.8).

\subsubsection{Okręgi?}

\begin{proposition}
    Niech $\Gamma$ będzie okręgiem o środku $O$ oraz promieniu $OA$.
    Wtedy prosta prostopadła do $OA$, która przechodzi przez $A$, jest styczną do okręgu, leżącą (poza punktem $A$) na zewnątrz okręgu $\Gamma$.
    Odwrotnie, każda prosta, która jest styczna w punkcie $A$ do okręgu $\Gamma$, musi być prostopadła do prostej $OA$.
\end{proposition} % Hartshorne 105

\begin{corollary}
    Przez każdy punkt okręgu przechodzi dokładnie jedna styczna do tego okręgu.
\end{corollary} % Hartshorne 105

\begin{corollary}
    Prosta, która nie jest styczna do okręgu i nie jest z nim rozłączna, musi przecinać go w dokładnie dwóch punktach.
\end{corollary} % Hartshorne 106

\begin{proposition}
    Niech $O_1, O_2, A$ będą trzema punktami.
    Następujące warunki są równoważne: punkty $A, O_1, O_2$ są współliniowe; okręgi o promieniach $O_1A$, $O_2A$ są styczne.
\end{proposition} % Hartshorne 105

\begin{corollary}
    Dwa okręgi, które nie są rozłączne i nie są styczne, mają dokładnie dwa punkty wspólne.
\end{corollary} % Hartshorne 106

Potrzebujemy dodatkowego aksjomatu, by okręgi, od których oczekujemy, że się przetną, naprawdę się przecięły:

\begin{axiom}[o części wspólnej dwóch okręgów, E]
    \label{axiom_e}
    Niech $\Gamma_1$, $\Gamma_2$ będą dwoma okręgami takimi, że każdy zawiera pewien punkt leżący wewnątrz drugiego.
    Wtedy istnieje punkt leżący na obydwu okręgach równocześnie.
\end{axiom}

Płaszczyznę Pitagorasa, która spełnia ten aksjomat, nazywamy płaszczyzną Euklidesa.

\begin{proposition}[o części wspólnej prostej i okręgu, LCI]
    Na płaszczyźnie Hilberta spełniającej aksjomat E, jeśli prosta $l$ zawiera punkt leżący wewnątrz okręgu $\Gamma$, to przecina ten okrąg.
\end{proposition}

Dopiero teraz jesteśmy w stanie uratować twierdzenia (I.1), (I.22), (III.I), (III.17) z Elementów Euklidesa, do tego wszystkie dalsze wyniki aż do III.19 (ponieważ III.20 i dalej wymagany jest aksjomat Playfaira).
% [23, Exercise 16.11] gives an example of I.22 failure.

\todofoot{chord cięciwa, saggita, sieczna}

\subsubsection{Twierdzenie Sylvestera-Gallaia}
\begin{theorem}[Sylvestera-Gallaia]
	Dla każdego skończonego zbioru punktów na płaszczyźnie istnieje prosta, która przechodzi przez dokładnie dwa albo wszystkie punkty.
\end{theorem}

Mamy wrażenie, że zaczęło się w 1893 roku, kiedy James Sylvester postawił problem.
Być może zainspirowała go konfiguracją Hessego\footnote{Konfiguracja Hessego to 12 prostych przez 9 punktów na zespolonej płaszczyźnie rzutowej, gdzie każdy punkt leży na 4 prostych, a każda prosta przechodzi przez 3 punkty}.
Herbert Woodall szybko zaproponował rozwiązanie, gdzie równie szybko wychwycono usterkę.
Dopiero w 1941 roku Eberhard Melchior udowodnił trochę mocniejsze stwierdzenie niż rzutowy dual ówczesnej hipotezy (że prostych przez dokładnie dwa punkty jest co najmniej trzy).
Nieświadomy tego, Paul ErdErdős postawił hipotezę na nowo w~1943 roku, a Tibor Gallai w 1944 roku dodał swój dowód (ponownie wykorzystując elementy geometrii rzutowej).
Wraz z upływem czasu pojawiały się inne, ciekawe rozumowania.
Na przykład Leroy Kelly wykorzystał własności metryki, co oburzyło Harolda Coxetera i skłoniło go do opublikowania kolejnego dowodu, korzystającego jedynie z aksjomatów geometrii uporządkowania.
(Aigner, Ziegler uważają dowód Kelly'ego za najlepszy).

Niech $t_2(n)$ oznacza minimalną liczbę prostych przez dwa punkty w dowolnym ułożeniu $n$ punktów.
Melchior pokazał, że $t_2(n) \ge 3$.
Wynik sukcesywnie poprawiano:
de Bruijn \cite{debruijn_1948} zapytał, czy $t_2(n)$ dąży do nieskończoności,
Theodore Motzkin \cite{motzkin_1951} udzielił twierdzącej odpowiedz, bo $t_2(n) \ge \sqrt{n}$.
Potem Gabriel Dirac \cite{dirac_1951} przypuścił, że $t_2(n) \ge \lfloor n/2\rfloor$, co nie zostawia wiele miejsca na poprawki, bo dla parzystych $n \ge 6$ zachodzi $t_2(n) \le n/2$, jak pokazał pomysłową konstrukcją Károly Böröczky.
Dla nieparzystych $n$ wiemy tylko, że ten kres jest realizowany dla $n = 7$ (Kelly, Moser \cite{kelly_1958} w 1958) i $n = 13$ (Crowe, McKee \cite{mckee_1968} w 1968).
Najnowszy wynik, o jakim nam wiadomo, to Csimy, Sawyera \cite{csima_1993}: że $t_2(n) \ge \lceil 6n/13 \rceil$.

\subsection{Aksjomaty Tarskiego}
\todofoot{Strona ,,Tarski's axioms'' na en-wiki}
% TODO: https://en.wikipedia.org/wiki/Tarski%27s_axioms

\subsection{Origami}
% TODO: https://en.wikipedia.org/wiki/Huzita–Hatori_axioms => ORIGAMI
\todofoot{Huzita-Hatori axioms}

%

\section{Konstrukcje klasyczne}
\subsection{Proste}
% TODO: Hartshorne s. 103
\begin{problem}
    Dwusieczna kąta.
\end{problem}

\begin{problem}
    Środek odcinka.
\end{problem}

\begin{problem}
    Prosta prostopadła do prostej, przechodząca przez punkt.
\end{problem}

\begin{problem}
    Prosta równoległa do prostej, przechodząca przez punkt.
\end{problem}





\subsection{Trudniejsze}


\begin{problem}
    Dany jest odcinek $AB$ oraz punkt $P$ wewnątrz okręgu.
    Skonstruować cięciwę tego okręgu, która przechodzi przez punkt $P$ o długości takiej samej, jak odcinek $AB$.
\end{problem}
% Hartshorne s. 26

\begin{problem}
    Dany jest odcinek $AB$, inny odcinek o długości $d$ oraz kąt $\alpha$.
    Skonstruować trójkąt $ABC$ tak, by kąt przy wierzchołku $C$ miał miarę $\alpha$, zaś suma długości ramion tego kąta była równa $d$.
\end{problem}
% Hartshorne s. 26

\begin{problem}
    Dane są dwa okręgi takie, że żaden nie jest zawarty w drugim.
    Skonstruować styczną do obydwu okręgów.
\end{problem}
% Hartshorne s. 26

\begin{problem}
    Dany jest okrąg $\Gamma$ oraz jego środek $O$.
    Skonstruować trzy przystające okręgi, które są styczne do pozostałych dwóch oraz do $\Gamma$. \hfill \emph{(13 kroków)}. % Hartshorne s. 51
\end{problem}
% Hartshorne s. 26

\begin{problem}
    Dany jest okrąg $\Gamma$ oraz dwa punkty $A$ i $B$.
    Skonstrować punkt $C$ na okręgu $\Gamma$ tak, by odcinek łączący punkty przecięcia prostych $CA$, $CB$ z okręgiem $\Gamma$ był równoległy do odcinka $AB$.
\end{problem}
% Hartshorne s. 58-59

\begin{problem}
    Skonstruować trzy parami styczne okręgi, każdy o innym promieniu, których środki nie są współliniowe. \hfill \emph{(7 kroków)}. % Hartshorne s. 62
\end{problem}

\subsection{Uszkodzone przyrządy}
\begin{problem}[połamana linijka]
    Dane są dwa punkty $A$ i $B$ na płaszczyźnie, odległe od siebie o około trzy nible.
    Mając do dyspozycji fragment linijki o długości jednej nibli oraz sprawny cyrkiel, narysować odcinek $AB$.
\end{problem}
% Hartshorne s. 25

\begin{problem}[zardzewiały cyrkiel]
    Dane są dwa punkty $A$ i $B$ na płaszczyźnie, odległe od siebie o około pięć nibli.
    Mając do dyspozycji zardzewiały cyrkiel, którym można kreślić jedynie okręgi o promieniu dwóch nibli, skonstruować trójkąt równoboczny oparty o bok $AB$.
\end{problem}

\begin{problem}
    Dany jest punkt $A$ leżący na prostej $l$.
    Skonstruować prostą prostopadłą do $l$ przechodzącą przez $A$ przy użyciu linijki i zardzewiałego cyrkla.
\end{problem}
% Hartshorne s. 25

\begin{problem}
    Dany jest punkt $A$ leżący ponad cztery nible od prostej $l$.
    Skonstruować prostą prostopadłą do $l$ przechodzącą przez $A$ przy użyciu linijki i zardzewiałego cyrkla.
\end{problem}
% Hartshorne s. 25

\begin{problem}
    Dane są trzy niewspółliniowe punkty $A$, $B$ oraz $C$.
    Skonstrować punkt $D$ na prostej $AC$ tak, żeby odcinki $AD$ oraz $AB$ były równej długości przy użyciu linijki i zardzewiałego cyrkla.
\end{problem}
% Hartshorne s. 26

\begin{problem}
    Dany jest odcinek $AB$ o długości ponad dwóch nibli oraz prosta $l$, która nie przechodzi przez końce odcinka.
    Skonstrować punkt $C$ na prostej $l$ tak, żeby odcinki $AB$ oraz $AC$ były równej długości przy użyciu linijki i zardzewiałego cyrkla.
\end{problem}
% Hartshorne s. 26

\begin{problem}
    Czy wszystkie konstrukcje, które można wykonać cyrklem i linijką, można wykonać też zardzewiałym cyrklem i linijką?
\end{problem}
% Hartshorne s. 26

\subsection{Wyrocznia w Delfach}

\subsection{Wielokąty foremne}
GUZICKI-12 **wielokąty foremne** które są konstruowalne? (tw. wantzla itd.) konstrukcje przybliżone pięciokąta - durer i da vinci.
$n = 3$, $n = 4$, $n = 6$ (proste)

\begin{problem}
    Skonstrować trójkąt równoboczny wpisany w okrąg, którego środek nie jest znany. \hfill \emph{(7 kroków)}
\end{problem}

\begin{problem}
    Skonstrować kwadrat. \hfill \emph{(9 kroków)}
\end{problem}

\begin{problem}
    Skonstrować pięciokąt foremny.
\end{problem}

Piszą o tym Hartshorne \cite[s. 45-49]{hartshorne2000}.
Jeśli mamy zadany jeden z jego boków, konstrukcja wymaga 11 kroków. % Hartshorne s. 51



$n = 17$

$n = 7$ (niemożliwe), możliwe ze znaczoną linijką: Hartshorne rozdział 30/31

\subsection{Stożkowe}
przecięcie prostej z parabolą (hartshorne s. 247)

s. 278 Hartshorne: problem Alhazen, równokąty widziane z dwóch punktów na okręgu

\subsection{Apolloniusz}
GUZICKI-19 **zadanie konstrukcyjne apolloniusza** wykorzystuje twierdzenie menelaosa



%

W 1803 roku Malfatti \cite{malfatti_1803} zainspirowany pewnym praktycznym zagadnieniem (wycinanie walców z graniastosłupa) postawi następujący problem:
\index[persons]{Malfatti, Gian Francesco}%

\begin{problem}[zadanie Malfattiego]
	\label{malfatti_problem}
	\index{zadanie!Malfattiego}%
	Dany jest trójkąt $\triangle ABC$.
	Skonstruować takie trzy parami styczne okręgi $\Gamma_A, \Gamma_B, \Gamma_C$, że okrąg $\Gamma_A$ (odpowiednio: $\Gamma_B$, $\Gamma_C$) jest wpisany w~kąt $\angle A$ (odpowiednio: $\angle B$, $\angle C$).
\end{problem}

% https://www.desmos.com/calculator/mqzextwkad?lang=pl
\begin{figure}[H] \centering
\begin{comment}
\begin{tikzpicture}[scale=.5]
\tkzDefPoints{0/0/A,10/2/B,6/7/C}
\tkzDefPoints{4.43012726/2.59439459/Oa}
\tkzDefCircle[R](Oa,1.67519375895) \tkzGetPoint{Oaa}
\tkzDrawCircle[line width=0.2mm](Oa,Oaa)

\tkzDefPoints{7.48168986/2.91734309/Ob}
\tkzDefCircle[R](Ob,1.39341015784) \tkzGetPoint{Obb}
\tkzDrawCircle[line width=0.2mm](Ob,Obb)

\tkzDefPoints{5.96721113/5.06490116/Oc}
\tkzDefCircle[R](Oc,1.23445046858) \tkzGetPoint{Occ}
\tkzDrawCircle[line width=0.2mm](Oc,Occ)

\tkzLabelPoint(A){$A$}
\tkzLabelPoint[anchor=center](Oa){$\Gamma_A$}
\tkzLabelPoint(B){$B$}
\tkzLabelPoint[anchor=center](Ob){$\Gamma_B$}
\tkzLabelPoint[above](C){$C$}
\tkzLabelPoint[anchor=center](Oc){$\Gamma_C$}
\tkzDrawPolygon[line width=0.3mm](A,B,C)
\end{tikzpicture}
\end{comment}
\caption{Trzy okręgi Malfattiego}
\end{figure}

Problem będzie rozważany na długo przed Malfattim, zajmie się nim Ajima Naonobu\footnote{Matematyk japoński, przypisze się mu wprowadzenie rachunku różniczkowo-całkowego do matematyki japońskiej.} w~XVIII wieku, a~jeszcze wcześniej Gilio de Cecco da Montepulciano w~rękopisie z~1384 roku.
\index[persons]{Ajima, Naonobu}%
\index[persons]{de Cecco da Montepulciano, Gilio}%

Malfatti wyprowadzi co następuje.
Niech $p$ będzie połową obwodu trójkąta, $r$ będzie promieniem okręgu wpisanego w~ten trójkąt zaś $d_A$, $d_B$, $d_C$ odległościami wierzchołków $A, B, C$ od środka tego okręgu.
Wtedy promienie okręgów Malfattiego wyrażają się wzorami
\begin{align}
	r_A & = \frac r 2 \cdot {\frac {s-r+d_A-d_B-d_C}{p-a}}, \\
	r_B & = \frac r 2 \cdot {\frac {s-r+d_B-d_A-d_C}{p-b}}, \\
	r_C & = \frac r 2 \cdot {\frac {s-r+d_C-d_A-d_B}{p-c}}.
\end{align}

Prostą konstrukcję okręgów opartą na dwustycznych zawdzięczymy Steinerowi \cite{steiner_1826} w~1826 roku;
\index[persons]{Steiner, Jakob}%
inne rozwiązania podadzą Lehmus \cite{lehmus_1819}, Catalan \cite{catalan_1846}, Adams \cite{adams_1846}, Derousseau \cite{derousseau_1895}, Pampuch \cite{pampuch_1904}.
% TODO: po poprawie bibliografii, podać tu index persons

(O~problemie napiszą też Bogdańska, Neugebauer \cite[s. 102]{neugebauer_2018}).

Malfatti postawi tak naprawdę inny problem: znalezienia trzech rozłącznych kół zawartych w~trójkącie, których suma pól jest maksymalna i~błędnie założy, że opisane wyżej okręgi stanowią rozwiązanie.
Pomyłkę zauważą najpierw bez dowodu Lob, Richmond \cite{lob_richmond_1930} w~1930 roku: z trójkąta równobocznego można wyciąć zachłannie kolejno trzy koła, ich łączna powierzchnia jest większa od powierzchni kół znalezionych przez Malfattiego o 1\%.
\index[persons]{Richmond, ?}%
\index[persons]{Lob, ?}%
Howard Eves powtórzy to dla stromych trójkątów równoramiennych o bardzo wąskiej podstawie i dużej wysokości około 1946 roku.
\index[persons]{Eves, Howard}%
% https://en.wikipedia.org/w/index.php?title=Howard_Eves&diff=831382284&oldid=750910758
Goldberg \cite{goldberg_1967} wykaże, że domniemanie Malfattiego nie daje nigdy kół o maksymalnej łącznej powierzchni.
Ostatnie słowo należy zaś do Zalgallera, Losa \cite{zalgaller_los_1992}, którzy znajdą trzy koła rozwiązujące problem Malfattiego w dowolnym trójkącie.
% TODO: Goldberg M., On the original Malfatti problem, Math. Mag. 40 (1967), 241-247.
\index[persons]{Zalgaller, VA?}%
\index[persons]{Los, GA?}%
% TODO: Zalgaller V.A., Los’ G.A., Solution of the Malfatti problem, Ukrain. Geom. Sb. 35 (1992), 14-33 (ang. J. Math. Sci. 72 (1994), 3163-3177).
% TODO: po poprawie bibliografii, podać tu index persons
% TODO: Lob, H.; Richmond, H. W. (1930), "On the Solutions of Malfatti's Problem for a Triangle", Proceedings of the London Mathematical Society, 2nd ser., 30 (1): 287-304, doi:10.1112/plms/s2-30.1.287.

Kryształowa kula nie potrafi przewidzieć, kto oceni, czy algorytm zachłanny zawsze znajduje $n \ge 4$ rozłącznych kół w trójkącie o maksymalnej łącznej powierzchni.

(O więcej niż jednym okręgu wpisanym w trójkąt pisaliśmy w podpodsekcji \ref{sssection_6_7_9_circles}).


\color{red}

\begin{problem}[zadanie Napoleona]
	Podzielić dany okrąg (bez znanego środka) na cztery łuki równej miary korzystając z cyrkla, ale nie linijki.
\end{problem}

Nie wiadomo, czy Napoleon wymyślił albo rozwiązał przedstawione wyżej zadanie konstrukcyjne.
Rozwiązanie: \cite[s. 116]{neugebauer} z wykorzystaniem okręgów Torricelliego.
\index{okrąg Torricelliego}%

\begin{problem}[zadanie Fermata]
	Dany jest trójkąt $ABC$.
	Znaleźć punkt $F$ taki, by suma $|FA| + |FB| + |FC|$ była możliwie najmniejsza.
\end{problem}

Powyższe zadanie rozwiązał Evangelista Torricelli, który dostał je w formie wyzwania od Fermata.
Rozwiązanie opublikował student Torricelliego, Viviani, w 1659 roku.
% TODO: Johnson, R. A. Modern Geometry: An Elementary Treatise on the Geometry of the Triangle and the Circle. Boston, MA: Houghton Mifflin, pp. 221-222, 1929.

% TODO: rozwiązanie https://en.wikipedia.org/wiki/Napoleon%27s_problem

\color{black}

Konstrukcje od \ref{delta_2024_12_start} do \ref{delta_2024_12_end} opisane są w czasopiśmie Delta, w numerze grudniowym z 2024 roku.
\todofoot{Dopisać cytowanie w formacie BibTeX}

\begin{geoconstruction}
    \label{delta_2024_12_start}
    Znając pięć punktów okręgu $\omega$, skonstruować styczną do $\omega$ w jednym z tych punktów.
\end{geoconstruction}

\begin{geoconstruction}
    Znając pięć punktów okręgu $\omega$, dla danej prostej $l$ przechodzącej przez jeden z nich wyznaczyć drugi punkt przecięcia $l$ i $\omega$.
\end{geoconstruction}

\begin{geoconstruction}
    Skonstruować środek jednego z dwóch okręgów mających dwa punkty wspólne.
\end{geoconstruction}

\begin{geoconstruction}
    \label{delta_2024_12_end}
    Skonstruować środek przynajmniej jednego z trzech okręgów nienależących do jednego pęku.
\end{geoconstruction}


%

%

\section{Geometrie nieeuklidesowe}
Tekst sekcji geometrie nieeuklidesowe.
\begin{enumerate}
	\item Aksjomat Arystotelesa.
	\item Lemat Proklusa.
	\item Aksjomat Claviusa.
	\item Aksjomat Clairauta.
	\item Aksjomat Simsona.
	\item Aksjomat Playfaire'a.
	Aksjomat Playfaira został nazwany na cześć szkockiego matematyka, który podał jego treść w podręczniku \emph{Elements of Geometry} z 1795 roku.
% % https://en.wikipedia.org/wiki/Playfair%27s_axiom
\index[persons]{Playfair, John}%
\index{aksjomat!Playfaira}%
	\item Aksjomat Wallisa.
	\item Aksjomat Bolyi.
	\item Czworokąt Saccheriego.
	\item Aksjomat Legendre'a.
	\item Model Poincarego.
	\item Geometria hiperboliczna.
\end{enumerate}

%

\section{Stereometria (wielościany)}
Tekst sekcji stereometria, na podstawie 8 rozdziału Hartshorne'a.



\subsection{Twierdzenie Sylvestera-Gallaia}
\begin{theorem}[Sylvestera-Gallaia]
	Dla każdego skończonego zbioru punktów na płaszczyźnie istnieje prosta, która przechodzi przez dokładnie dwa albo wszystkie punkty.
\end{theorem}

Mamy wrażenie, że zaczęło się w 1893 roku, kiedy James Sylvester postawił problem.
Być może zainspirowała go konfiguracją Hessego\footnote{Konfiguracja Hessego to 12 prostych przez 9 punktów na zespolonej płaszczyźnie rzutowej, gdzie każdy punkt leży na 4 prostych, a każda prosta przechodzi przez 3 punkty}.
Herbert Woodall szybko zaproponował rozwiązanie, gdzie równie szybko wychwycono usterkę.
Dopiero w 1941 roku Eberhard Melchior udowodnił trochę mocniejsze stwierdzenie niż rzutowy dual ówczesnej hipotezy (że prostych przez dokładnie dwa punkty jest co najmniej trzy).
Nieświadomy tego, Paul ErdErdős postawił hipotezę na nowo w~1943 roku, a Tibor Gallai w 1944 roku dodał swój dowód (ponownie wykorzystując elementy geometrii rzutowej).
Wraz z upływem czasu pojawiały się inne, ciekawe rozumowania.
Na przykład Leroy Kelly wykorzystał własności metryki, co oburzyło Harolda Coxetera i skłoniło go do opublikowania kolejnego dowodu, korzystającego jedynie z aksjomatów geometrii uporządkowania.
(Aigner, Ziegler uważają dowód Kelly'ego za najlepszy).

Niech $t_2(n)$ oznacza minimalną liczbę prostych przez dwa punkty w dowolnym ułożeniu $n$ punktów.
Melchior pokazał, że $t_2(n) \ge 3$.
Wynik sukcesywnie poprawiano:
de Bruijn \cite{debruijn_1948} zapytał, czy $t_2(n)$ dąży do nieskończoności,
Theodore Motzkin \cite{motzkin_1951} udzielił twierdzącej odpowiedz, bo $t_2(n) \ge \sqrt{n}$.
Potem Gabriel Dirac \cite{dirac_1951} przypuścił, że $t_2(n) \ge \lfloor n/2\rfloor$, co nie zostawia wiele miejsca na poprawki, bo dla parzystych $n \ge 6$ zachodzi $t_2(n) \le n/2$, jak pokazał pomysłową konstrukcją Károly Böröczky.
Dla nieparzystych $n$ wiemy tylko, że ten kres jest realizowany dla $n = 7$ (Kelly, Moser \cite{kelly_1958} w 1958) i $n = 13$ (Crowe, McKee \cite{mckee_1968} w 1968).
Najnowszy wynik, o jakim nam wiadomo, to Csimy, Sawyera \cite{csima_1993}: że $t_2(n) \ge \lceil 6n/13 \rceil$.

\subsection{Inwersje (UW-2)}
\begin{enumerate}
	\item Obrazy inwersyjne okręgów i prostych, konforemność inwersji, okręgi stałe inwersji, okręgi prostopadłe
	\item zmiana odległości przy inwersji, zmiana promienia okręgu przy inwersji,
	\item twierdzenie Ptolemeusza,
	\item łańcuchy Steinera
	\item formuła Kartezjusza
	\item formuła Fussa dla czworokątów,
	\item twierdzenie Feuerbacha.
\end{enumerate}

\subsection{Stożkowe (UW-2)}
\begin{enumerate}
	\item Ogniska elipsy i hiperboli, ognisko, kierownica i mimośród stożkowych, asymptoty hiperboli, konstrukcja stycznej do stożkowej, rzuty ustalonego ogniska na styczne, własności izogonalne stożkowych, równania kanoniczne stożkowych, elipsa jako przekrój walca.
	\item Ognisko, kierownica i mimośród stożkowej na przekroju stożka.
	\item Przekroje stożków ze sferami wpisanymi.
	\item Równanie ogólne stożkowej w układzie współrzędnych, duży i mały wyznacznik.
	\item Równania stożkowych we współrzędnych biegunowych.
\end{enumerate}

Neugebauer 262: w każdy właściwy czworobok zupełny da się wpisać dokładnie jedną parabolę, jej ogniskiem jest punkt Miquela czworoboku.
Jemieljanow: punkt Miquela właściwego czworoboku zupełnego leży na okręgu dziewięciu punktów trójkąta przekątnego tego czworoboku.
Droz-Farny: proste przechodzą przez ortocentrum trójkąta i są prostopadłe, wtedy środki odcinków leżą na jednej prostej.

\subsection{Przekształcenia afiniczne (UW-2)}
\begin{enumerate}
	\item Grupa przekształceń afinicznych od strony geometrycznej: powinowactwa osiowe, rozkład przekształcenia afinicznego na podobieństwo i powinowactwo osiowe, kierunki główne przekształcenia afinicznego.
	\item niezmienniczość stosunku pól przy przekształceniu afinicznym
	\item obraz okręgu przy przekształceniu afinicznym
\end{enumerate}

\subsection{UW-3}
\begin{enumerate}
	\item zna pojęcie płaszczyzny rzutowej rzeczywistej (równoważne sformułowania), dwustosunku, definicję przekształceń rzutowych łańcuchów, pęków, stożkowych, pęków stycznych do stożkowych. 
	\item Rozumie, czym są stożkowe w ujęciu rzutowym, zna typy stożkowych.
	\item Zna i potrafi stosować twierdzenia Steinera i Braikenridge'a-Maclaurina.
	\item Wie w jaki sposób określa się rzutowo ogniska i kierownice stożkowych.
\end{enumerate}

\subsection{Guzicki}
\begin{enumerate}
	\item Złoty podział i pięciokąt.
	\item Zagadnienie izoperymetryczne (6)
	\item nierówności geometryczne: stosunek sumy środkowych do obwodu leży między 3/4 i 1 (s. 355), $s <= p^2 / 3 \sqrt 3$ - przypomnienie nierówności izoperymetrycznej. nierówność eulera (R >= 2r), Mitrinovica, Leibniza, Weitzenbocka (s. 362). Twierdzenie Eulera: $d^2 = R^2 - 2Rr$. nierówność Erdosa-Mordella: P leży wewnątrz trójkąta, K L M to rzuty na boki. Wtedy PA + PB + PC >= 2 (PK + PL + PM). Mikołaj z Kuzy: $\sin x / x < (2 + \cos x) / 3$. Snellius-Huygens: $2 \sin x + \tan x > 3x$.
	\item przekątne w wielokącie, tw. Heinekena % n nieparzyste -> w n-kącie foremnym żadne trzy przekątne nie przecinają się -> https://arxiv.org/pdf/math/9508209v3 ... In the 1960s, Heineken [6] gave a delightful argument which generalized this to all odd n,
\end{enumerate}

\subsection{Starocie}
Twierdzenie Chasles'a: każda izometria płaszczyzny jest złożeniem co najwyżej trzech symetrii osiowych.
Symetria osiowa z poślizgiem.
Słowo Banacha.
Klasyfikacja podobieństw.
Okrąg siedmiu punktów. % https://mathworld.wolfram.com/BrocardCircle.html ?
Przekształcenia afiniczne i rzutowe.
% https://www.cut-the-knot.org/Curriculum/Geometry/HeronsProblem.shtml
% This one is a basic optimization problem. It's quite famous, being discussed in Heron's Catoptrica (On Mirrors from the Greek word Katoptron Catoptron = Mirror) that, in all likelihood, saw the light of day some 2000 years ago.
Pitagorasa % https://en.wikipedia.org/wiki/Pythagorean_theorem
% https://en.wikipedia.org/wiki/Spiral_of_Theodorus

gnomon % https://en.wikipedia.org/wiki/Theorem_of_the_gnomon

Czwarty aksjomat uporządkowania znalazł Moritz Pasch \cite{pasch_1882} w 1882 roku.
\index[persons]{Pasch, Moritz}
\index{aksjomat!Pascha}

\subsection{Zadania}
\textbf{Zadanie} (Guzicki, s. 304).
Na bokach $AB$, $BC$, $CD$ i $DA$ czworokąta wypukłego $ABCD$ zbudowano, na zewnątrz czworokąta, kwadraty $ABFE$, $BCHG$, $CDJI$ i $DALK$.
Punkty $P$, $Q$, $R$ i $S$ są odpowiednio środkami kwadratów $ABFE$, $BCHG$, $CDJI$ i $DALK$.
Udowodnij, że odcinki $PR$ i $QS$ są równej długości oraz wzajemnie prostopadłe.

\textbf{Zadanie} (Guzicki, s. 306).
(XLIV OM, zadanie 5/I).
Dana jest półpłaszczyzna oraz punkty $A$ i $C$ na jej krawędzi.
Dla każdego punktu $B$ tej półpłaszczyzny rozważamy kwadraty $ABKL$ i $BCMN$ leżące na zewnątrz trójkąta $ABC$.
Wyznaczają one odpowiadającą punktowi $B$ prostą $LM$.
Udowodnij, że wszystkie proste odpowiadające różnym położeniom punktu $B$ przechodzą przez jeden punkt.

\textbf{Zadanie} (Guzicki, s. 306).
Na bokach $AB$ i $AC$ trójkąta $ABC$ zbudowano, po jego zewnętrznej stronie, kwadraty $ABDE$ i $ACFG$.
Punkty $M$ i $N$ są odpowiednio środkami odcinków $DG$ i $EF$.
Wyznacz możliwe wartości wyrażenia $MN / BC$.

\textbf{Zadanie} (Guzicki, s. 307)
(TWIERDZENIE NAPOLEONA)
Na bokach $AB$, $BC$ i $CA$ trójkąta $ABC$ zbudowano, na zewnątrz trójkąta, trójkąty równoboczne $ABF$, $BCD$ i $CAE$.
Udowodnij, że środki tych trójkątów równobocznych są wierzchołkami trójkąta równobocznego.

\textbf{Zadanie} (Guzicki, s. 308)
Na bokach $AB$, $BC$ i $CA$ trójkąta $ABC$ wybrano odpowiednio punkty $D$, $E$ i $F$ tak, że $AD : DB = BE : EC = CF : FA$.
Udowodnij, że jeśli trójkąt $DEF$ jest równoboczny, to trójkąt $ABC$ też jest równoboczny.

\textbf{Zadanie} (Guzicki, s. 310)
(XLV OM, zadanie 7/I)
Na zewnątrz czworokąta wypukłego $ABCD$ budujemy trójkąty podobone $APB$, $BQC$, $CRD$, $DSA$ w ten sposób, że kąty $PAB, QBC, RCD, SDA$ są sobie równe i że kąty $PBA, QCB, RDS, SAD$ też są sobie równe.
Udowodnij, że jeśli czworokąt PQRS jest równoległobokiem, to czworokąt $ABCD$ też jest równoległobokiem.

%










% https://bookstore.ams.org/browse?Author=%22A.%20V.%20Akopyan%22
% https://geometry.ru/books/conic_e.pdf












\bibliography{geo-textbook}{}
\bibliographystyle{plain}

\raggedright
\indexprologue{\small Tekst prologu...}
\printindex

\indexprologue{\small Tekst prologu...}
\printindex[persons]

\end{document}







\section{Aksjomatyka}

Euklides wyróżnił kilka pojęć pierwotnych (takich jak punkt, który był dla Euklidesa \emph{tym, co nie ma żadnych części}) i pięć aksjomatów, przytoczonych za książką \emph{O Elementach Euklidesa}:

\begin{enumerate}
	\item Zakłada się, że od każdego punktu do każdego punktu można poprowadzić linię prostą.
	\item I że ograniczoną prostą można ciągle przedłużać po prostej.
	\item I że z każdego środka każdym rozwarciem można zakreślić kolo.
	\item I że wszystkie kąty proste są równe między sobą.
	\item I jeżeli prosta padająca na dwie proste tworzy po jednej stronie kąty wewnętrzne, które w sumie są mniejsze od dwóch prostych, to te proste przedłużone nieograniczenie schodzą się po tej stronie, po której kąty te w sumie są mniejsze od dwóch prostych.
\end{enumerate}

Pojęcia pierwotne i aksjomaty Euklidesa nie są jednak idealne.
Dlatego zamiast nich będziemy używać aksjomatów Hilberta podanych około 1899 roku.

\section{Aksjomaty Hilberta}
Aksjomatyka Hilberta używa trzech pojęć pierwotnych punktu, prostej, płaszczyzny oraz trzech relacji pierwotnych:
\begin{itemize}
	\item leżenia pomiędzy (jedna relacja między trójkami punktów),
	\item zawierania się  w (trzy relacje: między punktami i prostymi; punktami i płaszczyznami; prostymi i płaszczyznami) oraz
	\item przystawania (dwie relacje: między odcinkami; między kątami).
\end{itemize}
Będziemy czasem używać synonimów, takich jak: ,,punkt $A$ leży na prostej $a$'', ,,prosta $a$ przechodzi przez punkt $A$'', ,,prosta $a$ łączy punkty $A$ i $B$''.
Wymienimy najpierw wszystkie aksjomaty, a potem przeanalizujemy ich treść.

\begin{itemize}
	\item \textbf{aksjomaty incydencji}:
\begin{enumerate}
	\item Przez każde dwa punkty przechodzi dokładnie jedna prosta.
	\item Na każdej prostej leżą co najmniej dwa różne punkty.
	\item Pewne trzy punkty nie są współliniowe.
	\item (Pozostałe aksjomaty incydencji dotyczą przestrzeni trójwymiarowej).
\end{enumerate}
\item \textbf{aksjomat Playfaire'a}:
\begin{enumerate}
	\item Dla każdego punktu $A$ i każdej prostej $l$, istnieje co najwyżej jedna prosta równoległa do $l$, zawierająca $A$.
\end{enumerate}
\item \textbf{aksjomaty uporządkowania}: \begin{enumerate}
	\item Jeżeli punkt $B$ leży pomiędzy punktami $A$ i $C$, to leży też pomiędzy punktami $C$ i $A$, a wszystkie trzy leżą na jednej prostej.
	\item Między każdą parą punktów leży trzeci punkt.
	\item Dla każdych trzech punktów na prostej, tylko jeden z nich leży pomiędzy pozostałymi dwoma.
	\item (Pascha) Niech $A, B, C$ będą trzema niewspółliniiowymi punktami, zaś $l$ prostą, która nie przechodzi przez żaden z nich. Jeśli prosta $l$ zawiera punkt $D$ leżący między $A$ i $B$, to musi też zawierać punkt leżący między $A$ i $C$ albo punkt leżący między $B$ i $C$, ale nie obydwa te punkty.
\end{enumerate}
\item \textbf{aksjomaty przystawania} (zapis $\overline{AB} \cong \overline{CD}$ oznacza, że odcinki są przystające): \begin{enumerate}
	\item Niech $\overline{AB}$ będzie odcinkiem, a $r$ półprostą o początku w punkcie $C$. Istnieje dokładnie jeden punkt $D$ leżący na $r$ taki, że $\overline{AB} \cong \overline{CD}$.
	\item Jeśli $\overline{AB} \cong \overline{CD}$ i $\overline{AB} \cong \overline{EF}$, to $\overline{CD} \cong \overline{EF}$. Każdy odcinek przystaje do siebie.
	\item (dodawanie) Dane są trzy punkty $A, B, C$ na prostej takie, że $B$ leży pomiędzy $A$ i $C$; oraz trzy punkty $D, E, F$ na (być może innej) prostej takie, że $E$ leży pomiędzy $D$ i $F$.
	Jeśli $\overline{AB} \cong \overline{DE}$ i $\overline{BC} \cong \overline{EF}$, to $\overline{AC} \cong \overline{DF}$.
	\item (aksjomaty dla kątów)
	\item (aksjomaty dla kątów)
	\item (cecha przystawania bok-kąt-bok)
\end{enumerate}
\end{itemize}

Punkty nazywamy współliniowymi, kiedy istnieje prosta, która przechodzi przez każdy z~nich.
Aksjomat Playfaira został nazwany na cześć szkockiego matematyka, który podał jego treść w podręczniku \emph{Elements of Geometry} z 1795 roku.
% % https://en.wikipedia.org/wiki/Playfair%27s_axiom
\index[persons]{Playfair, John}%
\index{aksjomat!Playfaira}%
Potrzebna jest jeszcze definicja prostych równoległych:

\begin{definition}[równoległość]
	Dwie proste, które pokrywają się albo nie mają żadnych punktów wspólnych, nazywamy równoległymi.
\end{definition}

Czwarty aksjomat uporządkowania znalazł Moritz Pasch \cite{pasch_1882} w 1882 roku.
\index[persons]{Pasch, Moritz}
\index{aksjomat!Pascha}
Aksjomaty uporządkowania pozwalają mówić o odcinkach:

\begin{definition}[odcinek]
\index{odcinek}%
	Niech $A, B$ będą dwoma różnymi punktami.
	Zbiór punktów $A$, $B$ oraz wszystkich punktów leżących pomiędzy $A$ i $B$ nazywamy odcinkiem i oznaczamy $\overline{AB}$.
\end{definition}

\begin{definition}[trójkąt]
\index{trójkąt}%
\index{trójkąt!bok trójkąta}%
\index{trójkąt!wierzchołek trójkąta}%
	Niech $A, B, C$ będą trzema niewspółliniowymi punktami.
	Sumę odcinków $\overline{AB}$, $\overline{BC}$ i $\overline{AC}$ nazywamy trójkątem i oznaczamy $\triangle ABC$.
	Punkty $A, B, C$ są jego wierzchołkami, odcinki $\overline{AB}$, $\overline{BC}$ i $\overline{AC}$ bokami.
\end{definition}

Znając trójkąty, możemy wysłowić aksjomat Pascha inaczej: jeśli prosta $l$ przechodzi przez bok $\overline{AB}$ trójkąta $\triangle ABC$ i nie przechodzi przez wierzchołki $A, B$, to musi przecinać dokładnie jeden z boków $\overline{AC}$, $\overline{BC}$.

\begin{proposition}[rozdzielanie płaszczyzny]
	Niech $l$ będzie prostą.
	Zbiór punktów, które nie leżą na $l$ można podzielić na dwa niepuste podzbiory $S_1, S_2$ takie, że dwa punkty $A, B$ należą do tego samego zbioru ($S_1$ lub $S_2$) wtedy i tylko wtedy, gdy odcinek $\overline{AB}$ nie przecina prostej $l$.
\end{proposition}

Będziemy mówić, że dwa punkty leżą po tej samej stronie (albo po różnych stronach) prostej.

\begin{proof}
	Hartshorne \cite[s. 74--76]{hartshorne_2010}.
\end{proof}

\begin{proposition}[rozdzielanie prostej]
	Niech $l$ będzie prostą przechodzącą przez punkt $A$.
	Zbiór pozostałych punktów prostej $l$ można podzielić na dwa niepuste podzbiory $S_1, S_2$ takie, że dwa punkty $B, C$ należą do tego samego zbioru ($S_1$ lub $S_2$) wtedy i tylko wtedy, gdy punkt $A$ nie leży na odcinku $\overline{BC}$.
\end{proposition}

Znowu, pozwala to mówić o dwóch stronach prostej.

\begin{proof}
	Hartshorne \cite[s. 76--77]{hartshorne_2010}.
\end{proof}

\begin{definition}[półprosta]
	Niech $A, B$ będą dwoma punktami.
	Zbiór, do którego należą punkt $A$ oraz wszystkie punkty prostej $AB$, które leżą po tej samej stronie, co punkt $B$, nazywamy półprostą $\overrightarrow{AB}$ o początku w $A$.
\end{definition}

\begin{definition}[kąt]
	Sumę dwóch półprostych $\overrightarrow{AB}$ i $\overrightarrow{AC}$, które nie leżą na jednej prostej, nazywamy kątem i oznaczamy $\angle BAC$.
	Wnętrzem takiego kąta nazywamy zbiór punktów $D$ takich, że $D$ i $C$ leżą po tej samej stronie prostej $AB$, zaś $D$ i $B$ po tej samej stronie prostej $AC$.
\end{definition}

(W myśl tej definicji, nie istnieje kąt zerowy ani półpełny!).
Część wspólną wnętrz kątów $\angle ABC$, $\angle BCA$ i $\angle CAB$ nazywamy wnętrzem trójkąta $\triangle ABC$.

\begin{proposition}[o kuszy]
	Niech $\angle BAC$ będzie kątem, we wnętrzu którego leży punkt $D$.
	Wtedy półprosta $\overrightarrow{AD}$ przecina odcinek $\overline{BC}$.
\end{proposition}

\begin{proof}
	Hartshorne \cite[s. 77--78]{hartshorne_2010}.
\end{proof}

Trzeci aksjomat przystawania pozwala nam dodawać odcinki: jeśli dane są odcinki $\overline{AB}$ i $\overline{CD}$, zaś $r$ jest półprostą $\overrightarrow{AB}$ z punktem $E$ na sobie takim, że $\overline{CD} \cong \overline{BE}$, to możemy skonstruować sumę $AE = AB + CD$.

(Odejmowanie, porządek...)

\begin{definition}[płaszczyzna Hilberta]
	Zbiór punktów $\Pi$ z wyróżnionymi pewnymi podzbiorami (zwanymi liniami) oraz pojęciami leżenia pomiędzy, przystawania odcinków i przystawania kątów (tak jak opisaliśmy to wyżej), który spełnia wszystkie aksjomaty poza, być może, aksjomatem Pascha, nazywamy płaszczyzną Hilberta.
\end{definition}


Twierdzenie o dwusiecznej % https://en.wikipedia.org/wiki/Angle_bisector_theorem
The angle bisector theorem appears as Proposition 3 of Book VI in Euclid's Elements. 

The exterior angle theorem is Proposition 1.16 in Euclid's Elements, which states that the measure of an exterior angle of a triangle is greater than either of the measures of the remote interior angles. This is a fundamental result in absolute geometry because its proof does not depend upon the parallel postulate. % https://en.wikipedia.org/wiki/Exterior_angle_theorem

Konstrukcja pierwiastka z iloczynu:
The theorem is usually attributed to Euclid (ca. 360–280 BC), who stated it as a corollary to proposition 8 in book VI of his Elements. In proposition 14 of book II Euclid gives a method for squaring a rectangle, which essentially matches the method given here. Euclid however provides a different slightly more complicated proof for the correctness of the construction rather than relying on the geometric mean theorem.
% https://en.wikipedia.org/wiki/Geometric_mean_theorem


Hinge theorem % https://en.wikipedia.org/wiki/Hinge_theorem

twierdzenia geometrii koła:
- % https://en.wikipedia.org/wiki/Thales%27s_theorem
- The inscribed angle theorem states that an angle $\theta$ inscribed in a circle is half of the central angle $2\theta$ that subtends the same arc on the circle. 

% https://en.wikipedia.org/wiki/Intercept_theorem

% https://en.wikipedia.org/wiki/Inscribed_angle#Theorem

% https://en.wikipedia.org/wiki/Intersecting_chords_theorem
% https://en.wikipedia.org/wiki/Intersecting_secants_theorem
% https://en.wikipedia.org/wiki/Tangent%E2%80%93secant_theorem
% https://en.wikipedia.org/wiki/Power_of_a_point#Theorems
twierdzenie o siecznych


\subsection{Elementy, księga I}
Hartshorne analizuje teraz, które stwierdzenia z Elementów Euklidesa są nadal prawdziwe na płaszczyźnie Hilberta.
Nie opiszę tego lepiej, dlatego przedstawiam jedynie podsumowanie.

(I.1), czyli konstrukcja trójkąta równobocznego, nie wynika z aksjomatów płaszczyzny Hilberta (ćwiczenie 39.31).
(I.2) i (I.3) zastąpiliśmy aksjomatem C1, zaś (I.4) aksjomatem C6.
(I.5), że kąty przy podstawie trójkąta równoramiennego są przystające, nie wymaga zmian.
Tezę V w I księdze nazywa się często \emph{pons asinorum}, czyli mostem osłów; jeśli ktoś nie jest w stanie samodzielnie przeprowadzić tego dowodu, to nie może przekroczyć mostu i dalej studiować geometrii.
\index{pons asinorum}%
\index{most osłów|see {pons asinorum}}%
(I.6) to twierdzenie odwrotne do (I.5) i wymaga kosmetycznych zmian, podobnie jak (I.7).

Ale (I.8), czyli cecha przystawania bok-bok-bok musi zostać udowodniona zupełnie inaczej; nowe uzasadnienie zaproponował Hilbert.
\index{cecha przystawania bok-bok-bok}
Zaczynając od (I.9) mamy do czynienia z konstrukcjami cyrklem i linijką, co stanowi pewien problem, bo nie wiemy jeszcze, czy proste zawsze przecinają okręgi.
Hartshorne posiłkuje się słabszym twierdzeniem, że każdy odcinek może być podstawą trójkąta równoramiennego.
To wystarcza do naprawy (I.10) i (I.11), ale nie (I.12), że dowolny punkt można zrzutować prostopadle na prostą, która przez niego nie przechodzi.
Potrzeba znowu całkiem nowego rozumowania.

Tezy (I.13) do (I.21) są w porządku.
Teza (I.22) jest nie do uratowania; nie wiemy, czy dwa okręgi muszą się zawsze przecinać tak, jak oczekuje tego Euklides i istotnie ćwiczenie 16.11 u Hartshorne'a mówi, że w pewnych płaszczyznach Hilberta trójkąty użyte w dowodzie nie istnieją.
Dalej, (I.23) było dowodzone przy użyciu (I.22), ale u nas to jest po prostu aksjomat C4.
Tezy (I.24) do (I.27) i (I.31) znowu są w porządku.

Zatem wszystko, co pisze Euklides, od I.1 do I.28 bez I.1, I.22 można uratować.

Hartshorne 104
Definicja okręgu, środka, promienia

Hartshorne 105
Definicja stycznej


 % aksjomaty Euklidesa

\subsection{Aksjomaty Hilberta}
{\color{red}
Aksjomatyka Hilberta używa trzech pojęć pierwotnych punktu, prostej, płaszczyzny oraz trzech relacji pierwotnych:
\begin{itemize}
	\item leżenia pomiędzy (jedna relacja między trójkami punktów),
	\item zawierania się  w (trzy relacje: między punktami i prostymi; punktami i płaszczyznami; prostymi i płaszczyznami) oraz
	\item przystawania (dwie relacje: między odcinkami; między kątami).
\end{itemize}
Będziemy czasem używać synonimów, takich jak: ,,punkt $A$ leży na prostej $a$'', ,,prosta $a$ przechodzi przez punkt $A$'', ,,prosta $a$ łączy punkty $A$ i $B$''.
Wymienimy najpierw wszystkie aksjomaty, a potem przeanalizujemy ich treść.

\begin{itemize}
	\item \textbf{aksjomaty incydencji}:
\begin{enumerate}
	\item Przez każde dwa punkty przechodzi dokładnie jedna prosta.
	\item Na każdej prostej leżą co najmniej dwa różne punkty.
	\item Pewne trzy punkty nie są współliniowe.
	\item (Pozostałe aksjomaty incydencji dotyczą przestrzeni trójwymiarowej).
\end{enumerate}
\item \textbf{aksjomat Playfaire'a}:
\begin{enumerate}
	\item Dla każdego punktu $A$ i każdej prostej $l$, istnieje co najwyżej jedna prosta równoległa do $l$, zawierająca $A$.
\end{enumerate}
\item \textbf{aksjomaty uporządkowania}: \begin{enumerate}
	\item Jeżeli punkt $B$ leży pomiędzy punktami $A$ i $C$, to leży też pomiędzy punktami $C$ i $A$, a wszystkie trzy leżą na jednej prostej.
	\item Między każdą parą punktów leży trzeci punkt.
	\item Dla każdych trzech punktów na prostej, tylko jeden z nich leży pomiędzy pozostałymi dwoma.
	\item (Pascha) Niech $A, B, C$ będą trzema niewspółliniiowymi punktami, zaś $l$ prostą, która nie przechodzi przez żaden z nich. Jeśli prosta $l$ zawiera punkt $D$ leżący między $A$ i $B$, to musi też zawierać punkt leżący między $A$ i $C$ albo punkt leżący między $B$ i $C$, ale nie obydwa te punkty.
\end{enumerate}
\item \textbf{aksjomaty przystawania} (zapis $\overline{AB} \cong \overline{CD}$ oznacza, że odcinki są przystające): \begin{enumerate}
	\item Niech $\overline{AB}$ będzie odcinkiem, a $r$ półprostą o początku w punkcie $C$. Istnieje dokładnie jeden punkt $D$ leżący na $r$ taki, że $\overline{AB} \cong \overline{CD}$.
	\item Jeśli $\overline{AB} \cong \overline{CD}$ i $\overline{AB} \cong \overline{EF}$, to $\overline{CD} \cong \overline{EF}$. Każdy odcinek przystaje do siebie.
	\item (dodawanie) Dane są trzy punkty $A, B, C$ na prostej takie, że $B$ leży pomiędzy $A$ i $C$; oraz trzy punkty $D, E, F$ na (być może innej) prostej takie, że $E$ leży pomiędzy $D$ i $F$.
	Jeśli $\overline{AB} \cong \overline{DE}$ i $\overline{BC} \cong \overline{EF}$, to $\overline{AC} \cong \overline{DF}$.
	\item (aksjomaty dla kątów)
	\item (aksjomaty dla kątów)
	\item (cecha przystawania bok-kąt-bok)
\end{enumerate}
\end{itemize}

Punkty nazywamy współliniowymi, kiedy istnieje prosta, która przechodzi przez każdy z~nich.
Aksjomat Playfaira został nazwany na cześć szkockiego matematyka, który podał jego treść w podręczniku \emph{Elements of Geometry} z 1795 roku.
% % https://en.wikipedia.org/wiki/Playfair%27s_axiom
\index[persons]{Playfair, John}%
\index{aksjomat!Playfaira}%
Potrzebna jest jeszcze definicja prostych równoległych:

\begin{definition}[równoległość]
	Dwie proste, które pokrywają się albo nie mają żadnych punktów wspólnych, nazywamy równoległymi.
\end{definition}

Czwarty aksjomat uporządkowania znalazł Moritz Pasch \cite{pasch_1882} w 1882 roku.
\index[persons]{Pasch, Moritz}
\index{aksjomat!Pascha}
Aksjomaty uporządkowania pozwalają mówić o odcinkach:

\begin{definition}[odcinek]
\index{odcinek}%
	Niech $A, B$ będą dwoma różnymi punktami.
	Zbiór punktów $A$, $B$ oraz wszystkich punktów leżących pomiędzy $A$ i $B$ nazywamy odcinkiem i oznaczamy $\overline{AB}$.
\end{definition}

\begin{definition}[trójkąt]
\index{trójkąt}%
\index{trójkąt!bok trójkąta}%
\index{trójkąt!wierzchołek trójkąta}%
	Niech $A, B, C$ będą trzema niewspółliniowymi punktami.
	Sumę odcinków $\overline{AB}$, $\overline{BC}$ i $\overline{AC}$ nazywamy trójkątem i oznaczamy $\triangle ABC$.
	Punkty $A, B, C$ są jego wierzchołkami, odcinki $\overline{AB}$, $\overline{BC}$ i $\overline{AC}$ bokami.
\end{definition}

Znając trójkąty, możemy wysłowić aksjomat Pascha inaczej: jeśli prosta $l$ przechodzi przez bok $\overline{AB}$ trójkąta $\triangle ABC$ i nie przechodzi przez wierzchołki $A, B$, to musi przecinać dokładnie jeden z boków $\overline{AC}$, $\overline{BC}$.

\begin{proposition}[rozdzielanie płaszczyzny]
	Niech $l$ będzie prostą.
	Zbiór punktów, które nie leżą na $l$ można podzielić na dwa niepuste podzbiory $S_1, S_2$ takie, że dwa punkty $A, B$ należą do tego samego zbioru ($S_1$ lub $S_2$) wtedy i tylko wtedy, gdy odcinek $\overline{AB}$ nie przecina prostej $l$.
\end{proposition}

Będziemy mówić, że dwa punkty leżą po tej samej stronie (albo po różnych stronach) prostej.

\begin{proof}
	Hartshorne \cite[s. 74--76]{hartshorne_2010}.
\end{proof}

\begin{proposition}[rozdzielanie prostej]
	Niech $l$ będzie prostą przechodzącą przez punkt $A$.
	Zbiór pozostałych punktów prostej $l$ można podzielić na dwa niepuste podzbiory $S_1, S_2$ takie, że dwa punkty $B, C$ należą do tego samego zbioru ($S_1$ lub $S_2$) wtedy i tylko wtedy, gdy punkt $A$ nie leży na odcinku $\overline{BC}$.
\end{proposition}

Znowu, pozwala to mówić o dwóch stronach prostej.

\begin{proof}
	Hartshorne \cite[s. 76--77]{hartshorne_2010}.
\end{proof}

\begin{definition}[półprosta]
	Niech $A, B$ będą dwoma punktami.
	Zbiór, do którego należą punkt $A$ oraz wszystkie punkty prostej $AB$, które leżą po tej samej stronie, co punkt $B$, nazywamy półprostą $\overrightarrow{AB}$ o początku w $A$.
\end{definition}

\begin{definition}[kąt]
	Sumę dwóch półprostych $\overrightarrow{AB}$ i $\overrightarrow{AC}$, które nie leżą na jednej prostej, nazywamy kątem i oznaczamy $\angle BAC$.
	Wnętrzem takiego kąta nazywamy zbiór punktów $D$ takich, że $D$ i $C$ leżą po tej samej stronie prostej $AB$, zaś $D$ i $B$ po tej samej stronie prostej $AC$.
\end{definition}

(W myśl tej definicji, nie istnieje kąt zerowy ani półpełny!).
Część wspólną wnętrz kątów $\angle ABC$, $\angle BCA$ i $\angle CAB$ nazywamy wnętrzem trójkąta $\triangle ABC$.

\begin{proposition}[o kuszy]
	Niech $\angle BAC$ będzie kątem, we wnętrzu którego leży punkt $D$.
	Wtedy półprosta $\overrightarrow{AD}$ przecina odcinek $\overline{BC}$.
\end{proposition}

\begin{proof}
	Hartshorne \cite[s. 77--78]{hartshorne_2010}.
\end{proof}

Trzeci aksjomat przystawania pozwala nam dodawać odcinki: jeśli dane są odcinki $\overline{AB}$ i $\overline{CD}$, zaś $r$ jest półprostą $\overrightarrow{AB}$ z punktem $E$ na sobie takim, że $\overline{CD} \cong \overline{BE}$, to możemy skonstruować sumę $AE = AB + CD$.

(Odejmowanie, porządek...)

\begin{definition}[płaszczyzna Hilberta]
	Zbiór punktów $\Pi$ z wyróżnionymi pewnymi podzbiorami (zwanymi liniami) oraz pojęciami leżenia pomiędzy, przystawania odcinków i przystawania kątów (tak jak opisaliśmy to wyżej), który spełnia wszystkie aksjomaty poza, być może, aksjomatem Pascha, nazywamy płaszczyzną Hilberta.
\end{definition}


Twierdzenie o dwusiecznej % https://en.wikipedia.org/wiki/Angle_bisector_theorem
The angle bisector theorem appears as Proposition 3 of Book VI in Euclid's Elements. 

The exterior angle theorem is Proposition 1.16 in Euclid's Elements, which states that the measure of an exterior angle of a triangle is greater than either of the measures of the remote interior angles. This is a fundamental result in absolute geometry because its proof does not depend upon the parallel postulate. % https://en.wikipedia.org/wiki/Exterior_angle_theorem

Konstrukcja pierwiastka z iloczynu:
The theorem is usually attributed to Euclid (ca. 360–280 BC), who stated it as a corollary to proposition 8 in book VI of his Elements. In proposition 14 of book II Euclid gives a method for squaring a rectangle, which essentially matches the method given here. Euclid however provides a different slightly more complicated proof for the correctness of the construction rather than relying on the geometric mean theorem.
% https://en.wikipedia.org/wiki/Geometric_mean_theorem


Hinge theorem % https://en.wikipedia.org/wiki/Hinge_theorem

twierdzenia geometrii koła:
- % https://en.wikipedia.org/wiki/Thales%27s_theorem
- The inscribed angle theorem states that an angle $\theta$ inscribed in a circle is half of the central angle $2\theta$ that subtends the same arc on the circle. 

% https://en.wikipedia.org/wiki/Intercept_theorem

% https://en.wikipedia.org/wiki/Inscribed_angle#Theorem

% https://en.wikipedia.org/wiki/Intersecting_chords_theorem
% https://en.wikipedia.org/wiki/Intersecting_secants_theorem
% https://en.wikipedia.org/wiki/Tangent%E2%80%93secant_theorem
% https://en.wikipedia.org/wiki/Power_of_a_point#Theorems
twierdzenie o siecznych


\subsection{Elementy, księga I}
Hartshorne analizuje teraz, które stwierdzenia z Elementów Euklidesa są nadal prawdziwe na płaszczyźnie Hilberta.
Nie opiszę tego lepiej, dlatego przedstawiam jedynie podsumowanie.

(I.1), czyli konstrukcja trójkąta równobocznego, nie wynika z aksjomatów płaszczyzny Hilberta (ćwiczenie 39.31).
(I.2) i (I.3) zastąpiliśmy aksjomatem C1, zaś (I.4) aksjomatem C6.
(I.5), że kąty przy podstawie trójkąta równoramiennego są przystające, nie wymaga zmian.
Tezę V w I księdze nazywa się często \emph{pons asinorum}, czyli mostem osłów; jeśli ktoś nie jest w stanie samodzielnie przeprowadzić tego dowodu, to nie może przekroczyć mostu i dalej studiować geometrii.
\index{pons asinorum}%
\index{most osłów|see {pons asinorum}}%
(I.6) to twierdzenie odwrotne do (I.5) i wymaga kosmetycznych zmian, podobnie jak (I.7).

Ale (I.8), czyli cecha przystawania bok-bok-bok musi zostać udowodniona zupełnie inaczej; nowe uzasadnienie zaproponował Hilbert.
\index{cecha przystawania bok-bok-bok}
Zaczynając od (I.9) mamy do czynienia z konstrukcjami cyrklem i linijką, co stanowi pewien problem, bo nie wiemy jeszcze, czy proste zawsze przecinają okręgi.
Hartshorne posiłkuje się słabszym twierdzeniem, że każdy odcinek może być podstawą trójkąta równoramiennego.
To wystarcza do naprawy (I.10) i (I.11), ale nie (I.12), że dowolny punkt można zrzutować prostopadle na prostą, która przez niego nie przechodzi.
Potrzeba znowu całkiem nowego rozumowania.

Tezy (I.13) do (I.21) są w porządku.
Teza (I.22) jest nie do uratowania; nie wiemy, czy dwa okręgi muszą się zawsze przecinać tak, jak oczekuje tego Euklides i istotnie ćwiczenie 16.11 u Hartshorne'a mówi, że w pewnych płaszczyznach Hilberta trójkąty użyte w dowodzie nie istnieją.
Dalej, (I.23) było dowodzone przy użyciu (I.22), ale u nas to jest po prostu aksjomat C4.
Tezy (I.24) do (I.27) i (I.31) znowu są w porządku.

Zatem wszystko, co pisze Euklides, od I.1 do I.28 bez I.1, I.22 można uratować.

Hartshorne 104
Definicja okręgu, środka, promienia

Hartshorne 105
Definicja stycznej


} % aksjomaty Hilberta

\section{W przygotowaniu -- Geometria -- Uniwersytet Warszawski}
\subsection{Geometria I}
\subsubsection{X}
1. Przystawanie figur na płaszczyźnie. Cechy przystawania trójkątów. Własności równoległoboków. Problem Fagnano i problem Fermata. Kąty w okręgu: wpisane, kąty środkowe i kąty dopisane. Twierdzenia o kątach wpisanych, kątach środkowych i kątach dopisanych do okręgu. Kątowe warunki na istnienie okręgu przechodzącego przez cztery punkty. Zastosowanie: okrąg dziewięciu punktów, twierdzenie o prostej Simsona. Styczna do okręgu, okrąg wpisany w kąt. Okrąg wpisany w trójkąt, okręgi dopisane do trójkąta. Warunki istnienia okręgu stycznego do czterech prostych.

\subsubsection{X}
2. Stosunek podziału wektora. Twierdzenie Talesa, twierdzenie odwrotne i jego zastosowania. Pole. Pola wybranych figur, twierdzenie Pitagorasa. Pole zorientowane. Twierdzenie Newtona: środek okręgu wpisanego w czworokąt i środki przekątnych tego czworokąta są współliniowe. Twierdzenie Gaussa: środki przekątnych czworokąta zupełnego są współliniowe. Definicja jednokładności, podobieństwo figur. Cechy podobieństwa trójkątów. Stosunek pól figur podobnych. Iloczynowe warunki istnienia okręgu przechodzącego przez cztery punkty. Pojęcie potęgi punktu
względem okręgu. Twierdzenie Ptolemeusza.

\subsubsection{X}
3. Wielkości miarowe w trójkącie: wzór Herona, wzory na promienie okręgów wpisanych, dopisanych. Twierdzenie o dwusiecznej i okrąg Apoloniusza. Twierdzenie Cevy (wraz z trygonometryczną wersją), przykłady punktów szczególnych trójkąta: punkt Nagela, punkt Gergonne'a, punkt Lemoine'a. Punkty izogonalnie sprzężone w trójkącie. Twierdzenie Menelausa.

\subsubsection{Jednokładność}
Jednokładność.

Konstrukcja obrazu jednokładnego punktu, okręgu, prostej.

Środek jednokładności dwóch trójkątów.

Środki jednokładności dwóch okręgów.

Prosta Eulera w trójkącie (środek okręgu opisanego, środek ciężkości, ortocentrum).

Zastosowanie: Twierdzenie Pascala.

Twierdzenie Kirkmana: jeśli część wspólna dwóch trójkątów wpisanych w okrąg jest sześciokątem wypukłym, to główne przekątne tego sześciokąta przecinają się w jednym punkcie.

Grupa dylatacji na płaszczyźnie.

Twierdzenia o składaniu jednokładności i przesunięć, twierdzenie o środkach jednokładności trzech okręgów.

\subsubsection{Izometrie}
5. Grupa izometrii na płaszczyźnie. Konstrukcja obrazu punktu, okręgu, prostej przy translacji, obrocie i symetrii osiowej. Złożenie dwóch i złożenie trzech symetrii osiowych. Twierdzenia o składaniu izometrii. Klasyfikacja izometrii na płaszczyźnie. Izometrie parzyste i izometrie nieparzyste. Twierdzenie o redukcji. Twierdzenie Napoleona: środki ciężkości trójkątów równobocznych zbudowanych na bokach dowolnego trójkąta są wierzchołkami trójkąta równobocznego.

\subsubsection{Podobieństwa}
6. Grupa podobieństw płaszczyzny. Podobieństwa spiralne i odbicia dylatacyjne. Klasyfikacja podobieństw płaszczyzny.


\subsection{Geometria II UW}
1. Potęga punktu względem okręgu, oś potęgowa dwóch okręgów, środek potęgowy trzech okręgów, twierdzenie Brianchona, konstrukcja stycznej do okręgu samą linijką, okręgi współpękowe, twierdzenie Gaussa-Bodenmillera, twierdzenie o motylku, formuła Eulera na odległość między środkami okręgu opisanego i wpisanego (dla trójkąta), twierdzenie Ponceleta dla trójkąta.

2. Obrazy inwersyjne okręgów i prostych, konforemność inwersji, okręgi stałe inwersji, okręgi prostopadłe, zmiana odległości przy inwersji, twierdzenie Ptolemeusza, zmiana promienia okręgu przy inwersji, łańcuchy Steinera, formuła Kartezjusza, formuła Fussa dla czworokątów, twierdzenie Feuerbacha.

3. Ogniska elipsy i hiperboli, ognisko, kierownica i mimośród stożkowych, asymptoty hiperboli, konstrukcja stycznej do stożkowej, rzuty ustalonego ogniska na styczne, własności izogonalne stożkowych, równania kanoniczne stożkowych, elipsa jako przekrój walca. Ognisko, kierownica i mimośród stożkowej na przekroju stożka. Przekroje stożków ze sferami wpisanymi. Równanie ogólne stożkowej w układzie współrzędnych, duży i mały wyznacznik. Równania stożkowych we współrzędnych biegunowych.

4. Grupa przekształceń afinicznych od strony geometrycznej: powinowactwa osiowe, rozkład przekształcenia afinicznego na podobieństwo i powinowactwo osiowe, kierunki główne przekształcenia afinicznego, niezmienniczość stosunku pól przy przekształceniu afinicznym, obraz okręgu przy przekształceniu afinicznym
Literatura: 	

1. E. H. Askwith, D.D. A Course of Pure Geometry, Cambridge 1917.
2. H. Fukagawa, D. Pedoe, Japanese temple geometry problems. Sangaku, Charles Babbage Research Centre, Winnipeg 1989.
3. R. A. Johnson, Advanced Euclidean geometry: An elementary treatise on the geometry of the triangle and the circle, Dover Publications, Inc., New York 1960.
4. W. Pompe, Geometria kół, Wydawnictwo Szkolne OMEGA, Kraków 2019.
5. V. Prasolov, Zadaczi po planimietrii. Tom I-II (ros.), Nauka, Moskwa 1991

\subsection{Geometria III}
Geometria rzutowa: ujęcie od strony geometrycznej. Płaszczyzna rzutowa (rzeczywista), przekształcenia rzutowe prostych, pęków, stożkowych, pęków stycznych do stożkowych.
Twierdzenia Desarguesa, Pappusa, Pascala, Brianchona.
Dualność: biegun i biegunowa względem okręgu i stożkowych. Sprzężenie biegunowe. Inwolucje rzutowe, twierdzenia inwolucyjne. Pęki okręgów i stożkowych jako generatory inwolucji. Twierdzenie Ponceleta. Stożkowe w ujęciu rzutowym, twierdzenia Steinera i Braikenridge'a-Maclaurina. Rzutowe określenie ogniska i kierownicy stożkowych. Punkty urojone przecięcia prostej ze stożkową w ujęciu czysto geometrycznym.



\section{Trygonometria}
\subsection{Trygonometria.} Lorem ipsum dolor sit amet, consectetur adipiscing elit, sed do eiusmod tempor incididunt ut labore et dolore magna aliqua. Ut enim ad minim veniam, quis nostrud exercitation ullamco laboris nisi ut aliquip ex ea commodo consequat. Duis aute irure dolor in reprehenderit in voluptate velit esse cillum dolore eu fugiat nulla pariatur. Excepteur sint occaecat cupidatat non proident, sunt in culpa qui officia deserunt mollit anim id est laborum.
\subsubsection{Prawo sinusów}
Lorem ipsum dolor sit amet, consectetur adipiscing elit, sed do eiusmod tempor incididunt ut labore et dolore magna aliqua. Ut enim ad minim veniam, quis nostrud exercitation ullamco laboris nisi ut aliquip ex ea commodo consequat. Duis aute irure dolor in reprehenderit in voluptate velit esse cillum dolore eu fugiat nulla pariatur. Excepteur sint occaecat cupidatat non proident, sunt in culpa qui officia deserunt mollit anim id est laborum.
$$\frac{a}{\sin \alpha} = \frac{b}{\sin \beta} = \frac{c}{\sin \gamma} = 2R$$
% https://en.wikipedia.org/wiki/Law_of_sines

\subsubsection{Prawo cosinusów}
Lorem ipsum dolor sit amet, consectetur adipiscing elit, sed do eiusmod tempor incididunt ut labore et dolore magna aliqua. Ut enim ad minim veniam, quis nostrud exercitation ullamco laboris nisi ut aliquip ex ea commodo consequat. Duis aute irure dolor in reprehenderit in voluptate velit esse cillum dolore eu fugiat nulla pariatur. Excepteur sint occaecat cupidatat non proident, sunt in culpa qui officia deserunt mollit anim id est laborum.
$$c^2 = a^2 + b^2 - 2ab \cos \gamma$$
% https://en.wikipedia.org/wiki/Law_of_cosines

\textbf{Twierdzenie Stewarta}

\textbf{Wzór Brahmagupty}

\textbf{Twierdzenie Urquharta}

\textbf{Punkt i kąt Crelle'a-Brocarda}

\textbf{Twierdzenie o siódmym okręgu}

\textbf{Twierdzenie Caseya}

\textbf{Twierdzenie Taylora, okrąg, sześciokąt}

\textbf{Twierdzenie Eulera $1/4R^2$}

% https://en.wikipedia.org/wiki/Law_of_tangents

\subsubsection{Rozwiązywanie trójkątów}
Wzór Mollweide'a.
\index{wzór!Mollweide'a}%
Problem Hansena
\index{problem!Hansena}%
Problem Snelliusa-Pothenota.
\index{problem!Snelliusa-Pothenota}%
% https://en.wikipedia.org/wiki/Mollweide%27s_formula
% https://en.wikipedia.org/wiki/Snellius%E2%80%93Pothenot_problem
% https://en.wikipedia.org/wiki/Hansen%27s_problem

\section{Konstrukcje geometryczne}
\subsection{Twierdzenie Mohra-Mascheroniego}
Konstrukcje od \ref{delta_2024_12_start} do \ref{delta_2024_12_end} opisane są w czasopiśmie Delta, w numerze grudniowym z 2024 roku.
\todofoot{Dopisać cytowanie w formacie BibTeX}

\begin{geoconstruction}
    \label{delta_2024_12_start}
    Znając pięć punktów okręgu $\omega$, skonstruować styczną do $\omega$ w jednym z tych punktów.
\end{geoconstruction}

\begin{geoconstruction}
    Znając pięć punktów okręgu $\omega$, dla danej prostej $l$ przechodzącej przez jeden z nich wyznaczyć drugi punkt przecięcia $l$ i $\omega$.
\end{geoconstruction}

\begin{geoconstruction}
    Skonstruować środek jednego z dwóch okręgów mających dwa punkty wspólne.
\end{geoconstruction}

\begin{geoconstruction}
    \label{delta_2024_12_end}
    Skonstruować środek przynajmniej jednego z trzech okręgów nienależących do jednego pęku.
\end{geoconstruction}


Konstruowalna => stopień Q(x) nad Q to potęga 2, ale nie w drugą stronę.
Podwojenie sześcianu.
Trysekcja kąta.
<=: Hartshorne, papierowa strona 245.
17-kąt

\section{Stereometria}
\subsection{Pięć wielościanów}
Hartshorne: rozdział 8

\subsection{Cauchy's rigidity theorem}
Hartshorne: section 45

\subsection{Siamese dodecahedron}
John solid?
% https://en.wikipedia.org/wiki/Johnson_solid




% IMO problems
1959/4.
Construct a right triangle with given hypotenuse c such that the median
drawn to the hypotenuse is the geometric mean of the two legs of the triangle.

1959/5.
An arbitrary point M is selected in the interior of the segment AB. The
squares AM CD and M BEF are constructed on the same side of AB, with
the segments AM and M B as their respective bases. The circles circum-
scribed about these squares, with centers P and Q, intersect at M and also
at another point N. Let N ′ denote the point of intersection of the straight
lines AF and BC.
(a) Prove that the points N and N ′ coincide.
(b) Prove that the straight lines M N pass through a fixed point S indepen-
dent of the choice of M.
(c) Find the locus of the midpoints of the segments P Q as M varies between
A and B.
1959/6.
Two planes, P and Q, intersect along the line p. The point A is given in the
plane P, and the point C in the plane Q; neither of these points lies on the
straight line p. Construct an isosceles trapezoid ABCD (with AB parallel to
CD) in which a circle can be inscribed, and with vertices B and D lying in
the planes P and Q respectively.

% https://www.imo-official.org/problems.aspx


