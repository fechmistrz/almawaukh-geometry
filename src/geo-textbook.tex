\documentclass{greaseproof}
\begin{document}

% strona pierwsza
\thispagestyle{empty}
{\noindent\fontsize{18pt}{18pt}\selectfont Biblioteka Aleksandryjska, tom I}

\noindent\makebox[\linewidth]{\rule{\textwidth}{1pt}}

\newpage

% strona druga
\thispagestyle{empty}
\phantom{nothing}
\newpage

% strona trzecia
\thispagestyle{empty}
{\noindent\fontsize{18pt}{18pt}\selectfont Epafrodyt z Ptolemais}

\noindent\makebox[\linewidth]{\rule{\textwidth}{1pt}}

\vspace{10mm}

{\noindent\fontsize{24pt}{24pt}\selectfont \textbf{Geometria}}
\vspace{10mm}

{\noindent\fontsize{14pt}{14pt}\selectfont Wydanie zerowe (eksperymentalne)}

\newpage

% strona czwarta
\thispagestyle{empty}
\begin{figure}[H]
\begin{minipage}[b]{.48\linewidth}
{\noindent Epafrodyt z Eudoksos\\
do napisania\\
do napisania\\
do napisania}
\end{minipage}
\begin{minipage}[b]{.48\linewidth}
{\noindent do napisania\\
do napisania\\
do napisania\\
do napisania}
\end{minipage}
\end{figure}

{\noindent \textbf{Kategorie MSC 2020}\\do napisania} \vspace{5mm}

{\noindent \textbf{Tytuł oryginału}\\do napisania} \vspace{5mm}

{\noindent \textbf{Z greki tłumaczyła}\\do napisania} \vspace{5mm}

{\noindent \textbf{Okładkę zaprojektował}\\do napisania} \vspace{5mm}

{\noindent \textbf{Zredagował}\\do napisania} \vspace{5mm}

{\noindent \textbf{Zredagowała technicznie}\\do napisania} \vspace{5mm}

{\noindent \textbf{Złożyli i połamali}\\do napisania} \vspace{5mm}

{\noindent \textbf{Korekty dokonali}\\do napisania} \vfill

{\noindent Copyleft © 2024 by Antykwariat Czarnoksięski.
Książka, a żeby było śmieszniej także każda jej część, mogą być przedrukowywane oraz w jakikolwiek inny sposób reprodukowane czy powielane mechanicznie, fotooptycznie, zapisywane elektronicznie lub magnetycznie, oraz odczytywane w środkach publicznego przekazu bez pisemnej zgody wydawcy.
}

\vspace{5mm}
{
    \noindent
    Tekst udostępniany na licencji Creative Commons: uznanie autorstwa, użycie niekomercyjne. Przeczytaj więcej na \texttt{https://creativecommons.org/licenses/by-nc/4.0/deed.pl}.
}

\vspace{5mm}

{\noindent Przygotowano w systemie \TeX, wydrukowano na siarczystym papierze.}

% strona piąta
\newpage
\section*{Przedmowa}
Do napisania.

\begin{flushright}
Epafrodyt,\\gdzie, kiedy
\end{flushright}

\tableofcontents
% \pagestyle{fancy} % Enable default headers and footers again
\cleardoublepage % Start the following content on a new page

%

%

%

\section{Konstrukcje klasyczne}
\subsection{Proste}
% TODO: Hartshorne s. 103
\begin{problem}
    Dwusieczna kąta.
\end{problem}

\begin{problem}
    Środek odcinka.
\end{problem}

\begin{problem}
    Prosta prostopadła do prostej, przechodząca przez punkt.
\end{problem}

\begin{problem}
    Prosta równoległa do prostej, przechodząca przez punkt.
\end{problem}





\subsection{Trudniejsze}


\begin{problem}
    Dany jest odcinek $AB$ oraz punkt $P$ wewnątrz okręgu.
    Skonstruować cięciwę tego okręgu, która przechodzi przez punkt $P$ o długości takiej samej, jak odcinek $AB$.
\end{problem}
% Hartshorne s. 26

\begin{problem}
    Dany jest odcinek $AB$, inny odcinek o długości $d$ oraz kąt $\alpha$.
    Skonstruować trójkąt $ABC$ tak, by kąt przy wierzchołku $C$ miał miarę $\alpha$, zaś suma długości ramion tego kąta była równa $d$.
\end{problem}
% Hartshorne s. 26

\begin{problem}
    Dane są dwa okręgi takie, że żaden nie jest zawarty w drugim.
    Skonstruować styczną do obydwu okręgów.
\end{problem}
% Hartshorne s. 26

\begin{problem}
    Dany jest okrąg $\Gamma$ oraz jego środek $O$.
    Skonstruować trzy przystające okręgi, które są styczne do pozostałych dwóch oraz do $\Gamma$. \hfill \emph{(13 kroków)}. % Hartshorne s. 51
\end{problem}
% Hartshorne s. 26

\begin{problem}
    Dany jest okrąg $\Gamma$ oraz dwa punkty $A$ i $B$.
    Skonstrować punkt $C$ na okręgu $\Gamma$ tak, by odcinek łączący punkty przecięcia prostych $CA$, $CB$ z okręgiem $\Gamma$ był równoległy do odcinka $AB$.
\end{problem}
% Hartshorne s. 58-59

\begin{problem}
    Skonstruować trzy parami styczne okręgi, każdy o innym promieniu, których środki nie są współliniowe. \hfill \emph{(7 kroków)}. % Hartshorne s. 62
\end{problem}

\subsection{Uszkodzone przyrządy}
\begin{problem}[połamana linijka]
    Dane są dwa punkty $A$ i $B$ na płaszczyźnie, odległe od siebie o około trzy nible.
    Mając do dyspozycji fragment linijki o długości jednej nibli oraz sprawny cyrkiel, narysować odcinek $AB$.
\end{problem}
% Hartshorne s. 25

\begin{problem}[zardzewiały cyrkiel]
    Dane są dwa punkty $A$ i $B$ na płaszczyźnie, odległe od siebie o około pięć nibli.
    Mając do dyspozycji zardzewiały cyrkiel, którym można kreślić jedynie okręgi o promieniu dwóch nibli, skonstruować trójkąt równoboczny oparty o bok $AB$.
\end{problem}

\begin{problem}
    Dany jest punkt $A$ leżący na prostej $l$.
    Skonstruować prostą prostopadłą do $l$ przechodzącą przez $A$ przy użyciu linijki i zardzewiałego cyrkla.
\end{problem}
% Hartshorne s. 25

\begin{problem}
    Dany jest punkt $A$ leżący ponad cztery nible od prostej $l$.
    Skonstruować prostą prostopadłą do $l$ przechodzącą przez $A$ przy użyciu linijki i zardzewiałego cyrkla.
\end{problem}
% Hartshorne s. 25

\begin{problem}
    Dane są trzy niewspółliniowe punkty $A$, $B$ oraz $C$.
    Skonstrować punkt $D$ na prostej $AC$ tak, żeby odcinki $AD$ oraz $AB$ były równej długości przy użyciu linijki i zardzewiałego cyrkla.
\end{problem}
% Hartshorne s. 26

\begin{problem}
    Dany jest odcinek $AB$ o długości ponad dwóch nibli oraz prosta $l$, która nie przechodzi przez końce odcinka.
    Skonstrować punkt $C$ na prostej $l$ tak, żeby odcinki $AB$ oraz $AC$ były równej długości przy użyciu linijki i zardzewiałego cyrkla.
\end{problem}
% Hartshorne s. 26

\begin{problem}
    Czy wszystkie konstrukcje, które można wykonać cyrklem i linijką, można wykonać też zardzewiałym cyrklem i linijką?
\end{problem}
% Hartshorne s. 26

\subsection{Wyrocznia w Delfach}

\subsection{Wielokąty foremne}
GUZICKI-12 **wielokąty foremne** które są konstruowalne? (tw. wantzla itd.) konstrukcje przybliżone pięciokąta - durer i da vinci.
$n = 3$, $n = 4$, $n = 6$ (proste)

\begin{problem}
    Skonstrować trójkąt równoboczny wpisany w okrąg, którego środek nie jest znany. \hfill \emph{(7 kroków)}
\end{problem}

\begin{problem}
    Skonstrować kwadrat. \hfill \emph{(9 kroków)}
\end{problem}

\begin{problem}
    Skonstrować pięciokąt foremny.
\end{problem}

Piszą o tym Hartshorne \cite[s. 45-49]{hartshorne2000}.
Jeśli mamy zadany jeden z jego boków, konstrukcja wymaga 11 kroków. % Hartshorne s. 51



$n = 17$

$n = 7$ (niemożliwe), możliwe ze znaczoną linijką: Hartshorne rozdział 30/31

\subsection{Stożkowe}
przecięcie prostej z parabolą (hartshorne s. 247)

s. 278 Hartshorne: problem Alhazen, równokąty widziane z dwóch punktów na okręgu

\subsection{Apolloniusz}
GUZICKI-19 **zadanie konstrukcyjne apolloniusza** wykorzystuje twierdzenie menelaosa



%

W 1803 roku Malfatti \cite{malfatti_1803} zainspirowany pewnym praktycznym zagadnieniem (wycinanie walców z graniastosłupa) postawi następujący problem:
\index[persons]{Malfatti, Gian Francesco}%

\begin{problem}[zadanie Malfattiego]
	\label{malfatti_problem}
	\index{zadanie!Malfattiego}%
	Dany jest trójkąt $\triangle ABC$.
	Skonstruować takie trzy parami styczne okręgi $\Gamma_A, \Gamma_B, \Gamma_C$, że okrąg $\Gamma_A$ (odpowiednio: $\Gamma_B$, $\Gamma_C$) jest wpisany w~kąt $\angle A$ (odpowiednio: $\angle B$, $\angle C$).
\end{problem}

% https://www.desmos.com/calculator/mqzextwkad?lang=pl
\begin{figure}[H] \centering
\begin{comment}
\begin{tikzpicture}[scale=.5]
\tkzDefPoints{0/0/A,10/2/B,6/7/C}
\tkzDefPoints{4.43012726/2.59439459/Oa}
\tkzDefCircle[R](Oa,1.67519375895) \tkzGetPoint{Oaa}
\tkzDrawCircle[line width=0.2mm](Oa,Oaa)

\tkzDefPoints{7.48168986/2.91734309/Ob}
\tkzDefCircle[R](Ob,1.39341015784) \tkzGetPoint{Obb}
\tkzDrawCircle[line width=0.2mm](Ob,Obb)

\tkzDefPoints{5.96721113/5.06490116/Oc}
\tkzDefCircle[R](Oc,1.23445046858) \tkzGetPoint{Occ}
\tkzDrawCircle[line width=0.2mm](Oc,Occ)

\tkzLabelPoint(A){$A$}
\tkzLabelPoint[anchor=center](Oa){$\Gamma_A$}
\tkzLabelPoint(B){$B$}
\tkzLabelPoint[anchor=center](Ob){$\Gamma_B$}
\tkzLabelPoint[above](C){$C$}
\tkzLabelPoint[anchor=center](Oc){$\Gamma_C$}
\tkzDrawPolygon[line width=0.3mm](A,B,C)
\end{tikzpicture}
\end{comment}
\caption{Trzy okręgi Malfattiego}
\end{figure}

Problem będzie rozważany na długo przed Malfattim, zajmie się nim Ajima Naonobu\footnote{Matematyk japoński, przypisze się mu wprowadzenie rachunku różniczkowo-całkowego do matematyki japońskiej.} w~XVIII wieku, a~jeszcze wcześniej Gilio de Cecco da Montepulciano w~rękopisie z~1384 roku.
\index[persons]{Ajima, Naonobu}%
\index[persons]{de Cecco da Montepulciano, Gilio}%

Malfatti wyprowadzi co następuje.
Niech $p$ będzie połową obwodu trójkąta, $r$ będzie promieniem okręgu wpisanego w~ten trójkąt zaś $d_A$, $d_B$, $d_C$ odległościami wierzchołków $A, B, C$ od środka tego okręgu.
Wtedy promienie okręgów Malfattiego wyrażają się wzorami
\begin{align}
	r_A & = \frac r 2 \cdot {\frac {s-r+d_A-d_B-d_C}{p-a}}, \\
	r_B & = \frac r 2 \cdot {\frac {s-r+d_B-d_A-d_C}{p-b}}, \\
	r_C & = \frac r 2 \cdot {\frac {s-r+d_C-d_A-d_B}{p-c}}.
\end{align}

Prostą konstrukcję okręgów opartą na dwustycznych zawdzięczymy Steinerowi \cite{steiner_1826} w~1826 roku;
\index[persons]{Steiner, Jakob}%
inne rozwiązania podadzą Lehmus \cite{lehmus_1819}, Catalan \cite{catalan_1846}, Adams \cite{adams_1846}, Derousseau \cite{derousseau_1895}, Pampuch \cite{pampuch_1904}.
% TODO: po poprawie bibliografii, podać tu index persons

(O~problemie napiszą też Bogdańska, Neugebauer \cite[s. 102]{neugebauer_2018}).

Malfatti postawi tak naprawdę inny problem: znalezienia trzech rozłącznych kół zawartych w~trójkącie, których suma pól jest maksymalna i~błędnie założy, że opisane wyżej okręgi stanowią rozwiązanie.
Pomyłkę zauważą najpierw bez dowodu Lob, Richmond \cite{lob_richmond_1930} w~1930 roku: z trójkąta równobocznego można wyciąć zachłannie kolejno trzy koła, ich łączna powierzchnia jest większa od powierzchni kół znalezionych przez Malfattiego o 1\%.
\index[persons]{Richmond, ?}%
\index[persons]{Lob, ?}%
Howard Eves powtórzy to dla stromych trójkątów równoramiennych o bardzo wąskiej podstawie i dużej wysokości około 1946 roku.
\index[persons]{Eves, Howard}%
% https://en.wikipedia.org/w/index.php?title=Howard_Eves&diff=831382284&oldid=750910758
Goldberg \cite{goldberg_1967} wykaże, że domniemanie Malfattiego nie daje nigdy kół o maksymalnej łącznej powierzchni.
Ostatnie słowo należy zaś do Zalgallera, Losa \cite{zalgaller_los_1992}, którzy znajdą trzy koła rozwiązujące problem Malfattiego w dowolnym trójkącie.
% TODO: Goldberg M., On the original Malfatti problem, Math. Mag. 40 (1967), 241-247.
\index[persons]{Zalgaller, VA?}%
\index[persons]{Los, GA?}%
% TODO: Zalgaller V.A., Los’ G.A., Solution of the Malfatti problem, Ukrain. Geom. Sb. 35 (1992), 14-33 (ang. J. Math. Sci. 72 (1994), 3163-3177).
% TODO: po poprawie bibliografii, podać tu index persons
% TODO: Lob, H.; Richmond, H. W. (1930), "On the Solutions of Malfatti's Problem for a Triangle", Proceedings of the London Mathematical Society, 2nd ser., 30 (1): 287-304, doi:10.1112/plms/s2-30.1.287.

Kryształowa kula nie potrafi przewidzieć, kto oceni, czy algorytm zachłanny zawsze znajduje $n \ge 4$ rozłącznych kół w trójkącie o maksymalnej łącznej powierzchni.

(O więcej niż jednym okręgu wpisanym w trójkąt pisaliśmy w podpodsekcji \ref{sssection_6_7_9_circles}).


\color{red}

\begin{problem}[zadanie Napoleona]
	Podzielić dany okrąg (bez znanego środka) na cztery łuki równej miary korzystając z cyrkla, ale nie linijki.
\end{problem}

Nie wiadomo, czy Napoleon wymyślił albo rozwiązał przedstawione wyżej zadanie konstrukcyjne.
Rozwiązanie: \cite[s. 116]{neugebauer} z wykorzystaniem okręgów Torricelliego.
\index{okrąg Torricelliego}%

\begin{problem}[zadanie Fermata]
	Dany jest trójkąt $ABC$.
	Znaleźć punkt $F$ taki, by suma $|FA| + |FB| + |FC|$ była możliwie najmniejsza.
\end{problem}

Powyższe zadanie rozwiązał Evangelista Torricelli, który dostał je w formie wyzwania od Fermata.
Rozwiązanie opublikował student Torricelliego, Viviani, w 1659 roku.
% TODO: Johnson, R. A. Modern Geometry: An Elementary Treatise on the Geometry of the Triangle and the Circle. Boston, MA: Houghton Mifflin, pp. 221-222, 1929.

% TODO: rozwiązanie https://en.wikipedia.org/wiki/Napoleon%27s_problem

\color{black}

Konstrukcje od \ref{delta_2024_12_start} do \ref{delta_2024_12_end} opisane są w czasopiśmie Delta, w numerze grudniowym z 2024 roku.
\todofoot{Dopisać cytowanie w formacie BibTeX}

\begin{geoconstruction}
    \label{delta_2024_12_start}
    Znając pięć punktów okręgu $\omega$, skonstruować styczną do $\omega$ w jednym z tych punktów.
\end{geoconstruction}

\begin{geoconstruction}
    Znając pięć punktów okręgu $\omega$, dla danej prostej $l$ przechodzącej przez jeden z nich wyznaczyć drugi punkt przecięcia $l$ i $\omega$.
\end{geoconstruction}

\begin{geoconstruction}
    Skonstruować środek jednego z dwóch okręgów mających dwa punkty wspólne.
\end{geoconstruction}

\begin{geoconstruction}
    \label{delta_2024_12_end}
    Skonstruować środek przynajmniej jednego z trzech okręgów nienależących do jednego pęku.
\end{geoconstruction}


%

%

\section{Geometrie nieeuklidesowe}
Tekst sekcji geometrie nieeuklidesowe.
\begin{enumerate}
	\item Aksjomat Arystotelesa.
	\item Lemat Proklusa.
	\item Aksjomat Claviusa.
	\item Aksjomat Clairauta.
	\item Aksjomat Simsona.
	\item Aksjomat Playfaire'a.
	Aksjomat Playfaira został nazwany na cześć szkockiego matematyka, który podał jego treść w podręczniku \emph{Elements of Geometry} z 1795 roku.
% % https://en.wikipedia.org/wiki/Playfair%27s_axiom
\index[persons]{Playfair, John}%
\index{aksjomat!Playfaira}%
	\item Aksjomat Wallisa.
	\item Aksjomat Bolyi.
	\item Czworokąt Saccheriego.
	\item Aksjomat Legendre'a.
	\item Model Poincarego.
	\item Geometria hiperboliczna.
\end{enumerate}

%

\section{Stereometria (wielościany)}
Tekst sekcji stereometria, na podstawie 8 rozdziału Hartshorne'a.



\subsection{Twierdzenie Sylvestera-Gallaia}
\begin{theorem}[Sylvestera-Gallaia]
	Dla każdego skończonego zbioru punktów na płaszczyźnie istnieje prosta, która przechodzi przez dokładnie dwa albo wszystkie punkty.
\end{theorem}

Mamy wrażenie, że zaczęło się w 1893 roku, kiedy James Sylvester postawił problem.
Być może zainspirowała go konfiguracją Hessego\footnote{Konfiguracja Hessego to 12 prostych przez 9 punktów na zespolonej płaszczyźnie rzutowej, gdzie każdy punkt leży na 4 prostych, a każda prosta przechodzi przez 3 punkty}.
Herbert Woodall szybko zaproponował rozwiązanie, gdzie równie szybko wychwycono usterkę.
Dopiero w 1941 roku Eberhard Melchior udowodnił trochę mocniejsze stwierdzenie niż rzutowy dual ówczesnej hipotezy (że prostych przez dokładnie dwa punkty jest co najmniej trzy).
Nieświadomy tego, Paul ErdErdős postawił hipotezę na nowo w~1943 roku, a Tibor Gallai w 1944 roku dodał swój dowód (ponownie wykorzystując elementy geometrii rzutowej).
Wraz z upływem czasu pojawiały się inne, ciekawe rozumowania.
Na przykład Leroy Kelly wykorzystał własności metryki, co oburzyło Harolda Coxetera i skłoniło go do opublikowania kolejnego dowodu, korzystającego jedynie z aksjomatów geometrii uporządkowania.
(Aigner, Ziegler uważają dowód Kelly'ego za najlepszy).

Niech $t_2(n)$ oznacza minimalną liczbę prostych przez dwa punkty w dowolnym ułożeniu $n$ punktów.
Melchior pokazał, że $t_2(n) \ge 3$.
Wynik sukcesywnie poprawiano:
de Bruijn \cite{debruijn_1948} zapytał, czy $t_2(n)$ dąży do nieskończoności,
Theodore Motzkin \cite{motzkin_1951} udzielił twierdzącej odpowiedz, bo $t_2(n) \ge \sqrt{n}$.
Potem Gabriel Dirac \cite{dirac_1951} przypuścił, że $t_2(n) \ge \lfloor n/2\rfloor$, co nie zostawia wiele miejsca na poprawki, bo dla parzystych $n \ge 6$ zachodzi $t_2(n) \le n/2$, jak pokazał pomysłową konstrukcją Károly Böröczky.
Dla nieparzystych $n$ wiemy tylko, że ten kres jest realizowany dla $n = 7$ (Kelly, Moser \cite{kelly_1958} w 1958) i $n = 13$ (Crowe, McKee \cite{mckee_1968} w 1968).
Najnowszy wynik, o jakim nam wiadomo, to Csimy, Sawyera \cite{csima_1993}: że $t_2(n) \ge \lceil 6n/13 \rceil$.

\subsection{Inwersje (UW-2)}
\begin{enumerate}
	\item Obrazy inwersyjne okręgów i prostych, konforemność inwersji, okręgi stałe inwersji, okręgi prostopadłe
	\item zmiana odległości przy inwersji, zmiana promienia okręgu przy inwersji,
	\item twierdzenie Ptolemeusza,
	\item łańcuchy Steinera
	\item formuła Kartezjusza
	\item formuła Fussa dla czworokątów,
	\item twierdzenie Feuerbacha.
\end{enumerate}

\subsection{Stożkowe (UW-2)}
\begin{enumerate}
	\item Ogniska elipsy i hiperboli, ognisko, kierownica i mimośród stożkowych, asymptoty hiperboli, konstrukcja stycznej do stożkowej, rzuty ustalonego ogniska na styczne, własności izogonalne stożkowych, równania kanoniczne stożkowych, elipsa jako przekrój walca.
	\item Ognisko, kierownica i mimośród stożkowej na przekroju stożka.
	\item Przekroje stożków ze sferami wpisanymi.
	\item Równanie ogólne stożkowej w układzie współrzędnych, duży i mały wyznacznik.
	\item Równania stożkowych we współrzędnych biegunowych.
\end{enumerate}

Neugebauer 262: w każdy właściwy czworobok zupełny da się wpisać dokładnie jedną parabolę, jej ogniskiem jest punkt Miquela czworoboku.
Jemieljanow: punkt Miquela właściwego czworoboku zupełnego leży na okręgu dziewięciu punktów trójkąta przekątnego tego czworoboku.
Droz-Farny: proste przechodzą przez ortocentrum trójkąta i są prostopadłe, wtedy środki odcinków leżą na jednej prostej.

\subsection{Przekształcenia afiniczne (UW-2)}
\begin{enumerate}
	\item Grupa przekształceń afinicznych od strony geometrycznej: powinowactwa osiowe, rozkład przekształcenia afinicznego na podobieństwo i powinowactwo osiowe, kierunki główne przekształcenia afinicznego.
	\item niezmienniczość stosunku pól przy przekształceniu afinicznym
	\item obraz okręgu przy przekształceniu afinicznym
\end{enumerate}

\subsection{UW-3}
\begin{enumerate}
	\item zna pojęcie płaszczyzny rzutowej rzeczywistej (równoważne sformułowania), dwustosunku, definicję przekształceń rzutowych łańcuchów, pęków, stożkowych, pęków stycznych do stożkowych. 
	\item Rozumie, czym są stożkowe w ujęciu rzutowym, zna typy stożkowych.
	\item Zna i potrafi stosować twierdzenia Steinera i Braikenridge'a-Maclaurina.
	\item Wie w jaki sposób określa się rzutowo ogniska i kierownice stożkowych.
\end{enumerate}

\subsection{Guzicki}
\begin{enumerate}
	\item Złoty podział i pięciokąt.
	\item Zagadnienie izoperymetryczne (6)
	\item nierówności geometryczne: stosunek sumy środkowych do obwodu leży między 3/4 i 1 (s. 355), $s <= p^2 / 3 \sqrt 3$ - przypomnienie nierówności izoperymetrycznej. nierówność eulera (R >= 2r), Mitrinovica, Leibniza, Weitzenbocka (s. 362). Twierdzenie Eulera: $d^2 = R^2 - 2Rr$. nierówność Erdosa-Mordella: P leży wewnątrz trójkąta, K L M to rzuty na boki. Wtedy PA + PB + PC >= 2 (PK + PL + PM). Mikołaj z Kuzy: $\sin x / x < (2 + \cos x) / 3$. Snellius-Huygens: $2 \sin x + \tan x > 3x$.
	\item przekątne w wielokącie, tw. Heinekena % n nieparzyste -> w n-kącie foremnym żadne trzy przekątne nie przecinają się -> https://arxiv.org/pdf/math/9508209v3 ... In the 1960s, Heineken [6] gave a delightful argument which generalized this to all odd n,
\end{enumerate}

\subsection{Starocie}
Twierdzenie Chasles'a: każda izometria płaszczyzny jest złożeniem co najwyżej trzech symetrii osiowych.
Symetria osiowa z poślizgiem.
Słowo Banacha.
Klasyfikacja podobieństw.
Okrąg siedmiu punktów. % https://mathworld.wolfram.com/BrocardCircle.html ?
Przekształcenia afiniczne i rzutowe.
% https://www.cut-the-knot.org/Curriculum/Geometry/HeronsProblem.shtml
% This one is a basic optimization problem. It's quite famous, being discussed in Heron's Catoptrica (On Mirrors from the Greek word Katoptron Catoptron = Mirror) that, in all likelihood, saw the light of day some 2000 years ago.
Pitagorasa % https://en.wikipedia.org/wiki/Pythagorean_theorem
% https://en.wikipedia.org/wiki/Spiral_of_Theodorus

gnomon % https://en.wikipedia.org/wiki/Theorem_of_the_gnomon

Czwarty aksjomat uporządkowania znalazł Moritz Pasch \cite{pasch_1882} w 1882 roku.
\index[persons]{Pasch, Moritz}
\index{aksjomat!Pascha}

\subsection{Zadania}
\textbf{Zadanie} (Guzicki, s. 304).
Na bokach $AB$, $BC$, $CD$ i $DA$ czworokąta wypukłego $ABCD$ zbudowano, na zewnątrz czworokąta, kwadraty $ABFE$, $BCHG$, $CDJI$ i $DALK$.
Punkty $P$, $Q$, $R$ i $S$ są odpowiednio środkami kwadratów $ABFE$, $BCHG$, $CDJI$ i $DALK$.
Udowodnij, że odcinki $PR$ i $QS$ są równej długości oraz wzajemnie prostopadłe.

\textbf{Zadanie} (Guzicki, s. 306).
(XLIV OM, zadanie 5/I).
Dana jest półpłaszczyzna oraz punkty $A$ i $C$ na jej krawędzi.
Dla każdego punktu $B$ tej półpłaszczyzny rozważamy kwadraty $ABKL$ i $BCMN$ leżące na zewnątrz trójkąta $ABC$.
Wyznaczają one odpowiadającą punktowi $B$ prostą $LM$.
Udowodnij, że wszystkie proste odpowiadające różnym położeniom punktu $B$ przechodzą przez jeden punkt.

\textbf{Zadanie} (Guzicki, s. 306).
Na bokach $AB$ i $AC$ trójkąta $ABC$ zbudowano, po jego zewnętrznej stronie, kwadraty $ABDE$ i $ACFG$.
Punkty $M$ i $N$ są odpowiednio środkami odcinków $DG$ i $EF$.
Wyznacz możliwe wartości wyrażenia $MN / BC$.

\textbf{Zadanie} (Guzicki, s. 307)
(TWIERDZENIE NAPOLEONA)
Na bokach $AB$, $BC$ i $CA$ trójkąta $ABC$ zbudowano, na zewnątrz trójkąta, trójkąty równoboczne $ABF$, $BCD$ i $CAE$.
Udowodnij, że środki tych trójkątów równobocznych są wierzchołkami trójkąta równobocznego.

\textbf{Zadanie} (Guzicki, s. 308)
Na bokach $AB$, $BC$ i $CA$ trójkąta $ABC$ wybrano odpowiednio punkty $D$, $E$ i $F$ tak, że $AD : DB = BE : EC = CF : FA$.
Udowodnij, że jeśli trójkąt $DEF$ jest równoboczny, to trójkąt $ABC$ też jest równoboczny.

\textbf{Zadanie} (Guzicki, s. 310)
(XLV OM, zadanie 7/I)
Na zewnątrz czworokąta wypukłego $ABCD$ budujemy trójkąty podobone $APB$, $BQC$, $CRD$, $DSA$ w ten sposób, że kąty $PAB, QBC, RCD, SDA$ są sobie równe i że kąty $PBA, QCB, RDS, SAD$ też są sobie równe.
Udowodnij, że jeśli czworokąt PQRS jest równoległobokiem, to czworokąt $ABCD$ też jest równoległobokiem.

%























\bibliography{geo-textbook}{}
\bibliographystyle{plain}

\raggedright
\indexprologue{\small Tekst prologu...}
\printindex

\indexprologue{\small Tekst prologu...}
\printindex[persons]

\end{document}







\section{Aksjomatyka}

Euklides wyróżnił kilka pojęć pierwotnych (takich jak punkt, który był dla Euklidesa \emph{tym, co nie ma żadnych części}) i pięć aksjomatów, przytoczonych za książką \emph{O Elementach Euklidesa}:

\begin{enumerate}
	\item Zakłada się, że od każdego punktu do każdego punktu można poprowadzić linię prostą.
	\item I że ograniczoną prostą można ciągle przedłużać po prostej.
	\item I że z każdego środka każdym rozwarciem można zakreślić kolo.
	\item I że wszystkie kąty proste są równe między sobą.
	\item I jeżeli prosta padająca na dwie proste tworzy po jednej stronie kąty wewnętrzne, które w sumie są mniejsze od dwóch prostych, to te proste przedłużone nieograniczenie schodzą się po tej stronie, po której kąty te w sumie są mniejsze od dwóch prostych.
\end{enumerate}

Pojęcia pierwotne i aksjomaty Euklidesa nie są jednak idealne.
Dlatego zamiast nich będziemy używać aksjomatów Hilberta podanych około 1899 roku.

\section{Aksjomaty Hilberta}
Aksjomatyka Hilberta używa trzech pojęć pierwotnych punktu, prostej, płaszczyzny oraz trzech relacji pierwotnych:
\begin{itemize}
	\item leżenia pomiędzy (jedna relacja między trójkami punktów),
	\item zawierania się  w (trzy relacje: między punktami i prostymi; punktami i płaszczyznami; prostymi i płaszczyznami) oraz
	\item przystawania (dwie relacje: między odcinkami; między kątami).
\end{itemize}
Będziemy czasem używać synonimów, takich jak: ,,punkt $A$ leży na prostej $a$'', ,,prosta $a$ przechodzi przez punkt $A$'', ,,prosta $a$ łączy punkty $A$ i $B$''.
Wymienimy najpierw wszystkie aksjomaty, a potem przeanalizujemy ich treść.

\begin{itemize}
	\item \textbf{aksjomaty incydencji}:
\begin{enumerate}
	\item Przez każde dwa punkty przechodzi dokładnie jedna prosta.
	\item Na każdej prostej leżą co najmniej dwa różne punkty.
	\item Pewne trzy punkty nie są współliniowe.
	\item (Pozostałe aksjomaty incydencji dotyczą przestrzeni trójwymiarowej).
\end{enumerate}
\item \textbf{aksjomat Playfaire'a}:
\begin{enumerate}
	\item Dla każdego punktu $A$ i każdej prostej $l$, istnieje co najwyżej jedna prosta równoległa do $l$, zawierająca $A$.
\end{enumerate}
\item \textbf{aksjomaty uporządkowania}: \begin{enumerate}
	\item Jeżeli punkt $B$ leży pomiędzy punktami $A$ i $C$, to leży też pomiędzy punktami $C$ i $A$, a wszystkie trzy leżą na jednej prostej.
	\item Między każdą parą punktów leży trzeci punkt.
	\item Dla każdych trzech punktów na prostej, tylko jeden z nich leży pomiędzy pozostałymi dwoma.
	\item (Pascha) Niech $A, B, C$ będą trzema niewspółliniiowymi punktami, zaś $l$ prostą, która nie przechodzi przez żaden z nich. Jeśli prosta $l$ zawiera punkt $D$ leżący między $A$ i $B$, to musi też zawierać punkt leżący między $A$ i $C$ albo punkt leżący między $B$ i $C$, ale nie obydwa te punkty.
\end{enumerate}
\item \textbf{aksjomaty przystawania} (zapis $\overline{AB} \cong \overline{CD}$ oznacza, że odcinki są przystające): \begin{enumerate}
	\item Niech $\overline{AB}$ będzie odcinkiem, a $r$ półprostą o początku w punkcie $C$. Istnieje dokładnie jeden punkt $D$ leżący na $r$ taki, że $\overline{AB} \cong \overline{CD}$.
	\item Jeśli $\overline{AB} \cong \overline{CD}$ i $\overline{AB} \cong \overline{EF}$, to $\overline{CD} \cong \overline{EF}$. Każdy odcinek przystaje do siebie.
	\item (dodawanie) Dane są trzy punkty $A, B, C$ na prostej takie, że $B$ leży pomiędzy $A$ i $C$; oraz trzy punkty $D, E, F$ na (być może innej) prostej takie, że $E$ leży pomiędzy $D$ i $F$.
	Jeśli $\overline{AB} \cong \overline{DE}$ i $\overline{BC} \cong \overline{EF}$, to $\overline{AC} \cong \overline{DF}$.
	\item (aksjomaty dla kątów)
	\item (aksjomaty dla kątów)
	\item (cecha przystawania bok-kąt-bok)
\end{enumerate}
\end{itemize}

Punkty nazywamy współliniowymi, kiedy istnieje prosta, która przechodzi przez każdy z~nich.
Aksjomat Playfaira został nazwany na cześć szkockiego matematyka, który podał jego treść w podręczniku \emph{Elements of Geometry} z 1795 roku.
% % https://en.wikipedia.org/wiki/Playfair%27s_axiom
\index[persons]{Playfair, John}%
\index{aksjomat!Playfaira}%
Potrzebna jest jeszcze definicja prostych równoległych:

\begin{definition}[równoległość]
	Dwie proste, które pokrywają się albo nie mają żadnych punktów wspólnych, nazywamy równoległymi.
\end{definition}

Czwarty aksjomat uporządkowania znalazł Moritz Pasch \cite{pasch_1882} w 1882 roku.
\index[persons]{Pasch, Moritz}
\index{aksjomat!Pascha}
Aksjomaty uporządkowania pozwalają mówić o odcinkach:

\begin{definition}[odcinek]
\index{odcinek}%
	Niech $A, B$ będą dwoma różnymi punktami.
	Zbiór punktów $A$, $B$ oraz wszystkich punktów leżących pomiędzy $A$ i $B$ nazywamy odcinkiem i oznaczamy $\overline{AB}$.
\end{definition}

\begin{definition}[trójkąt]
\index{trójkąt}%
\index{trójkąt!bok trójkąta}%
\index{trójkąt!wierzchołek trójkąta}%
	Niech $A, B, C$ będą trzema niewspółliniowymi punktami.
	Sumę odcinków $\overline{AB}$, $\overline{BC}$ i $\overline{AC}$ nazywamy trójkątem i oznaczamy $\triangle ABC$.
	Punkty $A, B, C$ są jego wierzchołkami, odcinki $\overline{AB}$, $\overline{BC}$ i $\overline{AC}$ bokami.
\end{definition}

Znając trójkąty, możemy wysłowić aksjomat Pascha inaczej: jeśli prosta $l$ przechodzi przez bok $\overline{AB}$ trójkąta $\triangle ABC$ i nie przechodzi przez wierzchołki $A, B$, to musi przecinać dokładnie jeden z boków $\overline{AC}$, $\overline{BC}$.

\begin{proposition}[rozdzielanie płaszczyzny]
	Niech $l$ będzie prostą.
	Zbiór punktów, które nie leżą na $l$ można podzielić na dwa niepuste podzbiory $S_1, S_2$ takie, że dwa punkty $A, B$ należą do tego samego zbioru ($S_1$ lub $S_2$) wtedy i tylko wtedy, gdy odcinek $\overline{AB}$ nie przecina prostej $l$.
\end{proposition}

Będziemy mówić, że dwa punkty leżą po tej samej stronie (albo po różnych stronach) prostej.

\begin{proof}
	Hartshorne \cite[s. 74--76]{hartshorne_2010}.
\end{proof}

\begin{proposition}[rozdzielanie prostej]
	Niech $l$ będzie prostą przechodzącą przez punkt $A$.
	Zbiór pozostałych punktów prostej $l$ można podzielić na dwa niepuste podzbiory $S_1, S_2$ takie, że dwa punkty $B, C$ należą do tego samego zbioru ($S_1$ lub $S_2$) wtedy i tylko wtedy, gdy punkt $A$ nie leży na odcinku $\overline{BC}$.
\end{proposition}

Znowu, pozwala to mówić o dwóch stronach prostej.

\begin{proof}
	Hartshorne \cite[s. 76--77]{hartshorne_2010}.
\end{proof}

\begin{definition}[półprosta]
	Niech $A, B$ będą dwoma punktami.
	Zbiór, do którego należą punkt $A$ oraz wszystkie punkty prostej $AB$, które leżą po tej samej stronie, co punkt $B$, nazywamy półprostą $\overrightarrow{AB}$ o początku w $A$.
\end{definition}

\begin{definition}[kąt]
	Sumę dwóch półprostych $\overrightarrow{AB}$ i $\overrightarrow{AC}$, które nie leżą na jednej prostej, nazywamy kątem i oznaczamy $\angle BAC$.
	Wnętrzem takiego kąta nazywamy zbiór punktów $D$ takich, że $D$ i $C$ leżą po tej samej stronie prostej $AB$, zaś $D$ i $B$ po tej samej stronie prostej $AC$.
\end{definition}

(W myśl tej definicji, nie istnieje kąt zerowy ani półpełny!).
Część wspólną wnętrz kątów $\angle ABC$, $\angle BCA$ i $\angle CAB$ nazywamy wnętrzem trójkąta $\triangle ABC$.

\begin{proposition}[o kuszy]
	Niech $\angle BAC$ będzie kątem, we wnętrzu którego leży punkt $D$.
	Wtedy półprosta $\overrightarrow{AD}$ przecina odcinek $\overline{BC}$.
\end{proposition}

\begin{proof}
	Hartshorne \cite[s. 77--78]{hartshorne_2010}.
\end{proof}

Trzeci aksjomat przystawania pozwala nam dodawać odcinki: jeśli dane są odcinki $\overline{AB}$ i $\overline{CD}$, zaś $r$ jest półprostą $\overrightarrow{AB}$ z punktem $E$ na sobie takim, że $\overline{CD} \cong \overline{BE}$, to możemy skonstruować sumę $AE = AB + CD$.

(Odejmowanie, porządek...)

\begin{definition}[płaszczyzna Hilberta]
	Zbiór punktów $\Pi$ z wyróżnionymi pewnymi podzbiorami (zwanymi liniami) oraz pojęciami leżenia pomiędzy, przystawania odcinków i przystawania kątów (tak jak opisaliśmy to wyżej), który spełnia wszystkie aksjomaty poza, być może, aksjomatem Pascha, nazywamy płaszczyzną Hilberta.
\end{definition}


Twierdzenie o dwusiecznej % https://en.wikipedia.org/wiki/Angle_bisector_theorem
The angle bisector theorem appears as Proposition 3 of Book VI in Euclid's Elements. 

The exterior angle theorem is Proposition 1.16 in Euclid's Elements, which states that the measure of an exterior angle of a triangle is greater than either of the measures of the remote interior angles. This is a fundamental result in absolute geometry because its proof does not depend upon the parallel postulate. % https://en.wikipedia.org/wiki/Exterior_angle_theorem

Konstrukcja pierwiastka z iloczynu:
The theorem is usually attributed to Euclid (ca. 360–280 BC), who stated it as a corollary to proposition 8 in book VI of his Elements. In proposition 14 of book II Euclid gives a method for squaring a rectangle, which essentially matches the method given here. Euclid however provides a different slightly more complicated proof for the correctness of the construction rather than relying on the geometric mean theorem.
% https://en.wikipedia.org/wiki/Geometric_mean_theorem


Hinge theorem % https://en.wikipedia.org/wiki/Hinge_theorem

twierdzenia geometrii koła:
- % https://en.wikipedia.org/wiki/Thales%27s_theorem
- The inscribed angle theorem states that an angle $\theta$ inscribed in a circle is half of the central angle $2\theta$ that subtends the same arc on the circle. 

% https://en.wikipedia.org/wiki/Intercept_theorem

% https://en.wikipedia.org/wiki/Inscribed_angle#Theorem

% https://en.wikipedia.org/wiki/Intersecting_chords_theorem
% https://en.wikipedia.org/wiki/Intersecting_secants_theorem
% https://en.wikipedia.org/wiki/Tangent%E2%80%93secant_theorem
% https://en.wikipedia.org/wiki/Power_of_a_point#Theorems
twierdzenie o siecznych


\subsection{Elementy, księga I}
Hartshorne analizuje teraz, które stwierdzenia z Elementów Euklidesa są nadal prawdziwe na płaszczyźnie Hilberta.
Nie opiszę tego lepiej, dlatego przedstawiam jedynie podsumowanie.

(I.1), czyli konstrukcja trójkąta równobocznego, nie wynika z aksjomatów płaszczyzny Hilberta (ćwiczenie 39.31).
(I.2) i (I.3) zastąpiliśmy aksjomatem C1, zaś (I.4) aksjomatem C6.
(I.5), że kąty przy podstawie trójkąta równoramiennego są przystające, nie wymaga zmian.
Tezę V w I księdze nazywa się często \emph{pons asinorum}, czyli mostem osłów; jeśli ktoś nie jest w stanie samodzielnie przeprowadzić tego dowodu, to nie może przekroczyć mostu i dalej studiować geometrii.
\index{pons asinorum}%
\index{most osłów|see {pons asinorum}}%
(I.6) to twierdzenie odwrotne do (I.5) i wymaga kosmetycznych zmian, podobnie jak (I.7).

Ale (I.8), czyli cecha przystawania bok-bok-bok musi zostać udowodniona zupełnie inaczej; nowe uzasadnienie zaproponował Hilbert.
\index{cecha przystawania bok-bok-bok}
Zaczynając od (I.9) mamy do czynienia z konstrukcjami cyrklem i linijką, co stanowi pewien problem, bo nie wiemy jeszcze, czy proste zawsze przecinają okręgi.
Hartshorne posiłkuje się słabszym twierdzeniem, że każdy odcinek może być podstawą trójkąta równoramiennego.
To wystarcza do naprawy (I.10) i (I.11), ale nie (I.12), że dowolny punkt można zrzutować prostopadle na prostą, która przez niego nie przechodzi.
Potrzeba znowu całkiem nowego rozumowania.

Tezy (I.13) do (I.21) są w porządku.
Teza (I.22) jest nie do uratowania; nie wiemy, czy dwa okręgi muszą się zawsze przecinać tak, jak oczekuje tego Euklides i istotnie ćwiczenie 16.11 u Hartshorne'a mówi, że w pewnych płaszczyznach Hilberta trójkąty użyte w dowodzie nie istnieją.
Dalej, (I.23) było dowodzone przy użyciu (I.22), ale u nas to jest po prostu aksjomat C4.
Tezy (I.24) do (I.27) i (I.31) znowu są w porządku.

Zatem wszystko, co pisze Euklides, od I.1 do I.28 bez I.1, I.22 można uratować.

Hartshorne 104
Definicja okręgu, środka, promienia

Hartshorne 105
Definicja stycznej


 % aksjomaty Euklidesa

\subsection{Aksjomaty Hilberta}
{\color{red}
Aksjomatyka Hilberta używa trzech pojęć pierwotnych punktu, prostej, płaszczyzny oraz trzech relacji pierwotnych:
\begin{itemize}
	\item leżenia pomiędzy (jedna relacja między trójkami punktów),
	\item zawierania się  w (trzy relacje: między punktami i prostymi; punktami i płaszczyznami; prostymi i płaszczyznami) oraz
	\item przystawania (dwie relacje: między odcinkami; między kątami).
\end{itemize}
Będziemy czasem używać synonimów, takich jak: ,,punkt $A$ leży na prostej $a$'', ,,prosta $a$ przechodzi przez punkt $A$'', ,,prosta $a$ łączy punkty $A$ i $B$''.
Wymienimy najpierw wszystkie aksjomaty, a potem przeanalizujemy ich treść.

\begin{itemize}
	\item \textbf{aksjomaty incydencji}:
\begin{enumerate}
	\item Przez każde dwa punkty przechodzi dokładnie jedna prosta.
	\item Na każdej prostej leżą co najmniej dwa różne punkty.
	\item Pewne trzy punkty nie są współliniowe.
	\item (Pozostałe aksjomaty incydencji dotyczą przestrzeni trójwymiarowej).
\end{enumerate}
\item \textbf{aksjomat Playfaire'a}:
\begin{enumerate}
	\item Dla każdego punktu $A$ i każdej prostej $l$, istnieje co najwyżej jedna prosta równoległa do $l$, zawierająca $A$.
\end{enumerate}
\item \textbf{aksjomaty uporządkowania}: \begin{enumerate}
	\item Jeżeli punkt $B$ leży pomiędzy punktami $A$ i $C$, to leży też pomiędzy punktami $C$ i $A$, a wszystkie trzy leżą na jednej prostej.
	\item Między każdą parą punktów leży trzeci punkt.
	\item Dla każdych trzech punktów na prostej, tylko jeden z nich leży pomiędzy pozostałymi dwoma.
	\item (Pascha) Niech $A, B, C$ będą trzema niewspółliniiowymi punktami, zaś $l$ prostą, która nie przechodzi przez żaden z nich. Jeśli prosta $l$ zawiera punkt $D$ leżący między $A$ i $B$, to musi też zawierać punkt leżący między $A$ i $C$ albo punkt leżący między $B$ i $C$, ale nie obydwa te punkty.
\end{enumerate}
\item \textbf{aksjomaty przystawania} (zapis $\overline{AB} \cong \overline{CD}$ oznacza, że odcinki są przystające): \begin{enumerate}
	\item Niech $\overline{AB}$ będzie odcinkiem, a $r$ półprostą o początku w punkcie $C$. Istnieje dokładnie jeden punkt $D$ leżący na $r$ taki, że $\overline{AB} \cong \overline{CD}$.
	\item Jeśli $\overline{AB} \cong \overline{CD}$ i $\overline{AB} \cong \overline{EF}$, to $\overline{CD} \cong \overline{EF}$. Każdy odcinek przystaje do siebie.
	\item (dodawanie) Dane są trzy punkty $A, B, C$ na prostej takie, że $B$ leży pomiędzy $A$ i $C$; oraz trzy punkty $D, E, F$ na (być może innej) prostej takie, że $E$ leży pomiędzy $D$ i $F$.
	Jeśli $\overline{AB} \cong \overline{DE}$ i $\overline{BC} \cong \overline{EF}$, to $\overline{AC} \cong \overline{DF}$.
	\item (aksjomaty dla kątów)
	\item (aksjomaty dla kątów)
	\item (cecha przystawania bok-kąt-bok)
\end{enumerate}
\end{itemize}

Punkty nazywamy współliniowymi, kiedy istnieje prosta, która przechodzi przez każdy z~nich.
Aksjomat Playfaira został nazwany na cześć szkockiego matematyka, który podał jego treść w podręczniku \emph{Elements of Geometry} z 1795 roku.
% % https://en.wikipedia.org/wiki/Playfair%27s_axiom
\index[persons]{Playfair, John}%
\index{aksjomat!Playfaira}%
Potrzebna jest jeszcze definicja prostych równoległych:

\begin{definition}[równoległość]
	Dwie proste, które pokrywają się albo nie mają żadnych punktów wspólnych, nazywamy równoległymi.
\end{definition}

Czwarty aksjomat uporządkowania znalazł Moritz Pasch \cite{pasch_1882} w 1882 roku.
\index[persons]{Pasch, Moritz}
\index{aksjomat!Pascha}
Aksjomaty uporządkowania pozwalają mówić o odcinkach:

\begin{definition}[odcinek]
\index{odcinek}%
	Niech $A, B$ będą dwoma różnymi punktami.
	Zbiór punktów $A$, $B$ oraz wszystkich punktów leżących pomiędzy $A$ i $B$ nazywamy odcinkiem i oznaczamy $\overline{AB}$.
\end{definition}

\begin{definition}[trójkąt]
\index{trójkąt}%
\index{trójkąt!bok trójkąta}%
\index{trójkąt!wierzchołek trójkąta}%
	Niech $A, B, C$ będą trzema niewspółliniowymi punktami.
	Sumę odcinków $\overline{AB}$, $\overline{BC}$ i $\overline{AC}$ nazywamy trójkątem i oznaczamy $\triangle ABC$.
	Punkty $A, B, C$ są jego wierzchołkami, odcinki $\overline{AB}$, $\overline{BC}$ i $\overline{AC}$ bokami.
\end{definition}

Znając trójkąty, możemy wysłowić aksjomat Pascha inaczej: jeśli prosta $l$ przechodzi przez bok $\overline{AB}$ trójkąta $\triangle ABC$ i nie przechodzi przez wierzchołki $A, B$, to musi przecinać dokładnie jeden z boków $\overline{AC}$, $\overline{BC}$.

\begin{proposition}[rozdzielanie płaszczyzny]
	Niech $l$ będzie prostą.
	Zbiór punktów, które nie leżą na $l$ można podzielić na dwa niepuste podzbiory $S_1, S_2$ takie, że dwa punkty $A, B$ należą do tego samego zbioru ($S_1$ lub $S_2$) wtedy i tylko wtedy, gdy odcinek $\overline{AB}$ nie przecina prostej $l$.
\end{proposition}

Będziemy mówić, że dwa punkty leżą po tej samej stronie (albo po różnych stronach) prostej.

\begin{proof}
	Hartshorne \cite[s. 74--76]{hartshorne_2010}.
\end{proof}

\begin{proposition}[rozdzielanie prostej]
	Niech $l$ będzie prostą przechodzącą przez punkt $A$.
	Zbiór pozostałych punktów prostej $l$ można podzielić na dwa niepuste podzbiory $S_1, S_2$ takie, że dwa punkty $B, C$ należą do tego samego zbioru ($S_1$ lub $S_2$) wtedy i tylko wtedy, gdy punkt $A$ nie leży na odcinku $\overline{BC}$.
\end{proposition}

Znowu, pozwala to mówić o dwóch stronach prostej.

\begin{proof}
	Hartshorne \cite[s. 76--77]{hartshorne_2010}.
\end{proof}

\begin{definition}[półprosta]
	Niech $A, B$ będą dwoma punktami.
	Zbiór, do którego należą punkt $A$ oraz wszystkie punkty prostej $AB$, które leżą po tej samej stronie, co punkt $B$, nazywamy półprostą $\overrightarrow{AB}$ o początku w $A$.
\end{definition}

\begin{definition}[kąt]
	Sumę dwóch półprostych $\overrightarrow{AB}$ i $\overrightarrow{AC}$, które nie leżą na jednej prostej, nazywamy kątem i oznaczamy $\angle BAC$.
	Wnętrzem takiego kąta nazywamy zbiór punktów $D$ takich, że $D$ i $C$ leżą po tej samej stronie prostej $AB$, zaś $D$ i $B$ po tej samej stronie prostej $AC$.
\end{definition}

(W myśl tej definicji, nie istnieje kąt zerowy ani półpełny!).
Część wspólną wnętrz kątów $\angle ABC$, $\angle BCA$ i $\angle CAB$ nazywamy wnętrzem trójkąta $\triangle ABC$.

\begin{proposition}[o kuszy]
	Niech $\angle BAC$ będzie kątem, we wnętrzu którego leży punkt $D$.
	Wtedy półprosta $\overrightarrow{AD}$ przecina odcinek $\overline{BC}$.
\end{proposition}

\begin{proof}
	Hartshorne \cite[s. 77--78]{hartshorne_2010}.
\end{proof}

Trzeci aksjomat przystawania pozwala nam dodawać odcinki: jeśli dane są odcinki $\overline{AB}$ i $\overline{CD}$, zaś $r$ jest półprostą $\overrightarrow{AB}$ z punktem $E$ na sobie takim, że $\overline{CD} \cong \overline{BE}$, to możemy skonstruować sumę $AE = AB + CD$.

(Odejmowanie, porządek...)

\begin{definition}[płaszczyzna Hilberta]
	Zbiór punktów $\Pi$ z wyróżnionymi pewnymi podzbiorami (zwanymi liniami) oraz pojęciami leżenia pomiędzy, przystawania odcinków i przystawania kątów (tak jak opisaliśmy to wyżej), który spełnia wszystkie aksjomaty poza, być może, aksjomatem Pascha, nazywamy płaszczyzną Hilberta.
\end{definition}


Twierdzenie o dwusiecznej % https://en.wikipedia.org/wiki/Angle_bisector_theorem
The angle bisector theorem appears as Proposition 3 of Book VI in Euclid's Elements. 

The exterior angle theorem is Proposition 1.16 in Euclid's Elements, which states that the measure of an exterior angle of a triangle is greater than either of the measures of the remote interior angles. This is a fundamental result in absolute geometry because its proof does not depend upon the parallel postulate. % https://en.wikipedia.org/wiki/Exterior_angle_theorem

Konstrukcja pierwiastka z iloczynu:
The theorem is usually attributed to Euclid (ca. 360–280 BC), who stated it as a corollary to proposition 8 in book VI of his Elements. In proposition 14 of book II Euclid gives a method for squaring a rectangle, which essentially matches the method given here. Euclid however provides a different slightly more complicated proof for the correctness of the construction rather than relying on the geometric mean theorem.
% https://en.wikipedia.org/wiki/Geometric_mean_theorem


Hinge theorem % https://en.wikipedia.org/wiki/Hinge_theorem

twierdzenia geometrii koła:
- % https://en.wikipedia.org/wiki/Thales%27s_theorem
- The inscribed angle theorem states that an angle $\theta$ inscribed in a circle is half of the central angle $2\theta$ that subtends the same arc on the circle. 

% https://en.wikipedia.org/wiki/Intercept_theorem

% https://en.wikipedia.org/wiki/Inscribed_angle#Theorem

% https://en.wikipedia.org/wiki/Intersecting_chords_theorem
% https://en.wikipedia.org/wiki/Intersecting_secants_theorem
% https://en.wikipedia.org/wiki/Tangent%E2%80%93secant_theorem
% https://en.wikipedia.org/wiki/Power_of_a_point#Theorems
twierdzenie o siecznych


\subsection{Elementy, księga I}
Hartshorne analizuje teraz, które stwierdzenia z Elementów Euklidesa są nadal prawdziwe na płaszczyźnie Hilberta.
Nie opiszę tego lepiej, dlatego przedstawiam jedynie podsumowanie.

(I.1), czyli konstrukcja trójkąta równobocznego, nie wynika z aksjomatów płaszczyzny Hilberta (ćwiczenie 39.31).
(I.2) i (I.3) zastąpiliśmy aksjomatem C1, zaś (I.4) aksjomatem C6.
(I.5), że kąty przy podstawie trójkąta równoramiennego są przystające, nie wymaga zmian.
Tezę V w I księdze nazywa się często \emph{pons asinorum}, czyli mostem osłów; jeśli ktoś nie jest w stanie samodzielnie przeprowadzić tego dowodu, to nie może przekroczyć mostu i dalej studiować geometrii.
\index{pons asinorum}%
\index{most osłów|see {pons asinorum}}%
(I.6) to twierdzenie odwrotne do (I.5) i wymaga kosmetycznych zmian, podobnie jak (I.7).

Ale (I.8), czyli cecha przystawania bok-bok-bok musi zostać udowodniona zupełnie inaczej; nowe uzasadnienie zaproponował Hilbert.
\index{cecha przystawania bok-bok-bok}
Zaczynając od (I.9) mamy do czynienia z konstrukcjami cyrklem i linijką, co stanowi pewien problem, bo nie wiemy jeszcze, czy proste zawsze przecinają okręgi.
Hartshorne posiłkuje się słabszym twierdzeniem, że każdy odcinek może być podstawą trójkąta równoramiennego.
To wystarcza do naprawy (I.10) i (I.11), ale nie (I.12), że dowolny punkt można zrzutować prostopadle na prostą, która przez niego nie przechodzi.
Potrzeba znowu całkiem nowego rozumowania.

Tezy (I.13) do (I.21) są w porządku.
Teza (I.22) jest nie do uratowania; nie wiemy, czy dwa okręgi muszą się zawsze przecinać tak, jak oczekuje tego Euklides i istotnie ćwiczenie 16.11 u Hartshorne'a mówi, że w pewnych płaszczyznach Hilberta trójkąty użyte w dowodzie nie istnieją.
Dalej, (I.23) było dowodzone przy użyciu (I.22), ale u nas to jest po prostu aksjomat C4.
Tezy (I.24) do (I.27) i (I.31) znowu są w porządku.

Zatem wszystko, co pisze Euklides, od I.1 do I.28 bez I.1, I.22 można uratować.

Hartshorne 104
Definicja okręgu, środka, promienia

Hartshorne 105
Definicja stycznej


} % aksjomaty Hilberta

\section{W przygotowaniu -- Geometria -- Uniwersytet Warszawski}
\subsection{Geometria I}
\subsubsection{X}
1. Przystawanie figur na płaszczyźnie. Cechy przystawania trójkątów. Własności równoległoboków. Problem Fagnano i problem Fermata. Kąty w okręgu: wpisane, kąty środkowe i kąty dopisane. Twierdzenia o kątach wpisanych, kątach środkowych i kątach dopisanych do okręgu. Kątowe warunki na istnienie okręgu przechodzącego przez cztery punkty. Zastosowanie: okrąg dziewięciu punktów, twierdzenie o prostej Simsona. Styczna do okręgu, okrąg wpisany w kąt. Okrąg wpisany w trójkąt, okręgi dopisane do trójkąta. Warunki istnienia okręgu stycznego do czterech prostych.

\subsubsection{X}
2. Stosunek podziału wektora. Twierdzenie Talesa, twierdzenie odwrotne i jego zastosowania. Pole. Pola wybranych figur, twierdzenie Pitagorasa. Pole zorientowane. Twierdzenie Newtona: środek okręgu wpisanego w czworokąt i środki przekątnych tego czworokąta są współliniowe. Twierdzenie Gaussa: środki przekątnych czworokąta zupełnego są współliniowe. Definicja jednokładności, podobieństwo figur. Cechy podobieństwa trójkątów. Stosunek pól figur podobnych. Iloczynowe warunki istnienia okręgu przechodzącego przez cztery punkty. Pojęcie potęgi punktu
względem okręgu. Twierdzenie Ptolemeusza.

\subsubsection{X}
3. Wielkości miarowe w trójkącie: wzór Herona, wzory na promienie okręgów wpisanych, dopisanych. Twierdzenie o dwusiecznej i okrąg Apoloniusza. Twierdzenie Cevy (wraz z trygonometryczną wersją), przykłady punktów szczególnych trójkąta: punkt Nagela, punkt Gergonne'a, punkt Lemoine'a. Punkty izogonalnie sprzężone w trójkącie. Twierdzenie Menelausa.

\subsubsection{Jednokładność}
Jednokładność.

Konstrukcja obrazu jednokładnego punktu, okręgu, prostej.

Środek jednokładności dwóch trójkątów.

Środki jednokładności dwóch okręgów.

Prosta Eulera w trójkącie (środek okręgu opisanego, środek ciężkości, ortocentrum).

Zastosowanie: Twierdzenie Pascala.

Twierdzenie Kirkmana: jeśli część wspólna dwóch trójkątów wpisanych w okrąg jest sześciokątem wypukłym, to główne przekątne tego sześciokąta przecinają się w jednym punkcie.

Grupa dylatacji na płaszczyźnie.

Twierdzenia o składaniu jednokładności i przesunięć, twierdzenie o środkach jednokładności trzech okręgów.

\subsubsection{Izometrie}
5. Grupa izometrii na płaszczyźnie. Konstrukcja obrazu punktu, okręgu, prostej przy translacji, obrocie i symetrii osiowej. Złożenie dwóch i złożenie trzech symetrii osiowych. Twierdzenia o składaniu izometrii. Klasyfikacja izometrii na płaszczyźnie. Izometrie parzyste i izometrie nieparzyste. Twierdzenie o redukcji. Twierdzenie Napoleona: środki ciężkości trójkątów równobocznych zbudowanych na bokach dowolnego trójkąta są wierzchołkami trójkąta równobocznego.

\subsubsection{Podobieństwa}
6. Grupa podobieństw płaszczyzny. Podobieństwa spiralne i odbicia dylatacyjne. Klasyfikacja podobieństw płaszczyzny.


\subsection{Geometria II UW}
1. Potęga punktu względem okręgu, oś potęgowa dwóch okręgów, środek potęgowy trzech okręgów, twierdzenie Brianchona, konstrukcja stycznej do okręgu samą linijką, okręgi współpękowe, twierdzenie Gaussa-Bodenmillera, twierdzenie o motylku, formuła Eulera na odległość między środkami okręgu opisanego i wpisanego (dla trójkąta), twierdzenie Ponceleta dla trójkąta.

2. Obrazy inwersyjne okręgów i prostych, konforemność inwersji, okręgi stałe inwersji, okręgi prostopadłe, zmiana odległości przy inwersji, twierdzenie Ptolemeusza, zmiana promienia okręgu przy inwersji, łańcuchy Steinera, formuła Kartezjusza, formuła Fussa dla czworokątów, twierdzenie Feuerbacha.

3. Ogniska elipsy i hiperboli, ognisko, kierownica i mimośród stożkowych, asymptoty hiperboli, konstrukcja stycznej do stożkowej, rzuty ustalonego ogniska na styczne, własności izogonalne stożkowych, równania kanoniczne stożkowych, elipsa jako przekrój walca. Ognisko, kierownica i mimośród stożkowej na przekroju stożka. Przekroje stożków ze sferami wpisanymi. Równanie ogólne stożkowej w układzie współrzędnych, duży i mały wyznacznik. Równania stożkowych we współrzędnych biegunowych.

4. Grupa przekształceń afinicznych od strony geometrycznej: powinowactwa osiowe, rozkład przekształcenia afinicznego na podobieństwo i powinowactwo osiowe, kierunki główne przekształcenia afinicznego, niezmienniczość stosunku pól przy przekształceniu afinicznym, obraz okręgu przy przekształceniu afinicznym
Literatura: 	

1. E. H. Askwith, D.D. A Course of Pure Geometry, Cambridge 1917.
2. H. Fukagawa, D. Pedoe, Japanese temple geometry problems. Sangaku, Charles Babbage Research Centre, Winnipeg 1989.
3. R. A. Johnson, Advanced Euclidean geometry: An elementary treatise on the geometry of the triangle and the circle, Dover Publications, Inc., New York 1960.
4. W. Pompe, Geometria kół, Wydawnictwo Szkolne OMEGA, Kraków 2019.
5. V. Prasolov, Zadaczi po planimietrii. Tom I-II (ros.), Nauka, Moskwa 1991

\subsection{Geometria III}
Geometria rzutowa: ujęcie od strony geometrycznej. Płaszczyzna rzutowa (rzeczywista), przekształcenia rzutowe prostych, pęków, stożkowych, pęków stycznych do stożkowych.
Twierdzenia Desarguesa, Pappusa, Pascala, Brianchona.
Dualność: biegun i biegunowa względem okręgu i stożkowych. Sprzężenie biegunowe. Inwolucje rzutowe, twierdzenia inwolucyjne. Pęki okręgów i stożkowych jako generatory inwolucji. Twierdzenie Ponceleta. Stożkowe w ujęciu rzutowym, twierdzenia Steinera i Braikenridge'a-Maclaurina. Rzutowe określenie ogniska i kierownicy stożkowych. Punkty urojone przecięcia prostej ze stożkową w ujęciu czysto geometrycznym.



\section{Trygonometria}
\subsection{Trygonometria.} Lorem ipsum dolor sit amet, consectetur adipiscing elit, sed do eiusmod tempor incididunt ut labore et dolore magna aliqua. Ut enim ad minim veniam, quis nostrud exercitation ullamco laboris nisi ut aliquip ex ea commodo consequat. Duis aute irure dolor in reprehenderit in voluptate velit esse cillum dolore eu fugiat nulla pariatur. Excepteur sint occaecat cupidatat non proident, sunt in culpa qui officia deserunt mollit anim id est laborum.
\subsubsection{Prawo sinusów}
Lorem ipsum dolor sit amet, consectetur adipiscing elit, sed do eiusmod tempor incididunt ut labore et dolore magna aliqua. Ut enim ad minim veniam, quis nostrud exercitation ullamco laboris nisi ut aliquip ex ea commodo consequat. Duis aute irure dolor in reprehenderit in voluptate velit esse cillum dolore eu fugiat nulla pariatur. Excepteur sint occaecat cupidatat non proident, sunt in culpa qui officia deserunt mollit anim id est laborum.
$$\frac{a}{\sin \alpha} = \frac{b}{\sin \beta} = \frac{c}{\sin \gamma} = 2R$$
% https://en.wikipedia.org/wiki/Law_of_sines

\subsubsection{Prawo cosinusów}
Lorem ipsum dolor sit amet, consectetur adipiscing elit, sed do eiusmod tempor incididunt ut labore et dolore magna aliqua. Ut enim ad minim veniam, quis nostrud exercitation ullamco laboris nisi ut aliquip ex ea commodo consequat. Duis aute irure dolor in reprehenderit in voluptate velit esse cillum dolore eu fugiat nulla pariatur. Excepteur sint occaecat cupidatat non proident, sunt in culpa qui officia deserunt mollit anim id est laborum.
$$c^2 = a^2 + b^2 - 2ab \cos \gamma$$
% https://en.wikipedia.org/wiki/Law_of_cosines

\textbf{Twierdzenie Stewarta}

\textbf{Wzór Brahmagupty}

\textbf{Twierdzenie Urquharta}

\textbf{Punkt i kąt Crelle'a-Brocarda}

\textbf{Twierdzenie o siódmym okręgu}

\textbf{Twierdzenie Caseya}

\textbf{Twierdzenie Taylora, okrąg, sześciokąt}

\textbf{Twierdzenie Eulera $1/4R^2$}

% https://en.wikipedia.org/wiki/Law_of_tangents

\subsubsection{Rozwiązywanie trójkątów}
Wzór Mollweide'a.
\index{wzór!Mollweide'a}%
Problem Hansena
\index{problem!Hansena}%
Problem Snelliusa-Pothenota.
\index{problem!Snelliusa-Pothenota}%
% https://en.wikipedia.org/wiki/Mollweide%27s_formula
% https://en.wikipedia.org/wiki/Snellius%E2%80%93Pothenot_problem
% https://en.wikipedia.org/wiki/Hansen%27s_problem

\section{Konstrukcje geometryczne}
Konstrukcje od \ref{delta_2024_12_start} do \ref{delta_2024_12_end} opisane są w czasopiśmie Delta, w numerze grudniowym z 2024 roku.
\todofoot{Dopisać cytowanie w formacie BibTeX}

\begin{geoconstruction}
    \label{delta_2024_12_start}
    Znając pięć punktów okręgu $\omega$, skonstruować styczną do $\omega$ w jednym z tych punktów.
\end{geoconstruction}

\begin{geoconstruction}
    Znając pięć punktów okręgu $\omega$, dla danej prostej $l$ przechodzącej przez jeden z nich wyznaczyć drugi punkt przecięcia $l$ i $\omega$.
\end{geoconstruction}

\begin{geoconstruction}
    Skonstruować środek jednego z dwóch okręgów mających dwa punkty wspólne.
\end{geoconstruction}

\begin{geoconstruction}
    \label{delta_2024_12_end}
    Skonstruować środek przynajmniej jednego z trzech okręgów nienależących do jednego pęku.
\end{geoconstruction}


Konstruowalna => stopień Q(x) nad Q to potęga 2, ale nie w drugą stronę.
Podwojenie sześcianu.
Trysekcja kąta.
<=: Hartshorne, papierowa strona 245.
17-kąt

\section{Stereometria}
\subsection{Pięć wielościanów}
Hartshorne: rozdział 8

\subsection{Cauchy's rigidity theorem}
Hartshorne: section 45

\subsection{Siamese dodecahedron}
John solid?
% https://en.wikipedia.org/wiki/Johnson_solid




% IMO problems
1959/4.
Construct a right triangle with given hypotenuse c such that the median
drawn to the hypotenuse is the geometric mean of the two legs of the triangle.

1959/5.
An arbitrary point M is selected in the interior of the segment AB. The
squares AM CD and M BEF are constructed on the same side of AB, with
the segments AM and M B as their respective bases. The circles circum-
scribed about these squares, with centers P and Q, intersect at M and also
at another point N. Let N ′ denote the point of intersection of the straight
lines AF and BC.
(a) Prove that the points N and N ′ coincide.
(b) Prove that the straight lines M N pass through a fixed point S indepen-
dent of the choice of M.
(c) Find the locus of the midpoints of the segments P Q as M varies between
A and B.
1959/6.
Two planes, P and Q, intersect along the line p. The point A is given in the
plane P, and the point C in the plane Q; neither of these points lies on the
straight line p. Construct an isosceles trapezoid ABCD (with AB parallel to
CD) in which a circle can be inscribed, and with vertices B and D lying in
the planes P and Q respectively.

% https://www.imo-official.org/problems.aspx


