%

\chapter{Geometrie nieeuklidesowe}
Tekst sekcji geometrie nieeuklidesowe.
\begin{enumerate}
	\item Aksjomat Arystotelesa. \index{aksjomat!Arystotelesa}
	\item Lemat Proklusa: prosta nie może przecinać tylko jednej z dwóch prostych równoległych? \index{lemat!Proklusa}
	\item Aksjomat Claviusa. \index{aksjomat!Claviusa}
	\item Aksjomat Clairauta. \index{aksjomat!Clairauta}
	\item Aksjomat Simsona. \index{aksjomat!Simsona}
	\item Aksjomat Playfaire'a. \index{aksjomat!Playfaire'a}
	Aksjomat Playfaira został nazwany na cześć szkockiego matematyka, który podał jego treść w podręczniku \emph{Elements of Geometry} z 1795 roku.
% % https://en.wikipedia.org/wiki/Playfair%27s_axiom
\index[persons]{Playfair, John}%
	\item Aksjomat Wallisa. \index{aksjomat!Wallisa}
	\item Aksjomat Bolyi. \index{aksjomat!Bolyi}
	\item Czworokąt Saccheriego. % + https://en.wikipedia.org/wiki/Lambert_quadrilateral
	\index{czworokąt!Saccheriego}
	Giovanni Girolamo Saccheri był uczniem Tommasa Cevy, brata Giovanniego, autora twierdzenia Cevy.
	\item Aksjomat Legendre'a. \index{aksjomat!Legendre'a}
	\item Model Poincarego. \index{model Poincarego}
	\item Geometria hiperboliczna. \index{geometria!hiperboliczna}
\end{enumerate}

O geometrii hiperbolicznej wspomina się w numerach 2024/lipiec Delty.

%