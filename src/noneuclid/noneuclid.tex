%

\chapter{Geometrie nieeuklidesowe}
Tekst sekcji geometrie nieeuklidesowe.

\begin{proposition}[aksjomat Arystotelesa]
	Aksjomat Arystotelesa.
\index{aksjomat!Arystotelesa}%
\end{proposition}

\begin{proposition}[lemat Proklusa]
	Prosta nie może przecinać tylko jednej z dwóch prostych równoległych.
\index{lemat!Proklusa}%
\end{proposition}

\begin{proposition}[aksjomat Claviusa]
	Aksjomat Claviusa.
\index{aksjomat!Claviusa}%
\end{proposition}

\begin{proposition}[aksjomat Clairauta]
	Aksjomat Clairauta.
\index{aksjomat!Clairauta}%
\end{proposition}

\begin{proposition}[aksjomat Simsona]
	Aksjomat Simsona.
\index{aksjomat!Simsona}%
\end{proposition}

\begin{proposition}[aksjomat Playfaire'a]
	Aksjomat Playfaire'a.
\index{aksjomat!Playfaire'a}%
\end{proposition}

Aksjomat Playfaira został nazwany na cześć szkockiego matematyka, który podał jego treść w podręczniku \emph{Elements of Geometry} z 1795 roku.
% % https://en.wikipedia.org/wiki/Playfair%27s_axiom
\index[persons]{Playfair, John}%

\begin{proposition}[aksjomat Wallisa]
	Aksjomat Wallisa.
\index{aksjomat!Wallisa}%
\end{proposition}

\begin{proposition}[aksjomat Bolyi]
	Aksjomat Bolyi.
\index{aksjomat!Bolyi}%
\end{proposition}

\begin{definition}[czworokąt Saccheriego]
	Czworokąt Saccheriego.
% TODO: https://en.wikipedia.org/wiki/Lambert_quadrilateral
\index{czworokąt!Saccheriego}
\end{definition}

Giovanni Girolamo Saccheri był uczniem Tommasa Cevy, brata Giovanniego, autora twierdzenia Cevy.

\begin{proposition}[Aksjomat Legendre'a]
	Aksjomat Legendre'a.
\index{aksjomat!Legendre'a}%
\end{proposition}

\begin{proposition}[model Poincarego]
	Model Poincarego.
\index{model Poincarego}%
\end{proposition}

\begin{definition}[geometria hiperboliczna]
	Geometria hiperboliczna.
\index{geometria!hiperboliczna}
\end{definition}

O geometrii hiperbolicznej wspomina się w $\Delta_{24}^7$.

\begin{proposition}
	Jeśli geometria euklidesowa (na płaszczyźnie) jest niesprzeczna, to geometria Łobaczewskiego (na płaszczyźnie) też jest niesprzeczna.
\end{proposition}

Eves \cite[s. 347-353]{eves1_1972}.

%