%

Tekst sekcji geometrie nieeuklidesowe.

\begin{proposition}[aksjomat Arystotelesa]
	Aksjomat Arystotelesa.
\index{aksjomat!Arystotelesa}%
\end{proposition}

\begin{proposition}[lemat Proklusa]
	Prosta nie może przecinać tylko jednej z dwóch prostych równoległych.
\index{lemat!Proklusa}%
\end{proposition}

\begin{proposition}[aksjomat Claviusa]
	Aksjomat Claviusa.
\index{aksjomat!Claviusa}%
\end{proposition}

\begin{proposition}[aksjomat Clairauta]
	Aksjomat Clairauta.
\index{aksjomat!Clairauta}%
\end{proposition}

\begin{proposition}[aksjomat Simsona]
	Aksjomat Simsona.
\index{aksjomat!Simsona}%
\end{proposition}

\begin{proposition}[aksjomat Playfaire'a]
	Aksjomat Playfaire'a.
\index{aksjomat!Playfaire'a}%
\end{proposition}

Aksjomat Playfaira został nazwany na cześć szkockiego matematyka, który podał jego treść w podręczniku \emph{Elements of Geometry} z 1795 roku.
% % https://en.wikipedia.org/wiki/Playfair%27s_axiom
\index[persons]{Playfair, John}%

\begin{proposition}[aksjomat Wallisa]
	Aksjomat Wallisa.
\index{aksjomat!Wallisa}%
\end{proposition}

\begin{proposition}[aksjomat Bolyi]
	Aksjomat Bolyi.
\index{aksjomat!Bolyi}%
\end{proposition}

\begin{definition}[czworokąt Saccheriego]
	Czworokąt Saccheriego.
% TODO: https://en.wikipedia.org/wiki/Lambert_quadrilateral
\index{czworokąt!Saccheriego}
\end{definition}

Giovanni Girolamo Saccheri był uczniem Tommasa Cevy, brata Giovanniego, autora twierdzenia Cevy.

\begin{proposition}[aksjomat Legendre'a]
	Aksjomat Legendre'a.
\index{aksjomat!Legendre'a}%
\end{proposition}

\begin{definition}[model Poincarego]
	Model Poincarego.
\index{model Poincarego}%
\end{definition}

\begin{definition}[geometria hiperboliczna]
	Geometria hiperboliczna.
\index{geometria!hiperboliczna}
\end{definition}

O geometrii hiperbolicznej wspomina się w $\Delta_{24}^7$.

\begin{proposition}
	Jeśli geometria euklidesowa (na płaszczyźnie) jest niesprzeczna, to geometria Łobaczewskiego (na płaszczyźnie) też jest niesprzeczna.
\end{proposition}

Eves \cite[s. 347-353]{eves1_1972}.

W 1733 roku Giovanni Girolamo Saccheri podjął próbę zbadania, jak wyglądałaby geometria, gdyby piąty postulat Euklidesa okazał się fałszywy. Jego prace zapoczątkowały systematyczne rozważania nad możliwymi alternatywami dla geometrii euklidesowej.
Następnie w 1829 roku János Bolyai, Carl Friedrich Gauss oraz Nikołaj Łobaczewski -- niezależnie od siebie -- stworzyli geometrię hiperboliczną, czyli pierwszą w pełni rozwiniętą geometrię nieeuklidesową, w której piąty postulat Euklidesa nie obowiązuje.
Kolejnym ważnym krokiem było osiągnięcie Felixa Kleina z 1870 roku. Klein skonstruował wersję geometrii analitycznej dla geometrii Łobaczewskiego, wykazując jej wewnętrzną niesprzeczność oraz niezależność logiczną od piątego postulatu Euklidesa.
% model Kleida - Delta 2012-06

% Sferyczna osobno?
% https://en.wikipedia.org/wiki/Lexell's_theorem
\begin{proposition}[twierdzenie Lexella]
	Wierzchołki trójkątów sferycznych o ustalonej podstawie $AB$ i polu powierzchni leżą na okręgu małym (powstałym przez przecięcie sfery płaszczyzną) przechodzącym przez punkty antypodalne do wierzchołków $A$, $B$ z podstawy.
\end{proposition}
% https://en.wikipedia.org/wiki/Spherical_trigonometry => Girard

Anders Johan Lexell napisze około 1777 roku pracę na temat tego twierdzenia, razem z dwoma dowodami: trygonometrycznym i geometrycznym.
Inni podadzą inne dowody:
Leonhard Euler w 1778,
Adrien-Marie Legendre w 1800,
Jakob Steiner w 1827,
Carl Friedrich Gauss w 1841,
Paul Serret w 1855,
Joseph-Émile Barbier w 1864.

% https://www.deltami.edu.pl/2012/06/w-rozumowaniach-byl-blad/

%