%

\chapter{Geometrie nieeuklidesowe}
Tekst sekcji geometrie nieeuklidesowe.
\begin{enumerate}
	\item Aksjomat Arystotelesa.
	\item Lemat Proklusa: prosta nie może przecinać tylko jednej z dwóch prostych równoległych?
	\item Aksjomat Claviusa.
	\item Aksjomat Clairauta.
	\item Aksjomat Simsona.
	\item Aksjomat Playfaire'a.
	Aksjomat Playfaira został nazwany na cześć szkockiego matematyka, który podał jego treść w podręczniku \emph{Elements of Geometry} z 1795 roku.
% % https://en.wikipedia.org/wiki/Playfair%27s_axiom
\index[persons]{Playfair, John}%
\index{aksjomat!Playfaira}%
	\item Aksjomat Wallisa.
	\item Aksjomat Bolyi.
	\item Czworokąt Saccheriego. % + https://en.wikipedia.org/wiki/Lambert_quadrilateral
	\item Aksjomat Legendre'a.
	\item Model Poincarego.
	\item Geometria hiperboliczna.
\end{enumerate}

%