\subsection{Geometria pretalesowska}
\subsubsection{Cechy przystawania}
Cechy przystawania

\subsubsection{Kąty wierzchołkowe, naprzemianległe}
Kąty wierzchołkowe, naprzemianległe

%

\index{pons asinorum|(}

\subsubsection{Pons asinorum}
\index{most osłów patrz pons asinorum}
Most osłów (łacińskie \emph{,,pons asinorum''}) to tradycyjna nazwa dowodu twierdzenia, że kąty przy podstawie trójkąta równoramiennego są równe.
Podał go Euklides jako teza V w księdze I Elementów.
Mawiało się, że ci, którzy nie są w stanie samodzielnie przeprowadzić tego dedukcyjnego dowodu opartego na własnościach trójkątów przystających, nie może przekroczyć mostu i studiować dalej geometrii.

Bardziej przyziemnie Coxter \cite[s. 6-9]{coxeter_1991} zauważa, że rysunek wykonany przez Euklidesa przypomina most.
Wśród konsekwencji wymienia kilka wyników z Elementów: III.3, III.20, III.21, III.22, III.32, VI.2, VI.4, a potem III.35, III.36, VI.19, co prowadzi do dowodu twierdzenia Pitagorasa, czyli I.47. % TODO: sprawdzić, czy numeracja moja i Coxetera jest taka sama.
\index{twierdzenie!Pitagorasa}%
Coxeter podaje w formie ćwiczeń nierówność Erdős-Mordella (u nas podsekcja \ref{subsection_erdos_mordell}) oraz twierdzenie Steinera-Lehmusa (twierdzenie \ref{theorem_steiner_lehmus}).
% TODO: https://www.deltami.edu.pl/1990/08/elementarny-dowod-nierownosci-erdosa-mordella/
\todofoot{Przeczytać artykuł z Delty 1990, elementarny-dowod-nierownosci-erdosa-mordella}

Pierwsze dowody tego faktu podali jeszcze Euklides, komentujący jego prace Proklos zwany Diadochem oraz Pappus z Aleksandrii.
Współcześnie podaje się krótkie uzasadnienie w oparciu o dwusieczną kąta, ale Euklides nie mógł tak uczynić, ponieważ definiuje ją dopiero cztery tezy później w swoich Elementach.

O moście osłów piszą Coxeter 

\index{pons asinorum|)}

%

\subsubsection{Równoległobok (problem Fagnano i Fermata?)}
Równoległobok (problem Fagnano i Fermata?)

\subsubsection{Symetralna, okrąg opisany na trójkącie.}
Symetralna, okrąg opisany na trójkącie.

\subsubsection{Geometria koła i kątów, twierdzenie Apolloniusza (s. 22)}
Geometria koła i kątów, twierdzenie Apolloniusza (s. 22)

\subsubsection{Okrąg opisany na czworokącie}
Okrąg opisany na czworokącie.
Iloczynowe warunki istnienia okręgu przechodzącego przez cztery punkty.

Twierdzneie o prostej Wallace'a-Simsona.
Na UW: Zastosowanie: okrąg dziewięciu punktów, twierdzenie o prostej Simsona.

\subsubsection{Kąty w okręgu}
Kąty w okręgu: wpisane, kąty środkowe i kąty dopisane.
Twierdzenia o kątach wpisanych, kątach środkowych i kątach dopisanych do okręgu.
Kątowe warunki na istnienie okręgu przechodzącego przez cztery punkty.
Styczna do okręgu, okrąg wpisany w kąt.
Okrąg wpisany w trójkąt, okręgi dopisane do trójkąta.
Warunki istnienia okręgu stycznego do czterech prostych.

\subsubsection{Twierdzenie Miquela}
Twierdzenie Miquela

\subsubsection{Okrąg wpisany w czworokąt}
Okrąg wpisany w czworokąt

\subsubsection{Twierdzenie Pitagorasa}
Twierdzenie Pitagorasa