%

\section{Podobieństwo}
\subsection{Jednokładność}
Podobieństwo figur, trójkątów (cechy), stosunek pól figur podobnych.

%

Guzicki-3

\begin{theorem}[Talesa]
    Jeśli ramiona kąta płaskiego przetnie się 2 równoległymi prostymi:
    \begin{center}
        \begin{tikzpicture}
            \tkzDefPoint(0, 0.5){O}
            \tkzDefPoint(1.5, 0){A}
            \tkzDefPoint(2, 1){Ap}
            \tkzDefPointBy[homothety=center O ratio 1.618](A) \tkzGetPoint{B}
            \tkzDefLine[parallel=through B](A,Ap) \tkzGetPoint{Bp}
            \tkzInterLL(O,Ap)(B,Bp) \tkzGetPoint{Bpp}
            \tkzDrawPoints[fill=gray,opacity=.9](O,A,B,Ap,Bpp)
            \tkzLabelPoint[above](O){$O$}
            \tkzLabelPoint[below](A){$A$}
            \tkzLabelPoint[below](B){$A'$}
            \tkzLabelPoint[above left](Bpp){$B'$}
            \tkzLabelPoint[above left](Ap){$B$}
            \tkzDrawLine[thick](O,B)
            \tkzDrawLine[thick](O,Bpp)
            \tkzDrawLine[color=blue, thick](A,Ap)
            \tkzDrawLine[color=blue, thick](B,Bpp)
        \end{tikzpicture}
        \end{center}
    to długości odcinków wyznaczonych przez te proste na jednym z ramion kąta są proporcjonalne do długości odpowiednich odcinków na drugim ramieniu kąta, a zatem
    \begin{equation}
        \label{thales_ratio}
        \frac{|OA|}{|OA'|} = \frac{|OB|}{|OB'|} = \frac{|AB|}{|A'B'|}.
    \end{equation}
\end{theorem}
% TODO: https://en.wikipedia.org/wiki/Thales's_theorem

Tradycja przypisuje jego sformułowanie Talesowi z Miletu, chociaż znane było starożytnym Babilończykom i Egipcjanom.
\index[persons]{Tales z Miletu}%
% Pierwszy znany dowód pojawia się w Elementach Euklidesa.
Najstarszy zachowany dowód twierdzenia Talesa zamieszczony jest w VI. księdze Elementów Euklidesa. 
% https://en.wikipedia.org/wiki/Intercept_theorem#Claim_3

Piszą o nim Neugebauer, Bogdańska \cite[s. 48-56]{neugebauer_2018}.
Po angielsku znane jest jako \emph{Thales's theorem}, \emph{intercept theorem}, \emph{basic proportionality theorem} albo \emph{side splitter theorem}.

Prawdziwe jest również twierdzenie odwrotne:

\begin{proposition}[twierdzenie odwrotne do tw. Talesa]
    Jeżeli pewna prosta przecina boki $OA'$, $OB'$ trójkąta $OA'B'$ w różnych punktach $A$ i $B$ odpowiednio, a przy tym zachodzi równość \ref{thales_ratio}, to prosta ta jest równoległa do prostej $A'B'$.
\end{proposition}

Prostym wnioskiem z twierdzenia Talesa jest fakt \ref{hartshorne_52}, znajduje on zastosowanie w dowodzie:
% Neugebauer s. 52

\begin{theorem}[Varignona]
    Czworokąt $PQRS$, którego wierzchołki leżą na środkach boków $AB$, $BC$, $CD$, $DA$ czworokąta $ABCD$, jest równoległobokiem.
    Jego pole jest równe połowie pola czworokąta $ABCD$. % Neugebauer s. 61
\end{theorem}

W szczególności, czworokąt $ABCD$ nie musi być wypukły\footnote{Może być nawet ,,motylkiem'', to znaczy łamaną zamkniętą o czterech bokach, która ma samoprzecięcia.}.
Twierdzenie zostało nazwane na cześć Pierre'a Varignona pośmiertnie w 1731 roku.
\index[persons]{Varignon, Pierre}%
Co więcej,

\begin{proposition}
    Równoległobok Varignona jest rombem (prostokątem) wtedy i tylko wtedy, gdy przekątne czworokąta $ABCD$ są równej długości (są prostopadłe do siebie).
\index{równoległobok Varignona}%
\index{romb}%
\index{prostokąt}%
% de Villiers, Michael (2009), Some Adventures in Euclidean Geometry, Dynamic Mathematics Learning, p. 58, 169. ISBN 9780557102952.
\end{proposition}

%

% \subsection{Pole?}

\todofoot{Guzicki, rozdział 17}

Klaudiusz Ptolemeusz był astronomem, matematykiem i~geografem pochodzenia greckiego.
\index[persons]{Ptolemeusz, Klaudiusz}%
Urodzon w Tebaidzie (około roku 100), kształcił się, działał w~Aleksandrii; tam też zmarł około roku 170.
Napisał po grecku Μαθηματικὴ Σύνταξις, traktat w trzynastu księgach znany lepiej jako \emph{Almagest} zawierający kompendium wiedzy astronomicznej oraz matematyczny wykład teorii geocentrycznej.
Tam też znajduje się prawie\footnote{Ptolemeusz udowodnił równość, a nie nierówność, ale nazwa się przyjęła.} całe następujące twierdzenie:

% TODO: The Almagest was preserved, like many extant Greek scientific works, in Arabic manuscripts; the modern title is thought to be an Arabic corruption of the Greek name Hē Megistē Syntaxis ('The greatest treatise'), as the work was presumably known during late antiquity.[35] Because of its reputation, it was widely sought and translated twice into Latin in the 12th century, once in Sicily and again in Spain.[36] Ptolemy's planetary models, like those of the majority of his predecessors, were geocentric and almost universally accepted until the reappearance of heliocentric models during the Scientific Revolution.

\begin{theorem}[Ptolemeusza, 140 r.n.e.]
\index{nierówność!Ptolemeusza}%
\index{twierdzenie!Ptolemeusza}%
    W czworokącie wypukłym $ABCD$ zachodzi
    \begin{equation}
        |AC| \cdot |BD| \le |AB| \cdot |CD| + |BC| \cdot |AD|,
    \end{equation}
    z równością wtedy i tylko wtedy, gdy na czworokącie $ABCD$ można opisać okrąg.
\end{theorem}

To wystarczyło mu, żeby opracować ,,tablice cięciw'' (równoważne tablicom wartości funkcji trygonometrycznych), potrzebne do celów astronomicznych.
Wcześniejsze tablice Hipparchosa z~Nikei opisywały tylko wielokrotności kąta miary $\pi/24$.
\index[persons]{Hipparchos z Nikei}% % to jest ten, co go utopili?
\todofoot{Thurston, Hugh (1996), Early Astronomy, Springer, ISBN 978-0-387-94822-5, strony 235-236}

O twierdzeniu Ptolemeusza piszą Bogdańska, Neugebauer \cite[s. 62, 63]{neugebauer_2018}, Audin \cite[s. 108]{audin_2003}.
Delta: 2024/sierpień.
Angielska Wikipedia podpowiada, że założenie o wypukłości można pominąć, czwórka punktów nie musi nawet leżeć w~jednej płaszczyźnie -- ale wtedy równość zachodzi też wtedy, kiedy punkty są współliniowe.
% https://en.wikipedia.org/wiki/Ptolemy%27s_inequality

Twierdzenie Ptolemeusza można albo uogólnić (twierdzenie Caseya podamy za chwilę), albo wyprowadzić z niego jedno z kilku twierdzeń, które przypisuje się Lazarowi Carnotowi \cite{carnot_1803}.
\index[persons]{Carnot, Lazare}
% https://ru.wikipedia.org/w/index.php?title=Формула_Карно&oldid=8679639
% В ее доказательстве используется теорема Птолемея.

\begin{theorem}[Carnot, 1803?]
    \index{twierdzenie!Carnota}%
    Niech $ABC$ będzie trójkątem wpisanym w okrąg o środku $O$ i promieniu $R$ oraz opisanym na okręgu o promieniu $r$.
    Oznaczmy przez $OO_A$ (i analogicznie $OO_B$, $OO_C$) znakowaną odległość punktu $O$ od boku $BC$.
    Wtedy 
    \begin{equation}
        OO_A + OO_B + OO_C = R + r.
    \end{equation}
    (Odległość jest ujemna wtedy i tylko wtedy, gdy cały odcinek leży poza trójkątem).
    \index{twierdzenie!Carnota}%
\end{theorem}

Wynik ten znajduje znowu zastosowanie w dowodzie twierdzenia japońskiego. % TODO: Neugebauer s. 65
\index{twierdzenie!japońskie}

% TODO: https://en.wikipedia.org/wiki/Van_Schooten's_theorem

\begin{theorem}[Caseya, 1866]
\index{twierdzenie!Caseya}%
    Niech $\Gamma_1$, $\Gamma_2$, $\Gamma_3$, $\Gamma_4$ będą czterema okręgami ponumerowanymi zgodnie z ruchem wskazówek zegara, z których każdy styka się z piątym okręgiem $\Gamma$.
    Niechh $t_{ij}$ oznacza długość zewnętrznego odcinka stycznego łączącego okręgi $\Gamma_i$, $\Gamma_j$ (jeśli te stykają się z $\Gamma$ obydwa od wewnątrz lub obydwa od zewnątrz) albo długość wewnętrznego odcinka stycznego (w przeciwnym razie).
    Wówczas:
    \begin{equation}
        t_{12} \cdot t_{34} + t_{14} \cdot t_{23} = t_{13} \cdot t_{24}.
    \end{equation}
\end{theorem}

% https://en.wikipedia.org/wiki/Casey%27s_theorem
Twierdzenie podał John Casey (1820-1891), szanowany irlandzki geometra, który razem z Émilem Lemoinem uznawany jest za współzałożyciela nowoczesnej geometrii trójkątów i okręgów.
\index[persons]{Casey, John}%
\index[persons]{Lemoine, Émile}% % Émile Michel Hyacinthe Lemoine
Inny dowód wymyślił Max Zacharias \cite{zacharias_1942}.
\index[persons]{Zacharias, Max}%
Twierdzenie odwrotne do podanego przydaje się w najkrótszym znanym dowodzie twierdzenia Feuerbacha (że okrąg dziewięciu punktów jest styczny do okręgów dopisanych oraz wpisanego).
\index{okrąg!wpisany}%
\index{okrąg!dopisany}%
\index{twierdzenie!Feuerbacha}%
\index{okrąg!dziewięciu punktów}%

Znajdziemy je u Bogdańskiej, Neugebauera jako ćwiczenie \cite[s. 105]{neugebauer_2018}.

%


\subsection{Potęga punktu względem okręgu}

\begin{proposition}
\label{guzicki_6_11}%
    Dane są dwa niewspółśrodkowe okręgi $\omega_1$ i $\omega_2$.
    Miejscem geometrycznym punktów $P$ mających równe potęgi względem obu okręgów jest prosta prostopadła do prostje przechodzącej przez środki obu okręgów.
\index{potęga punktu}%
\end{proposition}

Patrz Guzicki \cite[s. 173, 174]{guzicki_2021}.
Prostą, której istnienie właśnie zasugerowaliśmy, nazywamy \textbf{osią potęgową} okręgów $\omega_1, \omega_2$.

\begin{corollary}
	Dane są trzy parami niewspółśrodkowe okręgi na płaszczyźnie: $\omega_1, \omega_2, \omega_3$.
	Jeśli środki tych okręgów są współliniowe, to osie potęgowe każdej pary są równoległe.
	W przeciwnym razie wszystkie trzy osie przecinają się w~jednym punkcie zwanym \textbf{środkiem potęgowym} tych trzech okręgów.
\end{corollary}

Patrz Guzicki \cite[s. 174]{guzicki_2021}.


UW zrobione:
Potęga punktu względem okręgu, oś potęgowa dwóch okręgów, środek potęgowy trzech okręgów.

UW niezrobione:
twierdzenie Brianchona, konstrukcja stycznej do okręgu samą linijką, okręgi współpękowe, twierdzenie Gaussa-Bodenmillera, twierdzenie o motylku, formuła Eulera na odległość między środkami okręgu opisanego i wpisanego (dla trójkąta), twierdzenie Ponceleta dla trójkąta.

\begin{definition}[potęga punktu względem okręgu]
	Jeżeli...
\end{definition}
\begin{proposition}[potęgowe kryterium współokręgowości]
	Jeżeli...
\end{proposition}
\begin{definition}[oś potęgowa]
	Jeżeli...
\end{definition}
\begin{theorem}[Monge'a]
	Jeżeli...
\end{theorem}
\begin{theorem}[Auberta]
	Jeżeli...
\end{theorem}


\subsection{Twierdzenie o dwusiecznej}
Okrąg Apolloniusza, Guzicki-4

\begin{proposition}[twierdzenie o dwusiecznej]
	Jeżeli...
\end{proposition}
\begin{theorem}[Lehmusa-Steinera]
	Jeżeli...
\end{theorem}
\begin{definition}[okrąg Apoloniusza]
	Jeżeli...
\end{definition}


\subsection{Twierdzenie Newtona i Gaussa?}
Twierdzenie Newtona: środek okręgu wpisanego w czworokąt i środki przekątnych tego czworokąta są współliniowe.
Twierdzenie Gaussa: środki przekątnych czworokąta zupełnego są współliniowe.

\subsection{Dwustosunek}

\subsection{Okręgi ortogonalne, pęki okręgów.}
Wie czym są pęki okręgów, zna ich podstawowe własności i potrafi stosować w konfiguracjach spokrewnionych z twierdzeniem Ponceleta.   

% T2.19 tutaj

Bogdańska, Neugebauer \cite[s. 267]{neugebauer_2018} na ostatniej stronie podają niespodziewanie informacją, że twierdzenie Ponceleta {\color{red}\textbf{(TODO: T2.19)}\color{black}} było motywem przewodnim całego skryptu.
% todo: podlinkować te cztery dowody po ich spisaniu
Zachęcają do uogólnienia czwartego dowodu dla poniższej wersji:

\begin{theorem}[Ponceleta, małe]
	Niech trójkąt $A_0 A_1 A_2$ będzie wpisany w~stożkową $C$ oraz opisany na stożkowej $D$.
	Wtedy każdy punkt $B_0$ stożkowej $C$ jest wierzchołkiem dokładnie jednego trójkąta $B_0 B_1 B_2$ wpisanego w~stożkową $C$ oraz opisanego na stożkowej $D$.
\end{theorem}

Oczywiście jest też wielkie twierdzenie Ponceleta, udowodnione przez, jak niezbyt trudno się domyślić, Victora Ponceleta \cite[s. 311-317]{poncelet_1865} (wg Bogdańskiej, Neugebauera w 1813 roku, wg angielskiej Wikipedii w 1822 roku):x

\begin{theorem}[Ponceleta, wielkie]
	Niech $C$ i $D$ będą dwiema stożkowymi, zaś $A_0, A_1, \ldots, A_{n-1}$ takimi punktami na stożkowej $C$, że proste $A_0A_1$, $A_1A_2$, \ldots, $A_{n-1}A_0$ są styczne do stożkowej $D$.
	Wtedy dla każdego punktu $B_0$ na stożkowej $C$ istnieją różne punkty $B_1, \ldots, B_{n-1}$, też na stożkowej $C$, że proste $B_0B_1$, $B_1B_2$, \ldots, $B_{n-1}B_0$ są styczne do stożkowej $D$.
\end{theorem}

Dowód można znaleźć na przykład u Akopiana, Zasławskiego \cite[s. 93, 61, 67, 115, 124]{akopyan_2007}.


\subsection{Prosta Eulera i okrąg Feuerbacha}
Prosta Eulera w trójkącie (środek okręgu opisanego, środek ciężkości, ortocentrum).
Wszystkie wysokości itd. przecinają się w jednym punkcie; prosta Eulera, okrąg Feuerbacha, punkt Torricellego/Fermata (Guzicki-8)
%

\subsection{Okrąg Feuerbacha}
W 1822 roku Karl Wilhelm Feuerbach, którego nazwiskiem nazywa się czasem okrąg dziewięciu punktów, zauważył, że sześć charakterystycznych punktów trójkąta – środki boków oraz spodki wysokości – leżą na wspólnym okręgu. Odkrycia tego dokonali wcześniej, w 1821 roku, Charles Brianchon i Jean-Victor Poncelet[3]. Jeszcze wcześniej, nad współokręgowością wspomnianych punktów zastanawiali się Benjamin Bevan (1804) i John Butterworth (1807)[3].
Krótko po Feuerbachu, matematyk Olry Terquem niezależnie udowodnił istnienie okręgu i jako pierwszy zauważył, że leżą na nim również środki odcinków łączących wierzchołki z ortocentrum. Terquem jako pierwszy użył również nazwy „okrąg dziewięciu punktów”[4].
Karl Wilhelm Feuerbach udowodnił, że w dowolnym trójkącie okrąg dziewięciu punktów jest styczny wewnętrznie do okręgu wpisanego i zewnętrznie do trzech okręgów dopisanych[5]. Punkt styczności okręgu wpisanego i okręgu dziewięciu punktów nazywa się często punktem Feuerbacha[6].
% Środek okręgu dziewięciu punktów leży na tzw. prostej Eulera, dokładnie w połowie odcinka pomiędzy ortocentrum tego trójkąta a środkiem okręgu na nim opisanego[7].
In geometry, the nine-point circle is a circle that can be constructed for any given triangle. It is so named because it passes through nine significant concyclic points defined from the triangle. These nine points are:
The midpoint of each side of the triangle
The foot of each altitude
The midpoint of the line segment from each vertex of the triangle to the orthocenter (where the three altitudes meet; these line segments lie on their respective altitudes).[1][2]
The nine-point circle is also known as Feuerbach's circle (after Karl Wilhelm Feuerbach), Euler's circle (after Leonhard Euler), Terquem's circle (after Olry Terquem), the six-points circle, the twelve-points circle, the n-point circle, the medioscribed circle, the mid circle or the circum-midcircle. Its center is the nine-point center of the triangle.[3][4]
Although he is credited for its discovery, Karl Wilhelm Feuerbach did not entirely discover the nine-point circle, but rather the six-point circle, recognizing the significance of the midpoints of the three sides of the triangle and the feet of the altitudes of that triangle. (See Fig. 1, points D, E, F, G, H, I.) (At a slightly earlier date, Charles Brianchon and Jean-Victor Poncelet had stated and proven the same theorem.) But soon after Feuerbach, mathematician Olry Terquem himself proved the existence of the circle. He was the first to recognize the added significance of the three midpoints between the triangle's vertices and the orthocenter. (See Fig. 1, points J, K, L.) Thus, Terquem was the first to use the name nine-point circle.
% The first major discovery that led to the discovery of the nine-point circle was by Benjamin Bevan in 1804 as he made a mathematical proposal that inevitably established the conclusions that “the nine point center bisects the distance between the circumcentre and the orthocenter, and that the radius of the nine-point circle is half the radius of the circumcircle”(Mackay, "History of the Nine Point Circle"). A mathematician by the name of John Butterworth later in 1804 proved this proposal and subsequent conclusions in mathematical journals, and in 1807 formed a key question for the further exploration of Benjamin Bevan’s proposed phenomenon.  He asks, “When the base and vertical angle are given, what is the locus of the centre of the circle passing through the three centres of the circles touching one side and the prolongation of the other two sides of a plane triangle?” in 1806.  In response a man by the name of John Whitley made the important discovery that the circumcircle of a triangle intersects two of the midpoints of the sides, two of the feet of the altitudes of the triangle, as well as two of the mid points of the segments intercepted between the orthocenter and the vertices.  At this point in time only seven of the nine points had been discovered. The discovery of the full nine-points and the full nine points were fully mentioned for the first time in 1821 by Jean-Victor Poncelet and his partner Bianchon in a mathematical journal.  Soon after in 1822 Karl Feurbach proved the existence of the same circle independently and received much of the credit for its discovery.  Up until this point in time there was no official name for this circle that had been discovered but in 1842 a man by the name of Olry Terquem coined the term the nine-point circle in an analytical proof investigating some of the subsequent properties of the circle.  Today we know of at least 25 important points that actually lie on the so called "Nine point circle" (Mackay, "History of the Nine Point Circle"). 
% The nine-point circle also passes through Kimberling centers X_i for i=11 (the Feuerbach point), 113, 114, 115 (center of the Kiepert hyperbola), 116, 117, 118, 119, 120, 121, 122, 123, 124, 125 (center of the Jerabek hyperbola), 126, 127, 128, 129, 130, 131, 132, 133, 134, 135, 136, 137, 138, 139, 1312, 1313, 1560, 1566, 2039, 2040, and 2679.
1821
The nine points are explicitly mentioned in Gergonne's Annales de Mathematiques , volume xi., in an article by Brianchon and Poncelet. This article contains the theorem establishing the characteristic property of the nine point circle.
1822
First enunciation of Feuerbach's Theorem, including the first published proof, appears in Karl Wilhelm Feuerbach's Eigenschaften einiger merkwiirdigen Punkte des geradlinigen Dreiecks, along with many other interesting proofs relating to the nine point circle.
1842
The circle is officially designated the "nine point circle" (le cercle des neuf points) by Terquem, one of the editors of the Nouvelles Annales. (see Volume I page 198). Terquem published the second analytical proof of the theorem that the nine point circle touches the incircle and the excircles.

%

\subsection{Trygonometria}

\subsubsection{Twierdzenie sinusów}

$$\frac{a}{\sin \alpha} = \frac{b}{\sin \beta} = \frac{c}{\sin \gamma} = 2R$$
% https://en.wikipedia.org/wiki/Law_of_sines

\subsubsection{Twierdzenie cosinusów}
\begin{proposition}[twierdzenie cosinusów]
	\label{twierdzenie_cosinusow}%
	\begin{equation}
		c^2 = a^2 + b^2 - 2ab \cos \gamma.
	\end{equation}
	% https://en.wikipedia.org/wiki/Law_of_cosines
\end{proposition}

\index[persons]{Archimedes}%

Wzory na promienie okręgów wpisanych, dopisanych.


\subsubsection{Zastosowania trygonometrii -- twierdzenie Urquharta}
Twierdzenie Urquharta

\subsubsection{Zastosowania trygonometrii -- punkt i kąt Crelle'a-Brocarda}
Punkt i kąt Crelle'a-Brocarda.

\subsubsection{Zastosowania trygonometrii -- twierdzenie o siódmym okręgu}
Twierdzenie o siódmym okręgu.

\subsubsection{Rozwiązywanie trójkątów}
Wzór Mollweide'a.
\index{wzór!Mollweide'a}%

Problem Hansena
\index{problem!Hansena}%

Problem Snelliusa-Pothenota.
\index{problem!Snelliusa-Pothenota}%

% https://en.wikipedia.org/wiki/Mollweide%27s_formula
% https://en.wikipedia.org/wiki/Snellius%E2%80%93Pothenot_problem
% https://en.wikipedia.org/wiki/Hansen%27s_problem


Twierdzenie Malfattiego.
Guzicki-11

\subsection{Bałagan}

\textbf{Twierdzenie Caseya} (nie duplikat Ptolemeusza?)

\textbf{Twierdzenie Taylora, okrąg, sześciokąt}
% https://en.wikipedia.org/wiki/Taylor_circle
{
    \emph{WIP: Taylor w 1882 roku zauważył, że rzuty spodków wysokości na pozostałe boki leżą na jednym okręgu.}
}

\textbf{Twierdzenie Eulera $1/4R^2$}

% https://en.wikipedia.org/wiki/Law_of_tangents


%