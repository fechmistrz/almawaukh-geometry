%

\section{Podobieństwo}
\subsection{Jednokładność}
Podobieństwo figur, trójkątów (cechy), stosunek pól figur podobnych.

%

Guzicki-3

\begin{theorem}[Talesa]
    Jeśli ramiona kąta płaskiego przetnie się 2 równoległymi prostymi:
    \begin{center}
\begin{comment}
        \begin{tikzpicture}
            \tkzDefPoint(0, 0.5){O}
            \tkzDefPoint(1.5, 0){A}
            \tkzDefPoint(2, 1){Ap}
            \tkzDefPointBy[homothety=center O ratio 1.618](A) \tkzGetPoint{B}
            \tkzDefLine[parallel=through B](A,Ap) \tkzGetPoint{Bp}
            \tkzInterLL(O,Ap)(B,Bp) \tkzGetPoint{Bpp}
            \tkzDrawPoints[fill=gray,opacity=.9](O,A,B,Ap,Bpp)
            \tkzLabelPoint[above](O){$O$}
            \tkzLabelPoint[below](A){$A$}
            \tkzLabelPoint[below](B){$A'$}
            \tkzLabelPoint[above left](Bpp){$B'$}
            \tkzLabelPoint[above left](Ap){$B$}
            \tkzDrawLine[thick](O,B)
            \tkzDrawLine[thick](O,Bpp)
            \tkzDrawLine[color=blue, thick](A,Ap)
            \tkzDrawLine[color=blue, thick](B,Bpp)
        \end{tikzpicture}
\end{comment}
        \end{center}
    to długości odcinków wyznaczonych przez te proste na jednym z ramion kąta są proporcjonalne do długości odpowiednich odcinków na drugim ramieniu kąta, a zatem
    \begin{equation}
        \label{thales_ratio}
        \frac{|OA|}{|OA'|} = \frac{|OB|}{|OB'|} = \frac{|AB|}{|A'B'|}.
    \end{equation}
\end{theorem}
% TODO: https://en.wikipedia.org/wiki/Thales's_theorem

Tradycja przypisuje jego sformułowanie Talesowi z Miletu, chociaż znane było starożytnym Babilończykom i Egipcjanom.
\index[persons]{Tales z Miletu}%
% Pierwszy znany dowód pojawia się w Elementach Euklidesa.
Najstarszy zachowany dowód twierdzenia Talesa zamieszczony jest w VI. księdze Elementów Euklidesa. 
% https://en.wikipedia.org/wiki/Intercept_theorem#Claim_3

Piszą o nim Neugebauer, Bogdańska \cite[s. 48-56]{neugebauer_2018}; Audin \cite[s. 24, 173]{audin_2003}.

Po angielsku znane jest jako \emph{Thales's theorem}, \emph{intercept theorem}, \emph{basic proportionality theorem} albo \emph{side splitter theorem}.

Prawdziwe jest również twierdzenie odwrotne:

\begin{proposition}[twierdzenie odwrotne do tw. Talesa]
    Jeżeli pewna prosta przecina boki $OA'$, $OB'$ trójkąta $OA'B'$ w różnych punktach $A$ i $B$ odpowiednio, a przy tym zachodzi równość \ref{thales_ratio}, to prosta ta jest równoległa do prostej $A'B'$.
\end{proposition}

Prostym wnioskiem z twierdzenia Talesa jest fakt \ref{hartshorne_52}, znajduje on zastosowanie w dowodzie:
% Neugebauer s. 52

\begin{theorem}[Varignona]
    Czworokąt $PQRS$, którego wierzchołki leżą na środkach boków $AB$, $BC$, $CD$, $DA$ czworokąta $ABCD$, jest równoległobokiem.
    Jego znakowane  (!) pole jest równe połowie pola czworokąta $ABCD$. % Neugebauer s. 61
\end{theorem}

% * The area of the Varignon parallelogram equals half the area of the original quadrilateral. This is true in convex, concave and crossed quadrilaterals provided the area of the latter is defined to be the difference of the areas of the two triangles it is composed of. => [[Varignon's theorem]]


W szczególności, czworokąt $ABCD$ nie musi być wypukły\footnote{Może być nawet ,,motylkiem'', to znaczy łamaną zamkniętą o czterech bokach, która ma samoprzecięcia.}.
Twierdzenie zostało nazwane na cześć Pierre'a Varignona pośmiertnie w 1731 roku.
\index[persons]{Varignon, Pierre}%
Co więcej,

\begin{proposition}
    Równoległobok Varignona jest rombem (prostokątem) wtedy i tylko wtedy, gdy przekątne czworokąta $ABCD$ są równej długości (są prostopadłe do siebie).
\index{równoległobok!Varignona}%
\index{romb}%
\index{prostokąt}%
% de Villiers, Michael (2009), Some Adventures in Euclidean Geometry, Dynamic Mathematics Learning, p. 58, 169. ISBN 9780557102952.
\end{proposition}

%

% \subsection{Pole?}

\input{section-01/ptolemy}


\subsection{Potęga punktu względem okręgu}

\begin{proposition}
\label{guzicki_6_11}%
    Dane są dwa niewspółśrodkowe okręgi $\omega_1$ i $\omega_2$.
    Miejscem geometrycznym punktów $P$ mających równe potęgi względem obu okręgów jest prosta prostopadła do prostje przechodzącej przez środki obu okręgów.
\index{potęga punktu}%
\end{proposition}

Patrz Guzicki \cite[s. 173, 174]{guzicki_2021}.
Prostą, której istnienie właśnie zasugerowaliśmy, nazywamy \textbf{osią potęgową} okręgów $\omega_1, \omega_2$.

\begin{corollary}
	Dane są trzy parami niewspółśrodkowe okręgi na płaszczyźnie: $\omega_1, \omega_2, \omega_3$.
	Jeśli środki tych okręgów są współliniowe, to osie potęgowe każdej pary są równoległe.
	W przeciwnym razie wszystkie trzy osie przecinają się w~jednym punkcie zwanym \textbf{środkiem potęgowym} tych trzech okręgów.
\end{corollary}

Patrz Guzicki \cite[s. 174]{guzicki_2021}.


UW zrobione:
Potęga punktu względem okręgu, oś potęgowa dwóch okręgów, środek potęgowy trzech okręgów.

UW niezrobione:
twierdzenie Brianchona, konstrukcja stycznej do okręgu samą linijką, okręgi współpękowe, twierdzenie Gaussa-Bodenmillera, twierdzenie o motylku, formuła Eulera na odległość między środkami okręgu opisanego i wpisanego (dla trójkąta), twierdzenie Ponceleta dla trójkąta.

\begin{definition}[potęga punktu względem okręgu]
	Jeżeli...
\end{definition}
\begin{proposition}[potęgowe kryterium współokręgowości]
	Jeżeli...
\end{proposition}
\begin{definition}[oś potęgowa]
	Jeżeli...
\end{definition}
\begin{theorem}[Monge'a]
	Jeżeli...
\end{theorem}
\begin{theorem}[Auberta]
	Jeżeli...
\end{theorem}


\subsection{Twierdzenie o dwusiecznej}
Okrąg Apolloniusza, Guzicki-4

\begin{proposition}[twierdzenie o dwusiecznej]
	Jeżeli...
\end{proposition}
\begin{theorem}[Lehmusa-Steinera]
	Jeżeli...
\end{theorem}
\begin{definition}[okrąg Apoloniusza]
	Jeżeli...
\end{definition}


\subsection{Twierdzenie Newtona i Gaussa?}
Twierdzenie Newtona: środek okręgu wpisanego w czworokąt i środki przekątnych tego czworokąta są współliniowe.
Twierdzenie Gaussa: środki przekątnych czworokąta zupełnego są współliniowe.

\subsection{Dwustosunek}

\subsection{Okręgi ortogonalne, pęki okręgów.}
Wie czym są pęki okręgów, zna ich podstawowe własności i potrafi stosować w konfiguracjach spokrewnionych z twierdzeniem Ponceleta.   

% T2.19 tutaj

Bogdańska, Neugebauer \cite[s. 267]{neugebauer_2018} na ostatniej stronie podają niespodziewanie informacją, że twierdzenie Ponceleta {\color{red}\textbf{(TODO: T2.19)}\color{black}} było motywem przewodnim całego skryptu.
% todo: podlinkować te cztery dowody po ich spisaniu
Zachęcają do uogólnienia czwartego dowodu dla poniższej wersji:

\begin{theorem}[Ponceleta, małe]
	Niech trójkąt $A_0 A_1 A_2$ będzie wpisany w~stożkową $C$ oraz opisany na stożkowej $D$.
	Wtedy każdy punkt $B_0$ stożkowej $C$ jest wierzchołkiem dokładnie jednego trójkąta $B_0 B_1 B_2$ wpisanego w~stożkową $C$ oraz opisanego na stożkowej $D$.
\end{theorem}

Oczywiście jest też wielkie twierdzenie Ponceleta, udowodnione przez, jak niezbyt trudno się domyślić, Victora Ponceleta \cite[s. 311-317]{poncelet_1865} (wg Bogdańskiej, Neugebauera w 1813 roku, wg angielskiej Wikipedii w 1822 roku):x

\begin{theorem}[Ponceleta, wielkie]
	Niech $C$ i $D$ będą dwiema stożkowymi, zaś $A_0, A_1, \ldots, A_{n-1}$ takimi punktami na stożkowej $C$, że proste $A_0A_1$, $A_1A_2$, \ldots, $A_{n-1}A_0$ są styczne do stożkowej $D$.
	Wtedy dla każdego punktu $B_0$ na stożkowej $C$ istnieją różne punkty $B_1, \ldots, B_{n-1}$, też na stożkowej $C$, że proste $B_0B_1$, $B_1B_2$, \ldots, $B_{n-1}B_0$ są styczne do stożkowej $D$.
\end{theorem}

Dowód można znaleźć na przykład u Akopiana, Zasławskiego \cite[s. 93, 61, 67, 115, 124]{akopyan_2007}.


\subsection{Prosta Eulera i okrąg Feuerbacha}
Prosta Eulera w trójkącie (środek okręgu opisanego, środek ciężkości, ortocentrum).
Wszystkie wysokości itd. przecinają się w jednym punkcie; prosta Eulera, okrąg Feuerbacha, punkt Torricellego/Fermata (Guzicki-8)
\input{section-01/feuerbach}

\subsection{Trygonometria}

\subsubsection{Twierdzenie sinusów}

$$\frac{a}{\sin \alpha} = \frac{b}{\sin \beta} = \frac{c}{\sin \gamma} = 2R$$
% https://en.wikipedia.org/wiki/Law_of_sines

\subsubsection{Twierdzenie cosinusów}
\begin{proposition}[twierdzenie cosinusów]
	\label{twierdzenie_cosinusow}%
	\begin{equation}
		c^2 = a^2 + b^2 - 2ab \cos \gamma.
	\end{equation}
	% https://en.wikipedia.org/wiki/Law_of_cosines
\end{proposition}

\index[persons]{Archimedes}%

Wzory na promienie okręgów wpisanych, dopisanych.


\subsubsection{Zastosowania trygonometrii -- twierdzenie Urquharta}
Twierdzenie Urquharta

\subsubsection{Zastosowania trygonometrii -- punkt i kąt Crelle'a-Brocarda}
Punkt i kąt Crelle'a-Brocarda.

\subsubsection{Zastosowania trygonometrii -- twierdzenie o siódmym okręgu}
Twierdzenie o siódmym okręgu.

\subsubsection{Rozwiązywanie trójkątów}
Wzór Mollweide'a.
\index{wzór!Mollweide'a}%

Problem Hansena
\index{problem!Hansena}%

Problem Snelliusa-Pothenota.
\index{problem!Snelliusa-Pothenota}%

% https://en.wikipedia.org/wiki/Mollweide%27s_formula
% https://en.wikipedia.org/wiki/Snellius%E2%80%93Pothenot_problem
% https://en.wikipedia.org/wiki/Hansen%27s_problem


Twierdzenie Malfattiego.
Guzicki-11

\subsection{Bałagan}

\textbf{Twierdzenie Caseya} (nie duplikat Ptolemeusza?)

\textbf{Twierdzenie Taylora, okrąg, sześciokąt}
% https://en.wikipedia.org/wiki/Taylor_circle
{
    \emph{WIP: Taylor w 1882 roku zauważył, że rzuty spodków wysokości na pozostałe boki leżą na jednym okręgu.}
}

\textbf{Twierdzenie Eulera $1/4R^2$}

% https://en.wikipedia.org/wiki/Law_of_tangents


%