
\subsection{Współliniowość, współpękowość}



Znamy trzy twierdzenia o współliniowości: ..., ... i twierdzenie o prostej Auberta ...

\begin{proposition}[twierdzenie Salmona]
	Dany jest okrąg oraz trzy jego różne cięciwy $PA$, $PB$, $PC$ takie, że przekrojem okręgów na średnicach $PA$, $PB$ (odpowiednio: $PB$, $PC$ i $PA$, $PC$) są punkty $P$, $M$ (odpowiednio: $P$, $K$ oraz $P$, $L$).
	Wtedy punkty $K$, $L$, $M$ są współliniowe.
\end{proposition}

\begin{proposition}[twierdzenie Menelaosa]
	...
	Wówczas punkty $K, L, M$ są współliniowe wtedy i tylko wtedy, gdy zachodzi
	\begin{equation}
		[AMB] [BKC] [CLA] = -1.
	\end{equation}
\end{proposition}
% https://en.wikipedia.org/wiki/Menelaus%27s_theorem
It is uncertain who actually discovered the theorem; however, the oldest extant exposition appears in Spherics by Menelaus. In this book, the plane version of the theorem is used as a lemma to prove a spherical version of the theorem.

% \begin{proposition}[twierdzenie Carnota???]
	% Neugebauer, strona 108.
% \end{proposition}

% https://en.wikipedia.org/wiki/Newton%E2%80%93Gauss_line#Existence_of_the_Newton%E2%88%92Gauss_line

\begin{proposition}
	Środki trzech przekątnych czworoboku zupełnego leżą na jednej prostej, zwaną prostą Newtona-Gaussa.
\end{proposition}

\begin{proposition}[twierdzenie Desargues'a]
	Neugebauer, strona 109.
	% https://en.wikipedia.org/wiki/Desargues%27s_theorem
\end{proposition}

\begin{proposition}[twierdzenie Pascala]
	Neugebauer, strona 113.
	% https://en.wikipedia.org/wiki/Pascal%27s_theorem
\end{proposition}

\begin{proposition}[twierdzenie Pappusa]
	Neugebauer, strona 114.
	% https://en.wikipedia.org/wiki/Pappus%27s_hexagon_theorem
\end{proposition}






1. menelaos, desargues, pascal, pappus (MENELAOS = guzicki-3)
1. Zna pojęcie inwolucji rzutowych.   Zna i potrafi stosować twierdzenia inwolucyjne Desarguesa.  
 
\subsubsection{Współpękowość}
zadanie newtona, okręgi torricelliego, ceva, carnot (suma kwadratów = suma kwadratów). Ceva = guzicki-3
twierdzenie Carnota: trzy proste są współpunktowe wtw AF2 + BD2 + CE2 = AE2 + BF2 + CD2. Wniosek: symetralne są współpunktowe. GUZICKI-6
UW: Twierdzenie Cevy (wraz z trygonometryczną wersją), przykłady punktów szczególnych trójkąta: punkt Nagela (Guzicki-4), punkt Gergonne'a (guzicki4), punkt Lemoine'a.


Ważnym kryterium współpękowości trzech czewian jest:

\begin{proposition}[twierdzenie Cevy (1678)]
	Dany jest trójkąt $ABC$ i trzy różne od wierzchołków punkty $K \in BC$, $L \in CA$, $M \in AB$.
	Wówczas czewiany $AK$, $BL$, $CM$ są współpękowe wtedy i tylko wtedy, gdy
	\begin{equation}
		[AMB] [BKC] [CLA] = 1.
	\end{equation}
\end{proposition}

% CARNOT
Uogólnieniem twierdzenia o współpękowości symetralnych boków trójkąta jest:

\begin{proposition}[twierdzenie Carnota]
	... wtedy i tylko wtedy, gdy
	\begin{equation}
		|AM|^2 + |BK|^2 + |CL|^2 = |AL|^2 + |CK|^2 + |BM|^2.
	\end{equation}
\end{proposition}
% CARNOT


\subsubsection{Do włączenia w powyższe podpodsekcje}
Twierdzenie Brianchona.
\begin{enumerate}
    \item Zna przykłady przekształceń rzutowych i umie je stosować w zadaniach i dowodach twierdzeń rzutowych (Desarguesa, Pappusa, Pascala, Brianchona). zna pojęcia: biegun i biegunowa i potrafi formułować twierdzenia dualne.  
    \item twierdzenia Newtona i Brianchona (s. 237) - GUZICKI 9
    \item Twierdzenie Kirkmana: jeśli część wspólna dwóch trójkątów wpisanych w okrąg jest sześciokątem wypukłym, to główne przekątne tego sześciokąta przecinają się w jednym punkcie. - TO JEST BARDZIEJ POD JEDNOKŁADNOŚĆ (UW)
    \item Wg Wiki, to jest wniosek z Desarguesa/Menelaos: twierdzenie o środkach jednokładności trzech okręgów, patrz TODO w kodzie źródłowym % (chyba https://atcm.mathandtech.org/EP2016/contributed/4052016_21160.pdf), na UW po: 	- Twierdzenia o składaniu jednokładności i przesunięć, 
\end{enumerate}

\subsubsection{Czewiany, symediany, twierdzenie van Aubela}
+ twierdzenie Steinera?

\subsection{Współrzędne barycentryczne}
Współrzędne barycentryczne

\subsection{Inwersja względem okręgu}
Patrz Guzicki-20: twierdzenie Ptolemeusza, zadanie Apolloniusza, zadanie Sangaku.

\subsection{Izogonalne}
Punkty izogonalnie sprzężone w trójkącie. + Twierdzenie Menelausa. (UW1)