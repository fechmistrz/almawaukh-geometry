%

\subsubsection{Księga III}
\paragraph{Definicje}
\begin{enumerate}
    \item [3.1] Dwa okręgi są przystające, kiedy mają równe średnice (lub równoważnie, promienie).
    \item [3.2] Definicja ...
    % Definicja 2. % Mówi się, że linia prosta dotyka koła, gdy będąc styczną z kołem przedłużona z obydwu stron nie przecina się z żadnej strony okręgu koła.
    \item [3.3] Dwa okręgi nazywamy stycznymi, kiedy mają dokładnie jeden punkt wspólny.
    \item [3.4] Definicja ...
    % Definicja 4. % Mówi się, że linie proste równoodległe są od środka koła, gdy prostopadłe ze środka koła na nie spuszczone są równe.
    \item [3.5] Definicja ...
    % Definicja 5. % Mówi się, że ta linia prosta bardziej jest odległa od środka koła, na którą prostopadła ze środka koła spuszczona jest większa.
    \item [3.6] Definicja ...
    % Definicja 6. % Odcinkiem koła jest figura czyli część koła ograniczona linią prostą i okręgiem koła.
    \item [3.7] Definicja ...
    % Definicja 7. % Kąt zaś odcinka jest ten, który się linią prostą i okręgiem koła zawiera.
    \item [3.8] Definicja ...
    % Definicja 8. % Jeżeli na okręgu koła wzięty będzie punkt i od niego będą poprowadzone linie proste do końców linii prostej za podstawę odcinkami służącej, kąt między tymi liniami prostymi zawarty jest kątem w odcinku.
    \item [3.9] Definicja ...
    % Definicja 9. % Kiedy zaś linie proste kąt zawierające zajmują część okręgu, mówi się, że kąt ten opiera się na okręgu koła.
    \item [3.10] Definicja ...
    % Definicja 10. % Jeżeli kąt ma swój wierzchołek we środku koła; figura czyli część koła zawarta między ramionami tegoż koła, to jest między promieniami i łukiem koła nazywa się wycinkiem koła.
    \item [3.11] Definicja ...
    % Definicja 11. % Odcinkami podobnymi kół nazywają się te, które zajmują kąty równe, lub w których kąty są równe między sobą.
\end{enumerate}

\paragraph{Twierdzenia}
\begin{enumerate}
    \item [3.1] Skonstruować środek danego okręgu. 
    \item [3.2] Twierdzenie ...
    % Twierdzenie 2. % Jeżeli na okręgu obierzemy dwa gdziekolwiek punkty, linia prosta łącząca te punkty padnie wewnątrz koła.
    \item [3.3] Twierdzenie ...
    % Twierdzenie 3. % Jeżeli w kole linia prosta przez środek poprowadzona przecina linie nie przez środek poprowadzoną na dwie równe części, będzie pierwsza prostopadła do drugiej; i jeżeli pierwsza jest prostopadła do drugiej, przecina ja na dwie równe części.
    \item [3.4] Twierdzenie ...
    % Twierdzenie 4. % Jeżeli w kole dwie linie proste, nie przez środek koła poprowadzone przecinają się nawzajem, nie przetną się na dwie równe części.
    \item [3.5] Dwa okręgi, które się przecinają, nie mogą być współśrodkowe. 
    \item [3.6] Twierdzenie ...
    % Twierdzenie 6. % Jeżeli dwa koła dotykają się wzajemnie, to wspólnego środka mieć nie mogą.
    \item [3.7] Twierdzenie ...
    % Twierdzenie 7. % Jeżeli na średnicy koła wzięty będzie punkt którykolwiek oprócz średnicy koła i od tego punktu poprowadzone linie proste do okręgu, ze wszystkich linii największa będzie część średnicy, na której znajduje się środek koła, a najmniejsza pozostała część średnicy, z innych zaś linii prostych każda bliższa przechodząca przez środek koła, większa będzie od odleglejszej, z tego na koniec punktu dwie tylko równe linie proste z obydwu stron najmniejszej linii prostej mogą być do okręgu poprowadzone.
    \item [3.8] Twierdzenie ...
    % Twierdzenie 8. % Jeżeli z punktu zewnątrz koła obranego, poprowadzone będą do okręgu linie proste, z których jedna przechodziła by przez środek koła a inne padały gdziekolwiek, z linii prostych padających na część okręgu wklęsłą, największa jest linia poprowadzona przez środek koła, z innych zaś linii każda bliższa przechodzącej przez środek jest większa od odleglejszej. Lecz z linii padających na cześć okręgu wypukłą, najmniejsza jest linia prosta zawarta między punktem zewnętrz koła i średnicą, z innych zaś linii prostych każda bliższa najmniejszej, mniejsza jest odleglejsza; na koniec dwie tylko równe linie proste z tego punktu po obydwu stronach najmniejszej linii prostej mogą być do okręgu poprowadzone.
    \item [3.9] Twierdzenie ...
    % Twierdzenie 9. % Jeżeli z punktu danego wewnątrz koła poprowadzimy do okręgu więcej niż dwie linie proste i te proste są miedzy sobą równe, punkt ten będzie środkiem koła.
    \item [3.10] Dwa okręgi, które się przecinają, przecinają się w dwóch punktach. 
    \item [3.11] Twierdzenie ...
    % Twierdzenie 11. % Jeżeli dwa koła stykają się ze sobą wewnątrz, linia łącząca środki tychże kół przedłużona pada na punkt dotykania się kół.
    \item [3.12] Twierdzenie ...
    % Twierdzenie 12. % Jeżeli dwa koła dotykają się ze sobą zewnętrznie, to linia prosta łącząca ich środki przechodzi przez punkt dotykania się.
    \item [3.13] Twierdzenie ...
    % Twierdzenie 13. % Okrąg koła nie może dotykać okręgu drugiego koła w więcej niż jednym punkcie, nieważne jest czy dotkniecie jest zewnętrzne bądź wewnętrzne.
    \item [3.14] Twierdzenie ...
    % Twierdzenie 14. % W kole linie proste równe, na okręgu jego zakończone, są równoodległe od środka; i linie proste które na okręgu jego zakończone są równoodległe od środka, są też miedzy sobą równe.
    \item [3.15] Twierdzenie ...
    % Twierdzenie 15. % Ze wszystkich linii prostych w kole poprowadzonych i na okręgu jego zakończonych, największa jest średnica, z innych zaś każda bliższa środka koła, większa jest od odleglejszej; i z dwóch linii prostych nierównych, większa bliższa jest środka koła od mniejszej.
    \item [3.16] Twierdzenie ... 
    % Twierdzenie 16. % Prostopadła do średnicy koła z końca jej wyprowadzona, pada cała zewnątrz koła, a między tą prostopadłą i okręgiem żadna inna linia prosta nie pada; albo tak samo: okrąg koła przechodzi miedzy prostopadłą do średnicy i linią prostą, która ze średnicą kąt ostry jakokolwiek wielki zawiera, czyli która zawiera kąt jakokolwiek mały z prostopadłą do średnicy.
    \item [3.17] Skonstruować styczną do danego okręgu, która przechodzi przez dany punkt.
    \item [3.18] Twierdzenie ...
    % Twierdzenie 18. % Jeżeli linia prosta dotyka się okręgu koła, a ze środka koła wyprowadzona będzie linia prosta do punktu dotykania się, to ta będzie prostopadła do stycznej.
    \item [3.19] Twierdzenie ...
    % Twierdzenie 19. % Jeżeli linia prosta dotyka okręgu koła, z punktu zaś dotknięcia wyprowadzona będzie do tej stycznej prostopadła, to na prostopadłej będzie środek koła.
    \item [3.20] Twierdzenie ...
    % Twierdzenie 20. % W kole, kąt mający wierzchołek we środku jest podwojeniem kata mającego swój wierzchołek na okręgu koła, gdyż tę samą podstawę okręgu mają za podstawę, czyli to samo gdy ramionami swymi tej samej części okręgu obejmują.
    \item [3.21] Twierdzenie ...
    % Twierdzenie 21. % Kąty w tym samym odcinku koła są między sobą równe.
    \item [3.22] Twierdzenie ...
    % Twierdzenie 22. % Kąty przeciwne czworokąta w koło wpisane są równe dwóm kątom prostym.
    \item [3.23] Twierdzenie ...
    % Twierdzenie 23. % Na tej samej linii prostej nie można wykreślić dwóch odcinków kół po tej samej stronie podobnych, które by nie przystawały do siebie.
    \item [3.24] Twierdzenie ...
    % Twierdzenie 24. % Wykreślone na równych liniach prostych podobne odcinki kół, są między sobą równe.
    \item [3.25] Twierdzenie ...
    % Twierdzenie 25. % Mając dany odcinek koła, opisać koła którego jest odcinkiem.
    \item [3.26] Twierdzenie ...
    % Twierdzenie 26. % W kołach równych, kąty równe w środkach lub przy okręgach wspierają się na równych łukach.
    \item [3.27] Twierdzenie ...
    % Twierdzenie 27. % W kołach równych, kąty we środkach lub przy okręgach, na równych łukach wspierające się, są między sobą równe.
    \item [3.28] Twierdzenie ...
    % Twierdzenie 28. % W kołach równych, cięciwy równe obejmują łuki równe, tak, że łuk większy większemu, mniejszy mniejszemu jest równy.
    \item [3.29] Twierdzenie ...
    % Twierdzenie 29. % W kołach równych, równe łuki obejmują cięciwy równe.
    \item [3.30] Podzielić dany Twierdzenie ...
    % Twierdzenie 30. % Dany łuk podzielić na dwie części.
    \item [3.31] Twierdzenie ...
    % Twierdzenie 31. % W kole, kąt w półkolu jest prosty; z katów zaś w odcinkach nierównych, kąt w większym odcinku mniejszy jest od prostego; a w mniejszym odcinku większy od prostego.
    \item [3.32] Twierdzenie ...
    % Twierdzenie 32. % Jeżeli okręgu koła dotyka linia prosta, z punktu zaś dotknięcia poprowadzona będzie cięciwa, kąty zawarte miedzy cięciwową i styczną, będą równe kątom w odcinkach koła na przemian.
    \item [3.33] Twierdzenie ...
    % Twierdzenie 33. % Na danej linii prostej wykreślić odcinek koła który by zawierał kąt równy kątowi danemu.
    \item [3.34] Twierdzenie ...
    % Twierdzenie 34. % Z koła danego oddzielić odcinek któryby zawierał kąt równy danemu kątowi.
    \item [3.35] Twierdzenie ...
    % Twierdzenie 35. % Jeżeli w kole dwie cięciwy przecinają się nawzajem, prostokąt zawarty odcinkami jednej cięciwy będzie równy prostokątowi zawartemu odcinkami drugiej cięciwy.
    \item [3.36] Twierdzenie ...
    % Twierdzenie 36. % Jeżeli z punktu za kołem obranego, poprowadzimy dwie linie proste, których jedna przecinałaby koło, a druga byłaby styczną; to prostokąt zawarty całą linia przecinającą i odcinkiem jej za kołem będzie równy kwadratowi ze stycznej.
    \item [3.37] Twierdzenie ...
    % Twierdzenie 37. % Jeżeli z dwóch linii prostych, od jednego punktu zewnątrz koła obranego poprowadzonych, jedna przecina koło, a druga pada na okrąg tego koła: i jeżeli prostokąt z całej linii przecinającej i odcinka jej za kołem będącego jest równy kwadratowi z linii padającej na okrąg koła, to linia będzie padająca na okrąg koła styczną.
\end{enumerate}

%