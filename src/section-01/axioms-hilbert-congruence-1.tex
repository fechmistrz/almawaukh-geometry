\subsubsection{Aksjomaty przystawania odcinków}
Listę aksjomatów rozszerzymy o trzy, które opisują niezdefiniowane pojęcie przystawania odcinków: relacji między odcinkami, którą oznaczamy przez $\cong$.

\begin{axiom}[przystawania, C1]
    Niech $A$ i $B$ będą punktami, zaś $l$ prostą przechodzącą przez punkt $C$.
    Wtedy po każdej stronie punktu $C$ istnieje punkt $D$ taki, że odcinki $AB \cong CD$ są przystające.
\end{axiom}

Ten aksjomat odpowiada konstrukcji (I.3) Euklidesa i pozwala przenosić odcinki.

\begin{axiom}[przystawania, C2]
    Jeśli $AB \cong CD$ oraz $AB \cong EF$, to $CD \cong EF$.
    Każdy odcinek jest przystający do siebie.
\end{axiom}

To było pierwsze pojęcie pierwotne dla Euklidesa.

\begin{axiom}[przystawania, C3]
    Niech $A$, $B$, $C$, $D$, $E$, $F$ będą takimi punktami, że $B$ leży między $A$ i $C$, zaś $E$ leży między $D$ i $F$.
    Jeśli $AB \cong DE$ i $BC \cong EF$, to $AC \cong DF$.
\end{axiom}

Ten aksjomat pozwala nam dodawać odcinki i znowu zastępuje pojęcie pierwotne Euklidesa.

\begin{proposition}[odejmowanie odcinków]
    Niech $A$, $B$, $C$, $D$, $E$, $F$ będą takimi punktami, że $B$ leży między $A$ i $C$, zaś $E$ i $F$ leżą na półprostej zaczynającej się w $D$.
    Jeśli $AB \cong DE$ i $AC \cong DF$, to punkt $E$ leży między $D$ i $F$, co więcej $BC \cong EF$.
\end{proposition}

Odcinek $BC$ traktujemy jako różnicę między $AC$ i $AB$.
Dla Euklidesa to było pojęcie pierwotne:

\begin{definition}[okrąg]
    Niech $O$, $A$ będą dwoma różnymi punktami.
    Zbiór punktów $B$ takich, że odcinki $OA$ i $OB$ są przystające nazywamy okręgiem o środku $O$ oraz promieniu $OA$; okręgi często oznacza się literą $\Gamma$.
\end{definition} % Hartshorne 89

\begin{proposition}
    Każda prosta, która przechodzi przez środek, przecina okrąg w dwóch punktach.
    Okrąg składa się z nieskończenie wielu punktów.
\end{proposition}

(Nie jest jasne, ile środków może mieć okrąg, ale Hartshorne \cite[s. 89]{hartshorne2000} obiecuje pokazać póżniej, że tylko jeden.
Później ma miejsce na stronie 104).

\begin{definition}[styczna]
    Niech $\Gamma$ będzie okręgiem, zaś $l$ prostą, która przecina $\Gamma$ w dokładnie jednym punkcie $A$.
    Mówimy, że $l$ jest styczną do okręgu $\Gamma$ w punkcie $A$.
\end{definition}

Podobnie mówimy, że dwa okręgi są styczne, jeśli mają jeden punkt wspólny.