%

\subsection{Aksjomaty Hilberta}
\subsubsection{Aksjomaty incydencji}
Aksjomaty incydencji I1, I2, I3.

\begin{proposition}
    Dwie różne proste mogą mieć co najwyżej jeden punkt wspólny.
\end{proposition}

\begin{definition}
    Dwie różne proste, które nie mają punktów wspólnych, nazywamy równoległymi.
    Każda prosta jest też równoległa do siebie.
\end{definition}

\begin{definition}[aksjomat Playfaira]
    Dla każdej prostej $l$ oraz punktu $A$, istnieje dokładnie jedna prosta przechodząca przez $A$, równoległa do $l$.
\end{definition}

\begin{example}
    Rozważmy zbiór pięciu punktów $A$, $B$, $C$, $D$, $E$, w którym proste są dowolnymi dwuelementowymi podzbiorami.
    Wtedy proste $AB$ i $AC$ mają punkt wspólny $A$, chociaż obydwie są równoległe do prostej $DE$.
    Aksjomat Playfaira nie jest spełniony.
\end{example}

\begin{proposition}
    Aksjomaty I1, I2, I3, P (Playfaira) są od siebie niezależne.
\end{proposition}

Hartshorne \cite[s. 69-70]{hartshorne2000} konstruuje modele geometrii, w których spełnione są dowolne trzy, ale nie czwarty z nich.

\begin{proposition}
    Płaszczyzna rzutowa to taki zbiór punktów oraz prostych (podzbiorów zbioru punktów), że: przez dwa różne punkty przechodzi dokładnie jedna prosta, każde dwie proste mają punkt wspólny, każda prosta ma co najmniej trzy punkty i nie wszystkie punkty są współliniowe.
    Każda płaszczyzna rzutowa ma co najmniej siedem punktów, dokładnie jedna płaszczyzna rzutowa ma dokładnie siedem punktów, każdy z wymienionych zdanie wcześniej aksjomatów jest niezależny od pozostałych.
    Co więcej, wynikają z nich wszystkie trzy aksjomaty incydencji.
    Jeśli istnieje prosta, która ma $n+1$ punktów, to płaszczyzna ma $n^2 + n + 1$ punktów.
\end{proposition} % Hartshorne 71

\begin{proposition}
    Płaszczyzna afiniczna to taki zbiór punktów i prostych, które spełniają aksjomaty incydencji oraz mocniejszą wersję aksjomatu Playfaira: dla każdej prostej $l$ i punktu $A$, dokładnie jedna prosta przechodzi przez punkt $A$ i jest równoległa do $l$>
    Każda prosta na płaszczyźnie afinicznej ma tyle samo punktów.
    Jeśli pewna prosta ma $n$ punktów, to płaszczyzna ma dokładnie $n^2$ punktów.
    Istnieją płaszczyzny rzutowe o $4$, $9$, $16$ i $25$ punktach, ale nie istnieje taka, która miałaby $36$ punktów.
\end{proposition} % Hartshorne 71, 72

\subsubsection{Aksjomaty leżenia pomiędzy}
B1, B2, B3, B4 (Pascha).

I1-I3 + B1-B4 wynika stąd, że każda prosta ma nieskończenie wiele punktów.

\begin{definition}[odcinek]
    Niech $A$, $B$ będą punktami.
    Zbiór złożony z punktów $A$, $B$ oraz punktów, które leżą między nimi, nazywamy odcinkiem i oznaczamy $\overline {AB}$.
\end{definition} % Hartshorne 74

\begin{definition}[trójkąt]
    Niech $A$, $B$, $C$ będą punktami.
    Sumę odcinków $AB$, $BC$, $AC$ nazywamy trójkątem, wspomniane odcinki -- jego bokami, zaś punkty $A$, $B$ i $C$ -- wierzchołkami.
\end{definition} % Hartshorne 74

\begin{proposition}
    Niech $l$ będzie prostą.
    Wtedy zbiór punktów, które nie leżą na prostej $l$ można rozbić na dwa niepuste zbiory $S_1$, $S_2$ takie, że: dwa punkty, które nie leżą na prostej $l$, należą do tego samego zbioru ($S_1$ lub $S_2$) wtedy i tylko wtedy, gdy odcinek $AB$ nie przecina prostej $l$.
\end{proposition} % Hartshorne 74

Zbiory $S_1$, $S_2$ nazywamy stronami prostej $l$.
Podobnie punkt wyznacza na prostej dwa zbiory, które leżą po różnych stronach tego punktu.

\begin{definition}[półprosta]
    Niech $A$, $B$ będą punktami.
    Zbiór złożony z punktów $A$, $B$ oraz punktów, które leżą po tej samej stronie punktu $A$ na prostej $AB$ co punkt $B$, nazywamy półprostą i oznaczamy $NIE WIEM JAK AB$.
\end{definition} % Hartshorne 77

\begin{definition}[kąt]
    Sumę dwóch półprostych $AB$, $AC$, które nie leżą na jednej prostej, nazywamy kątem, zaś punkt $A$ wierzchołkiem tego kąta.
    Wnętrze kąta $\angle BACS$ składa się z tych punktów $D$ takich, że $D$ i $C$ leżą po tej samej stronie prostej $AB$ oraz $D$ i $B$ leżą po tej samej stronie prostej $AC$.
\end{definition} % Hartshorne 77

W myśl tej definicji, nie ma kąta zerowego ani półpełnego.
Wnętrze trójkąta $ABC$ to część wspólna wnętrz kątów $\angle ABC$, $\angle BCA$, $\angle CAB$; jest wypukłe i niepuste.

\subsubsection{Aksjomaty przystawania odcinków}
Aksjomaty C1-C3

\begin{definition}[okrąg]
    Niech $O$, $A$ będą dwoma różnymi punktami.
    Zbiór punktów $B$ takich, że odcinki $OA$ i $OB$ są przystające nazywamy okręgiem o środku $O$ oraz promieniu $OA$; okręgi często oznacza się literą $\Gamma$.
\end{definition} % Hartshorne 89

\begin{proposition}
    Każda prosta, która przechodzi przez środek, przecina okrąg w dwóch punktach.
    Okrąg składa się z nieskończenie wielu punktów.
\end{proposition}

(Nie jest jasne, ile środków może mieć okrąg, ale Hartshorne \cite[s. 89]{hartshorne2000} obiecuje pokazać póżniej, że tylko jeden.
Później ma miejsce na stronie 104).

\begin{definition}[styczna]
    Niech $\Gamma$ będzie okręgiem, zaś $l$ prostą, która przecina $\Gamma$ w dokładnie jednym punkcie $A$.
    Mówimy, że $l$ jest styczną do okręgu $\Gamma$ w punkcie $A$.
\end{definition}

Podobnie mówimy, że dwa okręgi są styczne, jeśli mają jeden punkt wspólny.

\subsubsection{Aksjomaty przystawania kątów}
C4-C6

Kąty przyległe, prosty.

\subsubsection{Płaszczyzna Hilberta}

. . .

%