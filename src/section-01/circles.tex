\subsection{Okręgi}

\begin{proposition}
    Niech $\Gamma$ będzie okręgiem o środku $O$ oraz promieniu $OA$.
    Wtedy prosta prostopadła do $OA$, która przechodzi przez $A$, jest styczną do okręgu, leżącą (poza punktem $A$) na zewnątrz okręgu $\Gamma$.
    Odwrotnie, każda prosta, która jest styczna w punkcie $A$ do okręgu $\Gamma$, musi być prostopadła do prostej $OA$.
\end{proposition} % Hartshorne 105

\begin{corollary}
    Przez każdy punkt okręgu przechodzi dokładnie jedna styczna do tego okręgu.
\end{corollary} % Hartshorne 105

\begin{corollary}
    Prosta, która nie jest styczna do okręgu i nie jest z nim rozłączna, musi przecinać go w dokładnie dwóch punktach.
\end{corollary} % Hartshorne 106

\begin{proposition}
    Niech $O_1, O_2, A$ będą trzema punktami.
    Następujące warunki są równoważne: punkty $A, O_1, O_2$ są współliniowe; okręgi o promieniach $O_1A$, $O_2A$ są styczne.
\end{proposition} % Hartshorne 105

\begin{corollary}
    Dwa okręgi, które nie są rozłączne i nie są styczne, mają dokładnie dwa punkty wspólne.
\end{corollary} % Hartshorne 106


Kąty środkowe, wpisane, dopisane.
Okręgi opisane i wpisane w czworokąt.

\begin{proposition}[okrąg opisany na czworokącie]
	Niech $A$, $B$, $C$, $D$ będą czterema punktami na płaszczyźnie takimi, że $A$ i $B$ leżą po tej samej stronie prostej $CD$.
	Wtedy następujące warunki są równoważne: punkty $A$, $B$, $C$, $D$ leżą na jednym okręgu; kąty $\angle DAC$ i $\angle DBC$ są sobie równe; suma dwóch przeciwległych kątów czworokąta $ABCD$ ma miarę kąta półpełnego.
\end{proposition}

\begin{proposition}
	Niech $\Gamma$ będzie okręgiem opisanym na czworokącie $ABCD$.
	Niech $\Gamma_1$, $\Gamma_2$, $\Gamma_3$, $\Gamma_4$ będą dowolnymi okręgami, które przechodzą przez $AB$, $BC$, $CD$, $DA$.
	Wtedy ich cztery nowe punkty przecięcia tworzą czworokąt cykliczny.
\end{proposition}


Styczna do okręgu, okrąg wpisany w kąt.
Okrąg wpisany w trójkąt, okręgi dopisane do trójkąta.
Warunki istnienia okręgu stycznego do czterech prostych.

\begin{proposition}[twierdzenie o siecznych i stycznych]
	Jeżeli...
\end{proposition}

Geometria koła i kątów, twierdzenie Apolloniusza (s. 22)

\subsubsection{Twierdzenie Miquela}
Twierdzenie Miquela
\loremipsum
% https://en.wikipedia.org/wiki/Miquel%27s_theorem
Hartshorne jako ćwiczenie \cite[s. 61]{hartshorne2000} pisze, że tym razem punkt Miquela został nazwany na cześć osoby, która go odkryła w 1838 roku.

Twierdzneie o prostej Wallace'a-Simsona.
Na UW: Zastosowanie: okrąg dziewięciu punktów, twierdzenie o prostej Simsona.

\begin{proposition}
	Niech $P$ będzie dowolnym punktem leżącym na okręgu opisanym na trójkącie $ABC$, zaś $D$, $E$ oraz $F$ rzutami punktu $P$ na proste zawierające boki trójkąta $ABC$.
	Wtedy punkty $D$, $E$ oraz $F$ są współliniowe.
\end{proposition}

Hartshorne jako ćwiczenie \cite[s. 61]{hartshorne2000} pisze, że istnienie prostej Simsona jako pierwszy dowiódł Wallace w 1799 roku.
