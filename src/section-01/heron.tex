%

\index{wzór!Herona|(}
Guzicki \cite[s. 165-168]{guzicki_2021} wyprowadza wzór Herona z twierdzenia Pitagorasa.
\index{twierdzenie!Pitagorasa}
Oryginalny dowód Herona był dość skomplikowany, Guzicki \cite[s. 168-169]{guzicki_2021} wspomina o znacznie prostszym dowodzie geometrycznym, pochodzącym od Eulera.
\index[persons]{Euler, Leonhard}%

\begin{proposition}
	Niech $ABC$ będzie trójkątem o obwodzie $2p$ oraz polu powierzchni $S$.
	Wtedy
	\begin{equation}
		S \le \frac{p^2}{3 \sqrt{3}}
	\end{equation}
\end{proposition}

Guzicki wyprowadza tę nierówność izoperymetryczną ze wzoru Herona oraz nierówności między średnią arytmetyczną i geometryczną.

\index{wzór!Herona|)}

% TODO: wzór Herona (Guzicki-6), Brahmagupty

%