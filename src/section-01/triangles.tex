\subsection{Trójkąty}
\subsubsection{Nie wiem gdzie}

\begin{proposition}
	\label{hartshorne_52}
    Niech $AB$ będzie odcinkiem.
	Istnieje wtedy trójkąt równoramienny, którego podstawą jest $AB$.
\end{proposition}

Powyższe stwierdzenie jest ciekawe, bo jest prawdziwe na płaszczyźnie Hilberta, tzn. jego prawdziwość nie zależy od aksjomatu Pascha.
(W geometrii nieeuklidesowej może nie istnieć trójkąt równoboczny o danej podstawie).


\begin{proposition}
	\label{hartshorne_52}
    Niech $ABC$ będzie trójkątem, zaś $D$ i $E$ środkami odcinków $AB$ i $AC$.
	Wtedy odcinek $DE$ jest równoległy do odcinka $BC$ i dwa razy krótszy od niego.
\end{proposition}
% Hartshorne s. 52

Tego stwierdzenia nie ma w Elementach Euklidesa, ale można wyprowadzić je z księgi I (I.29, I.26, I.34), jak wspomina Hartshorne \cite[s. 52. 53]{hartshorne2000}.

\begin{corollary}
	Niech $ABC$ będzie trójkątem, zaś punkty $D$, $E$ i $F$ środkami jego boków.
	Wtedy cztery małe trójkąty utworzone na bokach $DE$, $EF$, $FD$ są przystające do siebie.
\end{corollary}

Do tego wniosku potrzeba dodatkowo cechy przystawania bok-bok-bok (I.8).

\begin{proposition}
	\label{srodkowe_przecinaja_sie}
	Środkowe trójkąta przecinają się w jednym punkcie zwanym centroidem i dzielą w stosunku $2 : 1$ licząc od wierzchołków.
\end{proposition}

Hartshorne \cite[s. 53, 54]{hartshorne2000} wnioskuje powyższe z \ref{hartshorne_52}.

\begin{proposition}
	\label{wysokosci_przecinaja_sie}
	Wysokości trójkąta (proste prostopadłe do podstawy przechodzące przez wierzchołek nieleżący na niej) przecinają się w jednym punkcie zwanym ortocentrum.
\end{proposition}

Hartshorne \cite[s. 52, 54]{hartshorne2000} pisze, że ten oraz poprzedni fakt (\ref{wysokosci_przecinaja_sie}, \ref{srodkowe_przecinaja_sie}) były znane Archimedesowi.

\begin{proposition}[prosta Eulera]
	\label{prosta_eulera}
	Środek okręgu opisanego na trójkącie, centroid oraz ortocentrum leżą na jednej prostej, zwanej prostą Eulera.
\end{proposition}

Hartshorne \cite[s. 54, 55]{hartshorne2000}.

\begin{proposition}[okrąg dziewięciu punktów]
	\label{okrag_dziewieciu_punktow}
	W każdym trójkącie środki boków, spodki wysokości oraz środki odcinków łączących ortocentrum z wierzchołkami leżą na jednym okręgu.
\end{proposition}

Hartshorne \cite[s. 57]{hartshorne2000}.
Środek tego okręgu leży na prostej Eulera (Hartshorne jako ćwiczenie \cite[s. 60]{hartshorne2000}).

\begin{proposition}
	\label{orthic_triangle}
	Niech $ABC$ będzie trójkątem ostrokątnym, zaś $K$, $L$ oraz $M$ spodkami jego wysokości.
	Wtedy wysokości trójkąta $ABC$ są dwusiecznymi kątów trójkąta $KLM$.
\end{proposition}

Hartshorne \cite[s. 58]{hartshorne2000}.



\subsubsection{Symetralna i okrąg opisany}
Symetralna i okrąg opisany
\loremipsum

\subsubsection{Ortocentrum}
Ortocentrum.
\loremipsum

%

\subsubsection{Twierdzenie Pitagorasa}
Najważniejszym twierdzeniem dotyczącym trójkątów prostokątnych jest twierdzenie Pitagorasa oraz twierdzenie do niego odwrotne.
Piszą o~nim Guzicki \cite[s. 160]{guzicki_2021}.

\begin{theorem}[Pitagorasa, ok. 500 r. p.n.e.]
\index{twierdzenie!Pitagorasa}%
    Niech $ABC$ będzie trójkątem prostokątnym, w~którym kąt przy wierzchołku $C$ jest prosty.
    Wtedy
    \begin{equation}
        |BC|^2 + |AC|^2 = |AB|^2.
    \end{equation}
    Odwrotnie, jeśli $ABC$ jest trójkątem takim, że $|BC|^2 + |AC|^2 = |AB|^2$, to trójkąt ten jest prostokątny, zaś kąt przy wierzchołku $C$ jest prosty.
\end{theorem}

Chociaż współcześnie powyższe twierdzenie przypisujemy Pitagorasowi z~Samos, to nie wiemy dokładnie, kto i~kiedy odkrył je jako pierwszy.
\index[persons]{Pitagoras z Samos}%
Było powszechnie stosowane w~okresie Starego Babilonu (XX-XVI wiek p.n.e.), a~więc na długo przed narodzinami Pitagorasa; pojawia się też w indyjskich i~chińskich tekstach matematycznych.
Papirus Berlin 6619 spisany ok. 1800 roku p.n.e. na terenach państwa egipskiego zawiera zadanie, którego rozwiązaniem jest trójka $(6, 8, 10)$.

Już w~szkole podstawowej uczniowie poznają trójkąt prostokątny o bokach długości $3, 4, 5$ wraz~z~legendą, że podobno Egipcjanie używali tego trójkąta do wyznaczania w terenie kątów prostych.

% TODO: rysunek z Guzickiego, stron 160

\begin{proposition}
    Mają miejsce następujące równości:
    \begin{equation}
        h = \frac{ab}{c}, \quad
        p = \frac{b^2}{c}, \quad
        q = \frac{a^2}{c}, \quad
        h^2 = pq.
    \end{equation}
\end{proposition}

Dowód wykorzystujący podobieństwa trójkątów można znaleźć u~Guzickiego \cite[s. 160, 161]{guzicki_2021}.

Twierdzenie Pitagorasa znajduje zastosowanie także przy wyznaczaniu niektórych miejsc geometrycznych.

\begin{proposition}
    Dane są dwa różne punkty $A$ i $B$ na płaszczyźnie oraz liczba rzeczywista $c$ taka, że $2c > |AB|^2$.
    Miejscem geometrycznym punktów $P$ o własności $|AP|^2 + |BP|^2 = c$ jest okrąg o środku w środku odcinka $AB$ i promieniu $r = \frac 1 2 \sqrt{2c - |AB|^2}$.
\end{proposition}

\begin{proposition}
    Dane są dwa różne punkty $A$ i $B$ na płaszczyźnie oraz liczba rzeczywista $c$.
    Miejscem geometrycznym punktów $P$ o własności $|AP|^2 - |BP|^2 = c$ jest prosta prostopadła do prostej $AB$.
\end{proposition}

Patrz Guzicki \cite[s. 170-173]{guzicki_2021} (Guzicki wprowadza potem osie i środki potęgowe jak w~fakcie \ref{guzicki_6_11}, a następnie twierdzenie \ref{guzicki_6_13} (Carnota)).

%

% https://en.wikipedia.org/wiki/Pythagorean_theorem liczne dowody, wiek Pitagorasa
% https://en.wikipedia.org/wiki/Xuan_tu
% https://en.wikipedia.org/wiki/Spiral_of_Theodorus
% https://en.wikipedia.org/wiki/Garfield%27s_proof_of_the_Pythagorean_theorem

%

\index{wzór!Herona|(}
Guzicki \cite[s. 165-168]{guzicki_2021} wyprowadza wzór Herona z twierdzenia Pitagorasa.
\index{twierdzenie!Pitagorasa}
Oryginalny dowód Herona był dość skomplikowany, Guzicki \cite[s. 168-169]{guzicki_2021} wspomina o znacznie prostszym dowodzie geometrycznym, pochodzącym od Eulera.
\index[persons]{Euler, Leonhard}%

\begin{proposition}
	Niech $ABC$ będzie trójkątem o obwodzie $2p$ oraz polu powierzchni $S$.
	Wtedy
	\begin{equation}
		S \le \frac{p^2}{3 \sqrt{3}}
	\end{equation}
\end{proposition}

Guzicki wyprowadza tę nierówność izoperymetryczną ze wzoru Herona oraz nierówności między średnią arytmetyczną i geometryczną.

\index{wzór!Herona|)}

% TODO: wzór Herona (Guzicki-6), Brahmagupty

%

\loremipsum

\subsubsection{Problemy Fagnano i Fermata}
Problemy Fagnano i Fermata
\loremipsum