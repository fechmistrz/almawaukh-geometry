%

\section{Konstrukcje klasyczne}
\subsection{Proste}
% TODO: Hartshorne s. 103
\begin{problem}
    Dwusieczna kąta.
\end{problem}

\begin{problem}
    Środek odcinka.
\end{problem}

\begin{problem}
    Prosta prostopadła do prostej, przechodząca przez punkt.
\end{problem}

\begin{problem}
    Prosta równoległa do prostej, przechodząca przez punkt.
\end{problem}





\subsection{Trudniejsze}


\begin{problem}
    Dany jest odcinek $AB$ oraz punkt $P$ wewnątrz okręgu.
    Skonstruować cięciwę tego okręgu, która przechodzi przez punkt $P$ o długości takiej samej, jak odcinek $AB$.
\end{problem}
% Hartshorne s. 26

\begin{problem}
    Dany jest odcinek $AB$, inny odcinek o długości $d$ oraz kąt $\alpha$.
    Skonstruować trójkąt $ABC$ tak, by kąt przy wierzchołku $C$ miał miarę $\alpha$, zaś suma długości ramion tego kąta była równa $d$.
\end{problem}
% Hartshorne s. 26

\begin{problem}
    Dane są dwa okręgi takie, że żaden nie jest zawarty w drugim.
    Skonstruować styczną do obydwu okręgów.
\end{problem}
% Hartshorne s. 26

\begin{problem}
    Dany jest okrąg $\Gamma$ oraz jego środek $O$.
    Skonstruować trzy przystające okręgi, które są styczne do pozostałych dwóch oraz do $\Gamma$. \hfill \emph{(13 kroków)}. % Hartshorne s. 51
\end{problem}
% Hartshorne s. 26

\begin{problem}
    Dany jest okrąg $\Gamma$ oraz dwa punkty $A$ i $B$.
    Skonstrować punkt $C$ na okręgu $\Gamma$ tak, by odcinek łączący punkty przecięcia prostych $CA$, $CB$ z okręgiem $\Gamma$ był równoległy do odcinka $AB$.
\end{problem}
% Hartshorne s. 58-59

\begin{problem}
    Skonstruować trzy parami styczne okręgi, każdy o innym promieniu, których środki nie są współliniowe. \hfill \emph{(7 kroków)}. % Hartshorne s. 62
\end{problem}

\subsection{Uszkodzone przyrządy}
\begin{problem}[połamana linijka]
    Dane są dwa punkty $A$ i $B$ na płaszczyźnie, odległe od siebie o około trzy nible.
    Mając do dyspozycji fragment linijki o długości jednej nibli oraz sprawny cyrkiel, narysować odcinek $AB$.
\end{problem}
% Hartshorne s. 25

\begin{problem}[zardzewiały cyrkiel]
    Dane są dwa punkty $A$ i $B$ na płaszczyźnie, odległe od siebie o około pięć nibli.
    Mając do dyspozycji zardzewiały cyrkiel, którym można kreślić jedynie okręgi o promieniu dwóch nibli, skonstruować trójkąt równoboczny oparty o bok $AB$.
\end{problem}

\begin{problem}
    Dany jest punkt $A$ leżący na prostej $l$.
    Skonstruować prostą prostopadłą do $l$ przechodzącą przez $A$ przy użyciu linijki i zardzewiałego cyrkla.
\end{problem}
% Hartshorne s. 25

\begin{problem}
    Dany jest punkt $A$ leżący ponad cztery nible od prostej $l$.
    Skonstruować prostą prostopadłą do $l$ przechodzącą przez $A$ przy użyciu linijki i zardzewiałego cyrkla.
\end{problem}
% Hartshorne s. 25

\begin{problem}
    Dane są trzy niewspółliniowe punkty $A$, $B$ oraz $C$.
    Skonstrować punkt $D$ na prostej $AC$ tak, żeby odcinki $AD$ oraz $AB$ były równej długości przy użyciu linijki i zardzewiałego cyrkla.
\end{problem}
% Hartshorne s. 26

\begin{problem}
    Dany jest odcinek $AB$ o długości ponad dwóch nibli oraz prosta $l$, która nie przechodzi przez końce odcinka.
    Skonstrować punkt $C$ na prostej $l$ tak, żeby odcinki $AB$ oraz $AC$ były równej długości przy użyciu linijki i zardzewiałego cyrkla.
\end{problem}
% Hartshorne s. 26

\begin{problem}
    Czy wszystkie konstrukcje, które można wykonać cyrklem i linijką, można wykonać też zardzewiałym cyrklem i linijką?
\end{problem}
% Hartshorne s. 26

\subsection{Wyrocznia w Delfach}

\subsection{Wielokąty foremne}
GUZICKI-12 **wielokąty foremne** które są konstruowalne? (tw. wantzla itd.) konstrukcje przybliżone pięciokąta - durer i da vinci.
$n = 3$, $n = 4$, $n = 6$ (proste)

\begin{problem}
    Skonstrować trójkąt równoboczny wpisany w okrąg, którego środek nie jest znany. \hfill \emph{(7 kroków)}
\end{problem}

\begin{problem}
    Skonstrować kwadrat. \hfill \emph{(9 kroków)}
\end{problem}

\begin{problem}
    Skonstrować pięciokąt foremny.
\end{problem}

Piszą o tym Hartshorne \cite[s. 45-49]{hartshorne2000}.
Jeśli mamy zadany jeden z jego boków, konstrukcja wymaga 11 kroków. % Hartshorne s. 51



$n = 17$

$n = 7$ (niemożliwe), możliwe ze znaczoną linijką: Hartshorne rozdział 30/31

\subsection{Stożkowe}
przecięcie prostej z parabolą (hartshorne s. 247)

s. 278 Hartshorne: problem Alhazen, równokąty widziane z dwóch punktów na okręgu

\subsection{Apolloniusz}
GUZICKI-19 **zadanie konstrukcyjne apolloniusza** wykorzystuje twierdzenie menelaosa



%

W 1803 roku Malfatti \cite{malfatti_1803} zainspirowany pewnym praktycznym zagadnieniem (wycinanie walców z graniastosłupa) postawi następujący problem:
\index[persons]{Malfatti, Gian Francesco}%

\begin{problem}[zadanie Malfattiego]
	\label{malfatti_problem}
	\index{zadanie!Malfattiego}%
	Dany jest trójkąt $\triangle ABC$.
	Skonstruować takie trzy parami styczne okręgi $\Gamma_A, \Gamma_B, \Gamma_C$, że okrąg $\Gamma_A$ (odpowiednio: $\Gamma_B$, $\Gamma_C$) jest wpisany w~kąt $\angle A$ (odpowiednio: $\angle B$, $\angle C$).
\end{problem}

\begin{figure}[H]
\begin{center}
\begin{tikzpicture}[scale=.5]
\tkzDefPoints{0/0/A,10/2/B,6/7/C}
\tkzDefPoints{4.43012726/2.59439459/Oa}
\tkzDefCircle[R](Oa,1.67519375895) \tkzGetPoint{Oaa}
\tkzDrawCircle[line width=0.2mm](Oa,Oaa)

\tkzDefPoints{7.48168986/2.91734309/Ob}
\tkzDefCircle[R](Ob,1.39341015784) \tkzGetPoint{Obb}
\tkzDrawCircle[line width=0.2mm](Ob,Obb)

\tkzDefPoints{5.96721113/5.06490116/Oc}
\tkzDefCircle[R](Oc,1.23445046858) \tkzGetPoint{Occ}
\tkzDrawCircle[line width=0.2mm](Oc,Occ)

\tkzLabelPoint(A){$A$}
\tkzLabelPoint[anchor=center](Oa){$\Gamma_A$}
\tkzLabelPoint(B){$B$}
\tkzLabelPoint[anchor=center](Ob){$\Gamma_B$}
\tkzLabelPoint[above](C){$C$}
\tkzLabelPoint[anchor=center](Oc){$\Gamma_C$}
\tkzDrawPolygon[line width=0.3mm](A,B,C)
\end{tikzpicture}
\end{center}
\caption{Trzy okręgi Malfattiego}
\end{figure}

Problem będzie rozważany na długo przed Malfattim, zajmie się nim Ajima Naonobu\footnote{Matematyk japoński, przypisze się mu wprowadzenie rachunku różniczkowo-całkowego do matematyki japońskiej.} w~XVIII wieku, a~jeszcze wcześniej Gilio de Cecco da Montepulciano w~rękopisie z~1384 roku.
\index[persons]{Ajima, Naonobu}%
\index[persons]{de Cecco da Montepulciano, Gilio}%

Malfatti wyprowadzi co następuje.
Niech $p$ będzie połową obwodu trójkąta, $r$ będzie promieniem okręgu wpisanego w~ten trójkąt zaś $d_A$, $d_B$, $d_C$ odległościami wierzchołków $A, B, C$ od środka tego okręgu.
Wtedy promienie okręgów Malfattiego wyrażają się wzorami
\begin{align}
	r_A & = \frac r 2 \cdot {\frac {s-r+d_A-d_B-d_C}{p-a}}, \\
	r_B & = \frac r 2 \cdot {\frac {s-r+d_B-d_A-d_C}{p-b}}, \\
	r_C & = \frac r 2 \cdot {\frac {s-r+d_C-d_A-d_B}{p-c}}.
\end{align}

Prostą konstrukcję okręgów opartą na dwustycznych zawdzięczymy Steinerowi \cite{steiner_1826} w~1826 roku;
\index[persons]{Steiner, Jakob}%
inne rozwiązania podadzą Lehmus \cite{lehmus_1819}, Catalan \cite{catalan_1846}, Adams \cite{adams_1846}, Derousseau \cite{derousseau_1895}, Pampuch \cite{pampuch_1904}.
% TODO: po poprawie bibliografii, podać tu index persons

(O~problemie napiszą też Bogdańska, Neugebauer \cite[s. 102]{neugebauer_2018}).

Malfatti postawi tak naprawdę inny problem: znalezienia trzech rozłącznych kół zawartych w~trójkącie, których suma pól jest maksymalna i~błędnie założy, że opisane wyżej okręgi stanowią rozwiązanie.
Pomyłkę zauważą najpierw bez dowodu Lob, Richmond \cite{lob_richmond_1930} w~1930 roku: z trójkąta równobocznego można wyciąć zachłannie kolejno trzy koła, ich łączna powierzchnia jest większa od powierzchni kół znalezionych przez Malfattiego o 1\%.
\index[persons]{Richmond, ?}%
\index[persons]{Lob, ?}%
Howard Eves powtórzy to dla stromych trójkątów równoramiennych o bardzo wąskiej podstawie i dużej wysokości około 1946 roku.
\index[persons]{Eves, Howard}%
% https://en.wikipedia.org/w/index.php?title=Howard_Eves&diff=831382284&oldid=750910758
Goldberg \cite{goldberg_1967} wykaże, że domniemanie Malfattiego nie daje nigdy kół o maksymalnej łącznej powierzchni.
Ostatnie słowo należy zaś do Zalgallera, Losa \cite{zalgaller_los_1992}, którzy znajdą trzy koła rozwiązujące problem Malfattiego w dowolnym trójkącie.
% TODO: Goldberg M., On the original Malfatti problem, Math. Mag. 40 (1967), 241–247.
\index[persons]{Zalgaller, VA?}%
\index[persons]{Los, GA?}%
% TODO: Zalgaller V.A., Los’ G.A., Solution of the Malfatti problem, Ukrain. Geom. Sb. 35 (1992), 14–33 (ang. J. Math. Sci. 72 (1994), 3163–3177).
% TODO: po poprawie bibliografii, podać tu index persons
% TODO: Lob, H.; Richmond, H. W. (1930), "On the Solutions of Malfatti's Problem for a Triangle", Proceedings of the London Mathematical Society, 2nd ser., 30 (1): 287–304, doi:10.1112/plms/s2-30.1.287.

Kryształowa kula nie potrafi przewidzieć, kto oceni, czy algorytm zachłanny zawsze znajduje $n \ge 4$ rozłącznych kół w trójkącie o maksymalnej łącznej powierzchni.

(O więcej niż jednym okręgu wpisanym w trójkąt pisaliśmy w podpodsekcji \ref{sssection_6_7_9_circles}).


\color{red}

\begin{problem}[zadanie Napoleona]
	Podzielić dany okrąg (bez znanego środka) na cztery łuki równej miary korzystając z cyrkla, ale nie linijki.
\end{problem}

Nie wiadomo, czy Napoleon wymyślił albo rozwiązał przedstawione wyżej zadanie konstrukcyjne.
Rozwiązanie: \cite[s. 116]{neugebauer} z wykorzystaniem okręgów Torricelliego.
\index{okrąg Torricelliego}%



% TODO: rozwiązanie https://en.wikipedia.org/wiki/Napoleon%27s_problem

\color{black}

Konstrukcje od \ref{delta_2024_12_start} do \ref{delta_2024_12_end} opisane są w czasopiśmie Delta, w numerze grudniowym z 2024 roku.
% TODO: opisać wszystkie siedem konstrukcji

\begin{geoconstruction}
    \label{delta_2024_12_start}
    Znając pięć punktów okręgu $\omega$, skonstruować styczną do $\omega$ w jednym z tych punktów.
\end{geoconstruction}
% Niech tymi punktami będą A, B, C, D, E. Przecinamy AB i CD w P, AC i BE w Q oraz PQ i DE w R. Wówczas prosta AR jest szukaną styczną (rys. 1). Podkreślmy, że do przeprowadzenia powyższej konstrukcji nie potrzebowaliśmy mieć narysowanego całego okręgu ω – wystarczyło tylko pięć znajdujących się na nim punktów. Uzasadnienie poprawności wymaga znajomości twierdzenia Pascala (patrz Deltoid z ∆9 14).

\begin{geoconstruction}
    Znając pięć punktów okręgu $\omega$, dla danej prostej $l$ przechodzącej przez jeden z nich wyznaczyć drugi punkt przecięcia $l$ i $\omega$.
\end{geoconstruction}

% W przypadku problemów ze znalezieniem rozwiązania polecam poszukać go
% w ∆6 17. Jesteśmy już gotowi do znalezienia środka okręgu samą linijką, jeśli
% mamy do dyspozycji jeszcze jeden, przecinający go okrąg.
% Konstrukcja 3. 
% Oznaczmy te okręgi przez ω1 i ω2, a ich punkty przecięcia przez A i B.
% Korzystając z konstrukcji 1, konstruujemy styczną do ω1 w punkcie B
% i przecinamy z ω2 w C. Przez A rysujemy prostą, która przecina ω1 w D,
% a ω2 w E. Oznaczmy przez F drugi punkt przecięcia prostej BD z ω2. Na koniec
% niech P będzie przecięciem BC i EF, a Q przecięciem BE i CF. Wówczas
% ?EFB= ?BAD= 180◦
% −?DBA−?ADB= 180◦
% −?DBA−?ABC= ?CBF.
% Oznacza to, że EB= CF, a prosta PQ zawiera średnicę ω2 (rys. 2).
% Po wybraniu innej prostej przechodzącej przez A skonstruujemy inną średnicę,
% i w konsekwencji środek ω2.
% Czytelnik z pewnością sam bez problemu wymyśli konstrukcje środka okręgu
% przy zadanych dwóch okręgach stycznych, a także przy zadanych dwóch
% okręgach współśrodkowych.
% W kolejnych konstrukcjach przyda się kilka pojęć.

\begin{geoconstruction}
    Skonstruować środek jednego z dwóch okręgów mających dwa punkty wspólne.
\end{geoconstruction}

% Rozważmy okrąg ω i dowolny punkt P nieleżący na tym okręgu. Przez punkt P
% poprowadźmy dwie sieczne, które przecinają ω w A i B oraz C i D. Niech proste
% AD i BC przecinają się w Q, a AC i BD przecinają się w R. Prostą QRbędziemy
% nazywać biegunową punktu P względem okręgu ω (rys. 3). Zauważmy, że może
% ona być wyznaczona wyłącznie przy użyciu linijki, nawet jeśli okrąg ω dany jest
% tylko w pięciu punktach (w takim przypadku korzystamy z konstrukcji 2).
% Biegunowe mają liczne i użyteczne własności. Na przykład jeśli P leży na
% zewnątrz ω,to biegunowa P przechodzi przez punkty styczności prostych stycznych
% do ω przechodzących przez P. Stąd dla punktów leżących na okręgu przyjmujemy,
% że biegunową jest styczna w tym punkcie. Zatem, wyznaczając biegunową, możemy
% skonstruować styczną do okręgu przechodzącą przez punkt na nim nieleżący.
% Inną użyteczną własnością jest fakt, że każda sieczna okręgu ω
% przechodząca przez P przecina ω w takich punktach A, B oraz biegunową
% P w takim Q, że AB dzieli harmonicznie PQ, tzn. AP
% BP = AQ
% BQ. Czytelnik
% może spróbować wymyślić, jak podzielić harmonicznie odcinek przy
% użyciu wyłącznie linijki (podpowiedź: warto przypomnieć sobie
% twierdzenia Cevy i Menelaosa).
% Kolejnym przydatnym obiektem będzie pęk okręgów. Jest to rodzina
% okręgów, którą jednoznacznie wyznaczają dwa niewspółśrodkowe okręgi.
% Pęki okręgów mają taką własność, że jeśli dwa okręgi należące do pęku
% przecinają się w dwóch punktach, to każdy okrąg z tego pęku przechodzi
% przez te dwa punkty (rys. 4), jeśli są styczne, to wszystkie są do siebie
% styczne w tym samym punkcie, oraz jeśli się nie przecinają, to żadne
% dwa się nie przecinają (rys. 5). Na potrzeby tego artykułu potraktujmy
% pęki okręgów jako „czarną skrzynkę”, zainteresowanych szczegółami
% odsyłam do krótkiego tekstu w tym wydaniu Delty (s. 20), który jest
% im poświęcony.
% Zachodzi następujące twierdzenie:
% Twierdzenie. Biegunowe dowolnego punktu P względem okręgów
% należących do jednego pęku są współpękowe.
% Punkt ten będziemy nazywali biegunowo sprzężonym do punktu P względem
% odpowiedniego pęku. Ponieważ pęk jest wyznaczony przez dwa okręgi,
% możemy też mówić o dwóch punktach sprzężonych względem pary okręgów.
% Powyższe twierdzenie wykorzystamy w kolejnych konstrukcjach. Ponieważ
% linijka nie pozwala na narysowanie okręgu, przez wyrażenie „skonstruować
% okrąg” będziemy określać wyznaczenie dowolnie wielu jego punktów.
% Konstrukcja 4. Mając dane okręgi λ i µ oraz punkt A na zewnątrz jednego
% z nich, skonstruować okrąg przechodzący przez A oraz należący do pęku
% wyznaczanego przez te okręgi.
% Niech A leży na zewnątrz okręgu λ. Z punktu A skonstruujmy styczną do λ
% w punkcie B. Następnie niech C będzie punktem biegunowo sprzężonym do
% punktu B względem λ i µ (zauważmy, że leży na AB). Konstruujemy teraz taki
% punkt D, że AD dzieli harmonicznie BC. Punkt D jest drugim obok A punktem
% szukanego okręgu (rys. 6). Gdyby okazało się, że D= C= A (tzn. gdyby AB było
% styczne do konstruowanego okręgu), to na początku konstrukcji powinniśmy wziąć
% „drugą styczną” z A do λ. Całą procedurę możemy teraz powtórzyć, biorąc D jako
% punkt startowy (i oczywiście punkt styczności do λ różny od B).

% Konstrukcja 5. Mając dane okręgi λ i µ oraz punkt A leżący wewnątrz nich,
% skonstruować okrąg przechodzący przez A oraz należący do pęku wyznaczanego
% przez te okręgi.
% W tym przypadku wyznaczamy punkt B, biegunowo sprzężony do A. Punkt ten
% leży na zewnątrz okręgów λ i µ, zatem możemy skonstruować dowolną liczbę
% punktów okręgu β przechodzącego przez B i należącego do pęku wyznaczanego
% przez te dwa okręgi (konstrukcja 4). Punkt A leży na zewnątrz β. Pokażemy, jak
% wykorzystać ten „dziurkowany” okrąg do odtworzenia konstrukcji 4.
% Problematyczny jest tylko pierwszy krok, czyli konstrukcja stycznej do β.
% Aby ją wyznaczyć, postępujemy następująco. Niech C będzie różnym od B
% punktem okręgu β. Wyznaczmy punkt D przecięcia prostej AC z okręgiem β
% (korzystamy z konstrukcji 2). Dalej konstruujemy taki E na AC, że AE dzieli
% harmonicznie CD. Prosta BE jest biegunową punktu A względem β, więc jej
% drugi punkt przecięcia z β to taki punkt F (rys. 7), że AF jest styczna do β
% (ponownie skorzystaliśmy z konstrukcji 2). Teraz na AF możemy wyznaczyć
% drugi obok A punkt szukanego okręgu i powtórzyć procedurę, rozpoczynając od
% tego punktu.
% Konstrukcja 6. Skonstruować środek przynajmniej jednego z czterech okręgów,
% z których żadne trzy nie należą do jednego pęku.
% Oznaczmy dane okręgi przez κ, λ, µ, ν. Zakładamy, że żadne dwa z nich nie
% mają punktów wspólnych ani nie są współśrodkowe.
% Wybierzmy punkt A na κ. Konstruujemy okręgi α i β przechodzące przez A oraz
% należące do pęków wyznaczonych odpowiednio przez λ i µ oraz µ i ν. Następnie
% wybieramy taki punkt B na α, że skonstruowana styczna w B do α przecina
% okrąg κ. Niech C będzie tym punktem przecięcia. Niech ponadto D i F będą
% punktami biegunowo sprzężonymi do punktów odpowiednio B i C względem
% pęku wyznaczonego przez okręgi α i β (rys. 8).
% Zauważmy, że E taki, że CE dzieli harmonicznie BD, leży na
% okręgu γ należącym do pęku wyznaczonego przez α i β oraz
% przechodzącym przez C (rozważamy ten okrąg, ale go nie
% konstruujemy). Ponadto prosta CF jest styczna do γ. Czytelnik,
% γ
% analizując ponownie rysunek 2, przekona się, że mamy wystarczająco
% danych, aby zastosować konstrukcję 3 dla okręgów γ i κ i uzyskać
% E
% średnicę κ. Drugą średnicę konstruujemy, zaczynając od innego
% punktu A.
% Odnotujmy, że konstrukcję da się powtórzyć, jeśli jeden z okręgów
% (u nas okrąg µ) jest dany tylko w pięciu punktach. Wynika to
% z możliwości wykonania konstrukcji, gdy okrąg µ jest dany tylko
% w 5 punktach, co z kolei jest konsekwencją poczynionej wcześniej
% uwagi o konstruowalności biegunowych.

\begin{geoconstruction}
    \label{delta_2024_12_end}
    Skonstruować środek przynajmniej jednego z trzech okręgów nienależących do jednego pęku.
\end{geoconstruction}
% Oznaczmy te okręgi przez κ, λ, µ. Wybieramy punkt A na κ i konstruujemy
% okrąg α należący do pęku wyznaczanego przez okręgi λ i µ oraz przechodzący
% przez A. Na okręgach κ i α wybieramy odpowiednio punkty B i C. Następnie
% prowadzimy dowolną prostą przez A i oznaczamy jej punkty przecięcia z κ
% i α przez P i Q, odpowiednio. Zauważmy, że jeśli prosta PQ będzie obracać
% się wokół punktu A, to punkt przecięcia R prostych PB i QC będzie zakreślał
% okrąg (jest to okrąg opisany na trójkącie, którego wierzchołkami są punkty B,
% C i różny od A punkt przecięcia α i κ). Oznaczmy go przez ν. Rysując zatem
% kolejne położenia prostej PQ, będziemy mogli konstruować kolejne punkty
% okręgu ν (rys. 9). Żadne trzy spośród κ, λ, µ, ν nie należą do jednego pęku.
% Stąd po skonstruowaniu pięciu punktów ν możemy powtórzyć konstrukcję 6.
% Czytelnikowi zastanawiającemu się, co z przypadkiem rozłącznych okręgów
% należących do jednego pęku, odpowiem, że wówczas środka okręgu nie da się
% skonstruować. Omówienie tego zagadnienia byłoby jednak zbyt długie, aby
% można je było zawrzeć w tym artykule.


%