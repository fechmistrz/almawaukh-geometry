%

\label{subsection_erdos_mordell}
Erdős w 1935 roku postawił problem dowodu tej nierówności; dowód przedstawili dwa lata później Mordell i D. F. Barrow (1937), choć nie był on zbyt elementarny. Później znaleziono prostsze dowody: Kazarinoff (1957), Bankoff (1958) oraz Alsina i Nelsen (2007).
% TODO: https://en.wikipedia.org/wiki/Erdős–Mordell_inequality#CITEREFErdős1935

\begin{theorem}[nierówność Erdősa-Mordella]
    Niech ..., wtedy ....
\end{theorem}

Twierdzenie podaje w formie ćwiczenia Coxeter \cite[s. 9]{coxeter_1991}

Wzmocnieniem nierówności (...) jest nierówność Barrowa:

% TODO: https://en.wikipedia.org/wiki/Barrow%27s_inequality

\begin{theorem}[nierówność Barrowa]
    Niech ..., wtedy ....
\end{theorem}

Nazwa ,,nierówność Barrowa'' jest używana co najmniej od 1961 roku; nie wiemy, co się stało w tym roku.
Dowód Barrowa został opublikowany w 1937 roku.
% TODO: Erdős, Paul; Mordell, L. J.; Barrow, David F. (1937), "Solution to problem 3740", American Mathematical Monthly, 44 (4): 252–254, doi:10.2307/2300713, JSTOR 2300713.

%