%

\label{subsection_erdos_mordell}
Erdős w 1935 roku postawi problem dowodu tej nierówności; dowód przedstawią dwa lata później Mordell i D. F. Barrow (1937), choć nie będzie on zbyt elementarny.
Później znajdzie się prostsze dowody: Kazarinoff (1957), Bankoff (1958) oraz Alsina i Nelsen (2007).
% TODO: https://en.wikipedia.org/wiki/Erdős–Mordell_inequality#CITEREFErdős1935

\begin{theorem}[nierówność Erdősa-Mordella]
    Niech $P$ będzie punktem wewnątrz trójkąta $\triangle ABC$, zaś $A_p, B_p, C_p$ spodkami punktu $P$ na boki trójkąta jak na rysunku \ref{erdos_mordell_barrowa}.
    Wtedy
    \begin{equation}
        |PA| + |PB| + |PC| \ge 2 (|PA_p| + |PB_p| + |PC_p|).
    \end{equation}
\end{theorem}


\begin{figure}[H] \centering
\begin{minipage}[b]{.45\linewidth}
\begin{center}
    \begin{comment}
    \begin{tikzpicture}[scale=.4]
    \tkzDefPoint(0, 0){A}
    \tkzDefPoint(10, 2){B}
    \tkzDefPoint(6, 7){C}
    \tkzDefPoint(5, 3){P}
    \tkzLabelPoint[below left](A){$A$}
    \tkzLabelPoint[below right](B){$B$}
    \tkzLabelPoint[above](C){$C$}
    \tkzLabelPoint[below left](P){$P$}
    \tkzDefPointsBy[projection=onto A--B](P){Pc}
    \tkzDefPointsBy[projection=onto B--C](P){Pa}
    \tkzDefPointsBy[projection=onto C--A](P){Pb}
    \tkzLabelPoint[above right](Pa){$A_p$}
    \tkzLabelPoint[above left](Pb){$B_p$}
    \tkzLabelPoint[below](Pc){$C_p$}

    \tkzDrawSegments[line width=0.2mm,dashed](P,Pa P,Pb P,Pc)
    \tkzDrawPolygon[line width=0.3mm](A,B,C)
    \tkzMarkRightAngles[size=0.5](P,Pa,C P,Pb,A P,Pc,B)
    \tkzDrawPoints[size=3,color=black,fill=black!50](A,B,C,P,Pc,Pb,Pa)
\end{tikzpicture}
\end{comment}
    \end{center}
    \subcaption{nierówność Erdősa-Mordella}
    \label{erdos_mordell_barrowa}
\end{minipage}
%
\begin{minipage}[b]{.45\linewidth}
\begin{center}\begin{comment}
    \begin{tikzpicture}[scale=.4]
    \tkzDefPoint(0, 0){A}
    \tkzDefPoint(10, 2){B}
    \tkzDefPoint(6, 7){C}
    \tkzDefPoint(5, 3){P}

    \tkzDefLine[bisector](A,P,B) \tkzGetPoint{prePc}
    \tkzInterLL(P,prePc)(A,B) \tkzGetPoint{Pc}
    \tkzDefLine[bisector](B,P,C) \tkzGetPoint{prePa}
    \tkzInterLL(P,prePa)(B,C) \tkzGetPoint{Pa}
    \tkzDefLine[bisector](C,P,A) \tkzGetPoint{prePb}
    \tkzInterLL(P,prePb)(C,A) \tkzGetPoint{Pb}

    \tkzLabelPoint[below left](A){$A$}
    \tkzLabelPoint[below right](B){$B$}
    \tkzLabelPoint[above](C){$C$}
    %\tkzLabelPoint[below left](P){$P$}
    \tkzLabelPoint[above right](Pa){$A_p$}
    \tkzLabelPoint[above left](Pb){$B_p$}
    \tkzLabelPoint[below](Pc){$C_p$}

    \tkzMarkAngle[arc=lll,size=1.2,mark=|||](A,P,Pc)
    \tkzMarkAngle[arc=lll,size=1.2,mark=|||](Pc,P,B)
    \tkzMarkAngle[arc=ll,size=1.2,mark=||](B,P,Pa)
    \tkzMarkAngle[arc=ll,size=1.2,mark=||](Pa,P,C)
    \tkzMarkAngle[arc=l,size=1.2,mark=|](C,P,Pb)
    \tkzMarkAngle[arc=l,size=1.2,mark=|](Pb,P,A)

    \tkzDrawSegments[line width=0.2mm](P,A P,B P,C)
    \tkzDrawSegments[line width=0.2mm,dashed](P,Pa P,Pb P,Pc)
    \tkzDrawPolygon[line width=0.3mm](A,B,C)
    \tkzDrawPoints[size=3,color=black,fill=black!50](A,B,C,P,Pc,Pb,Pa)
\end{tikzpicture}
\end{comment}
    \end{center}
    \subcaption{nierówność Barrowa}
    \label{erdos_mordell_barrowb}
\end{minipage}
\caption{}
\end{figure}

Twierdzenie poda w formie ćwiczenia Coxeter \cite[s. 9]{coxeter_1991}, Audin z licznymi wskazówkami \cite[s. 102]{audin_2003}.

Wzmocnieniem nierówności Erdősa-Mordella będzie nierówność Barrowa:

% TODO: https://en.wikipedia.org/wiki/Barrow%27s_inequality

\begin{theorem}[nierówność Barrowa]
    Niech $P$ będzie punktem wewnątrz trójkąta $\triangle ABC$, zaś $A_p$, $B_p$, $C_p$ punktami przecięć dwusiecznych trzech kątów wyznaczanych przez $P$ i pary wierzchołków trójkąta; tak jak na rysunku \ref{erdos_mordell_barrowb}.
    Wtedy
    \begin{equation}
        |PA| + |PB| + |PC| \ge 2 (|PA_p| + |PB_p| + |PC_p|).
    \end{equation}
\end{theorem}

Dowód Barrowa zostanie opublikowany w 1937 roku, ale nazwa ,,nierówność Barrowa'' będzie używana dopiero od 1961 roku; nie wiemy, co się wtedy stanie.
% TODO: Erdős, Paul; Mordell, L. J.; Barrow, David F. (1937), "Solution to problem 3740", American Mathematical Monthly, 44 (4): 252–254, doi:10.2307/2300713, JSTOR 2300713.

%