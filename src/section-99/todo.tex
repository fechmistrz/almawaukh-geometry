

\subsection{Inwersje (UW-2)}
\begin{enumerate}
	\item Obrazy inwersyjne okręgów i prostych, konforemność inwersji, okręgi stałe inwersji, okręgi prostopadłe
	\item zmiana odległości przy inwersji, zmiana promienia okręgu przy inwersji,
	\item twierdzenie Ptolemeusza,
	\item łańcuchy Steinera
	\item formuła Kartezjusza
	\item formuła Fussa dla czworokątów,
	\item twierdzenie Feuerbacha.
\end{enumerate}

Droz-Farny: proste przechodzą przez ortocentrum trójkąta i są prostopadłe, wtedy środki odcinków leżą na jednej prostej.



\subsection{Guzicki}
\begin{enumerate}
	\item Złoty podział i pięciokąt.
	\item przekątne w wielokącie, tw. Heinekena % n nieparzyste -> w n-kącie foremnym żadne trzy przekątne nie przecinają się -> https://arxiv.org/pdf/math/9508209v3 ... In the 1960s, Heineken [6] gave a delightful argument which generalized this to all odd n,
\end{enumerate}

\subsection{Starocie}
Okrąg siedmiu punktów. % https://mathworld.wolfram.com/BrocardCircle.html ?
% https://www.cut-the-knot.org/Curriculum/Geometry/HeronsProblem.shtml
% This one is a basic optimization problem. It's quite famous, being discussed in Heron's Catoptrica (On Mirrors from the Greek word Katoptron Catoptron = Mirror) that, in all likelihood, saw the light of day some 2000 years ago.
% https://en.wikipedia.org/wiki/Spiral_of_Theodorus

gnomon % https://en.wikipedia.org/wiki/Theorem_of_the_gnomon


%