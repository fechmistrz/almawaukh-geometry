

\subsection{Twierdzenie Sylvestera-Gallaia}
\begin{theorem}[Sylvestera-Gallaia]
	Dla każdego skończonego zbioru punktów na płaszczyźnie istnieje prosta, która przechodzi przez dokładnie dwa albo wszystkie punkty.
\end{theorem}

Mamy wrażenie, że zaczęło się w 1893 roku, kiedy James Sylvester postawił problem.
Być może zainspirowała go konfiguracją Hessego\footnote{Konfiguracja Hessego to 12 prostych przez 9 punktów na zespolonej płaszczyźnie rzutowej, gdzie każdy punkt leży na 4 prostych, a każda prosta przechodzi przez 3 punkty}.
Herbert Woodall szybko zaproponował rozwiązanie, gdzie równie szybko wychwycono usterkę.
Dopiero w 1941 roku Eberhard Melchior udowodnił trochę mocniejsze stwierdzenie niż rzutowy dual ówczesnej hipotezy (że prostych przez dokładnie dwa punkty jest co najmniej trzy).
Nieświadomy tego, Paul ErdErdős postawił hipotezę na nowo w~1943 roku, a Tibor Gallai w 1944 roku dodał swój dowód (ponownie wykorzystując elementy geometrii rzutowej).
Wraz z upływem czasu pojawiały się inne, ciekawe rozumowania.
Na przykład Leroy Kelly wykorzystał własności metryki, co oburzyło Harolda Coxetera i skłoniło go do opublikowania kolejnego dowodu, korzystającego jedynie z aksjomatów geometrii uporządkowania.
(Aigner, Ziegler uważają dowód Kelly'ego za najlepszy).

Niech $t_2(n)$ oznacza minimalną liczbę prostych przez dwa punkty w dowolnym ułożeniu $n$ punktów.
Melchior pokazał, że $t_2(n) \ge 3$.
Wynik sukcesywnie poprawiano:
de Bruijn \cite{debruijn_1948} zapytał, czy $t_2(n)$ dąży do nieskończoności,
Theodore Motzkin \cite{motzkin_1951} udzielił twierdzącej odpowiedz, bo $t_2(n) \ge \sqrt{n}$.
Potem Gabriel Dirac \cite{dirac_1951} przypuścił, że $t_2(n) \ge \lfloor n/2\rfloor$, co nie zostawia wiele miejsca na poprawki, bo dla parzystych $n \ge 6$ zachodzi $t_2(n) \le n/2$, jak pokazał pomysłową konstrukcją Károly Böröczky.
Dla nieparzystych $n$ wiemy tylko, że ten kres jest realizowany dla $n = 7$ (Kelly, Moser \cite{kelly_1958} w 1958) i $n = 13$ (Crowe, McKee \cite{mckee_1968} w 1968).
Najnowszy wynik, o jakim nam wiadomo, to Csimy, Sawyera \cite{csima_1993}: że $t_2(n) \ge \lceil 6n/13 \rceil$.

\subsection{Inwersje (UW-2)}
\begin{enumerate}
	\item Obrazy inwersyjne okręgów i prostych, konforemność inwersji, okręgi stałe inwersji, okręgi prostopadłe
	\item zmiana odległości przy inwersji, zmiana promienia okręgu przy inwersji,
	\item twierdzenie Ptolemeusza,
	\item łańcuchy Steinera
	\item formuła Kartezjusza
	\item formuła Fussa dla czworokątów,
	\item twierdzenie Feuerbacha.
\end{enumerate}

\subsection{Stożkowe (UW-2)}
\begin{enumerate}
	\item Ogniska elipsy i hiperboli, ognisko, kierownica i mimośród stożkowych, asymptoty hiperboli, konstrukcja stycznej do stożkowej, rzuty ustalonego ogniska na styczne, własności izogonalne stożkowych, równania kanoniczne stożkowych, elipsa jako przekrój walca.
	\item Ognisko, kierownica i mimośród stożkowej na przekroju stożka.
	\item Przekroje stożków ze sferami wpisanymi.
	\item Równanie ogólne stożkowej w układzie współrzędnych, duży i mały wyznacznik.
	\item Równania stożkowych we współrzędnych biegunowych.
\end{enumerate}

Neugebauer 262: w każdy właściwy czworobok zupełny da się wpisać dokładnie jedną parabolę, jej ogniskiem jest punkt Miquela czworoboku.
Jemieljanow: punkt Miquela właściwego czworoboku zupełnego leży na okręgu dziewięciu punktów trójkąta przekątnego tego czworoboku.
Droz-Farny: proste przechodzą przez ortocentrum trójkąta i są prostopadłe, wtedy środki odcinków leżą na jednej prostej.

\subsection{Przekształcenia afiniczne (UW-2)}
\begin{enumerate}
	\item Grupa przekształceń afinicznych od strony geometrycznej: powinowactwa osiowe, rozkład przekształcenia afinicznego na podobieństwo i powinowactwo osiowe, kierunki główne przekształcenia afinicznego.
	\item niezmienniczość stosunku pól przy przekształceniu afinicznym
	\item obraz okręgu przy przekształceniu afinicznym
\end{enumerate}

\subsection{UW-3}
\begin{enumerate}
	\item zna pojęcie płaszczyzny rzutowej rzeczywistej (równoważne sformułowania), dwustosunku, definicję przekształceń rzutowych łańcuchów, pęków, stożkowych, pęków stycznych do stożkowych. 
	\item Rozumie, czym są stożkowe w ujęciu rzutowym, zna typy stożkowych.
	\item Zna i potrafi stosować twierdzenia Steinera i Braikenridge'a-Maclaurina.
	\item Wie w jaki sposób określa się rzutowo ogniska i kierownice stożkowych.
\end{enumerate}

\subsection{Guzicki}
\begin{enumerate}
	\item Złoty podział i pięciokąt.
	\item Zagadnienie izoperymetryczne (6)
	\item nierówności geometryczne: stosunek sumy środkowych do obwodu leży między 3/4 i 1 (s. 355), $s <= p^2 / 3 \sqrt 3$ - przypomnienie nierówności izoperymetrycznej. nierówność eulera (R >= 2r), Mitrinovica, Leibniza, Weitzenbocka (s. 362). Twierdzenie Eulera: $d^2 = R^2 - 2Rr$. nierówność Erdosa-Mordella: P leży wewnątrz trójkąta, K L M to rzuty na boki. Wtedy PA + PB + PC >= 2 (PK + PL + PM). Mikołaj z Kuzy: $\sin x / x < (2 + \cos x) / 3$. Snellius-Huygens: $2 \sin x + \tan x > 3x$.
	\item przekątne w wielokącie, tw. Heinekena % n nieparzyste -> w n-kącie foremnym żadne trzy przekątne nie przecinają się -> https://arxiv.org/pdf/math/9508209v3 ... In the 1960s, Heineken [6] gave a delightful argument which generalized this to all odd n,
\end{enumerate}

\subsection{Starocie}
Twierdzenie Chasles'a: każda izometria płaszczyzny jest złożeniem co najwyżej trzech symetrii osiowych.
Symetria osiowa z poślizgiem.
Słowo Banacha.
Klasyfikacja podobieństw.
Okrąg siedmiu punktów. % https://mathworld.wolfram.com/BrocardCircle.html ?
Przekształcenia afiniczne i rzutowe.
% https://www.cut-the-knot.org/Curriculum/Geometry/HeronsProblem.shtml
% This one is a basic optimization problem. It's quite famous, being discussed in Heron's Catoptrica (On Mirrors from the Greek word Katoptron Catoptron = Mirror) that, in all likelihood, saw the light of day some 2000 years ago.
Pitagorasa % https://en.wikipedia.org/wiki/Pythagorean_theorem
% https://en.wikipedia.org/wiki/Spiral_of_Theodorus

gnomon % https://en.wikipedia.org/wiki/Theorem_of_the_gnomon

Czwarty aksjomat uporządkowania znalazł Moritz Pasch \cite{pasch_1882} w 1882 roku.
\index[persons]{Pasch, Moritz}
\index{aksjomat!Pascha}

\subsection{Zadania}
\textbf{Zadanie} (Guzicki, s. 304).
Na bokach $AB$, $BC$, $CD$ i $DA$ czworokąta wypukłego $ABCD$ zbudowano, na zewnątrz czworokąta, kwadraty $ABFE$, $BCHG$, $CDJI$ i $DALK$.
Punkty $P$, $Q$, $R$ i $S$ są odpowiednio środkami kwadratów $ABFE$, $BCHG$, $CDJI$ i $DALK$.
Udowodnij, że odcinki $PR$ i $QS$ są równej długości oraz wzajemnie prostopadłe.

\textbf{Zadanie} (Guzicki, s. 306).
(XLIV OM, zadanie 5/I).
Dana jest półpłaszczyzna oraz punkty $A$ i $C$ na jej krawędzi.
Dla każdego punktu $B$ tej półpłaszczyzny rozważamy kwadraty $ABKL$ i $BCMN$ leżące na zewnątrz trójkąta $ABC$.
Wyznaczają one odpowiadającą punktowi $B$ prostą $LM$.
Udowodnij, że wszystkie proste odpowiadające różnym położeniom punktu $B$ przechodzą przez jeden punkt.

\textbf{Zadanie} (Guzicki, s. 306).
Na bokach $AB$ i $AC$ trójkąta $ABC$ zbudowano, po jego zewnętrznej stronie, kwadraty $ABDE$ i $ACFG$.
Punkty $M$ i $N$ są odpowiednio środkami odcinków $DG$ i $EF$.
Wyznacz możliwe wartości wyrażenia $MN / BC$.

\textbf{Zadanie} (Guzicki, s. 307)
(TWIERDZENIE NAPOLEONA)
Na bokach $AB$, $BC$ i $CA$ trójkąta $ABC$ zbudowano, na zewnątrz trójkąta, trójkąty równoboczne $ABF$, $BCD$ i $CAE$.
Udowodnij, że środki tych trójkątów równobocznych są wierzchołkami trójkąta równobocznego.

\textbf{Zadanie} (Guzicki, s. 308)
Na bokach $AB$, $BC$ i $CA$ trójkąta $ABC$ wybrano odpowiednio punkty $D$, $E$ i $F$ tak, że $AD : DB = BE : EC = CF : FA$.
Udowodnij, że jeśli trójkąt $DEF$ jest równoboczny, to trójkąt $ABC$ też jest równoboczny.

\textbf{Zadanie} (Guzicki, s. 310)
(XLV OM, zadanie 7/I)
Na zewnątrz czworokąta wypukłego $ABCD$ budujemy trójkąty podobone $APB$, $BQC$, $CRD$, $DSA$ w ten sposób, że kąty $PAB, QBC, RCD, SDA$ są sobie równe i że kąty $PBA, QCB, RDS, SAD$ też są sobie równe.
Udowodnij, że jeśli czworokąt PQRS jest równoległobokiem, to czworokąt $ABCD$ też jest równoległobokiem.

%