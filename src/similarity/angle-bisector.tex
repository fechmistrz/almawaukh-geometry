% Coxeter s. 88 

\begin{proposition}[twierdzenie o dwusiecznej]
	Dany jest trójkąt $\triangle ABC$.
	Odcinek $CF$, gdzie $F$ leży na odcinku $AB$, jest dwusieczną kąta przy wierzchołku $C$ wtedy i tylko wtedy, gdy zachodzi równość:
	\begin{equation}
		\frac{|AF|}{|BF|} = \frac{|AC|}{|BC|}.
	\end{equation}
	\index{twierdzenie!o dwusiecznej}
\end{proposition}

To jest (VI.3) w Elementach Euklidesa.
Dowód korzysta z podobieństwa trójkątów, twierdzenia sinusów albo własności pola, patrz
\index{twierdzenie!sinusów}%
Guzicki \cite[s. 120]{guzicki_2021} (siedem różnych dowodów);
Bogdańska, Neugebauer \cite[s. 73]{neugebauer_2018},
Audin \cite[s. 102]{audin_2003} w formie ćwiczenia;
patrz też \cite[s. 11]{komisarczyk_2024}.

\begin{proposition} % Guzicki s. 126
	Punkt $P$ znajdujący się wewnątrz kąta wypukłego leży na dwusiecznej tego kąta wtedy i tylko wtedy, gdy odległości tego punktu od ramion kąta są równe.
\end{proposition}

% czemu to jest tu? czemu nie przenieść do trójkątów?
%

Wnioskiem z twierdzenia o dwusiecznej jest:

\begin{theorem}[Steinera-Lehmusa]
    \index{twierdzenie!Steinera-Lehmusa}%
    \label{theorem_steiner_lehmus}%
	Jeżeli dwie dwusieczne trójkąta są równej długości, to trójkąt ten jest równoramienny.
\end{theorem}

Po raz pierwszy wspomniał o nim Christian Lehmus w liście z 1840 roku do Charlesa Sturma, gdzie poprosił o czysto geometryczny dowód.
\index[persons]{Lehmus, Christian}%
\index[persons]{Sturm, Charles}%
Sturm przekazał prośbę do innych matematyków, jedną z pierwszych osób, która uporała się z problemem, był Jakob Steiner.
\index[persons]{Steiner, Jakob}%
Większość znanych dowodów przeprowadza się nie wprost: jeśli trójkąt nie jest równoramienny, to ma dwusieczne różnej długości.
Dowód można znaleźć u Hartshorne'a \cite[s. 11]{hartshorne2000}; Bogdańskiej, Neugebauera \cite[s. 74]{neugebauer_2018}.
Coxeter \cite[s. 32]{coxeter_1967} podaje je w formie ćwiczenia (po tym jak wcześniej \cite[s. 26, 33]{coxeter_1967} poprosi o~dowód tego samego dla dwusiecznej zamienionej na środkową lub wysokość, co znacząco obniża poziom trudności).
% https://www.algebra.com/algebra/homework/word/geometry/Medians-in-an-isosceles-triangle.lesson
Eves \cite[s. 19, 58]{eves1_1972} zachęca do dowodu z dopiskiem, że jest trudny. 

%

% TODO: https://zadania.info/d801/3107027

Twierdzenie o symetralnej % TODO czy to jest oficjalna nazwa
\index{twierdzenie!o symetralnej}
można wysłowić tak: miejsce geometryczne punktów $X$, dla których $|AX|/|BX| = 1$, jest prostą.
Około dwusetnego roku przed naszą erą Apoloniusz z Pergi udowodnił piękne uogólnienie tego faktu:

\begin{definition}[okrąg Apoloniusza] % TODO: Guzicki s. 129
	Dane są dwa różne punkty $A$, $B$ oraz liczba dodatnia $\lambda \neq 1$.
	Wtedy zbiór punktów 
	\begin{equation}
		\left\{X : \frac{|AX|}{|BX|} = \lambda \right\}
	\end{equation}
	jest okręgiem o środku na prostej $AB$ i promieniu równym
	\begin{equation}
		R = \frac{\lambda}{|\lambda^2 - 1|} \cdot |AB|.
	\end{equation}
	\index{okrąg!Apoloniusza}
\end{definition}

Coxeter \cite[s. 104, 105]{coxeter_1967} pisze, że ten okrąg odbija inwersyjnie punkty $A, B$ (a dla $\lambda = 1$ dostajemy symetralną odcinka).

To jest jeden z pięciu okręgów Apoloniusza, oprócz tego mamy dwie rodziny wzajemnie prostopadłych okręgów, okręgi Apoloniusza trójkąta (pomocne w znajdowaniu punktów izodynamicznych oraz prostej Lemoine'a), okrąg z problemu Apoloniusza styczny do trzech danych oraz fraktal zwany po angielsku uszczelką -- \emph{,,Apollonian gasket''}.
% https://en.wikipedia.org/wiki/Circles_of_Apollonius
\todofoot{Loksodromiczny ciąg Coxetera okręgów} % https://en.wikipedia.org/wiki/Coxeter%27s_loxodromic_sequence_of_tangent_circles

Piszą o nim Bogdańska, Neugebauer \cite[s. 74]{neugebauer_2018}