%

\section{Jednokładność}
Podobieństwo figur, trójkątów (cechy), stosunek pól figur podobnych.


\section{Okręgi ortogonalne, pęki okręgów.}
\index{okrąg!ortogonalny}
% \index{pęk okręgów}
Wie czym są pęki okręgów, zna ich podstawowe własności i potrafi stosować w konfiguracjach spokrewnionych z twierdzeniem Ponceleta.   
\index{twierdzenie!Ponceleta}

O pękach okręgów pisze Delta $\Delta_{24}^{12}$.

% T2.19 tutaj

% Poncelet przeniesiony do czworokątów dwuśrodkowych

\index{trygonometria|(}
\section{Trygonometria}
\subsection{Trygonometria.} Lorem ipsum dolor sit amet, consectetur adipiscing elit, sed do eiusmod tempor incididunt ut labore et dolore magna aliqua. Ut enim ad minim veniam, quis nostrud exercitation ullamco laboris nisi ut aliquip ex ea commodo consequat. Duis aute irure dolor in reprehenderit in voluptate velit esse cillum dolore eu fugiat nulla pariatur. Excepteur sint occaecat cupidatat non proident, sunt in culpa qui officia deserunt mollit anim id est laborum.
\subsubsection{Prawo sinusów}
Lorem ipsum dolor sit amet, consectetur adipiscing elit, sed do eiusmod tempor incididunt ut labore et dolore magna aliqua. Ut enim ad minim veniam, quis nostrud exercitation ullamco laboris nisi ut aliquip ex ea commodo consequat. Duis aute irure dolor in reprehenderit in voluptate velit esse cillum dolore eu fugiat nulla pariatur. Excepteur sint occaecat cupidatat non proident, sunt in culpa qui officia deserunt mollit anim id est laborum.
$$\frac{a}{\sin \alpha} = \frac{b}{\sin \beta} = \frac{c}{\sin \gamma} = 2R$$
% https://en.wikipedia.org/wiki/Law_of_sines

\subsubsection{Prawo cosinusów}
Lorem ipsum dolor sit amet, consectetur adipiscing elit, sed do eiusmod tempor incididunt ut labore et dolore magna aliqua. Ut enim ad minim veniam, quis nostrud exercitation ullamco laboris nisi ut aliquip ex ea commodo consequat. Duis aute irure dolor in reprehenderit in voluptate velit esse cillum dolore eu fugiat nulla pariatur. Excepteur sint occaecat cupidatat non proident, sunt in culpa qui officia deserunt mollit anim id est laborum.
$$c^2 = a^2 + b^2 - 2ab \cos \gamma$$
% https://en.wikipedia.org/wiki/Law_of_cosines

\textbf{Twierdzenie Stewarta}

\textbf{Wzór Brahmagupty}

\textbf{Twierdzenie Urquharta}

\textbf{Punkt i kąt Crelle'a-Brocarda}

\textbf{Twierdzenie o siódmym okręgu}

\textbf{Twierdzenie Caseya}

\textbf{Twierdzenie Taylora, okrąg, sześciokąt}

\textbf{Twierdzenie Eulera $1/4R^2$}

% https://en.wikipedia.org/wiki/Law_of_tangents

\subsubsection{Rozwiązywanie trójkątów}
Wzór Mollweide'a.
\index{wzór!Mollweide'a}%
Problem Hansena
\index{problem!Hansena}%
Problem Snelliusa-Pothenota.
\index{problem!Snelliusa-Pothenota}%
% https://en.wikipedia.org/wiki/Mollweide%27s_formula
% https://en.wikipedia.org/wiki/Snellius%E2%80%93Pothenot_problem
% https://en.wikipedia.org/wiki/Hansen%27s_problem
\index{trygonometria|)}

Twierdzenie Malfattiego.
Guzicki-11

\section{Bałagan}

\textbf{Twierdzenie Taylora, okrąg, sześciokąt}
% https://en.wikipedia.org/wiki/Taylor_circle
{
    \emph{WIP: Taylor w 1882 roku zauważył, że rzuty spodków wysokości na pozostałe boki leżą na jednym okręgu.}
	Hartshorne: s. 63
}

\textbf{Twierdzenie Eulera $1/4R^2$}

% https://en.wikipedia.org/wiki/Law_of_tangents


%