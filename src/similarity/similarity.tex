%

\chapter{Podobieństwo}
\section{Jednokładność}
Podobieństwo figur, trójkątów (cechy), stosunek pól figur podobnych.

\section{Twierdzenie Talesa}
%

Guzicki-3

\begin{theorem}[Talesa]
    Jeśli ramiona kąta płaskiego przetnie się 2 równoległymi prostymi:
    \begin{center}
\begin{comment}
        \begin{tikzpicture}
            \tkzDefPoint(0, 0.5){O}
            \tkzDefPoint(1.5, 0){A}
            \tkzDefPoint(2, 1){Ap}
            \tkzDefPointBy[homothety=center O ratio 1.618](A) \tkzGetPoint{B}
            \tkzDefLine[parallel=through B](A,Ap) \tkzGetPoint{Bp}
            \tkzInterLL(O,Ap)(B,Bp) \tkzGetPoint{Bpp}
            \tkzDrawPoints[fill=gray,opacity=.9](O,A,B,Ap,Bpp)
            \tkzLabelPoint[above](O){$O$}
            \tkzLabelPoint[below](A){$A$}
            \tkzLabelPoint[below](B){$A'$}
            \tkzLabelPoint[above left](Bpp){$B'$}
            \tkzLabelPoint[above left](Ap){$B$}
            \tkzDrawLine[thick](O,B)
            \tkzDrawLine[thick](O,Bpp)
            \tkzDrawLine[color=blue, thick](A,Ap)
            \tkzDrawLine[color=blue, thick](B,Bpp)
        \end{tikzpicture}
\end{comment}
        \end{center}
    to długości odcinków wyznaczonych przez te proste na jednym z ramion kąta są proporcjonalne do długości odpowiednich odcinków na drugim ramieniu kąta, a zatem
    \begin{equation}
        \label{thales_ratio}
        \frac{|OA|}{|OA'|} = \frac{|OB|}{|OB'|} = \frac{|AB|}{|A'B'|}.
    \end{equation}
\end{theorem}
% TODO: https://en.wikipedia.org/wiki/Thales's_theorem

Tradycja przypisuje jego sformułowanie Talesowi z Miletu, chociaż znane było starożytnym Babilończykom i Egipcjanom.
\index[persons]{Tales z Miletu}%
% Pierwszy znany dowód pojawia się w Elementach Euklidesa.
Najstarszy zachowany dowód twierdzenia Talesa zamieszczony jest w VI. księdze Elementów Euklidesa. 
% https://en.wikipedia.org/wiki/Intercept_theorem#Claim_3

Piszą o nim Neugebauer, Bogdańska \cite[s. 48-56]{neugebauer_2018}; Audin \cite[s. 24, 173]{audin_2003}.

Po angielsku znane jest jako \emph{Thales's theorem}, \emph{intercept theorem}, \emph{basic proportionality theorem} albo \emph{side splitter theorem}.

Prawdziwe jest również twierdzenie odwrotne:

\begin{proposition}[twierdzenie odwrotne do tw. Talesa]
    Jeżeli pewna prosta przecina boki $OA'$, $OB'$ trójkąta $OA'B'$ w różnych punktach $A$ i $B$ odpowiednio, a przy tym zachodzi równość \ref{thales_ratio}, to prosta ta jest równoległa do prostej $A'B'$.
\end{proposition}

Prostym wnioskiem z twierdzenia Talesa jest fakt \ref{hartshorne_52}, znajduje on zastosowanie w dowodzie:
% Neugebauer s. 52

\begin{theorem}[Varignona]
    Czworokąt $PQRS$, którego wierzchołki leżą na środkach boków $AB$, $BC$, $CD$, $DA$ czworokąta $ABCD$, jest równoległobokiem.
    Jego znakowane  (!) pole jest równe połowie pola czworokąta $ABCD$. % Neugebauer s. 61
\end{theorem}

% * The area of the Varignon parallelogram equals half the area of the original quadrilateral. This is true in convex, concave and crossed quadrilaterals provided the area of the latter is defined to be the difference of the areas of the two triangles it is composed of. => [[Varignon's theorem]]


W szczególności, czworokąt $ABCD$ nie musi być wypukły\footnote{Może być nawet ,,motylkiem'', to znaczy łamaną zamkniętą o czterech bokach, która ma samoprzecięcia.}.
Twierdzenie zostało nazwane na cześć Pierre'a Varignona pośmiertnie w 1731 roku.
\index[persons]{Varignon, Pierre}%
Co więcej,

\begin{proposition}
    Równoległobok Varignona jest rombem (prostokątem) wtedy i tylko wtedy, gdy przekątne czworokąta $ABCD$ są równej długości (są prostopadłe do siebie).
\index{równoległobok!Varignona}%
\index{romb}%
\index{prostokąt}%
% de Villiers, Michael (2009), Some Adventures in Euclidean Geometry, Dynamic Mathematics Learning, p. 58, 169. ISBN 9780557102952.
\end{proposition}

%

\section{Podobieństwo trójkątów}
\begin{definition}
	Dwa trójkąty nazywamy podobnymi...
	Liczbę $\lambda$... nazywamy skalą podobieństwa.
\end{definition}

\begin{proposition}[cecha podobieństwa BKB]
	Jeśli dla danych trójkątów...
\end{proposition}

\begin{proposition}[cecha podobieństwa BBB]
	Jeśli dla danych trójkątów...
\end{proposition}

% Przykład: zadanie 2.4 z Neugebauera, s. 60

% Jeżeli... ze skalą podobieństwa \lambda, to pola... \lambda^2.

% \section{Pole?}

\section{Twierdzenie o dwusiecznej}
% Coxeter s. 88 

\begin{proposition}[twierdzenie o dwusiecznej]
	Dany jest trójkąt $\triangle ABC$.
	Odcinek $CF$, gdzie $F$ leży na odcinku $AB$, jest dwusieczną kąta przy wierzchołku $C$ wtedy i tylko wtedy, gdy zachodzi równość:
	\begin{equation}
		\frac{|AF|}{|BF|} = \frac{|AC|}{|BC|}.
	\end{equation}
\end{proposition}

To jest (VI.3) w Elementach Euklidesa.
Dowód korzysta z podobieństwa trójkątów, twierdzenia sinusów albo własności pola, patrz Guzicki \cite[s. 120]{guzicki_2021} (siedem różnych dowodów); Bogdańska, Neugebauer \cite[s. 73]{neugebauer_2018}, Audin \cite[s. 102]{audin_2003} w formie ćwiczenia.

\begin{proposition} % Guzicki s. 126
	Punkt $P$ znajdujący się wewnątrz kąta wypukłego leży na dwusiecznej tego kąta wtedy i tylko wtedy, gdy odległości tego punktu od ramion kąta są równe.
\end{proposition}

% czemu to jest tu? czemu nie przenieść do trójkątów?
%

Wnioskiem z twierdzenia o dwusiecznej jest:

\begin{theorem}[Steinera-Lehmusa]
    \label{theorem_steiner_lehmus}
	Jeżeli dwie dwusieczne trójkąta są równej długości, to trójkąt ten jest równoramienny.
\end{theorem}

Po raz pierwszy wspomniał o nim Christian Lehmus w liście z 1840 roku do Charlesa Sturma, gdzie poprosił o czysto geometryczny dowód.
\index[persons]{Lehmus, Christian}%
\index[persons]{Sturm, Charles}%
Sturm przekazał prośbę do innych matematyków, jedną z pierwszych osób, która uporała się z problemem, był Jakob Steiner.
\index[persons]{Steiner, Jakob}%
Większość znanych dowodów przeprowadza się nie wprost: jeśli trójkąt nie jest równoramienny, to ma dwusieczne różnej długości.
Coxeter \cite[s. 9]{coxeter_1991} podaje je w formie ćwiczenia.
Dowód można znaleźć u Bogdańskiej, Neugebauera \cite[s. 74]{neugebauer_2018}.

%

% TODO: https://zadania.info/d801/3107027

Twierdzenie o symetralnej % TODO czy to jest oficjalna nazwa
można wysłowić tak: miejsce geometryczne punktów $X$, dla których $|AX|/|BX| = 1$, jest prostą.
Około dwusetnego roku przed naszą erą Apoloniusz z Pergi udowodnił piękne uogólnienie tego faktu:

\begin{definition}[okrąg Apoloniusza] % TODO: Guzicki s. 129
	Dane są dwa różne punkty $A$, $B$ oraz liczba dodatnia $\lambda \neq 1$.
	Wtedy zbiór punktów 
	\begin{equation}
		\left\{X : \frac{|AX|}{|BX|} = \lambda \right\}
	\end{equation}
	jest okręgiem o środku na prostej $AB$ i promieniu równym
	\begin{equation}
		R = \frac{\lambda}{|\lambda^2 - 1|} \cdot |AB|.
	\end{equation}
\end{definition}

To jest jeden z pięciu okręgów Apolloniusza, oprócz tego mamy dwie rodziny wzajemnie prostopadłych okręgów, okręgi Apoloniusza trójkąta (pomocne w znajdowaniu punktów izodynamicznych oraz prostej Lemoine'a), okrąg z problemu Apolloniusza styczny do trzech danych oraz fraktal zwany po angielsku uszczelką -- \emph{,,Apollonian gasket''}.
% https://en.wikipedia.org/wiki/Circles_of_Apollonius
\todofoot{Loksodromiczny ciąg Coxetera okręgów} % https://en.wikipedia.org/wiki/Coxeter%27s_loxodromic_sequence_of_tangent_circles

Piszą o nim Bogdańska, Neugebauer \cite[s. 74]{neugebauer_2018}



\section{Dwustosunek}

\section{Okręgi ortogonalne, pęki okręgów.}
Wie czym są pęki okręgów, zna ich podstawowe własności i potrafi stosować w konfiguracjach spokrewnionych z twierdzeniem Ponceleta.   

% T2.19 tutaj

% Poncelet przeniesiony do czworokątów dwuśrodkowych


\section{Trygonometria}
\todofoot{Nasir al Din al Tusi was the first to write a work on trigonometry independently of astronomy} % https://en.wikipedia.org/wiki/Nasir_al-Din_al-Tusi#Mathematics

\begin{proposition}
	Niech $\alpha, \beta, \gamma$ będą miarami kątów trójkąta o bokach $a, b, c$, polu $S$ i promieniu okręgu opisanego $R$.
	Wtedy
	\begin{equation}
		\tan \alpha + \tan \beta + \tan \gamma = \frac{4S}{a^2 + b^2 + c^2 - 8R^2}
	\end{equation}
\end{proposition}

\subsection{Twierdzenie sinusów}
\todofoot{Law of sines}
% TODO: https://en.wikipedia.org/wiki/Law_of_sines
\todofoot{ca. 1000 – Law of sines is discovered by Muslim mathematicians, but it is uncertain who discovers it first between Abu-Mahmud al-Khujandi, Abu Nasr Mansur, and Abu al-Wafa.}

$$\frac{a}{\sin \alpha} = \frac{b}{\sin \beta} = \frac{c}{\sin \gamma} = 2R$$


\subsection{Twierdzenie cosinusów}
Twierdzenie cosinusów jest bardzo stare.
Euklides rozpatrzy je w Elementach osobno dla trójkątów rozwartokątnych (II.12) i ostrokątnych (II.13).
% TODO: https://en.wikipedia.org/wiki/Law_of_cosines The cases of obtuse triangles and acute triangles (corresponding to the two cases of negative or positive cosine) are treated separately, in Propositions II.12 and II.13:[1]
Perski matematyk Jamshid al-Kashi znajdzie wartość $2\pi$ z~dokładnością do szesnastu cyfr, będzie autorem najdokładniejszych tablic trygonometrycznych swoich czasów i poda równoważną postać wzoru,
\index[persons]{al-Kashi, Jamshid}
\begin{equation}
	c = \sqrt{(b - a \cos \gamma)^2 + (a \sin \gamma)^2}
\end{equation}
dla ostrego kąta $\gamma$.
Taka sama metoda rozwiązywania trójkątów pojawi się w Europie w 1464 roku, kiedy Regiomontanus opublikuje \emph{De triangulis omnimodis} (czyli ,,O trójkątach wszelkiego rodzaju'').
\index[persons]{Regiomontanus}%
Współczesna forma twierdzenia to zasługa François Viète'a.
\index[persons]{Viète, François}%

\begin{proposition}[twierdzenie cosinusów]
	W trójkącie o bokach długości $a, b, c$ z kątem $\gamma$ naprzeciwko krawędzi $c$ zachodzi
	\label{twierdzenie_cosinusow}%
	\begin{equation}
		c^2 = a^2 + b^2 - 2ab \cos \gamma.
	\end{equation}
	% https://en.wikipedia.org/wiki/Law_of_cosines
\end{proposition}

https://www.deltami.edu.pl/2017/04/niemozliwe-wycinanki/

Znamy różne dowody tego twierdzenia: korzystające z twierdzenia Pitagorasa, trzech wysokości trójkąta, twierdzenia Ptolemeusza, geometrii koła albo twierdzenia sinusów.
We Francji twierdzenie cosinusów do późna będzie nazywane \emph{théorème d'Al-Kashi}, na cześć wspomnianego wyżej Persa.

Piszą o nim Guzicki \cite[s. 258]{guzicki_2021}.

\begin{theorem}[Stewarta, 1746]
\index{twierdzenie!Stewarta}
	W trójkącie $\triangle ABC$ o bokach długości $a, b, c$ poprowadzono czewianę z~wierzchołka $C$ do boku $AB$ o długości $d$, dzieląc ten bok na odcinki długości $n$ oraz $m$, jak na rysunku:
	\begin{center}
\begin{comment}
    \begin{tikzpicture}[scale=.4]
        %\tkzInit[xmin=-0.5,xmax=6.5, ymin=-0.5,ymax=4.5]
        % \tkzClip
        \tkzDefPoint(0, 0){A}
		\tkzDefPoint(3.25, 3.25){d}

		\tkzDefPoint(6, 0){AB}
        \tkzDefPoint(10, 0){B}
        \tkzDefPoint(1, 7){C}
        \tkzDefPoint(35:4.75){CC}
		\tkzDrawSegments(C,AB)
        \tkzDrawPolygon[line width=0.3mm](A,B,C)

        \tkzLabelPoint[below left](A){$A$}
        \tkzLabelPoint[below right](B){$B$}
        \tkzLabelPoint[above](C){$C$}

        \tkzLabelPoint(d){$d$}

		\tkzDrawPoints[size=3,color=black,fill=black!80](A,B,C,AB)
		\tkzDrawSegment[dim={$\,\,c\,\,$,-16pt,transform shape}](A,B)
		\tkzDrawSegment[dim={$\,\,n\,\,$,-8pt,transform shape}](A,AB)
		\tkzDrawSegment[dim={$\,\,m\,\,$,-8pt,transform shape}](AB,B)
		\tkzDrawSegment[dim={$\,\,b\,\,$,8pt,transform shape,sloped}](A,C)
		\tkzDrawSegment[dim={$\,\,a\,\,$,-8pt,transform shape,sloped}](B,C)
    \end{tikzpicture}
\end{comment}
    \end{center}
	Wtedy
	\begin{equation}
		b^2 m + c^2 n = a (d^2 + mn).
	\end{equation}
\end{theorem}

Twierdzenie bez dowodu opublikuje Matthew Stewart w 1746 roku, chociaż Coxeter przypuści, że mogło być znane nawet Archimedesowi.
\index[persons]{Stewart, Matthew}%
\index[persons]{Archimedes}%
% Coxeter, H.S.M.; Greitzer, S.L. (1967), Geometry Revisited, New Mathematical Library #19, The Mathematical Association of America, ISBN 0-88385-619-0 strona 6
Dowody podadzą też Simpson, Euler, Carnot i pewnie jeszcze ktoś.
\index[persons]{Simpson, Thomas}%
\index[persons]{Euler, Leonhard}%
\index[persons]{Carnot, L.M.N.}%
W przyszłości często pokazywać się będzie je jako zastosowanie twierdzenia cosinusów, tak jak Guzicki \cite[s. 265]{guzicki_2021}, Bogdańska, Neugebauer \cite[s. 90-91]{neugebauer_2018}; inne rozwiązanie skorzysta z odcinków skierowanych.
Znalezienie go to ćwiczenie podane przez Evesa \cite[s. 58]{eves1_1972}.

\begin{corollary}[twierdzenie Apoloniusza]
	% https://en.wikipedia.org/wiki/Apollonius%27s_theorem
	The theorem is found as proposition VII.122 of Pappus of Alexandria's Collection (c. 340 AD). It may have been in Apollonius of Perga's lost treatise Plane Loci (c. 200 BC), and was included in Robert Simson's 1749 reconstruction of that work.[1]
\end{corollary}

\begin{corollary}
	W trójkącie $\triangle ABC$ o bokach długości $a, b, c$ poprowadzono środkową oraz dwusieczną z~wierzchołka $C$.
	Długość środkowej wynosi
	\begin{equation}
		\sqrt{\frac{a^2 + b^2}{2} - \frac{c^2}{4}},
	\end{equation}
	zaś dwusieczna ma długość
	\begin{equation}
		\frac{\sqrt{ab (a+b+c)(a+b-c)}}{a+b}.
	\end{equation}
\end{corollary}

\begin{corollary}[z twierdzenia Stewarta]
	Odległość między środkiem ciężkości i środkiem okręgu opisanego na trójkącie wynosi
	\begin{equation}
		\sqrt{R^2 - \frac 19 \left(a^2 + b^2 + c^2\right)},
	\end{equation}
	w szczególności więc $(3R)^2 \ge a^2 + b^2 + c^2$, z równością dla trójkąta równobocznego.
\end{corollary}

Wniosek ten znajdziemy u Zetela \cite[s. 72]{zetel_2020}

\subsection{Więcej wzorów w powijakach z okręgami}
Wzory na promienie okręgów wpisanych, dopisanych.
$4R = r_a + r_b + r_c - r$ % Coxeter, s. 13; dopisz też bend okręgi Kartezjusza

Cztery okręgi ze znakiem Kartezjusza; Soddy.
Cztery nowe okręgi Beecrofta.
% ćwiczenie 2. ze strony 16
% https://en.wikipedia.org/wiki/Descartes%27_theorem
% https://dept.math.lsa.umich.edu/~lagarias/doc/descartes.pdf Soddy-Gossett,

\subsection{Rozwiązywanie trójkątów}

Podamy teraz kilka nowych zależności trygonometrycznych, które pomagają w rozwiązywaniu trójkątów.
Znamy już twierdzenia sinusów i cosinusów; w ogólności to drugie jest bezpieczniejsze od pierwszego (ponieważ z faktu, że $\sin \alpha = \frac 1 2$ nie wynika, czy kąt $\alpha$ jest ostry, czy rozwarty, cosinus zaś jest jednoznaczny).

W każdym z poniższych stwierdzeń mamy trójkąt o kątach $\alpha, \beta, \gamma$, bokach $a, b, c$, połowie obsowdu $p = \frac 1 2 (a + b + c)$, promieniu okręgu opisanego $R$ i wpisanego $r$.

\begin{proposition}
    Zachodzi
    \begin{equation}
    S = 2 R^2 \sin \alpha \sin \beta \sin \gamma.
    \end{equation}
\end{proposition}

\begin{proposition}
    Zachodzi
    \begin{equation}
        p = R (\sin \alpha + \sin \beta + \sin \theta).
    \end{equation}
\end{proposition}

\begin{proposition}
    Zachodzi
    \begin{equation}
        r = 4R \sin \frac \alpha 2 \sin \frac \beta 2 \sin \frac \gamma 2.
    \end{equation}
\end{proposition}

\begin{proposition}[wzór Newtona]
\index{wzór!Newtona}%
    Zachodzi
    \begin{equation}
        \frac{a + b}{c} = \frac{\cos \frac 1 2 (\alpha - \beta)}{\cos \frac 1 2 (\alpha + \beta)}.
    \end{equation}
\end{proposition}

\begin{proposition}[wzór Mollweidego]
\index{wzór!Mollweidego}%
    Zachodzi
    \begin{equation}
        \frac{a - b}{c} = \frac{\sin \frac 1 2 (\alpha - \beta)}{\sin \frac 1 2 (\alpha + \beta)}.
    \end{equation}
\end{proposition}

(Nie ma zgodności co do powyższych nazw).
Nieco bardziej geometryczną wersję wzorów podał Izaak Newton w 1707 roku, potem Friedrich von Oppel w 1746.
\index[persons]{Newton, Izaak}%
\index[persons]{von Oppel, Friedrich}%
Wyrażenia, których używamy po dziś dzień zawdzięczamy Thomasowi Simpsonowi z 1748 roku, oraz Karlowi Mollweidemu, który opublikował to samo w 1808 roku bez cytowania poprzedników.
\index[persons]{Simpson, Thomas}%
\index[persons]{Mollweide, Karl}%

\begin{proposition}[wzór Regiomontanusa]
\index{wzór!Regiomontanusa}%
    Zachodzi
    \begin{equation}
        \frac{a + b}{a- b} = \frac{\tan \frac 1 2 (\alpha + \beta)}{\tan \frac 1 2 (\alpha - \beta)}.
    \end{equation}
\end{proposition}

Regiomontanus był znany gorzej jako Johannes Müller von Königsberg.
\index[persons]{von Königsberg, Johannes|see{Regiomontanus}}%
\index[persons]{Regiomontanus}%


\subsection{Zastosowania trygonometrii -- twierdzenie Urquharta}
%

\begin{theorem}[Urquharta]
\label{theorem_urquhart}%
\index{twierdzenie!Urquharta}%
    Przy oznaczeniach jak na rysunku:
    \begin{center}
\begin{comment}
    \begin{tikzpicture}[scale=.4]
        \tkzDefPoint(0, 0){E}
        \tkzDefPoint(267:5){E1}
        \tkzDefPoint(250:5){E2}
        \tkzDefPoint(165:5){E3}
        \tkzDefPoint(135:5){E4}
        
        \tkzDefLine[tangent at=E1](E) \tkzGetPoint{tan1}
        \tkzDefLine[tangent at=E2](E) \tkzGetPoint{tan2}
        \tkzDefLine[tangent at=E3](E) \tkzGetPoint{tan3}
        \tkzDefLine[tangent at=E4](E) \tkzGetPoint{tan4}
        
        \tkzInterLL(E4,tan4)(E1,tan1) \tkzGetPoint{S1}
        \tkzInterLL(E4,tan4)(E2,tan2) \tkzGetPoint{S2}
        \tkzInterLL(E3,tan3)(E1,tan1) \tkzGetPoint{S3}
        \tkzInterLL(E3,tan3)(E2,tan2) \tkzGetPoint{S4}

        \tkzInterLL(S3,S4)(S1,S2) \tkzGetPoint{Gorny}
        \tkzInterLL(S1,S3)(S2,S4) \tkzGetPoint{Dolny}

        \tkzLabelPoint[below](S1){$A$}
        \tkzLabelPoint[above left](S2){$D$}
        \tkzLabelPoint(S3){$B$}
        \tkzLabelPoint[above right](S4){$C$}
        \tkzLabelPoint(Dolny){$E$}
        \tkzLabelPoint[right](Gorny){$F$}

        \tkzDrawSegments[line width=0.2mm](S1,Dolny S1,Gorny S3,Gorny S2,Dolny)
        \tkzDrawPoints[size=3,color=black,fill=black!50](S1,S2,S3,S4,Dolny,Gorny)
        % konstrukcja z https://en.wikipedia.org/wiki/Ex-tangential_quadrilateral
\end{tikzpicture}
\end{comment}
    \end{center}
    niech $|AB| + |BC| = |AD| + |CD|$.
    Wtedy $|AE| + |CE| = |AF| + |CF|$.
\end{theorem}

Twierdzenie \ref{theorem_urquhart} nie jest bardzo popularne, napiszą o nim Bogdańska, Neugebauer \cite[s. 97]{neugebauer_2018}, nieznany autor w $\Delta_{84}^{11}$ i pewnie ktoś jeszcze.
Nie do końca wiadomo, kto udowodni je pierwszy: być może będzie to de Morgan w 1841 roku; teza jest też przypadkiem granicznym innego wyniku Chaslesa, znanego około roku 1860.
Wreszcie Dan Pedoe \cite{pedoe_1976} przypisze to twierdzenie Malcolmowi Livingstonowi Urquhartowi\footnote{Matematyk australijski, będzie żyć w latach 1902-1966 i nie opublikuje żadnej pracy.}.
\index[persons]{Pedoe, Daniel}%
\index[persons]{Urquhart, Malcolm Livingston}%

%

\subsection{Zastosowania trygonometrii -- punkt i kąt Crelle'a-Brocarda}

Poznamy teraz dwa wyróżnione punkty trójkąta, opisane w 1875 roku przez oficera francuskiej armii, Henriego Brocarda oraz wiele lat wcześniej przez Augusta Crelle'a \cite{crelle_1816}, założyciela słynnego czasopisma matematycznego.
\index[persons]{Brocard, Henri}%
\index[persons]{Crelle, August}%

\begin{definition}
    Niech $\triangle ABC$ będzie trójkątem.
    Punkt $X$ leżący w jego wnętrzu taki, że kąty $\angle XAB$, $\angle XBC$, $\angle XCA$ są równej miary, nazywamy (pierwszym) punktem Crelle'a-Brocarda.
\end{definition}

\index{punkt Crelle'a-Brocarda}%
\index{kąt Crelle'a-Brocarda}%

Miarę wspomnianych kątów nazywamy kątem Crelle'a-Brocarda.
Jest jeszcze drugi punkt Crelle'a-Brocarda, gdzie odwracamy kolejność punktów: $\angle XBA = \angle XCB = \angle XAC$.

\begin{proposition}
    W każdym trójkącie istnieje (jedyny) punkt Crelle'a-Brocarda.
    Miara kąta Crelle'a-Brocarda $\omega$ spełnia związek
    \begin{equation}
        \cot \omega = \cot \alpha + \cot \beta + \cot \gamma,
    \end{equation}
    gdzie $\alpha, \beta, \gamma$ to miary kątów trójkąta.
    Ponadto, $0 \le \omega \le \pi/6$.
\end{proposition}

Bogdańska, Neugebauer \cite[s. 100]{neugebauer_2018} podają jako ćwiczenie:

\begin{proposition}
    Niech $X$ będzie punktem Crelle'a-Brocarda trójkąta $\triangle ABC$.
    Niech $R_a$, $R_b$ i $R_c$ oznaczają promienie okręgów opisanych na trójkątach $\triangle XBC$, $\triangle AXC$, $\triangle ABX$, zaś $R$ będzie jak zwykle promieniem okręgu opisanego na trójkącie $\triangle ABC$.
    Wtedy
    \begin{equation}
        R = \sqrt[3]{R_a R_b R_c}.
    \end{equation}
\end{proposition}

Peter Yiff \cite{yff_1963} postawił w 1963 roku (!) hipotezę, że $8 \omega^3 \le \alpha \beta \gamma$.
\index[persons]{Yiff, Peter}%
Dowód znalazł w 1974 roku Faruk Abi-Khuzam \cite{abikhuzam_1974}.
\index[persons]{Abi-Khuzam, Faruk}%
% P. Yff, "An analogue of the Brocard points" Amer. Math. Monthly , 70 (1963) pp. 495–501
% F. Abi–Khuzam, "Proof of Yff's conjecture on the Brocard angle of a triangle" Elem. Math. , 29 (1974) pp. 141–142



\subsection{Problem Hansena}
Problem Hansena
\index{problem!Hansena}%

\subsection{Problem Snelliusa-Pothenota}
Problem Snelliusa-Pothenota.
\index{problem!Snelliusa-Pothenota}%

% https://en.wikipedia.org/wiki/Mollweide%27s_formula
% https://en.wikipedia.org/wiki/Snellius%E2%80%93Pothenot_problem
% https://en.wikipedia.org/wiki/Hansen%27s_problem

Twierdzenie Malfattiego.
Guzicki-11

\section{Bałagan}

\textbf{Twierdzenie Taylora, okrąg, sześciokąt}
% https://en.wikipedia.org/wiki/Taylor_circle
{
    \emph{WIP: Taylor w 1882 roku zauważył, że rzuty spodków wysokości na pozostałe boki leżą na jednym okręgu.}
	Hartshorne: s. 63
}

\textbf{Twierdzenie Eulera $1/4R^2$}

% https://en.wikipedia.org/wiki/Law_of_tangents


%