%

\section{Jednokładność}
Podobieństwo figur, trójkątów (cechy), stosunek pól figur podobnych.

\index{twierdzenie!Talesa|(}
\section{Twierdzenie Talesa}
%

Guzicki-3

\begin{theorem}[Talesa]
    Jeśli ramiona kąta płaskiego przetnie się 2 równoległymi prostymi:
    \begin{center}
        \begin{tikzpicture}
            \tkzDefPoint(0, 0.5){O}
            \tkzDefPoint(1.5, 0){A}
            \tkzDefPoint(2, 1){Ap}
            \tkzDefPointBy[homothety=center O ratio 1.618](A) \tkzGetPoint{B}
            \tkzDefLine[parallel=through B](A,Ap) \tkzGetPoint{Bp}
            \tkzInterLL(O,Ap)(B,Bp) \tkzGetPoint{Bpp}
            \tkzDrawPoints[fill=gray,opacity=.9](O,A,B,Ap,Bpp)
            \tkzLabelPoint[above](O){$O$}
            \tkzLabelPoint[below](A){$A$}
            \tkzLabelPoint[below](B){$A'$}
            \tkzLabelPoint[above left](Bpp){$B'$}
            \tkzLabelPoint[above left](Ap){$B$}
            \tkzDrawLine[thick](O,B)
            \tkzDrawLine[thick](O,Bpp)
            \tkzDrawLine[color=blue, thick](A,Ap)
            \tkzDrawLine[color=blue, thick](B,Bpp)
        \end{tikzpicture}
        \end{center}
    to długości odcinków wyznaczonych przez te proste na jednym z ramion kąta są proporcjonalne do długości odpowiednich odcinków na drugim ramieniu kąta, a zatem
    \begin{equation}
        \label{thales_ratio}
        \frac{|OA|}{|OA'|} = \frac{|OB|}{|OB'|} = \frac{|AB|}{|A'B'|}.
    \end{equation}
\end{theorem}
% TODO: https://en.wikipedia.org/wiki/Thales's_theorem

Tradycja przypisuje jego sformułowanie Talesowi z Miletu, chociaż znane było starożytnym Babilończykom i Egipcjanom.
\index[persons]{Tales z Miletu}%
% Pierwszy znany dowód pojawia się w Elementach Euklidesa.
Najstarszy zachowany dowód twierdzenia Talesa zamieszczony jest w VI. księdze Elementów Euklidesa. 
% https://en.wikipedia.org/wiki/Intercept_theorem#Claim_3

Piszą o nim Neugebauer, Bogdańska \cite[s. 48-56]{neugebauer_2018}.
Po angielsku znane jest jako \emph{Thales's theorem}, \emph{intercept theorem}, \emph{basic proportionality theorem} albo \emph{side splitter theorem}.

Prawdziwe jest również twierdzenie odwrotne:

\begin{proposition}[twierdzenie odwrotne do tw. Talesa]
    Jeżeli pewna prosta przecina boki $OA'$, $OB'$ trójkąta $OA'B'$ w różnych punktach $A$ i $B$ odpowiednio, a przy tym zachodzi równość \ref{thales_ratio}, to prosta ta jest równoległa do prostej $A'B'$.
\end{proposition}

Prostym wnioskiem z twierdzenia Talesa jest fakt \ref{hartshorne_52}, znajduje on zastosowanie w dowodzie:
% Neugebauer s. 52

\begin{theorem}[Varignona]
    Czworokąt $PQRS$, którego wierzchołki leżą na środkach boków $AB$, $BC$, $CD$, $DA$ czworokąta $ABCD$, jest równoległobokiem.
    Jego pole jest równe połowie pola czworokąta $ABCD$. % Neugebauer s. 61
\end{theorem}

W szczególności, czworokąt $ABCD$ nie musi być wypukły\footnote{Może być nawet ,,motylkiem'', to znaczy łamaną zamkniętą o czterech bokach, która ma samoprzecięcia.}.
Twierdzenie zostało nazwane na cześć Pierre'a Varignona pośmiertnie w 1731 roku.
\index[persons]{Varignon, Pierre}%
Co więcej,

\begin{proposition}
    Równoległobok Varignona jest rombem (prostokątem) wtedy i tylko wtedy, gdy przekątne czworokąta $ABCD$ są równej długości (są prostopadłe do siebie).
\index{równoległobok Varignona}%
\index{romb}%
\index{prostokąt}%
% de Villiers, Michael (2009), Some Adventures in Euclidean Geometry, Dynamic Mathematics Learning, p. 58, 169. ISBN 9780557102952.
\end{proposition}

%
\index{twierdzenie!Talesa|)}

\section{Podobieństwo trójkątów}
\begin{definition}
	Dwa trójkąty nazywamy podobnymi...
	Liczbę $\lambda$... nazywamy skalą podobieństwa.
	\index{podobieństwo}
\end{definition}

\begin{proposition}[cecha podobieństwa BKB]
	Jeśli dla danych trójkątów...
	\index{cecha podobieństwa!bok-kąt-bok}
\end{proposition}

\begin{proposition}[cecha podobieństwa BBB]
	Jeśli dla danych trójkątów...
	\index{cecha podobieństwa!bok-bok-bok}
\end{proposition}

% Przykład: zadanie 2.4 z Neugebauera, s. 60

% Jeżeli... ze skalą podobieństwa \lambda, to pola... \lambda^2.

% \section{Pole?}

\section{Twierdzenie o dwusiecznej}
% Coxeter s. 88 

\begin{proposition}[twierdzenie o dwusiecznej]
	Dany jest trójkąt $\triangle ABC$.
	Odcinek $CF$, gdzie $F$ leży na odcinku $AB$, jest dwusieczną kąta przy wierzchołku $C$ wtedy i tylko wtedy, gdy zachodzi równość:
	\begin{equation}
		\frac{|AF|}{|BF|} = \frac{|AC|}{|BC|}.
	\end{equation}
	\index{twierdzenie!o dwusiecznej}
\end{proposition}

To jest (VI.3) w Elementach Euklidesa.
Dowód korzysta z podobieństwa trójkątów, twierdzenia sinusów albo własności pola, patrz Guzicki \cite[s. 120]{guzicki_2021} (siedem różnych dowodów); Bogdańska, Neugebauer \cite[s. 73]{neugebauer_2018}, Audin \cite[s. 102]{audin_2003} w formie ćwiczenia.
\index{twierdzenie!sinusów}

\begin{proposition} % Guzicki s. 126
	Punkt $P$ znajdujący się wewnątrz kąta wypukłego leży na dwusiecznej tego kąta wtedy i tylko wtedy, gdy odległości tego punktu od ramion kąta są równe.
\end{proposition}

% czemu to jest tu? czemu nie przenieść do trójkątów?
\index{twierdzenie!Steinera-Lehmusa|(}
%

Wnioskiem z twierdzenia o dwusiecznej jest:

\begin{theorem}[Steinera-Lehmusa]
    \index{twierdzenie!Steinera-Lehmusa}%
    \label{theorem_steiner_lehmus}%
	Jeżeli dwie dwusieczne trójkąta są równej długości, to trójkąt ten jest równoramienny.
\end{theorem}

Po raz pierwszy wspomniał o nim Christian Lehmus w liście z 1840 roku do Charlesa Sturma, gdzie poprosił o czysto geometryczny dowód.
\index[persons]{Lehmus, Christian}%
\index[persons]{Sturm, Charles}%
Sturm przekazał prośbę do innych matematyków, jedną z pierwszych osób, która uporała się z problemem, był Jakob Steiner.
\index[persons]{Steiner, Jakob}%
Większość znanych dowodów przeprowadza się nie wprost: jeśli trójkąt nie jest równoramienny, to ma dwusieczne różnej długości.
Dowód można znaleźć u Hartshorne'a \cite[s. 11]{hartshorne2000}; Bogdańskiej, Neugebauera \cite[s. 74]{neugebauer_2018}.
Coxeter \cite[s. 32]{coxeter_1967} podaje je w formie ćwiczenia (po tym jak wcześniej \cite[s. 26, 33]{coxeter_1967} poprosi o~dowód tego samego dla dwusiecznej zamienionej na środkową lub wysokość, co znacząco obniża poziom trudności).
% https://www.algebra.com/algebra/homework/word/geometry/Medians-in-an-isosceles-triangle.lesson
Eves \cite[s. 19, 58]{eves1_1972} zachęca do dowodu z dopiskiem, że jest trudny. 

%
\index{twierdzenie!Steinera-Lehmusa|)}

% TODO: https://zadania.info/d801/3107027

Twierdzenie o symetralnej % TODO czy to jest oficjalna nazwa
\index{twierdzenie!o symetralnej}
można wysłowić tak: miejsce geometryczne punktów $X$, dla których $|AX|/|BX| = 1$, jest prostą.
Około dwusetnego roku przed naszą erą Apoloniusz z Pergi udowodnił piękne uogólnienie tego faktu:

\begin{definition}[okrąg Apoloniusza] % TODO: Guzicki s. 129
	Dane są dwa różne punkty $A$, $B$ oraz liczba dodatnia $\lambda \neq 1$.
	Wtedy zbiór punktów 
	\begin{equation}
		\left\{X : \frac{|AX|}{|BX|} = \lambda \right\}
	\end{equation}
	jest okręgiem o środku na prostej $AB$ i promieniu równym
	\begin{equation}
		R = \frac{\lambda}{|\lambda^2 - 1|} \cdot |AB|.
	\end{equation}
	\index{okrąg!Apoloniusza}
\end{definition}

To jest jeden z pięciu okręgów Apoloniusza, oprócz tego mamy dwie rodziny wzajemnie prostopadłych okręgów, okręgi Apoloniusza trójkąta (pomocne w znajdowaniu punktów izodynamicznych oraz prostej Lemoine'a), okrąg z problemu Apoloniusza styczny do trzech danych oraz fraktal zwany po angielsku uszczelką -- \emph{,,Apollonian gasket''}.
% https://en.wikipedia.org/wiki/Circles_of_Apollonius
\todofoot{Loksodromiczny ciąg Coxetera okręgów} % https://en.wikipedia.org/wiki/Coxeter%27s_loxodromic_sequence_of_tangent_circles

Piszą o nim Bogdańska, Neugebauer \cite[s. 74]{neugebauer_2018}

\section{Dwustosunek}
\index{dwustosunek}

\section{Okręgi ortogonalne, pęki okręgów.}
\index{okrąg!ortogonalny}
\index{pęk okręgów}
Wie czym są pęki okręgów, zna ich podstawowe własności i potrafi stosować w konfiguracjach spokrewnionych z twierdzeniem Ponceleta.   
\index{twierdzenie!Ponceleta}

O pękach okręgów pisze Delta $\Delta_{24}^{12}$.

% T2.19 tutaj

% Poncelet przeniesiony do czworokątów dwuśrodkowych

\index{trygonometria|(}
\section{Trygonometria}
\todofoot{Nasir al Din al Tusi was the first to write a work on trigonometry independently of astronomy} % https://en.wikipedia.org/wiki/Nasir_al-Din_al-Tusi#Mathematics

\begin{proposition}
	Niech $\alpha, \beta, \gamma$ będą miarami kątów trójkąta o bokach $a, b, c$, polu $S$ i promieniu okręgu opisanego $R$.
	Wtedy
	\begin{equation}
		\tan \alpha + \tan \beta + \tan \gamma = \frac{4S}{a^2 + b^2 + c^2 - 8R^2}
	\end{equation}
\end{proposition}

\subsection{Twierdzenie sinusów}
\todofoot{Law of sines}
\index{twierdzenie!sinusów}
% TODO: https://en.wikipedia.org/wiki/Law_of_sines
\todofoot{ca. 1000 - Law of sines is discovered by Muslim mathematicians, but it is uncertain who discovers it first between Abu-Mahmud al-Khujandi, Abu Nasr Mansur, and Abu al-Wafa.}

$$\frac{a}{\sin \alpha} = \frac{b}{\sin \beta} = \frac{c}{\sin \gamma} = 2R$$

\subsection{Twierdzenie cosinusów}
\index{twierdzenie!cosinusów|(}%
Twierdzenie cosinusów jest bardzo stare.
Euklides rozpatruje je w Elementach osobno dla trójkątów rozwartokątnych (II.12) i ostrokątnych (II.13).
% TODO: https://en.wikipedia.org/wiki/Law_of_cosines The cases of obtuse triangles and acute triangles (corresponding to the two cases of negative or positive cosine) are treated separately, in Propositions II.12 and II.13:[1]
Perski matematyk Jamshid al-Kashi znalazł wartość $2\pi$ z~dokładnością do szesnastu cyfr, był autorem najdokładniejszych tablic trygonometrycznych swoich czasów i podał równoważną postać wzoru,
\index[persons]{al-Kashi, Jamshid}
\begin{equation}
	c = \sqrt{(b - a \cos \gamma)^2 + (a \sin \gamma)^2}
\end{equation}
dla ostrego kąta $\gamma$.
Taka sama metoda rozwiązywania trójkątów pojawiła się w Europie w 1464 roku, kiedy Regiomontanus opublikował \emph{De triangulis omnimodis} (czyli ,,O trójkątach wszelkiego rodzaju'').
\index[persons]{Regiomontanus}%
Współczesna forma twierdzenia to zasługa François Viète'a.
\index[persons]{Viète, François}%

\begin{proposition}[twierdzenie cosinusów]
	W trójkącie o bokach długości $a, b, c$ z kątem $\gamma$ naprzeciwko krawędzi $c$ zachodzi
	\label{twierdzenie_cosinusow}%
	\begin{equation}
		c^2 = a^2 + b^2 - 2ab \cos \gamma.
	\end{equation}
	% https://en.wikipedia.org/wiki/Law_of_cosines
\end{proposition}

Znamy różne dowody tego twierdzenia: korzystające z twierdzenia Pitagorasa, trzech wysokości trójkąta, twierdzenia Ptolemeusza, geometrii koła albo twierdzenia sinusów.
We Francji twierdzenie cosinusów do dzisiaj bywa nazywane \emph{théorème d'Al-Kashi}, na cześć wspomnianego wyżej Persa.

\begin{theorem}[Stewarta, 1746]
\index{twierdzenie!Stewarta}
	W trójkącie $\triangle ABC$ o bokach długości $a, b, c$ poprowadzono czewianę z~wierzchołka $C$ do boku $AB$ o długości $d$, dzieląc ten bok na odcinki długości $n$ oraz $m$, jak na rysunku:
	\begin{center}
\begin{comment}
    \begin{tikzpicture}[scale=.4]
        %\tkzInit[xmin=-0.5,xmax=6.5, ymin=-0.5,ymax=4.5]
        % \tkzClip
        \tkzDefPoint(0, 0){A}
		\tkzDefPoint(3.25, 3.25){d}

		\tkzDefPoint(6, 0){AB}
        \tkzDefPoint(10, 0){B}
        \tkzDefPoint(1, 7){C}
        \tkzDefPoint(35:4.75){CC}
		\tkzDrawSegments(C,AB)
        \tkzDrawPolygon[line width=0.3mm](A,B,C)

        \tkzLabelPoint[below left](A){$A$}
        \tkzLabelPoint[below right](B){$B$}
        \tkzLabelPoint[above](C){$C$}

        \tkzLabelPoint(d){$d$}

		\tkzDrawPoints[size=3,color=black,fill=black!80](A,B,C,AB)
		\tkzDrawSegment[dim={$\,\,c\,\,$,-16pt,transform shape}](A,B)
		\tkzDrawSegment[dim={$\,\,n\,\,$,-8pt,transform shape}](A,AB)
		\tkzDrawSegment[dim={$\,\,m\,\,$,-8pt,transform shape}](AB,B)
		\tkzDrawSegment[dim={$\,\,b\,\,$,8pt,transform shape,sloped}](A,C)
		\tkzDrawSegment[dim={$\,\,a\,\,$,-8pt,transform shape,sloped}](B,C)
    \end{tikzpicture}
\end{comment}
    \end{center}
	Wtedy
	\begin{equation}
		b^2 m + c^2 n = a (d^2 + mn).
	\end{equation}
\end{theorem}

Matthew Stewart opublikował to twierdzenie w 1746 roku, chociaż Coxeter przypuszcza, że mogło być znane nawet Archimedesowi.
\index[persons]{Stewart, Matthew}%
\index[persons]{Archimedes}%
% Coxeter, H.S.M.; Greitzer, S.L. (1967), Geometry Revisited, New Mathematical Library #19, The Mathematical Association of America, ISBN 0-88385-619-0 strona 6
Współcześnie często pokazuje się je jako zastosowanie twierdzenia cosinusów, tak jak Bogdańska, Neugebauer \cite[s. 90-91]{neugebauer_2018}.	

\begin{corollary}[twierdzenie Apoloniusza]
	% https://en.wikipedia.org/wiki/Apollonius%27s_theorem
	The theorem is found as proposition VII.122 of Pappus of Alexandria's Collection (c. 340 AD). It may have been in Apollonius of Perga's lost treatise Plane Loci (c. 200 BC), and was included in Robert Simson's 1749 reconstruction of that work.[1]
\end{corollary}

\begin{corollary}
	W trójkącie $\triangle ABC$ o bokach długości $a, b, c$ poprowadzono środkową oraz dwusieczną z~wierzchołka $C$.
	Długość środkowej wynosi
	\begin{equation}
		\sqrt{\frac{a^2 + b^2}{2} - \frac{c^2}{4}},
	\end{equation}
	zaś dwusieczna ma długość
	\begin{equation}
		\frac{\sqrt{ab (a+b+c)(a+b-c)}}{a+b}.
	\end{equation}
\end{corollary}
\index{twierdzenie!cosinusów|)}%

\subsection{Więcej wzorów w powijakach z okręgami}
Wzory na promienie okręgów wpisanych, dopisanych.
$4R = r_a + r_b + r_c - r$ % Coxeter, s. 13; dopisz też bend okręgi Kartezjusza

Cztery okręgi ze znakiem Kartezjusza; Soddy.
Cztery nowe okręgi Beecrofta.
\index{okrąg!Beecrofta}%
% ćwiczenie 2. ze strony 16
% https://en.wikipedia.org/wiki/Descartes%27_theorem
% https://dept.math.lsa.umich.edu/~lagarias/doc/descartes.pdf Soddy-Gossett,

\subsection{Rozwiązywanie trójkątów}
% https://en.wikipedia.org/wiki/Solution_of_triangles

Podamy teraz kilka nowych zależności trygonometrycznych, które pomagają w rozwiązywaniu trójkątów.
Znamy już twierdzenia sinusów i cosinusów; w ogólności to drugie jest bezpieczniejsze od pierwszego (ponieważ z faktu, że $\sin \alpha = \frac 1 2$ nie wynika, czy kąt $\alpha$ jest ostry, czy rozwarty, cosinus zaś jest jednoznaczny).

W każdym z poniższych stwierdzeń mamy trójkąt o kątach $\alpha, \beta, \gamma$, bokach $a, b, c$, połowie obsowdu $p = \frac 1 2 (a + b + c)$, promieniu okręgu opisanego $R$ i wpisanego $r$.

\begin{proposition}
    Zachodzi
    \begin{equation}
    S = 2 R^2 \sin \alpha \sin \beta \sin \gamma.
    \end{equation}
\end{proposition}

\begin{proposition}
    Zachodzi
    \begin{equation}
        p = R (\sin \alpha + \sin \beta + \sin \theta).
    \end{equation}
\end{proposition}

\begin{proposition}
    Zachodzi
    \begin{equation}
        r = 4R \sin \frac \alpha 2 \sin \frac \beta 2 \sin \frac \gamma 2.
    \end{equation}
\end{proposition}

\begin{proposition}[wzór Newtona]
\index{wzór!Newtona}%
    Zachodzi
    \begin{equation}
        \frac{a + b}{c} = \frac{\cos \frac 1 2 (\alpha - \beta)}{\cos \frac 1 2 (\alpha + \beta)}.
    \end{equation}
\end{proposition}

\begin{proposition}[wzór Mollweidego]
\index{wzór!Mollweidego}%
    Zachodzi
    \begin{equation}
        \frac{a - b}{c} = \frac{\sin \frac 1 2 (\alpha - \beta)}{\sin \frac 1 2 (\alpha + \beta)}.
    \end{equation}
\end{proposition}

(Nie ma zgodności co do powyższych nazw).
Nieco bardziej geometryczną wersję wzorów podał Izaak Newton w 1707 roku, potem Friedrich von Oppel w 1746.
\index[persons]{Newton, Izaak}%
\index[persons]{Oppel@von Oppel, Friedrich}%
Wyrażenia, których używamy po dziś dzień zawdzięczamy Thomasowi Simpsonowi z 1748 roku, oraz Karlowi Mollweidemu, który opublikował to samo w 1808 roku bez cytowania poprzedników.
\index[persons]{Simpson, Thomas}%
\index[persons]{Mollweide, Karl}%

\begin{proposition}[wzór Regiomontanusa]
\index{wzór!Regiomontanusa}%
    Zachodzi
    \begin{equation}
        \frac{a + b}{a- b} = \frac{\tan \frac 1 2 (\alpha + \beta)}{\tan \frac 1 2 (\alpha - \beta)}.
    \end{equation}
\end{proposition}
% TODO: https://en.wikipedia.org/wiki/Law_of_tangents

Regiomontanus był znany gorzej jako Johannes Müller von Königsberg.
\index[persons]{Königsberg@von Königsberg, Johannes|see{Regiomontanus}}%
\index[persons]{Regiomontanus}%


\subsection{Zastosowania trygonometrii -- twierdzenie Urquharta}
\begin{theorem}[Urquharta?]
\index{twierdzenie!Urquharta}%
    Przy oznaczeniach jak na rysunku, niech $|AB| + |BC| = |AD| + |CD|$.
    Wtedy $|AE| + |CE| = |AF| + |CF|$.
\end{theorem}

Dan Pedoe
% Pedoe, D. "The Most 'Elementary' Theorem of Euclidean Geometry." Math. Mag. 49, 40-42, 1976.
przypisuje to twierdzenie Malcolmowi Urquhartowi\footnote{Matematyk australijski, żył w latach 1902-1966 i nie opublikował żadnej pracy}.
\index{Urquhart, Malcolm}%
Wiemy jednak, że de Morgan opublikował swój dowód już w 1841 roku; samo zaś twierdzenie jest przypadkiem granicznym innego wyniku Chaslesa, znanego w latach 186x.
Poznaliśmy je, tak jak wiele innych, z książki Bogdańskiej, Neugebauera \cite[s. 97]{neugebauer_2018}.

\subsection{Zastosowania trygonometrii -- punkt i kąt Crelle'a-Brocarda}
%

Poznamy teraz dwa wyróżnione punkty trójkąta, opisane w 1875 roku przez oficera francuskiej armii, Henriego Brocarda oraz wiele lat wcześniej przez Augusta Crelle'a \cite{crelle_1816}, założyciela słynnego czasopisma matematycznego.
\index[persons]{Brocard, Henri}%
\index[persons]{Crelle, August}%

\begin{definition}
\label{punkty_brocarda}%
    Niech $\triangle ABC$ będzie trójkątem.
    Punkt $X$ leżący w jego wnętrzu taki, że kąty $\angle XAB$, $\angle XBC$, $\angle XCA$ są równej miary, nazywamy (pierwszym) punktem Crelle'a-Brocarda.
    \index{punkt!Crelle'a-Brocarda}%
    \begin{center}
\begin{comment}
    \begin{tikzpicture}[scale=.75]
        \tkzInit[xmin=-0.5,xmax=6.5, ymin=-0.5,ymax=4.5]
        \tkzClip
        \tkzDefPoint(0, 0){A}
        \tkzDefPoint(6, 1){B}
        \tkzDefPoint(1.5, 4){C}
        \tkzLabelPoint[below left](A){$A$}
        \tkzLabelPoint[right](B){$B$}
        \tkzLabelPoint[above](C){$C$}

        \tkzDefLine[mediator](A,B) \tkzGetPoints{AB1}{AB2}
        \tkzDefLine[orthogonal=through B](B,C) \tkzGetPoint{BC3}
        \tkzInterLL(AB1,AB2)(B,BC3) \tkzGetPoint{S1}
        %
        \tkzDefLine[mediator](B,C) \tkzGetPoints{BC1}{BC2}
        \tkzDefLine[orthogonal=through C](A,C) \tkzGetPoint{AC3}
        \tkzInterLL(BC1,BC2)(C,AC3) \tkzGetPoint{S2}
        %
        \tkzInterCC(S1,B)(S2,C) \tkzGetPoints{Bro1}{Bro2} % two circles
        \tkzLabelPoint[above right](Bro1){$X$}
        \tkzDrawSegments[line width=0.2mm](A,Bro1 B,Bro1 C,Bro1)
        \tkzFillAngle[fill=black!30,size=1](B,A,Bro1)
        \tkzFillAngle[fill=black!30,size=1](C,B,Bro1)
        \tkzFillAngle[fill=black!30,size=1](A,C,Bro1)
        \tkzDrawPolygon[line width=0.4mm](A,B,C)
        \tkzDrawPoints[size=3,color=black,fill=black!50](Bro1)
    \end{tikzpicture}
\end{comment}
    \end{center}
    % Construct a circle through vertices A and B, tangent to side BC of the triangle.
    % Symmetrically, construct the other two circles.
    % These three circles intersect at the first Brocard Point of triangle ABC.
    % https://www.geogebra.org/m/MT679Keu
\end{definition}

Miarę wspomnianych kątów nazywamy kątem Crelle'a-Brocarda.
\index{kąt!Crelle'a-Brocarda}%
Jest jeszcze drugi punkt Crelle'a-Brocarda, gdzie odwracamy kolejność punktów: $\angle XBA = \angle XCB = \angle XAC$.

\begin{proposition}
    W każdym trójkącie istnieje (jedyny) punkt Crelle'a-Brocarda.
    Miara kąta Crelle'a-Brocarda $\omega$ spełnia związek
    \begin{equation}
        \cot \omega = \cot \alpha + \cot \beta + \cot \gamma,
    \end{equation}
    gdzie $\alpha, \beta, \gamma$ to miary kątów trójkąta.
    Ponadto, $0 \le \omega \le \pi/6$.
\end{proposition}

Bogdańska, Neugebauer \cite[s. 100]{neugebauer_2018} podadzą jako ćwiczenie:

\begin{proposition}
    Niech $X$ będzie punktem Crelle'a-Brocarda trójkąta $\triangle ABC$.
    Niech $R_a$, $R_b$ i $R_c$ oznaczają promienie okręgów opisanych na trójkątach $\triangle XBC$, $\triangle AXC$, $\triangle ABX$, zaś $R$ będzie jak zwykle promieniem okręgu opisanego na trójkącie $\triangle ABC$.
    Wtedy
    \begin{equation}
        R = \sqrt[3]{R_a R_b R_c}.
    \end{equation}
\end{proposition}

Peter Yiff \cite{yff_1963} postawi w 1963 roku (!) hipotezę, że $8 \omega^3 \le \alpha \beta \gamma$.
\index[persons]{Yiff, Peter}%
Dowód znajdzie w 1974 roku Faruk Abi-Khuzam \cite{abikhuzam_1974}.
\index[persons]{Abi-Khuzam, Faruk}%

Mamy jeszcze mało ciekawy dla kogoś o wiedzy tak nikłej jak my okrąg Crelle'a-Brocarda.
% PRZECZYTANO: https://mathworld.wolfram.com/BrocardCircle.html
\index{okrąg!Crelle'a-Brocarda}%
Jego średnicą jest odcinek łączący środek okręgu opisanego z punktem Lemoine'a.
\index{okrąg!opisany}%
\index{punkt!Lemoine'a}%
Przechodzi przez pierwszy i drugi punkt Crelle'a-Brocarda oraz wierzchołki celowo niezdefiniowanego tu trójkąta Brocarda, co uzasadnia stosowaną czasami nazwę ,,okrąg siedmiu punktów''.
\index{okrąg!siedmiu punktów}%
Ma promień
\begin{equation}
    R \cdot \frac{\sqrt{1 - 4 \sin^2 \omega}}{2 \cos \omega}.
\end{equation}

%

\subsection{Problem Hansena}
Problem Hansena
\index{problem!Hansena}%

\subsection{Zastosowania trygonometrii -- Problem Snelliusa-Pothenota}
% TODO: https://en.wikipedia.org/wiki/Snellius-Pothenot_problem
Problem Snelliusa-Pothenota.
\index{problem!Snelliusa-Pothenota}%

% https://en.wikipedia.org/wiki/Skinny_triangle

% https://en.wikipedia.org/wiki/Rule_of_marteloio - napisać, że tego nie będzie

% https://en.wikipedia.org/wiki/Mollweide%27s_formula
% https://en.wikipedia.org/wiki/Mollweide's_formula
% https://en.wikipedia.org/wiki/Hansen%27s_problem

\index{trygonometria|)}

Twierdzenie Malfattiego.
Guzicki-11

\section{Bałagan}

\textbf{Twierdzenie Taylora, okrąg, sześciokąt}
% https://en.wikipedia.org/wiki/Taylor_circle
{
    \emph{WIP: Taylor w 1882 roku zauważył, że rzuty spodków wysokości na pozostałe boki leżą na jednym okręgu.}
	Hartshorne: s. 63
}

\textbf{Twierdzenie Eulera $1/4R^2$}

% https://en.wikipedia.org/wiki/Law_of_tangents


%