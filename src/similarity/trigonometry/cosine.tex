Twierdzenie cosinusów pojawia się już w Elementach Euklidesa jako (II.12) dla trójkątów rozwartokątnych i (II.13) dla trójkątów ostrokątnych.
% https://en.wikipedia.org/wiki/Law_of_cosines The cases of obtuse triangles and acute triangles (corresponding to the two cases of negative or positive cosine) are treated separately, in Propositions II.12 and II.13:[1]

\begin{proposition}[twierdzenie cosinusów]
	\label{twierdzenie_cosinusow}%
	\begin{equation}
		c^2 = a^2 + b^2 - 2ab \cos \gamma.
	\end{equation}
	% https://en.wikipedia.org/wiki/Law_of_cosines
\end{proposition}

Znamy różne dowody tego twierdzenia: korzystające z twierdzenia Pitagorasa, trzech wysokości trójkąta, twierdzenia Ptolemeusza, geometrii koła albo twierdzenia sinusów.

Perski matematyk Jamshid al-Kashi, autor najdokładniejszych tablic trygonometrycznych swoich czasów, obliczył $2\pi$ z dokładnością do szesnastu cyfr i podał równoważną postać wzoru,
\index[persons]{al-Kashi, Jamshid}
\begin{equation}
	c = \sqrt{(b - a \cos \gamma)^2 + (a \sin \gamma)^2}
\end{equation}
dla ostrego kąta $\gamma$.
(We Francji twierdzenie cosinusów do dzisiaj bywa nazywane twierdzeniem \emph{théorème d'Al-Kashi}).
Taka sama metoda rozwiązywania trójkątów pojawiła się w Europie w 1464 roku, kiedy Regiomontanus opublikował \emph{De triangulis omnimodis} (O trójkątach wszelkiego rodzaju).
\index[persons]{Regiomontanus}%
Współczesna forma twierdzenia to zasługa François Viète'a.
\index[persons]{Viète, François}%

\begin{theorem}[Stewarta, 1746]
\index{twierdzenie!Stewarta}
	Niech $a, b, c$ będą długościami boków trójkąta $ABC$, w którym poprowadzono czewianę $d$ z wierzchołka $C$ do boku $a$, dzieląc go na odcinki długości $n$ oraz $m$ (sąsiadujące odpowiednio z bokami $b$ oraz $c$).
	Wtedy
	\begin{equation}
		b^2 m + c^2 n = a (d^2 + mn).
	\end{equation}
\end{theorem}

Matthew Stewart opublikował to twierdzenie w 1746 roku, chociaż Coxeter przypuszcza, że mogło być znane nawet Archimedesowi.
\index[persons]{Stewart, Matthew}%
\index[persons]{Archimedes}%
% Coxeter, H.S.M.; Greitzer, S.L. (1967), Geometry Revisited, New Mathematical Library #19, The Mathematical Association of America, ISBN 0-88385-619-0 strona 6
Współcześnie często pokazuje się je jako zastosowanie twierdzenia cosinusów, tak jak Bogdańska, Neugebauer \cite[s. 90-91]{neugebauer_2018}.	

\begin{corollary}
	Niech $\triangle ABC$ będzie trójkątem o bokach długości $a$, $b$, $c$.
	Wtedy długość środkowej poprowadzonej z wierzchołka $C$ wynosi
	\begin{equation}
		\sqrt{\frac{a^2 + b^2}{2} - \frac{c^2}{4}},
	\end{equation}
	zaś dwusieczna stamtąd ma długość
	\begin{equation}
		\frac{\sqrt{ab (a+b+c)(a+b-c)}}{a+b}.
	\end{equation}
\end{corollary}