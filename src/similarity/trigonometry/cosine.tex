\begin{proposition}[twierdzenie cosinusów]
	\label{twierdzenie_cosinusow}%
	\begin{equation}
		c^2 = a^2 + b^2 - 2ab \cos \gamma.
	\end{equation}
	% https://en.wikipedia.org/wiki/Law_of_cosines
\end{proposition}

\begin{theorem}[Stewarta, 1746]
\index{twierdzenie!Stewarta}
	Niech $a, b, c$ będą długościami boków trójkąta $ABC$, w którym poprowadzono czewianę $d$ z wierzchołka $C$ do boku $a$, dzieląc go na odcinki długości $n$ oraz $m$ (sąsiadujące odpowiednio z bokami $b$ oraz $c$).
	Wtedy
	\begin{equation}
		b^2 m + c^2 n = a (d^2 + mn).
	\end{equation}
\end{theorem}

Matthew Stewart opublikował to twierdzenie w 1746 roku, chociaż Coxeter przypuszcza, że mogło być znane nawet Archimedesowi.
\index[persons]{Stewart, Matthew}%
\index[persons]{Archimedes}%
% Coxeter, H.S.M.; Greitzer, S.L. (1967), Geometry Revisited, New Mathematical Library #19, The Mathematical Association of America, ISBN 0-88385-619-0 strona 6
Współcześnie często pokazuje się je jako zastosowanie twierdzenia cosinusów, tak jak Bogdańska, Neugebauer \cite[s. 90-91]{neugebauer_2018}.	

\begin{corollary}
	Niech $\triangle ABC$ będzie trójkątem o bokach długości $a$, $b$, $c$.
	Wtedy długość środkowej poprowadzonej z wierzchołka $C$ wynosi
	\begin{equation}
		\sqrt{\frac{a^2 + b^2}{2} - \frac{c^2}{4}},
	\end{equation}
	zaś dwusieczna stamtąd ma długość
	\begin{equation}
		\frac{\sqrt{ab (a+b+c)(a+b-c)}}{a+b}.
	\end{equation}
\end{corollary}