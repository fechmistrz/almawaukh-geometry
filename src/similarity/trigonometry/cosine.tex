Twierdzenie cosinusów jest bardzo stare.
Euklides rozpatrzy je w Elementach osobno dla trójkątów rozwartokątnych (II.12) i ostrokątnych (II.13).
% TODO: https://en.wikipedia.org/wiki/Law_of_cosines The cases of obtuse triangles and acute triangles (corresponding to the two cases of negative or positive cosine) are treated separately, in Propositions II.12 and II.13:[1]
Perski matematyk Jamshid al-Kashi znajdzie wartość $2\pi$ z~dokładnością do szesnastu cyfr, będzie autorem najdokładniejszych tablic trygonometrycznych swoich czasów i poda równoważną postać wzoru,
\index[persons]{al-Kashi, Jamshid}
\begin{equation}
	c = \sqrt{(b - a \cos \gamma)^2 + (a \sin \gamma)^2}
\end{equation}
dla ostrego kąta $\gamma$.
Taka sama metoda rozwiązywania trójkątów pojawi się w Europie w 1464 roku, kiedy Regiomontanus opublikuje \emph{De triangulis omnimodis} (czyli ,,O trójkątach wszelkiego rodzaju'').
\index[persons]{Regiomontanus}%
Współczesna forma twierdzenia to zasługa François Viète'a.
\index[persons]{Viète, François}%

\begin{proposition}[twierdzenie cosinusów]
	W trójkącie o bokach długości $a, b, c$ z kątem $\gamma$ naprzeciwko krawędzi $c$ zachodzi
	\label{twierdzenie_cosinusow}%
	\begin{equation}
		c^2 = a^2 + b^2 - 2ab \cos \gamma.
	\end{equation}
	% https://en.wikipedia.org/wiki/Law_of_cosines
\end{proposition}

https://www.deltami.edu.pl/2017/04/niemozliwe-wycinanki/

Znamy różne dowody tego twierdzenia: korzystające z twierdzenia Pitagorasa, trzech wysokości trójkąta, twierdzenia Ptolemeusza, geometrii koła albo twierdzenia sinusów.
We Francji twierdzenie cosinusów do późna będzie nazywane \emph{théorème d'Al-Kashi}, na cześć wspomnianego wyżej Persa.

Piszą o nim Guzicki \cite[s. 258]{guzicki_2021}.

\begin{theorem}[Stewarta, 1746]
\index{twierdzenie!Stewarta}
	W trójkącie $\triangle ABC$ o bokach długości $a, b, c$ poprowadzono czewianę z~wierzchołka $C$ do boku $AB$ o długości $d$, dzieląc ten bok na odcinki długości $n$ oraz $m$, jak na rysunku:
	\begin{center}
\begin{comment}
    \begin{tikzpicture}[scale=.4]
        %\tkzInit[xmin=-0.5,xmax=6.5, ymin=-0.5,ymax=4.5]
        % \tkzClip
        \tkzDefPoint(0, 0){A}
		\tkzDefPoint(3.25, 3.25){d}

		\tkzDefPoint(6, 0){AB}
        \tkzDefPoint(10, 0){B}
        \tkzDefPoint(1, 7){C}
        \tkzDefPoint(35:4.75){CC}
		\tkzDrawSegments(C,AB)
        \tkzDrawPolygon[line width=0.3mm](A,B,C)

        \tkzLabelPoint[below left](A){$A$}
        \tkzLabelPoint[below right](B){$B$}
        \tkzLabelPoint[above](C){$C$}

        \tkzLabelPoint(d){$d$}

		\tkzDrawPoints[size=3,color=black,fill=black!80](A,B,C,AB)
		\tkzDrawSegment[dim={$\,\,c\,\,$,-16pt,transform shape}](A,B)
		\tkzDrawSegment[dim={$\,\,n\,\,$,-8pt,transform shape}](A,AB)
		\tkzDrawSegment[dim={$\,\,m\,\,$,-8pt,transform shape}](AB,B)
		\tkzDrawSegment[dim={$\,\,b\,\,$,8pt,transform shape,sloped}](A,C)
		\tkzDrawSegment[dim={$\,\,a\,\,$,-8pt,transform shape,sloped}](B,C)
    \end{tikzpicture}
\end{comment}
    \end{center}
	Wtedy
	\begin{equation}
		b^2 m + c^2 n = a (d^2 + mn).
	\end{equation}
\end{theorem}

Twierdzenie bez dowodu opublikuje Matthew Stewart w 1746 roku, chociaż Coxeter przypuści, że mogło być znane nawet Archimedesowi.
\index[persons]{Stewart, Matthew}%
\index[persons]{Archimedes}%
% Coxeter, H.S.M.; Greitzer, S.L. (1967), Geometry Revisited, New Mathematical Library #19, The Mathematical Association of America, ISBN 0-88385-619-0 strona 6
Dowody podadzą też Simpson, Euler, Carnot i pewnie jeszcze ktoś.
\index[persons]{Simpson, Thomas}%
\index[persons]{Euler, Leonhard}%
\index[persons]{Carnot, L.M.N.}%
W przyszłości często pokazywać się będzie je jako zastosowanie twierdzenia cosinusów, tak jak Guzicki \cite[s. 265]{guzicki_2021}, Bogdańska, Neugebauer \cite[s. 90-91]{neugebauer_2018}; inne rozwiązanie skorzysta z odcinków skierowanych.
Znalezienie go to ćwiczenie podane przez Evesa \cite[s. 58]{eves1_1972}.

\begin{corollary}[twierdzenie Apoloniusza]
	% https://en.wikipedia.org/wiki/Apollonius%27s_theorem
	The theorem is found as proposition VII.122 of Pappus of Alexandria's Collection (c. 340 AD). It may have been in Apollonius of Perga's lost treatise Plane Loci (c. 200 BC), and was included in Robert Simson's 1749 reconstruction of that work.[1]
\end{corollary}

\begin{corollary}
	W trójkącie $\triangle ABC$ o bokach długości $a, b, c$ poprowadzono środkową oraz dwusieczną z~wierzchołka $C$.
	Długość środkowej wynosi
	\begin{equation}
		\sqrt{\frac{a^2 + b^2}{2} - \frac{c^2}{4}},
	\end{equation}
	zaś dwusieczna ma długość
	\begin{equation}
		\frac{\sqrt{ab (a+b+c)(a+b-c)}}{a+b}.
	\end{equation}
\end{corollary}

\begin{corollary}[z twierdzenia Stewarta]
	Odległość między środkiem ciężkości i środkiem okręgu opisanego na trójkącie wynosi
	\begin{equation}
		\sqrt{R^2 - \frac 19 \left(a^2 + b^2 + c^2\right)},
	\end{equation}
	w szczególności więc $(3R)^2 \ge a^2 + b^2 + c^2$, z równością dla trójkąta równobocznego.
\end{corollary}

Wniosek ten znajdziemy u Zetela \cite[s. 72]{zetel_2020}