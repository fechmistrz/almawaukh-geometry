
Poznamy teraz dwa wyróżnione punkty trójkąta, opisane w 1875 roku przez oficera francuskiej armii, Henriego Brocarda oraz wiele lat wcześniej przez Augusta Crelle'a \cite{crelle_1816}, założyciela słynnego czasopisma matematycznego.
\index[persons]{Brocard, Henri}%
\index[persons]{Crelle, August}%

\begin{definition}
    Niech $\triangle ABC$ będzie trójkątem.
    Punkt $X$ leżący w jego wnętrzu taki, że kąty $\angle XAB$, $\angle XBC$, $\angle XCA$ są równej miary, nazywamy (pierwszym) punktem Crelle'a-Brocarda.
\end{definition}

\index{punkt Crelle'a-Brocarda}%
\index{kąt Crelle'a-Brocarda}%

Miarę wspomnianych kątów nazywamy kątem Crelle'a-Brocarda.
Jest jeszcze drugi punkt Crelle'a-Brocarda, gdzie odwracamy kolejność punktów: $\angle XBA = \angle XCB = \angle XAC$.

\begin{proposition}
    W każdym trójkącie istnieje (jedyny) punkt Crelle'a-Brocarda.
    Miara kąta Crelle'a-Brocarda $\omega$ spełnia związek
    \begin{equation}
        \cot \omega = \cot \alpha + \cot \beta + \cot \gamma,
    \end{equation}
    gdzie $\alpha, \beta, \gamma$ to miary kątów trójkąta.
    Ponadto, $0 \le \omega \le \pi/6$.
\end{proposition}

Bogdańska, Neugebauer \cite[s. 100]{neugebauer_2018} podają jako ćwiczenie:

\begin{proposition}
    Niech $X$ będzie punktem Crelle'a-Brocarda trójkąta $\triangle ABC$.
    Niech $R_a$, $R_b$ i $R_c$ oznaczają promienie okręgów opisanych na trójkątach $\triangle XBC$, $\triangle AXC$, $\triangle ABX$, zaś $R$ będzie jak zwykle promieniem okręgu opisanego na trójkącie $\triangle ABC$.
    Wtedy
    \begin{equation}
        R = \sqrt[3]{R_a R_b R_c}.
    \end{equation}
\end{proposition}

Peter Yiff \cite{yff_1963} postawił w 1963 roku (!) hipotezę, że $8 \omega^3 \le \alpha \beta \gamma$.
\index[persons]{Yiff, Peter}%
Dowód znalazł w 1974 roku Faruk Abi-Khuzam \cite{abikhuzam_1974}.
\index[persons]{Abi-Khuzam, Faruk}%
% P. Yff, "An analogue of the Brocard points" Amer. Math. Monthly , 70 (1963) pp. 495–501
% F. Abi–Khuzam, "Proof of Yff's conjecture on the Brocard angle of a triangle" Elem. Math. , 29 (1974) pp. 141–142


\todofoot{https://mathworld.wolfram.com/BrocardCircle.html}