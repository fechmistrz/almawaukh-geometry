%

Poznamy teraz dwa wyróżnione punkty trójkąta, opisane w 1875 roku przez oficera francuskiej armii, Henriego Brocarda oraz wiele lat wcześniej przez Augusta Crelle'a \cite{crelle_1816}, założyciela słynnego czasopisma matematycznego.
\index[persons]{Brocard, Henri}%
\index[persons]{Crelle, August}%

\begin{definition}
    Niech $\triangle ABC$ będzie trójkątem.
    Punkt $X$ leżący w jego wnętrzu taki, że kąty $\angle XAB$, $\angle XBC$, $\angle XCA$ są równej miary, nazywamy (pierwszym) punktem Crelle'a-Brocarda.
    \index{punkt!Crelle'a-Brocarda}%
    \begin{center}
\begin{comment}
    \begin{tikzpicture}[scale=.75]
        \tkzInit[xmin=-0.5,xmax=6.5, ymin=-0.5,ymax=4.5]
        \tkzClip
        \tkzDefPoint(0, 0){A}
        \tkzDefPoint(6, 1){B}
        \tkzDefPoint(1.5, 4){C}
        \tkzLabelPoint[below left](A){$A$}
        \tkzLabelPoint[right](B){$B$}
        \tkzLabelPoint[above](C){$C$}

        \tkzDefLine[mediator](A,B) \tkzGetPoints{AB1}{AB2}
        \tkzDefLine[orthogonal=through B](B,C) \tkzGetPoint{BC3}
        \tkzInterLL(AB1,AB2)(B,BC3) \tkzGetPoint{S1}
        %
        \tkzDefLine[mediator](B,C) \tkzGetPoints{BC1}{BC2}
        \tkzDefLine[orthogonal=through C](A,C) \tkzGetPoint{AC3}
        \tkzInterLL(BC1,BC2)(C,AC3) \tkzGetPoint{S2}
        %
        \tkzInterCC(S1,B)(S2,C) \tkzGetPoints{Bro1}{Bro2} % two circles
        \tkzLabelPoint[above right](Bro1){$X$}
        \tkzDrawSegments[line width=0.2mm](A,Bro1 B,Bro1 C,Bro1)
        \tkzFillAngle[fill=black!30,size=1](B,A,Bro1)
        \tkzFillAngle[fill=black!30,size=1](C,B,Bro1)
        \tkzFillAngle[fill=black!30,size=1](A,C,Bro1)
        \tkzDrawPolygon[line width=0.4mm](A,B,C)
        \tkzDrawPoints[size=3,color=black,fill=black!50](Bro1)
    \end{tikzpicture}
\end{comment}
    \end{center}
    % Construct a circle through vertices A and B, tangent to side BC of the triangle.
    % Symmetrically, construct the other two circles.
    % These three circles intersect at the first Brocard Point of triangle ABC.
    % https://www.geogebra.org/m/MT679Keu
\end{definition}

Miarę wspomnianych kątów nazywamy kątem Crelle'a-Brocarda.
\index{kąt!Crelle'a-Brocarda}%
Jest jeszcze drugi punkt Crelle'a-Brocarda, gdzie odwracamy kolejność punktów: $\angle XBA = \angle XCB = \angle XAC$.

\begin{proposition}
    W każdym trójkącie istnieje (jedyny) punkt Crelle'a-Brocarda.
    Miara kąta Crelle'a-Brocarda $\omega$ spełnia związek
    \begin{equation}
        \cot \omega = \cot \alpha + \cot \beta + \cot \gamma,
    \end{equation}
    gdzie $\alpha, \beta, \gamma$ to miary kątów trójkąta.
    Ponadto, $0 \le \omega \le \pi/6$.
\end{proposition}

Bogdańska, Neugebauer \cite[s. 100]{neugebauer_2018} podają jako ćwiczenie:

\begin{proposition}
    Niech $X$ będzie punktem Crelle'a-Brocarda trójkąta $\triangle ABC$.
    Niech $R_a$, $R_b$ i $R_c$ oznaczają promienie okręgów opisanych na trójkątach $\triangle XBC$, $\triangle AXC$, $\triangle ABX$, zaś $R$ będzie jak zwykle promieniem okręgu opisanego na trójkącie $\triangle ABC$.
    Wtedy
    \begin{equation}
        R = \sqrt[3]{R_a R_b R_c}.
    \end{equation}
\end{proposition}

Peter Yiff \cite{yff_1963} postawił w 1963 roku (!) hipotezę, że $8 \omega^3 \le \alpha \beta \gamma$.
\index[persons]{Yiff, Peter}%
Dowód znalazł w 1974 roku Faruk Abi-Khuzam \cite{abikhuzam_1974}.
\index[persons]{Abi-Khuzam, Faruk}%

Mamy jeszcze mało ciekawy dla kogoś o wiedzy tak nikłej jak my okrąg Crelle'a-Brocarda.
% PRZECZYTANO: https://mathworld.wolfram.com/BrocardCircle.html
\index{okrąg!Crelle'a-Brocarda}%
Jego średnicą jest odcinek łączący środek okręgu opisanego z punktem Lemoine'a.
\index{okrąg!opisany}%
\index{punkt!Lemoine'a}%
Przechodzi przez pierwszy i drugi punkt Crelle'a-Brocarda oraz wierzchołki celowo niezdefiniowanego tu trójkąta Brocarda, co uzasadnia stosowaną czasami nazwę ,,okrąg siedmiu punktów''.
\index{okrąg!siedmiu punktów}%
Ma promień
\begin{equation}
    R \cdot \frac{\sqrt{1 - 4 \sin^2 \omega}}{2 \cos \omega}.
\end{equation}

%