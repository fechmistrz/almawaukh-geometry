
Podamy teraz kilka zależności trygonometrycznych, które pomagają w rozwiązywaniu trójkątów.

\begin{proposition}
    Niech $S$ oznacza pole trójkąta o kątach $\alpha, \beta, \gamma$ i promieniu koła opisanego $R$.
    Wtedy
    \begin{equation}
    S = 2 R^2 \sin \alpha \sin \beta \sin \gamma.
    \end{equation}
\end{proposition}

\begin{proposition}
    Niech $p$ oznacza połowę obwodu trójkąta o kątach $\alpha, \beta, \gamma$ i promieniu koła opisanego $R$.
    Wtedy
    \begin{equation}
        p = R (\sin \alpha + \sin \beta + \sin \theta).
    \end{equation}
\end{proposition}

\begin{proposition}
    Niech $r$ oznacza promień koła wpisanego w trójkąt o kątach $\alpha, \beta, \gamma$ i promieniu koła opisanego $R$.
    Wtedy
    \begin{equation}
        r = 4R \sin \frac \alpha 2 \sin \frac \beta 2 \sin \frac \gamma 2.
    \end{equation}
\end{proposition}

\begin{proposition}[wzór Newtona]
\index{wzór!Newtona}%
    Przy domyślnych oznaczeniach, zachodzi:
    \begin{equation}
        \frac{a + b}{c} = \frac{\cos \frac 1 2 (\alpha - \beta)}{\cos \frac 1 2 (\alpha + \beta)}.
    \end{equation}
\end{proposition}

\begin{proposition}[wzór Mollweidego]
\index{wzór!Mollweidego}%
    Przy domyślnych oznaczeniach, zachodzi:
    \begin{equation}
        \frac{a - b}{c} = \frac{\sin \frac 1 2 (\alpha - \beta)}{\sin \frac 1 2 (\alpha + \beta)}.
    \end{equation}
\end{proposition}

(Nie ma zgodności co do powyższych nazw).
Nieco bardziej geometryczną wersję wzorów podał Izaak Newton w 1707 roku, potem Friedrich von Oppel w 1746.
\index[persons]{Newton, Izaak}%
\index[persons]{von Oppel, Friedrich}%
Wyrażenia, których używamy po dziś dzień zawdzięczamy Thomasowi Simpsonowi z 1748 roku, oraz Karlowi Mollweidemu, który opublikował to samo w 1808 roku bez cytowania poprzedników.
\index[persons]{Simpson, Thomas}%
\index[persons]{Mollweide, Karl}%

\begin{proposition}[wzór Regiomontanusa]
\index{wzór!Regiomontanusa}%
    Przy domyślnych oznaczeniach, zachodzi:
    \begin{equation}
        \frac{a + b}{a- b} = \frac{\tan \frac 1 2 (\alpha + \beta)}{\tan \frac 1 2 (\alpha - \beta)}.
    \end{equation}
\end{proposition}

Regiomontanus był znany gorzej jako Johannes Müller von Königsberg.
\index[persons]{von Königsberg, Johannes}
\index[persons]{Regiomontanus}
