%

Twierdzenie odkryje Abu al-Wafa Buzjani w X wieku oraz niezależnie Ibn Mu'adh al-Jayyani (arabski matematyk, autor pierwszego traktatu o trygonometrii sferycznej) w XI wieku.
\index[persons]{Buzjani, Abu al-Wafa}%
\index[persons]{Ibn Mu'adh al-Jayyani}%
% The law of tangents for spherical triangles was described in the 13th century by Persian mathematician Nasir al-Din al-Tusi (1201–1274), who also presented the law of sines for plane triangles in his five-volume work Treatise on the Quadrilateral.
Po polsku często padać będzie nazwa wzór Regiomontana.

\begin{proposition}[twierdzenie tangensów]
	W trójkącie o bokach długości $a, b, c$ z kątami $\alpha$, $\beta$ naprzeciwko krawędzi $a$, $b$ zachodzi
	\label{twierdzenie_cosinusow}%
	\begin{equation}
		\frac{a-b}{a+b} = \frac{\tan \frac 1 2 (\alpha - \beta)}{\tan \frac 1 2 (\alpha + \beta)}.
	\end{equation}
\end{proposition}

Prosty dowód wykorzystuje twierdzenie sinusów.
Inny korzysta ze wzorów Mollweidege (to jest faktu \ref{wzor_mollweidego}), które wyprowadza się i tak z twierdzenia sinusów.

Dopóki nie zostanie wynaleziony komputer (albo chociaż kalkulator elektroniczny), twierdzenie tangensów będzie mieć przewagę nad twierdzeniem kosinusów, ponieważ nie wymaga sięgania po tablice logarytmiczne do znalezienia pierwiastka
\begin{equation}
c = \sqrt{a^2 + b^2 - 2 a b \cos \gamma}.
\end{equation}

Po wynalezieniu rzeczonej maszyny, przewaga nie zniknie.
Twierdzenie tangensów pozwala uniknąć błędów numerycznych, kiedy $a \approx b$, zaś kąt $\gamma$ jest bardzo mały.

%