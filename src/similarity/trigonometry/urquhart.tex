\begin{theorem}[Urquharta?]
\index{twierdzenie!Urquharta}%
    Przy oznaczeniach jak na rysunku, niech $|AB| + |BC| = |AD| + |CD|$.
    Wtedy $|AE| + |CE| = |AF| + |CF|$.
\end{theorem}

Dan Pedoe
% Pedoe, D. "The Most 'Elementary' Theorem of Euclidean Geometry." Math. Mag. 49, 40-42, 1976.
przypisuje to twierdzenie Malcolmowi Urquhartowi\footnote{Matematyk australijski, żył w latach 1902-1966 i nie opublikował żadnej pracy}.
\index{Urquhart, Malcolm}%
Wiemy jednak, że de Morgan opublikował swój dowód już w 1841 roku; samo zaś twierdzenie jest przypadkiem granicznym innego wyniku Chaslesa, znanego w latach 186x.
Poznaliśmy je, tak jak wiele innych, z książki Bogdańskiej, Neugebauera \cite[s. 97]{neugebauer_2018}.