%

\begin{theorem}[Urquharta?]
\index{twierdzenie!Urquharta}%
    Przy oznaczeniach jak na rysunku:
    \begin{center}\begin{tikzpicture}[scale=.4]
        \tkzDefPoint(0, 0){E}
        \tkzDefPoint(267:5){E1}
        \tkzDefPoint(250:5){E2}
        \tkzDefPoint(165:5){E3}
        \tkzDefPoint(135:5){E4}
        
        \tkzDefLine[tangent at=E1](E) \tkzGetPoint{tan1}
        \tkzDefLine[tangent at=E2](E) \tkzGetPoint{tan2}
        \tkzDefLine[tangent at=E3](E) \tkzGetPoint{tan3}
        \tkzDefLine[tangent at=E4](E) \tkzGetPoint{tan4}
        
        \tkzInterLL(E4,tan4)(E1,tan1) \tkzGetPoint{S1}
        \tkzInterLL(E4,tan4)(E2,tan2) \tkzGetPoint{S2}
        \tkzInterLL(E3,tan3)(E1,tan1) \tkzGetPoint{S3}
        \tkzInterLL(E3,tan3)(E2,tan2) \tkzGetPoint{S4}

        \tkzInterLL(S3,S4)(S1,S2) \tkzGetPoint{Gorny}
        \tkzInterLL(S1,S3)(S2,S4) \tkzGetPoint{Dolny}

        \tkzLabelPoint[below](S1){$A$}
        \tkzLabelPoint[above left](S2){$D$}
        \tkzLabelPoint(S3){$B$}
        \tkzLabelPoint[above right](S4){$C$}
        \tkzLabelPoint(Dolny){$E$}
        \tkzLabelPoint[right](Gorny){$F$}

        \tkzDrawSegments[line width=0.2mm](S1,Dolny S1,Gorny S3,Gorny S2,Dolny)
        \tkzDrawPoints[size=3,color=black,fill=black!50](S1,S2,S3,S4,Dolny,Gorny)
        % konstrukcja z https://en.wikipedia.org/wiki/Ex-tangential_quadrilateral
\end{tikzpicture}\end{center}
    niech $|AB| + |BC| = |AD| + |CD|$.
    Wtedy $|AE| + |CE| = |AF| + |CF|$.
\end{theorem}

Dan Pedoe
% Pedoe, D. "The Most 'Elementary' Theorem of Euclidean Geometry." Math. Mag. 49, 40-42, 1976.
przypisuje to twierdzenie Malcolmowi Urquhartowi\footnote{Matematyk australijski, żył w latach 1902-1966 i nie opublikował żadnej pracy}.
\index{Urquhart, Malcolm}%
Wiemy jednak, że de Morgan opublikował swój dowód już w 1841 roku; samo zaś twierdzenie jest przypadkiem granicznym innego wyniku Chaslesa, znanego w latach 186x.
Poznaliśmy je, tak jak wiele innych, z książki Bogdańskiej, Neugebauera \cite[s. 97]{neugebauer_2018}.

%