

Wszystkie ściany wielościanów archimedesowych są wielokątami foremnymi, ale mogą występować w dwóch lub trzech rodzajach.
Układ ścian we wszystkich wierzchołkach jest taki sam.
Istnieje 13 unikatowych wielościanów archimedesowych:
\begin{enumerate}
\item czworościan ścięty (truncated tetrahedron)
\item sześcian ścięty (truncated cube)
\item sześcio-ośmiościan (cuboctahedron)
\item ośmiościan ścięty (truncated octahedron)
\item dwunastościan ścięty (truncated dodecahedron)
\item dwudziesto-dwunastościan (icosidodecahedron)
\item dwudziestościan ścięty (truncated icosahedron)
\item sześcio-ośmiościan rombowy mały (rhombicuboctahedron)
\item sześcio-ośmiościan rombowy wielki (rhombitruncated cuboctahedron)
\item sześcio-ośmiościan przycięty (snub cuboctahedron)
\item dwudziesto-dwunastościan rombowy mały (rhombicosidodecahedron)
\item dwudziesto-dwunastościan rombowy wielki (rhombitruncated icosidodecahedron)
\item dwudziesto-dwunastościan przycięty (snub icosidodecahedron).
\end{enumerate}
Niektórzy zaliczają do tej kategorii jeszcze dwie nieskończone serie wypukłych wielościanów półforemnych:
\begin{itemize}
\item płóforemne graniastosłupy (o ścianach bocznych będących kwadratami)
\item półforemne antygraniastosłupy (o ścianach bocznych będących trójkątami równobocznymi).
\end{itemize} 

Niektórzy (wiele osób) mieszają wielościany archimedesowe z półforemnymi.

\begin{definition}[wielościan półforemny]
    Trzynaście wielościanów archimedesowych i dwie nieskończone serie: graniastosłupów oraz antygraniastosłupów, a czasami także $J_{37}$ (np. zdaniem Branko Grünbauma) nazywamy wielościanami półforemnymi.
\end{definition}

Pierwsza definicja mówiła, że wielościan ma mieć foremne ściany, a jego grupa symetrii ma działać tranzytywnie na wierzchołkach; później takie obiekty nazwie się jednorodnymi.
Wielościan $J_{37}$ ma identyczne naroża, ale przez skręcenie nie ma tranzytywności (i zapewne przez to pojawił się w druku dopiero w 1905 roku, a potem został przez przypadek odkryty na nowo: przez Jeffreya Charlesa Percy'ego Millera w 1930 i niezależnie przez Ashkinuze)
% TODO: Ashkinuse, V.G.: On the number of semiregular polyhedra. -> rosyjski mathscinet

% http://matematyka.wroc.pl/book/historia-1
% http://matematyka.wroc.pl/book/wielościany-archimedesowe-platońskie