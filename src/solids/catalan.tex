%

\begin{definition}[wielościan dualny]
    W ustalonym wielościanie usuńmy wierzchołki i krawędzie, zastąpmy każdą ścianę jej środkiem, po czym połączmy te punkty, które należały do sąsiednich ścian.
    Tak otrzymana bryła nazywa się wielościanem dualnym.
\end{definition}

Bryły dualne do wielościanów platońskich są platońskie, a dualne do Keplera-Poinsota są bryłami Keplera-Poinsota, więc nie dają żadnych nowych obiektów

\begin{definition}[wielościan Catalana]
    Wielościan dualny do wielościanu archimedesowego nazywamy wielościanem Catalana.
\end{definition}

Opisze je Eugene Catalan w 1865 roku.
% Diudea (2018), p. 39 // Heil & Martini (1993), p. 352
% The Catalan solids are face-transitive or isohedral meaning that their faces are symmetric to one another, but they are not vertex-transitive because their vertices are not symmetric. Their dual, the Archimedean solids, are vertex-transitive but not face-transitive. Each Catalan solid has constant dihedral angles, meaning the angle between any two adjacent faces is the same.[1] Additionally, two Catalan solids, the rhombic dodecahedron and rhombic triacontahedron, are edge-transitive, meaning their edges are symmetric to each other.[citation needed] Some Catalan solids were discovered by Johannes Kepler during his study of zonohedra, and Eugene Catalan completed the list of the thirteen solids in 1865.[3]
Są to\footnote{\raggedright{Po angielsku: triakistetrahedron, triakisoctahedron, triakisicosahedron, tetrakishexahedron, pentakisdodecahedron, hexakisoctahedron, hexakisicosahedron, rhombic dodecahedron, rhombic triacontahedron, strombic hexecontahedron, strombic icositetrahedron, pentagonal icositetrahedron, pentagonal hexecontahedron}}:
 
\begin{itemize}
\item (czworościan, ośmiościan lub dwudziestościan) potrójny,
\item sześciościan poczwórny,
\item dwunastościan piątkowy,
\item (ośmiościan lub dwudziestościan) szóstkowy,
\item (dwunastościan lub trzydziestościan) rombowy,
\item (sześćdziestościan lub dwudziestoczterościan) deltoidowy,
\item (dwudziestoczterościanl lub sześćdziestościan) pięciokątny.
\end{itemize}
% TODO: sześćdziestościan deltoidowy (strombic hexecontahedron), % Taką nazwę zaproponował prof. Roman Duda, tłumacząc ponad 40 lat temu książkę "Modele matematyczne" Cundy'ego i Rolleta i tak już zostało, choć rzeczywiście to trochę dziwnie brzmi.

%