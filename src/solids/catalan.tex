%

\begin{definition}[wielościan dualny]
    W ustalonym wielościanie usuńmy wierzchołki i krawędzie, zastąpmy każdą ścianę jej środkiem, po czym połączmy te punkty, które należały do sąsiednich ścian.
    Tak otrzymana bryła nazywa się wielościanem dualnym.
\end{definition}

Bryły dualne do wielościanów platońskich są platońskie, a dualne do Keplera-Poinsota są bryłami Keplera-Poinsota, więc nie dają żadnych nowych obiektów

\begin{definition}[wielościan Catalana]
    Wielościan dualny do wielościanu archimedesowego nazywamy wielościanem Catalana.
\end{definition}

Opisze je Eugene Catalan w 1865 roku.
% Diudea (2018), p. 39 // Heil & Martini (1993), p. 352
Są to\footnote{\raggedright{Po angielsku: triakistetrahedron, triakisoctahedron, triakisicosahedron, tetrakishexahedron, pentakisdodecahedron, hexakisoctahedron, hexakisicosahedron, rhombic dodecahedron, rhombic triacontahedron, strombic hexecontahedron, strombic icositetrahedron, pentagonal icositetrahedron, pentagonal hexecontahedron}}:
 
\begin{itemize}
\item (czworościan, ośmiościan lub dwudziestościan) potrójny,
\item sześciościan poczwórny,
\item dwunastościan piątkowy,
\item (ośmiościan lub dwudziestościan) szóstkowy,
\item (dwunastościan lub trzydziestościan) rombowy,
\item (sześćdziestościan lub dwudziestoczterościan) deltoidowy,
\item (dwudziestoczterościanl lub sześćdziestościan) pięciokątny.
\end{itemize}
% TODO: sześćdziestościan deltoidowy (strombic hexecontahedron), % Taką nazwę zaproponował prof. Roman Duda, tłumacząc ponad 40 lat temu książkę "Modele matematyczne" Cundy'ego i Rolleta i tak już zostało, choć rzeczywiście to trochę dziwnie brzmi.

%