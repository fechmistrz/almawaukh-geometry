%

Wielościany Catalana to bryły dualne do archimedesowych. Takie bryły po raz pierwszy zostały opisane w 1865 roku przez belgijskiego matematyka Eugene Catalana - stąd ich nazwa.
Okazuje się, że bryły dualne do wielościanów platońskich są platońskie, a dualne do Keplera-Poinsota są bryłami Keplera-Poinsota, więc nie dają żadnych nowych obiektów. Dają je dopiero wielościany archimedesowe.
Ponieważ dla danego wielościanu jego bryła dualna wyznaczona jest jednoznacznie, mamy 13 wielościanów Catalana, bo jest 13 wielościanów archimedesowych. Są to:
 
\begin{itemize}
\item czworościan potrójny (triakistetrahedron),
\item ośmiościan potrójny (triakisoctahedron),
\item dwudziestościan potrójny (triakisicosahedron),
\item sześciościan poczwórny (tetrakishexahedron),
\item dwunastościan piątkowy (pentakisdodecahedron),
\item ośmiościan szóstkowy (hexakisoctahedron),
\item dwudziestościan szóstkowy (hexakisicosahedron),
\item dwunastościan rombowy (rhombic dodecahedron),
\item trzydziestościan rombowy (rhombic triacontahedron),
\item sześćdziestościan deltoidowy (strombic hexecontahedron),
\item dwudziestoczterościan deltoidowy (strombic icositetrahedron),
\item dwudziestoczterościan pięciokątny (pentagonal icositetrahedron),
\item sześćdziestościan pieciokątny (pentagonal hexecontahedron).
\end{itemize}
% sześćdziestościan deltoidowy (strombic hexecontahedron), % Taką nazwę zaproponował prof. Roman Duda, tłumacząc ponad 40 lat temu książkę "Modele matematyczne" Cundy'ego i Rolleta i tak już zostało, choć rzeczywiście to trochę dziwnie brzmi.

%