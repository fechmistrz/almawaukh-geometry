\subsubsection{Deltościany}
Jest osiem wielościanów wypukłych, których ściany są trójkątami równobocznymi: trzy platońskie bryły (czworościan, ośmiościan i dwudziestościan foremny) oraz pięć wielościanów Johnsona ($J_{12}$, $J_{13}$, $J_{17}$, $J_{51}$, $J_{84}$).
Pokażą to w 1947 roku Hans Freudenthal i Bartel Leendert van der Waerden \cite{Freudenthal_1947} w obskurnym duńskim żurnalu.
% Freudenthal, H.; van der Waerden, B. L. (1947), "On an assertion of Euclid", Simon Stevin, 25: 115–121, MR 0021687
% https://mathscinet.ams.org/mathscinet/relay-station?mr=0021687
Ich wynik zreferuje później Adam Gajda w $\Delta_{84}^{4}$.

Czasami nazywa się je deltościanami, przez podobieństwo litery $\Delta$ do ich ścian.
Niewypukłych brył o tej własności jest nieskończnie wiele (!), mogą mieć dowolną parzystą liczbę ścian większą niż sześć.

% https://en.wikipedia.org/wiki/Deltahedron  % http://matematyka.wroc.pl/book/deltosciany
\section{Walce i stożki}
% https://en.wikipedia.org/wiki/Cylinder

\subsubsection{Sześciościany}
Jest siedem sześciościanów wypukłych (i trzy niewypukłe).
Heinz Schumann oraz Bronisław Pabich napiszą krótki artykuł w $\Delta_{24}^{11}$, gdzie uzasadnią te liczby diagramami Schlegela.
\index{diagram Schlegela}%
(Wielościany wypukłe mają 6/0/0, 5/0/1, 4/2/0, 3/2/1, 2/4/0, 2/2/2, 0/6/0 ścian o trzech, czterech, pięciu krawędziach).
% TODO: DELTA 2001 luty Kordos

% Ilejesttypówczworościanów? Dziwnepytanie,oczywiścienieskończeniewiele.Sąbardziejspłaszczone, wydłużone,szpiczaste,sąteżprawidłowe,foremne Jeślijednakpominiemy długościkrawędzi,kątypłaskieidwuścienneitp.,zachowamytylkoogólną strukturę,towidzimy,żejesttylkojedentypczworościanu.Nieistnieje czworościan,którymiałbyścianęnietrójkątną.Wprzypadkupięciościanów mamydwatypy:typostrosłupaopodstawieczworokątaoraztypgraniastosłupa opodstawietrójkąta.Pierwszymajednąścianęczworokątnąiczterytrójkątne, adrugitrzyczworokątneidwietrójkątne.Innychniema.Gdyzetniemy wierzchołekczworościanu,todostaniemypięciościandrugiegotypu.Powołując sięnaJohnaMcClelana(artystęzWoodstock),MartinGardnerzadałpytanie: ilejesttypówsześciościanów(wdomyślewypukłych)? Należyuważaćnaprzypadkiwyglądającenapozórróżnie,ajednakdające tensamtyp.Możnapójśćdalejizapytaćoliczbętypówsiedmiościanów, ośmiościanówitd.Możejestjakiśogólnyschematpostępowania? Gardnerprzytacza7różnychtypówwypukłychsześciościanów,dodając,iż nieznaprostegodowodu,żeniemainnych.Informujeteż,żeistnieją34rodzaje wypukłychsiedmiościanów,257ośmiościanówi2606dziewięciościanów.Natomiast niewypukłychsześciościanówmamytrzytypy, 26siedmiościanówi277ośmiościanów delta 2011-01 Dowódtwierdzenia,żeistniejedokładnie siedemwypukłychsześciościanów,można znaleźćwpracyDonaldaCrowe’a,Euler’s formulaforpolyhedraandrelatedtopics w:A.Beck,M.Bleicher,D.Crowe, ExcursionsintoMathematics,Worth, 1969,str.29–30. Dlaspecjalistówpodamyprofesjonalne źródła: P.J.Federico, Enumerationofpolyhedra:Thenumber of 9-hedra,J.Combin.Theory7(1969), 155–161; Polyhedrawith4or8faces,Geom. Dedicata3(1975),469–481; Thenumberofpolyhedra,PhilipsRes. Rep.30(1975),220–231.