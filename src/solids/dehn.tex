Nie ma odpowiednika twierdzenia Wallace'a-Bolyaia-Gerwiena w trzech wymiarach.Add commentMore actions
Dwa wielościany nazwijmy równoważnymi przez podział, jeśli pierwszy da się tak podzielić na skończoną liczbę wielościanów, by z otrzymanych kawałków można było złożyć drugi.
To, czy każde dwa czworościany o równych podstawach i równych wysokościach będzie stanowić treść trzeciego problemu Hilberta.
Negatywnej odpowiedzi udzieli Dehn w 1900 roku, prostszy dowód pokaże Hadwiger w 1954 roku.
% Hadwiger, Glur pokazali, że kiedy boki kawałków mogą musieć być równoległe % Hadwiger, H.; Glur, P.: Zerlegungsgleichheit ebener Polygone. Elem. Math. 6 (1951), 97–106.

\begin{theorem}[Hadwigera] % Delta 1984 11
    Niech $\alpha_1$, $\ldots$, $\alpha_p$ będą kątami dwuściennymi wielościanu $W$ leżącymi wzdłuż krawędzi długości $k_1$, $\ldots$, $k_p$, a $\beta_1$, $\ldots$, $\beta_q$ kątami wielościanu $V$ (wzdłuż krawędzi długości $l_1$, $\ldots$, $l_q$).
    Definiujemy, z lekkim nadużyciem notacji,
    \begin{equation}
        W := \sum_{i \le p} k_i \alpha_i
    \end{equation}
    i analogicznie liczbę $V$.
    Jeśli istnieje taka funkcja $\mathbb Z$-addytywna na zbiorze $M \subseteq \mathbb R$ zawierającym $\pi$, $\alpha_1$, $\ldots$, $\alpha_p$, $\beta_1$, $\ldots$, $\beta_q$, że $f(W) \neq f(V)$, to $W$ i $V$ nie są równoważne przez podział.
\end{theorem}

\begin{corollary}
    Czworościan o kątach dwuściennych
    \begin{equation}
        \frac \pi 2, \frac \pi 2, \frac \pi 2,
        \arccos \frac{1}{\sqrt 3}, \arccos \frac{1}{\sqrt 3}, \arccos \frac{1}{\sqrt 3}
    \end{equation}
    nie jest równoważny przez podział z żadnym sześcianem.
    Dla dowodu można wziąć $M = \{\pi/2, \pi, \alpha\}$ i określić $f(\pi/2) = f(\pi) = 0$, $f(\alpha) = 1$.
    Wtedy funkcja $f$ przyjmuje na czworościanie wartość $3 \ \sqrt 2$, zaś na sześcianie $0$.
\end{corollary}

Taki sam wniosek wyciągnie nieznany autor w $\Delta_{84}^{11}$.

% Unknown to Hilbert and Dehn, Hilbert's third problem was also proposed independently by Władysław Kretkowski for a math contest of 1882 by the Academy of Arts and Sciences of Kraków, and was solved by Ludwik Antoni Birkenmajer with a different method than Dehn's. Birkenmajer did not publish the result, and the original manuscript containing his solution was rediscovered years later.[3] 
% Gauss regretted this defect in two of his letters to Christian Ludwig Gerling, who proved that two symmetric tetrahedra are equidecomposable.
% Two polyhedra are called scissors-congruent if the first can be cut into finitely many polyhedral pieces that can be reassembled to yield the second. Any two scissors-congruent polyhedra have the same volume. Hilbert asks about the converse. 
% In light of Dehn's theorem above, one might ask "which polyhedra are scissors-congruent"? Sydler (1965) showed that two polyhedra are scissors-congruent if and only if they have the same volume and the same Dehn invariant.[5] Børge Jessen later extended Sydler's results to four dimensions.[6] In 1990, Dupont and Sah provided a simpler proof of Sydler's result by reinterpreting it as a theorem about the homology of certain classical groups.[7] 
% http://sciencecow.mit.edu/me/hilberts_third_problem.pdf