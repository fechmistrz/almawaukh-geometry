Jest osiem wielościanów wypukłych, których ściany są trójkątami równobocznymi: trzy platońskie bryły (czworościan, ośmiościan i dwudziestościan foremny) oraz pięć wielościanów Johnsona ($J_{12}$, $J_{13}$, $J_{17}$, $J_{51}$, $J_{84}$).
Pokażą to w 1947 roku Hans Freudenthal i Bartel Leendert van der Waerden \cite{Freudenthal_1947} w obskurnym duńskim żurnalu.
% Freudenthal, H.; van der Waerden, B. L. (1947), "On an assertion of Euclid", Simon Stevin, 25: 115–121, MR 0021687
% https://mathscinet.ams.org/mathscinet/relay-station?mr=0021687
Ich wynik zreferuje później Adam Gajda w $\Delta_{84}^{4}$.

Czasami nazywa się je deltościanami, przez podobieństwo litery $\Delta$ do ich ścian.
Niewypukłych brył o tej własności jest nieskończnie wiele (!), mogą mieć dowolną parzystą liczbę ścian większą niż sześć.

% https://en.wikipedia.org/wiki/Deltahedron 