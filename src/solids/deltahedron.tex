Jest osiem wielościanów wypukłych, których ściany są trójkątami równobocznymi (czasami nazywa się je deltaścianami, przez podobieństwo litery $\Delta$ do ich ścian):
\begin{enumerate}
\item 1...
\item 2...
\item 3...
\item 4...
\item 5...
\item 6...
\item 7...
\item 8...
\end{enumerate}
(trzy bryły platońskie i pięć wielościanów Johnsona).
% https://en.wikipedia.org/wiki/Deltahedron tu jest lista, ale nie jestem pewien, jakie jest polskie nazewnictwo
% There are subclasses of non-convex deltahedra. Cundy (1952) shows that they may be discovered by finding the number of varying vertex's types. A set of vertices is considered the same type as long as there are subgroups of the polyhedron's same group transitive on the set. Cundy shows that the great icosahedron is the only non-convex deltahedron with a single type of vertex. There are seventeen non-convex deltahedra with two types of vertex, and soon the other eleven deltahedra were later added by Olshevsky,[12] Other subclasses are the isohedral deltahedron that was later discovered by both McNeill and Shephard (2000),[13] and the spiral deltahedron constructed by the strips of equilateral triangles was discovered by Trigg (1978).[14] 
% Delta 1984/04
Pokażą to w 1947 roku Hans Freudenthal i Bartel Leendert van der Waerden \cite{Freudenthal_1947} w obskurnym duńskim żurnalu.
% Freudenthal, H.; van der Waerden, B. L. (1947), "On an assertion of Euclid", Simon Stevin, 25: 115–121, MR 0021687
% https://mathscinet.ams.org/mathscinet/relay-station?mr=0021687
Ich wynik zreferuje później Adam Gajda w $\Delta_{84}^{4}$.