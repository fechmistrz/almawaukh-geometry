Panuje nieziemski bałagan, jeśli chodzi o terminologię dla wielościanów lub, ogólniej, wielotopów. By uniknąć ryzyka zgubienia się, podamy jedynie podstawowe definicje, a nie twierdzenia.

Z każdym wielościanem związana jest grupa jego izometrii: takich przekształceń (odbić, obrotów) bryły na siebie, która zachowuje odległości.

Tylko odwzorowanie tożsamościowe przenosi flagę na siebie.

**Definicja** (symetrie wielościanów) Wielościan nazywamy izohedralnym (izotoksalnym, izogonalnym), jeśli jego grupa izometrii działa tranzytywnie na zbiorze ścian (krawędzi, wierzchołków).

**Przykład** Wielościany platońskie są izogonalne, izotoksalne i izohedralne jednocześnie.

Edmund Hess, Max Brückner, a później także Branko Grünbaum będą badać wielościany szlachetne czyli takie, które są izohedralne oraz izogonalne. Wiadomo, że wielościany platońskie, Keplera-Poinsota są szlachetne, poza nimi dwusfenoid i ogólniej, dowolne sfenoidy (wielościany koronne?). -> % https://en.wikipedia.org/wiki/Noble_polyhedron

**Przykład**. Wielościany Catalana, podwójne ostrosłupy, latawcościany są izohedralne. Bryły dwoiste do nich, czyli wielościany archimedesowe, graniastosłupy i antygraniastosłupy są izogonalne.

Wielościan dwoisty do izohedralnego jest zawsze izogonalny

Wielosciany foremne maja jesli maja identyczne foremen sciany i naroza.

Wielościany archimedesowe są wypukłe, mają foremne ściany (dwoch lub trzech rodzajow) i są izogonalne, ale nie izohedralne.

Catalana dualne do archimedesowych.

Uniform maja foremne sciany oraz vertex-transitive.

Wielościan Johnsona(-Zalgallera) to wypukły wielościan, którego ściany są wielokątami foremnymi.


Przez analogię do wielokątów wypukłych mamy:

\begin{definition}[wielościan wypukły]
    Obszar skończony przestrzeni przekrój półprzestrzeni...
    Część wycięta z płaszczyzny przez pozostałe jest wielokątem, nazywamy go ścianą.
    Wspólny bok dwóch ścian to krawędź, wspólny koniec dwóch krawędzi to wierzchołek.
\end{definition}

Najbardziej znanymi są ostrosłupy i graniastosłupy.
Oprócz nich opowiemy także o podwójnych ostrosłupach, latawcościanach, różnych mniej lub bardziej foremnych bryłach.
Natomiast nie podamy definicji wielościanu (niewypukłego), ponieważ nie ma powszechnie akceptowanej.

\begin{definition}[flaga]
    Zbiór złożony ze ściany, krawędzi oraz wierzchołka takich, że każde leży na poprzednim, nazywamy flagą.
\end{definition} % https://en.wikipedia.org/wiki/Flag_(geometry)

