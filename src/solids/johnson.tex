%

\begin{definition}
    Wielościan wypukły, który nie jest jednolity, ale wszystkie jego ściany są foremne, nazywamy wielościanem Johnsona albo Johnsona-Zalgallera.
\end{definition}

Norman Woodason Johnson \cite{johnson_1966} opublikuje w pracy doktorskiej z 1966 roku, a spisanej pod opieką samego Coxetera, listę 92 brył, które spełniają ten warunek, zaś Wiktor Abramowicz Zalgaller \cite{zalgaller_1969} udowodni po trzech latach (1969), że lista ta jest kompletna.
Często oznacza się je po prostu $J_{\ldots}$, gdzie indeks dolny mówi, którą pozycję na liście zajmuje.

\begin{itemize}
    \item \textbf{piramida} (czworokątna: $J_1$, pięciokątna: $J_2$),
    \item \textbf{kopuła} (trójkątna: $J_3$, czworokątna: $J_4$, pięciokątna: $J_5$),
    \item \textbf{rotunda} pięciokątna: $J_6$
    \item wydłużona \textbf{piramida} (trójkątna: $J_7$, czworokątna: $J_8$, pięciokątna: $J_9$),
    \item skrętnie wydłużona \textbf{piramida} (czworokątna: $J_{10}$, pięciokątna $J_{11}$),
    \item \textbf{dwupiramida} (trójkątna: $J_{12}$, pięciokątna: $J_{13
    }$),
    %
    \item wydłużona \textbf{dwupiramida} (trójkątna: $J_{14}$, czworokątna: $J_{15}$, pięciokątna: $J_{16}$),
    \item skrętnie wydłużona \textbf{dwupiramida} czworokątna: $J_{17}$,
    \item wydłużona \textbf{kopuła} (trójkątna: $J_{18}$, czworokątna: $J_{19}$, pięciokątna: $J_{20}$),
    %
    \item [$J_{21}$] {wydłużona rotunda pięciokątna}
    \item [$J_{22}$] {skrętnie wydłużona kopuła trójkątna}
    \item [$J_{23}$] {skrętnie wydłużona kopuła czworokątna}
    \item [$J_{24}$] {skrętnie wydłużona kopuła pięciokątna}
    \item [$J_{25}$] {skrętnie wydłużona rotunda pięciokątna}
    \item [$J_{26}$] {podwójny graniastosłup trójkątny}
    \item podwójna \textbf{kopuła} (trójkątna: $J_{27}$, czworokątna: $J_{28}$, czworokątna skręcona: $J_{29}$, pięciokątna: $J_{30}$),
%%%%%%%%%
    \item [$J_{31}$] {dwukopuła pięciokątna skręcona}
    \item [$J_{32}$] {kopuło-rotunda pięciokątna}
    \item [$J_{33}$] {kopuło-rotunda pięciokątna skręcona}
    \item [$J_{34}$] {dwurotunda pięciokątna}
    \item wydłużona \textbf{dwukopuła} (trójkątna: $J_{35}$, trójkątna skręcona: $J_{36}$, czworokątna skręcona: $J_{37}$, pięciokątna: $J_{38}$, pięciokątna skręcona: $J_{39}$),
    \item [$J_{40}$] {wydłużona kopuło-rotunda pięciokątna}
    \item [$J_{41}$] {wydłużona kopułorotunda pięciokątna skręcona}
    \item [$J_{42}$] {wydłużona dwurotunda pięciokątna}
    \item [$J_{43}$] {wydłużona dwurotunda pięciokątna skręcona}
    \item [$J_{44}$] {skrętnie wydłużona dwukopuła trójkątna}
    \item [$J_{45}$] {skrętnie wydłużona dwukopuła czworokątna}
    \item [$J_{46}$] {skrętnie wydłużona dwukopuła pięciokątna}
    \item [$J_{47}$] {skrętnie wydłużona kopuło-rotunda pięciokątna}
    \item [$J_{48}$] {skrętnie wydłużona dwurotunda pięciokątna}
    \item [$J_{49}$] {powiększony graniastosłup trójkątny}
    \item [$J_{50}$] {podwójnie powiększony graniastosłup trójkątny}
    \item [$J_{51}$] {potrójnie powiększony graniastosłup trójkątny}
    \item [$J_{52}$] {powiększony graniastosłup pięciokątny}
    \item [$J_{53}$] {podwójnie powiększony graniastosłup pieciokątny}
    \item [$J_{54}$] {powiększony graniastosłup sześciokątny}
    \item [$J_{55}$] {podwójnie osiowo powiększony graniastosłup sześciokątny}
    \item [$J_{56}$] {podwójnie powiększony graniastosłup sześciokątny}
    \item [$J_{57}$] {potrójnie powiększony graniastosłup sześciokątny}
    \item [$J_{58}$] {powiększony dwunastościan}
    \item [$J_{59}$] {podwójnie osiowo powiększony dwunastościan}
    \item {(podwójnie: $J_{60}$, potrójnie: $J_{61}$) powiększony \textbf{dwunastościan}},
    \item [$J_{62}$] {podwójnie obcięty dwudziestościan}
    \item [$J_{63}$] {potrójnie obcięty dwudziestościan}
    \item [$J_{64}$] {powiększony potrójnie obcięty dwudziestościan}
    \item [$J_{65}$] {powiększony czworościan ścięty}
    \item [$J_{66}$] {powiększony sześcian ścięty}
    \item [$J_{67}$] {podwójnie powiększony sześcian ścięty}
    \item [$J_{68}$] {powiększony dwunastościan ścięty}
    \item [$J_{69}$] {podwójnie osiowo powiększony dwunastościan ścięty}
    \item [$J_{70}$] {podwójnie powiększony dwunastościan ścięty}
    \item [$J_{71}$] {potrójnie powiększony dwunastościan ścięty}
    \item [$J_{72}$] {dwudziesto-dwunastościan rombowy skręcony}
    \item [$J_{73}$] {dwudziesto-dwunastościan rombowy podwójnie osiowo skręcony}
    \item [$J_{74}$] {dwudziesto-dwunastościan rombowy podwójnie skośnie skręcony}
    \item [$J_{75}$] {dwudziesto-dwunastościan rombowy potrojnie skręcony}
    \item [$J_{76}$] {dwudziesto-dwunastościan rombowy obcięty}
    \item [$J_{77}$] {przekręcony osiowo dwudziesto-dwunastościan rombowy obcięty}
    \item [$J_{78}$] {przekręcony skośnie dwudziesto-dwunastościan rombowy obcięty}
    \item [$J_{79}$] {podwójnie przekręcony dwudziesto-dwunastościan rombowy obcięty}
    \item [$J_{80}$] {dwudziesto-dwunastościan rombowy podwójnie osiowo obcięty}
    \item [$J_{81}$] {dwudziesto-dwunastościan rombowy podwójnie skośnie obcięty}
    \item [$J_{82}$] {przekręcony dwudziesto-dwunastościan rombowy podwójnie obcięty}
    \item [$J_{83}$] {dwudziesto-dwunastościan rombowy potrójnie obcięty}
    \item {\textbf{dwuklinoid} przycięty: $J_{84}$,}
    \item [$J_{85}$] {antygraniastosłup czworokątny przycięty}
    \item {\textbf{klinokorona}}: $J_{86}$,
    \item [$J_{87}$] {powiększona klinokorona}
    \item [$J_{88}$] {klinomega- korona}
    \item [$J_{89}$] {hebeklinomegakorona}
    \item [$J_{90}$] {klinocingulum podwójne}
    \item [$J_{91}$] {dwusoczewkowa rotunda podwójna}
    \item [$J_{92}$] {hebeklinorotunda trójkątna}
\end{itemize}

% https://en.wikipedia.org/wiki/Johnson_solid
% http://matematyka.wroc.pl/book/wielosciany-johnsona
% {Siamese dodecahedron} https://en.wikipedia.org/wiki/Snub_disphenoid

%