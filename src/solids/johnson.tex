%

Wśród wielościanów wypukłych oprócz brył platońskich, archimedesowych oraz półforemnych graniastosłupów i antygraniastosłupów istnieją jeszcze 92 wielościany o foremnych ścianach (lecz nieprzystających wierzchołkach).
Oznaczamy je $J_{\#}$, gdzie kratka zastępuje numer porządkowy.

\begin{itemize}
    \item piramida (czworokątna: $J_1$, pięciokątna: $J_2$),
    \item kopuła (trójkątna: $J_3$, czworokątna: $J_4$, pięciokątna: $J_5$),
    \item \emph{(equilateral square: $J_1$, pentagonal: $J_2$) pyramid},
    \item \emph{(triangular: $J_3$, square: $J_4$, pentagonal: $J_5$) cupola},
    \item [$J_{6}$] {rotunda pięciokątna}
                     (\emph{pentagonal rotunda})
    \item [$J_{7}$] {wydłużona piramida trójkątna}
                     (\emph{elongated triangular pyramid})
    \item [$J_{8}$] {wydłużona piramida czworokątna}
                     (\emph{elongated square pyramid})
    \item [$J_{9}$] {wydłużona piramida pięciokątna}
                     (\emph{elongated pentagonal pyramid})
    \item [$J_{10}$] {skrętnie wydłużona piramida czworokątna}
                     (\emph{gyroelongated square pyramid})
    \item [$J_{11}$] {skrętnie wydłużona piramida pięciokątna}
                     (\emph{gyroelongated pentagonal pyramid})
    \item [$J_{12}$] {dwupiramida trójkątna}
                     (\emph{triangular bipyramid})
    \item [$J_{13}$] {dwupiramida pięciokątna}
                     (\emph{pentagonal bipyramid})
    \item [$J_{14}$] {wydłużona dwupiramida trójkątna}
                     (\emph{elongated triangular bipyramid})
    \item [$J_{15}$] {wydłużona dwupiramida czworokątna}
                     (\emph{elongated square bipyramid})
    \item [$J_{16}$] {wydłużona dwupiramida pięciokątna}
                     (\emph{elongated pentagonal bipyramid})
    \item [$J_{17}$] {skrętnie wydłużona dwupiramida czworokątna}
                     (\emph{gyroelongated square bipyramid})
    \item [$J_{18}$] {wydłużona kopuła trójkątna}
                     (\emph{elongated triangular cupola})
    \item [$J_{19}$] {wydłużona kopuła czworokątna}
                     (\emph{elongated square cupola})
    \item [$J_{20}$] {wydłużona kopuła pięciokątna}
                     (\emph{elongated pentagonal cupola})
    \item [$J_{21}$] {wydłużona rotunda pięciokątna}
                     (\emph{elongated pentagonal rotunda})
    \item [$J_{22}$] {skrętnie wydłużona kopuła trójkątna}
                     (\emph{gyroelongated triangular cupola})
    \item [$J_{23}$] {skrętnie wydłużona kopuła czworokątna}
                     (\emph{gyroelongated square cupola})
    \item [$J_{24}$] {skrętnie wydłużona kopuła pięciokątna}
                     (\emph{gyroelongated pentagonal cupola})
    \item [$J_{25}$] {skrętnie wydłużona rotunda pięciokątna}
                     (\emph{gyroelongated pentagonal rotunda})
    \item [$J_{26}$] {podwójny graniastosłup trójkątny}
                     (\emph{gyrobifastigium})
    \item [$J_{27}$] {podwójna kopuła trójkątna}
                     (\emph{triangular orthobicupola})
    \item [$J_{28}$] {podwójna kopuła czworokątna}
                     (\emph{square orthobicupola})
    \item [$J_{29}$] {podwójna kopuła czworokątna skręcona}
                     (\emph{square gyrobicupola})
    \item [$J_{30}$] {podwójna kopuła pięciokątna}
                     (\emph{pentagonal orthobicupola})
    \item [$J_{31}$] {dwukopuła pięciokątna skręcona}
                     (\emph{pentagonal gyrobicupola})
    \item [$J_{32}$] {kopuło-rotunda pięciokątna}
                     (\emph{pentagonal orthocupolarotunda})
    \item [$J_{33}$] {kopuło-rotunda pięciokątna skręcona}
                     (\emph{pentagonal gyrocupolarotunda})
    \item [$J_{34}$] {dwurotunda pięciokątna}
                     (\emph{pentagonal orthobirotunda})
    \item wydłużona dwukopuła (trójkątna: $J_{35}$, trójkątna skręcona: $J_{36}$, czworokątna skręcona: $J_{37}$, pięciokątna: $J_{38}$, pięciokątna skręcona: $J_{36}$),
    \item [$J_{35}$] {wydłużona dwukopuła trójkątna}
                     (\emph{elongated triangular orthobicupola})
    \item [$J_{36}$] {wydłużona dwukopuła trójkątna skręcona}
                     (\emph{elongated triangular gyrobicupola})
    \item [$J_{37}$] {wydłużona dwukopuła czworokątna skręcona}
                     (\emph{elongated square gyrobicupola})
    \item [$J_{38}$] {wydłużona dwukopuła pięciokątna}
                     (\emph{elongated pentagonal orthobicupola})
    \item [$J_{39}$] {wydłużona dwukopuła pięciokątna skręcona}
                     (\emph{elongated pentagonal gyrobicupola})
    \item [$J_{40}$] {wydłużona kopuło-rotunda pięciokątna}
                     (\emph{elongated pentagonal orthocupolarotunda})
    \item [$J_{41}$] {wydłużona kopułorotunda pięciokątna skręcona}
                     (\emph{elongated pentagonal gyrocupolarotunda})
    \item [$J_{42}$] {wydłużona dwurotunda pięciokątna}
                     (\emph{elongated pentagonal orthobirotunda})
    \item [$J_{43}$] {wydłużona dwurotunda pięciokątna skręcona}
                     (\emph{elongated pentagonal gyrobirotunda})
    \item [$J_{44}$] {skrętnie wydłużona dwukopuła trójkątna}
                     (\emph{gyroelongated triangular bicupola})
    \item [$J_{45}$] {skrętnie wydłużona dwukopuła czworokątna}
                     (\emph{gyroelongated square bicupola})
    \item [$J_{46}$] {skrętnie wydłużona dwukopuła pięciokątna}
                     (\emph{gyroelongated pentagonal bicupola})
    \item [$J_{47}$] {skrętnie wydłużona kopuło-rotunda pięciokątna}
                     (\emph{gyroelongated pentagonal cupolarotunda})
    \item [$J_{48}$] {skrętnie wydłużona dwurotunda pięciokątna}
                     (\emph{gyroelongated pentagonal birotunda})
    \item [$J_{49}$] {powiększony graniastosłup trójkątny}
                     (\emph{augmented triangular prism})
    \item [$J_{50}$] {podwójnie powiększony graniastosłup trójkątny}
                     (\emph{biaugmented triangular prism})
    \item [$J_{51}$] {potrójnie powiększony graniastosłup trójkątny}
                     (\emph{triaugmented triangular prism})
    \item [$J_{52}$] {powiększony graniastosłup pięciokątny}
                     (\emph{augmented pentagonal prism})
    \item [$J_{53}$] {podwójnie powiększony graniastosłup pieciokątny}
                     (\emph{biaugmented pentagonal prism})
    \item [$J_{54}$] {powiększony graniastosłup sześciokątny}
                     (\emph{augmented hexagonal prism})
    \item [$J_{55}$] {podwójnie osiowo powiększony graniastosłup sześciokątny}
                     (\emph{parabiaugmented hexagonal prism})
    \item [$J_{56}$] {podwójnie powiększony graniastosłup sześciokątny}
                     (\emph{metabiaugmented hexagonal prism})
    \item [$J_{57}$] {potrójnie powiększony graniastosłup sześciokątny}
                     (\emph{triaugmented hexagonal prism})
    \item [$J_{58}$] {powiększony dwunastościan}
                     (\emph{augmented dodecahedron})
    \item [$J_{59}$] {podwójnie osiowo powiększony dwunastościan}
                     (\emph{parabiaugmented dodecahedron})
    \item [$J_{60}$] {podwójnie powiększony dwunastościan}
                     (\emph{metabiaugmented dodecahedron})
    \item [$J_{61}$] {potrójnie powiększony dwunastościan}
                     (\emph{triaugmented dodecahedron})
    \item [$J_{62}$] {podwójnie obcięty dwudziestościan}
                     (\emph{metabidiminished icosahedron})
    \item [$J_{63}$] {potrójnie obcięty dwudziestościan}
                     (\emph{tridiminished icosahedron})
    \item [$J_{64}$] {powiększony potrójnie obcięty dwudziestościan}
                     (\emph{augmented tridiminished icosahedron})
    \item [$J_{65}$] {powiększony czworościan ścięty}
                     (\emph{augmented truncated tetrahedron})
    \item [$J_{66}$] {powiększony sześcian ścięty}
                     (\emph{augmented truncated cube})
    \item [$J_{67}$] {podwójnie powiększony sześcian ścięty}
                     (\emph{biaugmented truncated cube})
    \item [$J_{68}$] {powiększony dwunastościan ścięty}
                     (\emph{augmented truncated dodecahedron})
    \item [$J_{69}$] {podwójnie osiowo powiększony dwunastościan ścięty}
                     (\emph{parabiaugmented truncated dodecahedron})
    \item [$J_{70}$] {podwójnie powiększony dwunastościan ścięty}
                     (\emph{metabiaugmented truncated dodecahedron})
    \item [$J_{71}$] {potrójnie powiększony dwunastościan ścięty}
                     (\emph{triaugmented truncated dodecahedron})
    \item [$J_{72}$] {dwudziesto-dwunastościan rombowy skręcony}
                     (\emph{gyrate rhombicosidodecahedron})
    \item [$J_{73}$] {dwudziesto-dwunastościan rombowy podwójnie osiowo skręcony}
                     (\emph{parabigyrate rhombicosidodecahedron})
    \item [$J_{74}$] {dwudziesto-dwunastościan rombowy podwójnie skośnie skręcony}
                     (\emph{metabigyrate rhombicosidodecahedron})
    \item [$J_{75}$] {dwudziesto-dwunastościan rombowy potrojnie skręcony}
                     (\emph{trigyrate rhombicosidodecahedron})
    \item [$J_{76}$] {dwudziesto-dwunastościan rombowy obcięty}
                     (\emph{diminished rhombicosidodecahedron})
    \item [$J_{77}$] {przekręcony osiowo dwudziesto-dwunastościan rombowy obcięty}
                     (\emph{paragyrate diminished rhombicosidodecahedron})
    \item [$J_{78}$] {przekręcony skośnie dwudziesto-dwunastościan rombowy obcięty}
                     (\emph{metagyrate diminished rhombicosidodecahedron})
    \item [$J_{79}$] {podwójnie przekręcony dwudziesto-dwunastościan rombowy obcięty}
                     (\emph{bigyrate diminished rhombicosidodecahedron})
    \item [$J_{80}$] {dwudziesto-dwunastościan rombowy podwójnie osiowo obcięty}
                     (\emph{parabidiminished rhombicosidodecahedron})
    \item [$J_{81}$] {dwudziesto-dwunastościan rombowy podwójnie skośnie obcięty}
                     (\emph{metabidiminished rhombicosidodecahedron})
    \item [$J_{82}$] {przekręcony dwudziesto-dwunastościan rombowy podwójnie obcięty}
                     (\emph{gyrate bidiminished rhombicosidodecahedron})
    \item [$J_{83}$] {dwudziesto-dwunastościan rombowy potrójnie obcięty}
                     (\emph{tridiminished rhombicosidodecahedron})
    \item [$J_{84}$] {dwuklinoid przycięty}
                     (\emph{snub disphenoid})
    \item [$J_{85}$] {antygraniastosłup czworokątny przycięty}
                     (\emph{snub square antiprism})
    \item [$J_{86}$] {klinokorona}
                     (\emph{sphenocorona})
    \item [$J_{87}$] {powiększona klinokorona}
                     (\emph{augmented sphenocorona})
    \item [$J_{88}$] {klinomega- korona}
                     (\emph{sphenomegacorona})
    \item [$J_{89}$] {hebeklinomegakorona}
                     (\emph{hebesphenomegacorona})
    \item [$J_{90}$] {klinocingulum podwójne}
                     (\emph{disphenocingulum})
    \item [$J_{91}$] {dwusoczewkowa rotunda podwójna}
                     (\emph{bilunabirotunda})
    \item [$J_{92}$] {hebeklinorotunda trójkątna}
                     (\emph{triangular hebesphenorotunda})
\end{itemize}

Zostaną one opisane w 1966 roku przez Normana Johnsona w jego pracy doktorskiej napisanej pod kierunkiem Coxetera.


Wielościan Johnsona(-Zalgallera) to wypukły wielościan, którego ściany są wielokątami foremnymi. (Zazwyczaj wyklucza się wielościany jednorodne: platońskie, archimedesowe, graniastosłupy, antygraniastosłupy).


% https://en.wikipedia.org/wiki/Johnson_solid
% http://matematyka.wroc.pl/book/wielosciany-johnsona
% {Siamese dodecahedron} https://en.wikipedia.org/wiki/Snub_disphenoid

%