\todofoot{1806 - Louis Poinsot discovers the two remaining Kepler-Poinsot polyhedra.}
\todofoot{1619 - Johannes Kepler discovers two of the Kepler-Poinsot polyhedra}

% https://en.wikipedia.org/wiki/Great_icosahedron

Wielościany Keplera-Poinsota
Wielościany Keplera-Poinsota są odpowiednikami brył platońskich w świecie wielościanów niewypukłych. Również ich ściany są przystającymi wielokątami foremnymi i w każdym wierzchołku spotyka się taka sama liczba ścian. Tym razem jednak ściany mogą być wielokątami gwiaździstymi. Dopuszczona jest także możliwość przenikania ścian (tzn. przecinania się poza krawędziami).
Istnieją tylko 4 wielościany foremne niewypukłe:
dwunastościan gwiaździsty mały (small stellated dodecahedron),
dwunastościan wielki (great dodecahedron),
dwunastościan gwiaździsty wielki (great stellated dodecahedron),
dwudziestościan wielki (great icosahedron). 