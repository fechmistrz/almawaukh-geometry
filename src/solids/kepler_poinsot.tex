\todofoot{1806 - Louis Poinsot discovers the two remaining Kepler-Poinsot polyhedra.}
\todofoot{1619 - Johannes Kepler discovers two of the Kepler-Poinsot polyhedra}

% https://en.wikipedia.org/wiki/Great_icosahedron

Wielościany Keplera-Poinsota
Wielościany Keplera-Poinsota są odpowiednikami brył platońskich w świecie wielościanów niewypukłych. Również ich ściany są przystającymi wielokątami foremnymi i w każdym wierzchołku spotyka się taka sama liczba ścian. Tym razem jednak ściany mogą być wielokątami gwiaździstymi. Dopuszczona jest także możliwość przenikania ścian (tzn. przecinania się poza krawędziami).
Istnieją tylko 4 wielościany foremne niewypukłe:
dwunastościan gwiaździsty mały (small stellated dodecahedron),
dwunastościan wielki (great dodecahedron),
dwunastościan gwiaździsty wielki (great stellated dodecahedron),
dwudziestościan wielki (great icosahedron). 

https://www.deltami.edu.pl/1974/01/wielosciany-gwiazdziste/

Wielokąty gwiaździste takie jak pentagram był znane starożytnym, ale z jakiegoś powodu nie użyją ich do budowy wielościanów.

Most, if not all, of the Kepler–Poinsot polyhedra were known of in some form or other before Kepler. A small stellated dodecahedron appears in a marble tarsia (inlay panel) on the floor of St. Mark's Basilica, Venice, Italy. It dates from the 15th century and is sometimes attributed to Paolo Uccello.[7]

In his Perspectiva corporum regularium (Perspectives of the regular solids), a book of woodcuts published in 1568, Wenzel Jamnitzer depicts the great stellated dodecahedron and a great dodecahedron (both shown below). There is also a truncated version of the small stellated dodecahedron.[8] It is clear from the general arrangement of the book that he regarded only the five Platonic solids as regular.

Zmieni to J. Kepler, zauważy bowiem, że w definicji wielościanów platońskich można opuścić założenie wypukłości.

Dwieście lat później Louis Poinsot dopuści "star vertex figures" (circuits around each corner), co doprowadzi go do odkrcia dwóch nowych (i dwóch starych nan owo).

Nazywamy je wielościanami Keplera-Poinsota. Nazwy pochodzą od Cayleya, dopiero z połowi 19tego wieku!!

Three years later, Augustin Cauchy proved the list complete by stellating the Platonic solids, and almost half a century after that, in 1858, Bertrand provided a more elegant proof by faceting them. => z https://en.wikipedia.org/wiki/Kepler%E2%80%93Poinsot_polyhedron

* [Kepler]%(https://en.wikipedia.org/wiki/Johannes_Kepler "Johannes Kepler") (1619) discovered two of the regular[Kepler–Poinsot polyhedra](https://en.wikipedia.org/wiki/Kepler%E2%80%93Poinsot_polyhedra "Kepler–Poinsot polyhedra"), the[small stellated dodecahedron](https://en.wikipedia.org/wiki/Small_stellated_dodecahedron "Small stellated dodecahedron") and[great stellated dodecahedron](https://en.wikipedia.org/wiki/Great_stellated_dodecahedron "Great stellated dodecahedron").

* [Louis Poinsot]%(https://en.wikipedia.org/wiki/Louis_Poinsot "Louis Poinsot") (1809) discovered the other two, the[great dodecahedron](https://en.wikipedia.org/wiki/Great_dodecahedron "Great dodecahedron") and[great icosahedron](https://en.wikipedia.org/wiki/Great_icosahedron "Great icosahedron").

* The set of four was proven complete by[Augustin-Louis Cauchy]%(https://en.wikipedia.org/wiki/Augustin-Louis_Cauchy "Augustin-Louis Cauchy") in 1813 and named by[Arthur Cayley](https://en.wikipedia.org/wiki/Arthur_Cayley "Arthur Cayley") in 1859.
