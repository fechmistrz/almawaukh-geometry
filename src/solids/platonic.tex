%

Wszystkie ściany wielościanów platońskich są przystającymi wielokątami foremnymi, w każdym wierzchołku spotyka się taka sama liczba ścian.
Już od starożytni odkryją, że takich brył jest pięć:
\begin{itemize}
\item czworościan foremny,
\item sześciościan foremny, czyli sześcian,
\item ośmiościan foremny,
\item dwunastościan foremny,
\item dwudziestościan foremny.
\end{itemize}

Coxeter 166-167.

BRYŁA DWOISTA

OCTAEDRON ELEVATUM

Niektórzy będą mylić wielościany foremne z platońskimi (!).

Pierwszym, który rozpoznał cechy wspólne całej piątki i zaliczył te wielościany do jednej rodziny był starożytny matematyk grecki Teajtetos (IV w. p.n.e.). 
Przyjaciel Teajtetosa -- Platon włączył je do swojego systemu filozoficznego, stąd nazwa.

% https://en.wikipedia.org/wiki/Platonic_solid
% \section{Pięć wielościanów} Hartshorne: rozdział 8
% \section{Cauchy's rigidity theorem} Hartshorne: section 45

%