%

Wszystkie ściany wielościanów platońskich są przystającymi wielokątami foremnymi, w każdym wierzchołku spotyka się taka sama liczba ścian.
Pozostaje zagadką, kto pierwszy odkryje wszystkie foremne wielościany wypukłe, ale Teajtet, uczeń Platona udowodni, że jest ich dokładnie pięć:
\begin{itemize}
\item czworościan foremny,
\item sześciościan foremny, czyli sześcian,
\item ośmiościan foremny,
\item dwunastościan foremny,
\item dwudziestościan foremny.
\end{itemize}

(Dwunastościan zostanie odkryty jako ostatni).
Niektórzy będą mylić wielościany foremne z platońskimi (!).

Czworościan ma trzy osie symetii: to proste przechodzące przez środki przeciwległych krawędzi. % https://www.deltami.edu.pl/1993/04/rozne-roznosci/

Coxeter 166-167.

\begin{proposition}
    Niech $ABCD$ będzie czworościanem.
    Wtedy następujące warunki są równoważne:
    \begin{enumerate}
        \item Wszystkie ściany są przystające.
        \item Wszystkie ściany są trójkątami ostrokątnymi o takim samym promieniu okręgu opisanego.
        \item Suma kątów płaskich przy każdym wierzchołku wynosi $\pi$.
        \item Sumy kątów płaskich przy trzech dowolnych wierzchołkach wynosi $\pi$.
        \item Siatka czworościanu jest trójkątem ostrokątnym podzielonym na cztery przystające trójkąty.
        \item Kąty $\angle BAC$, $\angle ABD$, $\angle ACD$, $\angle BDC$ są równe.
        \item Przeciwległe krawędzie są równe.
        \item Trzy odcinki łączące środki przeciwległych krawędzi są parami prostopadłe.
        \item Wszystkie ściany mają równe pola.
        \item Rzut czworościanu $ABCD$ na dowolną płaszczyznę równoległą do dwóch przeciwległych krawędzi jest prostokątem.
        \item Każdy odcinek łączący środki przeciwległych krawędzi jest prostopadły do tych krawędzi.
    \end{enumerate} 
\end{proposition}

Dowód wszystkich 11 implikacji znajdziemy w artykulu Pompego, $\Delta_{94}^3$. % https://www.deltami.edu.pl/1994/03/o-czworoscianie-rownosciennym/

OCTAEDRON ELEVATUM

https://www.deltami.edu.pl/2020/08/wielosciany-w-wieloscianach-czyli-matematyka-eksperymentalna/

% https://en.wikipedia.org/wiki/Platonic_solid
% \section{Pięć wielościanów} Hartshorne: rozdział 8
% \section{Cauchy's rigidity theorem} Hartshorne: section 45

%