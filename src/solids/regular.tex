**Definicja**. Wielościan nazywamy foremnym, jeśli jego grupa izometrii działa tranzytywnie na zbiorze flag.

Klasycznie podawane były inne definicje, z czego najpowszechniejsza: idetnyczne ściany foremne, identyczne naroża.

**Fakt**. Istnieje dziewięć albo czternaście wielościanów foremnych: pięć brył platońskich, cztery gwieździste (Keplera-Poinsota) i pięć złożeń (czasami nieuznawanych za wielościany foremne).


\subsection{Wielościany platońskie}
%

Wszystkie ściany wielościanów platońskich są przystającymi wielokątami foremnymi, w każdym wierzchołku spotyka się taka sama liczba ścian.
Już od starożytni odkryją, że takich brył jest pięć:
\begin{itemize}
\item czworościan foremny,
\item sześciościan foremny, czyli sześcian,
\item ośmiościan foremny,
\item dwunastościan foremny,
\item dwudziestościan foremny.
\end{itemize}

Niektórzy będą mylić wielościany foremne z platońskimi (!).

Pierwszym, który rozpoznał cechy wspólne całej piątki i zaliczył te wielościany do jednej rodziny był starożytny matematyk grecki Teajtetos (IV w. p.n.e.). 
Przyjaciel Teajtetosa -- Platon włączył je do swojego systemu filozoficznego, stąd nazwa.

% https://en.wikipedia.org/wiki/Platonic_solid
% \section{Pięć wielościanów} Hartshorne: rozdział 8
% \section{Cauchy's rigidity theorem} Hartshorne: section 45

%

\subsection{Wielościany Keplera-Poinsota}
\todofoot{1806 - Louis Poinsot discovers the two remaining Kepler-Poinsot polyhedra.}
\todofoot{1619 - Johannes Kepler discovers two of the Kepler-Poinsot polyhedra}

% https://en.wikipedia.org/wiki/Great_icosahedron

Wielościany Keplera-Poinsota
Wielościany Keplera-Poinsota są odpowiednikami brył platońskich w świecie wielościanów niewypukłych. Również ich ściany są przystającymi wielokątami foremnymi i w każdym wierzchołku spotyka się taka sama liczba ścian. Tym razem jednak ściany mogą być wielokątami gwiaździstymi. Dopuszczona jest także możliwość przenikania ścian (tzn. przecinania się poza krawędziami).
Istnieją tylko 4 wielościany foremne niewypukłe:
dwunastościan gwiaździsty mały (small stellated dodecahedron),
dwunastościan wielki (great dodecahedron),
dwunastościan gwiaździsty wielki (great stellated dodecahedron),
dwudziestościan wielki (great icosahedron). 

\subsection{Złożenia?}
In geometry, a polyhedral compound is a figure that is composed of several polyhedra sharing a common centre. They are the three-dimensional analogs of polygonal compounds such as the hexagram.

A regular polyhedral compound can be defined as a compound which, like a regular polyhedron, is vertex-transitive, edge-transitive, and face-transitive. Unlike thec ase of polyhedra, this is not equivalent to the symmetry group acting transitively on its flags; the compound of two tetrahedra is the only regular compound with that property.

(czemu nie jest flag-transitive???)

najbardziej znany: stella octangula

% https://en.wikipedia.org/wiki/Compound_of_four_cubes -> to nie jest regular

