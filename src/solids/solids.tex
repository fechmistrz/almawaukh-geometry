

\section{Preludium}
\input{solids/intro}

W każdym czworościanie istnieje wierzchołek, przy którym trzy kąty płąskie są ostre. % https://www.deltami.edu.pl/2013/12/katy-trojscienne/

% TODO: https://en.wikipedia.org/wiki/Wallace-Bolyai-Gerwien_theorem to nie działa w 3d a jest ok w 2d
% TODO: https://www.deltami.edu.pl/2025/03/wycinanki-i-ukladanki/
\section{Objętość. Trzeci problem Hilberta}
\input{solids/dehn}

\section{Rodzaje brył}
\subsection{Graniastosłupy i ostrosłupy}
\input{solids/pyramid}

\subsubsection{Pryzmatoid}
Pryzmatoid to bryła, której wierzchołki leżą na dwóch płaszczyznach równoległych.
Jej objętość wyraża wzór $(l + 4m + h)/ 6$, gdzie $l$ to pole dolnego przekroju, $m$ środkowego, $h$ górnego. % eves s. 40
% TODO https://en.wikipedia.org/wiki/Prismatoid#Prismatoid_families


\subsection{Wielościany foremne}
**Definicja**. Wielościan nazywamy foremnym, jeśli jego grupa izometrii działa tranzytywnie na zbiorze flag.

Klasycznie podawane były inne definicje, z czego najpowszechniejsza: idetnyczne ściany foremne, identyczne naroża.

**Fakt**. Istnieje dziewięć albo czternaście wielościanów foremnych: pięć brył platońskich, cztery gwieździste (Keplera-Poinsota) i pięć złożeń (czasami nieuznawanych za wielościany foremne).


\subsection{Wielościany platońskie}
%

Wszystkie ściany wielościanów platońskich są przystającymi wielokątami foremnymi, w każdym wierzchołku spotyka się taka sama liczba ścian.
Pozostaje zagadką, kto pierwszy odkryje wszystkie foremne wielościany wypukłe, ale Teajtet, uczeń Platona udowodni, że jest ich dokładnie pięć:
\begin{itemize}
\item czworościan foremny,
\item sześciościan foremny, czyli sześcian,
\item ośmiościan foremny,
\item dwunastościan foremny,
\item dwudziestościan foremny.
\end{itemize}

(Dwunastościan zostanie odkryty jako ostatni).
Niektórzy będą mylić wielościany foremne z platońskimi (!).

Coxeter 166-167.

OCTAEDRON ELEVATUM

% https://en.wikipedia.org/wiki/Platonic_solid
% \section{Pięć wielościanów} Hartshorne: rozdział 8
% \section{Cauchy's rigidity theorem} Hartshorne: section 45

%

\subsection{Wielościany Keplera-Poinsota}
\todofoot{1806 - Louis Poinsot discovers the two remaining Kepler-Poinsot polyhedra.}
\todofoot{1619 - Johannes Kepler discovers two of the Kepler-Poinsot polyhedra}

% https://en.wikipedia.org/wiki/Great_icosahedron
small stellated dodecahedron 
great stellated dodecahedron 
great dodecahedron 
great icosahedron 


\subsection{Złożenia?}
In geometry, a polyhedral compound is a figure that is composed of several polyhedra sharing a common centre. They are the three-dimensional analogs of polygonal compounds such as the hexagram.

A regular polyhedral compound can be defined as a compound which, like a regular polyhedron, is vertex-transitive, edge-transitive, and face-transitive. Unlike thec ase of polyhedra, this is not equivalent to the symmetry group acting transitively on its flags; the compound of two tetrahedra is the only regular compound with that property.

(czemu nie jest flag-transitive???)

najbardziej znany: stella octangula

% https://en.wikipedia.org/wiki/Compound_of_four_cubes -> to nie jest regular



\subsection{Wielościany archimedesowe i Catalana}


Wszystkie ściany wielościanów archimedesowych są wielokątami foremnymi, ale mogą występować w dwóch lub trzech rodzajach.
Układ ścian we wszystkich wierzchołkach jest taki sam.
Istnieje 13 unikatowych wielościanów archimedesowych:
\begin{enumerate}
\item czworościan ścięty (truncated tetrahedron)
\item sześcian ścięty (truncated cube)
\item sześcio-ośmiościan (cuboctahedron)
\item ośmiościan ścięty (truncated octahedron)
\item dwunastościan ścięty (truncated dodecahedron)
\item dwudziesto-dwunastościan (icosidodecahedron)
\item dwudziestościan ścięty (truncated icosahedron)
\item sześcio-ośmiościan rombowy mały (rhombicuboctahedron)
\item sześcio-ośmiościan rombowy wielki (rhombitruncated cuboctahedron)
\item sześcio-ośmiościan przycięty (snub cuboctahedron)
\item dwudziesto-dwunastościan rombowy mały (rhombicosidodecahedron)
\item dwudziesto-dwunastościan rombowy wielki (rhombitruncated icosidodecahedron)
\item dwudziesto-dwunastościan przycięty (snub icosidodecahedron).
\end{enumerate}
Niektórzy zaliczają do tej kategorii jeszcze dwie nieskończone serie wypukłych wielościanów półforemnych:
\begin{itemize}
\item płóforemne graniastosłupy (o ścianach bocznych będących kwadratami)
\item półforemne antygraniastosłupy (o ścianach bocznych będących trójkątami równobocznymi).
\end{itemize} 

Niektórzy (wiele osób) mieszają wielościany archimedesowe z półforemnymi.

\begin{definition}[wielościan półforemny]
    Trzynaście wielościanów archimedesowych i dwie nieskończone serie: graniastosłupów oraz antygraniastosłupów, a czasami także $J_{37}$ (np. zdaniem Branko Grünbauma) nazywamy wielościanami półforemnymi.
\end{definition}

Pierwsza definicja mówiła, że wielościan ma mieć foremne ściany, a jego grupa symetrii ma działać tranzytywnie na wierzchołkach; później takie obiekty nazwie się jednorodnymi.
Wielościan $J_{37}$ ma identyczne naroża, ale przez skręcenie nie ma tranzytywności (i zapewne przez to pojawił się w druku dopiero w 1905 roku, a potem został przez przypadek odkryty na nowo: przez Jeffreya Charlesa Percy'ego Millera w 1930 i niezależnie przez Ashkinuze)
% TODO: Ashkinuse, V.G.: On the number of semiregular polyhedra. -> rosyjski mathscinet

% http://matematyka.wroc.pl/book/historia-1
% http://matematyka.wroc.pl/book/wielościany-archimedesowe-platońskie
\begin{itemize}
\item Catalan solids: Rhombic dodecahedron 
\item Tetrakis hexahedron 
\item Triakis tetrahedron 
\item Catalan solid 
\item Triakis icosahedron 
\item Deltoidal icositetrahedron 
\item Disdyakis triacontahedron 
\item Deltoidal hexecontahedron 
\item Disdyakis dodecahedron 
\item Rhombic triacontahedron 
\item Pentagonal hexecontahedron 
\item Pentagonal icositetrahedron 
\item Pentakis dodecahedron 
\item Triakis octahedron 
\end{itemize}

\subsection{Wielościany Johnsona-Zalgallera}
%

\begin{definition}
    Wielościan wypukły, który nie jest jednolity, ale wszystkie jego ściany są foremne, nazywamy wielościanem Johnsona albo Johnsona-Zalgallera.
\end{definition}

Norman Woodason Johnson \cite{johnson_1966} opublikuje w pracy doktorskiej z 1966 roku, a spisanej pod opieką samego Coxetera, listę 92 brył, które spełniają ten warunek, zaś Wiktor Abramowicz Zalgaller \cite{zalgaller_1969} udowodni po trzech latach (1969), że lista ta jest kompletna.
Często oznacza się je po prostu $J_{\ldots}$, gdzie indeks dolny mówi, którą pozycję na liście zajmuje.

\begin{itemize}
    \item piramida (czworokątna: $J_1$, pięciokątna: $J_2$),
    \item kopuła (trójkątna: $J_3$, czworokątna: $J_4$, pięciokątna: $J_5$),
    \item \emph{(equilateral square: $J_1$, pentagonal: $J_2$) pyramid},
    \item \emph{(triangular: $J_3$, square: $J_4$, pentagonal: $J_5$) cupola},
    \item [$J_{6}$] {rotunda pięciokątna}
                     (\emph{pentagonal rotunda})
    \item [$J_{7}$] {wydłużona piramida trójkątna}
                     (\emph{elongated triangular pyramid})
    \item [$J_{8}$] {wydłużona piramida czworokątna}
                     (\emph{elongated square pyramid})
    \item [$J_{9}$] {wydłużona piramida pięciokątna}
                     (\emph{elongated pentagonal pyramid})
    \item [$J_{10}$] {skrętnie wydłużona piramida czworokątna}
                     (\emph{gyroelongated square pyramid})
    \item [$J_{11}$] {skrętnie wydłużona piramida pięciokątna}
                     (\emph{gyroelongated pentagonal pyramid})
    \item [$J_{12}$] {dwupiramida trójkątna}
                     (\emph{triangular bipyramid})
    \item [$J_{13}$] {dwupiramida pięciokątna}
                     (\emph{pentagonal bipyramid})
    \item [$J_{14}$] {wydłużona dwupiramida trójkątna}
                     (\emph{elongated triangular bipyramid})
    \item [$J_{15}$] {wydłużona dwupiramida czworokątna}
                     (\emph{elongated square bipyramid})
    \item [$J_{16}$] {wydłużona dwupiramida pięciokątna}
                     (\emph{elongated pentagonal bipyramid})
    \item [$J_{17}$] {skrętnie wydłużona dwupiramida czworokątna}
                     (\emph{gyroelongated square bipyramid})
    \item [$J_{18}$] {wydłużona kopuła trójkątna}
                     (\emph{elongated triangular cupola})
    \item [$J_{19}$] {wydłużona kopuła czworokątna}
                     (\emph{elongated square cupola})
    \item [$J_{20}$] {wydłużona kopuła pięciokątna}
                     (\emph{elongated pentagonal cupola})
    \item [$J_{21}$] {wydłużona rotunda pięciokątna}
                     (\emph{elongated pentagonal rotunda})
    \item [$J_{22}$] {skrętnie wydłużona kopuła trójkątna}
                     (\emph{gyroelongated triangular cupola})
    \item [$J_{23}$] {skrętnie wydłużona kopuła czworokątna}
                     (\emph{gyroelongated square cupola})
    \item [$J_{24}$] {skrętnie wydłużona kopuła pięciokątna}
                     (\emph{gyroelongated pentagonal cupola})
    \item [$J_{25}$] {skrętnie wydłużona rotunda pięciokątna}
                     (\emph{gyroelongated pentagonal rotunda})
    \item [$J_{26}$] {podwójny graniastosłup trójkątny}
                     (\emph{gyrobifastigium})
    \item [$J_{27}$] {podwójna kopuła trójkątna}
                     (\emph{triangular orthobicupola})
    \item [$J_{28}$] {podwójna kopuła czworokątna}
                     (\emph{square orthobicupola})
    \item [$J_{29}$] {podwójna kopuła czworokątna skręcona}
                     (\emph{square gyrobicupola})
    \item [$J_{30}$] {podwójna kopuła pięciokątna}
                     (\emph{pentagonal orthobicupola})
    \item [$J_{31}$] {dwukopuła pięciokątna skręcona}
                     (\emph{pentagonal gyrobicupola})
    \item [$J_{32}$] {kopuło-rotunda pięciokątna}
                     (\emph{pentagonal orthocupolarotunda})
    \item [$J_{33}$] {kopuło-rotunda pięciokątna skręcona}
                     (\emph{pentagonal gyrocupolarotunda})
    \item [$J_{34}$] {dwurotunda pięciokątna}
                     (\emph{pentagonal orthobirotunda})
    \item wydłużona dwukopuła (trójkątna: $J_{35}$, trójkątna skręcona: $J_{36}$, czworokątna skręcona: $J_{37}$, pięciokątna: $J_{38}$, pięciokątna skręcona: $J_{36}$),
    \item [$J_{35}$] {wydłużona dwukopuła trójkątna}
                     (\emph{elongated triangular orthobicupola})
    \item [$J_{36}$] {wydłużona dwukopuła trójkątna skręcona}
                     (\emph{elongated triangular gyrobicupola})
    \item [$J_{37}$] {wydłużona dwukopuła czworokątna skręcona}
                     (\emph{elongated square gyrobicupola})
    \item [$J_{38}$] {wydłużona dwukopuła pięciokątna}
                     (\emph{elongated pentagonal orthobicupola})
    \item [$J_{39}$] {wydłużona dwukopuła pięciokątna skręcona}
                     (\emph{elongated pentagonal gyrobicupola})
    \item [$J_{40}$] {wydłużona kopuło-rotunda pięciokątna}
                     (\emph{elongated pentagonal orthocupolarotunda})
    \item [$J_{41}$] {wydłużona kopułorotunda pięciokątna skręcona}
                     (\emph{elongated pentagonal gyrocupolarotunda})
    \item [$J_{42}$] {wydłużona dwurotunda pięciokątna}
                     (\emph{elongated pentagonal orthobirotunda})
    \item [$J_{43}$] {wydłużona dwurotunda pięciokątna skręcona}
                     (\emph{elongated pentagonal gyrobirotunda})
    \item [$J_{44}$] {skrętnie wydłużona dwukopuła trójkątna}
                     (\emph{gyroelongated triangular bicupola})
    \item [$J_{45}$] {skrętnie wydłużona dwukopuła czworokątna}
                     (\emph{gyroelongated square bicupola})
    \item [$J_{46}$] {skrętnie wydłużona dwukopuła pięciokątna}
                     (\emph{gyroelongated pentagonal bicupola})
    \item [$J_{47}$] {skrętnie wydłużona kopuło-rotunda pięciokątna}
                     (\emph{gyroelongated pentagonal cupolarotunda})
    \item [$J_{48}$] {skrętnie wydłużona dwurotunda pięciokątna}
                     (\emph{gyroelongated pentagonal birotunda})
    \item [$J_{49}$] {powiększony graniastosłup trójkątny}
                     (\emph{augmented triangular prism})
    \item [$J_{50}$] {podwójnie powiększony graniastosłup trójkątny}
                     (\emph{biaugmented triangular prism})
    \item [$J_{51}$] {potrójnie powiększony graniastosłup trójkątny}
                     (\emph{triaugmented triangular prism})
    \item [$J_{52}$] {powiększony graniastosłup pięciokątny}
                     (\emph{augmented pentagonal prism})
    \item [$J_{53}$] {podwójnie powiększony graniastosłup pieciokątny}
                     (\emph{biaugmented pentagonal prism})
    \item [$J_{54}$] {powiększony graniastosłup sześciokątny}
                     (\emph{augmented hexagonal prism})
    \item [$J_{55}$] {podwójnie osiowo powiększony graniastosłup sześciokątny}
                     (\emph{parabiaugmented hexagonal prism})
    \item [$J_{56}$] {podwójnie powiększony graniastosłup sześciokątny}
                     (\emph{metabiaugmented hexagonal prism})
    \item [$J_{57}$] {potrójnie powiększony graniastosłup sześciokątny}
                     (\emph{triaugmented hexagonal prism})
    \item [$J_{58}$] {powiększony dwunastościan}
                     (\emph{augmented dodecahedron})
    \item [$J_{59}$] {podwójnie osiowo powiększony dwunastościan}
                     (\emph{parabiaugmented dodecahedron})
    \item [$J_{60}$] {podwójnie powiększony dwunastościan}
                     (\emph{metabiaugmented dodecahedron})
    \item [$J_{61}$] {potrójnie powiększony dwunastościan}
                     (\emph{triaugmented dodecahedron})
    \item [$J_{62}$] {podwójnie obcięty dwudziestościan}
                     (\emph{metabidiminished icosahedron})
    \item [$J_{63}$] {potrójnie obcięty dwudziestościan}
                     (\emph{tridiminished icosahedron})
    \item [$J_{64}$] {powiększony potrójnie obcięty dwudziestościan}
                     (\emph{augmented tridiminished icosahedron})
    \item [$J_{65}$] {powiększony czworościan ścięty}
                     (\emph{augmented truncated tetrahedron})
    \item [$J_{66}$] {powiększony sześcian ścięty}
                     (\emph{augmented truncated cube})
    \item [$J_{67}$] {podwójnie powiększony sześcian ścięty}
                     (\emph{biaugmented truncated cube})
    \item [$J_{68}$] {powiększony dwunastościan ścięty}
                     (\emph{augmented truncated dodecahedron})
    \item [$J_{69}$] {podwójnie osiowo powiększony dwunastościan ścięty}
                     (\emph{parabiaugmented truncated dodecahedron})
    \item [$J_{70}$] {podwójnie powiększony dwunastościan ścięty}
                     (\emph{metabiaugmented truncated dodecahedron})
    \item [$J_{71}$] {potrójnie powiększony dwunastościan ścięty}
                     (\emph{triaugmented truncated dodecahedron})
    \item [$J_{72}$] {dwudziesto-dwunastościan rombowy skręcony}
                     (\emph{gyrate rhombicosidodecahedron})
    \item [$J_{73}$] {dwudziesto-dwunastościan rombowy podwójnie osiowo skręcony}
                     (\emph{parabigyrate rhombicosidodecahedron})
    \item [$J_{74}$] {dwudziesto-dwunastościan rombowy podwójnie skośnie skręcony}
                     (\emph{metabigyrate rhombicosidodecahedron})
    \item [$J_{75}$] {dwudziesto-dwunastościan rombowy potrojnie skręcony}
                     (\emph{trigyrate rhombicosidodecahedron})
    \item [$J_{76}$] {dwudziesto-dwunastościan rombowy obcięty}
                     (\emph{diminished rhombicosidodecahedron})
    \item [$J_{77}$] {przekręcony osiowo dwudziesto-dwunastościan rombowy obcięty}
                     (\emph{paragyrate diminished rhombicosidodecahedron})
    \item [$J_{78}$] {przekręcony skośnie dwudziesto-dwunastościan rombowy obcięty}
                     (\emph{metagyrate diminished rhombicosidodecahedron})
    \item [$J_{79}$] {podwójnie przekręcony dwudziesto-dwunastościan rombowy obcięty}
                     (\emph{bigyrate diminished rhombicosidodecahedron})
    \item [$J_{80}$] {dwudziesto-dwunastościan rombowy podwójnie osiowo obcięty}
                     (\emph{parabidiminished rhombicosidodecahedron})
    \item [$J_{81}$] {dwudziesto-dwunastościan rombowy podwójnie skośnie obcięty}
                     (\emph{metabidiminished rhombicosidodecahedron})
    \item [$J_{82}$] {przekręcony dwudziesto-dwunastościan rombowy podwójnie obcięty}
                     (\emph{gyrate bidiminished rhombicosidodecahedron})
    \item [$J_{83}$] {dwudziesto-dwunastościan rombowy potrójnie obcięty}
                     (\emph{tridiminished rhombicosidodecahedron})
    \item [$J_{84}$] {dwuklinoid przycięty}
                     (\emph{snub disphenoid})
    \item [$J_{85}$] {antygraniastosłup czworokątny przycięty}
                     (\emph{snub square antiprism})
    \item [$J_{86}$] {klinokorona}
                     (\emph{sphenocorona})
    \item [$J_{87}$] {powiększona klinokorona}
                     (\emph{augmented sphenocorona})
    \item [$J_{88}$] {klinomega- korona}
                     (\emph{sphenomegacorona})
    \item [$J_{89}$] {hebeklinomegakorona}
                     (\emph{hebesphenomegacorona})
    \item [$J_{90}$] {klinocingulum podwójne}
                     (\emph{disphenocingulum})
    \item [$J_{91}$] {dwusoczewkowa rotunda podwójna}
                     (\emph{bilunabirotunda})
    \item [$J_{92}$] {hebeklinorotunda trójkątna}
                     (\emph{triangular hebesphenorotunda})
\end{itemize}

% https://en.wikipedia.org/wiki/Johnson_solid
% http://matematyka.wroc.pl/book/wielosciany-johnsona
% {Siamese dodecahedron} https://en.wikipedia.org/wiki/Snub_disphenoid

%

\subsection{Wielościany jednorodne}
Zostawione na koniec, bo ich uzytecznoscj est ograniczona.

Uniform maja foremne sciany oraz vertex-transitive. Mamy dwie nieskonczone serie i 75 konkretnych modeli.

Dziela sie na:

- 9 regular (face, edge, to juz bylo). 5 platonskich, 4 keplera-poinsot. Zostaje 66

- quasi-regular (tylko edge-)

- 2 archimedesowe: There are only two convex quasiregular polyhedra: the cuboctahedron and the icosidodecahedron.

- 14 kwaziregularnych 'uniform star polyhedra'

- semi-regular (ani to, ani to).

- 11 archimedesowych: te, ktorych nie wymienilismy wczesniej XD

- 39 semiregular uniform star polyhedra

Lista usp jest tu: % https://en.wikipedia.org/wiki/Uniform_star_polyhedron

(Wygflada na to, ze Nie mozna byc tylko face-, ale dowodu brak).

Trzeba je wysumowac do 75.

5 platon, 13 archimedes. Zostaje 57.

he non-convex star polyhedra as in 4 Kepler-Poinsothe non-convex star polyhedra

and 53 % [uniform star polyhedra](https://en.wikipedia.org/wiki/Uniform_star_polyhedra "Uniform star polyhedra")—14 [quasiregular](https://en.wikipedia.org/wiki/Quasiregular_polyhedron#Nonconvex_examples "Quasiregular polyhedron") and 39 semiregular

There are two infinite classes of uniform polyhedra, together with % 75 other polyhedra. They are 2 infinite classes of [prisms](https://en.wikipedia.org/wiki/Prism_(geometry)) "Prism (geometry)") and [antiprisms](https://en.wikipedia.org/wiki/Antiprism "Antiprism"), the convex polyhedrons as in 5 [Platonic solids](https://en.wikipedia.org/wiki/Platonic_solid "Platonic solid") and 13 [Archimedean solids](https://en.wikipedia.org/wiki/Archimedean_solid "Archimedean solid")—2 [quasiregular](https://en.wikipedia.org/wiki/Quasiregular_polyhedron "Quasiregular polyhedron") and 11 [semiregular](https://en.wikipedia.org/wiki/Semiregular_polyhedron "Semiregular polyhedron")— the non-convex star polyhedra as in 4 [Kepler–Poinsot polyhedra](https://en.wikipedia.org/wiki/Kepler%E2%80%93Poinsot_polyhedra "Kepler–Poinsot polyhedra") and 53 [uniform star polyhedra](https://en.wikipedia.org/wiki/Uniform_star_polyhedra "Uniform star polyhedra")—14 [quasiregular](https://en.wikipedia.org/wiki/Quasiregular_polyhedron#Nonconvex_examples "Quasiregular polyhedron")

and 39 semiregular. There are also many degenerate uniform polyhedra

with pairs of edges that coincide, including one found by John Skilling

called the [great disnub dirhombidodecahedron] % (https://en.wikipedia.org/wiki/Great_disnub_dirhombidodecahedron "Great disnub dirhombidodecahedron"), Skilling's figure.^[**[**1**]**](https://en.wikipedia.org/wiki/Uniform_polyhedron#cite_note-FOOTNOTEDiudea2018https://books.google.com/books?id=p_06DwAAQBAJ&pg=PA40_40]-1)^

[Coxeter, Longuet-Higgins  % &amp; Miller (1954)](https://en.wikipedia.org/wiki/Uniform_polyhedron#CITEREFCoxeterLonguet-HigginsMiller1954) define uniform polyhedra to be [vertex-transitive](https://en.wikipedia.org/wiki/Vertex-transitive "Vertex-transitive")

polyhedra with regular faces.

% TEGO NIE BĘDZIE: {Wielościany Goldberga}

\subsection{Wielościany Szilassiego}
O wielościanach Szilassiego pisze $\Delta_{24}^4$.

\subsection{Walce i stożki}
% https://en.wikipedia.org/wiki/Cylinder

\subsection{Kule}
% https://en.wikipedia.org/wiki/On_the_Sphere_and_Cylinder pierwszy raz wzór objętość kuli

\section{Zliczanie małych wielościanów}
\subsubsection{Deltościany}
Jest osiem wielościanów wypukłych, których ściany są trójkątami równobocznymi: trzy platońskie bryły (czworościan, ośmiościan i dwudziestościan foremny) oraz pięć wielościanów Johnsona ($J_{12}$, $J_{13}$, $J_{17}$, $J_{51}$, $J_{84}$).
Pokażą to w 1947 roku Hans Freudenthal i Bartel Leendert van der Waerden \cite{Freudenthal_1947} w obskurnym duńskim żurnalu.
% Freudenthal, H.; van der Waerden, B. L. (1947), "On an assertion of Euclid", Simon Stevin, 25: 115–121, MR 0021687
% https://mathscinet.ams.org/mathscinet/relay-station?mr=0021687
Ich wynik zreferuje później Adam Gajda w $\Delta_{84}^{4}$.

Czasami nazywa się je deltościanami, przez podobieństwo litery $\Delta$ do ich ścian.
Niewypukłych brył o tej własności jest nieskończnie wiele (!), mogą mieć dowolną parzystą liczbę ścian większą niż sześć.

% https://en.wikipedia.org/wiki/Deltahedron  % http://matematyka.wroc.pl/book/deltosciany
\section{Walce i stożki}
% https://en.wikipedia.org/wiki/Cylinder

\subsubsection{Sześciościany}
Jest siedem sześciościanów wypukłych (i trzy niewypukłe).
Heinz Schumann oraz Bronisław Pabich napiszą krótki artykuł w $\Delta_{24}^{11}$, gdzie uzasadnią te liczby diagramami Schlegela.
\index{diagram Schlegela}%
(Wielościany wypukłe mają 6/0/0, 5/0/1, 4/2/0, 3/2/1, 2/4/0, 2/2/2, 0/6/0 ścian o trzech, czterech, pięciu krawędziach).
% TODO: DELTA 2001 luty Kordos
Oprócz tego są trzy typy niewypukłych sześciościanów, które powstają przez wycięcie czworościanu z czworościanu.
% Cvetković Dragoš i Milenko Petrić, ,,A table of connected graphs on six vertices”, Discrete Mathematics 50 (1984): 37–49.

\subsubsection{Wielościany z $n$ przekątnymi}
Jerzy Bednarczuk w $\Delta_{23}^{12}$ pokaże bez dowodu:

\begin{proposition}
    Istnieją co najmniej 3 wielościany wypukłe z jedną przekątną.
\end{proposition}

\begin{proposition}
    Istnieje co najmniej 8 wielościanów wypukłych z dwiema przekątnymi.     
\end{proposition}


% Ilejesttypówczworościanów? Dziwnepytanie,oczywiścienieskończeniewiele.Sąbardziejspłaszczone, wydłużone,szpiczaste,sąteżprawidłowe,foremne Jeślijednakpominiemy długościkrawędzi,kątypłaskieidwuścienneitp.,zachowamytylkoogólną strukturę,towidzimy,żejesttylkojedentypczworościanu.Nieistnieje czworościan,którymiałbyścianęnietrójkątną.Wprzypadkupięciościanów mamydwatypy:typostrosłupaopodstawieczworokątaoraztypgraniastosłupa opodstawietrójkąta.Pierwszymajednąścianęczworokątnąiczterytrójkątne, adrugitrzyczworokątneidwietrójkątne.Innychniema.Gdyzetniemy wierzchołekczworościanu,todostaniemypięciościandrugiegotypu.Powołując sięnaJohnaMcClelana(artystęzWoodstock),MartinGardnerzadałpytanie: ilejesttypówsześciościanów(wdomyślewypukłych)? Należyuważaćnaprzypadkiwyglądającenapozórróżnie,ajednakdające tensamtyp.Możnapójśćdalejizapytaćoliczbętypówsiedmiościanów, ośmiościanówitd.Możejestjakiśogólnyschematpostępowania? Gardnerprzytacza7różnychtypówwypukłychsześciościanów,dodając,iż nieznaprostegodowodu,żeniemainnych.Informujeteż,żeistnieją34rodzaje wypukłychsiedmiościanów,257ośmiościanówi2606dziewięciościanów.Natomiast niewypukłychsześciościanówmamytrzytypy, 26siedmiościanówi277ośmiościanów delta 2011-01 Dowódtwierdzenia,żeistniejedokładnie siedemwypukłychsześciościanów,można znaleźćwpracyDonaldaCrowe’a,Euler’s formulaforpolyhedraandrelatedtopics w:A.Beck,M.Bleicher,D.Crowe, ExcursionsintoMathematics,Worth, 1969,str.29–30. Dlaspecjalistówpodamyprofesjonalne źródła: P.J.Federico, Enumerationofpolyhedra:Thenumber of 9-hedra,J.Combin.Theory7(1969), 155–161; Polyhedrawith4or8faces,Geom. Dedicata3(1975),469–481; Thenumberofpolyhedra,PhilipsRes. Rep.30(1975),220–231.

\section{Do przerobienia}


\subsection{Twierdzenie Sylvestera-Gallaia}
\begin{theorem}[Sylvestera-Gallaia]
	Dla każdego skończonego zbioru punktów na płaszczyźnie istnieje prosta, która przechodzi przez dokładnie dwa albo wszystkie punkty.
\end{theorem}

Mamy wrażenie, że zaczęło się w 1893 roku, kiedy James Sylvester postawił problem.
Być może zainspirowała go konfiguracją Hessego\footnote{Konfiguracja Hessego to 12 prostych przez 9 punktów na zespolonej płaszczyźnie rzutowej, gdzie każdy punkt leży na 4 prostych, a każda prosta przechodzi przez 3 punkty}.
Herbert Woodall szybko zaproponował rozwiązanie, gdzie równie szybko wychwycono usterkę.
Dopiero w 1941 roku Eberhard Melchior udowodnił trochę mocniejsze stwierdzenie niż rzutowy dual ówczesnej hipotezy (że prostych przez dokładnie dwa punkty jest co najmniej trzy).
Nieświadomy tego, Paul ErdErdős postawił hipotezę na nowo w~1943 roku, a Tibor Gallai w 1944 roku dodał swój dowód (ponownie wykorzystując elementy geometrii rzutowej).
Wraz z upływem czasu pojawiały się inne, ciekawe rozumowania.
Na przykład Leroy Kelly wykorzystał własności metryki, co oburzyło Harolda Coxetera i skłoniło go do opublikowania kolejnego dowodu, korzystającego jedynie z aksjomatów geometrii uporządkowania.
(Aigner, Ziegler uważają dowód Kelly'ego za najlepszy).

Niech $t_2(n)$ oznacza minimalną liczbę prostych przez dwa punkty w dowolnym ułożeniu $n$ punktów.
Melchior pokazał, że $t_2(n) \ge 3$.
Wynik sukcesywnie poprawiano:
de Bruijn \cite{debruijn_1948} zapytał, czy $t_2(n)$ dąży do nieskończoności,
Theodore Motzkin \cite{motzkin_1951} udzielił twierdzącej odpowiedz, bo $t_2(n) \ge \sqrt{n}$.
Potem Gabriel Dirac \cite{dirac_1951} przypuścił, że $t_2(n) \ge \lfloor n/2\rfloor$, co nie zostawia wiele miejsca na poprawki, bo dla parzystych $n \ge 6$ zachodzi $t_2(n) \le n/2$, jak pokazał pomysłową konstrukcją Károly Böröczky.
Dla nieparzystych $n$ wiemy tylko, że ten kres jest realizowany dla $n = 7$ (Kelly, Moser \cite{kelly_1958} w 1958) i $n = 13$ (Crowe, McKee \cite{mckee_1968} w 1968).
Najnowszy wynik, o jakim nam wiadomo, to Csimy, Sawyera \cite{csima_1993}: że $t_2(n) \ge \lceil 6n/13 \rceil$.

\subsection{Inwersje (UW-2)}
\begin{enumerate}
	\item Obrazy inwersyjne okręgów i prostych, konforemność inwersji, okręgi stałe inwersji, okręgi prostopadłe
	\item zmiana odległości przy inwersji, zmiana promienia okręgu przy inwersji,
	\item twierdzenie Ptolemeusza,
	\item łańcuchy Steinera
	\item formuła Kartezjusza
	\item formuła Fussa dla czworokątów,
	\item twierdzenie Feuerbacha.
\end{enumerate}

\subsection{Stożkowe (UW-2)}
\begin{enumerate}
	\item Ogniska elipsy i hiperboli, ognisko, kierownica i mimośród stożkowych, asymptoty hiperboli, konstrukcja stycznej do stożkowej, rzuty ustalonego ogniska na styczne, własności izogonalne stożkowych, równania kanoniczne stożkowych, elipsa jako przekrój walca.
	\item Ognisko, kierownica i mimośród stożkowej na przekroju stożka.
	\item Przekroje stożków ze sferami wpisanymi.
	\item Równanie ogólne stożkowej w układzie współrzędnych, duży i mały wyznacznik.
	\item Równania stożkowych we współrzędnych biegunowych.
\end{enumerate}

Neugebauer 262: w każdy właściwy czworobok zupełny da się wpisać dokładnie jedną parabolę, jej ogniskiem jest punkt Miquela czworoboku.
Jemieljanow: punkt Miquela właściwego czworoboku zupełnego leży na okręgu dziewięciu punktów trójkąta przekątnego tego czworoboku.
Droz-Farny: proste przechodzą przez ortocentrum trójkąta i są prostopadłe, wtedy środki odcinków leżą na jednej prostej.

\subsection{Przekształcenia afiniczne (UW-2)}
\begin{enumerate}
	\item Grupa przekształceń afinicznych od strony geometrycznej: powinowactwa osiowe, rozkład przekształcenia afinicznego na podobieństwo i powinowactwo osiowe, kierunki główne przekształcenia afinicznego.
	\item niezmienniczość stosunku pól przy przekształceniu afinicznym
	\item obraz okręgu przy przekształceniu afinicznym
\end{enumerate}

\subsection{UW-3}
\begin{enumerate}
	\item zna pojęcie płaszczyzny rzutowej rzeczywistej (równoważne sformułowania), dwustosunku, definicję przekształceń rzutowych łańcuchów, pęków, stożkowych, pęków stycznych do stożkowych. 
	\item Rozumie, czym są stożkowe w ujęciu rzutowym, zna typy stożkowych.
	\item Zna i potrafi stosować twierdzenia Steinera i Braikenridge'a-Maclaurina.
	\item Wie w jaki sposób określa się rzutowo ogniska i kierownice stożkowych.
\end{enumerate}

\subsection{Guzicki}
\begin{enumerate}
	\item Złoty podział i pięciokąt.
	\item Zagadnienie izoperymetryczne (6)
	\item nierówności geometryczne: stosunek sumy środkowych do obwodu leży między 3/4 i 1 (s. 355), $s <= p^2 / 3 \sqrt 3$ - przypomnienie nierówności izoperymetrycznej. nierówność eulera (R >= 2r), Mitrinovica, Leibniza, Weitzenbocka (s. 362). Twierdzenie Eulera: $d^2 = R^2 - 2Rr$. nierówność Erdosa-Mordella: P leży wewnątrz trójkąta, K L M to rzuty na boki. Wtedy PA + PB + PC >= 2 (PK + PL + PM). Mikołaj z Kuzy: $\sin x / x < (2 + \cos x) / 3$. Snellius-Huygens: $2 \sin x + \tan x > 3x$.
	\item przekątne w wielokącie, tw. Heinekena % n nieparzyste -> w n-kącie foremnym żadne trzy przekątne nie przecinają się -> https://arxiv.org/pdf/math/9508209v3 ... In the 1960s, Heineken [6] gave a delightful argument which generalized this to all odd n,
\end{enumerate}

\subsection{Starocie}
Twierdzenie Chasles'a: każda izometria płaszczyzny jest złożeniem co najwyżej trzech symetrii osiowych.
Symetria osiowa z poślizgiem.
Słowo Banacha.
Klasyfikacja podobieństw.
Okrąg siedmiu punktów. % https://mathworld.wolfram.com/BrocardCircle.html ?
Przekształcenia afiniczne i rzutowe.
% https://www.cut-the-knot.org/Curriculum/Geometry/HeronsProblem.shtml
% This one is a basic optimization problem. It's quite famous, being discussed in Heron's Catoptrica (On Mirrors from the Greek word Katoptron Catoptron = Mirror) that, in all likelihood, saw the light of day some 2000 years ago.
Pitagorasa % https://en.wikipedia.org/wiki/Pythagorean_theorem
% https://en.wikipedia.org/wiki/Spiral_of_Theodorus

gnomon % https://en.wikipedia.org/wiki/Theorem_of_the_gnomon

Czwarty aksjomat uporządkowania znalazł Moritz Pasch \cite{pasch_1882} w 1882 roku.
\index[persons]{Pasch, Moritz}
\index{aksjomat!Pascha}

\subsection{Zadania}
\textbf{Zadanie} (Guzicki, s. 304).
Na bokach $AB$, $BC$, $CD$ i $DA$ czworokąta wypukłego $ABCD$ zbudowano, na zewnątrz czworokąta, kwadraty $ABFE$, $BCHG$, $CDJI$ i $DALK$.
Punkty $P$, $Q$, $R$ i $S$ są odpowiednio środkami kwadratów $ABFE$, $BCHG$, $CDJI$ i $DALK$.
Udowodnij, że odcinki $PR$ i $QS$ są równej długości oraz wzajemnie prostopadłe.

\textbf{Zadanie} (Guzicki, s. 306).
(XLIV OM, zadanie 5/I).
Dana jest półpłaszczyzna oraz punkty $A$ i $C$ na jej krawędzi.
Dla każdego punktu $B$ tej półpłaszczyzny rozważamy kwadraty $ABKL$ i $BCMN$ leżące na zewnątrz trójkąta $ABC$.
Wyznaczają one odpowiadającą punktowi $B$ prostą $LM$.
Udowodnij, że wszystkie proste odpowiadające różnym położeniom punktu $B$ przechodzą przez jeden punkt.

\textbf{Zadanie} (Guzicki, s. 306).
Na bokach $AB$ i $AC$ trójkąta $ABC$ zbudowano, po jego zewnętrznej stronie, kwadraty $ABDE$ i $ACFG$.
Punkty $M$ i $N$ są odpowiednio środkami odcinków $DG$ i $EF$.
Wyznacz możliwe wartości wyrażenia $MN / BC$.

\textbf{Zadanie} (Guzicki, s. 307)
(TWIERDZENIE NAPOLEONA)
Na bokach $AB$, $BC$ i $CA$ trójkąta $ABC$ zbudowano, na zewnątrz trójkąta, trójkąty równoboczne $ABF$, $BCD$ i $CAE$.
Udowodnij, że środki tych trójkątów równobocznych są wierzchołkami trójkąta równobocznego.

\textbf{Zadanie} (Guzicki, s. 308)
Na bokach $AB$, $BC$ i $CA$ trójkąta $ABC$ wybrano odpowiednio punkty $D$, $E$ i $F$ tak, że $AD : DB = BE : EC = CF : FA$.
Udowodnij, że jeśli trójkąt $DEF$ jest równoboczny, to trójkąt $ABC$ też jest równoboczny.

\textbf{Zadanie} (Guzicki, s. 310)
(XLV OM, zadanie 7/I)
Na zewnątrz czworokąta wypukłego $ABCD$ budujemy trójkąty podobone $APB$, $BQC$, $CRD$, $DSA$ w ten sposób, że kąty $PAB, QBC, RCD, SDA$ są sobie równe i że kąty $PBA, QCB, RDS, SAD$ też są sobie równe.
Udowodnij, że jeśli czworokąt PQRS jest równoległobokiem, to czworokąt $ABCD$ też jest równoległobokiem.

%

% TODO: https://en.wikipedia.org/wiki/Nicolo_Tartaglia