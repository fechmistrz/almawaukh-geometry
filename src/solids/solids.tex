
Tekst intro rozdziału stereometria.

\section{Wielościany}
\subsection{Ostrosłupy}

Eves \cite[s.4]{eves_1963} napisze, że około 1850 lat przed Chrystusem znany będzie wzór na dokładną objętość ostrosłupa czworokątnego ściętego, o podstawie długości $a$, $b$ i wysokości $h$:
\begin{equation}
	V = \frac 1 3 h (a^2 + ab + b^2).
\end{equation}
(Po angielsku taką bryłę nazywa się \emph{frustum}, co w łacinie znaczy \emph{kęs, kawałek}).
Wzór pojawi się bez dowodu w papirusie moskiewskim, nazwanym tak, ponieważ jego pierwszym właścicielem spoza Egiptu będzie Władimir Goleniszczew, rosyjski egiptolog i kolekcjoner sztuki.
\index[persons]{Goleniszczew, Władimir}%

Pryzmatoid to bryła, której wierzchołki leżą na dwóch płaszczyznach równoległych.
Jej objętość wyraża wzór $(l + 4m + h)/ 6$, gdzie $l$ to pole dolnego przekroju, $m$ środkowego, $h$ górnego. % eves s. 40
% TODO https://en.wikipedia.org/wiki/Prismatoid#Prismatoid_families

\subsection{Graniastosłupy}

% https://en.wikipedia.org/wiki/Polyhedron
% https://en.wikipedia.org/wiki/Dual_polyhedron
% https://www.youtube.com/watch?v=yAEveAOH2KwI => Schwarz lantern

% TODO: https://en.wikipedia.org/wiki/Wallace-Bolyai-Gerwien_theorem to nie działa w 3d a jest ok w 2d
Nie ma odpowiednika twierdzenia Wallace'a-Bolyaia-Gerwiena w trzech wymiarach.
Dwa wielościany nazwijmy równoważnymi przez podział, jeśli pierwszy da się tak podzielić na skończoną liczbę wielościanów, by z otrzymanych kawałków można było złożyć drugi.
To, czy każde dwa czworościany o równych podstawach i równych wysokościach będzie stanowić treść trzeciego problemu Hilberta.
Negatywnej odpowiedzi udzieli Dehn w 1900 roku, prostszy dowód pokaże Hadwiger w 1954 roku.
% Hadwiger, Glur pokazali, że kiedy boki kawałków mogą musieć być równoległe % Hadwiger, H.; Glur, P.: Zerlegungsgleichheit ebener Polygone. Elem. Math. 6 (1951), 97–106.

\begin{theorem}[Hadwigera] % Delta 1984 11
    Niech $\alpha_1$, $\ldots$, $\alpha_p$ będą kątami dwuściennymi wielościanu $W$ leżącymi wzdłuż krawędzi długości $k_1$, $\ldots$, $k_p$, a $\beta_1$, $\ldots$, $\beta_q$ kątami wielościanu $V$ (wzdłuż krawędzi długości $l_1$, $\ldots$, $l_q$).
    Definiujemy, z lekkim nadużyciem notacji,
    \begin{equation}
        W := \sum_{i \le p} k_i \alpha_i
    \end{equation}
    i analogicznie liczbę $V$.
    Jeśli istnieje taka funkcja $\mathbb Z$-addytywna na zbiorze $M \subseteq \mathbb R$ zawierającym $\pi$, $\alpha_1$, $\ldots$, $\alpha_p$, $\beta_1$, $\ldots$, $\beta_q$, że $f(W) \neq f(V)$, to $W$ i $V$ nie są równoważne przez podział.
\end{theorem}

\begin{corollary}
    Czworościan o kątach dwuściennych
    \begin{equation}
        \frac \pi 2, \frac \pi 2, \frac \pi 2,
        \arccos \frac{1}{\sqrt 3}, \arccos \frac{1}{\sqrt 3}, \arccos \frac{1}{\sqrt 3}
    \end{equation}
    nie jest równoważny przez podział z żadnym sześcianem.
    Dla dowodu można wziąć $M = \{\pi/2, \pi, \alpha\}$ i określić $f(\pi/2) = f(\pi) = 0$, $f(\alpha) = 1$.
    Wtedy funkcja $f$ przyjmuje na czworościanie wartość $3 \ \sqrt 2$, zaś na sześcianie $0$.
\end{corollary}

Taki sam wniosek wyciągnie nieznany autor w $\Delta_{84}^{11}$.

% Unknown to Hilbert and Dehn, Hilbert's third problem was also proposed independently by Władysław Kretkowski for a math contest of 1882 by the Academy of Arts and Sciences of Kraków, and was solved by Ludwik Antoni Birkenmajer with a different method than Dehn's. Birkenmajer did not publish the result, and the original manuscript containing his solution was rediscovered years later.[3] 
% Gauss regretted this defect in two of his letters to Christian Ludwig Gerling, who proved that two symmetric tetrahedra are equidecomposable.
% Two polyhedra are called scissors-congruent if the first can be cut into finitely many polyhedral pieces that can be reassembled to yield the second. Any two scissors-congruent polyhedra have the same volume. Hilbert asks about the converse. 
% In light of Dehn's theorem above, one might ask "which polyhedra are scissors-congruent"? Sydler (1965) showed that two polyhedra are scissors-congruent if and only if they have the same volume and the same Dehn invariant.[5] Børge Jessen later extended Sydler's results to four dimensions.[6] In 1990, Dupont and Sah provided a simpler proof of Sydler's result by reinterpreting it as a theorem about the homology of certain classical groups.[7] 
% http://sciencecow.mit.edu/me/hilberts_third_problem.pdf

\subsection{Zliczanie małych wielościanów}
\subsubsection{Deltaściany}
Jest osiem wielościanów wypukłych, których ściany są trójkątami równobocznymi: trzy platońskie bryły (czworościan, ośmiościan i dwudziestościan foremny) oraz pięć wielościanów Johnsona ($J_{12}$, $J_{13}$, $J_{17}$, $J_{51}$, $J_{84}$).
Pokażą to w 1947 roku Hans Freudenthal i Bartel Leendert van der Waerden \cite{Freudenthal_1947} w obskurnym duńskim żurnalu.
% Freudenthal, H.; van der Waerden, B. L. (1947), "On an assertion of Euclid", Simon Stevin, 25: 115–121, MR 0021687
% https://mathscinet.ams.org/mathscinet/relay-station?mr=0021687
Ich wynik zreferuje później Adam Gajda w $\Delta_{84}^{4}$.

Czasami nazywa się je deltościanami, przez podobieństwo litery $\Delta$ do ich ścian.
Niewypukłych brył o tej własności jest nieskończnie wiele (!), mogą mieć dowolną parzystą liczbę ścian większą niż sześć.

% https://en.wikipedia.org/wiki/Deltahedron 

Jest siedem sześciościanów wypukłych (i trzy niewypukłe).
Heinz Schumann oraz Bronisław Pabich napiszą krótki artykuł w $\Delta_{24}^{11}$, gdzie uzasadnią te liczby diagramami Schlegela.
\index{diagram Schlegela}%
(Wielościany wypukłe mają 6/0/0, 5/0/1, 4/2/0, 3/2/1, 2/4/0, 2/2/2, 0/6/0 ścian o trzech, czterech, pięciu krawędziach).
% TODO: DELTA 2001 luty Kordos

\subsection{Wielościany platońskie}
% https://en.wikipedia.org/wiki/Platonic_solid
% \section{Pięć wielościanów} Hartshorne: rozdział 8
% \section{Cauchy's rigidity theorem} Hartshorne: section 45

\subsubsection{Czworościan}
Czworościan

\subsubsection{Sześcian}
Sześcian

\subsubsection{Ośmiościan}
Ośmiościan
% https://en.wikipedia.org/wiki/Octahedron

\subsubsection{Dwunastościan}
Dwunastościan

\subsubsection{Dwudziestościan}
Dwudziestościan
% https://en.wikipedia.org/wiki/Regular_icosahedron

\todofoot{Platonic solid} 

\subsection{Wielościany archimedesowe}
\todofoot{1955 - H. S. M. Coxeter et al. publish the complete list of uniform polyhedron}

\subsection{Wielościany Keplera-Poinsota}
\todofoot{1806 - Louis Poinsot discovers the two remaining Kepler-Poinsot polyhedra.}
\todofoot{1619 - Johannes Kepler discovers two of the Kepler-Poinsot polyhedra}

% https://en.wikipedia.org/wiki/Great_icosahedron

Wielościany Keplera-Poinsota
Wielościany Keplera-Poinsota są odpowiednikami brył platońskich w świecie wielościanów niewypukłych. Również ich ściany są przystającymi wielokątami foremnymi i w każdym wierzchołku spotyka się taka sama liczba ścian. Tym razem jednak ściany mogą być wielokątami gwiaździstymi. Dopuszczona jest także możliwość przenikania ścian (tzn. przecinania się poza krawędziami).
Istnieją tylko 4 wielościany foremne niewypukłe:
dwunastościan gwiaździsty mały (small stellated dodecahedron),
dwunastościan wielki (great dodecahedron),
dwunastościan gwiaździsty wielki (great stellated dodecahedron),
dwudziestościan wielki (great icosahedron). 

\subsection{Wielościany Catalana}
%

Wielościany Catalana to bryły dualne do archimedesowych. Takie bryły po raz pierwszy zostały opisane w 1865 roku przez belgijskiego matematyka Eugene Catalana - stąd ich nazwa.
Okazuje się, że bryły dualne do wielościanów platońskich są platońskie, a dualne do Keplera-Poinsota są bryłami Keplera-Poinsota, więc nie dają żadnych nowych obiektów. Dają je dopiero wielościany archimedesowe.
Ponieważ dla danego wielościanu jego bryła dualna wyznaczona jest jednoznacznie, mamy 13 wielościanów Catalana, bo jest 13 wielościanów archimedesowych. Są to:
 
\begin{itemize}
\item czworościan potrójny (triakistetrahedron),
\item ośmiościan potrójny (triakisoctahedron),
\item dwudziestościan potrójny (triakisicosahedron),
\item sześciościan poczwórny (tetrakishexahedron),
\item dwunastościan piątkowy (pentakisdodecahedron),
\item ośmiościan szóstkowy (hexakisoctahedron),
\item dwudziestościan szóstkowy (hexakisicosahedron),
\item dwunastościan rombowy (rhombic dodecahedron),
\item trzydziestościan rombowy (rhombic triacontahedron),
\item sześćdziestościan deltoidowy (strombic hexecontahedron),
\item dwudziestoczterościan deltoidowy (strombic icositetrahedron),
\item dwudziestoczterościan pięciokątny (pentagonal icositetrahedron),
\item sześćdziestościan pieciokątny (pentagonal hexecontahedron).
\end{itemize}
% sześćdziestościan deltoidowy (strombic hexecontahedron), % Taką nazwę zaproponował prof. Roman Duda, tłumacząc ponad 40 lat temu książkę "Modele matematyczne" Cundy'ego i Rolleta i tak już zostało, choć rzeczywiście to trochę dziwnie brzmi.

%

\subsection{Wielościany Johnsona}
% https://en.wikipedia.org/wiki/Johnson_solid

% \section{Siamese dodecahedron} https://en.wikipedia.org/wiki/Snub_disphenoid
\begin{itemize}
\item List of Johnson solids 
\item Johnson solid 
\item Elongated square gyrobicupola 
\item Triaugmented triangular prism 
\item Square pyramid 
\item Pentagonal bipyramid 
\item Elongated square cupola 
\item Elongated pentagonal bipyramid 
\item Elongated triangular cupola 
\item Triangular bipyramid 
\item Gyroelongated square bipyramid 
\item Gyroelongated pentagonal pyramid 
\item Sphenomegacorona 
\item Snub disphenoid 
\item Pentagonal pyramid 
\item Gyroelongated pentagonal bicupola 
\item Gyroelongated square pyramid 
\item Cupola (geometry) 
\item Elongated triangular orthobicupola 
\item Gyroelongated triangular cupola 
\item Gyroelongated square cupola 
\item Elongated pentagonal rotunda 
\item Gyroelongated pentagonal cupola 
\item Gyroelongated pentagonal rotunda 
\item Elongated square bipyramid 
\item Square cupola 
\item Triangular cupola 
\item Bilunabirotunda 
\item Elongated square pyramid 
\item Elongated pentagonal pyramid 
\item Elongated pentagonal cupola 
\item Pentagonal cupola 
\item Pentagonal rotunda 
\item Augmented triangular prism 
\item Augmented truncated tetrahedron 
\item Biaugmented triangular prism 
\item Square orthobicupola 
\item Triangular hebesphenorotunda 
\item Gyrobifastigium 
\item Elongated triangular pyramid 
\item Elongated triangular bipyramid 
\item Augmented hexagonal prism 
\item Elongated triangular gyrobicupola 
\item Snub square antiprism 
\item Augmented pentagonal prism 
\item Biaugmented pentagonal prism 
\item Gyroelongated square bicupola 
\item Triaugmented truncated dodecahedron 
\item Hebesphenomegacorona 
\item Triangular orthobicupola 
\item Gyrate rhombicosidodecahedron 
\item Sphenocorona 
\item Disphenocingulum 
\item Metabidiminished rhombicosidodecahedron 
\item Pentagonal gyrobicupola 
\item Tridiminished icosahedron 
\item Tridiminished rhombicosidodecahedron 
\item Birotunda 
\item Pentagonal orthobirotunda 
\item Rotunda (geometry) 
\item Square gyrobicupola 
\item Augmented sphenocorona 
\item Elongated pentagonal gyrobirotunda 
\item Elongated pentagonal orthobicupola 
\item Gyroelongated triangular bicupola 
\item Metabidiminished icosahedron 
\item Gyroelongated pentagonal birotunda 
\item Gyroelongated pentagonal cupolarotunda 
\item Parabiaugmented dodecahedron 
\item Trigyrate rhombicosidodecahedron 
\item Elongated pentagonal gyrobicupola 
\item Parabidiminished rhombicosidodecahedron 
\item Elongated pentagonal orthobirotunda 
\item Augmented truncated cube 
\item Elongated pentagonal gyrocupolarotunda 
\item Pentagonal orthobicupola 
\item Elongated pentagonal orthocupolarotunda 
\item Metabigyrate rhombicosidodecahedron 
\item Bigyrate diminished rhombicosidodecahedron 
\item Metagyrate diminished rhombicosidodecahedron 
\item Gyrate bidiminished rhombicosidodecahedron 
\item Diminished rhombicosidodecahedron 
\item Paragyrate diminished rhombicosidodecahedron 
\item Parabigyrate rhombicosidodecahedron 
\item Parabiaugmented hexagonal prism 
\item Metabiaugmented hexagonal prism 
\item Metabiaugmented dodecahedron 
\item Pentagonal gyrocupolarotunda 
\item Triaugmented dodecahedron 
\item Triaugmented hexagonal prism 
\item Pentagonal orthocupolarotunda 
\item Parabiaugmented truncated dodecahedron 
\item Augmented truncated dodecahedron 
\item Biaugmented truncated cube 
\item Augmented dodecahedron 
\item Metabiaugmented truncated dodecahedron 
\item Augmented tridiminished icosahedron
\end{itemize}

\subsection{Wielościany Goldberga}
\todofoot{Goldberg polyhedron} % https://en.wikipedia.org/wiki/Goldberg_polyhedron

\subsection{Wielościany Szilassiego}
O wielościanach Szilassiego pisze $\Delta_{24}^4$.



\section{Kule, walce}
% https://en.wikipedia.org/wiki/On_the_Sphere_and_Cylinder pierwszy raz wzór objętość kuli