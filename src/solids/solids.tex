
Tekst intro rozdziału stereometria.

Panuje nieziemski bałagan, jeśli chodzi o terminologię dla wielościanów lub, ogólniej, wielotopów. By uniknąć ryzyka zgubienia się, podamy jedynie podstawowe definicje, a nie twierdzenia.


Z każdym wielościanem związana jest grupa jego izometrii: takich przekształceń (odbić, obrotów) bryły na siebie, która zachowuje odległości.

Tylko odwzorowanie tożsamościowe przenosi flagę na siebie.

**Definicja** (symetrie wielościanów) Wielościan nazywamy izohedralnym (izotoksalnym, izogonalnym), jeśli jego grupa izometrii działa tranzytywnie na zbiorze ścian (krawędzi, wierzchołków).

**Przykład** Wielościany platońskie są izogonalne, izotoksalne i izohedralne jednocześnie.

Edmund Hess, Max Brückner, a później także Branko Grünbaum będą badać wielościany szlachetne czyli takie, które są izohedralne oraz izogonalne. Wiadomo, że wielościany platońskie, Keplera-Poinsota są szlachetne, poza nimi dwusfenoid i ogólniej, dowolne sfenoidy (wielościany koronne?). -> % https://en.wikipedia.org/wiki/Noble_polyhedron

**Przykład**. Wielościany Catalana, podwójne ostrosłupy, latawcościany są izohedralne. Bryły dwoiste do nich, czyli wielościany archimedesowe, graniastosłupy i antygraniastosłupy są izogonalne.

Wielościan dwoisty do izohedralnego jest zawsze izogonalny

Wielosciany foremne maja jesli maja identyczne foremen sciany i naroza.

Wielościany archimedesowe są wypukłe, mają foremne ściany (dwoch lub trzech rodzajow) i są izogonalne, ale nie izohedralne.

Catalana dualne do archimedesowych.

Uniform maja foremne sciany oraz vertex-transitive.

Wielościan Johnsona(-Zalgallera) to wypukły wielościan, którego ściany są wielokątami foremnymi.


Przez analogię do wielokątów wypukłych mamy:

\begin{definition}[wielościan wypukły]
    Obszar skończony przestrzeni przekrój półprzestrzeni...
    Część wycięta z płaszczyzny przez pozostałe jest wielokątem, nazywamy go ścianą.
    Wspólny bok dwóch ścian to krawędź, wspólny koniec dwóch krawędzi to wierzchołek.
\end{definition}

Najbardziej znanymi są ostrosłupy i graniastosłupy.
Oprócz nich opowiemy także o podwójnych ostrosłupach, latawcościanach, różnych mniej lub bardziej foremnych bryłach.
Natomiast nie podamy definicji wielościanu (niewypukłego), ponieważ nie ma powszechnie akceptowanej.

\begin{definition}[flaga]
    Zbiór złożony ze ściany, krawędzi oraz wierzchołka takich, że każde leży na poprzednim, nazywamy flagą.
\end{definition} % https://en.wikipedia.org/wiki/Flag_(geometry)


\section{Ostrosłupy}
\input{solids/pyramid}

\section{Graniastosłupy}

\subsubsection{Pryzmatoid}
Pryzmatoid to bryła, której wierzchołki leżą na dwóch płaszczyznach równoległych.
Jej objętość wyraża wzór $(l + 4m + h)/ 6$, gdzie $l$ to pole dolnego przekroju, $m$ środkowego, $h$ górnego. % eves s. 40
% TODO https://en.wikipedia.org/wiki/Prismatoid#Prismatoid_families


\section{Wielościany foremne}
**Definicja**. Wielościan nazywamy foremnym, jeśli jego grupa izometrii działa tranzytywnie na zbiorze flag.

Klasycznie podawane były inne definicje, z czego najpowszechniejsza: idetnyczne ściany foremne, identyczne naroża.

**Fakt**. Istnieje dziewięć albo czternaście wielościanów foremnych: pięć brył platońskich, cztery gwieździste (Keplera-Poinsota) i pięć złożeń (czasami nieuznawanych za wielościany foremne).


\subsection{Wielościany platońskie}
%

Wszystkie ściany wielościanów platońskich są przystającymi wielokątami foremnymi, w każdym wierzchołku spotyka się taka sama liczba ścian.
Pozostaje zagadką, kto pierwszy odkryje wszystkie foremne wielościany wypukłe, ale Teajtet, uczeń Platona udowodni, że jest ich dokładnie pięć:
\begin{itemize}
\item czworościan foremny,
\item sześciościan foremny, czyli sześcian,
\item ośmiościan foremny,
\item dwunastościan foremny,
\item dwudziestościan foremny.
\end{itemize}

(Dwunastościan zostanie odkryty jako ostatni).
Niektórzy będą mylić wielościany foremne z platońskimi (!).

Coxeter 166-167.

OCTAEDRON ELEVATUM

% https://en.wikipedia.org/wiki/Platonic_solid
% \section{Pięć wielościanów} Hartshorne: rozdział 8
% \section{Cauchy's rigidity theorem} Hartshorne: section 45

%

\subsection{Wielościany Keplera-Poinsota}
\todofoot{1806 - Louis Poinsot discovers the two remaining Kepler-Poinsot polyhedra.}
\todofoot{1619 - Johannes Kepler discovers two of the Kepler-Poinsot polyhedra}

% https://en.wikipedia.org/wiki/Great_icosahedron
small stellated dodecahedron 
great stellated dodecahedron 
great dodecahedron 
great icosahedron 


\subsection{Złożenia?}
In geometry, a polyhedral compound is a figure that is composed of several polyhedra sharing a common centre. They are the three-dimensional analogs of polygonal compounds such as the hexagram.

A regular polyhedral compound can be defined as a compound which, like a regular polyhedron, is vertex-transitive, edge-transitive, and face-transitive. Unlike thec ase of polyhedra, this is not equivalent to the symmetry group acting transitively on its flags; the compound of two tetrahedra is the only regular compound with that property.

(czemu nie jest flag-transitive???)

najbardziej znany: stella octangula

% https://en.wikipedia.org/wiki/Compound_of_four_cubes -> to nie jest regular



\section{Wielościany archimedesowe i Catalana}


Wszystkie ściany wielościanów archimedesowych są wielokątami foremnymi, ale mogą występować w dwóch lub trzech rodzajach.
Układ ścian we wszystkich wierzchołkach jest taki sam.
Istnieje 13 unikatowych wielościanów archimedesowych:
\begin{enumerate}
\item czworościan ścięty (truncated tetrahedron)
\item sześcian ścięty (truncated cube)
\item sześcio-ośmiościan (cuboctahedron)
\item ośmiościan ścięty (truncated octahedron)
\item dwunastościan ścięty (truncated dodecahedron)
\item dwudziesto-dwunastościan (icosidodecahedron)
\item dwudziestościan ścięty (truncated icosahedron)
\item sześcio-ośmiościan rombowy mały (rhombicuboctahedron)
\item sześcio-ośmiościan rombowy wielki (rhombitruncated cuboctahedron)
\item sześcio-ośmiościan przycięty (snub cuboctahedron)
\item dwudziesto-dwunastościan rombowy mały (rhombicosidodecahedron)
\item dwudziesto-dwunastościan rombowy wielki (rhombitruncated icosidodecahedron)
\item dwudziesto-dwunastościan przycięty (snub icosidodecahedron).
\end{enumerate}
Niektórzy zaliczają do tej kategorii jeszcze dwie nieskończone serie wypukłych wielościanów półforemnych:
\begin{itemize}
\item płóforemne graniastosłupy (o ścianach bocznych będących kwadratami)
\item półforemne antygraniastosłupy (o ścianach bocznych będących trójkątami równobocznymi).
\end{itemize} 

Niektórzy (wiele osób) mieszają wielościany archimedesowe z półforemnymi.

\begin{definition}[wielościan półforemny]
    Trzynaście wielościanów archimedesowych i dwie nieskończone serie: graniastosłupów oraz antygraniastosłupów, a czasami także $J_{37}$ (np. zdaniem Branko Grünbauma) nazywamy wielościanami półforemnymi.
\end{definition}

Pierwsza definicja mówiła, że wielościan ma mieć foremne ściany, a jego grupa symetrii ma działać tranzytywnie na wierzchołkach; później takie obiekty nazwie się jednorodnymi.
Wielościan $J_{37}$ ma identyczne naroża, ale przez skręcenie nie ma tranzytywności (i zapewne przez to pojawił się w druku dopiero w 1905 roku, a potem został przez przypadek odkryty na nowo: przez Jeffreya Charlesa Percy'ego Millera w 1930 i niezależnie przez Ashkinuze)
% TODO: Ashkinuse, V.G.: On the number of semiregular polyhedra. -> rosyjski mathscinet

% http://matematyka.wroc.pl/book/historia-1
% http://matematyka.wroc.pl/book/wielościany-archimedesowe-platońskie

\subsection{Wielościany Catalana}
\begin{itemize}
\item Catalan solids: Rhombic dodecahedron 
\item Tetrakis hexahedron 
\item Triakis tetrahedron 
\item Catalan solid 
\item Triakis icosahedron 
\item Deltoidal icositetrahedron 
\item Disdyakis triacontahedron 
\item Deltoidal hexecontahedron 
\item Disdyakis dodecahedron 
\item Rhombic triacontahedron 
\item Pentagonal hexecontahedron 
\item Pentagonal icositetrahedron 
\item Pentakis dodecahedron 
\item Triakis octahedron 
\end{itemize}

\subsection{Wielościany Johnsona}
%

\begin{definition}
    Wielościan wypukły, który nie jest jednolity, ale wszystkie jego ściany są foremne, nazywamy wielościanem Johnsona albo Johnsona-Zalgallera.
\end{definition}

Norman Woodason Johnson \cite{johnson_1966} opublikuje w pracy doktorskiej z 1966 roku, a spisanej pod opieką samego Coxetera, listę 92 brył, które spełniają ten warunek, zaś Wiktor Abramowicz Zalgaller \cite{zalgaller_1969} udowodni po trzech latach (1969), że lista ta jest kompletna.
Często oznacza się je po prostu $J_{\ldots}$, gdzie indeks dolny mówi, którą pozycję na liście zajmuje.

\begin{itemize}
    \item piramida (czworokątna: $J_1$, pięciokątna: $J_2$),
    \item kopuła (trójkątna: $J_3$, czworokątna: $J_4$, pięciokątna: $J_5$),
    \item \emph{(equilateral square: $J_1$, pentagonal: $J_2$) pyramid},
    \item \emph{(triangular: $J_3$, square: $J_4$, pentagonal: $J_5$) cupola},
    \item [$J_{6}$] {rotunda pięciokątna}
                     (\emph{pentagonal rotunda})
    \item [$J_{7}$] {wydłużona piramida trójkątna}
                     (\emph{elongated triangular pyramid})
    \item [$J_{8}$] {wydłużona piramida czworokątna}
                     (\emph{elongated square pyramid})
    \item [$J_{9}$] {wydłużona piramida pięciokątna}
                     (\emph{elongated pentagonal pyramid})
    \item [$J_{10}$] {skrętnie wydłużona piramida czworokątna}
                     (\emph{gyroelongated square pyramid})
    \item [$J_{11}$] {skrętnie wydłużona piramida pięciokątna}
                     (\emph{gyroelongated pentagonal pyramid})
    \item [$J_{12}$] {dwupiramida trójkątna}
                     (\emph{triangular bipyramid})
    \item [$J_{13}$] {dwupiramida pięciokątna}
                     (\emph{pentagonal bipyramid})
    \item [$J_{14}$] {wydłużona dwupiramida trójkątna}
                     (\emph{elongated triangular bipyramid})
    \item [$J_{15}$] {wydłużona dwupiramida czworokątna}
                     (\emph{elongated square bipyramid})
    \item [$J_{16}$] {wydłużona dwupiramida pięciokątna}
                     (\emph{elongated pentagonal bipyramid})
    \item [$J_{17}$] {skrętnie wydłużona dwupiramida czworokątna}
                     (\emph{gyroelongated square bipyramid})
    \item [$J_{18}$] {wydłużona kopuła trójkątna}
                     (\emph{elongated triangular cupola})
    \item [$J_{19}$] {wydłużona kopuła czworokątna}
                     (\emph{elongated square cupola})
    \item [$J_{20}$] {wydłużona kopuła pięciokątna}
                     (\emph{elongated pentagonal cupola})
    \item [$J_{21}$] {wydłużona rotunda pięciokątna}
                     (\emph{elongated pentagonal rotunda})
    \item [$J_{22}$] {skrętnie wydłużona kopuła trójkątna}
                     (\emph{gyroelongated triangular cupola})
    \item [$J_{23}$] {skrętnie wydłużona kopuła czworokątna}
                     (\emph{gyroelongated square cupola})
    \item [$J_{24}$] {skrętnie wydłużona kopuła pięciokątna}
                     (\emph{gyroelongated pentagonal cupola})
    \item [$J_{25}$] {skrętnie wydłużona rotunda pięciokątna}
                     (\emph{gyroelongated pentagonal rotunda})
    \item [$J_{26}$] {podwójny graniastosłup trójkątny}
                     (\emph{gyrobifastigium})
    \item [$J_{27}$] {podwójna kopuła trójkątna}
                     (\emph{triangular orthobicupola})
    \item [$J_{28}$] {podwójna kopuła czworokątna}
                     (\emph{square orthobicupola})
    \item [$J_{29}$] {podwójna kopuła czworokątna skręcona}
                     (\emph{square gyrobicupola})
    \item [$J_{30}$] {podwójna kopuła pięciokątna}
                     (\emph{pentagonal orthobicupola})
    \item [$J_{31}$] {dwukopuła pięciokątna skręcona}
                     (\emph{pentagonal gyrobicupola})
    \item [$J_{32}$] {kopuło-rotunda pięciokątna}
                     (\emph{pentagonal orthocupolarotunda})
    \item [$J_{33}$] {kopuło-rotunda pięciokątna skręcona}
                     (\emph{pentagonal gyrocupolarotunda})
    \item [$J_{34}$] {dwurotunda pięciokątna}
                     (\emph{pentagonal orthobirotunda})
    \item wydłużona dwukopuła (trójkątna: $J_{35}$, trójkątna skręcona: $J_{36}$, czworokątna skręcona: $J_{37}$, pięciokątna: $J_{38}$, pięciokątna skręcona: $J_{36}$),
    \item [$J_{35}$] {wydłużona dwukopuła trójkątna}
                     (\emph{elongated triangular orthobicupola})
    \item [$J_{36}$] {wydłużona dwukopuła trójkątna skręcona}
                     (\emph{elongated triangular gyrobicupola})
    \item [$J_{37}$] {wydłużona dwukopuła czworokątna skręcona}
                     (\emph{elongated square gyrobicupola})
    \item [$J_{38}$] {wydłużona dwukopuła pięciokątna}
                     (\emph{elongated pentagonal orthobicupola})
    \item [$J_{39}$] {wydłużona dwukopuła pięciokątna skręcona}
                     (\emph{elongated pentagonal gyrobicupola})
    \item [$J_{40}$] {wydłużona kopuło-rotunda pięciokątna}
                     (\emph{elongated pentagonal orthocupolarotunda})
    \item [$J_{41}$] {wydłużona kopułorotunda pięciokątna skręcona}
                     (\emph{elongated pentagonal gyrocupolarotunda})
    \item [$J_{42}$] {wydłużona dwurotunda pięciokątna}
                     (\emph{elongated pentagonal orthobirotunda})
    \item [$J_{43}$] {wydłużona dwurotunda pięciokątna skręcona}
                     (\emph{elongated pentagonal gyrobirotunda})
    \item [$J_{44}$] {skrętnie wydłużona dwukopuła trójkątna}
                     (\emph{gyroelongated triangular bicupola})
    \item [$J_{45}$] {skrętnie wydłużona dwukopuła czworokątna}
                     (\emph{gyroelongated square bicupola})
    \item [$J_{46}$] {skrętnie wydłużona dwukopuła pięciokątna}
                     (\emph{gyroelongated pentagonal bicupola})
    \item [$J_{47}$] {skrętnie wydłużona kopuło-rotunda pięciokątna}
                     (\emph{gyroelongated pentagonal cupolarotunda})
    \item [$J_{48}$] {skrętnie wydłużona dwurotunda pięciokątna}
                     (\emph{gyroelongated pentagonal birotunda})
    \item [$J_{49}$] {powiększony graniastosłup trójkątny}
                     (\emph{augmented triangular prism})
    \item [$J_{50}$] {podwójnie powiększony graniastosłup trójkątny}
                     (\emph{biaugmented triangular prism})
    \item [$J_{51}$] {potrójnie powiększony graniastosłup trójkątny}
                     (\emph{triaugmented triangular prism})
    \item [$J_{52}$] {powiększony graniastosłup pięciokątny}
                     (\emph{augmented pentagonal prism})
    \item [$J_{53}$] {podwójnie powiększony graniastosłup pieciokątny}
                     (\emph{biaugmented pentagonal prism})
    \item [$J_{54}$] {powiększony graniastosłup sześciokątny}
                     (\emph{augmented hexagonal prism})
    \item [$J_{55}$] {podwójnie osiowo powiększony graniastosłup sześciokątny}
                     (\emph{parabiaugmented hexagonal prism})
    \item [$J_{56}$] {podwójnie powiększony graniastosłup sześciokątny}
                     (\emph{metabiaugmented hexagonal prism})
    \item [$J_{57}$] {potrójnie powiększony graniastosłup sześciokątny}
                     (\emph{triaugmented hexagonal prism})
    \item [$J_{58}$] {powiększony dwunastościan}
                     (\emph{augmented dodecahedron})
    \item [$J_{59}$] {podwójnie osiowo powiększony dwunastościan}
                     (\emph{parabiaugmented dodecahedron})
    \item [$J_{60}$] {podwójnie powiększony dwunastościan}
                     (\emph{metabiaugmented dodecahedron})
    \item [$J_{61}$] {potrójnie powiększony dwunastościan}
                     (\emph{triaugmented dodecahedron})
    \item [$J_{62}$] {podwójnie obcięty dwudziestościan}
                     (\emph{metabidiminished icosahedron})
    \item [$J_{63}$] {potrójnie obcięty dwudziestościan}
                     (\emph{tridiminished icosahedron})
    \item [$J_{64}$] {powiększony potrójnie obcięty dwudziestościan}
                     (\emph{augmented tridiminished icosahedron})
    \item [$J_{65}$] {powiększony czworościan ścięty}
                     (\emph{augmented truncated tetrahedron})
    \item [$J_{66}$] {powiększony sześcian ścięty}
                     (\emph{augmented truncated cube})
    \item [$J_{67}$] {podwójnie powiększony sześcian ścięty}
                     (\emph{biaugmented truncated cube})
    \item [$J_{68}$] {powiększony dwunastościan ścięty}
                     (\emph{augmented truncated dodecahedron})
    \item [$J_{69}$] {podwójnie osiowo powiększony dwunastościan ścięty}
                     (\emph{parabiaugmented truncated dodecahedron})
    \item [$J_{70}$] {podwójnie powiększony dwunastościan ścięty}
                     (\emph{metabiaugmented truncated dodecahedron})
    \item [$J_{71}$] {potrójnie powiększony dwunastościan ścięty}
                     (\emph{triaugmented truncated dodecahedron})
    \item [$J_{72}$] {dwudziesto-dwunastościan rombowy skręcony}
                     (\emph{gyrate rhombicosidodecahedron})
    \item [$J_{73}$] {dwudziesto-dwunastościan rombowy podwójnie osiowo skręcony}
                     (\emph{parabigyrate rhombicosidodecahedron})
    \item [$J_{74}$] {dwudziesto-dwunastościan rombowy podwójnie skośnie skręcony}
                     (\emph{metabigyrate rhombicosidodecahedron})
    \item [$J_{75}$] {dwudziesto-dwunastościan rombowy potrojnie skręcony}
                     (\emph{trigyrate rhombicosidodecahedron})
    \item [$J_{76}$] {dwudziesto-dwunastościan rombowy obcięty}
                     (\emph{diminished rhombicosidodecahedron})
    \item [$J_{77}$] {przekręcony osiowo dwudziesto-dwunastościan rombowy obcięty}
                     (\emph{paragyrate diminished rhombicosidodecahedron})
    \item [$J_{78}$] {przekręcony skośnie dwudziesto-dwunastościan rombowy obcięty}
                     (\emph{metagyrate diminished rhombicosidodecahedron})
    \item [$J_{79}$] {podwójnie przekręcony dwudziesto-dwunastościan rombowy obcięty}
                     (\emph{bigyrate diminished rhombicosidodecahedron})
    \item [$J_{80}$] {dwudziesto-dwunastościan rombowy podwójnie osiowo obcięty}
                     (\emph{parabidiminished rhombicosidodecahedron})
    \item [$J_{81}$] {dwudziesto-dwunastościan rombowy podwójnie skośnie obcięty}
                     (\emph{metabidiminished rhombicosidodecahedron})
    \item [$J_{82}$] {przekręcony dwudziesto-dwunastościan rombowy podwójnie obcięty}
                     (\emph{gyrate bidiminished rhombicosidodecahedron})
    \item [$J_{83}$] {dwudziesto-dwunastościan rombowy potrójnie obcięty}
                     (\emph{tridiminished rhombicosidodecahedron})
    \item [$J_{84}$] {dwuklinoid przycięty}
                     (\emph{snub disphenoid})
    \item [$J_{85}$] {antygraniastosłup czworokątny przycięty}
                     (\emph{snub square antiprism})
    \item [$J_{86}$] {klinokorona}
                     (\emph{sphenocorona})
    \item [$J_{87}$] {powiększona klinokorona}
                     (\emph{augmented sphenocorona})
    \item [$J_{88}$] {klinomega- korona}
                     (\emph{sphenomegacorona})
    \item [$J_{89}$] {hebeklinomegakorona}
                     (\emph{hebesphenomegacorona})
    \item [$J_{90}$] {klinocingulum podwójne}
                     (\emph{disphenocingulum})
    \item [$J_{91}$] {dwusoczewkowa rotunda podwójna}
                     (\emph{bilunabirotunda})
    \item [$J_{92}$] {hebeklinorotunda trójkątna}
                     (\emph{triangular hebesphenorotunda})
\end{itemize}

% https://en.wikipedia.org/wiki/Johnson_solid
% http://matematyka.wroc.pl/book/wielosciany-johnsona
% {Siamese dodecahedron} https://en.wikipedia.org/wiki/Snub_disphenoid

%

\subsection{Wielościany jednorodne}
Zostawione na koniec, bo ich uzytecznoscj est ograniczona.

Uniform maja foremne sciany oraz vertex-transitive. Mamy dwie nieskonczone serie i 75 konkretnych modeli.

Dziela sie na:

- 9 regular (face, edge, to juz bylo). 5 platonskich, 4 keplera-poinsot. Zostaje 66

- quasi-regular (tylko edge-)

- 2 archimedesowe: There are only two convex quasiregular polyhedra: the cuboctahedron and the icosidodecahedron.

- 14 kwaziregularnych 'uniform star polyhedra'

- semi-regular (ani to, ani to).

- 11 archimedesowych: te, ktorych nie wymienilismy wczesniej XD

- 39 semiregular uniform star polyhedra

Lista usp jest tu: % https://en.wikipedia.org/wiki/Uniform_star_polyhedron

(Wygflada na to, ze Nie mozna byc tylko face-, ale dowodu brak).

Trzeba je wysumowac do 75.

5 platon, 13 archimedes. Zostaje 57.

he non-convex star polyhedra as in 4 Kepler-Poinsothe non-convex star polyhedra

and 53 % [uniform star polyhedra](https://en.wikipedia.org/wiki/Uniform_star_polyhedra "Uniform star polyhedra")—14 [quasiregular](https://en.wikipedia.org/wiki/Quasiregular_polyhedron#Nonconvex_examples "Quasiregular polyhedron") and 39 semiregular

There are two infinite classes of uniform polyhedra, together with % 75 other polyhedra. They are 2 infinite classes of [prisms](https://en.wikipedia.org/wiki/Prism_(geometry)) "Prism (geometry)") and [antiprisms](https://en.wikipedia.org/wiki/Antiprism "Antiprism"), the convex polyhedrons as in 5 [Platonic solids](https://en.wikipedia.org/wiki/Platonic_solid "Platonic solid") and 13 [Archimedean solids](https://en.wikipedia.org/wiki/Archimedean_solid "Archimedean solid")—2 [quasiregular](https://en.wikipedia.org/wiki/Quasiregular_polyhedron "Quasiregular polyhedron") and 11 [semiregular](https://en.wikipedia.org/wiki/Semiregular_polyhedron "Semiregular polyhedron")— the non-convex star polyhedra as in 4 [Kepler–Poinsot polyhedra](https://en.wikipedia.org/wiki/Kepler%E2%80%93Poinsot_polyhedra "Kepler–Poinsot polyhedra") and 53 [uniform star polyhedra](https://en.wikipedia.org/wiki/Uniform_star_polyhedra "Uniform star polyhedra")—14 [quasiregular](https://en.wikipedia.org/wiki/Quasiregular_polyhedron#Nonconvex_examples "Quasiregular polyhedron")

and 39 semiregular. There are also many degenerate uniform polyhedra

with pairs of edges that coincide, including one found by John Skilling

called the [great disnub dirhombidodecahedron] % (https://en.wikipedia.org/wiki/Great_disnub_dirhombidodecahedron "Great disnub dirhombidodecahedron"), Skilling's figure.^[**[**1**]**](https://en.wikipedia.org/wiki/Uniform_polyhedron#cite_note-FOOTNOTEDiudea2018https://books.google.com/books?id=p_06DwAAQBAJ&pg=PA40_40]-1)^

[Coxeter, Longuet-Higgins  % &amp; Miller (1954)](https://en.wikipedia.org/wiki/Uniform_polyhedron#CITEREFCoxeterLonguet-HigginsMiller1954) define uniform polyhedra to be [vertex-transitive](https://en.wikipedia.org/wiki/Vertex-transitive "Vertex-transitive")

polyhedra with regular faces.


\subsection{Wielościany Goldberga}
\todofoot{Goldberg polyhedron} % https://en.wikipedia.org/wiki/Goldberg_polyhedron


\subsection{Wielościany Szilassiego}
O wielościanach Szilassiego pisze $\Delta_{24}^4$.

\section{Historia}



KratkaKratkaKratka historia

KratkaKratkaKratkaKratka Regular convex polyhedra

* The[Platonic solids](https://en.wikipedia.org/wiki/Platonic%_solid "Platonic solid") date back to the classical Greeks and were studied by the[Pythagoreans](https://en.wikipedia.org/wiki/Pythagoreanism "Pythagoreanism"),[Plato](https://en.wikipedia.org/wiki/Plato "Plato") (c. 424 – 348 BC),[Theaetetus](https://en.wikipedia.org/wiki/Theaetetus%_(mathematician)) "Theaetetus (mathematician)") (c. 417 BC – 369 BC),[Timaeus of Locri](https://en.wikipedia.org/wiki/Timaeus%_of%_Locri "Timaeus of Locri") (c. 420–380 BC), and[Euclid](https://en.wikipedia.org/wiki/Euclid "Euclid") (fl. 300 BC). The[Etruscans](https://en.wikipedia.org/wiki/Etruscans "Etruscans") discovered the regular dodecahedron before 500 BC.^[**[**3**]**](https://en.wikipedia.org/wiki/Uniform%_polyhedronKratkacite%_note-3)^

KratkaKratkaKratkaKratka Nonregular uniform convex polyhedra

* The[cuboctahedron](https://en.wikipedia.org/wiki/Cuboctahedron "Cuboctahedron") was known by[Plato](https://en.wikipedia.org/wiki/Plato "Plato").

* [Archimedes](https://en.wikipedia.org/wiki/Archimedes "Archimedes") (287 BC – 212 BC) discovered all of the 13[Archimedean solids](https://en.wikipedia.org/wiki/Archimedean%_solid "Archimedean solid"). His original book on the subject was lost, but[Pappus of Alexandria](https://en.wikipedia.org/wiki/Pappus%_of%_Alexandria "Pappus of Alexandria") (c. 290 – c. 350 AD) mentioned Archimedes listed 13 polyhedra.

* [Piero della Francesca](https://en.wikipedia.org/wiki/Piero%_della%_Francesca "Piero della Francesca")

(1415 – 1492) rediscovered the five truncations of the Platonic

solids—truncated tetrahedron, truncated octahedron, truncated cube,

truncated dodecahedron, and truncated icosahedron—and included

illustrations and calculations of their metric properties in his book*[De quinque corporibus regularibus](https://en.wikipedia.org/wiki/De%_quinque%_corporibus%_regularibus "De quinque corporibus regularibus")* . He also discussed the cuboctahedron in a different book.^[**[**4**]**](https://en.wikipedia.org/wiki/Uniform%_polyhedronKratkacite%_note-4)^

* [Luca Pacioli](https://en.wikipedia.org/wiki/Luca%_Pacioli "Luca Pacioli") plagiarized Francesca's work in*[De divina proportione](https://en.wikipedia.org/wiki/De%_divina%_proportione "De divina proportione")* in 1509, adding the[rhombicuboctahedron](https://en.wikipedia.org/wiki/Rhombicuboctahedron "Rhombicuboctahedron"), calling it an*icosihexahedron* for its 26 faces, which was drawn by[Leonardo da Vinci](https://en.wikipedia.org/wiki/Leonardo%_da%_Vinci "Leonardo da Vinci").

* [Johannes Kepler](https://en.wikipedia.org/wiki/Johannes%_Kepler "Johannes Kepler") (1571–1630) was the first to publish the complete list of[Archimedean solids](https://en.wikipedia.org/wiki/Archimedean%_solid "Archimedean solid"), in 1619. He also identified the infinite families of uniform[prisms and antiprisms](https://en.wikipedia.org/wiki/Prismatic%_uniform%_polyhedron "Prismatic uniform polyhedron").

KratkaKratkaKratka Other 53 nonregular star polyhedra

* Of the remaining 53,[Edmund Hess](https://en.wikipedia.org/wiki/Edmund%_Hess "Edmund Hess")

(1878) discovered 2, Albert Badoureau (1881) discovered 36 more, and

Pitsch (1881) independently discovered 18, of which 3 had not previously

been discovered. Together these gave 41 polyhedra.

* The geometer[H.S.M. Coxeter](https://en.wikipedia.org/wiki/Harold%_Scott%_MacDonald%_Coxeter "Harold Scott MacDonald Coxeter") discovered the remaining twelve in collaboration with[J. C. P. Miller](https://en.wikipedia.org/wiki/J.%_C.%_P.%_Miller "J. C. P. Miller") (1930–1932) but did not publish.[M.S. Longuet-Higgins](https://en.wikipedia.org/wiki/Michael%_S.%_Longuet-Higgins "Michael S. Longuet-Higgins") and[H.C. Longuet-Higgins](https://en.wikipedia.org/wiki/H.C.%_Longuet-Higgins "H.C. Longuet-Higgins") independently discovered eleven of these. Lesavre and Mercier rediscovered five of them in 1947.

* [Coxeter, Longuet-Higgins % &amp; Miller (1954)](https://en.wikipedia.org/wiki/Uniform%_polyhedronKratkaCITEREFCoxeterLonguet-HigginsMiller1954) published the list of uniform polyhedra.

* [Sopov (1970)](https://en.wikipedia.org/wiki/Uniform%_polyhedronKratkaCITEREFSopov1970) proved their conjecture that the list was complete.

* In 1974,[Magnus Wenninger](https://en.wikipedia.org/wiki/Magnus%_Wenninger "Magnus Wenninger") published his book[*Polyhedron models*](https://en.wikipedia.org/wiki/List%_of%_Wenninger%_polyhedron%_models "List of Wenninger polyhedron models"), which lists all 75 nonprismatic uniform polyhedra, with many previously unpublished names given to them by[Norman Johnson](https://en.wikipedia.org/wiki/Norman%_Johnson%_(mathematician)) "Norman Johnson (mathematician)").

* [Skilling (1975)](https://en.wikipedia.org/wiki/Uniform%_polyhedronKratkaCITEREFSkilling1975)

independently proved the completeness and showed that if the definition

of uniform polyhedron is relaxed to allow edges to coincide then there

is just one extra possibility (the[great disnub dirhombidodecahedron](https://en.wikipedia.org/wiki/Great%_disnub%_dirhombidodecahedron "Great disnub dirhombidodecahedron")).

* In 1987,[Edmond Bonan](https://en.wikipedia.org/wiki/Edmond%_Bonan "Edmond Bonan") drew all the uniform polyhedra and their duals in 3D with a Turbo Pascal program called**Polyca** .

Most of them were shown during the International Stereoscopic Union

Congress held in 1993, at the Congress Theatre, Eastbourne, England; and

again in 2005 at the Kursaal of Besançon, % France.^[**[**5**]**](https://en.wikipedia.org/wiki/Uniform%_polyhedronKratkacite%_note-5)^

* In 1993, Zvi Har'El % (1949–2008)^[**[**6**]**](https://en.wikipedia.org/wiki/Uniform%_polyhedronKratkacite%_note-6)^ produced a complete kaleidoscopic construction of the uniform polyhedra and duals with a computer program called**Kaleido** and summarized it in a paper*Uniform Solution for Uniform Polyhedra* , counting figures 1-80.^[**[**7**]**](https://en.wikipedia.org/wiki/Uniform%_polyhedronKratkacite%_note-7)^

* Also in 1993, R. Mäder ported this Kaleido solution to[Mathematica](https://en.wikipedia.org/wiki/Mathematica) with a slightly different % indexing system.^[**[**8**]**](https://en.wikipedia.org/wiki/Uniform%_polyhedronKratkacite%_note-8)^

* In 2002 Peter W. Messer discovered a minimal set of closed-form

expressions for determining the main combinatorial and metrical

quantities of any uniform polyhedron (and its dual) given only its[Wythoff symbol](https://en.wikipedia.org/wiki/Wythoff%_symbol "Wythoff symbol").^[**[**9**]**](https://en.wikipedia.org/wiki/Uniform%_polyhedronKratkacite%_note-9)^

KratkaKratkaKratka archimedesowe -> semiregular

Kepler coined kategorię wielościanów półforemnych w *Harmonice Mundi* (1619) jako 13 wielościanów archimedesowych, graniastosłupy, antygraniastosłupy, i dwa wielościany Catalana, krawędziowo tranzytywne **rhombic dodecahedron + rhombic triacontahedron** i niejawnie **trigonal trapezohedron**.

Thorold Gosset zdefiniuje (*Thorold Gosset On the Regular and Semi-Regular Figures in Space of n Dimensions, Messenger of Mathematics, Macmillan, 1900*) ogólniejsze pojęcie wielotopu półforemnego, z którego wynika, że

Coxeter et al. 1954 używali terminu "wielościany półforemne" do sklasyfikowania jednorodnych wielościanów, których symbol Wythoffa to "p q | r", co obejmuje tylko sześć wielościanów archimedesowych, graniastosłupy, ale nie antygraniastosłupy, ale za to liczne niewypukłe bryły.

W 1973 zmieni zdanie i przyjmie definicje Gosseta bez komentarza.

Cromwell 1997 napisze, że wielościany półforemne to archimedesowe, Catalana, chociaz w dalszej czesci (tej samej książki???) uzna wielościany Catalana, że nie są półforemne.

https://en.wikipedia.org/wiki/Semiregular%_polyhedron

https://en.wikipedia.org/wiki/Quasiregular%_polyhedron => raczej krtkie

frustrum (frusta) to część bryły wycięta przez dwie płaszczyzny równoległ


% https://en.wikipedia.org/wiki/Polyhedron
% https://en.wikipedia.org/wiki/Dual_polyhedron
% https://www.youtube.com/watch?v=yAEveAOH2KwI => Schwarz lantern

% TODO: https://en.wikipedia.org/wiki/Wallace-Bolyai-Gerwien_theorem to nie działa w 3d a jest ok w 2d
\subsection{Nie ma odpowiednika}
Nie ma odpowiednika twierdzenia Wallace'a-Bolyaia-Gerwiena w trzech wymiarach.
Dwa wielościany nazwijmy równoważnymi przez podział, jeśli pierwszy da się tak podzielić na skończoną liczbę wielościanów, by z otrzymanych kawałków można było złożyć drugi.
To, czy każde dwa czworościany o równych podstawach i równych wysokościach będzie stanowić treść trzeciego problemu Hilberta.
Negatywnej odpowiedzi udzieli Dehn w 1900 roku, prostszy dowód pokaże Hadwiger w 1954 roku.
% Hadwiger, Glur pokazali, że kiedy boki kawałków mogą musieć być równoległe % Hadwiger, H.; Glur, P.: Zerlegungsgleichheit ebener Polygone. Elem. Math. 6 (1951), 97–106.

\begin{theorem}[Hadwigera] % Delta 1984 11
    Niech $\alpha_1$, $\ldots$, $\alpha_p$ będą kątami dwuściennymi wielościanu $W$ leżącymi wzdłuż krawędzi długości $k_1$, $\ldots$, $k_p$, a $\beta_1$, $\ldots$, $\beta_q$ kątami wielościanu $V$ (wzdłuż krawędzi długości $l_1$, $\ldots$, $l_q$).
    Definiujemy, z lekkim nadużyciem notacji,
    \begin{equation}
        W := \sum_{i \le p} k_i \alpha_i
    \end{equation}
    i analogicznie liczbę $V$.
    Jeśli istnieje taka funkcja $\mathbb Z$-addytywna na zbiorze $M \subseteq \mathbb R$ zawierającym $\pi$, $\alpha_1$, $\ldots$, $\alpha_p$, $\beta_1$, $\ldots$, $\beta_q$, że $f(W) \neq f(V)$, to $W$ i $V$ nie są równoważne przez podział.
\end{theorem}

\begin{corollary}
    Czworościan o kątach dwuściennych
    \begin{equation}
        \frac \pi 2, \frac \pi 2, \frac \pi 2,
        \arccos \frac{1}{\sqrt 3}, \arccos \frac{1}{\sqrt 3}, \arccos \frac{1}{\sqrt 3}
    \end{equation}
    nie jest równoważny przez podział z żadnym sześcianem.
    Dla dowodu można wziąć $M = \{\pi/2, \pi, \alpha\}$ i określić $f(\pi/2) = f(\pi) = 0$, $f(\alpha) = 1$.
    Wtedy funkcja $f$ przyjmuje na czworościanie wartość $3 \ \sqrt 2$, zaś na sześcianie $0$.
\end{corollary}

Taki sam wniosek wyciągnie nieznany autor w $\Delta_{84}^{11}$.

% Unknown to Hilbert and Dehn, Hilbert's third problem was also proposed independently by Władysław Kretkowski for a math contest of 1882 by the Academy of Arts and Sciences of Kraków, and was solved by Ludwik Antoni Birkenmajer with a different method than Dehn's. Birkenmajer did not publish the result, and the original manuscript containing his solution was rediscovered years later.[3] 
% Gauss regretted this defect in two of his letters to Christian Ludwig Gerling, who proved that two symmetric tetrahedra are equidecomposable.
% Two polyhedra are called scissors-congruent if the first can be cut into finitely many polyhedral pieces that can be reassembled to yield the second. Any two scissors-congruent polyhedra have the same volume. Hilbert asks about the converse. 
% In light of Dehn's theorem above, one might ask "which polyhedra are scissors-congruent"? Sydler (1965) showed that two polyhedra are scissors-congruent if and only if they have the same volume and the same Dehn invariant.[5] Børge Jessen later extended Sydler's results to four dimensions.[6] In 1990, Dupont and Sah provided a simpler proof of Sydler's result by reinterpreting it as a theorem about the homology of certain classical groups.[7] 
% http://sciencecow.mit.edu/me/hilberts_third_problem.pdf

\subsection{Zliczanie małych wielościanów}
\subsubsection{Deltościany}
Jest osiem wielościanów wypukłych, których ściany są trójkątami równobocznymi: trzy platońskie bryły (czworościan, ośmiościan i dwudziestościan foremny) oraz pięć wielościanów Johnsona ($J_{12}$, $J_{13}$, $J_{17}$, $J_{51}$, $J_{84}$).
Pokażą to w 1947 roku Hans Freudenthal i Bartel Leendert van der Waerden \cite{Freudenthal_1947} w obskurnym duńskim żurnalu.
% Freudenthal, H.; van der Waerden, B. L. (1947), "On an assertion of Euclid", Simon Stevin, 25: 115–121, MR 0021687
% https://mathscinet.ams.org/mathscinet/relay-station?mr=0021687
Ich wynik zreferuje później Adam Gajda w $\Delta_{84}^{4}$.

Czasami nazywa się je deltościanami, przez podobieństwo litery $\Delta$ do ich ścian.
Niewypukłych brył o tej własności jest nieskończnie wiele (!), mogą mieć dowolną parzystą liczbę ścian większą niż sześć.

% https://en.wikipedia.org/wiki/Deltahedron  % http://matematyka.wroc.pl/book/deltosciany

\subsubsection{Sześciościany}
Jest siedem sześciościanów wypukłych (i trzy niewypukłe).
Heinz Schumann oraz Bronisław Pabich napiszą krótki artykuł w $\Delta_{24}^{11}$, gdzie uzasadnią te liczby diagramami Schlegela.
\index{diagram Schlegela}%
(Wielościany wypukłe mają 6/0/0, 5/0/1, 4/2/0, 3/2/1, 2/4/0, 2/2/2, 0/6/0 ścian o trzech, czterech, pięciu krawędziach).
% TODO: DELTA 2001 luty Kordos

\section{Kule, walce}
% https://en.wikipedia.org/wiki/On_the_Sphere_and_Cylinder pierwszy raz wzór objętość kuli