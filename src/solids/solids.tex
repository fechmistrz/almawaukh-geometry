
Tekst intro rozdziału stereometria.

\section{Wielościany}
% https://en.wikipedia.org/wiki/Polyhedron
% https://en.wikipedia.org/wiki/Dual_polyhedron
% https://www.youtube.com/watch?v=yAEveAOH2KwI => Schwarz lantern
% https://en.wikipedia.org/wiki/Wallace-Bolyai-Gerwien_theorem to nie działa w 3d a jest ok w 2d

Jest siedem sześciościanów wypukłych (i trzy niewypukłe).
Heinz Schumann oraz Bronisław Pabich napiszą krótki artykuł w $\Delta_{24}^{11}$, gdzie uzasadnią te liczby diagramami Schlegela.
\index{diagram Schlegela}%
(Wielościany wypukłe mają 6/0/0, 5/0/1, 4/2/0, 3/2/1, 2/4/0, 2/2/2, 0/6/0 ścian o trzech, czterech, pięciu krawędziach).
% TODO: DELTA 2001 luty Kordos

\subsection{Wielościany platońskie}
% https://en.wikipedia.org/wiki/Platonic_solid
% \section{Pięć wielościanów} Hartshorne: rozdział 8
% \section{Cauchy's rigidity theorem} Hartshorne: section 45
\subsubsection{Czworościan}
\subsubsection{Sześcian}
\subsubsection{Ośmiościan}
\subsubsection{Dwunastościan}
\subsubsection{Dwudziestościan}
% https://en.wikipedia.org/wiki/Regular_icosahedron
\todofoot{Platonic solid} 

\subsection{Wielościany archimedesowe}
\todofoot{1955 - H. S. M. Coxeter et al. publish the complete list of uniform polyhedron}

\subsection{Wielościany Keplera-Poinsota}
\todofoot{1806 - Louis Poinsot discovers the two remaining Kepler-Poinsot polyhedra.}
\todofoot{1619 - Johannes Kepler discovers two of the Kepler-Poinsot polyhedra}

% https://en.wikipedia.org/wiki/Great_icosahedron

Wielościany Keplera-Poinsota
Wielościany Keplera-Poinsota są odpowiednikami brył platońskich w świecie wielościanów niewypukłych. Również ich ściany są przystającymi wielokątami foremnymi i w każdym wierzchołku spotyka się taka sama liczba ścian. Tym razem jednak ściany mogą być wielokątami gwiaździstymi. Dopuszczona jest także możliwość przenikania ścian (tzn. przecinania się poza krawędziami).
Istnieją tylko 4 wielościany foremne niewypukłe:
dwunastościan gwiaździsty mały (small stellated dodecahedron),
dwunastościan wielki (great dodecahedron),
dwunastościan gwiaździsty wielki (great stellated dodecahedron),
dwudziestościan wielki (great icosahedron). 

\subsection{Wielościany Catalana}
%

Wielościany Catalana to bryły dualne do archimedesowych. Takie bryły po raz pierwszy zostały opisane w 1865 roku przez belgijskiego matematyka Eugene Catalana - stąd ich nazwa.
Okazuje się, że bryły dualne do wielościanów platońskich są platońskie, a dualne do Keplera-Poinsota są bryłami Keplera-Poinsota, więc nie dają żadnych nowych obiektów. Dają je dopiero wielościany archimedesowe.
Ponieważ dla danego wielościanu jego bryła dualna wyznaczona jest jednoznacznie, mamy 13 wielościanów Catalana, bo jest 13 wielościanów archimedesowych. Są to:
 
\begin{itemize}
\item czworościan potrójny (triakistetrahedron),
\item ośmiościan potrójny (triakisoctahedron),
\item dwudziestościan potrójny (triakisicosahedron),
\item sześciościan poczwórny (tetrakishexahedron),
\item dwunastościan piątkowy (pentakisdodecahedron),
\item ośmiościan szóstkowy (hexakisoctahedron),
\item dwudziestościan szóstkowy (hexakisicosahedron),
\item dwunastościan rombowy (rhombic dodecahedron),
\item trzydziestościan rombowy (rhombic triacontahedron),
\item sześćdziestościan deltoidowy (strombic hexecontahedron),
\item dwudziestoczterościan deltoidowy (strombic icositetrahedron),
\item dwudziestoczterościan pięciokątny (pentagonal icositetrahedron),
\item sześćdziestościan pieciokątny (pentagonal hexecontahedron).
\end{itemize}
% sześćdziestościan deltoidowy (strombic hexecontahedron), % Taką nazwę zaproponował prof. Roman Duda, tłumacząc ponad 40 lat temu książkę "Modele matematyczne" Cundy'ego i Rolleta i tak już zostało, choć rzeczywiście to trochę dziwnie brzmi.

%

\subsection{Wielościany Johnsona}
% https://en.wikipedia.org/wiki/Johnson_solid

% \section{Siamese dodecahedron} https://en.wikipedia.org/wiki/Snub_disphenoid
\begin{itemize}
\item List of Johnson solids 
\item Johnson solid 
\item Elongated square gyrobicupola 
\item Triaugmented triangular prism 
\item Square pyramid 
\item Pentagonal bipyramid 
\item Elongated square cupola 
\item Elongated pentagonal bipyramid 
\item Elongated triangular cupola 
\item Triangular bipyramid 
\item Gyroelongated square bipyramid 
\item Gyroelongated pentagonal pyramid 
\item Sphenomegacorona 
\item Snub disphenoid 
\item Pentagonal pyramid 
\item Gyroelongated pentagonal bicupola 
\item Gyroelongated square pyramid 
\item Cupola (geometry) 
\item Elongated triangular orthobicupola 
\item Gyroelongated triangular cupola 
\item Gyroelongated square cupola 
\item Elongated pentagonal rotunda 
\item Gyroelongated pentagonal cupola 
\item Gyroelongated pentagonal rotunda 
\item Elongated square bipyramid 
\item Square cupola 
\item Triangular cupola 
\item Bilunabirotunda 
\item Elongated square pyramid 
\item Elongated pentagonal pyramid 
\item Elongated pentagonal cupola 
\item Pentagonal cupola 
\item Pentagonal rotunda 
\item Augmented triangular prism 
\item Augmented truncated tetrahedron 
\item Biaugmented triangular prism 
\item Square orthobicupola 
\item Triangular hebesphenorotunda 
\item Gyrobifastigium 
\item Elongated triangular pyramid 
\item Elongated triangular bipyramid 
\item Augmented hexagonal prism 
\item Elongated triangular gyrobicupola 
\item Snub square antiprism 
\item Augmented pentagonal prism 
\item Biaugmented pentagonal prism 
\item Gyroelongated square bicupola 
\item Triaugmented truncated dodecahedron 
\item Hebesphenomegacorona 
\item Triangular orthobicupola 
\item Gyrate rhombicosidodecahedron 
\item Sphenocorona 
\item Disphenocingulum 
\item Metabidiminished rhombicosidodecahedron 
\item Pentagonal gyrobicupola 
\item Tridiminished icosahedron 
\item Tridiminished rhombicosidodecahedron 
\item Birotunda 
\item Pentagonal orthobirotunda 
\item Rotunda (geometry) 
\item Square gyrobicupola 
\item Augmented sphenocorona 
\item Elongated pentagonal gyrobirotunda 
\item Elongated pentagonal orthobicupola 
\item Gyroelongated triangular bicupola 
\item Metabidiminished icosahedron 
\item Gyroelongated pentagonal birotunda 
\item Gyroelongated pentagonal cupolarotunda 
\item Parabiaugmented dodecahedron 
\item Trigyrate rhombicosidodecahedron 
\item Elongated pentagonal gyrobicupola 
\item Parabidiminished rhombicosidodecahedron 
\item Elongated pentagonal orthobirotunda 
\item Augmented truncated cube 
\item Elongated pentagonal gyrocupolarotunda 
\item Pentagonal orthobicupola 
\item Elongated pentagonal orthocupolarotunda 
\item Metabigyrate rhombicosidodecahedron 
\item Bigyrate diminished rhombicosidodecahedron 
\item Metagyrate diminished rhombicosidodecahedron 
\item Gyrate bidiminished rhombicosidodecahedron 
\item Diminished rhombicosidodecahedron 
\item Paragyrate diminished rhombicosidodecahedron 
\item Parabigyrate rhombicosidodecahedron 
\item Parabiaugmented hexagonal prism 
\item Metabiaugmented hexagonal prism 
\item Metabiaugmented dodecahedron 
\item Pentagonal gyrocupolarotunda 
\item Triaugmented dodecahedron 
\item Triaugmented hexagonal prism 
\item Pentagonal orthocupolarotunda 
\item Parabiaugmented truncated dodecahedron 
\item Augmented truncated dodecahedron 
\item Biaugmented truncated cube 
\item Augmented dodecahedron 
\item Metabiaugmented truncated dodecahedron 
\item Augmented tridiminished icosahedron
\end{itemize}

\subsection{Wielościany Szilassiego}
O wielościanach Szilassiego pisze Delta (2024/kwiecień)

\todofoot{Goldberg polyhedron} % https://en.wikipedia.org/wiki/Goldberg_polyhedron

\subsection{Kule, walce}
% https://en.wikipedia.org/wiki/On_the_Sphere_and_Cylinder pierwszy raz wzór objętość kuli