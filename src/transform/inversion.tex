Inwersję odkryje wiele osób niezależnie od siebie, na przykład Jakob Steiner (1824), Lambert Adolphe Jacques Quetelet (1825), Giusto Bellavitis (1836), John William Stubbs i John Kells Ingram (1842?) albo lord Kelvin (właściwie William Thomson 1845).

\begin{definition}
    Dany jest okrąg $\Gamma$ o środku $O$ i promeniu $r$ oraz punkt $P$, różny od $O$.
    Wtedy jedyny punkt $P'$ leżący na półprostej $OP$ taki, że $|OP| \cdot |OP'| = r^2$ nazywamy inwersją punktu $P$ względem okręgu $\Gamma$.
\end{definition}

O inwersji piszą Coxeter \cite[s. 93-101, 107-112]{coxeter_1967} (który wspomni inwersor Peaucelliera\footnote{Po angielsku odpowiednio Peaucellier's cell, Harts' linkage.}, a właściwie Peaucelliera-Lipkina (!) na s.~98, przyrząd wynaleziony w 1864 roku do kreślenia obrazu inwersyjnego danej trajektorii, a chwilę poźniej urządzenie Harta służące do tego samego celu).

Czasami do płaszczyzny dokłada się $\infty$, punkt w nieskończoności: przyjmujemy, że inwersja odwzorowuje $\infty$ na punkt $O$, zaś punkt $O$ na $\infty$; czyni to ją automorfizmem płaszczyzny.

\begin{proposition}
    Inwersje są kątowierne: każdy kąt ma taką samą miarę jak jego obraz względem inwersji.
\end{proposition}

\begin{proposition}
    Inwersje przenoszą okręgi ortogonalne na okręgi ortogonalne, włączając proste jako przypadki szczególne.
\end{proposition}

Wykreślenie obrazu inwersyjnego punktu przy użyciu cyrkla i linijki nie jest trudne, chyba że uprzemy się, by nie rozpatrywać przypadków ,,punkt leży wewnątrz okręgu'' oraz ,,punkt leży na zewnątrz okręgu'' osobno.
Stosowny przepis poda Surajit Dutta około 2014 roku.
% Dutta, Surajit (2014) A simple property of isosceles triangles with applications Archived 2018-04-21 at the Wayback Machine, Forum Geometricorum 14: 237–240 
\index[persons]{Dutta, Surajit}%

Wiele trudnych problemów geometrycznych staje się prostsze po zastosowaniu inwersji.
Na przykład środek okręgu przed i po odwróceniu są współliniowe ze środkiem inwersji, skąd wynika, że prosta Eulera trójkąta między punktami styczności okręgów dopisanych przechodzi przez pewne ważne punkty.
% https://en.wikipedia.org/wiki/Inversive_geometry#Application

Proste prostopadłe do $PP'$, które przechodzą przez jeden z punktów, są biegunowymi drugiego końca odcinka (bieguna).
Bieguny i biegunowe mają swoje ciekawe własności.

% Coxeter s. 77: Magnus 1831 wymyślił ten termin
% tamże: Peaucellier's cell; Hart's linkage
% % https://en.wikipedia.org/wiki/Sacred_Mathematics

Istnieje odpowiednik klasyfikacji \ref{podobienstwa_klasyfikacja}:

\begin{proposition}
    Każde przekształcenie płaszczyzny inwersyjnej zachowujące okręgi jest albo podobieństwem, albo złożeniem inwersji i izometrii.
\end{proposition}

Stąd wynika, że każde takie przekształcenie jest złożeniem co najwyżej czterech inwersji.
Coxeter \cite[s. 108]{coxeter_1967} pisze o trzech rodzajach przekształceń: eliptycznych, parabolicznych i hiperbolicznych, co sprowadza się w szczególnym przypadku do obrotów, translacji i dylatacji.

\begin{proposition}
    Złożenie czterech inwersji, które nie jest złożeniem dwóch inwersji, nazywamy przekształceniem loksodromicznym.
\end{proposition}

Nazwa ma greckie korzenie: loksodroma to linia przecinająca wszystkie południki pod tym samym kątem (λοξός ukośny, δρόμος linia).

%