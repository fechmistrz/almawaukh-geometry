Inwersja.

O inwersji piszą Coxeter \cite[s. 93-101, 107-112]{coxeter_1967} (który wspomina inwersor Peaucelliera na s. 98, przyrząd wynaleziony w 1864 roku do kreślenia obrazu inwersyjnego danej trajektorii oraz urządzenie Harta służące do tego samego celu).

Inwersja zachowuje kąty, przenosi okręgi ortogonalne na okręgi ortogonalne, włączając proste jako przypadki szczególne.

Peaucellier's cell.
Harts' linkage.
% Coxeter s. 77: Magnus 1831 wymyślił ten termin
% tamże: Peaucellier's cell; Hart's linkage
% % https://en.wikipedia.org/wiki/Sacred_Mathematics


Istnieje odpowiednik klasyfikacji \ref{podobienstwa_klasyfikacja}:

\begin{proposition}
    Każde przekształcenie płaszczyzny inwersyjnej zachowujące okręgi jest albo podobieństwem, albo złożeniem inwersji i izometrii.
\end{proposition}

Stąd wynika, że każde takie przekształcenie jest złożeniem co najwyżej czterech inwersji.
Coxeter \cite[s. 108]{coxeter_1967} pisze o trzech rodzajach przekształceń: eliptycznych, parabolicznych i hiperbolicznych, co sprowadza się w szczególnym przypadku do obrotów, translacji i dylatacji.

\begin{proposition}
    Złożenie czterech inwersji, które nie jest złożeniem dwóch inwersji, nazywamy przekształceniem loksodromicznym.
\end{proposition}

Nazwa ma greckie korzenie: loksdroma to linia przecinająca wszystkie południki pod tym samym kątem (λοξός ukośny, δρόμος linia).

%