\todofoot{Izometrie Coxetera, s. 29-36, 47 -- Hjelmslev}

\begin{definition}
    Odwzorowanie $f \colon \Pi \to \Pi$ płaszczyzny w siebie takie, że dla dowolnych punktów $A, B \in \Pi$ mamy
    \begin{equation}
        |AB| = |f(A) f(B)|,
    \end{equation}
    nazywamy izometrią.
\end{definition}

Izometrie to odwzorowania, które zachowują odległości.
Izometrie można składać, każda jest bijekcją i funkcja odwrotna do niej też jest izometrią.

\begin{proposition}
\label{delta_1997_9_start}%
    Izometria, która ma punkt stały, przenosi wszystkie okręgi wokół tego punktu na siebie.
\end{proposition}

\begin{proposition}
    Izometria, która ma dwa punkty stałe, ma ich nieskończenie wiele (wszystkie punkty prostej przechodzącej przez dwa punkty).
\end{proposition}

\begin{proposition}
    Izometria, która ma trzy niewspółliniowe punkty stałe, jest odwzorowaniem tożsamościowym i~wszystkie punkty są jej punktami stałymi.
\end{proposition}

\begin{proposition}
\label{delta_1997_9_end}%
    Izometria, której wartości w trzech niewspółliniowych punktach są znane, jest jednoznacznie wyznaczona przez nie.
\end{proposition}

Cztery powyższe fakty to początek przepisu, jaki Marek Kordos poda w $\Delta_{97}^9$ na wynik związany z paradoksalnym rozkładem kuli.