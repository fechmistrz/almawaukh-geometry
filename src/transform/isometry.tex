\todofoot{Izometrie Coxetera, s. 29-36, 47 -- Hjelmslev}

\begin{definition}
    Odwzorowanie płaszczyzny w siebie, które zachowuje długość każdego odcinka, nazywamy izometrią płaszczyzny albo krócej izometrią.
\end{definition}

Izometrie to odwzorowania, które zachowują odległości.
Izometrie można składać, każda jest bijekcją i funkcja odwrotna do niej też jest izometrią.
Patrz do Coxetera \cite[s. 45-52, 56-63]{coxeter_1967}.

\begin{proposition}
\label{delta_1997_9_start}%
    Izometria, która ma punkt stały, przenosi wszystkie okręgi wokół tego punktu na siebie.
\end{proposition}

\begin{proposition}
    Izometria, która ma dwa punkty stałe, ma ich nieskończenie wiele (wszystkie punkty prostej przechodzącej przez dwa punkty).
\end{proposition}

\begin{proposition}
    Izometria, która ma trzy niewspółliniowe punkty stałe, jest odwzorowaniem tożsamościowym i~wszystkie punkty są jej punktami stałymi.
\end{proposition}

\begin{proposition}
\label{delta_1997_9_end}%
    Izometria, której wartości w trzech niewspółliniowych punktach są znane, jest jednoznacznie wyznaczona przez nie.
\end{proposition}

Cztery powyższe fakty to początek przepisu, jaki Marek Kordos poda w $\Delta_{97}^9$ na wynik związany z paradoksalnym rozkładem kuli.

Mamy też:

\begin{theorem}[Hjelmsleva]
    Jeżeli wszystkie punkty $P$ prostej przekształcają się izometrycznie na wszystkie punkty $P'$ innej prostej, to wszystkie środki odcinków $PP'$ są różne i współliniowe, albo wszystkie sprowadzają się do jednego punktu.
\end{theorem}

% https://deltami.edu.pl/media/issues/1984/08/delta-1984-08.pdf ?

Patrz też do Coxetera \cite[s. 63, 64]{coxeter_1967}.