
\begin{definition}[symetria osiowa]
    Niech $k$ będzie ustaloną prostą, $X$ punktem, zaś $X'$ jego rzutem na prostą $k$.
    Oznaczmy przez $w$ wektor $2X X'$.
    Wtedy funkcję $S_k$ zadaną przez
    \begin{equation}
    S_k(X) = T_{w}(X)
    \end{equation}
    nazywamy symetrią osiową o osi $k$.
\end{definition}

(Chodzi o to, by rzut punktu $X$ na prostą $k$ był środkiem odcinka łączącego argument oraz wartość funkcji $S_k$).
Symetrie osiowe nie mają estetycznego wzoru, jeśli nie chcemy wprowadzać do książki macierzy: punkt $(x_0, y_0)$ odbity względem osi o równaniu $ax + by + c = 0$ przechodzi na
\begin{equation}
    \left(
    x_0 - 2a \cdot \frac{ax_0 + by_0 + c}{a^2 + b^2},
    y_0 - 2b \cdot \frac{ax_0 + by_0 + c}{a^2 + b^2}
    \right),
\end{equation}

\begin{proposition}
\label{kordos_banach_6}%
    Złożenie dwóch symetrii osiowych, których osie są równoległe, jest translacją (o wektor dwa razy dłuższy niż odległość między osiami, prostopadły do obydwu).
\end{proposition}

\begin{proposition}
\label{klkm1}%
    Niech $k, l$ będą prostymi.
    Wtedy $S_k \circ S_l \circ S_k = S_m$, gdzie $m$ to prosta będąca odbiciem prostej $l$ względem $k$.
\end{proposition}

Jeśli figura ma dokładnie dwie osi symetri, to muszą być prostopadłe.
