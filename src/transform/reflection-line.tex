
\begin{definition}[symetria osiowa]
    Niech $k$ będzie prostą.
    Funkcję $S_k$ zadaną następująco: $S_k(X) = Y$ wtedy i tylko wtedy, gdy rzut punktu $X$ na prostą $k$ jest środkiem odcinka $XY$ nazywamy symetrią osiową o osi $k$.
\end{definition}

Symetrie osiowe nie mają estetycznego wzoru, jeśli nie chcemy wprowadzać do książki macierzy (nie chcemy).

\begin{proposition}
    \label{kordos_banach_6}%
    Złożenie dwóch symetrii osiowych, których osie są równoległe, jest translacją.
\end{proposition}

\begin{proposition}
    Złożenie trzech symetrii osiowych jest symetrią osiową.
\end{proposition}