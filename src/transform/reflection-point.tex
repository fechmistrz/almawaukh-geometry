
\begin{definition}[symetria środkowa]
    Niech $A = (x_A, y_A)$ będzie ustalonym punktem.
    Funkcję $S_A$ zadaną następująco:
    \begin{equation}
        S_A(x_0, y_0) = (2x_A - x_0, 2y_A - y_0)
    \end{equation}
    nazywamy symetrią środkową o środku $A$.
\end{definition}

(Chodzi o to, by punkt $A$ był środkiem odcinka łączącego argument oraz wartość funkcji $S_A$).

\begin{example}
    Symetrie osiowe oraz środkowe to przykłady inwolucji.
\end{example}

\begin{proposition}
    Złożenie dwóch symetrii środkowych jest translacją o wektor dwa razy dłuższy od tego, który łączy ich środki.
\end{proposition}

Symetrie środkowe można emulować dwoma symetriami osiowymi:

\begin{proposition}
\label{two_axial_one_point}
    Niech $k, l$ będą dwiema prostymi prostopadłymi do siebie, które przechodzą przez punkt $A$.
    Wtedy $S_k \circ S_l = S_l \circ S_k = S_A$.
\end{proposition}

Równoważnie: jeżeli punkt $A$ leży na prostej $k$, to $S_A \circ S_k = S_k \circ S_A = S_l$, gdzie prosta $l$ jest prostopadła do prostej $k$ i przechodzi przez $A$.
% Ćwiczenie 4.26, Neugebauer s. 203

\begin{proposition}
    Złożenie trzech symetrii środkowych jest symetrią środkową: $S_A \circ S_B \circ S_C = S_D$, gdzie punkt $D$ jest przesunięciem punktu $A$ o wektor $BC$.
\end{proposition}

To jest, po krótkim zastanowieniu, treść twierdzenia Varignona.
\index{twierdzenie!Varignona}%
Mamy odpowiednik faktu \ref{klkm1}:

\begin{proposition}
\label{klkm2}
    Niech $K, L$ będą punktami.
    Wtedy $S_K \circ S_L \circ S_K = S_M$, gdzie $M$ to punkt będący odbiciem punktu $L$ względem $K$.
\end{proposition}

Jeśli figura ma dwa środki symetrii, to ma ich nieskończenie wiele.

%