
\begin{definition}[symetria środkowa]
    Niech $A$ będzie punktem.
    Funkcję $S_A$ zadaną następująco: $S_A(X) = Y$ wtedy i tylko wtedy, gdy punkt $A$ jest środkiem odcinka $XY$ nazywamy symetrią środkową o środku $A$.
\end{definition}

\begin{example}
    Symetrie osiowe oraz środkowe to przykłady inwolucji.
\end{example}

\begin{proposition}
    Złożenie trzech symetrii środkowych jest symetrią środkową.
    Dokładniej, jeśli punkt $D$ jest przesunięciem punktu $A$ o wektor $BC$, to $S_A \circ S_B \circ S_C = S_D$.
\end{proposition}

To jest, po krótkim zastanowieniu, treść twierdzenia Varignona.
\index{twierdzenie!Varignona}%

Jeśli figura ma dokładnie dwie osi symetri, to muszą być prostopadłe.

% Ćwiczenie 4.26, Neugebauer s. 203
\begin{proposition}
    Jeżeli punkt $A$ leży na prostej $k$, to $S_A \circ S_k = S_k \circ S_A = S_l$, gdzie prosta $l$ jest prostopadła do prostej $k$ i przechodzi przez $A$.
\end{proposition}