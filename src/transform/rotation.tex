
\begin{definition}[obrót]
    Niech $A$ będziez punktem, zaś $\alpha$ miarą kąta.
    Funkcję $R_A^\alpha$ zadaną następująco: $R_A^\alpha(X) = Y$ wtedy i tylko wtedy, gdy kąt $\angle XAY$ ma miarę $\alpha$ i $|AY| = |AX|$ nazywamy obrotem o kąt $\alpha$ wokół punktu $A$ zwanego środkiem obrotu.
\end{definition}

\begin{proposition}
    Złożenie dwóch symetrii osiowych, których osie przecinają się pod kątem $\alpha$ jest obrotem o kąt $2 \alpha$ wokół punktu przecięcia wspomnianych osi.
\end{proposition}

\begin{proposition}
\label{for_banach_11}%
    Złożenie dwóch obrotów jest obrotem lub translacją.
\end{proposition}