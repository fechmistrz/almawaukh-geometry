
\begin{definition}[obrót]
    Niech $A$ będziez punktem, zaś $\alpha$ miarą kąta.
    Funkcję $R_A^\alpha$ zadaną następująco: $R_A^\alpha(X) = Y$ wtedy i tylko wtedy, gdy kąt $\angle XAY$ ma miarę $\alpha$ i $|AY| = |AX|$ nazywamy obrotem o kąt $\alpha$ wokół punktu $A$ zwanego środkiem obrotu.
\end{definition}

Oczywiście i tutaj można podać wzór.
Wyznacza się go najpierw dla obrotów wokół zera, by następnie zauważyć, że dowolny obrót sprowadza się do tamtego przy użyciu dwóch translacji (jednej, by przenieść się na początek i drugiej, by wrócić):

\begin{equation}
    R_A^\alpha(x,y) = \begin{pmatrix}
        x_A + (x_0 - x_A) \cos \alpha - (y_0 - y_A) \sin \alpha\\
        y_A + (x_0 - x_A) \sin \alpha - (y_0 - y_A) \cos \alpha
    \end{pmatrix}.
\end{equation}

Symetrie środkowe nazywa się (rzadko) półobrotem, czyli obrotem o kąt $\pi$; robi tak Eves \cite[s. 105]{eves1_1972}.

\begin{proposition}
    Złożenie dwóch symetrii osiowych, których osie przecinają się pod kątem $\alpha$ jest obrotem o kąt $2 \alpha$ wokół punktu przecięcia wspomnianych osi.
\end{proposition}

To uogólnienie faktu \ref{two_axial_one_point}, tam było $\alpha = \pi$, a półobrót (obrót o kąt $\pi$) jest tym samym, co symetria środkowa.

\begin{proposition}
\label{for_banach_11}%
    Niech $A \neq B$ będą dwoma punktami, zaś $\alpha, \beta$ dwiema miarami kątów.
    Jeśli $\alpha + \beta$ nie jest wielokrotnością kąta pełnego, $2 \pi$, to $R_B^\beta \circ R_A^\alpha = R_C^{\alpha + \beta}$, gdzie $C$ jest takim punktem, by trójkąt $\triangle ABC$ miał kąty $\alpha/2$ przy $A$, $\beta / 2$ przy $B$.

    W przeciwnym przypadku złożenie tych dwóch obrotów jest translacją.
\end{proposition}