\todofoot{Coxeter, s. 67}

\begin{definition}[podobieństwo]
    Niech $P \colon \Pi \to \Pi$ będzie przekształceniem geometrycznym, zaś $s > 0$ dodatnią liczbą taką, że dla każdej pary punktów $A, B \in \Pi$ zachodzi
    \begin{equation}
        d(P(A), P(B)) = s \cdot d(A, B).
    \end{equation}
    Wtedy przekształcenie $P$ nazywamy podobieństwem o skali $s$.
\end{definition}

Podobieństwa są bijekcjami, zachowują miary kątów, a~obrazami prostych są proste.

\begin{proposition}
    Jeżeli $\lambda \mu \neq 1$, to $J_A^\lambda \circ J_B^\mu = J_C^{\lambda \mu}$, gdzie $[CBA] = (1-\lambda) / (\lambda \mu - 1)$.
\end{proposition}

To jest ćwiczenie 4.75 \cite[s. 217]{neugebauer_2018}.

\begin{proposition}
    Każde podobieństwo jest izometrią, symetrią dylatacyjną albo podobieństwem spiralnym.
\end{proposition}

Wynika to z klasyfikacji izometrii, patrz Bogdańska, Neugebauer \cite[s. 220]{neugebauer_2018}.

\begin{itemize}
        \item izometrią,
        \item podobieństwem spiralnym (złożeniem obrotu i~jednokładności o~tym samym środku) albo
        \item symetrią dylatacyjną (złożeniem jednokładności z~symetrią osiową o~osi przechodzącej przez środek jednokładności).
    \end{itemize}

\begin{proposition}
    Każde podobieństwo o skali $s \neq 1$ ma dokładnie jeden punkt stały.
\end{proposition}

\subsection{Symetrie dylatacyjne}

\subsection{Podobieństwa spiralne}