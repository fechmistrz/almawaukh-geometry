\section{Przekształcenia geometryczne}
Przekształcenia geometryczne

\subsection{Izometrie}
Izometrie
\todofoot{Izometrie Coxetera, s. 29-36, 47 -- Hjelmslev}

\begin{enumerate}
    \item Konstrukcja obrazu punktu, okręgu, prostej przy translacji, obrocie i symetrii osiowej.
    \item Złożenie dwóch i złożenie trzech symetrii osiowych.
    \item Twierdzenia o składaniu izometrii.
    \item Klasyfikacja izometrii na płaszczyźnie.
    \item Izometrie parzyste i izometrie nieparzyste.
    \item Twierdzenie o redukcji.
    \item Twierdzenie Napoleona: środki ciężkości trójkątów równobocznych zbudowanych na bokach dowolnego trójkąta są wierzchołkami trójkąta równobocznego.
\end{enumerate}

\subsubsection{Symetria osiowa}
Symetria osiowa.
\todofoot{Symetria osiowa z poślizgiem.}
\todofoot{Twierdzenie Chasles'a: każda izometria płaszczyzny jest złożeniem co najwyżej trzech symetrii osiowych.}

\subsubsection{Symetria środkowa}
Symetria środkowa

\subsection{Jednokładność -- podobieństwo?}
\todofoot{Coxeter, s. 67}

\begin{definition}[podobieństwo]
    Niech $P \colon \Pi \to \Pi$ będzie przekształceniem geometrycznym, zaś $s > 0$ dodatnią liczbą taką, że dla każdej pary punktów $A, B \in \Pi$ zachodzi
    \begin{equation}
        d(P(A), P(B)) = s \cdot d(A, B).
    \end{equation}
    Wtedy przekształcenie $P$ nazywamy podobieństwem o skali $s$.
\end{definition}

Podobieństwa są bijekcjami, zachowują miary kątów, a~obrazami prostych są proste.

\begin{proposition}
    Jeżeli $\lambda \mu \neq 1$, to $J_A^\lambda \circ J_B^\mu = J_C^{\lambda \mu}$, gdzie $[CBA] = (1-\lambda) / (\lambda \mu - 1)$.
\end{proposition}

To jest ćwiczenie 4.75 \cite[s. 217]{neugebauer_2018}.

\begin{proposition}
    Każde podobieństwo jest izometrią, podobieństwem spiralnym (złożeniem obrotu i~jednokładności o~tym samym środku) albo symetrią dylatacyjną (złożeniem jednokładności z~symetrią osiową o~osi przechodzącej przez środek jednokładności).
\end{proposition}

Wynika to z klasyfikacji izometrii, patrz Bogdańska, Neugebauer \cite[s. 220]{neugebauer_2018}.

\begin{proposition}
    Każde podobieństwo o skali $s \neq 1$ ma dokładnie jeden punkt stały.
\end{proposition}

\todofoot{Słowo Banacha.}

%
