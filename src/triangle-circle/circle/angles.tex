
\begin{proposition}
    Kąt środkowy jest dwa razy większy od kąta wpisanego opartego na tym samym łuku.
\end{proposition}
To jest Guzicki \cite[s. 11]{guzicki_2021}.

\begin{corollary}
    Kąty wpisane oparte na tym samym łuku (ogólniej: na równych łukach) są równe.
\end{corollary}

\begin{corollary}
    Kąty oparty na półokręgu jest prosty.
\end{corollary}

\begin{corollary}
    \label{ab_twice_pi}
    Kąty wpisane oparte na dwóch różnych łukach $AB$ dają w sumie $\pi$.
\end{corollary}

\begin{proposition}
    Kąt dopisany do okręgu jest równy kątowi wpisanemu opartemu na łuku zawartym w danym kącie dopisanym.
\end{proposition}
% Guzicki s. 18

Geometria koła i kątów, twierdzenie Apolloniusza (s. 22)