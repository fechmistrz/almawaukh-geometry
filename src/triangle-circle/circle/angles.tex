
\begin{proposition}
    Kąt środkowy jest dwa razy większy od kąta wpisanego opartego na tym samym łuku.
    \index{kąt!środkowy}
    \index{kąt!wpisany}
\end{proposition}
To jest Guzicki \cite[s. 11]{guzicki_2021}, Audin \cite[s. 74]{audin_2003}.

\begin{corollary}
    Kąty wpisane oparte na tym samym łuku (ogólniej: na równych łukach) są równe.
\end{corollary}

Audin \cite[s. 75]{audin_2003}.

\begin{corollary}
    Kąty oparty na półokręgu jest prosty.
\end{corollary}

\begin{corollary}
    \label{ab_twice_pi}
    Kąty wpisane oparte na dwóch różnych łukach $AB$ dają w sumie $\pi$.
\end{corollary}

\begin{proposition}
    Kąt dopisany do okręgu jest równy kątowi wpisanemu opartemu na łuku zawartym w danym kącie dopisanym.
    \index{kąt!dopisany}
\end{proposition}
% Guzicki s. 18

% TODO https://en.wikipedia.org/wiki/Inscribed_angle

Geometria koła i kątów, twierdzenie Apoloniusza (s. 22)
\index{twierdzenie!Apoloniusza?}