
Materiał przytoczony tutaj będzie łatwy do odnalezienia w~dowolnym podręczniku geometrii, dla przykładu u~Guzickiego \cite[s. 11-13, 18, 19]{guzicki_2021}, Audina \cite[s. 74, 75]{audin_2003}, Neugebauera \cite[s. 22, 23]{neugebauer_2018}.

\begin{proposition}
\index{kąt!środkowy}%
\index{kąt!wpisany}%
    Kąt środkowy jest dwa razy większy od kąta wpisanego opartego na tym samym łuku.
\begin{figure}[H]\centering%
\begin{comment}
\begin{tikzpicture}[scale=.35]
    \tkzDefPoint(0, 0){Zero}
    \tkzDefPoint(100:5){A}
    \tkzDefPoint(100:3.5){Aa}
    \tkzDefPoint(230:5){B}
    \tkzDefPoint(285:1.5){Oo}
    \tkzDefPoint(340:5){C}
    \tkzDrawCircle[line width=0.5mm](Zero,A)
    \tkzLabelPoint[above left](Zero){$O$}
    \tkzLabelPoint[above](A){$A$}
    \tkzLabelPoint[below left](B){$B$}
    \tkzLabelPoint[below right](C){$C$}
    \tkzDrawPolygons[line width=0.3mm](A,B,Zero,C)
    \tkzMarkAngle(B,Zero,C)
    \tkzMarkAngle(B,A,C)
    \tkzDrawPoints[size=3,color=black,fill=red!50](A,B,C,Zero)
    \tkzLabelPoint[anchor=center](Aa){$\alpha$}
    \tkzLabelPoint[anchor=center](Oo){$2\alpha$}
\end{tikzpicture}
\end{comment}
    \caption{kąt środkowy jest dwa razy większy od wpisanego}
\end{figure}
\end{proposition}
% PRZECZYTANO: https://en.wikipedia.org/wiki/Inscribed_angle

To będzie (III.20).

\begin{corollary}
    Kąty wpisane oparte na tym samym łuku (ogólniej: na równych łukach) są równe.
\end{corollary}

Neuegebauer (i nikt poza nim) nazwie nie wiedzieć czemu powyższy wniosek twierdzeniem Apolloniusza.

\begin{corollary}
    Kąty oparty na półokręgu jest prosty.
\end{corollary}

Grecki pisarz Diogenes Laertios przypisze to stwierdzenie Talesowi, chociaż nazwa przyjmie się jedynie w krajach anglosaskich.
\index[persons]{Tales}%
\index[persons]{Laertios, Diogenes}%
Można wykorzystać je do konstrukcji stycznej.
\index{styczna}%

\begin{corollary}
    \label{ab_twice_pi}
    Kąty wpisane oparte na dwóch różnych łukach $AB$ dają w sumie kąt półpełny.
\end{corollary}

Wykorzystamy ten wniosek później (fakt \ref{prp_incircle}) do opisania, na których czworokątach można opisać okrąg.

\begin{proposition}
    Kąt dopisany, między styczną do okręgu a jego cięciwą przechodzącą przez punkt styczności, jest równy kątowi wpisanemu w ten okrąg, opartemu na odpowiednim łuku: $\angle ABC = \angle CAD$.
    \index{kąt!dopisany}
\begin{figure}[H] \centering
\begin{comment}
\begin{tikzpicture}[scale=.4]
    \tkzDefPoint(0, 0){Zero}
    \tkzDefPoint(50:5){A}
    \tkzDefPoint(230:5){B}
    \tkzDefPoint(300:5){C}
    \tkzDefLine[tangent at=A](Zero) \tkzGetPoint{DD}
    \tkzDefPointsBy[projection=onto A--DD](C){D}
    \tkzDrawCircle[line width=0.5mm](Zero,A)
    \tkzDrawLines[add= 1 and 1, line width=0.3mm](A,D)
    \tkzDrawPolygons[line width=0.3mm](A,B,C)
    \tkzMarkAngle[arc=l,size=1.2,mark=||](C,B,A)
    \tkzMarkAngle[arc=l,size=1.2,mark=||](C,A,D)
    \tkzDrawPoints[size=3,color=black,fill=black!50](A,B,C,D)
    \tkzLabelPoint[above left](Zero){$O$}
    \tkzLabelPoint[above right](A){$A$}
    \tkzLabelPoint[below left](B){$B$}
    \tkzLabelPoint[below right](C){$C$}
    \tkzLabelPoint[above right](D){$D$}
\end{tikzpicture}
\end{comment}
    \caption{kąt dopisane jest równy wpisanemu}
\end{figure}
\end{proposition}
