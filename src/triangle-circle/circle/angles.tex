
\begin{proposition}
    Kąt środkowy jest dwa razy większy od kąta wpisanego opartego na tym samym łuku.
    \index{kąt!środkowy}
    \index{kąt!wpisany}
\end{proposition}
% PRZECZYTANO: https://en.wikipedia.org/wiki/Inscribed_angle

O tym samym co (III.20) piszą Guzicki \cite[s. 11-13]{guzicki_2021}, Audin \cite[s. 74, 75]{audin_2003}.

\begin{corollary}
    Kąty wpisane oparte na tym samym łuku (ogólniej: na równych łukach) są równe.
\end{corollary}

\begin{corollary}
    Kąty oparty na półokręgu jest prosty.
\end{corollary}

Grecki pisarz Diogenes Laertios przypisze to stwierdzenie Talesowi.
\index[persons]{Tales}%
\index[persons]{Laertios, Diogenes}%
Można wykorzystać je do konstrukcji stycznej.
\index{styczna}%

\begin{corollary}
    \label{ab_twice_pi}
    Kąty wpisane oparte na dwóch różnych łukach $AB$ dają w sumie kąt półpełny.
\end{corollary}

Wykorzystamy ten wniosek później (fakt \ref{prp_incircle}) do opisania, na których czworokątach można opisać okrąg.

\begin{proposition}
    Kąt dopisany do okręgu jest równy kątowi wpisanemu opartemu na łuku zawartym w danym kącie dopisanym.
    \index{kąt!dopisany}
\end{proposition}
% Guzicki s. 18

\todofoot{Geometria koła i kątów, twierdzenie Apoloniusza (s. 22)}
\index{twierdzenie!Apoloniusza?}