\subsubsection{Kąty środkowe, wpisane, dopisane}

\begin{proposition}
    Kąt środkowy jest dwa razy większy od kąta wpisanego opartego na tym samym łuku.
    \index{kąt!środkowy}
    \index{kąt!wpisany}
\end{proposition}
To jest Guzicki \cite[s. 11]{guzicki_2021}, Audin \cite[s. 74]{audin_2003}.

\begin{corollary}
    Kąty wpisane oparte na tym samym łuku (ogólniej: na równych łukach) są równe.
\end{corollary}

Audin \cite[s. 75]{audin_2003}.

\begin{corollary}
    Kąty oparty na półokręgu jest prosty.
\end{corollary}

\begin{corollary}
    \label{ab_twice_pi}
    Kąty wpisane oparte na dwóch różnych łukach $AB$ dają w sumie $\pi$.
\end{corollary}

\begin{proposition}
    Kąt dopisany do okręgu jest równy kątowi wpisanemu opartemu na łuku zawartym w danym kącie dopisanym.
    \index{kąt!dopisany}
\end{proposition}
% Guzicki s. 18

% TODO https://en.wikipedia.org/wiki/Inscribed_angle

Geometria koła i kątów, twierdzenie Apoloniusza (s. 22)
\index{twierdzenie!Apoloniusza?}

\subsubsection{Okręgi wpisane i opisane}

Okrąg wpisany, opisany, dopisany: Audin \cite[s. 98]{audin_2003}.


Styczna do okręgu, okrąg wpisany w kąt.
Okrąg wpisany w trójkąt, okręgi dopisane do trójkąta.
Warunki istnienia okręgu stycznego do czterech prostych.

Euklides (III.35):

% https://en.wikipedia.org/wiki/Intersecting_chords_theorem
\begin{proposition}[twierdzenie o cięciwach]
    \label{prop_intersecting_chords}
	Niech punkty $A$, $B$ leżą na pewnej prostej przechodzącej przez punkt $S$, zaś punkty $C$, $D$ leżą na innej prostej przez ten punkt.
	Wtedy cztery punkty $A$, $B$, $C$, $D$ leżą na jednym okręgu wtedy i tylko wtedy, gdy:
	\begin{equation}
		|AS| \cdot |BS| = |CS| \cdot |DS|.
	\end{equation}
\end{proposition}

Po angielsku nazywa się to ,,\emph{intersecting chords theorem}''; Bogdańska i Neugebauer nazywają to nie wiedzieć czemu ,,potęgowym kryterium współokręgowości''.
Ech.
Elementarny dowód korzysta z~podobieństwa trójkątów $\triangle ASD$ i $\triangle BSC$.
Nieelementarny zauważa, że obydwa iloczyny są równe (poza znakiem) potędze punktu $S$ względem okręgu przez punkty $A$, $B$, $C$, $D$.
Definicję \ref{def_power_point} potęgi punktu podamy już za kilka stron.

\begin{proposition}[twierdzenie o siecznych i stycznych]
	Jeżeli... (Neugebauer s. 66)
\end{proposition}

\todofoot{twierdzenie o motylku} % https://en.wikipedia.org/wiki/Butterfly_theorem

\subsubsection{Twierdzenia Ptolemeusza, Carnota, Caseya}
\todofoot{Guzicki, rozdział 17}

Klaudiusz Ptolemeusz był astronomem, matematykiem i~geografem pochodzenia greckiego.
\index[persons]{Ptolemeusz, Klaudiusz}%
Urodzon w Tebaidzie (około roku 100), kształcił się, działał w~Aleksandrii; tam też zmarł około roku 170.
Napisał po grecku Μαθηματικὴ Σύνταξις, traktat w trzynastu księgach znany lepiej jako \emph{Almagest} zawierający kompendium wiedzy astronomicznej oraz matematyczny wykład teorii geocentrycznej.
Tam też znajduje się prawie\footnote{Ptolemeusz udowodnił równość, a nie nierówność, ale nazwa się przyjęła.} całe następujące twierdzenie:

% TODO: The Almagest was preserved, like many extant Greek scientific works, in Arabic manuscripts; the modern title is thought to be an Arabic corruption of the Greek name Hē Megistē Syntaxis ('The greatest treatise'), as the work was presumably known during late antiquity.[35] Because of its reputation, it was widely sought and translated twice into Latin in the 12th century, once in Sicily and again in Spain.[36] Ptolemy's planetary models, like those of the majority of his predecessors, were geocentric and almost universally accepted until the reappearance of heliocentric models during the Scientific Revolution.

\begin{theorem}[Ptolemeusza, 140 r.n.e.]
\index{nierówność!Ptolemeusza}%
\index{twierdzenie!Ptolemeusza}%
    W czworokącie wypukłym $ABCD$ zachodzi
    \begin{equation}
        |AC| \cdot |BD| \le |AB| \cdot |CD| + |BC| \cdot |AD|,
    \end{equation}
    z równością wtedy i tylko wtedy, gdy na czworokącie $ABCD$ można opisać okrąg.
\end{theorem}

To wystarczyło mu, żeby opracować ,,tablice cięciw'' (równoważne tablicom wartości funkcji trygonometrycznych), potrzebne do celów astronomicznych.
Wcześniejsze tablice Hipparchosa z~Nikei opisywały tylko wielokrotności kąta miary $\pi/24$.
\index[persons]{Hipparchos z Nikei}% % to jest ten, co go utopili?
\todofoot{Thurston, Hugh (1996), Early Astronomy, Springer, ISBN 978-0-387-94822-5, strony 235-236}

O twierdzeniu Ptolemeusza piszą Bogdańska, Neugebauer \cite[s. 62, 63]{neugebauer_2018}, Audin \cite[s. 108]{audin_2003}.
Delta: 2024/sierpień.
Angielska Wikipedia podpowiada, że założenie o wypukłości można pominąć, czwórka punktów nie musi nawet leżeć w~jednej płaszczyźnie -- ale wtedy równość zachodzi też wtedy, kiedy punkty są współliniowe.
% https://en.wikipedia.org/wiki/Ptolemy%27s_inequality

Twierdzenie Ptolemeusza można albo uogólnić (twierdzenie Caseya podamy za chwilę), albo wyprowadzić z niego jedno z kilku twierdzeń, które przypisuje się Lazarowi Carnotowi \cite{carnot_1803}.
\index[persons]{Carnot, Lazare}
% https://ru.wikipedia.org/w/index.php?title=Формула_Карно&oldid=8679639
% В ее доказательстве используется теорема Птолемея.

\begin{theorem}[Carnot, 1803?]
    \index{twierdzenie!Carnota}%
    Niech $ABC$ będzie trójkątem wpisanym w okrąg o środku $O$ i promieniu $R$ oraz opisanym na okręgu o promieniu $r$.
    Oznaczmy przez $OO_A$ (i analogicznie $OO_B$, $OO_C$) znakowaną odległość punktu $O$ od boku $BC$.
    Wtedy 
    \begin{equation}
        OO_A + OO_B + OO_C = R + r.
    \end{equation}
    (Odległość jest ujemna wtedy i tylko wtedy, gdy cały odcinek leży poza trójkątem).
    \index{twierdzenie!Carnota}%
\end{theorem}

Wynik ten znajduje znowu zastosowanie w dowodzie twierdzenia japońskiego. % TODO: Neugebauer s. 65
\index{twierdzenie!japońskie}

% TODO: https://en.wikipedia.org/wiki/Van_Schooten's_theorem

\begin{theorem}[Caseya, 1866]
\index{twierdzenie!Caseya}%
    Niech $\Gamma_1$, $\Gamma_2$, $\Gamma_3$, $\Gamma_4$ będą czterema okręgami ponumerowanymi zgodnie z ruchem wskazówek zegara, z których każdy styka się z piątym okręgiem $\Gamma$.
    Niechh $t_{ij}$ oznacza długość zewnętrznego odcinka stycznego łączącego okręgi $\Gamma_i$, $\Gamma_j$ (jeśli te stykają się z $\Gamma$ obydwa od wewnątrz lub obydwa od zewnątrz) albo długość wewnętrznego odcinka stycznego (w przeciwnym razie).
    Wówczas:
    \begin{equation}
        t_{12} \cdot t_{34} + t_{14} \cdot t_{23} = t_{13} \cdot t_{24}.
    \end{equation}
\end{theorem}

% https://en.wikipedia.org/wiki/Casey%27s_theorem
Twierdzenie podał John Casey (1820-1891), szanowany irlandzki geometra, który razem z Émilem Lemoinem uznawany jest za współzałożyciela nowoczesnej geometrii trójkątów i okręgów.
\index[persons]{Casey, John}%
\index[persons]{Lemoine, Émile}% % Émile Michel Hyacinthe Lemoine
Inny dowód wymyślił Max Zacharias \cite{zacharias_1942}.
\index[persons]{Zacharias, Max}%
Twierdzenie odwrotne do podanego przydaje się w najkrótszym znanym dowodzie twierdzenia Feuerbacha (że okrąg dziewięciu punktów jest styczny do okręgów dopisanych oraz wpisanego).
\index{okrąg!wpisany}%
\index{okrąg!dopisany}%
\index{twierdzenie!Feuerbacha}%
\index{okrąg!dziewięciu punktów}%

Znajdziemy je u Bogdańskiej, Neugebauera jako ćwiczenie \cite[s. 105]{neugebauer_2018}.

%

\subsubsection{Potęga punktu względem okręgu}
\index{potęga punktu względem okręgu} % zamienić na zakres stron

Inwersja: Audin \cite[s. 84]{audin_2003}.
% zachowuje kąty

\todofoot{Coxeter s. 85, coaxial circles}

Potęgę punktu wprowadzi Jakob Steiner\footnote{Praca \emph{Einige geometrische Betrachtungen}, strona 164} w 1826 roku.
\index[persons]{Steiner, Jakob}%
Użyje jej, aby znaleźć okrąg przecinający cztery dane okręgi pod tym samym kątem; rozwiązać problem Apoloniusza (problem \ref{problem_apolloniusza}: znaleźć okrąg styczny do trzech); skonstruować okręgi Malfattiego (problem \ref{malfatti_problem}: trzy okręgi, które są styczne do pozostałych dwóch oraz boków zadanego trójkąta).
% TODO: index, ref
\index{problem!Apoloniusza}%
\index{okrąg!Malfattiego}%
Wydaje się, że ta lista jest niekompletna.

Wśród elementarnych zastosowań można wskazać dowód twierdzenia o przecinających się cięciwach (fakt \ref{prop_intersecting_chords}).

\begin{definition}[potęga punktu względem okręgu]
	\label{def_power_point}
	Dane są okrąg $\Gamma$ o środku $O$ i promieniu $r$ oraz dowolny punkt $A$.
	Liczbę rzeczywistą
	\begin{equation}
		\Pi(A) := |OA|^2 - r^2
	\end{equation}
	nazywamy potęgą punktu $A$ względem okręgu $\Gamma$.
\end{definition}

Pisze o tym Audin \cite[s. 89]{audin_2003}, nieznany autor w $\Delta_{84}^{11}$.

Twierdzenie Pitagorasa dostarcza prostej interpretacji wielkości $\Pi(A)$, gdy punkt $A$ leży na zewnątrz okręgu: jest to długość stycznej do okręgu, która przechodzi przez punkt.
\index{twierdzenie!Pitagorasa}

\begin{proposition}
	Środki odcinków wspólnych stycznych do dwóch okręgów leżą na jednej prostej. % Neugebauer s. 70
\end{proposition}

Nie wiemy, jak trudny jest dowód powyższego bez poniższego.

\begin{proposition}[oś potęgowa istnieje]
\label{guzicki_6_11}%
    Dane są dwa niewspółśrodkowe okręgi $\Gamma_1(O_1, r_1)$ oraz $\Gamma_2(O_2, r_2)$.
    Wtedy miejscem geometrycznym punktów $P$ mających równe potęgi względem obu okręgów:
	\begin{equation}
		\{S : \Pi(S, \Gamma_1) = \Pi(S, \Gamma_2)\}
	\end{equation}
	jest prosta prostopadła do prostej przechodzącej przez środki obu okręgów.
	Nazywamy ją \emph{osią potęgową} okręgów $\Gamma_1$, $\Gamma_2$.
	\index{oś potęgowa}%
	\index{potęgowa!oś|see{oś potęgowa}}
\end{proposition}

O osi potęgowej piszą Bogdańska, Neugebauer \cite[s. 69]{neugebauer_2018}; Guzicki \cite[s. 173, 174]{guzicki_2021}, Audin \cite[s. 89]{audin_2003}, Eves \cite[s. 92]{eves1_1972}.
Oś potęgowa okręgów stycznych to prosta styczna do obydwu okręgów; oś potęgowa okręgów, które się przecinają w dwóch punktach, to prosta przez punkty przecięcia.

\begin{proposition}[środek potęgowy]
	Dane są trzy parami niewspółśrodkowe okręgi $\Gamma_1, \Gamma_2, \Gamma_3$.
	Wtedy albo środki tych okręgów są współliniowe (i trzy osie potęgowe każdej pary są równoległe), albo wszystkie trzy osie przecinają się w~jednym punkcie: \emph{środku potęgowym} okręgów.
	\index{środek potęgowy}%
	\index{potęgowy!środek|see{środek potęgowy}}
\end{proposition}

Nie jest jasne, dlaczego niektórzy nazywają to twierdzeniem Monge'a (jak Bogdańska, Neugebauer \cite[s. ???]{neugebauer_2018}) albo też rozwiązaniem problemu Monge'a (jak Dörrie \cite[s. 151]{dorrie_1965}).
\index{twierdzenie!Monge'a}%
Ale jest jasne, że pozwala łatwo skonstruować oś potęgową dwóch okręgów: wystarczy przeciąć je jednocześnie trzecim okręgiem (na dwa różne sposoby).

Patrz Guzicki \cite[s. 174]{guzicki_2021}.

% Neugebauer s. 71
\begin{proposition}
	Ortocentrum trójkąta jest środkiem potęgowym rodziny wszystkich okręgów o średnicach będących czewianami (odcinkami łączącymi wierzchołek z przeciwległym bokiem).
\end{proposition}

\begin{theorem}[Auberta]
	Dane są cztery proste w położeniu ogólnym (żadne trzy nie przechodzą przez jeden punkt, żadne dwie nie są równoległe).
	Wówczas ortocentra czterech trójkątów wyznaczonych przez trójki tych prostych leża na jednej prostej, zwanej prostą Auberta.
	\index{twierdzenie!Auberta}
\end{theorem}

Nie udało nam się ustalić, kim był Aubert, ale najprawdopodobniej udowodnił to twierdzenie w 1899 roku.
\index[persons]{Aubert, urodzny przed 1899}
Prosta Auberta znana jest czasami jako prosta Steinera.
% TODO: https://el-m-wikipedia-org.translate.goog/wiki/Ευθεία_Σίμσον_(τετράπλευρο)?_x_tr_sl=auto&_x_tr_tl=pl&_x_tr_hl=pl&_x_tr_pto=wapp
% https://el-m-wikipedia-org.translate.goog/wiki/Θεώρημα_Miquel_(τετράπλευρο)?_x_tr_sl=auto&_x_tr_tl=pl&_x_tr_hl=pl&_x_tr_pto=wapp
% https://el-m-wikipedia-org.translate.goog/wiki/Ευθεία_Νεύτωνα-Γκάους?_x_tr_sl=auto&_x_tr_tl=pl&_x_tr_hl=pl&_x_tr_pto=wapp
Patrz też \cite[s. 95]{eves1_1972}: trzy okręgi ze średnicami na przekątnych czworoboku zupełnego są coaxial (współosiowe?); środki trzech przekątnych leżą na prostej prostopadłej do Auberta.

% pencil
O pencilu piszą Audin \cite[s. 92-98]{audin_2003}, Eves \cite[s. 94]{eves1_1972}.

\subsubsection{Okrąg dziewięciu punktów (Feuerbacha) i prosta Eulera}
%

Prosta Eulera to pierwsza w szkolnej geometrii trójka punktów współliniowych.
Przyszła na świat w żurnalu ,,Novi commentarii Academiae Petropolitanae (ad annum 1765)'', w artykule ,,Solutio facilis problematum quorundam geometricorum difficillimorum''\footnote{Łatwe rozwiązanie niektórych najtrudniejszych problemów geometrycznych.}.

\begin{proposition}[prosta Eulera]
	\label{prosta_eulera}
	Środek okręgu opisanego na nierównobocznym trójkącie, środek ciężkości oraz ortocentrum leżą na jednej prostej, zwanej prostą Eulera.
\end{proposition}
\todofoot{Coxeter, s. 17}

Prosta Eulera jest prostopadła do jednego z boków wtedy i tylko wtedy, gdy trójkąt jest równoramienny.
Piszą o niej Hartshorne \cite[s. 54, 55]{hartshorne2000}, Bogdańska, Neugebauer \cite[s. 84]{neugebauer_2018}.

\begin{proposition}[okrąg dziewięciu punktów]
	\label{okrag_dziewieciu_punktow}
	W każdym trójkącie środki boków, spodki wysokości oraz środki odcinków łączących ortocentrum z wierzchołkami leżą na jednym okręgu.
	Jego środek pokrywa się ze środkiem odcinka łączącego środek okręgu opisanego z ortocentrum, zaś jego promień jest dwukrotnie krótszy od promienia okręgu opisanego.
\end{proposition}

Hartshorne \cite[s. 57, 60]{hartshorne2000}, Bogdańska, Neugebauer \cite[s. 85, 86]{neugebauer_2018}.

Temat badali Benjamin Bevan (który zasugerował środek oraz promień) i John Butterworth (który udowodnił podejrzenia Bevana) na początku XIX wieku.
\index[persons]{Bevan, Benjamin}%
\index[persons]{Butterworth, John}%
To,  że środki boków i spodki wysokości leżą na wspólnym okręgu, zostało zauważone w 1821 roku przez Charles Brianchona i Jean-Victora Ponceleta.
\index[persons]{Brianchon, Charles}%
\index[persons]{Poncelet, Jean-Victor}%
Tego samego odkrycia dokonał rok później Karl Feuerbach; a krótko po nim Olry Terquem zauważył, że leży na nim dziewięć, a nie tylko sześć wspomnianych punktów.
\todofoot{The nine-point circle also passes through Kimberling centers Xi for i=11 (the Feuerbach point), 113, 114, 115 (center of the Kiepert hyperbola), 116, 117, 118, 119, 120, 121, 122, 123, 124, 125 (center of the Jerabek hyperbola), 126, 127, 128, 129, 130, 131, 132, 133, 134, 135, 136, 137, 138, 139, 1312, 1313, 1560, 1566, 2039, 2040, and 2679.}
\todofoot{Karl Wilhelm Feuerbach's Eigenschaften einiger merkwiirdigen Punkte des geradlinigen Dreiecks, along with many other interesting proofs relating to the nine point circle.}
\index[persons]{Feuerbach, Karl}%
\index[persons]{Terquem, Olry}%
Terquemowi (1842) zawdzięczamy nazwę ,,okrąg dziewięciu punktów''.
\todofoot{The circle is officially designated the "nine point circle" (le cercle des neuf points) by Terquem, one of the editors of the Nouvelles Annales. (see Volume I page 198).}

Feuerbach udowodnił też, że:

\begin{theorem}[Feuerbacha]
	Okrąg dziewięciu punktów jest styczny wewnętrznie do okręgu wpisanego (w punkcie Feuerbacha) i zewnętrznie do trzech okręgów dopisanych.
\end{theorem}

\todofoot{Wszystkie wysokości itd. przecinają się w jednym punkcie; prosta Eulera, okrąg Feuerbacha (Guzicki-8)}
punkt Torricellego/Fermata (Guzicki-8)

\begin{proposition}
	\label{orthic_triangle}
	Niech $ABC$ będzie trójkątem ostrokątnym, zaś $K$, $L$ oraz $M$ spodkami jego wysokości.
	Wtedy wysokości trójkąta $ABC$ są dwusiecznymi kątów trójkąta $KLM$.
\end{proposition}

Hartshorne \cite[s. 58]{hartshorne2000}.

%

\subsubsection{Twierdzenia o sześciu, siedmiu i dziewięciu okręgach}
\label{sssection_6_7_9_circles}
% TODO: pięciu, Guzicki ps. 36
Cecil John Alvin Evelyn, G. B. (pełne imiona nieznane) Money-Coutts i John Alfred Tyrrell znaleźli około 1974 roku piękne twierdzenia o okręgach wpisanych, które pokazują nam, jak wiele wyników geometrii elementarnej oczekuje na swoje odkrycie.

\begin{proposition}[o sześciu okręgach]
	Dany jest trójkąt $\triangle ABC$.
	Niech $\Gamma_1, \Gamma_2, \ldots, \Gamma_6$ będą okręgami wpisanymi kolejno w kąty przy wierzchołkach $A$, $B$, $C$, $A$, $B$, $C$, $A$ takimi, że każdy jest styczny do poprzedniego.
	Wtedy okręgi $\Gamma_6$ i $\Gamma_1$ też są do siebie styczne.
\end{proposition}

Pisze o tym Bogdańska, Neugebauer \cite[s. 101]{neugebauer_2018}: wykorzystują punkt Crelle'a-Brocarda, wzór Herona i trochę trygonometrii.
Tabacznikow, Iwanow \cite{ivanov_tabachnikov_2016} pokazali, że jeśli osłabimy założenia: okręgi nie muszą zawierać się w~trójkącie i~wystarczy, że będą styczne do prostych zawierających boki trójkąta, to nadal ciąg okręgów jest od pewnego miejsca okresowy z okresem równym sześć, ale osiągnięcie tego stanu może wymagać dowolnie wielu kroków.
\index[persons]{Tabacznikow, Siergiej (Табачников, Сергей Львович)}%
\index[persons]{Iwanow, Denis (Иванов, Денис)}%

\begin{proposition}[o siedmiu okręgach]
	Dany jest okrąg $\Gamma$ oraz sześć okręgów stycznych do niego tak, że każdy jest też styczny do swoich dwóch sąsiadów.
	Wtedy trzy proste łączące przeciwległe punkty styczności przecinają się w jednym punkcie.
\end{proposition}

\begin{proposition}[o dziewięciu okręgach]
	Niech $\Gamma_1$, $\Gamma_2$, $\Gamma_3$ będą trzema okręgami na płaszczyźnie, zaś okrąg $\Gamma_4$ będzie styczny do $\Gamma_2$ i $\Gamma_3$.
	Kreślimy ciąg okręgów stycznych do poprzednich:
	$\Gamma_5$ do $\Gamma_1$, $\Gamma_3$ i $\Gamma_4$,
	$\Gamma_6$ do $\Gamma_1$, $\Gamma_2$ i $\Gamma_5$,
	$\Gamma_7$ do $\Gamma_2$, $\Gamma_3$ i $\Gamma_6$,
	$\Gamma_8$ do $\Gamma_1$, $\Gamma_3$ i $\Gamma_7$,
	$\Gamma_9$ do $\Gamma_1$, $\Gamma_2$ i $\Gamma_8$.
	Wtedy przy dobrym wyborze, który okrąg styczny wziąć, okrąg $\Gamma_4$ jest też styczny do $\Gamma_2$, $\Gamma_3$ i $\Gamma_9$.
\end{proposition}

\subsubsection{Inversive?}
łańcuchy Steinera % https://en.wikipedia.org/wiki/Steiner_chain The method of circle inversion is helpful in treating Steiner chains. => Audin Audin \cite[s. 109]{audin_2003} używa porism

formuła Kartezjusza % https://en.wikipedia.org/wiki/Descartes%27_theorem
miana odległości przy inwersji, zmiana promienia okręgu przy inwersji, % https://en.wikipedia.org/wiki/Inversive_geometry#Relation_to_Erlangen_program

% pole: https://math.stackexchange.com/questions/2715752/how-was-the-area-formula-for-a-circle-a-pi-r2-derived-before-the-introdu