\begin{proposition}[okrąg opisany na czworokącie]
	\label{prp_incircle}
	Niech $A$, $B$, $C$, $D$ będą czterema punktami na płaszczyźnie takimi, że $A$ i $B$ leżą po tej samej stronie prostej $CD$.
	Wtedy następujące warunki są równoważne: punkty $A$, $B$, $C$, $D$ leżą na jednym okręgu; kąty $\angle DAC$ i $\angle DBC$ są sobie równe; suma dwóch przeciwległych kątów czworokąta $ABCD$ ma miarę kąta półpełnego.
\end{proposition}

Jedna ze wspomnianych implikacji to wniosek \ref{ab_twice_pi}.

\begin{proposition}[okrąg wpisany w czworokąt]
	\label{prp_excircle}
	Niech $A$, $B$, $C$, $D$ będą czterema punktami na płaszczyźnie takimi, że...
	\todofoot{Dokończyć okręgi wpisane}
\end{proposition}

\begin{proposition}
	Niech $\Gamma$ będzie okręgiem opisanym na czworokącie $ABCD$.
	Niech $\Gamma_1$, $\Gamma_2$, $\Gamma_3$, $\Gamma_4$ będą dowolnymi okręgami, które przechodzą przez $AB$, $BC$, $CD$, $DA$.
	Wtedy ich cztery nowe punkty przecięcia tworzą czworokąt cykliczny.
\end{proposition}

Styczna do okręgu, okrąg wpisany w kąt.
Okrąg wpisany w trójkąt, okręgi dopisane do trójkąta.
Warunki istnienia okręgu stycznego do czterech prostych.

Euklides (III.35):

% https://en.wikipedia.org/wiki/Intersecting_chords_theorem
\begin{proposition}[twierdzenie o cięciwach]
    \label{prop_intersecting_chords}
	Niech punkty $A$, $B$ leżą na pewnej prostej przechodzącej przez punkt $S$, zaś punkty $C$, $D$ leżą na innej prostej przez ten punkt.
	Wtedy cztery punkty $A$, $B$, $C$, $D$ leżą na jednym okręgu wtedy i tylko wtedy, gdy:
	\begin{equation}
		|AS| \cdot |BS| = |CS| \cdot |DS|.
	\end{equation}
\end{proposition}

Po angielsku nazywa się to ,,\emph{intersecting chords theorem}''; Bogdańska i Neugebauer nazywają to nie wiedzieć czemu ,,potęgowym kryterium współokręgowości''.
Ech.
Elementarny dowód korzysta z~podobieństwa trójkątów $\triangle ASD$ i $\triangle BSC$.
Nieelementarny zauważa, że obydwa iloczyny są równe (poza znakiem) potędze punktu $S$ względem okręgu przez punkty $A$, $B$, $C$, $D$.
Definicję \ref{def_power_point} potęgi punktu podamy już za kilka stron.

\begin{proposition}[twierdzenie o siecznych i stycznych]
	Jeżeli... (Neugebauer s. 66)
\end{proposition}

\todofoot{twierdzenie o motylku} % https://en.wikipedia.org/wiki/Butterfly_theorem