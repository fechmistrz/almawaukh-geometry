
Okrąg wpisany, opisany, dopisany: Audin \cite[s. 98]{audin_2003}.
\index{okrąg!wpisany}
\index{okrąg!opisany}
\index{okrąg!dopisany}

Są jeszcze trzy inne okręgi styczne do wszystkich trzech prostych, na których leżą boki trójkąta.
Nazywamy je okręgami dopisanymi.
\index{okrąg dopisany}

Styczna do okręgu, okrąg wpisany w kąt.
\index{styczność}
Styczne są równej długości.
Dwustyczne (do dwóch okręgów).
\index{dwustyczna}
Okrąg wpisany w trójkąt, okręgi dopisane do trójkąta.
Warunki istnienia okręgu stycznego do czterech prostych.

Euklides (III.35):

% https://en.wikipedia.org/wiki/Intersecting_chords_theorem
\begin{proposition}[twierdzenie o cięciwach]
    \label{prop_intersecting_chords}
	Niech punkty $A$, $B$ leżą na pewnej prostej przechodzącej przez punkt $S$, zaś punkty $C$, $D$ leżą na innej prostej przez ten punkt.
	Wtedy cztery punkty $A$, $B$, $C$, $D$ leżą na jednym okręgu wtedy i tylko wtedy, gdy:
	\begin{equation}
		|AS| \cdot |BS| = |CS| \cdot |DS|.
	\end{equation}
	\index{cięciwa}%
	\index{twierdzenie!o cięciwach}%
\end{proposition}

Po angielsku nazywa się to ,,\emph{intersecting chords theorem}''; Bogdańska i Neugebauer nazywają to nie wiedzieć czemu ,,potęgowym kryterium współokręgowości''.
Ech.
Elementarny dowód korzysta z~podobieństwa trójkątów $\triangle ASD$ i $\triangle BSC$.
Nieelementarny zauważa, że obydwa iloczyny są równe (poza znakiem) potędze punktu $S$ względem okręgu przez punkty $A$, $B$, $C$, $D$.
Definicję \ref{def_power_point} potęgi punktu podamy już za kilka stron.
\index{potęga punktu}%

\begin{proposition}[twierdzenie o siecznych i stycznych]
	Jeżeli... (Neugebauer s. 66)
	\index{twierdzenie!o siecznych i stycznych}
\end{proposition}

\todofoot{twierdzenie o motylku} % https://en.wikipedia.org/wiki/Butterfly_theorem
\index{twierdzenie!o motylku}