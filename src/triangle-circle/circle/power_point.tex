\index{potęga punktu względem okręgu} % zamienić na zakres stron
\todofoot{Coxeter s. 85, coaxial circles}

Potęgę punktu wprowadził Jakob Steiner\footnote{Praca \emph{Einige geometrische Betrachtungen}, strona 164} w 1826 roku.
\index[persons]{Steiner, Jakob}%
Użył jej, aby znaleźć okrąg przecinający cztery dane okręgi pod tym samym kątem; rozwiązać problem Apolloniusza; skonstruować okręgi Malfattiego (problem \ref{malfatti_problem}: trzy okręgi, które są styczne do pozostałych dwóch oraz boków zadanego trójkąta).
% TODO: index, ref
\index{problem Apolloniusza}%
\index{okręgi Malfattiego}%
Wydaje się, że ta lista jest niekompletna.

Wśród elementarnych zastosowań można wskazać dowód twierdzenia o przecinających się cięciwach (fakt \ref{prop_intersecting_chords})

\begin{definition}[potęga punktu względem okręgu]
	\label{def_power_point}
	Dane są okrąg $\Gamma$ o środku $O$ i promieniu $r$ oraz dowolny punkt $A$.
	Liczbę rzeczywistą
	\begin{equation}
		\Pi(A) := |OA|^2 - r^2
	\end{equation}
	nazywamy potęgą punktu $A$ względem okręgu $\Gamma$.
\end{definition}

Twierdzenie Pitagorasa dostarcza prostej interpretacji wielkości $\Pi(A)$, gdy punkt $A$ leży na zewnątrz okręgu: jest to długość stycznej do okręgu, która przechodzi przez punkt.

\begin{proposition}
	Środki odcinków wspólnych stycznych do dwóch okręgów leżą na jednej prostej. % Neugebauer s. 70
\end{proposition}

Nie wiemy, jak trudny jest dowód powyższego bez poniższego.

\begin{proposition}[oś potęgowa istnieje]
\label{guzicki_6_11}%
    Dane są dwa niewspółśrodkowe okręgi $\Gamma_1(O_1, r_1)$ oraz $\Gamma_2(O_2, r_2)$.
    Wtedy miejscem geometrycznym punktów $P$ mających równe potęgi względem obu okręgów:
	\begin{equation}
		\{S : \Pi(S, \Gamma_1) = \Pi(S, \Gamma_2)\}
	\end{equation}
	jest prosta prostopadła do prostej przechodzącej przez środki obu okręgów.
	Nazywamy ją \emph{osią potęgową} okręgów $\Gamma_1$, $\Gamma_2$.
\end{proposition}

O osi potęgowej piszą Bogdańska, Neugebauer \cite[s. 69]{neugebauer_2018}; Guzicki \cite[s. 173, 174]{guzicki_2021}.
Oś potęgowa okręgów stycznych to prosta styczna do obydwu okręgów; oś potęgowa okręgów, które się przecinają w dwóch punktach, to prosta przez punkty przecięcia.

\begin{proposition}[środek potęgowy]
	Dane są trzy parami niewspółśrodkowe okręgi $\Gamma_1, \Gamma_2, \Gamma_3$.
	Wtedy albo środki tych okręgów są współliniowe (i trzy osie potęgowe każdej pary są równoległe), albo wszystkie trzy osie przecinają się w~jednym punkcie: \emph{środku potęgowym} okręgów.
\end{proposition}

Nie jest jasne, dlaczego niektórzy nazywają to twierdzeniem Monge'a (jak Bogdańska, Neugebauer \cite[s. ???]{neugebauer_2018}) albo też rozwiązaniem problemu Monge'a (jak Dörrie \cite[s. 151]{dorrie_1965}).
Ale jest jasne, że pozwala łatwo skonstruować oś potęgową dwóch okręgów: wystarczy przeciąć je jednocześnie trzecim okręgiem (na dwa różne sposoby).

Patrz Guzicki \cite[s. 174]{guzicki_2021}.

% Neugebauer s. 71
\begin{proposition}
	Ortocentrum trójkąta jest środkiem potęgowym rodziny wszystkich okręgów o średnicach będących czewianami (odcinkami łączącymi wierzchołek z przeciwległym bokiem).
\end{proposition}

\begin{theorem}[Auberta]
	Dane są cztery proste w położeniu ogólnym (żadne trzy nie przechodzą przez jeden punkt, żadne dwie nie są równoległe).
	Wówczas ortocentra czterech trójkątów wyznaczonych przez trójki tych prostych leża na jednej prostej, zwanej prostą Auberta.
\end{theorem}

Nie udało nam się ustalić, kim był Aubert, ale najprawdopodobniej udowodnił to twierdzenie w 1899 roku.
Prosta Auberta znana jest czasami jako prosta Steinera.
% TODO: https://el-m-wikipedia-org.translate.goog/wiki/Ευθεία_Σίμσον_(τετράπλευρο)?_x_tr_sl=auto&_x_tr_tl=pl&_x_tr_hl=pl&_x_tr_pto=wapp
% https://el-m-wikipedia-org.translate.goog/wiki/Θεώρημα_Miquel_(τετράπλευρο)?_x_tr_sl=auto&_x_tr_tl=pl&_x_tr_hl=pl&_x_tr_pto=wapp
% https://el-m-wikipedia-org.translate.goog/wiki/Ευθεία_Νεύτωνα-Γκάους?_x_tr_sl=auto&_x_tr_tl=pl&_x_tr_hl=pl&_x_tr_pto=wapp

