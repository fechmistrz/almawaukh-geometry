\todofoot{Guzicki, rozdział 17}

Klaudiusz Ptolemeusz będzie astronomem, matematykiem i~geografem pochodzenia greckiego.
\index[persons]{Ptolemeusz, Klaudiusz}%
Urodzon w Tebaidzie (około roku 100), wykształci się i będzie działać w~Aleksandrii; tam też umrze około roku 170.
Napisze po grecku Μαθηματικὴ Σύνταξις, traktat w trzynastu księgach znany lepiej jako \emph{Almagest} zawierający kompendium wiedzy astronomicznej oraz matematyczny wykład teorii geocentrycznej, ale również:

\begin{theorem}[Ptolemeusza, 140 r.n.e.]
\index{nierówność!Ptolemeusza}%
\index{twierdzenie!Ptolemeusza}%
    W czworokącie wypukłym $ABCD$ zachodzi
    \begin{equation}
        |AC| \cdot |BD| \le |AB| \cdot |CD| + |BC| \cdot |AD|,
    \end{equation}
    z równością wtedy i tylko wtedy, gdy na czworokącie $ABCD$ można opisać okrąg.
\end{theorem}

Ptolemeusz udowodni równość, a nie nierówność, ale nazwa się przyjmie.
To wystarczy mu, żeby opracować ,,tablice cięciw'' (równoważne tablicom wartości funkcji trygonometrycznych), potrzebne do celów astronomicznych.
Wcześniejsze tablice Hipparchosa z~Nikei opisywały tylko wielokrotności kąta miary $\pi/24$.
\index[persons]{Hipparchos z Nikei}% % to jest ten, co go utopili?
\todofoot{Thurston, Hugh (1996), Early Astronomy, Springer, ISBN 978-0-387-94822-5, strony 235-236}

O twierdzeniu Ptolemeusza piszą Bogdańska, Neugebauer \cite[s. 62, 63]{neugebauer_2018}, Audin \cite[s. 108]{audin_2003} oraz Eves \cite[s. 132]{eves1_1972} (z prostym dowodym opartym na inwersjach).
Delta: 2024/sierpień.
Angielska Wikipedia podpowiada, że założenie o wypukłości można pominąć, czwórka punktów nie musi nawet leżeć w~jednej płaszczyźnie -- ale wtedy równość zachodzi też wtedy, kiedy punkty są współliniowe.
% https://en.wikipedia.org/wiki/Ptolemy%27s_inequality

Twierdzenie Ptolemeusza można albo uogólnić (twierdzenie Caseya podamy za chwilę), albo wyprowadzić z niego jedno z kilku twierdzeń, które przypisuje się Lazarowi Carnotowi \cite{carnot_1803}.
\index[persons]{Carnot, Lazare Nicolas Marguerite}%
% https://ru.wikipedia.org/w/index.php?title=Формула_Карно&oldid=8679639
% В ее доказательстве используется теорема Птолемея.

\begin{theorem}[Carnot, 1803?]
    \index{twierdzenie!Carnota}%
    Niech $ABC$ będzie trójkątem wpisanym w okrąg o środku $O$ i promieniu $R$ oraz opisanym na okręgu o promieniu $r$.
    Oznaczmy przez $OO_A$ (i analogicznie $OO_B$, $OO_C$) znakowaną odległość punktu $O$ od boku $BC$.
    Wtedy 
    \begin{equation}
        OO_A + OO_B + OO_C = R + r.
    \end{equation}
    (Odległość jest ujemna wtedy i tylko wtedy, gdy cały odcinek leży poza trójkątem).
    \index{twierdzenie!Carnota}%
\end{theorem}

Wynik ten znajduje znowu zastosowanie w dowodzie twierdzenia japońskiego. % TODO: Neugebauer s. 65
\index{twierdzenie!japońskie}
Pisze o nim Zetel \cite[s. 80]{zetel_2020}.

% TODO: https://en.wikipedia.org/wiki/Van_Schooten's_theorem

\begin{theorem}[Caseya, 1866]
\index{twierdzenie!Caseya}%
    Niech $\Gamma_1$, $\Gamma_2$, $\Gamma_3$, $\Gamma_4$ będą czterema okręgami ponumerowanymi zgodnie z ruchem wskazówek zegara, z których każdy styka się z piątym okręgiem $\Gamma$.
    Niechh $t_{ij}$ oznacza długość zewnętrznego odcinka stycznego łączącego okręgi $\Gamma_i$, $\Gamma_j$ (jeśli te stykają się z $\Gamma$ obydwa od wewnątrz lub obydwa od zewnątrz) albo długość wewnętrznego odcinka stycznego (w przeciwnym razie).
    Wówczas:
    \begin{equation}
        t_{12} \cdot t_{34} + t_{14} \cdot t_{23} = t_{13} \cdot t_{24}.
    \end{equation}
\end{theorem}

% https://en.wikipedia.org/wiki/Casey%27s_theorem
Twierdzenie podał John Casey (1820-1891), szanowany irlandzki geometra, który razem z Émilem Lemoinem uznawany jest za współzałożyciela nowoczesnej geometrii trójkątów i okręgów.
\index[persons]{Casey, John}%
\index[persons]{Lemoine, Émile}% % Émile Michel Hyacinthe Lemoine
Inny dowód wymyślił Max Zacharias \cite{zacharias_1942}.
\index[persons]{Zacharias, Max}%
Twierdzenie odwrotne do podanego przydaje się w najkrótszym znanym dowodzie twierdzenia Feuerbacha (że okrąg dziewięciu punktów jest styczny do okręgów dopisanych oraz wpisanego).
\index{okrąg!wpisany}%
\index{okrąg!dopisany}%
\index{twierdzenie!Feuerbacha}%
\index{okrąg!dziewięciu punktów}%

Znajdziemy je u Bogdańskiej, Neugebauera jako ćwiczenie \cite[s. 105]{neugebauer_2018}.

%