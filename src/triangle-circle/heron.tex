%

\index{wzór!Herona|(}
Guzicki \cite[s. 165-168]{guzicki_2021} wyprowadza wzór Herona z twierdzenia Pitagorasa.
\index{twierdzenie!Pitagorasa}
Oryginalny dowód Herona był dość skomplikowany, Guzicki \cite[s. 168-169]{guzicki_2021} wspomina o znacznie prostszym dowodzie geometrycznym, pochodzącym od Eulera.
\index[persons]{Euler, Leonhard}%
(Chociaż wynik przypisujemy obecnie Heronowi, został odkryty przez Archimedesa -- Coxeter \cite[s. 12]{coxeter_1991} odsyła do van der Waerdena \cite{MISSING_CITATION}).
% This remarkable expression, which we shall use in § 18.4, is attributed to Heron of Alexandria (about 60 a.d.), but it was really discovered by Archimedes. (See B. L. van der Waerden, Science Awakening, Oxford University Press, New York, 1961, pp. 228, 277.) 

\begin{proposition}
	Niech $ABC$ będzie trójkątem o obwodzie $2p$ oraz polu powierzchni $S$.
	Wtedy
	\begin{equation}
		S \le \frac{p^2}{3 \sqrt{3}}
	\end{equation}
\end{proposition}

Guzicki wyprowadza tę nierówność izoperymetryczną ze wzoru Herona oraz nierówności między średnią arytmetyczną i geometryczną.

% TODO: Coxeter - Introduction to Geometry, s. 12

\index{wzór!Herona|)}

% TODO: wzór Herona (Guzicki-6), Brahmagupty

%