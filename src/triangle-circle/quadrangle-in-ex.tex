\begin{proposition}[okrąg opisany na czworokącie]
\index{okrąg!opisany}%
\label{prp_incircle}
	Niech $A$, $B$, $C$, $D$ będą czterema punktami na płaszczyźnie takimi, że $A$ i $B$ leżą po tej samej stronie prostej $CD$.
	Wtedy następujące warunki są równoważne:
	\begin{itemize}
		\item punkty $A$, $B$, $C$, $D$ leżą na jednym okręgu;
		\item kąty $\angle ACB$ i $\angle ADB$ są sobie równe;
		\item suma dwóch przeciwległych kątów czworokąta wypukłego wyznaczonego przez wierzchłki $A$, $B$, $C$, $D$ ma miarę kąta półpełnego.
	\end{itemize}
\end{proposition}

Jedna ze wspomnianych implikacji to wniosek \ref{ab_twice_pi}.
Piszą o tym Guzicki \cite[s. 11-13, 16, 17]{guzicki_2021}.

\begin{proposition}[okrąg wpisany w czworokąt]
	\label{prp_excircle}
	Niech $A$, $B$, $C$, $D$ będą czterema punktami na płaszczyźnie takimi, że...
	\todofoot{Dokończyć okręgi wpisane}
	\index{okrąg!wpisany}%
\end{proposition}

\begin{proposition}
	Niech $\Gamma$ będzie okręgiem opisanym na czworokącie $ABCD$.
	Niech $\Gamma_1$, $\Gamma_2$, $\Gamma_3$, $\Gamma_4$ będą dowolnymi okręgami, które przechodzą przez $AB$, $BC$, $CD$, $DA$.
	Wtedy ich cztery nowe punkty przecięcia tworzą czworokąt cykliczny.
	\index{czworokąt!cykliczny}%
	% https://en.wikipedia.org/wiki/Cyclic_quadrilateral
\end{proposition}

% \subsection{Twierdzenie Miquela}
Twierdzenie Miquela
\index{twierdzenie!Miquela}%
\loremipsum
\todofoot{artykuł na en-wiki ,,Miquel's theorem''} % https://en.wikipedia.org/wiki/Miquel%27s_theorem

Hartshorne jako ćwiczenie \cite[s. 61]{hartshorne2000} pisze, że tym razem punkt Miquela został nazwany na cześć osoby, która go odkryła w 1838 roku.
\index{punkt!Miquela}%
\todofoot{Guzicki ps. 29, 32}
Audin \cite[s. 104]{audin_2003} jako ,,the pivot'' (dla niego twierdzenie Miquela mówi o czterech okręgach)
% w en-wiki Miquels' theorem to jest six citcle ^^^