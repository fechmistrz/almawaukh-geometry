%

\begin{definition}[czworokąt]
    Niech $A$, $B$, $C$, $D$ będą czterema punktami, z których żadne trzy nie są współliniowe, takimi że odcinki $AB$, $BC$, $CD$, $DA$ nie mają części wspólnej poza końcami.
    Wtedy sumę tych odcinków nazywamy czworokątem.
\end{definition}


Quadri (Latin for 4) + latus (side). Tetragon = tetra (grecki 4) + gon (corner, angle), like polygon, pentagon. Quadrangle.Prosty (bez samoprzeciec) jest wypukly lub wklesly, albo z samoprzecieciami. Mucha albo motyl.

Irregular quadrilateral (GBr) lub trapezium (NA) nie ma pary bokow rownoleglych (kiedys nazywal sie trapezoidem w GBr)

Trapez posiada jedna (lub dwie) pary bokow rownoleglych (UK trapezium, US trapezoid). Kazdy rownoleglobok jest trapezem.Trapez, ktory posiada os symetrii, ktora jest tez symetralna dwoch przeciwleglych bokow, nazywamy rownoramiennym. (!!! Please do NOT define an isosceles trapezoid as having legs equal. Doing so would make all parallelograms isosceles trapezoids, which we know is wrong. )

Rownoleglobok posiada dwie pary bokow rownoleglych, albo przeciwlegle boki rownej dlugosci, albo przeciwlegle katy rownej miary, albo przekatne polowia sie wzajemnie. Rownolegloboki obejmuje romby, romboidy. (Parallelogram)

Romby (rhombus, rhomb) posiadaja cztery boki rownej dlugosci, albo prostopadle przekatne polowiace sie.

Romboid to rownoleglobok, ktory nie jest rombem, bo posiada boki roznej dlugosci. Niektorzy dodaja, ze musi miec katy roznej miary, wykluczajac w ten sposob prostokaty. Rzadko uzywana klasa.

Prostokat ma cztery katy proste, albo przekatne rownej dlugosci polowiace sie. Wsrod prostokatow wyrozniamy kwadraty (ktore maja wszystkie boki tej samej dlugosci) oraz oblongi (ktore nie). Kwadraty to dokladnie prostokaty, ktore sa tez rombami; maja rowne boki o katy.

Kite ma dwie pary sasiednich bokow rownej dlugosci, wiec jedna z przekatnych dzieli go na przystajace trojkaty. Wynika stad, ze przekatne sa prostopadle. Kite obejmuje romby. Kite prosty to taki, ktory ma dwa katy proste naprzeciw siebie, mozna na takim opisac kolo. (HJEMLSJEV!).

Czworokat cykliczny to taki, ktory...

---

A dart (or arrowhead) is a concave quadrilateral with bilateral symmetry like a kite, but where one interior angle is reflex. See Kite.
A self-intersecting quadrilateral is called variously a cross-quadrilateral, crossed quadrilateral, butterfly quadrilateral or bow-tie quadrilateral. In a crossed quadrilateral, the four "interior" angles on either side of the crossing (two acute and two reflex, all on the left or all on the right as the figure is traced out) add up to 720°.

% https://en.wikipedia.org/wiki/Rectangle

To jest definicja Hartshorne'a \cite[s. 80]{hartshorne2000}.

Pokaż, że przekątne rombu rozcinają go na cztery przystające trójkąty prostokątne. % romb cztery te same boki

Pokaż, że przekątne prostokąta są równej długości i dzielą się na połowy. % prostokąt cztery kąty proste

\subsection{Opisane, wpisane}
\begin{proposition}[okrąg opisany na czworokącie]
	\label{prp_incircle}
	Niech $A$, $B$, $C$, $D$ będą czterema punktami na płaszczyźnie takimi, że $A$ i $B$ leżą po tej samej stronie prostej $CD$.
	Wtedy następujące warunki są równoważne: punkty $A$, $B$, $C$, $D$ leżą na jednym okręgu; kąty $\angle DAC$ i $\angle DBC$ są sobie równe; suma dwóch przeciwległych kątów czworokąta $ABCD$ ma miarę kąta półpełnego.
\end{proposition}

Jedna ze wspomnianych implikacji to wniosek \ref{ab_twice_pi}.

\begin{proposition}[okrąg wpisany w czworokąt]
	\label{prp_excircle}
	Niech $A$, $B$, $C$, $D$ będą czterema punktami na płaszczyźnie takimi, że...
	\todofoot{Dokończyć okręgi wpisane}
\end{proposition}

\begin{proposition}
	Niech $\Gamma$ będzie okręgiem opisanym na czworokącie $ABCD$.
	Niech $\Gamma_1$, $\Gamma_2$, $\Gamma_3$, $\Gamma_4$ będą dowolnymi okręgami, które przechodzą przez $AB$, $BC$, $CD$, $DA$.
	Wtedy ich cztery nowe punkty przecięcia tworzą czworokąt cykliczny.
\end{proposition}

% \subsection{Twierdzenie Miquela}
Twierdzenie Miquela
\loremipsum
\todofoot{artykuł na en-wiki ,,Miquel's theorem''} % https://en.wikipedia.org/wiki/Miquel%27s_theorem


Hartshorne jako ćwiczenie \cite[s. 61]{hartshorne2000} pisze, że tym razem punkt Miquela został nazwany na cześć osoby, która go odkryła w 1838 roku.
\todofoot{Guzicki ps. 29, 32}
Audin \cite[s. 104]{audin_2003} jako ,,the pivot'' (dla niego twierdzenie Miquela mówi o czterech okręgach)
% w en-wiki Miquels' theorem to jest six citcle ^^^

\subsection{Czoworokąty dwuśrodkowe}
\input{triangle-circle/quadrangle-bicentric}

\subsection{Czoworobok zupełny}

\begin{proposition}
	Środki trzech przekątnych czworoboku zupełnego leżą na jednej prostej, zwaną prostą Newtona-Gaussa.
	\todofoot{Twierdzenie Newtona: środek okręgu ego w czworokąt i środki przekątnych tego czworokąta są współliniowe.}
	\todofoot{Twierdzenie Gaussa: środki przekątnych czworokąta zupełnego są współliniowe.}
\end{proposition}

\todofoot{twierdzenie Gaussa-Bodenmillera}

\todofoot{Jemieljanow: punkt Miquela właściwego czworoboku zupełnego leży na okręgu dziewięciu punktów trójkąta przekątnego tego czworoboku.}
\todofoot{Neugebauer 262: w każdy właściwy czworobok zupełny da się wpisać dokładnie jedną parabolę, jej ogniskiem jest punkt Miquela czworoboku.}

