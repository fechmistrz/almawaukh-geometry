%

\begin{definition}[czworokąt]
    Niech $A$, $B$, $C$, $D$ będą czterema punktami, z których żadne trzy nie są współliniowe, takimi że odcinki $AB$, $BC$, $CD$, $DA$ nie mają części wspólnej poza końcami.
    Wtedy sumę tych odcinków nazywamy czworokątem.
	\index{czworokąt}
\end{definition}

% https://en.wikipedia.org/wiki/Square
% https://en.wikipedia.org/wiki/Rhomboid

To jest definicja Hartshorne'a \cite[s. 80]{hartshorne2000}.

Po angielsku \emph{quadrangle}, z łacińskiego \emph{quadri} określającego liczbę cztery oraz \emph{latus}, czyli boku (a więc technicznie czworobok, nie czworokąt); istnieje też rzadko używane słowo \emph{tetragon} (przez analogię do \emph{polygon, pentagon}) z greckiego \emph{tetra} oraz \emph{gon}, znaczącego tym razem kąt.
Niektórzy dopuszczają czworokąty z samoprzecięciem, nazywając je muchą albo motylem.
\index{motyl}
\index{mucha}

Czworokąty, które nie mają samoprzecięć, mogą być wypukłe albo wklęsłe.
Wypukłe tworzą bogatszą rodzinę.
W krajach anglosaskich wyróżnia się czworokąty nieregularne, które nie mają żadnej pary boków równoległych (\emph{trapezium} w amerykańskiej odmianie języka).

\begin{definition}[trapez]
	Czworokąt, który posiada jedną lub dwie pary boków równoległych nazywamy trapezem.
	\index{trapez}
\end{definition}

Po brytyjsku \emph{trapezium}, po amerykańsku \emph{trapezoid} -- widać więc kolizję z terminem na czworokąt ,,w ogólnym położeniu''.
Wyjaśnimy to po podaniu pozostałych definicji. % https://en.wikipedia.org/wiki/Trapezoid

\begin{definition}[trapez równoramienny]
	Czworokąt, który posiada parę boków równoległych, którego kąty przy podstawie są równej miary, nazywamy trapezem równoramiennym.
	\index{trapez!równoramienny}%
\end{definition}

W szczególności, nie można definiować trapezu równoramiennego jako trapezu, który ma ramiona równej długości!
To czyniłoby wszystkie równoległoboki trapezami równoramiennymi, co nie ma żadnych zalet.
\index{równoległobok}%
Inne, równoważne definicje mówią, że jest to czworokąt z osią symetrii, która połowi parę przeciwległych boków albo trapez, który ma przekątne równej długości.

\begin{definition}[równoległobok]
	Czworokąt, który posiada dwie pary boków równoległych, nazywamy równoległobokiem.
	\index{równoległobok}
\end{definition}

Każdy równoległobok (po angielsku \emph{parallelogram}) jest też trapezem.
Alternatywne definicje głoszą, że ma przeciwległe boki równej długości, albo równe przeciwległe kąty, albo połowiące się wzajemnie przekątne.

\begin{definition}[romb]
	Czworokąt, który posiada cztery boki tej samej długości, nazywamy rombem.
	\index{romb}
\end{definition}

Równoważnie, romby to te czworokąty, których prostopadłe przekątne połowią się i rozcinają romb na cztery przystające trójkąty (prostokątne).
Angielski nazywa je \emph{rhombus}.

\begin{definition}[prostokąt]
	Czworokąt, który posiada cztery kąty tej samej miary, nazywamy prostokątem.
	\index{prostokąt}%
\end{definition}

Nazwa bierze się z tego, że wszystkie jego kąty są proste.
Alternatywnie, prostokąty to dokładnie te czworokąty, których przekątne równej długości połowią się.
Szczególnymi prostokątami są

\begin{definition}[kwadrat]
	Czworokąt, który posiada cztery kąty tej samej miary i wszystkie boki tej samej długości, nazywamy kwadratem.
	\index{kwadrat}
\end{definition}

Kwadraty (\emph{squares}) to dokładnie te prostokąty (\emph{rectangles}), które są też rombami.

\begin{definition}[latawiec]
	Czworokąt, który ma dwie pary sąsiednich boków równej długości, nazywamy latawcem.
	\index{latawiec}
	\index{deltoid|see{latawiec}}
\end{definition}

W Polsce bardziej popularne będzie określenie deltoid.
Latawce mają jedną przekątną, która dzieli je na dwa przystające trójkąty, w szczególności jego przekątne są prostopadłe.
Na latawcu, który ma dwa kąty proste naprzeciw siebie można opisać okrąg.
Romby są szczególnym przypadkiem latawców.

Czworokąty wklęsłe mają jeden kąt, którego miara wynosi między $\pi$ i $2\pi$ radianów.
Wyróżniamy wśród nich rzutki (z angielskiego \emph{dart, arrowhead}), które mają oś symetrii albo równoważnie, dwie pary sąsiednich boków równej długości, tak jak latawce.
\index{rzutka}

W starożytnej Grecji czworokąty podzieli się na rozłączne klasy: kwadraty, podłużne prostokąty (po angielsku \emph{oblong}), niekwadratowe romby, romboidy (równoległoboki, które nie są ani rombami, ani prostokątami) oraz \emph{trapezia} (τραπέζιον znaczy dosłownie stolik), które nie pasowały do wcześniej wymienionych.
Filozof neoplatoński Proklos zwany Diadochem napisze wpływowy komentarz do prac Euklidesa, gdzie wprowadzi trochę inny podział zgodnie z tym, co obmyśli około 100 lat p.n.e. Posejdonios z Rodos (nauczyciel Cycerona).
Czworokąt będzie równoległobokiem (kwadratem, podłużnym prostokątem, rombem, romboidem) albo nie (z parą boków równoległych, jako trapezium równoramienne lub ukośne albo bez niej -- trapezoidem).
Wszystkie języki europejskie odziedziczą klasyfikację czworokątów po Proklosie... poza angielskim, gdzie znaczenie trapezoidu i trapezium odwróci słownik wydany przez Charlesa Huttona w 1795 roku.
Porządek przywróci się około 1875 roku, ale amerykańskiego angielskiego już nikt nie uratuje.

\subsection{Opisane, wpisane}

Na rombie, który nie jest kwadratem, nie można opisać okręgu.

\begin{proposition}[okrąg opisany na czworokącie]
\index{okrąg!opisany}%
\label{prp_incircle}
	Niech $A$, $B$, $C$, $D$ będą czterema punktami na płaszczyźnie takimi, że $A$ i $B$ leżą po tej samej stronie prostej $CD$.
	Wtedy następujące warunki są równoważne:
	\begin{itemize}
		\item punkty $A$, $B$, $C$, $D$ leżą na jednym okręgu;
		\item kąty $\angle ACB$ i $\angle ADB$ są sobie równe;
		\item suma dwóch przeciwległych kątów czworokąta wypukłego wyznaczonego przez wierzchłki $A$, $B$, $C$, $D$ ma miarę kąta półpełnego.
	\end{itemize}
\end{proposition}

Jedna ze wspomnianych implikacji to wniosek \ref{ab_twice_pi}.
Piszą o tym Guzicki \cite[s. 11-13, 16, 17]{guzicki_2021}.

W prostokąt, który nie jest kwadratem, nie można wpisać okręgu.

\begin{proposition}[okrąg wpisany w czworokąt]
	\label{prp_excircle}
	Dany jest czworokąt wypukły $ABCD$.
	Wtedy następujące warunki są równoważne:
	\begin{itemize}
		\item w czworokąt $ABCD$ można wpisać okrąg,
		\item $AB + CD = AD + BC$.
	\end{itemize}
	\index{okrąg!wpisany}%
\end{proposition}

Piszą o tym Guzicki \cite[s. 231-237]{guzicki_2021}.

\begin{proposition}[twierdzenie Newtona?]
	Dany jest czworokąt $ABCD$, w który wpisano okrąg, styczny do boków $AB$, $BC$, $CD$, $DA$ w punktach $K$, $L$, $M$, $N$.
	Wówczas proste $AC$, $BD$, $KM$ i $LN$ są współpękowe.
\end{proposition}

Piszą o tym Guzicki \cite[s. 237, 238]{guzicki_2021}, że jest to szczególny przypadek twierdzenia Brianchona, którego dowodzi najpierw dla sześciokąta.

\begin{proposition}[twierdzenie Miquela]
	Niech $ABC$ będzie trójkątem, na bokach którego ($AB$, $BC$, $AC$) wybrano punkty $C'$, $A'$, $B'$.
	Wtedy okręgi opisane na trójkątach $AB'C'$, $A'BC'$, $A'B'C$ przecinają się w~jednym punkcie.
\end{proposition}

Sformułowanie mówi tylko o trójkątach, ale dowód wykorzystuje własności czworokątów cyklicznych.
Hartshorne jako ćwiczenie \cite[s. 61]{hartshorne2000} napisze, że tym razem punkt Miquela zostanie nazwany na cześć osoby, która go odkryje w 1838 roku.
\index{punkt!Miquela}%
\todofoot{Guzicki ps. 29, 32}
Audin \cite[s. 104]{audin_2003} jako ,,the pivot'' (dla niego twierdzenie Miquela mówi o czterech okręgach, czyli nasz fakt \ref{miquel6}).

\begin{proposition}[twierdzenie Miquela o pięciokącie]
	Niech $ABCDE$ będzie pięciokątem wypukłym, którego przedłużenia boków przecinają się w punktach $F$, $G$, $H$, $I$, $K$.
	Na trójkątach $CFD$, $DGE$, $EHA$, $AIB$, $BKC$ opisano okręgi.
	Wtedy ich nowe punkty przecięcia (różne od $A$, $B$, $C$, $D$, $E$) leżą na jednym okręgu. 
\end{proposition}

% Miquel, A. "Mémoire de Géométrie." J. de mathématiques pures et appliquées de Liouville 1, 485-487, 1838. ???

\begin{proposition}[twierdzenie Miquela o sześciu okręgach]
\label{miquel6}%
	Niech $\Gamma$ będzie okręgiem opisanym na czworokącie $ABCD$.
	Niech $\Gamma_1$, $\Gamma_2$, $\Gamma_3$, $\Gamma_4$ będą dowolnymi okręgami, które przechodzą przez $AB$, $BC$, $CD$, $DA$.
	Wtedy ich cztery nowe punkty przecięcia tworzą czworokąt cykliczny.
	\index{czworokąt!cykliczny}%
	% https://en.wikipedia.org/wiki/Miquel%27s_theorem#Miquel's_six_circle_theorem
	% It is also known as the four circles theorem and while generally attributed to Jakob Steiner the only known published proof was given by Miquel.[11]
	% Ostermann, Alexander; Wanner, Gerhard (2012), Geometry by its History, Springer, ISBN 978-3-642-29162-3 STRONA=352
\end{proposition}

\index{twierdzenie!Miquela}%

\subsection{Czoworokąty dwuśrodkowe}
\begin{definition}
	Czworokąt, który jest jednocześnie wpisany w pewien okrąg i opisany na innym okręgu, nazywamy dwuśrodkowym.
	\index{czworokąt!dwuśrodkowy}%
\end{definition}

Przykładami takich czworokątów są kwadraty, prostokątne latawce i niektóre równoramienne trapezy.
\index{kwadrat}%
\index{latawiec}%
\index{trapez!równoramienny}%
Ich pełną charakteryzację można uzyskać przez połączenie warunków \ref{prp_incircle} oraz \ref{prp_excircle}.
Ale mamy też inne opisy.

\begin{proposition}
	Niech $ABCD$ będzie czworokątem opisanym na okręgu $\Gamma$, który dotyka go w punktach $W$ (na odcinku $AB$), $X$ (na $BC$), $Y$ (na $CD$), $Z$ (na $AD$).
	Wtedy następujące warunki są równoważne:
	\begin{itemize}
		\item na czworokącie $ABCD$ można opisać okrąg,
		\item odcinki $WY$ i $XZ$ są prostopadłe,
		\item $|AW|/|BW| = |DY|/|CY|$,
		\item $|AC|/|BD| = (|AW| + |CY|) / (|BX| + |DZ|)$,
		\item równoległobok Varignona jest prostokątem. \index{równoległobok!Varignona}
	\end{itemize}
	\todofoot{brakujący rysunek}
\end{proposition}

% TODO? https://en.wikipedia.org/wiki/Bicentric_quadrilateral#Construction
% https://en.wikipedia.org/wiki/Bicentric_polygon

Pole powierzchni czworokąta dwuśrodkowego o bokach długości $a, b, c, d$ wynosi $S = \sqrt{abcd}$, jest to prosty wniosek ze wzoru Brahmagupty \ref{brahmagupta_formula}.
\index{wzór!Brahmagupty}%
Mamy $4r^2 \le S \le 2R^2$, a nawet
\begin{equation}
	S \le r^2 \left(1 + \sqrt{\left(\frac{2R}{r}\right)^2 + 1} \right),
\end{equation}
z równością wtedy i tylko wtedy, kiedy czworokąt jest prostokątnym latawcem.
Inne nierówności, których źródła nie podamy, to
\begin{equation}
	2 \sqrt {S} \le p \le r + \sqrt{r^2 + 4R^2},
\end{equation}
gdzie $p$ to połowa obwodu albo 
\begin{equation}
	S \le \frac 1 6 \left(ab + ac + ad + bc + bd + cd\right).
\end{equation}
Nierówność
\begin{equation}
	R \ge \sqrt 2 r
\end{equation}
z równością tylko dla kwadratu jest nietrywialna, dowiódł jej Fejes Tóth w 1948 roku.
\index[persons]{Tóth, Fejes}%
% TODO: citation missing.

\begin{theorem}[Fussa, 1792]
	Niech $x$ oznacza odległość między środkami okręgu wpisanego i opisanego na czworokącie dwuśrodkowym.
	Wtedy
	\begin{equation}
		\frac{1}{(R-x)^2} + \frac{1}{(R+x)^2} = \frac{1}{r^2}.
	\end{equation}
	\index{twierdzenie!Fussa}
\end{theorem}
% TODO: to jest odpowiednik wzoru Eulera dla trójkątów

Wyznaczając $x$ z twierdzenia Fussa i rozwiązując nierówność $x^2 \ge 0$ dochodzimy znowu do nierówności Tótha.
Aż dziwne, że nikt tego wcześniej nie zrobił.
Nicolaus Fuss był szwajcarskim matematykiem, który spędził większość swego życia w Rosji.
\index[persons]{Fuss, Nicolaus}

\begin{proposition}
	Punkt przecięcia przekątnych, środek okręgu wpisanego i środek okręgu opisanego na czworokącie dwuśrodkowym są współliniowe.\todofoot{Bogomolny, Alex, Collinearity in Bicentric Quadrilaterals [9], 2004.}
	\index{współliniowy}%
\end{proposition}

\begin{proposition}
	Niech $ABCD$ będzie czworokątem dwuśrodkowym, zaś $O$ środkiem okręgu opisanego.
	Wtedy środki okręgów wpisanych w trójkąty $\triangle OAB$, $\triangle OBC$, $\triangle OCD$, $\triangle ODA$ leżą na jednym okręgu.\todofoot{Alexey A. Zaslavsky, One property of bicentral quadrilaterals, 2019, [11]}
\end{proposition}

Wreszcie Klamkin pokazał w 1967 roku, że
\begin{proposition}
    Niech $p, q$ będą długościami przekątnych czworokąta dwuśrodkowego o bokach długości $a$, $b$, $c$, $d$.
    Wtedy
    \begin{equation}
        8 pq \le (a + b + c + d)^2
    \end{equation}
\end{proposition}





Bogdańska, Neugebauer \cite[s. 267]{neugebauer_2018} na ostatniej stronie podają niespodziewanie informacją, że twierdzenie Ponceleta {\color{red}\textbf{(TODO: T2.19)}\color{black}} było motywem przewodnim całego skryptu.
% todo: podlinkować te cztery dowody po ich spisaniu
Zachęcają do uogólnienia czwartego dowodu dla poniższej wersji:

\begin{theorem}[Ponceleta, małe]
	Niech trójkąt $A_0 A_1 A_2$ będzie wpisany w~stożkową $C$ oraz opisany na stożkowej $D$.
	Wtedy każdy punkt $B_0$ stożkowej $C$ jest wierzchołkiem dokładnie jednego trójkąta $B_0 B_1 B_2$ wpisanego w~stożkową $C$ oraz opisanego na stożkowej $D$.
	\index{twierdzenie!Poneceleta, małe i duże}%
\end{theorem}

Oczywiście jest też wielkie twierdzenie Ponceleta, udowodnione przez, jak niezbyt trudno się domyślić, Victora Ponceleta \cite[s. 311-317]{poncelet_1865} (wg Bogdańskiej, Neugebauera w 1813 roku, wg angielskiej Wikipedii w 1822 roku).
\index[persons]{Poncelet, Victor}%

\begin{theorem}[Ponceleta, wielkie]
\label{big_poncelet}%
	Niech $C$ i $D$ będą dwiema stożkowymi, zaś $A_0, A_1, \ldots, A_{n-1}$ takimi punktami na stożkowej $C$, że proste $A_0A_1$, $A_1A_2$, \ldots, $A_{n-1}A_0$ są styczne do stożkowej $D$.
	Wtedy dla każdego punktu $B_0$ na stożkowej $C$ istnieją różne punkty $B_1, \ldots, B_{n-1}$, też na stożkowej $C$, że proste $B_0B_1$, $B_1B_2$, \ldots, $B_{n-1}B_0$ są styczne do stożkowej $D$.
\end{theorem}

Dowód można znaleźć na przykład u Akopiana, Zasławskiego \cite[s. 93, 61, 67, 115, 124]{akopyan_2007}.

\index{czworobok zupełny|(}%
\index{czworokąt zupełny|see{czworobok zupełny}}%
\subsection{Czoworobok zupełny}
Pojęcie opisane tutaj jest charakterystyczne dla geometrii incydencji albo rzutowej.

\begin{definition}
	Cztery punkty, z których żadne trzy nie są współliniowe, leżące na sześciu prostych przez sześć par punktów, nazywamy czworokątem zupełnym.
\end{definition}

Sześć prostych wyznacza dodatkowe trzy punkty, które nazywamy przekątniowymi.

\begin{definition}
	Cztery proste, z których żadne trzy nie są współpękowe, przecinające się parami w sześciu różnych punktach, nazywamy czworobokiem zupełnym.
\end{definition}

Lachlan nazwie w 1893 roku czworoboki zupełne tetragramami, zaś czworokąty tetrastygmami.
\index[persons]{Lachlan, Robert}
Podobnież zdarzy się później napotkać jeszcze te słowa.
\index{tetragram}%
\index{tetrastygma}%
\index[persons]{Lachlan, Robert}%
% Lachlan, Robert (1893). An Elementary Treatise on Modern Pure Geometry. London, New York: Macmillan and Co.
Inne określenie na czworoboki zupełne to konfiguracje Pascha, szczególnie w kontekście potrójnych układów Steinera.
\index{układ Steinera, potrójny}%
\index{konfiguracja Pascha}

\begin{theorem}[Newtona-Gaussa]
	Środki trzech przekątnych czworoboku zupełnego leżą na jednej prostej, zwaną prostą Newtona-Gaussa.
	\index{twierdzenie!Newtona-Gaussa}%
	\index{prosta!Newtona-Gaussa}%
\end{theorem}

Różne dowody korzystają z własności pola, iloczynu zewnętrznego albo twierdzenie Menelaosa.
\index{iloczyn zewnętrzny}%
\index{twierdzenie!Menelaosa}%
\todofoot{en-wiki Newton-Gauss line}

% TODO: https://en.wikipedia.org/wiki/Newton-Gauss_line#:~:text=In%20geometry%2C%20the%20Newton-Gauss,diagonals%20of%20a%20complete%20quadrilateral.
% 	\todofoot{Twierdzenie Newtona: środek okręgu ego w czworokąt i środki przekątnych tego czworokąta są współliniowe.}

% https://en.wikipedia.org/wiki/Newton%E2%80%93Gauss_line#Existence_of_the_Newton%E2%88%92Gauss_line

\begin{theorem}[Gaussa-Bodenmillera]
	Trzy okręgi, których średnicami są przekątne czworoboku zupełnego, są współosiowie.
	% The theorem of Gauss and Bodenmiller states that the three circles whose diameters are the diagonals of a complete quadrilateral are coaxal.[8]
	\index{twierdzenie!Gaussa-Bodenmillera}
\end{theorem}

\begin{theorem}[Jemieljanowa?]
	punkt Miquela właściwego czworoboku zupełnego leży na okręgu dziewięciu punktów trójkąta przekątnego tego czworoboku.?
	\index{twierdzenie!Jemieljanowa}
\end{theorem}

\begin{proposition}
	Neugebauer 262: w każdy właściwy czworobok zupełny da się wpisać dokładnie jedną parabolę, jej ogniskiem jest punkt Miquela czworoboku.
	\todofoot{Neugebauer 262: w każdy właściwy czworobok zupełny da się wpisać dokładnie jedną parabolę, jej ogniskiem jest punkt Miquela czworoboku.}
\end{proposition}

\index{czworobok zupełny|)}%

% The two '''bimedians''' of a convex quadrilateral are the line segments that connect the midpoints of opposite sides. They intersect at the "vertex centroid" of the quadrilateral (see [[Quadrilateral#Remarkable points and lines in a convex quadrilateral|§ Remarkable points and lines in a convex quadrilateral]] below).

% The four '''maltitudes''' of a convex quadrilateral are the perpendiculars to a side—through the midpoint of the opposite side.

% https://en.wikipedia.org/wiki/Antiparallelogram => STYCZNE OKRĘGU

% The heyday of synthetic geometry can be considered to have been the 19th century, when analytic methods based on coordinates and calculus were ignored by some geometers such as Jakob Steiner, in favor of a purely synthetic development of projective geometry. For example, the treatment of the projective plane starting from axioms of incidence is actually a broader theory (with more models) than is found by starting with a vector space of dimension three. Projective geometry has in fact the simplest and most elegant synthetic expression of any geometry.

%