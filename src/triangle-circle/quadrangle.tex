%

\begin{definition}[czworokąt]
    Niech $A$, $B$, $C$, $D$ będą czterema punktami, z których żadne trzy nie są współliniowe, takimi że odcinki $AB$, $BC$, $CD$, $DA$ nie mają części wspólnej poza końcami.
    Wtedy sumę tych odcinków nazywamy czworokątem.
\end{definition}

% https://en.wikipedia.org/wiki/Rectangle

To jest definicja Hartshorne'a \cite[s. 80]{hartshorne2000}.

Pokaż, że przekątne rombu rozcinają go na cztery przystające trójkąty prostokątne. % romb cztery te same boki

Pokaż, że przekątne prostokąta są równej długości i dzielą się na połowy. % prostokąt cztery kąty proste

\subsection{Opisane, wpisane}
\begin{proposition}[okrąg opisany na czworokącie]
	\label{prp_incircle}
	Niech $A$, $B$, $C$, $D$ będą czterema punktami na płaszczyźnie takimi, że $A$ i $B$ leżą po tej samej stronie prostej $CD$.
	Wtedy następujące warunki są równoważne: punkty $A$, $B$, $C$, $D$ leżą na jednym okręgu; kąty $\angle DAC$ i $\angle DBC$ są sobie równe; suma dwóch przeciwległych kątów czworokąta $ABCD$ ma miarę kąta półpełnego.
\end{proposition}

Jedna ze wspomnianych implikacji to wniosek \ref{ab_twice_pi}.

\begin{proposition}[okrąg wpisany w czworokąt]
	\label{prp_excircle}
	Niech $A$, $B$, $C$, $D$ będą czterema punktami na płaszczyźnie takimi, że...
	\todofoot{Dokończyć okręgi wpisane}
\end{proposition}

\begin{proposition}
	Niech $\Gamma$ będzie okręgiem opisanym na czworokącie $ABCD$.
	Niech $\Gamma_1$, $\Gamma_2$, $\Gamma_3$, $\Gamma_4$ będą dowolnymi okręgami, które przechodzą przez $AB$, $BC$, $CD$, $DA$.
	Wtedy ich cztery nowe punkty przecięcia tworzą czworokąt cykliczny.
\end{proposition}

% \subsection{Twierdzenie Miquela}
Twierdzenie Miquela
\loremipsum
\todofoot{artykuł na en-wiki ,,Miquel's theorem''} % https://en.wikipedia.org/wiki/Miquel%27s_theorem


Hartshorne jako ćwiczenie \cite[s. 61]{hartshorne2000} pisze, że tym razem punkt Miquela został nazwany na cześć osoby, która go odkryła w 1838 roku.
\todofoot{Guzicki ps. 29, 32}
Audin \cite[s. 104]{audin_2003} jako ,,the pivot'' (dla niego twierdzenie Miquela mówi o czterech okręgach)
% w en-wiki Miquels' theorem to jest six citcle ^^^

\subsection{Czoworokąty dwuśrodkowe}
\input{triangle-circle/quadrangle-bicentric}

\subsection{Czoworobok zupełny}

\begin{proposition}
	Środki trzech przekątnych czworoboku zupełnego leżą na jednej prostej, zwaną prostą Newtona-Gaussa.
	\todofoot{Twierdzenie Newtona: środek okręgu ego w czworokąt i środki przekątnych tego czworokąta są współliniowe.}
	\todofoot{Twierdzenie Gaussa: środki przekątnych czworokąta zupełnego są współliniowe.}
\end{proposition}

\todofoot{twierdzenie Gaussa-Bodenmillera}

\todofoot{Jemieljanow: punkt Miquela właściwego czworoboku zupełnego leży na okręgu dziewięciu punktów trójkąta przekątnego tego czworoboku.}
\todofoot{Neugebauer 262: w każdy właściwy czworobok zupełny da się wpisać dokładnie jedną parabolę, jej ogniskiem jest punkt Miquela czworoboku.}

