%

\subsection{Czworokąty: równoległoboki, prostkąty, romby, kwadraty, trapezy, deltoidy}

\begin{definition}[czworokąt]
    Niech $A$, $B$, $C$, $D$ będą czterema punktami, z których żadne trzy nie są współliniowe, takimi że odcinki $AB$, $BC$, $CD$, $DA$ nie mają części wspólnej poza końcami.
    Wtedy sumę tych odcinków nazywamy czworokątem.
\end{definition}

To jest definicja Hartshorne'a \cite[s. 80]{hartshorne2000}.

Pokaż, że przekątne rombu rozcinają go na cztery przystające trójkąty prostokątne. % romb cztery te same boki

Pokaż, że przekątne prostokąta są równej długości i dzielą się na połowy. % prostokąt cztery kąty proste

\subsubsection{Czoworobok zupełny}


\begin{proposition}
	Środki trzech przekątnych czworoboku zupełnego leżą na jednej prostej, zwaną prostą Newtona-Gaussa.
	\todofoot{Twierdzenie Newtona: środek okręgu wpisanego w czworokąt i środki przekątnych tego czworokąta są współliniowe.}
	\todofoot{Twierdzenie Gaussa: środki przekątnych czworokąta zupełnego są współliniowe.}
\end{proposition}

\todofoot{twierdzenie Gaussa-Bodenmillera}

\todofoot{Jemieljanow: punkt Miquela właściwego czworoboku zupełnego leży na okręgu dziewięciu punktów trójkąta przekątnego tego czworoboku.}
\todofoot{Neugebauer 262: w każdy właściwy czworobok zupełny da się wpisać dokładnie jedną parabolę, jej ogniskiem jest punkt Miquela czworoboku.}

%