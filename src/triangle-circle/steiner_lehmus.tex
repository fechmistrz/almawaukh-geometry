%

\begin{theorem}[Steinera-Lehmusa]
    \label{theorem_steiner_lehmus}
	Jeżeli dwie dwusieczne trójkąta są równej długości, to trójkąt ten jest równoramienny.
\end{theorem}

Twierdzenie podaje w formie ćwiczenia Coxeter \cite[s. 9]{coxeter_1991}.
% TODO: Twierdzenie to jest zaskakująco trudne w dowodzie i zostało po raz pierwszy sformułowane w 1840 roku przez C. L. Lehmusa w liście do C. Sturma. C. Sturm przekazał to zapytanie J. Steinerowi, który podał jeden z pierwszych jego dowodów.

%