\section{Trójkąty i koła}
\subsection{Cechy przystawania}
Cechy przystawania
\loremipsum
Hartshorne s. 99

\subsection{Kąty wierzchołkowe, przyległe, odpowiadające i naprzemianległe}
Kąty wierzchołkowe, przyległe, odpowiadające i naprzemianległe
\loremipsum

% Pompe s. 5: dwie proste przecięte trzecią wyznaczają kąty tej samej miary <=> proste są równoległe (naprzemianległe)
% Pompe s. 5: suma kątów trójkąta to 180


% TODO: https://en.wikipedia.org/wiki/Triangle

\subsubsection{Cechy przystawania}
Cechy przystawania
\loremipsum
Hartshorne s. 99

%

\index{pons asinorum|(}

\subsubsection{Pons asinorum}
\index{most osłów patrz pons asinorum}
Most osłów (łacińskie \emph{,,pons asinorum''}) to tradycyjna nazwa dowodu twierdzenia, że kąty przy podstawie trójkąta równoramiennego są równe.
Podał go Euklides jako teza V w księdze I Elementów.
Mawiało się, że ci, którzy nie są w stanie samodzielnie przeprowadzić tego dedukcyjnego dowodu opartego na własnościach trójkątów przystających, nie może przekroczyć mostu i studiować dalej geometrii.

Bardziej przyziemnie Coxter \cite[s. 6-9]{coxeter_1991} zauważa, że rysunek wykonany przez Euklidesa przypomina most.
Wśród konsekwencji wymienia kilka wyników z Elementów: III.3, III.20, III.21, III.22, III.32, VI.2, VI.4, a potem III.35, III.36, VI.19, co prowadzi do dowodu twierdzenia Pitagorasa, czyli I.47. % TODO: sprawdzić, czy numeracja moja i Coxetera jest taka sama.
\index{twierdzenie!Pitagorasa}%
Coxeter podaje w formie ćwiczeń nierówność Erdős-Mordella (u nas podsekcja \ref{subsection_erdos_mordell}) oraz twierdzenie Steinera-Lehmusa (twierdzenie \ref{theorem_steiner_lehmus}).
% TODO: https://www.deltami.edu.pl/1990/08/elementarny-dowod-nierownosci-erdosa-mordella/
\todofoot{Przeczytać artykuł z Delty 1990, elementarny-dowod-nierownosci-erdosa-mordella}

Pierwsze dowody tego faktu podali jeszcze Euklides, komentujący jego prace Proklos zwany Diadochem oraz Pappus z Aleksandrii.
Współcześnie podaje się krótkie uzasadnienie w oparciu o dwusieczną kąta, ale Euklides nie mógł tak uczynić, ponieważ definiuje ją dopiero cztery tezy później w swoich Elementach.

O moście osłów piszą Coxeter 

\index{pons asinorum|)}

%

%

\begin{proposition}[nierówność trójkąta]
	Niech $ABC$ będzie trojkątem.
	Wtedy suma odcinków $AB$ i $BC$ jest dłuższa niż $AC$.
	\index{nierówność trójkąta}
\end{proposition} % https://en.wikipedia.org/wiki/Triangle_inequality#Euclidean_geometry I.20

Nierówność trójkąta nie jest wnioskiem z aksjomatów I1-I3, B1-B4, C1-C3, ponieważ nie zachodzi w następującym modelu: płaszczyzna to zbiór $\mathbb R^2$, ze standardowymi punktami i prostymi, ale niestandardową metryką
\begin{equation}
	d((x_1, y_1), (x_2, y_2)) = \begin{cases}
		\sqrt{(x_1-x_2)^2 + (y_1-y_2)^2} & \text{jeśli } x_1 = x_2 \vee y_1 = y_2, \\
		2 \sqrt{(x_1-x_2)^2 + (y_1-y_2)^2} & \text{w przeciwnym wypadku}
	\end{cases}.
\end{equation}

(Powyższy przykład opisał Hartshorne \cite[s. 90]{hartshorne2000}).

%

%

\subsubsection{Twierdzenie Pitagorasa}
Najważniejszym twierdzeniem dotyczącym trójkątów prostokątnych jest twierdzenie Pitagorasa oraz twierdzenie do niego odwrotne.
Piszą o~nim Guzicki \cite[s. 160]{guzicki_2021}.

\begin{theorem}[Pitagorasa, ok. 500 r. p.n.e.]
\index{twierdzenie!Pitagorasa}%
    Niech $ABC$ będzie trójkątem prostokątnym, w~którym kąt przy wierzchołku $C$ jest prosty.
    Wtedy
    \begin{equation}
        |BC|^2 + |AC|^2 = |AB|^2.
    \end{equation}
    Odwrotnie, jeśli $ABC$ jest trójkątem takim, że $|BC|^2 + |AC|^2 = |AB|^2$, to trójkąt ten jest prostokątny, zaś kąt przy wierzchołku $C$ jest prosty.
\end{theorem}

Chociaż współcześnie powyższe twierdzenie przypisujemy Pitagorasowi z~Samos, to nie wiemy dokładnie, kto i~kiedy odkrył je jako pierwszy.
\index[persons]{Pitagoras z Samos}%
Było powszechnie stosowane w~okresie Starego Babilonu (XX-XVI wiek p.n.e.), a~więc na długo przed narodzinami Pitagorasa; pojawia się też w indyjskich i~chińskich tekstach matematycznych.
Papirus Berlin 6619 spisany ok. 1800 roku p.n.e. na terenach państwa egipskiego zawiera zadanie, którego rozwiązaniem jest trójka $(6, 8, 10)$.

Już w~szkole podstawowej uczniowie poznają trójkąt prostokątny o bokach długości $3, 4, 5$ wraz~z~legendą, że podobno Egipcjanie używali tego trójkąta do wyznaczania w terenie kątów prostych.

% TODO: rysunek z Guzickiego, stron 160

\begin{proposition}
    Mają miejsce następujące równości:
    \begin{equation}
        h = \frac{ab}{c}, \quad
        p = \frac{b^2}{c}, \quad
        q = \frac{a^2}{c}, \quad
        h^2 = pq.
    \end{equation}
\end{proposition}

Dowód wykorzystujący podobieństwa trójkątów można znaleźć u~Guzickiego \cite[s. 160, 161]{guzicki_2021}.

Twierdzenie Pitagorasa znajduje zastosowanie także przy wyznaczaniu niektórych miejsc geometrycznych.

\begin{proposition}
    Dane są dwa różne punkty $A$ i $B$ na płaszczyźnie oraz liczba rzeczywista $c$ taka, że $2c > |AB|^2$.
    Miejscem geometrycznym punktów $P$ o własności $|AP|^2 + |BP|^2 = c$ jest okrąg o środku w środku odcinka $AB$ i promieniu $r = \frac 1 2 \sqrt{2c - |AB|^2}$.
\end{proposition}

\begin{proposition}
    Dane są dwa różne punkty $A$ i $B$ na płaszczyźnie oraz liczba rzeczywista $c$.
    Miejscem geometrycznym punktów $P$ o własności $|AP|^2 - |BP|^2 = c$ jest prosta prostopadła do prostej $AB$.
\end{proposition}

Patrz Guzicki \cite[s. 170-173]{guzicki_2021} (Guzicki wprowadza potem osie i środki potęgowe jak w~fakcie \ref{guzicki_6_11}, a następnie twierdzenie \ref{guzicki_6_13} (Carnota)).

%

% https://en.wikipedia.org/wiki/Pythagorean_theorem liczne dowody, wiek Pitagorasa
% https://en.wikipedia.org/wiki/Xuan_tu
% https://en.wikipedia.org/wiki/Spiral_of_Theodorus
% https://en.wikipedia.org/wiki/Garfield%27s_proof_of_the_Pythagorean_theorem

\subsubsection{Wzór Herona}
%

\index{wzór!Herona|(}
Guzicki \cite[s. 165-168]{guzicki_2021} wyprowadza wzór Herona z twierdzenia Pitagorasa.
\index{twierdzenie!Pitagorasa}
Oryginalny dowód Herona był dość skomplikowany, Guzicki \cite[s. 168-169]{guzicki_2021} wspomina o znacznie prostszym dowodzie geometrycznym, pochodzącym od Eulera.
\index[persons]{Euler, Leonhard}%

\begin{proposition}
	Niech $ABC$ będzie trójkątem o obwodzie $2p$ oraz polu powierzchni $S$.
	Wtedy
	\begin{equation}
		S \le \frac{p^2}{3 \sqrt{3}}
	\end{equation}
\end{proposition}

Guzicki wyprowadza tę nierówność izoperymetryczną ze wzoru Herona oraz nierówności między średnią arytmetyczną i geometryczną.

\index{wzór!Herona|)}

% TODO: wzór Herona (Guzicki-6), Brahmagupty

%

\subsubsection{Symetralna i okrąg opisany}
Symetralna i okrąg opisany
\loremipsum

\subsubsection{Ortocentrum}
Ortocentrum.
\loremipsum

\subsubsection{Problemy Fagnano i Fermata}
Problemy Fagnano i Fermata
\todofoot{Coxeter, s. 20, 21}

\begin{problem}[zadanie Fermata]
	Dany jest trójkąt $ABC$.
	Znaleźć punkt $F$ taki, by suma $|FA| + |FB| + |FC|$ była możliwie najmniejsza.
\end{problem}

Powyższe zadanie rozwiązał Evangelista Torricelli, który dostał je w formie wyzwania od Fermata.
Rozwiązanie opublikował student Torricelliego, Viviani, w 1659 roku.
% TODO: Johnson, R. A. Modern Geometry: An Elementary Treatise on the Geometry of the Triangle and the Circle. Boston, MA: Houghton Mifflin, pp. 221-222, 1929.
\todofoot{Punkt Fermata (punkt Torricellego) – punkt w trójkącie, którego suma odległości od wierzchołków trójkąta jest najmniejsza z możliwych. Pierwszy raz problem konstrukcji takiego punktu został rozwiązany przez Fermata w prywatnym liście.}. %https://pl.wikipedia.org/wiki/Punkt_Fermata

\subsubsection{Nierówności trójkątne}

\begin{proposition}[nierówność izoperymetryczna]
	Dany jest trójkąt o połowie obwodu $p$ oraz polu $S$.
	Wtedy 
	\begin{equation}
		S \le \frac{p^2}{3 \sqrt 3},
	\end{equation}
	zatem wśród trójkątów o ustalonym obwodzie największe pole ma trójkąt równoboczny.
	\index{nierówność!izoperymetryczna}
\end{proposition}

Guzicki \cite[s. 169, 170]{guzicki_2021} wyprowadza nierówność izoperymetryczną ze wzoru Herona oraz nierówności między średnią arytmetyczną i geometryczną.
\index{wzór!Herona}

% trójwymiarowy odpowiednik to hipoteza: https://math.stackexchange.com/questions/4044670/what-is-the-largest-volume-of-a-polyhedron-whose-skeleton-has-total-length-1-is

\begin{proposition}[stosunek sumy środkowych do obwodu]
	Niech... % stosunek sumy środkowych do obwodu leży między 3/4 i 1 (s. 355),
	% TODO: https://en.wikipedia.org/wiki/Isoperimetric_inequality ?
\end{proposition}

\begin{proposition}[nierówność Eulera]
	$R \ge 2r$
	% TODO: Twierdzenie Eulera: $d^2 = R^2 - 2Rr$. % Audin \cite[s. 110]{audin_2003} podaje ten fakt w formie ćwiczenia.

	% TODO: Eulera: R >= 2r https://en.wikipedia.org/wiki/Euler%27s_theorem_in_geometry
	\todofoot{formuła Eulera na odległość między środkami okręgu opisanego i wpisanego (dla trójkąta)}
	\index{nierówność!Eulera}
\end{proposition}

\begin{proposition}[nierówność Mitrinovica]
	Niech...
	\index{nierówność!Mitrinovica}
\end{proposition}

\begin{proposition}[nierówność Leibniza]
	Niech...
	\index{nierówność!Leibniza}
\end{proposition}

\begin{proposition}[nierówność Weitzenbocka]
	Niech...
	% TODO: https://en.wikipedia.org/wiki/Hadwiger–Finsler_inequality => Weitzenbock
	% TODO: https://en.wikipedia.org/wiki/Pedoe%27s_inequality => Weitzenbock
	\index{nierówność!Weitzenbocka}
\end{proposition}

Snellius-Huygens: $2 \sin x + \tan x > 3x$.
\index{nierówność!Snelliusa-Huygensa}

\index{nierówność!Erdősa-Mordella|(}
%

\label{subsection_erdos_mordell}
Erdős w 1935 roku postawi problem dowodu tej nierówności; dowód przedstawią dwa lata później Mordell i D. F. Barrow (1937), choć nie będzie on zbyt elementarny.
Później znajdzie się prostsze dowody: Kazarinoff (1957), Bankoff (1958) oraz Alsina i Nelsen (2007).
% TODO: https://en.wikipedia.org/wiki/Erdős–Mordell_inequality#CITEREFErdős1935

\begin{theorem}[nierówność Erdősa-Mordella]
    Niech $P$ będzie punktem wewnątrz trójkąta $\triangle ABC$, zaś $A_p, B_p, C_p$ spodkami punktu $P$ na boki trójkąta jak na rysunku \ref{erdos_mordell_barrowa}.
    Wtedy
    \begin{equation}
        |PA| + |PB| + |PC| \ge 2 (|PA_p| + |PB_p| + |PC_p|).
    \end{equation}
\end{theorem}


\begin{figure}[H] \centering
\begin{minipage}[b]{.45\linewidth}
\begin{center}
    \begin{comment}
    \begin{tikzpicture}[scale=.4]
    \tkzDefPoint(0, 0){A}
    \tkzDefPoint(10, 2){B}
    \tkzDefPoint(6, 7){C}
    \tkzDefPoint(5, 3){P}
    \tkzLabelPoint[below left](A){$A$}
    \tkzLabelPoint[below right](B){$B$}
    \tkzLabelPoint[above](C){$C$}
    \tkzLabelPoint[below left](P){$P$}
    \tkzDefPointsBy[projection=onto A--B](P){Pc}
    \tkzDefPointsBy[projection=onto B--C](P){Pa}
    \tkzDefPointsBy[projection=onto C--A](P){Pb}
    \tkzLabelPoint[above right](Pa){$A_p$}
    \tkzLabelPoint[above left](Pb){$B_p$}
    \tkzLabelPoint[below](Pc){$C_p$}

    \tkzDrawSegments[line width=0.2mm,dashed](P,Pa P,Pb P,Pc)
    \tkzDrawPolygon[line width=0.3mm](A,B,C)
    \tkzMarkRightAngles[size=0.5](P,Pa,C P,Pb,A P,Pc,B)
    \tkzDrawPoints[size=3,color=black,fill=black!50](A,B,C,P,Pc,Pb,Pa)
\end{tikzpicture}
\end{comment}
    \end{center}
    \subcaption{nierówność Erdősa-Mordella}
    \label{erdos_mordell_barrowa}
\end{minipage}
%
\begin{minipage}[b]{.45\linewidth}
\begin{center}\begin{comment}
    \begin{tikzpicture}[scale=.4]
    \tkzDefPoint(0, 0){A}
    \tkzDefPoint(10, 2){B}
    \tkzDefPoint(6, 7){C}
    \tkzDefPoint(5, 3){P}

    \tkzDefLine[bisector](A,P,B) \tkzGetPoint{prePc}
    \tkzInterLL(P,prePc)(A,B) \tkzGetPoint{Pc}
    \tkzDefLine[bisector](B,P,C) \tkzGetPoint{prePa}
    \tkzInterLL(P,prePa)(B,C) \tkzGetPoint{Pa}
    \tkzDefLine[bisector](C,P,A) \tkzGetPoint{prePb}
    \tkzInterLL(P,prePb)(C,A) \tkzGetPoint{Pb}

    \tkzLabelPoint[below left](A){$A$}
    \tkzLabelPoint[below right](B){$B$}
    \tkzLabelPoint[above](C){$C$}
    %\tkzLabelPoint[below left](P){$P$}
    \tkzLabelPoint[above right](Pa){$A_p$}
    \tkzLabelPoint[above left](Pb){$B_p$}
    \tkzLabelPoint[below](Pc){$C_p$}

    \tkzMarkAngle[arc=lll,size=1.2,mark=|||](A,P,Pc)
    \tkzMarkAngle[arc=lll,size=1.2,mark=|||](Pc,P,B)
    \tkzMarkAngle[arc=ll,size=1.2,mark=||](B,P,Pa)
    \tkzMarkAngle[arc=ll,size=1.2,mark=||](Pa,P,C)
    \tkzMarkAngle[arc=l,size=1.2,mark=|](C,P,Pb)
    \tkzMarkAngle[arc=l,size=1.2,mark=|](Pb,P,A)

    \tkzDrawSegments[line width=0.2mm](P,A P,B P,C)
    \tkzDrawSegments[line width=0.2mm,dashed](P,Pa P,Pb P,Pc)
    \tkzDrawPolygon[line width=0.3mm](A,B,C)
    \tkzDrawPoints[size=3,color=black,fill=black!50](A,B,C,P,Pc,Pb,Pa)
\end{tikzpicture}
\end{comment}
    \end{center}
    \subcaption{nierówność Barrowa}
    \label{erdos_mordell_barrowb}
\end{minipage}
\caption{}
\end{figure}

Twierdzenie poda w formie ćwiczenia Coxeter \cite[s. 9]{coxeter_1991}, Audin z licznymi wskazówkami \cite[s. 102]{audin_2003}.

Wzmocnieniem nierówności Erdősa-Mordella będzie nierówność Barrowa:

% TODO: https://en.wikipedia.org/wiki/Barrow%27s_inequality

\begin{theorem}[nierówność Barrowa]
    Niech $P$ będzie punktem wewnątrz trójkąta $\triangle ABC$, zaś $A_p$, $B_p$, $C_p$ punktami przecięć dwusiecznych trzech kątów wyznaczanych przez $P$ i pary wierzchołków trójkąta; tak jak na rysunku \ref{erdos_mordell_barrowb}.
    Wtedy
    \begin{equation}
        |PA| + |PB| + |PC| \ge 2 (|PA_p| + |PB_p| + |PC_p|).
    \end{equation}
\end{theorem}

Dowód Barrowa zostanie opublikowany w 1937 roku, ale nazwa ,,nierówność Barrowa'' będzie używana dopiero od 1961 roku; nie wiemy, co się wtedy stanie.
% TODO: Erdős, Paul; Mordell, L. J.; Barrow, David F. (1937), "Solution to problem 3740", American Mathematical Monthly, 44 (4): 252–254, doi:10.2307/2300713, JSTOR 2300713.

% % barrow tu jest
\index{nierówność!Erdősa-Mordella|)}

% TODO: https://en.wikipedia.org/wiki/Ono%27s_inequality

Mikołaj z Kuzy: $\sin x / x < (2 + \cos x) / 3$.
\index{nierówność!Mikołaja z Kuzy}
\index[persons]{Mikołaj z Kuzy}

\todofoot{Coxeter, s. 12}

\subsubsection{Nie wiem gdzie}

\begin{proposition}
	\label{srodkowe_przecinaja_sie}
	Środkowe trójkąta przecinają się w jednym punkcie zwanym środkiem ciężkości (po ang. \emph{centroid}?) i dzielą w stosunku $2 : 1$ licząc od wierzchołków.
\end{proposition}

Hartshorne \cite[s. 53, 54]{hartshorne2000} wnioskuje powyższe z \ref{hartshorne_52}.
Podobnie postępują Bogdańska, Neugebauer (chociaż oni wyprowadzają fakt \ref{hartshorne_52} z twierdzenia Talesa).

\begin{proposition}
	\label{wysokosci_przecinaja_sie}
	Wysokości trójkąta (proste prostopadłe do podstawy przechodzące przez wierzchołek nieleżący na niej) przecinają się w jednym punkcie zwanym ortocentrum.
\end{proposition}

Hartshorne \cite[s. 52, 54]{hartshorne2000} pisze, że ten oraz poprzedni fakt (\ref{wysokosci_przecinaja_sie}, \ref{srodkowe_przecinaja_sie}) były znane Archimedesowi.
Fakt zostaje powtórzony \cite[s. 119-120]{hartshorne2000}, by pokazać zastosowanie geometrii analitycznej.

Twierdzneie o prostej Wallace'a-Simsona. % https://en.wikipedia.org/wiki/Simson_line


\begin{proposition}
	Niech $P$ będzie dowolnym punktem leżącym na okręgu opisanym na trójkącie $ABC$, zaś $D$, $E$ oraz $F$ rzutami punktu $P$ na proste zawierające boki trójkąta $ABC$.
	Wtedy punkty $D$, $E$ oraz $F$ są współliniowe.
\end{proposition}

Hartshorne jako ćwiczenie \cite[s. 61]{hartshorne2000} pisze, że istnienie prostej Simsona jako pierwszy dowiódł Wallace w 1799 roku.


Droz-Farny: proste przechodzą przez ortocentrum trójkąta i są prostopadłe, wtedy środki odcinków leżą na jednej prostej. % https://en.wikipedia.org/wiki/Droz-Farny_line_theorem

%

% TODO: https://math.stackexchange.com/questions/3332146/is-it-possible-to-find-such-an-angle-using-only-angle-chasing
% TODO: https://math.stackexchange.com/questions/3009635/japanese-temple-problem-from-1844

%

\begin{definition}[czworokąt]
    Niech $A$, $B$, $C$, $D$ będą czterema punktami, z których żadne trzy nie są współliniowe, takimi że odcinki $AB$, $BC$, $CD$, $DA$ nie mają części wspólnej poza końcami.
    Wtedy sumę tych odcinków nazywamy czworokątem.
\end{definition}


Quadri (Latin for 4) + latus (side). Tetragon = tetra (grecki 4) + gon (corner, angle), like polygon, pentagon. Quadrangle.Prosty (bez samoprzeciec) jest wypukly lub wklesly, albo z samoprzecieciami. Mucha albo motyl.

Irregular quadrilateral (GBr) lub trapezium (NA) nie ma pary bokow rownoleglych (kiedys nazywal sie trapezoidem w GBr)

Trapez posiada jedna (lub dwie) pary bokow rownoleglych (UK trapezium, US trapezoid). Kazdy rownoleglobok jest trapezem.Trapez, ktory posiada os symetrii, ktora jest tez symetralna dwoch przeciwleglych bokow, nazywamy rownoramiennym. (!!! Please do NOT define an isosceles trapezoid as having legs equal. Doing so would make all parallelograms isosceles trapezoids, which we know is wrong. )

Rownoleglobok posiada dwie pary bokow rownoleglych, albo przeciwlegle boki rownej dlugosci, albo przeciwlegle katy rownej miary, albo przekatne polowia sie wzajemnie. Rownolegloboki obejmuje romby, romboidy. (Parallelogram)

Romby (rhombus, rhomb) posiadaja cztery boki rownej dlugosci, albo prostopadle przekatne polowiace sie.

Romboid to rownoleglobok, ktory nie jest rombem, bo posiada boki roznej dlugosci. Niektorzy dodaja, ze musi miec katy roznej miary, wykluczajac w ten sposob prostokaty. Rzadko uzywana klasa.

Prostokat ma cztery katy proste, albo przekatne rownej dlugosci polowiace sie. Wsrod prostokatow wyrozniamy kwadraty (ktore maja wszystkie boki tej samej dlugosci) oraz oblongi (ktore nie). Kwadraty to dokladnie prostokaty, ktore sa tez rombami; maja rowne boki o katy.

Kite ma dwie pary sasiednich bokow rownej dlugosci, wiec jedna z przekatnych dzieli go na przystajace trojkaty. Wynika stad, ze przekatne sa prostopadle. Kite obejmuje romby. Kite prosty to taki, ktory ma dwa katy proste naprzeciw siebie, mozna na takim opisac kolo. (HJEMLSJEV!).

Czworokat cykliczny to taki, ktory...

---

A dart (or arrowhead) is a concave quadrilateral with bilateral symmetry like a kite, but where one interior angle is reflex. See Kite.
A self-intersecting quadrilateral is called variously a cross-quadrilateral, crossed quadrilateral, butterfly quadrilateral or bow-tie quadrilateral. In a crossed quadrilateral, the four "interior" angles on either side of the crossing (two acute and two reflex, all on the left or all on the right as the figure is traced out) add up to 720°.

% https://en.wikipedia.org/wiki/Rectangle

To jest definicja Hartshorne'a \cite[s. 80]{hartshorne2000}.

Pokaż, że przekątne rombu rozcinają go na cztery przystające trójkąty prostokątne. % romb cztery te same boki

Pokaż, że przekątne prostokąta są równej długości i dzielą się na połowy. % prostokąt cztery kąty proste

\subsection{Opisane, wpisane}
\begin{proposition}[okrąg opisany na czworokącie]
	\label{prp_incircle}
	Niech $A$, $B$, $C$, $D$ będą czterema punktami na płaszczyźnie takimi, że $A$ i $B$ leżą po tej samej stronie prostej $CD$.
	Wtedy następujące warunki są równoważne: punkty $A$, $B$, $C$, $D$ leżą na jednym okręgu; kąty $\angle DAC$ i $\angle DBC$ są sobie równe; suma dwóch przeciwległych kątów czworokąta $ABCD$ ma miarę kąta półpełnego.
\end{proposition}

Jedna ze wspomnianych implikacji to wniosek \ref{ab_twice_pi}.

\begin{proposition}[okrąg wpisany w czworokąt]
	\label{prp_excircle}
	Niech $A$, $B$, $C$, $D$ będą czterema punktami na płaszczyźnie takimi, że...
	\todofoot{Dokończyć okręgi wpisane}
\end{proposition}

\begin{proposition}
	Niech $\Gamma$ będzie okręgiem opisanym na czworokącie $ABCD$.
	Niech $\Gamma_1$, $\Gamma_2$, $\Gamma_3$, $\Gamma_4$ będą dowolnymi okręgami, które przechodzą przez $AB$, $BC$, $CD$, $DA$.
	Wtedy ich cztery nowe punkty przecięcia tworzą czworokąt cykliczny.
\end{proposition}

% \subsection{Twierdzenie Miquela}
Twierdzenie Miquela
\loremipsum
\todofoot{artykuł na en-wiki ,,Miquel's theorem''} % https://en.wikipedia.org/wiki/Miquel%27s_theorem


Hartshorne jako ćwiczenie \cite[s. 61]{hartshorne2000} pisze, że tym razem punkt Miquela został nazwany na cześć osoby, która go odkryła w 1838 roku.
\todofoot{Guzicki ps. 29, 32}
Audin \cite[s. 104]{audin_2003} jako ,,the pivot'' (dla niego twierdzenie Miquela mówi o czterech okręgach)
% w en-wiki Miquels' theorem to jest six citcle ^^^

\subsection{Czoworokąty dwuśrodkowe}
\input{triangle-circle/quadrangle-bicentric}

\subsection{Czoworobok zupełny}

\begin{proposition}
	Środki trzech przekątnych czworoboku zupełnego leżą na jednej prostej, zwaną prostą Newtona-Gaussa.
	\todofoot{Twierdzenie Newtona: środek okręgu ego w czworokąt i środki przekątnych tego czworokąta są współliniowe.}
	\todofoot{Twierdzenie Gaussa: środki przekątnych czworokąta zupełnego są współliniowe.}
\end{proposition}

\todofoot{twierdzenie Gaussa-Bodenmillera}

\todofoot{Jemieljanow: punkt Miquela właściwego czworoboku zupełnego leży na okręgu dziewięciu punktów trójkąta przekątnego tego czworoboku.}
\todofoot{Neugebauer 262: w każdy właściwy czworobok zupełny da się wpisać dokładnie jedną parabolę, jej ogniskiem jest punkt Miquela czworoboku.}



\subsection{Okręgi}
\subsubsection{Kąty środkowe, wpisane, dopisane}

\begin{proposition}
    Kąt środkowy jest dwa razy większy od kąta wpisanego opartego na tym samym łuku.
    \index{kąt!środkowy}
    \index{kąt!wpisany}
\end{proposition}
% PRZECZYTANO: https://en.wikipedia.org/wiki/Inscribed_angle

O tym samym co (III.20) piszą Guzicki \cite[s. 11-13]{guzicki_2021}, Audin \cite[s. 74, 75]{audin_2003}.

\begin{corollary}
    Kąty wpisane oparte na tym samym łuku (ogólniej: na równych łukach) są równe.
\end{corollary}

\begin{corollary}
    Kąty oparty na półokręgu jest prosty.
\end{corollary}

Grecki pisarz Diogenes Laertios przypisze to stwierdzenie Talesowi.
\index[persons]{Tales}%
\index[persons]{Laertios, Diogenes}%
Można wykorzystać je do konstrukcji stycznej.
\index{styczna}%

\begin{corollary}
    \label{ab_twice_pi}
    Kąty wpisane oparte na dwóch różnych łukach $AB$ dają w sumie kąt półpełny.
\end{corollary}

Wykorzystamy ten wniosek później (fakt \ref{prp_incircle}) do opisania, na których czworokątach można opisać okrąg.

\begin{proposition}
    Kąt dopisany do okręgu jest równy kątowi wpisanemu opartemu na łuku zawartym w danym kącie dopisanym.
    \index{kąt!dopisany}
\end{proposition}
% Guzicki s. 18

\todofoot{Geometria koła i kątów, twierdzenie Apoloniusza (s. 22)}
\index{twierdzenie!Apoloniusza?}

\subsubsection{Okręgi wpisane i opisane}

Okrąg wpisany, opisany, dopisany: Audin \cite[s. 98]{audin_2003}.
\index{okrąg!wpisany}
\index{okrąg!opisany}
\index{okrąg!dopisany}

Styczna do okręgu, okrąg wpisany w kąt.
\index{styczność}
Styczne są równej długości.
Dwustyczne (do dwóch okręgów).
\index{dwustyczna}
Okrąg wpisany w trójkąt, okręgi dopisane do trójkąta.
Warunki istnienia okręgu stycznego do czterech prostych.

Euklides (III.35):

% https://en.wikipedia.org/wiki/Intersecting_chords_theorem
\begin{proposition}[twierdzenie o cięciwach]
    \label{prop_intersecting_chords}
	Niech punkty $A$, $B$ leżą na pewnej prostej przechodzącej przez punkt $S$, zaś punkty $C$, $D$ leżą na innej prostej przez ten punkt.
	Wtedy cztery punkty $A$, $B$, $C$, $D$ leżą na jednym okręgu wtedy i tylko wtedy, gdy:
	\begin{equation}
		|AS| \cdot |BS| = |CS| \cdot |DS|.
	\end{equation}
	\index{cięciwa}%
	\index{twierdzenie!o cięciwach}%
\end{proposition}

Po angielsku nazywa się to ,,\emph{intersecting chords theorem}''; Bogdańska i Neugebauer nazywają to nie wiedzieć czemu ,,potęgowym kryterium współokręgowości''.
Ech.
Elementarny dowód korzysta z~podobieństwa trójkątów $\triangle ASD$ i $\triangle BSC$.
Nieelementarny zauważa, że obydwa iloczyny są równe (poza znakiem) potędze punktu $S$ względem okręgu przez punkty $A$, $B$, $C$, $D$.
Definicję \ref{def_power_point} potęgi punktu podamy już za kilka stron.
\index{potęga punktu}%

\begin{proposition}[twierdzenie o siecznych i stycznych]
	Jeżeli... (Neugebauer s. 66)
	\index{twierdzenie!o siecznych i stycznych}
\end{proposition}

\todofoot{twierdzenie o motylku} % https://en.wikipedia.org/wiki/Butterfly_theorem
\index{twierdzenie!o motylku}

\subsubsection{Twierdzenia Ptolemeusza, Carnota, Caseya}
\input{triangle-circle/circle/ptolemy}

\subsubsection{Potęga punktu względem okręgu}
\index{potęga punktu względem okręgu} % zamienić na zakres stron

Inwersja: Audin \cite[s. 84]{audin_2003}.
% zachowuje kąty

\todofoot{Coxeter s. 85, coaxial circles}

Potęgę punktu wprowadził Jakob Steiner\footnote{Praca \emph{Einige geometrische Betrachtungen}, strona 164} w 1826 roku.
\index[persons]{Steiner, Jakob}%
Użył jej, aby znaleźć okrąg przecinający cztery dane okręgi pod tym samym kątem; rozwiązać problem Apolloniusza (problem \ref{problem_apolloniusza}: znaleźć okrąg styczny do trzech); skonstruować okręgi Malfattiego (problem \ref{malfatti_problem}: trzy okręgi, które są styczne do pozostałych dwóch oraz boków zadanego trójkąta).
% TODO: index, ref
\index{problem Apolloniusza}%
\index{okręgi Malfattiego}%
Wydaje się, że ta lista jest niekompletna.

Wśród elementarnych zastosowań można wskazać dowód twierdzenia o przecinających się cięciwach (fakt \ref{prop_intersecting_chords}).

\begin{definition}[potęga punktu względem okręgu]
	\label{def_power_point}
	Dane są okrąg $\Gamma$ o środku $O$ i promieniu $r$ oraz dowolny punkt $A$.
	Liczbę rzeczywistą
	\begin{equation}
		\Pi(A) := |OA|^2 - r^2
	\end{equation}
	nazywamy potęgą punktu $A$ względem okręgu $\Gamma$.
\end{definition}

Pisze o tym Audin \cite[s. 89]{audin_2003}.

Twierdzenie Pitagorasa dostarcza prostej interpretacji wielkości $\Pi(A)$, gdy punkt $A$ leży na zewnątrz okręgu: jest to długość stycznej do okręgu, która przechodzi przez punkt.

\begin{proposition}
	Środki odcinków wspólnych stycznych do dwóch okręgów leżą na jednej prostej. % Neugebauer s. 70
\end{proposition}

Nie wiemy, jak trudny jest dowód powyższego bez poniższego.

\begin{proposition}[oś potęgowa istnieje]
\label{guzicki_6_11}%
    Dane są dwa niewspółśrodkowe okręgi $\Gamma_1(O_1, r_1)$ oraz $\Gamma_2(O_2, r_2)$.
    Wtedy miejscem geometrycznym punktów $P$ mających równe potęgi względem obu okręgów:
	\begin{equation}
		\{S : \Pi(S, \Gamma_1) = \Pi(S, \Gamma_2)\}
	\end{equation}
	jest prosta prostopadła do prostej przechodzącej przez środki obu okręgów.
	Nazywamy ją \emph{osią potęgową} okręgów $\Gamma_1$, $\Gamma_2$.
\end{proposition}

O osi potęgowej piszą Bogdańska, Neugebauer \cite[s. 69]{neugebauer_2018}; Guzicki \cite[s. 173, 174]{guzicki_2021}, Audin \cite[s. 89]{audin_2003}.
Oś potęgowa okręgów stycznych to prosta styczna do obydwu okręgów; oś potęgowa okręgów, które się przecinają w dwóch punktach, to prosta przez punkty przecięcia.

\begin{proposition}[środek potęgowy]
	Dane są trzy parami niewspółśrodkowe okręgi $\Gamma_1, \Gamma_2, \Gamma_3$.
	Wtedy albo środki tych okręgów są współliniowe (i trzy osie potęgowe każdej pary są równoległe), albo wszystkie trzy osie przecinają się w~jednym punkcie: \emph{środku potęgowym} okręgów.
\end{proposition}

Nie jest jasne, dlaczego niektórzy nazywają to twierdzeniem Monge'a (jak Bogdańska, Neugebauer \cite[s. ???]{neugebauer_2018}) albo też rozwiązaniem problemu Monge'a (jak Dörrie \cite[s. 151]{dorrie_1965}).
Ale jest jasne, że pozwala łatwo skonstruować oś potęgową dwóch okręgów: wystarczy przeciąć je jednocześnie trzecim okręgiem (na dwa różne sposoby).

Patrz Guzicki \cite[s. 174]{guzicki_2021}.

% Neugebauer s. 71
\begin{proposition}
	Ortocentrum trójkąta jest środkiem potęgowym rodziny wszystkich okręgów o średnicach będących czewianami (odcinkami łączącymi wierzchołek z przeciwległym bokiem).
\end{proposition}

\begin{theorem}[Auberta]
	Dane są cztery proste w położeniu ogólnym (żadne trzy nie przechodzą przez jeden punkt, żadne dwie nie są równoległe).
	Wówczas ortocentra czterech trójkątów wyznaczonych przez trójki tych prostych leża na jednej prostej, zwanej prostą Auberta.
\end{theorem}

Nie udało nam się ustalić, kim był Aubert, ale najprawdopodobniej udowodnił to twierdzenie w 1899 roku.
Prosta Auberta znana jest czasami jako prosta Steinera.
% TODO: https://el-m-wikipedia-org.translate.goog/wiki/Ευθεία_Σίμσον_(τετράπλευρο)?_x_tr_sl=auto&_x_tr_tl=pl&_x_tr_hl=pl&_x_tr_pto=wapp
% https://el-m-wikipedia-org.translate.goog/wiki/Θεώρημα_Miquel_(τετράπλευρο)?_x_tr_sl=auto&_x_tr_tl=pl&_x_tr_hl=pl&_x_tr_pto=wapp
% https://el-m-wikipedia-org.translate.goog/wiki/Ευθεία_Νεύτωνα-Γκάους?_x_tr_sl=auto&_x_tr_tl=pl&_x_tr_hl=pl&_x_tr_pto=wapp

% pencil
O pencilu piszą Audin \cite[s. 92-98]{audin_2003}.

\subsubsection{Okrąg dziewięciu punktów (Feuerbacha) i prosta Eulera}
%

Prosta Eulera to pierwsza w szkolnej geometrii trójka punktów współliniowych.
Przyszła na świat w~1765 roku, w żurnalu \emph{Novi commentarii Academiae Petropolitanae}, w artykule \emph{Solutio facilis problematum quorundam geometricorum difficillimorum}, czyli jakbyśmy powiedzieli po polsku, ,,Łatwe rozwiązanie niektórych najtrudniejszych problemów geometrycznych''.

\begin{proposition}[prosta Eulera]
	\label{prosta_eulera}
	Środek okręgu opisanego na nierównobocznym trójkącie, środek ciężkości oraz ortocentrum leżą na jednej prostej, zwanej prostą Eulera.
	\index{prosta!Eulera}
\end{proposition}

Piszą o niej
Coxeter \cite[s. 32, 33]{coxeter_1967},
Hartshorne \cite[s. 54, 55]{hartshorne2000},
Bogdańska, Neugebauer \cite[s. 84]{neugebauer_2018},
Audin \cite[s. 61]{audin_2003}.
W $\Delta_{84}^{4}$ podany będzie przepis, jak skonstruować trójkąt, w którym prosta Eulera ma zadane położenie względem podstawy.

\begin{proposition}
	Prosta Eulera jest prostopadła do jednego z boków trójkąta wtedy i tylko wtedy, gdy trójkąt jest równoramienny.
\end{proposition}

\begin{proposition}[okrąg dziewięciu punktów]
\label{okrag_dziewieciu_punktow}%
	W każdym trójkącie środki boków, spodki wysokości oraz środki odcinków łączących ortocentrum z wierzchołkami leżą na jednym okręgu.
	Jego środek pokrywa się ze środkiem odcinka łączącego środek okręgu opisanego z ortocentrum, zaś jego promień jest dwukrotnie krótszy od promienia okręgu opisanego.
	\index{okrąg!dziewięciu punktów}
\end{proposition}

% TODO: https://en.wikipedia.org/wiki/Nine-point_circle

Coxeter \cite[s. 34, 35, 88]{coxeter_1967}, Hartshorne \cite[s. 57, 60]{hartshorne2000}, Bogdańska, Neugebauer \cite[s. 85, 86]{neugebauer_2018}.
Audin \cite[s. 62]{audin_2003}.

Temat badali Benjamin Bevan (który zasugerował środek oraz promień) i John Butterworth (który udowodnił podejrzenia Bevana) na początku XIX wieku.
\index[persons]{Bevan, Benjamin}%
\index[persons]{Butterworth, John}%
To, że środki boków i spodki wysokości leżą na wspólnym okręgu, zostało zauważone w 1821 roku przez Charles Brianchona i Jean-Victora Ponceleta.
\index[persons]{Brianchon, Charles}%
\index[persons]{Poncelet, Jean-Victor}%
Tego samego odkrycia dokonał rok później Karl Feuerbach; a krótko po nim Olry Terquem zauważył, że leży na nim dziewięć, a nie tylko sześć wspomnianych punktów.
\todofoot{The nine-point circle also passes through Kimberling centers Xi for i=11 (the Feuerbach point), 113, 114, 115 (center of the Kiepert hyperbola), 116, 117, 118, 119, 120, 121, 122, 123, 124, 125 (center of the Jerabek hyperbola), 126, 127, 128, 129, 130, 131, 132, 133, 134, 135, 136, 137, 138, 139, 1312, 1313, 1560, 1566, 2039, 2040, and 2679.}
\todofoot{Karl Wilhelm Feuerbach's Eigenschaften einiger merkwiirdigen Punkte des geradlinigen Dreiecks, along with many other interesting proofs relating to the nine point circle.}
\index[persons]{Feuerbach, Karl}%
\index[persons]{Terquem, Olry}%
Terquemowi (1842) zawdzięczamy nazwę ,,okrąg dziewięciu punktów''.
\todofoot{The circle is officially designated the "nine point circle" (le cercle des neuf points) by Terquem, one of the editors of the Nouvelles Annales. (see Volume I page 198).}

Feuerbach udowodnił też, że:

\begin{theorem}[Feuerbacha]
\label{punkt_feuerbacha}%
	Okrąg dziewięciu punktów jest styczny wewnętrznie do okręgu wpisanego (w punkcie Feuerbacha) i zewnętrznie do trzech okręgów dopisanych.
	\index{twierdzenie!Feuerbacha}%
\end{theorem}

Coxeter \cite[s. 99]{coxeter_1967}, Audin \cite[s. 110]{audin_2003} podają ten fakt w formie ćwiczenia.

punkt Torricellego/Fermata (Guzicki-8)
Audin \cite[s. 105]{audin_2003} podaje ten fakt w formie ćwiczenia.
% problem Fermeta?
% TODO: jaki fakt dokładnie podaje

\begin{proposition}
	\label{orthic_triangle}
	Niech $ABC$ będzie trójkątem ostrokątnym, zaś $K$, $L$ oraz $M$ spodkami jego wysokości.
	Wtedy wysokości trójkąta $ABC$ są dwusiecznymi kątów trójkąta $KLM$.
\end{proposition}

Hartshorne \cite[s. 58]{hartshorne2000}.

%

% UW niezrobione:
\subsubsection{W powijakach: twierdzenie Brianchona}
twierdzenie Brianchona

\subsubsection{W powijakach: okręgi współpękowe}
okręgi współpękowe

\subsubsection{W powijakach: twierdzenie Gaussa-Bodenmillera}
twierdzenie Gaussa-Bodenmillera

\subsubsection{W powijakach: twierdzenie Ponceleta dla trójkąta.}
twierdzenie Ponceleta dla trójkąta


