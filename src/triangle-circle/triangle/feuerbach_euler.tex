%

Prosta Eulera to pierwsza w szkolnej geometrii trójka punktów współliniowych.
Przyszła na świat w~1765 roku, w żurnalu \emph{Novi commentarii Academiae Petropolitanae}, w artykule \emph{Solutio facilis problematum quorundam geometricorum difficillimorum}, czyli jakbyśmy powiedzieli po polsku, ,,Łatwe rozwiązanie niektórych najtrudniejszych problemów geometrycznych''.

\begin{proposition}[prosta Eulera]
	\label{prosta_eulera}
	Środek okręgu opisanego na nierównobocznym trójkącie, środek ciężkości oraz ortocentrum leżą na jednej prostej, zwanej prostą Eulera.
	\index{prosta!Eulera}
\end{proposition}

Piszą o niej
Coxeter \cite[s. 32, 33]{coxeter_1967},
Hartshorne \cite[s. 54, 55]{hartshorne2000},
Bogdańska, Neugebauer \cite[s. 84]{neugebauer_2018},
Audin \cite[s. 61]{audin_2003}.
W $\Delta_{84}^{4}$ podany będzie przepis, jak skonstruować trójkąt, w którym prosta Eulera ma zadane położenie względem podstawy.

\begin{proposition}
	Prosta Eulera jest prostopadła do jednego z boków trójkąta wtedy i tylko wtedy, gdy trójkąt jest równoramienny.
\end{proposition}

\begin{proposition}[okrąg dziewięciu punktów]
\label{okrag_dziewieciu_punktow}%
	W każdym trójkącie środki boków, spodki wysokości oraz środki odcinków łączących ortocentrum z wierzchołkami leżą na jednym okręgu.
	Jego środek pokrywa się ze środkiem odcinka łączącego środek okręgu opisanego z ortocentrum, zaś jego promień jest dwukrotnie krótszy od promienia okręgu opisanego.
	\index{okrąg!dziewięciu punktów}
\end{proposition}

% TODO: https://en.wikipedia.org/wiki/Nine-point_circle

Coxeter \cite[s. 34, 35, 88]{coxeter_1967}, Hartshorne \cite[s. 57, 60]{hartshorne2000}, Bogdańska, Neugebauer \cite[s. 85, 86]{neugebauer_2018}.
Audin \cite[s. 62]{audin_2003}.

Temat badali Benjamin Bevan (który zasugerował środek oraz promień) i John Butterworth (który udowodnił podejrzenia Bevana) na początku XIX wieku.
\index[persons]{Bevan, Benjamin}%
\index[persons]{Butterworth, John}%
To, że środki boków i spodki wysokości leżą na wspólnym okręgu, zostało zauważone w 1821 roku przez Charles Brianchona i Jean-Victora Ponceleta.
\index[persons]{Brianchon, Charles}%
\index[persons]{Poncelet, Jean-Victor}%
Tego samego odkrycia dokonał rok później Karl Feuerbach; a krótko po nim Olry Terquem zauważył, że leży na nim dziewięć, a nie tylko sześć wspomnianych punktów.
\todofoot{The nine-point circle also passes through Kimberling centers Xi for i=11 (the Feuerbach point), 113, 114, 115 (center of the Kiepert hyperbola), 116, 117, 118, 119, 120, 121, 122, 123, 124, 125 (center of the Jerabek hyperbola), 126, 127, 128, 129, 130, 131, 132, 133, 134, 135, 136, 137, 138, 139, 1312, 1313, 1560, 1566, 2039, 2040, and 2679.}
\todofoot{Karl Wilhelm Feuerbach's Eigenschaften einiger merkwiirdigen Punkte des geradlinigen Dreiecks, along with many other interesting proofs relating to the nine point circle.}
\index[persons]{Feuerbach, Karl}%
\index[persons]{Terquem, Olry}%
Terquemowi (1842) zawdzięczamy nazwę ,,okrąg dziewięciu punktów''.
\todofoot{The circle is officially designated the "nine point circle" (le cercle des neuf points) by Terquem, one of the editors of the Nouvelles Annales. (see Volume I page 198).}

Feuerbach udowodnił też, że:

\begin{theorem}[Feuerbacha]
\label{punkt_feuerbacha}%
	Okrąg dziewięciu punktów jest styczny wewnętrznie do okręgu wpisanego (w punkcie Feuerbacha) i zewnętrznie do trzech okręgów dopisanych.
	\index{twierdzenie!Feuerbacha}%
\end{theorem}

Coxeter \cite[s. 99]{coxeter_1967}, Audin \cite[s. 110]{audin_2003} podają ten fakt w formie ćwiczenia.

punkt Torricellego/Fermata (Guzicki-8)
Audin \cite[s. 105]{audin_2003} podaje ten fakt w formie ćwiczenia.
% problem Fermeta?
% TODO: jaki fakt dokładnie podaje

\begin{proposition}
	\label{orthic_triangle}
	Niech $ABC$ będzie trójkątem ostrokątnym, zaś $K$, $L$ oraz $M$ spodkami jego wysokości.
	Wtedy wysokości trójkąta $ABC$ są dwusiecznymi kątów trójkąta $KLM$.
\end{proposition}

Hartshorne \cite[s. 58]{hartshorne2000}.

%