%

% TODO: https://www.cut-the-knot.org/Curriculum/Geometry/HeronsProblem.shtml#explanation

\index{wzór!Herona|(}
\begin{proposition}[wzór Herona, 60 r.n.e.?]
    Pole trójkąta o bokach długości $a, b, c$ wyraża się wzorem
    \begin{equation}
        S = \sqrt{p(p-a)(p-b)(p-c)},
    \end{equation}
    gdzie $p$ oznacza połowę obwodu.
\end{proposition}

Wynik przypisujemy obecne Heronowi z Aleksandrii, który podał dość skomplikowany dowód w dziele \emph{,,Metrica''}, chociaż mógł być znany nawet Archimedesowi dwieście lat wcześniej.
\index[persons]{Heron z Aleksandrii}%
Coxeter \cite[s. 12]{coxeter_1991} przeczytał van der Waerdena \cite[s. 228, 277]{waerden_1961} i przez to twierdzi, że wzór był już wcześniej znany Archimedesowi.
\index[persons]{Archimedes}%
Guzicki \cite[s. 165-169]{guzicki_2021} wyprowadza go z twierdzenia Pitagorasa i~wspomina, że prosty geometryczny dowód podał też Euler.
\index[persons]{Euler, Leonhard}%
\index{twierdzenie!Pitagorasa}
Wreszcie Bogdańska, Neugebauer \cite[s. 92]{neugebauer_2018} korzystają z twierdzenia cosinusów.

Istnieje częściowe uogólnienie wzoru Herona dla czworokątów:

\begin{proposition}[wzór Brahmagupty]
    \label{brahmagupta_formula}
    Pole czworokąta wypukłego, na którym można opisać okrąg, o bokach długości $a, b, c, d$ wyraża się wzorem
    \begin{equation}
        S = \sqrt{(p-a)(p-b)(p-c)(p-d)},
    \end{equation}
    gdzie $p$ oznacza połowę obwodu.
\end{proposition}

Dowód polega na zastosowaniu wzoru Herona dwa razy do trójkątów podobnych lub ponownym skorzystaniu z dobrodziejstw trygonometrii.
Założenie o wypukłości i cykliczności można pominąć, jak odkryli równocześnie Carl Bretschneider i Karl von Staudt:
\index[persons]{von Staudt, Karl}%
\index[persons]{Bretschneider, Carl}%

\begin{proposition}[wzór Bretschneidera, 1842]
    Pole czworokąta o bokach długości $a, b, c, d$ i sumie miar dwóch przeciwległych kątów wewnętrznych $2 \varphi$ wyraża się wzorem
    \begin{equation}
        S = \sqrt{(p-a)(p-b)(p-c)(p-d) - abcd \cos^2  \varphi},
    \end{equation}
    gdzie $p$ oznacza połowę obwodu.
\end{proposition}

Inny, równie przydatny wzór znalazł Julian Coolidge:
\todofoot{J. L. Coolidge, "A Historically Interesting Formula for the Area of a Quadrilateral", American Mathematical Monthly, 46 (1939) pp. 345-347.}
\index[persons]{Coolidge, Julian}%
pole czworokąta o bokach długości $a, b, c, d$ i przekątnych długości $e, f$ wyraża się wzorem
\begin{equation}
    S = \sqrt{(p-a)(p-b)(p-c)(p-d) - \frac 1 4 (ac + bd + pq)(ac + bd - pq)},
\end{equation}
gdzie $p$ oznacza połowę obwodu.

\index{wzór!Herona|)}

%