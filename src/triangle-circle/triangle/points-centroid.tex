\begin{definition}[środkowa]
\index{środkowa}%
    Niech $\triangle ABC$ będzie trójkątem.
    Odcinek łaczący wierzchołek (na przykład $A$) ze środkiem przeciwległego boku (w naszym przykładzie $BC$) nazywamy środkową.
\end{definition}

Środkowe dzielą trójkąt na sześć mniejszych o równych polach.

\begin{proposition}
\label{srodkowe_przecinaja_sie}%
\index{środek ciężkości}%
    Trzy środkowe trójkąta przecinają się w jednym punkcie zwanym środkiem ciężkości i dzielą się w~stosunku $2$ do $1$, licząc od wierzchołków.
\end{proposition}
% % Coxeter, Introduction to Geometry, s. 10 <- przeczytaj to, nie tylko cytuj! + ćwiczenia: 3/4 <= 1

Środkowe to po angielsku \emph{medians}, przecinają się w \emph{centroid}.
Polska nazwa bierze się z tego, że środek ciężkości fizycznego modelu trójkąta wykonanego z jednolitego materiału znajduje się właśnie tam.
\index{punkt!Spiekera}%
Angielska nazwa powstanie w 1814 roku z powodu niechęci do tego, co nie jest czysto geometryczne; niechęci, której nie będzie widać w innych europejskich językach.
(Środek ciężkości brzegu trójkąta nazywa się punktem Spiekera, patrz fakt \ref{punkt_spiekera}).

Hartshorne \cite[s. 52-54]{hartshorne2000} wnioskuje \ref{srodkowe_przecinaja_sie} z faktu \ref{hartshorne_52} (później zaś \cite[s. 119-120]{hartshorne2000} powtórzy dowód z~maszynerią geometrii analitycznej).
\index[persons]{Archimedes}%
Podobnie zrobią Guzicki \cite[s. 220]{guzicki_2021} albo Bogdańska, Neugebauer, z tym że ta dwójka uwikła twierdzenie Talesa w uzasadnienie \ref{hartshorne_52} z twierdzenia Talesa.
\index{twierdzenie!Talesa}%
Zetel \cite[s. 14]{zetel_2020} pokaże, że twierdzenie Cevy daje natychmiastowy dowód istnienia środka ciężkości.
% TODO: Ich długość można obliczyć z https://en.wikipedia.org/wiki/Apollonius%27s_theorem => to jest z https://en.wikipedia.org/wiki/Median_(geometry)#Formulas_involving_the_medians'_lengths

Historia faktów \ref{wysokosci_przecinaja_sie} oraz \ref{srodkowe_przecinaja_sie} ma wiele elementów wspólnych, więc omówimy je razem.
Żaden z nich nie pojawi się w Elementach Euklidesach i właściwie nie wiadomo, kto odkryje je pierwszy.
Uczeni arabscy stwierdzą, że był to Archimedes, choć nie ma ku temu niezbitych dowodów.
\index[persons]{Archimedes}%
Pappus będzie świadom istnienia ortocentrum, ale najstarsze znane uzasadnienie (w komentarzu al-Nasawiego) pochodzi z XI wieku.
\index[persons]{Pappus}%
\index[persons]{al-Nasawi, Ali ibn Ahmad}%

Fakt \ref{srodkowe_przecinaja_sie} ma trójwymiarową kontynuację:

\begin{proposition}
    Cztery odcinki łączące wierzchołki czworościanu ze środkami przeciwległych ścian przecinają się w~jednym punkcie, który dzieli je w stosunku $3$ do $1$.
\end{proposition}

Niektórzy stosują określenie ,,twierdzenie Commandino'', ponieważ Federico Commandino napisze w 1565 roku pracę \emph{De centro gravitatis solidorum} (O środkach ciężkości brył), chociaż może nie być pierwszym, który je odkrył.
\index{twierdzenie!Commandino}%
\index[persons]{Commandino, Federico}%
Podejrzewa się, że Francesco Maurolico pozna je wcześniej.
(Friederich Eduard Reusch znajdzie uogólnienie, które w zdegenerowanym przypadku prowadzi znowu do twierdzenia \ref{theorem_varignon} Varignona).
\index[persons]{Reusch, Friederich Eduard}%