Z każdym trójkątem związane są pewne specjalne punkty, internetowa lista \emph{Encyclopedia of Triangle Centers} wymieni ich co najmniej 68 547.
My nie mamy aż tyle miejsca, więc ograniczymy się do najważniejszych.

$X(1)$ to środek okręgu wpisanego, 
$X(2)$ to środek ciężkości,
$X(3)$ to środek okręgu opisanego,
$X(4)$ to ortocentrum.
Te cztery punkty opisujemy poniżej.

$X(5)$ to środek okręgu dziewięciu punktów z faktu \ref{okrag_dziewieciu_punktow},
$X(6)$ to punkt Lemoine'a (Grebego) z faktu \ref{punkt_lemoine},
$X(7)$ to punkt Gergonne'a z~faktu \ref{punkt_gergonne},
$X(8)$ to punkt Nagela z faktu \ref{punkt_nagela}, 
$X(9)$ to mittenpunkt (punkt Lemoine'a trójkąta rozpiętego przez środki okręgów dopisanych; leży na prostych łączących środek ciężkości z punktem Gergonne'a, środek okręgu wpisanego z punktem Lemoine'a oraz ortocentrum ze środkiem Spiekera), % https://en.wikipedia.org/wiki/Mittenpunkt
$X(10)$ to środek okręgu Spiekera z faktu \ref{punkt_spiekera},
$X(11)$ to punkt Feuerbacha z twierdzenia \ref{punkt_feuerbacha},
$X(13)$ to punkt Fermata z zadania \ref{punkt_fermata},
$X(17)$, $X(18)$ to punkty Napoleona,
$X(39)$ to środek odcinka łączącego punkty Brocarda z definicji \ref{punkty_brocarda}.
Lista jest długa, a jej końca nie widać.

\subsubsection{Trzy symetralne}
\begin{proposition}
    \label{symetralne_przecinaja_sie}
    Trzy symetralne boków trójkąta przecinają się w jednym punkcie, środku okręgu opisanego na tym trójkącie.
\end{proposition}

Wynika to bezpośrednio z uwagi za definicją \ref{def_symetralna}: w trójkącie $\triangle ABC$, symetralna boku $AB$ (odpowiednio: $AC$) zawiera środki okręgów przechodzących przez punkty $A$ oraz $B$ (przez punkty $A$ oraz $C$), zatem trzecia symetralna musi przejść przez punkt wspólny dwóch pierwszych.

Hartshorne \cite[s. 16]{hartshorne2000} podaje to w formie ćwiczenia ze wskazówką, by spojrzeć na (IV.5).
Audin \cite[s. 61]{audin_2003} też, ale bez wskazówki.
Inne odsyłacze to Neugebauer \cite[s. 19]{neugebauer_2018}.

\begin{proposition}
	Środek okręgu opisanego na trójkącie leży wewnątrz tego trójkąta (odpowiednio: na jego brzegu, na zewnątrz trójkąta) wtedy i tylko wtedy, gdy trójkąt jest ostrokątny (odpowiednio: prostokątny, rozwartokątny).
\end{proposition}

Uogólnieniem faktu \ref{symetralne_przecinaja_sie} o współpękowości symetralnych boków trójkąta jest:

\begin{theorem}[Carnota]
\index{twierdzenie!Carnota}%
\label{guzicki_6_13}%
	Dany jest trójkąt $ABC$ i punkty $D, E, F$ leżące odpowiednio na prostych $BC, CA, AB$.
	Niech prosta $k$ (odpowiednio: $l$, $m$) przechodzi przez punkt $D$ ($E$, $F$) i będzie prostopadła do prostej $BC$ ($CA$, $AB$).
	Wtedy proste $k$, $l$, $m$ mają punkt wspólny wtedy i tylko wtedy, gdy
	\begin{equation}
		|AF|^2 + |BD|^2 + |CE|^2 = |AE|^2 + |BF|^2 + |CD|^2.
	\end{equation}
\end{theorem}
% TODO: https://en.wikipedia.org/wiki/Carnot%27s_theorem_(perpendiculars)

Guzicki \cite[s. 176]{guzicki_2021} wyprowadza je z twierdzenia Pitagorasa, co pozwala mu dojść do wniosku, że okręgi opisany i wpisany oraz ortocentrum istnieją.
\index{twierdzenie!Pitagorasa}

\subsubsection{Trzy dwusieczne}
\begin{proposition}
    Trzy dwusieczne kątów trójkąta przecinają się w jednym punkcie, środku okręgu wpisanego w ten trójkąt.
\end{proposition}

Uzasadnienie jest kompletnie analogiczne.
Po angielsku mamy krótkie określenia \emph{excenter, excircle, incenter, incircle}.
Hartshorne \cite[s. 16]{hartshorne2000} podaje to w formie ćwiczenia ze wskazówką, by spojrzeć na (IV.4), z czym totalnie się zgadzamy.

\subsubsection{Trzy wysokości}
\begin{definition}[wysokość]
\index{wysokość trójkąta}%
    Niech $\triangle ABC$ będzie trójkątem, w którym przez punkt $C$ poprowadzono prostą prostopadłą do boku $AB$.
    Odcinek leżący na tej prostej, którego jeden koniec znajduje się w punkcie $C$, zaś drugi leży na boku $AB$, nazywamy wysokością opuszczoną z punktu $C$, drugi jego koniec to spodek wysokości.
\end{definition}

% TODO: Sometimes an arbitrary edge is chosen to be the base, in which case the opposite vertex is called the apex; the shortest segment between the base and apex is the height. The area of a triangle equals one-half the product of height and base length.

\begin{proposition}
\label{wysokosci_przecinaja_sie}%
	Trzy wysokości trójkąta (albo w przypadku trójkąta rozwartokątnego, przedłużenia wysokości) przecinają się w jednym punkcie zwanym ortocentrum (dawniej: środku ortycznym).
\index{ortocentrum}%
\end{proposition}

Dowodów we współczesnej literaturze nie brakuje: są u Pompego \cite[s. 38]{pompe_2022}, gdzie zmyślnie używa równoległoboków albo Guzickiego \cite[s. 218]{guzicki_2021}.
Warto też przeczytać tekst Hartshorne'a \cite[s. 54]{hartshorne2000} oraz Audina \cite[s. 61]{audin_2003}.
(Po angielsku mamy \emph{altitude}, \emph{foot of the altitude}, \emph{orthocenter}).

\subsubsection{Trzy środkowe}
\begin{definition}[środkowa]
\index{środkowa}%
    Niech $\triangle ABC$ będzie trójkątem.
    Odcinek łaczący wierzchołek (na przykład $A$) ze środkiem przeciwległego boku (w naszym przykładzie $BC$) nazywamy środkową.
\end{definition}

Środkowe dzielą trójkąt na sześć mniejszych o równych polach.

\begin{proposition}
\label{srodkowe_przecinaja_sie}%
\index{środek ciężkości}%
    Trzy środkowe trójkąta przecinają się w jednym punkcie zwanym środkiem ciężkości i dzielą się w~stosunku $2$ do $1$, licząc od wierzchołków.
\end{proposition}
% % Coxeter, Introduction to Geometry, s. 10 <- przeczytaj to, nie tylko cytuj! + ćwiczenia: 3/4 <= 1

Środkowe to po angielsku \emph{medians}, przecinają się w \emph{centroid}.
Polska nazwa bierze się z tego, że środek ciężkości fizycznego modelu trójkąta wykonanego z jednolitego materiału znajduje się właśnie tam.
\index{punkt!Spiekera}%
Angielska nazwa powstanie w 1814 roku z powodu niechęci do tego, co nie jest czysto geometryczne; niechęci, której nie będzie widać w innych europejskich językach.
(Środek ciężkości brzegu trójkąta nazywa się punktem Spiekera, patrz fakt \ref{punkt_spiekera}).

Hartshorne \cite[s. 52-54]{hartshorne2000} wnioskuje \ref{srodkowe_przecinaja_sie} z faktu \ref{hartshorne_52} (później zaś \cite[s. 119-120]{hartshorne2000} powtórzy dowód z~maszynerią geometrii analitycznej).
\index[persons]{Archimedes}%
Podobnie zrobią Guzicki \cite[s. 220]{guzicki_2021} albo Bogdańska, Neugebauer, z tym że ta dwójka uwikła twierdzenie Talesa w uzasadnienie \ref{hartshorne_52} z twierdzenia Talesa.
\index{twierdzenie!Talesa}%
Zetel \cite[s. 14]{zetel_2020} pokaże, że twierdzenie Cevy daje natychmiastowy dowód istnienia środka ciężkości.
% TODO: Ich długość można obliczyć z https://en.wikipedia.org/wiki/Apollonius%27s_theorem => to jest z https://en.wikipedia.org/wiki/Median_(geometry)#Formulas_involving_the_medians'_lengths

Historia faktów \ref{wysokosci_przecinaja_sie} oraz \ref{srodkowe_przecinaja_sie} ma wiele elementów wspólnych, więc omówimy je razem.
Żaden z nich nie pojawi się w Elementach Euklidesach i właściwie nie wiadomo, kto odkryje je pierwszy.
Uczeni arabscy stwierdzą, że był to Archimedes, choć nie ma ku temu niezbitych dowodów.
\index[persons]{Archimedes}%
Pappus będzie świadom istnienia ortocentrum, ale najstarsze znane uzasadnienie (w komentarzu al-Nasawiego) pochodzi z XI wieku.
\index[persons]{Pappus}%
\index[persons]{al-Nasawi, Ali ibn Ahmad}%

Fakt \ref{srodkowe_przecinaja_sie} ma trójwymiarową kontynuację:

\begin{proposition}
    Cztery odcinki łączące wierzchołki czworościanu ze środkami przeciwległych ścian przecinają się w~jednym punkcie, który dzieli je w stosunku $3$ do $1$.
\end{proposition}

Niektórzy stosują określenie ,,twierdzenie Commandino'', ponieważ Federico Commandino napisze w 1565 roku pracę \emph{De centro gravitatis solidorum} (O środkach ciężkości brył), chociaż może nie być pierwszym, który je odkrył.
\index{twierdzenie!Commandino}%
\index[persons]{Commandino, Federico}%
Podejrzewa się, że Francesco Maurolico pozna je wcześniej.
(Friederich Eduard Reusch znajdzie uogólnienie, które w zdegenerowanym przypadku prowadzi znowu do twierdzenia \ref{theorem_varignon} Varignona).
\index[persons]{Reusch, Friederich Eduard}%

%