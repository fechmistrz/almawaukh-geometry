%

\index{pons asinorum|(}

\subsubsection{Pons asinorum}
\index{most osłów patrz pons asinorum}
Most osłów (łacińskie \emph{,,pons asinorum''}) to tradycyjna nazwa dowodu twierdzenia, że kąty przy podstawie trójkąta równoramiennego są równe.
Podał go Euklides jako teza V w księdze I Elementów.
Mawiało się, że ci, którzy nie są w stanie samodzielnie przeprowadzić tego dedukcyjnego dowodu opartego na własnościach trójkątów przystających, nie może przekroczyć mostu i studiować dalej geometrii.

Bardziej przyziemnie Coxter \cite[s. 6-9]{coxeter_1991} zauważa, że rysunek wykonany przez Euklidesa przypomina most.
Wśród konsekwencji wymienia kilka wyników z Elementów: III.3, III.20, III.21, III.22, III.32, VI.2, VI.4, a potem III.35, III.36, VI.19, co prowadzi do dowodu twierdzenia Pitagorasa, czyli I.47. % TODO: sprawdzić, czy numeracja moja i Coxetera jest taka sama.
\index{twierdzenie!Pitagorasa}%
Coxeter podaje w formie ćwiczeń nierówność Erdős-Mordella (u nas podsekcja \ref{subsection_erdos_mordell}) oraz twierdzenie Steinera-Lehmusa (twierdzenie \ref{theorem_steiner_lehmus}).
% TODO: https://www.deltami.edu.pl/1990/08/elementarny-dowod-nierownosci-erdosa-mordella/
\todofoot{Przeczytać artykuł z Delty 1990, elementarny-dowod-nierownosci-erdosa-mordella}

Pierwsze dowody tego faktu podali jeszcze Euklides, komentujący jego prace Proklos zwany Diadochem oraz Pappus z Aleksandrii.
Współcześnie podaje się krótkie uzasadnienie w oparciu o dwusieczną kąta, ale Euklides nie mógł tak uczynić, ponieważ definiuje ją dopiero cztery tezy później w swoich Elementach.

% TODO: https://en.wikipedia.org/wiki/Pons_asinorum

\index{pons asinorum|)}

%