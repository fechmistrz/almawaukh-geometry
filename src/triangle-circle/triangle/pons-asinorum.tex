%

,,\emph{Pons asinorum}'', czyli most osłów, to tradycyjna nazwa twierdzenia (I.5), że kąty przy podstawie trójkąta równoramiennego są równe.
Ci, którzy nie są w stanie samodzielnie przeprowadzić jego dedukcyjnego dowodu opartego na własnościach trójkątów przystających, nie mogą przekroczyć mostu i studiować dalej geometrii.
Bardziej przyziemnie Coxeter \cite[s. 22-24]{coxeter_1967} zauważa, że rysunek wykonany przez Euklidesa przypomni most.
Wśród konsekwencji wymienia wyniki z~Elementów: (III.3), (III.20), (III.21), (III.22), (III.32), (VI.2), (VI.4), a potem (III.35), (III.36), (VI.19), co prowadzi do dowodu twierdzenia Pitagorasa, czyli (I.47). % TODO: sprawdzić, czy numeracja moja i Coxetera jest taka sama.
\index{twierdzenie!Pitagorasa}%

\begin{figure}[H] \centering
\begin{comment}
\begin{tikzpicture}[scale=.5]
    \tkzDefPoint(90:-1){A}
    \tkzDefPoint(-55:5){C}
    \tkzDefPoint(235:5){B}
    \tkzDefPoint(-90:8){X}

    \tkzLabelPoint[above](A){$A$}
    \tkzLabelPoint[left](B){$B$}
    \tkzLabelPoint[right](C){$C$}
    \tkzInterLC(A,B)(A,X) \tkzGetPoints{XX}{D} % line and circle
    \tkzLabelPoint[left](D){$D$}
    \tkzDefLine[parallel=through D](B,C) \tkzGetPoint{XXX}
    \tkzInterLL(D,XXX)(A,C) \tkzGetPoint{E} % line and circle
    \tkzLabelPoint[right](E){$E$}
    
    \tkzMarkSegments[mark=|](A,B A,C)
    \tkzMarkSegments[mark=||](B,D C,E)
    \tkzDrawLines[add= 0 and 0, line width=0.2mm](B,E C,D)
    \tkzDrawLines[add= 0 and 0.5, line width=0.2mm](B,D C,E)
    \tkzDrawPolygon[line width=0.5mm](A,B,C)
    \tkzDrawPoints[size=4,color=black,fill=black!50](A,B,C,D,E)
\end{tikzpicture}
\end{comment}
    \caption{most osłów}
\end{figure}

Pierwsze dowody tego faktu podadzą jeszcze Euklides, komentujący jego prace Proklos, a także (dużo krócej\footnote{Pappus zauważa, że trójkąt $\triangle ABC$ przystaje do siebie $\triangle ACB$, więc stosowne kąty przy podstawie też są przystajace.}) Pappus z Aleksandrii.
\index[persons]{Proklos zwany Diadochem}%
\index[persons]{Pappus z Aleksandrii}%
Przyszłość przyniesie jeszcze jedno uzasadnienie, zaczynające się od wykreślenia dwusiecznej z kąta przy wierzchołku.
\index{dwusieczna}%
Euklides nie zrobi tego przede wszystkim ze względu na kolejność wykładanego materiału: dwusieczna pojawi się cztery tezy później, a nie można korzystać z wyników, których prawdziwości dopiero się pokaże.

% PRZECZYTANO: https://en.wikipedia.org/wiki/Pons_asinorum

%