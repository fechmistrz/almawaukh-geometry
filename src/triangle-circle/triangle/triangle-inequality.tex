
\subsection{Nierówność trójkąta}

\begin{proposition}[nierówność trójkąta]
	Niech $ABC$ będzie trojkątem.
	Wtedy suma odcinków $AB$ i $BC$ jest dłuższa niż $AC$.
	\index{nierówność trójkąta}
\end{proposition}

Nierówność trójkąta nie jest wnioskiem z aksjomatów I1-I3, B1-B4, C1-C3, ponieważ nie zachodzi w następującym modelu: płaszczyzna to zbiór $\mathbb R^2$, ze standardowymi punktami i prostymi, ale niestandardową metryką
\begin{equation}
	d((x_1, y_1), (x_2, y_2)) = \begin{cases}
		\sqrt{(x_1-x_2)^2 + (y_1-y_2)^2} & \text{jeśli } x_1 = x_2 \vee y_1 = y_2, \\
		2 \sqrt{(x_1-x_2)^2 + (y_1-y_2)^2} & \text{w przeciwnym wypadku}
	\end{cases}.
\end{equation}

(Powyższy przykład opisał Hartshorne \cite[s. 90]{hartshorne2000}).