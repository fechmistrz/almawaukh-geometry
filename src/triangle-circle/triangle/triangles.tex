\subsection{Trójkąty}

\subsubsection{Cechy przystawania}
Cechy przystawania
\loremipsum
Hartshorne s. 99

%

\index{pons asinorum|(}

\subsubsection{Pons asinorum}
\index{most osłów patrz pons asinorum}
Most osłów (łacińskie \emph{,,pons asinorum''}) to tradycyjna nazwa dowodu twierdzenia, że kąty przy podstawie trójkąta równoramiennego są równe.
Podał go Euklides jako teza V w księdze I Elementów.
Mawiało się, że ci, którzy nie są w stanie samodzielnie przeprowadzić tego dedukcyjnego dowodu opartego na własnościach trójkątów przystających, nie może przekroczyć mostu i studiować dalej geometrii.

Bardziej przyziemnie Coxter \cite[s. 6-9]{coxeter_1991} zauważa, że rysunek wykonany przez Euklidesa przypomina most.
Wśród konsekwencji wymienia kilka wyników z Elementów: III.3, III.20, III.21, III.22, III.32, VI.2, VI.4, a potem III.35, III.36, VI.19, co prowadzi do dowodu twierdzenia Pitagorasa, czyli I.47. % TODO: sprawdzić, czy numeracja moja i Coxetera jest taka sama.
\index{twierdzenie!Pitagorasa}%
Coxeter podaje w formie ćwiczeń nierówność Erdős-Mordella (u nas podsekcja \ref{subsection_erdos_mordell}) oraz twierdzenie Steinera-Lehmusa (twierdzenie \ref{theorem_steiner_lehmus}).
% TODO: https://www.deltami.edu.pl/1990/08/elementarny-dowod-nierownosci-erdosa-mordella/
\todofoot{Przeczytać artykuł z Delty 1990, elementarny-dowod-nierownosci-erdosa-mordella}

Pierwsze dowody tego faktu podali jeszcze Euklides, komentujący jego prace Proklos zwany Diadochem oraz Pappus z Aleksandrii.
Współcześnie podaje się krótkie uzasadnienie w oparciu o dwusieczną kąta, ale Euklides nie mógł tak uczynić, ponieważ definiuje ją dopiero cztery tezy później w swoich Elementach.

O moście osłów piszą Coxeter 

\index{pons asinorum|)}

%

%

\begin{proposition}[nierówność trójkąta]
	Niech $ABC$ będzie trojkątem.
	Wtedy suma odcinków $AB$ i $BC$ jest dłuższa niż $AC$.
	\index{nierówność trójkąta}
\end{proposition} % https://en.wikipedia.org/wiki/Triangle_inequality#Euclidean_geometry I.20

Nierówność trójkąta nie jest wnioskiem z aksjomatów I1-I3, B1-B4, C1-C3, ponieważ nie zachodzi w następującym modelu: płaszczyzna to zbiór $\mathbb R^2$, ze standardowymi punktami i prostymi, ale niestandardową metryką
\begin{equation}
	d((x_1, y_1), (x_2, y_2)) = \begin{cases}
		\sqrt{(x_1-x_2)^2 + (y_1-y_2)^2} & \text{jeśli } x_1 = x_2 \vee y_1 = y_2, \\
		2 \sqrt{(x_1-x_2)^2 + (y_1-y_2)^2} & \text{w przeciwnym wypadku}
	\end{cases}.
\end{equation}

(Powyższy przykład opisał Hartshorne \cite[s. 90]{hartshorne2000}).

%

%

\subsubsection{Twierdzenie Pitagorasa}
Najważniejszym twierdzeniem dotyczącym trójkątów prostokątnych jest twierdzenie Pitagorasa oraz twierdzenie do niego odwrotne.
Piszą o~nim Guzicki \cite[s. 160]{guzicki_2021}.

\begin{theorem}[Pitagorasa, ok. 500 r. p.n.e.]
\index{twierdzenie!Pitagorasa}%
    Niech $ABC$ będzie trójkątem prostokątnym, w~którym kąt przy wierzchołku $C$ jest prosty.
    Wtedy
    \begin{equation}
        |BC|^2 + |AC|^2 = |AB|^2.
    \end{equation}
    Odwrotnie, jeśli $ABC$ jest trójkątem takim, że $|BC|^2 + |AC|^2 = |AB|^2$, to trójkąt ten jest prostokątny, zaś kąt przy wierzchołku $C$ jest prosty.
\end{theorem}

Chociaż współcześnie powyższe twierdzenie przypisujemy Pitagorasowi z~Samos, to nie wiemy dokładnie, kto i~kiedy odkrył je jako pierwszy.
\index[persons]{Pitagoras z Samos}%
Było powszechnie stosowane w~okresie Starego Babilonu (XX-XVI wiek p.n.e.), a~więc na długo przed narodzinami Pitagorasa; pojawia się też w indyjskich i~chińskich tekstach matematycznych.
Papirus Berlin 6619 spisany ok. 1800 roku p.n.e. na terenach państwa egipskiego zawiera zadanie, którego rozwiązaniem jest trójka $(6, 8, 10)$.

Już w~szkole podstawowej uczniowie poznają trójkąt prostokątny o bokach długości $3, 4, 5$ wraz~z~legendą, że podobno Egipcjanie używali tego trójkąta do wyznaczania w terenie kątów prostych.

% TODO: rysunek z Guzickiego, stron 160

\begin{proposition}
    Mają miejsce następujące równości:
    \begin{equation}
        h = \frac{ab}{c}, \quad
        p = \frac{b^2}{c}, \quad
        q = \frac{a^2}{c}, \quad
        h^2 = pq.
    \end{equation}
\end{proposition}

Dowód wykorzystujący podobieństwa trójkątów można znaleźć u~Guzickiego \cite[s. 160, 161]{guzicki_2021}.

Twierdzenie Pitagorasa znajduje zastosowanie także przy wyznaczaniu niektórych miejsc geometrycznych.

\begin{proposition}
    Dane są dwa różne punkty $A$ i $B$ na płaszczyźnie oraz liczba rzeczywista $c$ taka, że $2c > |AB|^2$.
    Miejscem geometrycznym punktów $P$ o własności $|AP|^2 + |BP|^2 = c$ jest okrąg o środku w środku odcinka $AB$ i promieniu $r = \frac 1 2 \sqrt{2c - |AB|^2}$.
\end{proposition}

\begin{proposition}
    Dane są dwa różne punkty $A$ i $B$ na płaszczyźnie oraz liczba rzeczywista $c$.
    Miejscem geometrycznym punktów $P$ o własności $|AP|^2 - |BP|^2 = c$ jest prosta prostopadła do prostej $AB$.
\end{proposition}

Patrz Guzicki \cite[s. 170-173]{guzicki_2021} (Guzicki wprowadza potem osie i środki potęgowe jak w~fakcie \ref{guzicki_6_11}, a następnie twierdzenie \ref{guzicki_6_13} (Carnota)).

%

% https://en.wikipedia.org/wiki/Pythagorean_theorem liczne dowody, wiek Pitagorasa
% https://en.wikipedia.org/wiki/Xuan_tu
% https://en.wikipedia.org/wiki/Spiral_of_Theodorus
% https://en.wikipedia.org/wiki/Garfield%27s_proof_of_the_Pythagorean_theorem

\subsubsection{Wzór Herona}
%

\index{wzór!Herona|(}
Guzicki \cite[s. 165-168]{guzicki_2021} wyprowadza wzór Herona z twierdzenia Pitagorasa.
\index{twierdzenie!Pitagorasa}
Oryginalny dowód Herona był dość skomplikowany, Guzicki \cite[s. 168-169]{guzicki_2021} wspomina o znacznie prostszym dowodzie geometrycznym, pochodzącym od Eulera.
\index[persons]{Euler, Leonhard}%

\begin{proposition}
	Niech $ABC$ będzie trójkątem o obwodzie $2p$ oraz polu powierzchni $S$.
	Wtedy
	\begin{equation}
		S \le \frac{p^2}{3 \sqrt{3}}
	\end{equation}
\end{proposition}

Guzicki wyprowadza tę nierówność izoperymetryczną ze wzoru Herona oraz nierówności między średnią arytmetyczną i geometryczną.

\index{wzór!Herona|)}

% TODO: wzór Herona (Guzicki-6), Brahmagupty

%

\subsubsection{Symetralna i okrąg opisany}
Symetralna i okrąg opisany
\loremipsum

\subsubsection{Ortocentrum}
Ortocentrum.
\loremipsum

\subsubsection{Problemy Fagnano i Fermata}
Problemy Fagnano i Fermata
\todofoot{Coxeter, s. 20, 21}
\loremipsum

\subsubsection{Nierówności trójkątne}

\begin{proposition}[nierówność izoperymetryczna]
	Dany jest trójkąt o połowie obwodu $p$ oraz polu $S$.
	Wtedy 
	\begin{equation}
		S \le \frac{p^2}{3 \sqrt 3},
	\end{equation}
	zatem wśród trójkątów o ustalonym obwodzie największe pole ma trójkąt równoboczny.
	\index{nierówność!izoperymetryczna}
\end{proposition}

Guzicki \cite[s. 169, 170]{guzicki_2021} wyprowadza nierówność izoperymetryczną ze wzoru Herona oraz nierówności między średnią arytmetyczną i geometryczną.
\index{wzór!Herona}

% trójwymiarowy odpowiednik to hipoteza: https://math.stackexchange.com/questions/4044670/what-is-the-largest-volume-of-a-polyhedron-whose-skeleton-has-total-length-1-is

\begin{proposition}[stosunek sumy środkowych do obwodu]
	Niech... % stosunek sumy środkowych do obwodu leży między 3/4 i 1 (s. 355),
	% TODO: https://en.wikipedia.org/wiki/Isoperimetric_inequality ?
\end{proposition}

\begin{proposition}[nierówność Eulera]
	$R \ge 2r$
	% TODO: Twierdzenie Eulera: $d^2 = R^2 - 2Rr$. % Audin \cite[s. 110]{audin_2003} podaje ten fakt w formie ćwiczenia.

	% TODO: Eulera: R >= 2r https://en.wikipedia.org/wiki/Euler%27s_theorem_in_geometry
	\todofoot{formuła Eulera na odległość między środkami okręgu opisanego i wpisanego (dla trójkąta)}
	\index{nierówność!Eulera}
\end{proposition}

\begin{proposition}[nierówność Mitrinovica]
	Niech...
	\index{nierówność!Mitrinovica}
\end{proposition}

\begin{proposition}[nierówność Leibniza]
	Niech...
	\index{nierówność!Leibniza}
\end{proposition}

\begin{proposition}[nierówność Weitzenbocka]
	Niech...
	% TODO: https://en.wikipedia.org/wiki/Hadwiger–Finsler_inequality => Weitzenbock
	% TODO: https://en.wikipedia.org/wiki/Pedoe%27s_inequality => Weitzenbock
	\index{nierówność!Weitzenbocka}
\end{proposition}

Snellius-Huygens: $2 \sin x + \tan x > 3x$.
\index{nierówność!Snelliusa-Huygensa}

\index{nierówność!Erdősa-Mordella|(}
%

\label{subsection_erdos_mordell}
Erdős w 1935 roku postawi problem dowodu tej nierówności; dowód przedstawią dwa lata później Mordell i D. F. Barrow (1937), choć nie będzie on zbyt elementarny.
Później znajdzie się prostsze dowody: Kazarinoff (1957), Bankoff (1958) oraz Alsina i Nelsen (2007).
% TODO: https://en.wikipedia.org/wiki/Erdős–Mordell_inequality#CITEREFErdős1935

\begin{theorem}[nierówność Erdősa-Mordella]
    Niech $P$ będzie punktem wewnątrz trójkąta $\triangle ABC$, zaś $A_p, B_p, C_p$ spodkami punktu $P$ na boki trójkąta jak na rysunku \ref{erdos_mordell_barrowa}.
    Wtedy
    \begin{equation}
        |PA| + |PB| + |PC| \ge 2 (|PA_p| + |PB_p| + |PC_p|).
    \end{equation}
\end{theorem}


\begin{figure}[H] \centering
\begin{minipage}[b]{.45\linewidth}
\begin{center}
    \begin{comment}
    \begin{tikzpicture}[scale=.4]
    \tkzDefPoint(0, 0){A}
    \tkzDefPoint(10, 2){B}
    \tkzDefPoint(6, 7){C}
    \tkzDefPoint(5, 3){P}
    \tkzLabelPoint[below left](A){$A$}
    \tkzLabelPoint[below right](B){$B$}
    \tkzLabelPoint[above](C){$C$}
    \tkzLabelPoint[below left](P){$P$}
    \tkzDefPointsBy[projection=onto A--B](P){Pc}
    \tkzDefPointsBy[projection=onto B--C](P){Pa}
    \tkzDefPointsBy[projection=onto C--A](P){Pb}
    \tkzLabelPoint[above right](Pa){$A_p$}
    \tkzLabelPoint[above left](Pb){$B_p$}
    \tkzLabelPoint[below](Pc){$C_p$}

    \tkzDrawSegments[line width=0.2mm,dashed](P,Pa P,Pb P,Pc)
    \tkzDrawPolygon[line width=0.3mm](A,B,C)
    \tkzMarkRightAngles[size=0.5](P,Pa,C P,Pb,A P,Pc,B)
    \tkzDrawPoints[size=3,color=black,fill=black!50](A,B,C,P,Pc,Pb,Pa)
\end{tikzpicture}
\end{comment}
    \end{center}
    \subcaption{nierówność Erdősa-Mordella}
    \label{erdos_mordell_barrowa}
\end{minipage}
%
\begin{minipage}[b]{.45\linewidth}
\begin{center}\begin{comment}
    \begin{tikzpicture}[scale=.4]
    \tkzDefPoint(0, 0){A}
    \tkzDefPoint(10, 2){B}
    \tkzDefPoint(6, 7){C}
    \tkzDefPoint(5, 3){P}

    \tkzDefLine[bisector](A,P,B) \tkzGetPoint{prePc}
    \tkzInterLL(P,prePc)(A,B) \tkzGetPoint{Pc}
    \tkzDefLine[bisector](B,P,C) \tkzGetPoint{prePa}
    \tkzInterLL(P,prePa)(B,C) \tkzGetPoint{Pa}
    \tkzDefLine[bisector](C,P,A) \tkzGetPoint{prePb}
    \tkzInterLL(P,prePb)(C,A) \tkzGetPoint{Pb}

    \tkzLabelPoint[below left](A){$A$}
    \tkzLabelPoint[below right](B){$B$}
    \tkzLabelPoint[above](C){$C$}
    %\tkzLabelPoint[below left](P){$P$}
    \tkzLabelPoint[above right](Pa){$A_p$}
    \tkzLabelPoint[above left](Pb){$B_p$}
    \tkzLabelPoint[below](Pc){$C_p$}

    \tkzMarkAngle[arc=lll,size=1.2,mark=|||](A,P,Pc)
    \tkzMarkAngle[arc=lll,size=1.2,mark=|||](Pc,P,B)
    \tkzMarkAngle[arc=ll,size=1.2,mark=||](B,P,Pa)
    \tkzMarkAngle[arc=ll,size=1.2,mark=||](Pa,P,C)
    \tkzMarkAngle[arc=l,size=1.2,mark=|](C,P,Pb)
    \tkzMarkAngle[arc=l,size=1.2,mark=|](Pb,P,A)

    \tkzDrawSegments[line width=0.2mm](P,A P,B P,C)
    \tkzDrawSegments[line width=0.2mm,dashed](P,Pa P,Pb P,Pc)
    \tkzDrawPolygon[line width=0.3mm](A,B,C)
    \tkzDrawPoints[size=3,color=black,fill=black!50](A,B,C,P,Pc,Pb,Pa)
\end{tikzpicture}
\end{comment}
    \end{center}
    \subcaption{nierówność Barrowa}
    \label{erdos_mordell_barrowb}
\end{minipage}
\caption{}
\end{figure}

Twierdzenie poda w formie ćwiczenia Coxeter \cite[s. 9]{coxeter_1991}, Audin z licznymi wskazówkami \cite[s. 102]{audin_2003}.

Wzmocnieniem nierówności Erdősa-Mordella będzie nierówność Barrowa:

% TODO: https://en.wikipedia.org/wiki/Barrow%27s_inequality

\begin{theorem}[nierówność Barrowa]
    Niech $P$ będzie punktem wewnątrz trójkąta $\triangle ABC$, zaś $A_p$, $B_p$, $C_p$ punktami przecięć dwusiecznych trzech kątów wyznaczanych przez $P$ i pary wierzchołków trójkąta; tak jak na rysunku \ref{erdos_mordell_barrowb}.
    Wtedy
    \begin{equation}
        |PA| + |PB| + |PC| \ge 2 (|PA_p| + |PB_p| + |PC_p|).
    \end{equation}
\end{theorem}

Dowód Barrowa zostanie opublikowany w 1937 roku, ale nazwa ,,nierówność Barrowa'' będzie używana dopiero od 1961 roku; nie wiemy, co się wtedy stanie.
% TODO: Erdős, Paul; Mordell, L. J.; Barrow, David F. (1937), "Solution to problem 3740", American Mathematical Monthly, 44 (4): 252–254, doi:10.2307/2300713, JSTOR 2300713.

% % barrow tu jest
\index{nierówność!Erdősa-Mordella|)}

% TODO: https://en.wikipedia.org/wiki/Ono%27s_inequality

Mikołaj z Kuzy: $\sin x / x < (2 + \cos x) / 3$.
\index{nierówność!Mikołaja z Kuzy}
\index[persons]{Mikołaj z Kuzy}

\todofoot{Coxeter, s. 12}

\subsubsection{Nie wiem gdzie}

\begin{proposition}
	\label{srodkowe_przecinaja_sie}
	Środkowe trójkąta przecinają się w jednym punkcie zwanym środkiem ciężkości (po ang. \emph{centroid}?) i dzielą w stosunku $2 : 1$ licząc od wierzchołków.
\end{proposition}

Hartshorne \cite[s. 53, 54]{hartshorne2000} wnioskuje powyższe z \ref{hartshorne_52}.
Podobnie postępują Bogdańska, Neugebauer (chociaż oni wyprowadzają fakt \ref{hartshorne_52} z twierdzenia Talesa).

\begin{proposition}
	\label{wysokosci_przecinaja_sie}
	Wysokości trójkąta (proste prostopadłe do podstawy przechodzące przez wierzchołek nieleżący na niej) przecinają się w jednym punkcie zwanym ortocentrum.
\end{proposition}

Hartshorne \cite[s. 52, 54]{hartshorne2000} pisze, że ten oraz poprzedni fakt (\ref{wysokosci_przecinaja_sie}, \ref{srodkowe_przecinaja_sie}) były znane Archimedesowi.
Fakt zostaje powtórzony \cite[s. 119-120]{hartshorne2000}, by pokazać zastosowanie geometrii analitycznej.
