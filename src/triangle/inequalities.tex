
\begin{proposition}[nierówność izoperymetryczna]
	Dany jest trójkąt o połowie obwodu $p$ oraz polu $S$.
	Wtedy 
	\begin{equation}
		S \le \frac{p^2}{3 \sqrt 3},
	\end{equation}
	zatem wśród trójkątów o ustalonym obwodzie największe pole ma trójkąt równoboczny.
	\index{nierówność!izoperymetryczna}
\end{proposition}

Guzicki \cite[s. 169, 170]{guzicki_2021} wyprowadza nierówność izoperymetryczną ze wzoru Herona oraz nierówności między średnią arytmetyczną i~geometryczną.
\index{wzór!Herona}%
Odpowiednik tego w trzech wymiarach znany będzie jako hipoteza (Zdzisława Aleksandra) Melzaka: podana około roku 1965, będzie musiała długo czekać na swoje rozwiązanie.
\index[persons]{Melzak, Zdzisław Aleksander}%
% trójwymiarowy odpowiednik to hipoteza: https://math.stackexchange.com/questions/4044670/what-is-the-largest-volume-of-a-polyhedron-whose-skeleton-has-total-length-1-is

\begin{proposition}[stosunek sumy środkowych do obwodu]
	Niech... % stosunek sumy środkowych do obwodu leży między 3/4 i 1 (s. 355),
	% TODO: https://en.wikipedia.org/wiki/Isoperimetric_inequality ?
\end{proposition}

\begin{proposition}[nierówność Eulera]
\index{nierówność!Eulera}%
	Niech $R, r, d$ oznaczają kolejno promienie okręgu opisanego i wpisanego w~trójkąt oraz odległość między środkami tych okręgów.
	Wtedy
	\begin{equation}
	\frac{1}{R-d} + \frac{1}{R+d} = \frac 1 r,
	\end{equation}
	lub równoważnie $d^2 = R \cdot (R - 2r)$.
	Wynika stąd, że $R \ge 2r$.
\end{proposition}

Leonhard Euler opublikuje ten wynik w 1765 roku, chociaż wcześniej zrobi to William Chapple, w pracy \emph{,,An essay on the properties of triangles inscribed in and circumscribed about two given circles''} z 1746 roku, na dole strony 123.
% Leversha, Gerry; Smith, G. C. (November 2007), "Euler and triangle geometry", The Mathematical Gazette, 91 (522): 436–452, doi:10.1017/S0025557200182087, JSTOR 40378417, S2CID 125341434
% TODO: https://en.wikipedia.org/wiki/Euler%27s_theorem_in_geometry a stornger version... [6]

Trójwymiarowym odpowiednikiem jest:

\begin{proposition}[nierówność Grace'a-Danielssona]
\index{nierówność!Grace'a-Danielssona}%
	Dla każdej pary sfer o promieniach $r, R$ takich, że $r < R$ istnieje czworościan zawarty w dużej sferze i zawierający małą sferę wtedy i tylko wtedy, gdy
	\begin{equation}d^2 \le (R+r)(R - 3r).\end{equation}
\end{proposition}
% TODO: Grace, J.H. (1918), Proc. London Math. (ed.), Tetrahedra in relation to spheres and quadrics, vol. Soc.17, pp. 259–271
% Danielsson, G. (1952), Johan Grundt Tanums Forlag (ed.), Proof of the inequality d2≤(R+r)(R−3r) for the distance between the centres of the circumscribed and inscribed spheres of a tetrahedron, pp. 101–105

\begin{proposition}[nierówność Mitrinovica]
	Niech...
	\index{nierówność!Mitrinovica}
\end{proposition}

\begin{proposition}[nierówność Leibniza]
	Niech...
	\index{nierówność!Leibniza}
\end{proposition}

\begin{proposition}[nierówność Weitzenbocka]
	Niech...
	% TODO: https://en.wikipedia.org/wiki/Hadwiger-Finsler_inequality => Weitzenbock
	% TODO: https://en.wikipedia.org/wiki/Pedoe%27s_inequality => Weitzenbock
	% https://en.wikipedia.org/wiki/Weitzenböck's_inequality
	\index{nierówność!Weitzenbocka}
\end{proposition}
% https://www.ime.usp.br/~toscano/disc/2022/GreenbergGeometry.pdf ps. 34

Snellius-Huygens: $2 \sin x + \tan x > 3x$.
\index{nierówność!Snelliusa-Huygensa}

\index{nierówność!Erdősa-Mordella|(}
%

\label{subsection_erdos_mordell}
Erdős w 1935 roku postawi problem dowodu tej nierówności; dowód przedstawią dwa lata później Mordell i D. F. Barrow (1937), choć nie będzie on zbyt elementarny.
Później znajdzie się prostsze dowody: Kazarinoff (1957), Bankoff (1958) oraz Alsina i Nelsen (2007).
% TODO: https://en.wikipedia.org/wiki/Erdős–Mordell_inequality#CITEREFErdős1935

\begin{theorem}[nierówność Erdősa-Mordella]
    Niech $P$ będzie punktem wewnątrz trójkąta $\triangle ABC$, zaś $A_p, B_p, C_p$ spodkami punktu $P$ na boki trójkąta jak na rysunku \ref{erdos_mordell_barrowa}.
    Wtedy
    \begin{equation}
        |PA| + |PB| + |PC| \ge 2 (|PA_p| + |PB_p| + |PC_p|).
    \end{equation}
\end{theorem}


\begin{figure}[H] \centering
\begin{minipage}[b]{.45\linewidth}
\begin{center}\begin{tikzpicture}[scale=.4]
    \tkzDefPoint(0, 0){A}
    \tkzDefPoint(10, 2){B}
    \tkzDefPoint(6, 7){C}
    \tkzDefPoint(5, 3){P}
    \tkzLabelPoint[below left](A){$A$}
    \tkzLabelPoint[below right](B){$B$}
    \tkzLabelPoint[above](C){$C$}
    \tkzLabelPoint[below left](P){$P$}
    \tkzDefPointsBy[projection=onto A--B](P){Pc}
    \tkzDefPointsBy[projection=onto B--C](P){Pa}
    \tkzDefPointsBy[projection=onto C--A](P){Pb}
    \tkzLabelPoint[above right](Pa){$A_p$}
    \tkzLabelPoint[above left](Pb){$B_p$}
    \tkzLabelPoint[below](Pc){$C_p$}

    \tkzDrawSegments[line width=0.2mm,dashed](P,Pa P,Pb P,Pc)
    \tkzDrawPolygon[line width=0.3mm](A,B,C)
    \tkzMarkRightAngles[size=0.5](P,Pa,C P,Pb,A P,Pc,B)
    \tkzDrawPoints[size=3,color=black,fill=black!50](A,B,C,P,Pc,Pb,Pa)
\end{tikzpicture}\end{center}
    \subcaption{nierówność Erdősa-Mordella}
    \label{erdos_mordell_barrowa}
\end{minipage}
%
\begin{minipage}[b]{.45\linewidth}
\begin{center}\begin{tikzpicture}[scale=.4]
    \tkzDefPoint(0, 0){A}
    \tkzDefPoint(10, 2){B}
    \tkzDefPoint(6, 7){C}
    \tkzDefPoint(5, 3){P}

    \tkzDefLine[bisector](A,P,B) \tkzGetPoint{prePc}
    \tkzInterLL(P,prePc)(A,B) \tkzGetPoint{Pc}
    \tkzDefLine[bisector](B,P,C) \tkzGetPoint{prePa}
    \tkzInterLL(P,prePa)(B,C) \tkzGetPoint{Pa}
    \tkzDefLine[bisector](C,P,A) \tkzGetPoint{prePb}
    \tkzInterLL(P,prePb)(C,A) \tkzGetPoint{Pb}

    \tkzLabelPoint[below left](A){$A$}
    \tkzLabelPoint[below right](B){$B$}
    \tkzLabelPoint[above](C){$C$}
    %\tkzLabelPoint[below left](P){$P$}
    \tkzLabelPoint[above right](Pa){$A_p$}
    \tkzLabelPoint[above left](Pb){$B_p$}
    \tkzLabelPoint[below](Pc){$C_p$}

    \tkzMarkAngle[arc=lll,size=1.2,mark=|||](A,P,Pc)
    \tkzMarkAngle[arc=lll,size=1.2,mark=|||](Pc,P,B)
    \tkzMarkAngle[arc=ll,size=1.2,mark=||](B,P,Pa)
    \tkzMarkAngle[arc=ll,size=1.2,mark=||](Pa,P,C)
    \tkzMarkAngle[arc=l,size=1.2,mark=|](C,P,Pb)
    \tkzMarkAngle[arc=l,size=1.2,mark=|](Pb,P,A)

    \tkzDrawSegments[line width=0.2mm](P,A P,B P,C)
    \tkzDrawSegments[line width=0.2mm,dashed](P,Pa P,Pb P,Pc)
    \tkzDrawPolygon[line width=0.3mm](A,B,C)
    \tkzDrawPoints[size=3,color=black,fill=black!50](A,B,C,P,Pc,Pb,Pa)
\end{tikzpicture}\end{center}
    \subcaption{nierówność Barrowa}
    \label{erdos_mordell_barrowb}
\end{minipage}
\caption{}
\end{figure}

Twierdzenie poda w formie ćwiczenia Coxeter \cite[s. 9]{coxeter_1991}, Audin z licznymi wskazówkami \cite[s. 102]{audin_2003}.

Wzmocnieniem nierówności Erdősa-Mordella będzie nierówność Barrowa:

% TODO: https://en.wikipedia.org/wiki/Barrow%27s_inequality

\begin{theorem}[nierówność Barrowa]
    Niech $P$ będzie punktem wewnątrz trójkąta $\triangle ABC$, zaś $A_p$, $B_p$, $C_p$ punktami przecięć dwusiecznych trzech kątów wyznaczanych przez $P$ i pary wierzchołków trójkąta; tak jak na rysunku \ref{erdos_mordell_barrowb}.
    Wtedy
    \begin{equation}
        |PA| + |PB| + |PC| \ge 2 (|PA_p| + |PB_p| + |PC_p|).
    \end{equation}
\end{theorem}

Dowód Barrowa zostanie opublikowany w 1937 roku, ale nazwa ,,nierówność Barrowa'' będzie używana dopiero od 1961 roku; nie wiemy, co się wtedy stanie.
% TODO: Erdős, Paul; Mordell, L. J.; Barrow, David F. (1937), "Solution to problem 3740", American Mathematical Monthly, 44 (4): 252–254, doi:10.2307/2300713, JSTOR 2300713.

% % barrow tu jest
\index{nierówność!Erdősa-Mordella|)}

% TODO: https://en.wikipedia.org/wiki/Ono%27s_inequality

Mikołaj z Kuzy:
\begin{equation}
	\frac{\sin x}{x} < \frac{2 + \cos x}{3}.
\end{equation}
\index{nierówność!Mikołaja z Kuzy}
\index[persons]{Mikołaj z Kuzy}

\todofoot{Coxeter, s. 12}

% https://en.wikipedia.org/wiki/Ono's_inequality