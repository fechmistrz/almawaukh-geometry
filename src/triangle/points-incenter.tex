\begin{proposition}
    Trzy dwusieczne kątów trójkąta przecinają się w jednym punkcie, środku okręgu wpisanego w ten trójkąt.
\end{proposition}

Uzasadnienie jest kompletnie analogiczne do istnienia środku okręgu opisanego, ale można też podać fikuśny dowód dookoła: z twierdzenia o dwusiecznej i Cevy, jak Zetel \cite[s. 14]{zetel_2020}.
Hartshorne \cite[s. 16]{hartshorne2000} podaje to w formie ćwiczenia ze wskazówką, by spojrzeć na (IV.4), z czym totalnie się zgadzamy.

Po angielsku mamy krótkie określenia \emph{excenter, excircle, incenter, incircle}.
