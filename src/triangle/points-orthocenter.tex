\begin{definition}[wysokość]
\index{wysokość trójkąta}%
    Niech $\triangle ABC$ będzie trójkątem, w którym przez punkt $C$ poprowadzono prostą prostopadłą do boku $AB$.
    Odcinek leżący na tej prostej, którego jeden koniec znajduje się w punkcie $C$, zaś drugi leży na boku $AB$, nazywamy wysokością opuszczoną z punktu $C$, drugi jego koniec to spodek wysokości.
\end{definition}

% TODO: Sometimes an arbitrary edge is chosen to be the base, in which case the opposite vertex is called the apex; the shortest segment between the base and apex is the height. The area of a triangle equals one-half the product of height and base length.

\begin{proposition}
\label{wysokosci_przecinaja_sie}%
	Trzy wysokości trójkąta (albo w przypadku trójkąta rozwartokątnego, przedłużenia wysokości) przecinają się w jednym punkcie zwanym ortocentrum (dawniej: środku ortycznym).
\index{ortocentrum}%
\end{proposition}

Dowodów we współczesnej literaturze nie brakuje: są u Pompego \cite[s. 38]{pompe_2022}, gdzie zmyślnie używa równoległoboków albo Guzickiego \cite[s. 218]{guzicki_2021}.
Warto też przeczytać tekst Hartshorne'a \cite[s. 54]{hartshorne2000} oraz Audina \cite[s. 61]{audin_2003}.
(Po angielsku mamy \emph{altitude}, \emph{foot of the altitude}, \emph{orthocenter}).

\begin{proposition}
    Dany jest trójkąt $ABC$ oraz jego ortocentrum $H$.
    Odbicia symetryczne punktu $H$ względem prostych zawierających boki trójkąta $ABC$ leżą na okręgu opisanym na tym trójkącie.
\end{proposition}

Dowód tego prostego stwierdzenia przedstawia Neugebauer \cite[s. 28]{neugebauer_2018}.