
% \begin{definition}[prosta Simsona]
% \index{prosta!Wallace'a-Simsona}%
% 	Niech $P$ będzie dowolnym punktem leżącym na okręgu opisanym na trójkącie $ABC$, zaś $D$, $E$ oraz $F$ rzutami punktu $P$ na proste zawierające boki trójkąta $ABC$.
	% Wtedy punkty $D$, $E$ oraz $F$ są współliniowe, leżą na prostej zwaną prostą Simsona.
% \end{definition}

\begin{proposition}[twierdzenie Wallace'a]
	Dany jest trójkąt $ABC$ oraz punkt $P$.
	Niech $P_a$, $P_b$, $P_c$ oznacza rzuty punktu $P$ na proste $BC$, $AC$, $AB$.
	Wtedy następujące warunki są równoważne: punkt $P$ leży na okręgu opisanym na $\triangle ABC$; punkty $P_a$, $P_b$, $P_c$ są współliniowe.
\end{proposition}

Kiedy punkt $P$ leży na wspomnianym okręgu, prosta $P_aP_b$ nazywa się prostą Simsona; na cześć Roberta Simsona, szkockiego matematyka żyjącego w latach 1687-1768; chociaż pierwszą pracę na jej temat opublikuje Wallace w 1799 roku.
Być może właściwsze byłoby pisać ,,prosta Wallace'a-Simsona''.

Współliniowość spodków punktu $P$ na boki trójkąta jest twierdzeniem z dowodem u Zetela \cite[s. 55]{zetel_2020}, ćwiczeniem u Audina \cite[s. 104]{audin_2003} i Hartshorne'a \cite[s. 61]{hartshorne2000} i Neugebauera \cite[s. 25]{neugebauer_2018}.
% TODO: https://en.wikipedia.org/wiki/Simson_line
\todofoot{HARTSHORNE PRZECHODZI DO PUNKTU MIQUELA}

% PROSTA STEINERA? Audin ta sama strona
% http://users.math.uoc.gr/~pamfilos/eGallery/problems/SteinerLine.html

Z twierdzenia o prostej Wallace'a-Simsona wynika, jak u Zetela \cite[s. 57]{zetel_2020}:

\begin{proposition}[twierdzenie Salmona]
	Dany jest okrąg oraz trzy jego różne cięciwy $PA$, $PB$, $PC$ takie, że przekrojem okręgów na średnicach $PA$, $PB$ (odpowiednio: $PB$, $PC$ i $PA$, $PC$) są punkty $P$, $M$ (odpowiednio: $P$, $K$ oraz $P$, $L$).
	Wtedy punkty $K$, $L$, $M$ są współliniowe.
	\index{twierdzenie!Salmona}% % https://ru.wikipedia.org/wiki/Теорема_Сальмона
\end{proposition}

