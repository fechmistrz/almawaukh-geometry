%

\begin{proposition}[punkt i okrąg Spiekera]
\label{punkt_spiekera}%
    Niech $\triangle ABC$ będzie trójkątem.
    Środek okręgu wpisanego w trójkąt, którego wierzchołkami są środki boków trójkąta $ABC$, nazywamy jego punktem Spiekera.    
\end{proposition}

O punkcie Spiekera pisze też Zetel \cite[s. 22]{zetel_2020}, chociaż nie nazywa go tak, być może dlatego, że Theodor Spieker będzie dziewiętnastowiecznym geometrą z Niemiec (Poczdamu)?
\index[persons]{Spieker, Theodor}

%