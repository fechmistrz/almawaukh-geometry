%

Wnioskiem z twierdzenia o dwusiecznej jest:

\begin{theorem}[Steinera-Lehmusa]
    \index{twierdzenie!Steinera-Lehmusa}%
    \label{theorem_steiner_lehmus}%
	Jeżeli dwie dwusieczne trójkąta są równej długości, to trójkąt ten jest równoramienny.
\end{theorem}

Po raz pierwszy wspomniał o nim Christian Lehmus w liście z 1840 roku do Charlesa Sturma, gdzie poprosił o czysto geometryczny dowód.
\index[persons]{Lehmus, Christian}%
\index[persons]{Sturm, Charles}%
Sturm przekazał prośbę do innych matematyków, jedną z pierwszych osób, która uporała się z problemem, był Jakob Steiner.
\index[persons]{Steiner, Jakob}%
Większość znanych dowodów przeprowadza się nie wprost: jeśli trójkąt nie jest równoramienny, to ma dwusieczne różnej długości.
Dowód można znaleźć u Hartshorne'a \cite[s. 11]{hartshorne2000}; Bogdańskiej, Neugebauera \cite[s. 74]{neugebauer_2018}.
Coxeter \cite[s. 32]{coxeter_1967} podaje je w formie ćwiczenia (po tym jak wcześniej \cite[s. 26, 33]{coxeter_1967} poprosi o~dowód tego samego dla dwusiecznej zamienionej na środkową lub wysokość, co znacząco obniża poziom trudności).
% https://www.algebra.com/algebra/homework/word/geometry/Medians-in-an-isosceles-triangle.lesson

%