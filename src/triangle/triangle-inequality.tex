%

Zajmiemy się teraz (I.20).

\begin{proposition}[nierówność trójkąta]
\index{nierówność!trójkąta}%
	Niech $ABC$ będzie trojkątem.
	Wtedy suma odcinków $AB$ i $BC$ jest dłuższa niż $AC$.
\end{proposition}
% PRZECZYTANO: https://en.wikipedia.org/wiki/Triangle_inequality

\begin{proof}
	Wynika to ze wzoru Herona (fakt \ref{prp_heron}) i tego, że pole trójkąta jest nieujemne.
\end{proof}

\begin{corollary}
	Niech $a \ge b \ge c$ będą bokami trójkąta.
	Wtedy
	% \begin{equation}
	% 	1 < \frac{a + c}{b} < 3 
	% \end{equation}
	% oraz
	\begin{equation}
		1 \le \min \left(\frac ab, \frac bc\right) \le \phi = \frac {1 + \sqrt 5}{2}.
		% American Mathematical Monthly, pp. 49-50, 1954. 
	\end{equation}
\end{corollary}

Nierówność trójkąta nie jest wnioskiem z aksjomatów I1-I3, B1-B4, C1-C3, ponieważ nie zachodzi w następującym modelu (ukradniętym Hartshorne'owi \cite[s. 90]{hartshorne2000}):

\begin{example}
	Rozpatrujemy zbiór $\mathbb R^2$ jako płaszczyznę ze standardowymi punktami oraz prostymi, ale niestandardową metryką
	\begin{equation}
		d((x_1, y_1), (x_2, y_2)) = \begin{cases}
			\sqrt{(x_1-x_2)^2 + (y_1-y_2)^2} & \text{jeśli } x_1 = x_2 \vee y_1 = y_2, \\
			2 \sqrt{(x_1-x_2)^2 + (y_1-y_2)^2} & \text{w przeciwnym wypadku}
		\end{cases}.
	\end{equation}
	Wtedy nierówność trójkąta nie zachodzi.
\end{example}

% section{Problemy Fagnano i Fermata}
% https://en.wikipedia.org/wiki/Fagnano's_problem

\begin{problem}[zadanie Fagnano]
	Dany jest trójkąt ostrokątny $ABC$.
	Wpisać w niego trójkąt $UVW$ o możliwie najmniejszym obwodzie.
\index{zadanie!Fermata}%
\end{problem}

Coxeter \cite[s. 36, 37]{coxeter_1967} pokaże tak jak Fejer, że rozwiązaniem zadania jest trójkąt spodkowy (zwany ortycznym).
Audin \cite[s. 101]{audin_2003} podaje ten fakt w formie ćwiczenia. % todo: fagnano czy gemrat?

\begin{problem}[zadanie Fermata]
	\label{punkt_fermata}
	Dany jest trójkąt ostrokątny $ABC$.
	Znaleźć punkt $F$ taki, by suma $|FA| + |FB| + |FC|$ była możliwie najmniejsza.
\index{zadanie!Fermata}%
\end{problem}

\todofoot{Zadanie Fermata -- Neugebauer, s. 117.}

Powyższe zadanie rozwiąże Evangelista Torricelli (dlatego też punkt $F$ nazywa się czasem punktem Torricellego; robi tak Guzicki \cite[s. 224-228]{guzicki_2021}), który dostanie je w formie wyzwania od Fermata.
\index[persons]{Torricelli, Evangelista}%.
Rozwiązanie opublikuje student Torricelliego, Vincenzo Viviani, w 1659 roku.
\index[persons]{Viviani, Vincenzo}
% TODO: Johnson, R. A. Modern Geometry: An Elementary Treatise on the Geometry of the Triangle and the Circle. Boston, MA: Houghton Mifflin, pp. 221-222, 1929.
Coxeter \cite[s. 37]{coxeter_1967} przytoczy rozwiązanie Hofmanna\todofoot{J E Hoffman, Elementare Losung einer Mimimumsaufgabe 1929}
\index{zadanie!Fagnano}

%