%

Zajmiemy się teraz (I.20).

\begin{proposition}[nierówność trójkąta]
\index{nierówność!trójkąta}%
	Niech $ABC$ będzie trojkątem.
	Wtedy suma odcinków $AB$ i $BC$ jest dłuższa niż $AC$.
\end{proposition}
% PRZECZYTANO: https://en.wikipedia.org/wiki/Triangle_inequality

\begin{corollary}
	Niech $a \ge b \ge c$ będą bokami trójkąta.
	Wtedy
	% \begin{equation}
	% 	1 < \frac{a + c}{b} < 3
	% \end{equation}
	% oraz
	\begin{equation}
		1 \le \min \left(\frac ab, \frac bc\right) \le \phi = \frac {1 + \sqrt 5}{2}.
		% American Mathematical Monthly, pp. 49-50, 1954. 
	\end{equation}
\end{corollary}

\begin{proposition}
	Nierówność trójkąta wynika ze wzoru Herona (fakt \ref{prp_heron}) i tego, że pole trójkąta jest nieujemne.
\end{proposition}

Nierówność trójkąta nie jest wnioskiem z aksjomatów I1-I3, B1-B4, C1-C3, ponieważ nie zachodzi w następującym modelu (ukradniętym Hartshorne'owi \cite[s. 90]{hartshorne2000}):

\begin{example}
	Rozpatrujemy zbiór $\mathbb R^2$ jako płaszczyznę ze standardowymi punktami oraz prostymi, ale niestandardową metryką
	\begin{equation}
		d((x_1, y_1), (x_2, y_2)) = \begin{cases}
			\sqrt{(x_1-x_2)^2 + (y_1-y_2)^2} & \text{jeśli } x_1 = x_2 \vee y_1 = y_2, \\
			2 \sqrt{(x_1-x_2)^2 + (y_1-y_2)^2} & \text{w przeciwnym wypadku}
		\end{cases}.
	\end{equation}
	Wtedy nierówność trójkąta nie zachodzi.
\end{example}

%