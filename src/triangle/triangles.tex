
% TODO: https://en.wikipedia.org/wiki/Triangle

\index{pons asinorum|(}
\section{Pons asinorum}
%

\index{pons asinorum|(}

\subsubsection{Pons asinorum}
\index{most osłów patrz pons asinorum}
Most osłów (łacińskie \emph{,,pons asinorum''}) to tradycyjna nazwa dowodu twierdzenia, że kąty przy podstawie trójkąta równoramiennego są równe.
Podał go Euklides jako teza V w księdze I Elementów.
Mawiało się, że ci, którzy nie są w stanie samodzielnie przeprowadzić tego dedukcyjnego dowodu opartego na własnościach trójkątów przystających, nie może przekroczyć mostu i studiować dalej geometrii.

Bardziej przyziemnie Coxter \cite[s. 6-9]{coxeter_1991} zauważa, że rysunek wykonany przez Euklidesa przypomina most.
Wśród konsekwencji wymienia kilka wyników z Elementów: III.3, III.20, III.21, III.22, III.32, VI.2, VI.4, a potem III.35, III.36, VI.19, co prowadzi do dowodu twierdzenia Pitagorasa, czyli I.47. % TODO: sprawdzić, czy numeracja moja i Coxetera jest taka sama.
\index{twierdzenie!Pitagorasa}%
Coxeter podaje w formie ćwiczeń nierówność Erdős-Mordella (u nas podsekcja \ref{subsection_erdos_mordell}) oraz twierdzenie Steinera-Lehmusa (twierdzenie \ref{theorem_steiner_lehmus}).
% TODO: https://www.deltami.edu.pl/1990/08/elementarny-dowod-nierownosci-erdosa-mordella/
\todofoot{Przeczytać artykuł z Delty 1990, elementarny-dowod-nierownosci-erdosa-mordella}

Pierwsze dowody tego faktu podali jeszcze Euklides, komentujący jego prace Proklos zwany Diadochem oraz Pappus z Aleksandrii.
Współcześnie podaje się krótkie uzasadnienie w oparciu o dwusieczną kąta, ale Euklides nie mógł tak uczynić, ponieważ definiuje ją dopiero cztery tezy później w swoich Elementach.

O moście osłów piszą Coxeter 

\index{pons asinorum|)}

%
\index{pons asinorum|)}
\index{most osłów|see{pons asinorum}}

\section{Nierówność trójkąta}
%

\begin{proposition}[nierówność trójkąta]
	Niech $ABC$ będzie trojkątem.
	Wtedy suma odcinków $AB$ i $BC$ jest dłuższa niż $AC$.
	\index{nierówność trójkąta}
\end{proposition} % https://en.wikipedia.org/wiki/Triangle_inequality#Euclidean_geometry I.20

Nierówność trójkąta nie jest wnioskiem z aksjomatów I1-I3, B1-B4, C1-C3, ponieważ nie zachodzi w następującym modelu: płaszczyzna to zbiór $\mathbb R^2$, ze standardowymi punktami i prostymi, ale niestandardową metryką
\begin{equation}
	d((x_1, y_1), (x_2, y_2)) = \begin{cases}
		\sqrt{(x_1-x_2)^2 + (y_1-y_2)^2} & \text{jeśli } x_1 = x_2 \vee y_1 = y_2, \\
		2 \sqrt{(x_1-x_2)^2 + (y_1-y_2)^2} & \text{w przeciwnym wypadku}
	\end{cases}.
\end{equation}

(Powyższy przykład opisał Hartshorne \cite[s. 90]{hartshorne2000}).

%

\section{Punkty szczególne}
Z każdym trójkątem związane są pewne specjalne punkty, internetowa lista \emph{Encyclopedia of Triangle Centers} wymieni ich co najmniej 68 547.
My nie mamy aż tyle miejsca, więc ograniczymy się do najważniejszych.

$X(1)$ to środek okręgu wpisanego, 
$X(2)$ to środek ciężkości,
$X(3)$ to środek okręgu opisanego,
$X(4)$ to ortocentrum.
Te cztery punkty opisujemy poniżej.

$X(5)$ to środek okręgu dziewięciu punktów z faktu \ref{okrag_dziewieciu_punktow},
$X(6)$ to punkt Lemoine'a (Grebego) z faktu \ref{punkt_lemoine},
$X(7)$ to punkt Gergonne'a z~faktu \ref{punkt_gergonne},
$X(8)$ to punkt Nagela z faktu \ref{punkt_nagela}, 
$X(9)$ to mittenpunkt (punkt Lemoine'a trójkąta rozpiętego przez środki okręgów dopisanych; leży na prostych łączących środek ciężkości z punktem Gergonne'a, środek okręgu wpisanego z punktem Lemoine'a oraz ortocentrum ze środkiem Spiekera), % https://en.wikipedia.org/wiki/Mittenpunkt
$X(10)$ to środek okręgu Spiekera z faktu \ref{punkt_spiekera},
$X(11)$ to punkt Feuerbacha z twierdzenia \ref{punkt_feuerbacha},
$X(13)$ to punkt Fermata z zadania \ref{punkt_fermata},
$X(17)$, $X(18)$ to punkty Napoleona,
$X(39)$ to środek odcinka łączącego punkty Brocarda z definicji \ref{punkty_brocarda}.
Lista jest długa, a jej końca nie widać.

\begin{proposition}
    \label{symetralne_przecinaja_sie}
    Trzy symetralne boków trójkąta przecinają się w jednym punkcie, środku okręgu opisanego na tym trójkącie.
\end{proposition}

Wynika to bezpośrednio z uwagi za definicją \ref{def_symetralna}: w trójkącie $\triangle ABC$, symetralna boku $AB$ (odpowiednio: $AC$) zawiera środki okręgów przechodzących przez punkty $A$ oraz $B$ (przez punkty $A$ oraz $C$), zatem trzecia symetralna musi przejść przez punkt wspólny dwóch pierwszych.

Hartshorne \cite[s. 16]{hartshorne2000} podaje to w formie ćwiczenia ze wskazówką, by spojrzeć na (IV.5).
Audin \cite[s. 61]{audin_2003} też, ale bez wskazówki.

Uogólnieniem faktu \ref{symetralne_przecinaja_sie} o współpękowości symetralnych boków trójkąta jest:

\begin{theorem}[Carnota]
\index{twierdzenie!Carnota}%
\label{guzicki_6_13}%
	Dany jest trójkąt $ABC$ i punkty $D, E, F$ leżące odpowiednio na prostych $BC, CA, AB$.
	Niech prosta $k$ (odpowiednio: $l$, $m$) przechodzi przez punkt $D$ ($E$, $F$) i będzie prostopadła do prostej $BC$ ($CA$, $AB$).
	Wtedy proste $k$, $l$, $m$ mają punkt wspólny wtedy i tylko wtedy, gdy
	\begin{equation}
		|AF|^2 + |BD|^2 + |CE|^2 = |AE|^2 + |BF|^2 + |CD|^2.
	\end{equation}
\end{theorem}
% TODO: https://en.wikipedia.org/wiki/Carnot%27s_theorem_(perpendiculars)

Guzicki \cite[s. 176]{guzicki_2021} wyprowadza je z twierdzenia Pitagorasa, co pozwala mu dojść do wniosku, że okręgi opisany i wpisany oraz ortocentrum istnieją.
\index{twierdzenie!Pitagorasa}

\begin{proposition}
    Trzy dwusieczne kątów trójkąta przecinają się w jednym punkcie, środku okręgu wpisanego w ten trójkąt.
\end{proposition}

Uzasadnienie jest kompletnie analogiczne.
Po angielsku mamy krótkie określenia \emph{excenter, excircle, incenter, incircle}.
Hartshorne \cite[s. 16]{hartshorne2000} podaje to w formie ćwiczenia ze wskazówką, by spojrzeć na (IV.4).

\begin{definition}[wysokość]
\index{wysokość trójkąta}%
    Niech $\triangle ABC$ będzie trójkątem, w którym przez punkt $C$ poprowadzono prostą prostopadłą do boku $AB$.
    Odcinek leżący na tej prostej, którego jeden koniec znajduje się w punkcie $C$, zaś drugi leży na boku $AB$, nazywamy wysokością opuszczoną z punktu $C$, drugi jego koniec to spodek wysokości.
\end{definition}

% TODO: Sometimes an arbitrary edge is chosen to be the base, in which case the opposite vertex is called the apex; the shortest segment between the base and apex is the height. The area of a triangle equals one-half the product of height and base length.

\begin{proposition}
\label{wysokosci_przecinaja_sie}%
	Trzy wysokości trójkąta (albo w przypadku trójkąta rozwartokątnego, przedłużenia wysokości) przecinają się w jednym punkcie zwanym ortocentrum (dawniej: środku ortycznym).
\index{ortocentrum}%
\end{proposition}

Po angielsku mamy \emph{altitude}, \emph{foot of the altitude}, \emph{orthocenter}.
Pompe \cite[s. 38]{pompe_2022} pokaże dowód, który zmyślnie używa równoległoboków.
Oprócz niego warto przeczytać tekst Hartshorne'a \cite[s. 54]{hartshorne2000}. albo Audina \cite[s. 61]{audin_2003}, gdzie poda się ten fakt w formie ćwiczenia.
Cały dowód pokaże Guzicki \cite[s. 218]{guzicki_2021}.

\begin{definition}[środkowa]
\index{środkowa}%
    Niech $\triangle ABC$ będzie trójkątem.
    Odcinek łaczący wierzchołek (na przykład $A$) ze środkiem przeciwległego boku (w naszym przykładzie $BC$) nazywamy środkową.
\end{definition}

Środkowe dzielą trójkąt na sześć mniejszych o równych polach.

\begin{proposition}
\label{srodkowe_przecinaja_sie}%
\index{środek ciężkości}%
    Trzy środkowe trójkąta przecinają się w jednym punkcie zwanym środkiem ciężkości i dzielą się w~stosunku $2$ do $1$, licząc od wierzchołków.
\end{proposition}
% % Coxeter, Introduction to Geometry, s. 10 <- przeczytaj to, nie tylko cytuj! + ćwiczenia: 3/4 <= 1

Środkowe to po angielsku \emph{medians}, przecinają się w \emph{centroid}.
Polska nazwa bierze się z tego, że środek ciężkości fizycznego modelu trójkąta wykonanego z jednolitego materiału znajduje się właśnie tam.
(Ale warto też wiedzieć o punkcie Spiekera).

Hartshorne \cite[s. 52-54]{hartshorne2000} wnioskuje \ref{srodkowe_przecinaja_sie} z faktu \ref{hartshorne_52} z krótkim komentarzem, że dwa ostatnie stwierdzenia (\ref{wysokosci_przecinaja_sie}, \ref{srodkowe_przecinaja_sie}) były znane Archimedesowi, później zaś \cite[s. 119-120]{hartshorne2000} powtórzy dowód z~maszynerią geometrii analitycznej.
\index[persons]{Archimedes}%
Podobnie zrobi Guzicki \cite[s. 220]{guzicki_2021} albo (wnioskując \ref{hartshorne_52} z twierdzenia Talesa) Bogdańska, Neugebauer.
\index{twierdzenie!Talesa}%
% TODO: Ich długość można obliczyć z https://en.wikipedia.org/wiki/Apollonius%27s_theorem => to jest z https://en.wikipedia.org/wiki/Median_(geometry)#Formulas_involving_the_medians'_lengths

Analogiczne stwierdzenie dla czworościanu mówi o dzieleniu się w stosunku $3$ do $1$ i bywa określane twierdzeniem Commandino, ponieważ Federico Commandino napisze w 1565 roku pracę \emph{De Centro Gravitatis Solidorum} (O środkach ciężkości brył), chociaż może nie być pierwszym, który je odkrył.
\index{twierdzenie!Commandino}%
\index[persons]{Commandino, Federico}%
Podejrzewa się, że Francesco Maurolico pozna je wcześniej.
(Friederich Eduard Reusch znajdzie uogólnienie, które w zdegenerowanym przypadku prowadzi znowu do twierdzenia \ref{theorem_varignon} Varignona).

\section{Twierdzenie Pitagorasa}
%

\subsubsection{Twierdzenie Pitagorasa}
Najważniejszym twierdzeniem dotyczącym trójkątów prostokątnych jest twierdzenie Pitagorasa oraz twierdzenie do niego odwrotne.
Piszą o~nim Guzicki \cite[s. 160]{guzicki_2021}.

\begin{theorem}[Pitagorasa, ok. 500 r. p.n.e.]
\index{twierdzenie!Pitagorasa}%
    Niech $ABC$ będzie trójkątem prostokątnym, w~którym kąt przy wierzchołku $C$ jest prosty.
    Wtedy
    \begin{equation}
        |BC|^2 + |AC|^2 = |AB|^2.
    \end{equation}
    Odwrotnie, jeśli $ABC$ jest trójkątem takim, że $|BC|^2 + |AC|^2 = |AB|^2$, to trójkąt ten jest prostokątny, zaś kąt przy wierzchołku $C$ jest prosty.
\end{theorem}

Chociaż współcześnie powyższe twierdzenie przypisujemy Pitagorasowi z~Samos, to nie wiemy dokładnie, kto i~kiedy odkrył je jako pierwszy.
\index[persons]{Pitagoras z Samos}%
Było powszechnie stosowane w~okresie Starego Babilonu (XX-XVI wiek p.n.e.), a~więc na długo przed narodzinami Pitagorasa; pojawia się też w indyjskich i~chińskich tekstach matematycznych.
Papirus Berlin 6619 spisany ok. 1800 roku p.n.e. na terenach państwa egipskiego zawiera zadanie, którego rozwiązaniem jest trójka $(6, 8, 10)$.

Już w~szkole podstawowej uczniowie poznają trójkąt prostokątny o bokach długości $3, 4, 5$ wraz~z~legendą, że podobno Egipcjanie używali tego trójkąta do wyznaczania w terenie kątów prostych.

% TODO: rysunek z Guzickiego, stron 160

\begin{proposition}
    Mają miejsce następujące równości:
    \begin{equation}
        h = \frac{ab}{c}, \quad
        p = \frac{b^2}{c}, \quad
        q = \frac{a^2}{c}, \quad
        h^2 = pq.
    \end{equation}
\end{proposition}

Dowód wykorzystujący podobieństwa trójkątów można znaleźć u~Guzickiego \cite[s. 160, 161]{guzicki_2021}.

Twierdzenie Pitagorasa znajduje zastosowanie także przy wyznaczaniu niektórych miejsc geometrycznych.

\begin{proposition}
    Dane są dwa różne punkty $A$ i $B$ na płaszczyźnie oraz liczba rzeczywista $c$ taka, że $2c > |AB|^2$.
    Miejscem geometrycznym punktów $P$ o własności $|AP|^2 + |BP|^2 = c$ jest okrąg o środku w środku odcinka $AB$ i promieniu $r = \frac 1 2 \sqrt{2c - |AB|^2}$.
\end{proposition}

\begin{proposition}
    Dane są dwa różne punkty $A$ i $B$ na płaszczyźnie oraz liczba rzeczywista $c$.
    Miejscem geometrycznym punktów $P$ o własności $|AP|^2 - |BP|^2 = c$ jest prosta prostopadła do prostej $AB$.
\end{proposition}

Patrz Guzicki \cite[s. 170-173]{guzicki_2021} (Guzicki wprowadza potem osie i środki potęgowe jak w~fakcie \ref{guzicki_6_11}, a następnie twierdzenie \ref{guzicki_6_13} (Carnota)).

%

% https://en.wikipedia.org/wiki/Pythagorean_theorem liczne dowody, wiek Pitagorasa
% https://en.wikipedia.org/wiki/Xuan_tu
% https://en.wikipedia.org/wiki/Spiral_of_Theodorus
% https://en.wikipedia.org/wiki/Garfield%27s_proof_of_the_Pythagorean_theorem

\index{wzór!Herona|(}
\subsection{Wzór Herona}
%

\index{wzór!Herona|(}
Guzicki \cite[s. 165-168]{guzicki_2021} wyprowadza wzór Herona z twierdzenia Pitagorasa.
\index{twierdzenie!Pitagorasa}
Oryginalny dowód Herona był dość skomplikowany, Guzicki \cite[s. 168-169]{guzicki_2021} wspomina o znacznie prostszym dowodzie geometrycznym, pochodzącym od Eulera.
\index[persons]{Euler, Leonhard}%

\begin{proposition}
	Niech $ABC$ będzie trójkątem o obwodzie $2p$ oraz polu powierzchni $S$.
	Wtedy
	\begin{equation}
		S \le \frac{p^2}{3 \sqrt{3}}
	\end{equation}
\end{proposition}

Guzicki wyprowadza tę nierówność izoperymetryczną ze wzoru Herona oraz nierówności między średnią arytmetyczną i geometryczną.

\index{wzór!Herona|)}

% TODO: wzór Herona (Guzicki-6), Brahmagupty

%
\index{wzór!Herona|)}

\section{Okrąg dziewięciu punktów (Feuerbacha) i prosta Eulera}
%

Prosta Eulera to pierwsza w szkolnej geometrii trójka punktów współliniowych.
Przyszła na świat w~1765 roku, w żurnalu \emph{Novi commentarii Academiae Petropolitanae}, w artykule \emph{Solutio facilis problematum quorundam geometricorum difficillimorum}, czyli jakbyśmy powiedzieli po polsku, ,,Łatwe rozwiązanie niektórych najtrudniejszych problemów geometrycznych''.

\begin{proposition}[prosta Eulera]
	\label{prosta_eulera}
	Środek okręgu opisanego na nierównobocznym trójkącie, środek ciężkości oraz ortocentrum leżą na jednej prostej, zwanej prostą Eulera.
	\index{prosta!Eulera}
\end{proposition}

Piszą o niej
Coxeter \cite[s. 32, 33]{coxeter_1967},
Hartshorne \cite[s. 54, 55]{hartshorne2000},
Bogdańska, Neugebauer \cite[s. 84]{neugebauer_2018},
Audin \cite[s. 61]{audin_2003}.
W $\Delta_{84}^{4}$ podany będzie przepis, jak skonstruować trójkąt, w którym prosta Eulera ma zadane położenie względem podstawy.

\begin{proposition}
	Prosta Eulera jest prostopadła do jednego z boków trójkąta wtedy i tylko wtedy, gdy trójkąt jest równoramienny.
\end{proposition}

\begin{proposition}[okrąg dziewięciu punktów]
\label{okrag_dziewieciu_punktow}%
	W każdym trójkącie środki boków, spodki wysokości oraz środki odcinków łączących ortocentrum z wierzchołkami leżą na jednym okręgu.
	Jego środek pokrywa się ze środkiem odcinka łączącego środek okręgu opisanego z ortocentrum, zaś jego promień jest dwukrotnie krótszy od promienia okręgu opisanego.
	\index{okrąg!dziewięciu punktów}
\end{proposition}

% TODO: https://en.wikipedia.org/wiki/Nine-point_circle

Coxeter \cite[s. 34, 35, 88]{coxeter_1967}, Hartshorne \cite[s. 57, 60]{hartshorne2000}, Bogdańska, Neugebauer \cite[s. 85, 86]{neugebauer_2018}.
Audin \cite[s. 62]{audin_2003}.

Temat badali Benjamin Bevan (który zasugerował środek oraz promień) i John Butterworth (który udowodnił podejrzenia Bevana) na początku XIX wieku.
\index[persons]{Bevan, Benjamin}%
\index[persons]{Butterworth, John}%
To, że środki boków i spodki wysokości leżą na wspólnym okręgu, zostało zauważone w 1821 roku przez Charles Brianchona i Jean-Victora Ponceleta.
\index[persons]{Brianchon, Charles}%
\index[persons]{Poncelet, Jean-Victor}%
Tego samego odkrycia dokonał rok później Karl Feuerbach; a krótko po nim Olry Terquem zauważył, że leży na nim dziewięć, a nie tylko sześć wspomnianych punktów.
\todofoot{The nine-point circle also passes through Kimberling centers Xi for i=11 (the Feuerbach point), 113, 114, 115 (center of the Kiepert hyperbola), 116, 117, 118, 119, 120, 121, 122, 123, 124, 125 (center of the Jerabek hyperbola), 126, 127, 128, 129, 130, 131, 132, 133, 134, 135, 136, 137, 138, 139, 1312, 1313, 1560, 1566, 2039, 2040, and 2679.}
\todofoot{Karl Wilhelm Feuerbach's Eigenschaften einiger merkwiirdigen Punkte des geradlinigen Dreiecks, along with many other interesting proofs relating to the nine point circle.}
\index[persons]{Feuerbach, Karl}%
\index[persons]{Terquem, Olry}%
Terquemowi (1842) zawdzięczamy nazwę ,,okrąg dziewięciu punktów''.
\todofoot{The circle is officially designated the "nine point circle" (le cercle des neuf points) by Terquem, one of the editors of the Nouvelles Annales. (see Volume I page 198).}

Feuerbach udowodnił też, że:

\begin{theorem}[Feuerbacha]
\label{punkt_feuerbacha}%
	Okrąg dziewięciu punktów jest styczny wewnętrznie do okręgu wpisanego (w punkcie Feuerbacha) i zewnętrznie do trzech okręgów dopisanych.
	\index{twierdzenie!Feuerbacha}%
\end{theorem}

Coxeter \cite[s. 99]{coxeter_1967}, Audin \cite[s. 110]{audin_2003} podają ten fakt w formie ćwiczenia.

punkt Torricellego/Fermata (Guzicki-8)
Audin \cite[s. 105]{audin_2003} podaje ten fakt w formie ćwiczenia.
% problem Fermeta?
% TODO: jaki fakt dokładnie podaje

\begin{proposition}
	\label{orthic_triangle}
	Niech $ABC$ będzie trójkątem ostrokątnym, zaś $K$, $L$ oraz $M$ spodkami jego wysokości.
	Wtedy wysokości trójkąta $ABC$ są dwusiecznymi kątów trójkąta $KLM$.
\end{proposition}

Hartshorne \cite[s. 58]{hartshorne2000}.

%

\section{Nierówności trójkątne}

\begin{proposition}[nierówność izoperymetryczna]
	Dany jest trójkąt o połowie obwodu $p$ oraz polu $S$.
	Wtedy 
	\begin{equation}
		S \le \frac{p^2}{3 \sqrt 3},
	\end{equation}
	zatem wśród trójkątów o ustalonym obwodzie największe pole ma trójkąt równoboczny.
	\index{nierówność!izoperymetryczna}
\end{proposition}

Guzicki \cite[s. 169, 170]{guzicki_2021} wyprowadza nierówność izoperymetryczną ze wzoru Herona oraz nierówności między średnią arytmetyczną i geometryczną.
\index{wzór!Herona}

% trójwymiarowy odpowiednik to hipoteza: https://math.stackexchange.com/questions/4044670/what-is-the-largest-volume-of-a-polyhedron-whose-skeleton-has-total-length-1-is

\begin{proposition}[stosunek sumy środkowych do obwodu]
	Niech... % stosunek sumy środkowych do obwodu leży między 3/4 i 1 (s. 355),
	% TODO: https://en.wikipedia.org/wiki/Isoperimetric_inequality ?
\end{proposition}

\begin{proposition}[nierówność Eulera]
	$R \ge 2r$
	% TODO: Twierdzenie Eulera: $d^2 = R^2 - 2Rr$. % Audin \cite[s. 110]{audin_2003} podaje ten fakt w formie ćwiczenia.

	% TODO: Eulera: R >= 2r https://en.wikipedia.org/wiki/Euler%27s_theorem_in_geometry
	\todofoot{formuła Eulera na odległość między środkami okręgu opisanego i wpisanego (dla trójkąta)}
	\index{nierówność!Eulera}
\end{proposition}

\begin{proposition}[nierówność Mitrinovica]
	Niech...
	\index{nierówność!Mitrinovica}
\end{proposition}

\begin{proposition}[nierówność Leibniza]
	Niech...
	\index{nierówność!Leibniza}
\end{proposition}

\begin{proposition}[nierówność Weitzenbocka]
	Niech...
	% TODO: https://en.wikipedia.org/wiki/Hadwiger–Finsler_inequality => Weitzenbock
	% TODO: https://en.wikipedia.org/wiki/Pedoe%27s_inequality => Weitzenbock
	\index{nierówność!Weitzenbocka}
\end{proposition}

Snellius-Huygens: $2 \sin x + \tan x > 3x$.
\index{nierówność!Snelliusa-Huygensa}

\index{nierówność!Erdősa-Mordella|(}
%

\label{subsection_erdos_mordell}
Erdős w 1935 roku postawi problem dowodu tej nierówności; dowód przedstawią dwa lata później Mordell i D. F. Barrow (1937), choć nie będzie on zbyt elementarny.
Później znajdzie się prostsze dowody: Kazarinoff (1957), Bankoff (1958) oraz Alsina i Nelsen (2007).
% TODO: https://en.wikipedia.org/wiki/Erdős–Mordell_inequality#CITEREFErdős1935

\begin{theorem}[nierówność Erdősa-Mordella]
    Niech $P$ będzie punktem wewnątrz trójkąta $\triangle ABC$, zaś $A_p, B_p, C_p$ spodkami punktu $P$ na boki trójkąta jak na rysunku \ref{erdos_mordell_barrowa}.
    Wtedy
    \begin{equation}
        |PA| + |PB| + |PC| \ge 2 (|PA_p| + |PB_p| + |PC_p|).
    \end{equation}
\end{theorem}


\begin{figure}[H] \centering
\begin{minipage}[b]{.45\linewidth}
\begin{center}
    \begin{comment}
    \begin{tikzpicture}[scale=.4]
    \tkzDefPoint(0, 0){A}
    \tkzDefPoint(10, 2){B}
    \tkzDefPoint(6, 7){C}
    \tkzDefPoint(5, 3){P}
    \tkzLabelPoint[below left](A){$A$}
    \tkzLabelPoint[below right](B){$B$}
    \tkzLabelPoint[above](C){$C$}
    \tkzLabelPoint[below left](P){$P$}
    \tkzDefPointsBy[projection=onto A--B](P){Pc}
    \tkzDefPointsBy[projection=onto B--C](P){Pa}
    \tkzDefPointsBy[projection=onto C--A](P){Pb}
    \tkzLabelPoint[above right](Pa){$A_p$}
    \tkzLabelPoint[above left](Pb){$B_p$}
    \tkzLabelPoint[below](Pc){$C_p$}

    \tkzDrawSegments[line width=0.2mm,dashed](P,Pa P,Pb P,Pc)
    \tkzDrawPolygon[line width=0.3mm](A,B,C)
    \tkzMarkRightAngles[size=0.5](P,Pa,C P,Pb,A P,Pc,B)
    \tkzDrawPoints[size=3,color=black,fill=black!50](A,B,C,P,Pc,Pb,Pa)
\end{tikzpicture}
\end{comment}
    \end{center}
    \subcaption{nierówność Erdősa-Mordella}
    \label{erdos_mordell_barrowa}
\end{minipage}
%
\begin{minipage}[b]{.45\linewidth}
\begin{center}\begin{comment}
    \begin{tikzpicture}[scale=.4]
    \tkzDefPoint(0, 0){A}
    \tkzDefPoint(10, 2){B}
    \tkzDefPoint(6, 7){C}
    \tkzDefPoint(5, 3){P}

    \tkzDefLine[bisector](A,P,B) \tkzGetPoint{prePc}
    \tkzInterLL(P,prePc)(A,B) \tkzGetPoint{Pc}
    \tkzDefLine[bisector](B,P,C) \tkzGetPoint{prePa}
    \tkzInterLL(P,prePa)(B,C) \tkzGetPoint{Pa}
    \tkzDefLine[bisector](C,P,A) \tkzGetPoint{prePb}
    \tkzInterLL(P,prePb)(C,A) \tkzGetPoint{Pb}

    \tkzLabelPoint[below left](A){$A$}
    \tkzLabelPoint[below right](B){$B$}
    \tkzLabelPoint[above](C){$C$}
    %\tkzLabelPoint[below left](P){$P$}
    \tkzLabelPoint[above right](Pa){$A_p$}
    \tkzLabelPoint[above left](Pb){$B_p$}
    \tkzLabelPoint[below](Pc){$C_p$}

    \tkzMarkAngle[arc=lll,size=1.2,mark=|||](A,P,Pc)
    \tkzMarkAngle[arc=lll,size=1.2,mark=|||](Pc,P,B)
    \tkzMarkAngle[arc=ll,size=1.2,mark=||](B,P,Pa)
    \tkzMarkAngle[arc=ll,size=1.2,mark=||](Pa,P,C)
    \tkzMarkAngle[arc=l,size=1.2,mark=|](C,P,Pb)
    \tkzMarkAngle[arc=l,size=1.2,mark=|](Pb,P,A)

    \tkzDrawSegments[line width=0.2mm](P,A P,B P,C)
    \tkzDrawSegments[line width=0.2mm,dashed](P,Pa P,Pb P,Pc)
    \tkzDrawPolygon[line width=0.3mm](A,B,C)
    \tkzDrawPoints[size=3,color=black,fill=black!50](A,B,C,P,Pc,Pb,Pa)
\end{tikzpicture}
\end{comment}
    \end{center}
    \subcaption{nierówność Barrowa}
    \label{erdos_mordell_barrowb}
\end{minipage}
\caption{}
\end{figure}

Twierdzenie poda w formie ćwiczenia Coxeter \cite[s. 9]{coxeter_1991}, Audin z licznymi wskazówkami \cite[s. 102]{audin_2003}.

Wzmocnieniem nierówności Erdősa-Mordella będzie nierówność Barrowa:

% TODO: https://en.wikipedia.org/wiki/Barrow%27s_inequality

\begin{theorem}[nierówność Barrowa]
    Niech $P$ będzie punktem wewnątrz trójkąta $\triangle ABC$, zaś $A_p$, $B_p$, $C_p$ punktami przecięć dwusiecznych trzech kątów wyznaczanych przez $P$ i pary wierzchołków trójkąta; tak jak na rysunku \ref{erdos_mordell_barrowb}.
    Wtedy
    \begin{equation}
        |PA| + |PB| + |PC| \ge 2 (|PA_p| + |PB_p| + |PC_p|).
    \end{equation}
\end{theorem}

Dowód Barrowa zostanie opublikowany w 1937 roku, ale nazwa ,,nierówność Barrowa'' będzie używana dopiero od 1961 roku; nie wiemy, co się wtedy stanie.
% TODO: Erdős, Paul; Mordell, L. J.; Barrow, David F. (1937), "Solution to problem 3740", American Mathematical Monthly, 44 (4): 252–254, doi:10.2307/2300713, JSTOR 2300713.

% % barrow tu jest
\index{nierówność!Erdősa-Mordella|)}

% TODO: https://en.wikipedia.org/wiki/Ono%27s_inequality

Mikołaj z Kuzy: $\sin x / x < (2 + \cos x) / 3$.
\index{nierówność!Mikołaja z Kuzy}
\index[persons]{Mikołaj z Kuzy}

\todofoot{Coxeter, s. 12}

\section{Okrąg Spiekera}
\begin{proposition}[okrąg Spiekera]
    \label{punkt_spiekera}
\end{proposition}

\section{Nie wiem gdzie}

Droz-Farny: proste przechodzą przez ortocentrum trójkąta i są prostopadłe, wtedy środki odcinków leżą na jednej prostej. % https://en.wikipedia.org/wiki/Droz-Farny_line_theorem

%

% TODO: https://math.stackexchange.com/questions/3332146/is-it-possible-to-find-such-an-angle-using-only-angle-chasing
% TODO: https://math.stackexchange.com/questions/3009635/japanese-temple-problem-from-1844

\todofoot{Calabi triangle} % https://en.wikipedia.org/wiki/Calabi_triangle
\index{trójkąt!Calabiego}

% https://en.wikipedia.org/wiki/Pompeiu%27s_theorem 1936

% TODO: https://en.wikipedia.org/wiki/Heilbronn_triangle_problem

% https://en.wikipedia.org/wiki/Spieker_circle

\section{Twierdzenie Talesa}
%

Guzicki-3

\begin{theorem}[Talesa]
    Jeśli ramiona kąta płaskiego przetnie się 2 równoległymi prostymi:
    \begin{center}
\begin{comment}
        \begin{tikzpicture}
            \tkzDefPoint(0, 0.5){O}
            \tkzDefPoint(1.5, 0){A}
            \tkzDefPoint(2, 1){Ap}
            \tkzDefPointBy[homothety=center O ratio 1.618](A) \tkzGetPoint{B}
            \tkzDefLine[parallel=through B](A,Ap) \tkzGetPoint{Bp}
            \tkzInterLL(O,Ap)(B,Bp) \tkzGetPoint{Bpp}
            \tkzDrawPoints[fill=gray,opacity=.9](O,A,B,Ap,Bpp)
            \tkzLabelPoint[above](O){$O$}
            \tkzLabelPoint[below](A){$A$}
            \tkzLabelPoint[below](B){$A'$}
            \tkzLabelPoint[above left](Bpp){$B'$}
            \tkzLabelPoint[above left](Ap){$B$}
            \tkzDrawLine[thick](O,B)
            \tkzDrawLine[thick](O,Bpp)
            \tkzDrawLine[color=blue, thick](A,Ap)
            \tkzDrawLine[color=blue, thick](B,Bpp)
        \end{tikzpicture}
\end{comment}
        \end{center}
    to długości odcinków wyznaczonych przez te proste na jednym z ramion kąta są proporcjonalne do długości odpowiednich odcinków na drugim ramieniu kąta, a zatem
    \begin{equation}
        \label{thales_ratio}
        \frac{|OA|}{|OA'|} = \frac{|OB|}{|OB'|} = \frac{|AB|}{|A'B'|}.
    \end{equation}
\end{theorem}
% TODO: https://en.wikipedia.org/wiki/Thales's_theorem

Tradycja przypisuje jego sformułowanie Talesowi z Miletu, chociaż znane było starożytnym Babilończykom i Egipcjanom.
\index[persons]{Tales z Miletu}%
% Pierwszy znany dowód pojawia się w Elementach Euklidesa.
Najstarszy zachowany dowód twierdzenia Talesa zamieszczony jest w VI. księdze Elementów Euklidesa. 
% https://en.wikipedia.org/wiki/Intercept_theorem#Claim_3

Piszą o nim Neugebauer, Bogdańska \cite[s. 48-56]{neugebauer_2018}; Audin \cite[s. 24, 173]{audin_2003}.

Po angielsku znane jest jako \emph{Thales's theorem}, \emph{intercept theorem}, \emph{basic proportionality theorem} albo \emph{side splitter theorem}.

Prawdziwe jest również twierdzenie odwrotne:

\begin{proposition}[twierdzenie odwrotne do tw. Talesa]
    Jeżeli pewna prosta przecina boki $OA'$, $OB'$ trójkąta $OA'B'$ w różnych punktach $A$ i $B$ odpowiednio, a przy tym zachodzi równość \ref{thales_ratio}, to prosta ta jest równoległa do prostej $A'B'$.
\end{proposition}

Prostym wnioskiem z twierdzenia Talesa jest fakt \ref{hartshorne_52}, znajduje on zastosowanie w dowodzie:
% Neugebauer s. 52

\begin{theorem}[Varignona]
    Czworokąt $PQRS$, którego wierzchołki leżą na środkach boków $AB$, $BC$, $CD$, $DA$ czworokąta $ABCD$, jest równoległobokiem.
    Jego znakowane  (!) pole jest równe połowie pola czworokąta $ABCD$. % Neugebauer s. 61
\end{theorem}

% * The area of the Varignon parallelogram equals half the area of the original quadrilateral. This is true in convex, concave and crossed quadrilaterals provided the area of the latter is defined to be the difference of the areas of the two triangles it is composed of. => [[Varignon's theorem]]


W szczególności, czworokąt $ABCD$ nie musi być wypukły\footnote{Może być nawet ,,motylkiem'', to znaczy łamaną zamkniętą o czterech bokach, która ma samoprzecięcia.}.
Twierdzenie zostało nazwane na cześć Pierre'a Varignona pośmiertnie w 1731 roku.
\index[persons]{Varignon, Pierre}%
Co więcej,

\begin{proposition}
    Równoległobok Varignona jest rombem (prostokątem) wtedy i tylko wtedy, gdy przekątne czworokąta $ABCD$ są równej długości (są prostopadłe do siebie).
\index{równoległobok!Varignona}%
\index{romb}%
\index{prostokąt}%
% de Villiers, Michael (2009), Some Adventures in Euclidean Geometry, Dynamic Mathematics Learning, p. 58, 169. ISBN 9780557102952.
\end{proposition}

%

\section{Podobieństwo trójkątów}
\begin{definition}
	Dwa trójkąty nazywamy podobnymi...
	Liczbę $\lambda$... nazywamy skalą podobieństwa.
	\index{podobieństwo}
\end{definition}

\begin{proposition}[cecha podobieństwa BKB]
	Jeśli dla danych trójkątów...
	\index{cecha podobieństwa!bok-kąt-bok}
\end{proposition}

\begin{proposition}[cecha podobieństwa BBB]
	Jeśli dla danych trójkątów...
	\index{cecha podobieństwa!bok-bok-bok}
\end{proposition}

% Przykład: zadanie 2.4 z Neugebauera, s. 60

% Jeżeli... ze skalą podobieństwa \lambda, to pola... \lambda^2.

% \section{Pole?}

\section{Twierdzenie o dwusiecznej}
\input{similarity/angle-bisector}
